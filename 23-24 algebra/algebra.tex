\documentclass[a4paper]{article}
\usepackage[utf8]{inputenc} % standard unicode
\usepackage[italian]{babel} % corretta sillabazione in italiano
\usepackage{geometry} % per impostare margini e layout pagina
\usepackage{amssymb} % per l'ambiente matematico
\usepackage{amsmath} % per l'ambiente matematico
\usepackage{enumitem} % per elenchi puntati
\usepackage{multirow} % per celle che si espandono su più righe
\usepackage{tabularx} % per tabelle con larghezza flessibile
\usepackage{booktabs} % per linee orizzontali tabelle
\usepackage{hyperref} % per collegamenti
\usepackage{graphicx} % per immagini
\usepackage{listings} % per codice
\usepackage{xcolor} % per colori nel codice
\usepackage{dirtytalk} % per le ""

% per margini
\geometry{a4paper,left=25mm, right=25mm, bottom=25mm, top=30mm}

% per centrare testo nelle tabelleX
\renewcommand\tabularxcolumn[1]{m{#1}}

% percorso delle immagini da inserire
\graphicspath{ {./ } }

\title{Appunti di algebra lineare e geometria}
\author{Giacomo Simonetto}
\date{Secondo semetre 2023-24}

\begin{document}

% -------------------------------------- Copertina e indice ---------------------------------------
\maketitle
\begin{abstract}
	Appunti del corso di algebra lineare e geometria della facoltà di Ingegneria Informatica dell'Università di Padova.
\end{abstract}

\newpage

\tableofcontents

\newpage

% ----------------------------------------- introduzione ------------------------------------------
\section{Definizioni  di anello commutativo e campo e spazio vettoriale}
\subsection{Anello commutativo}
Un anello commutativo con unità è un insieme in cui sono definite due operazioni \(+\) e \(\times\) e che soddisfa le seguenti
proprietà:
\begin{itemize}
	\item[-] per la somma \(+\):
	\begin{enumerate}
		\item proprietà associativa
		\item proprietà commutativa
		\item esistenza dell'elemento neutro
		\item esistenza dell'elemento opposto
	\end{enumerate}
	\item[-] per il prodotto \(\times\):
	\begin{enumerate}[resume]
		\item proprietà associativa
		\item proprietà commutativa
		\item proprietà distributiva
		\item esistenza dell'elemento neutro
	\end{enumerate}
\end{itemize}

\subsection{Campo}
Un campo \(k\) è un insieme in cui sono definite le due operazioni \(+\) e \(\times\), che soddisfa sia le 8 proprietà dell'anello
commutativo, sia la seguente:
\begin{enumerate}[topsep=3pt, itemsep=0pt] \setcounter{enumi}{8}
	\item esistenza dell'elemento inverso: \(\forall a \in k \; \text{con} \; a \neq 0 \;\; \exists b \in k \; \text{tale che} \; a \cdot b =  b \cdot a = 1\)
\end{enumerate}

In un campo \(k\) valgono le seguenti proprietà:
\begin{enumerate}[topsep=3pt, itemsep=0pt]
	\item si possono risolvere equazioni di primo grado: \(ax + b = 0 \quad \Rightarrow \quad x = -b/a\)
	\item per ogni elemento di \(k\) vale \(a \cdot 0 = 0\)
\end{enumerate}

\subsection{Spazio vettoriale}
Uno spazio vettoriale \(V\) su un campo \(k\) è un insieme di vettori non vuoto dotato delle operazioni:
\begin{enumerate}
	\item somma \(+\): \(\qquad \qquad \qquad \qquad \qquad f : \begin{matrix}
		V \times V &\to &V \\
		(v, w) &\mapsto &v + w
	\end{matrix}\)
	\item prodotto per uno scalare \(\times\): \(\qquad \;\; f : \begin{matrix}
		k \times V &\to &V \\
		(\alpha, v) &\mapsto &\alpha \cdot w
	\end{matrix}\)
\end{enumerate}
Inoltre devovo valere le 8 proprietà di anello commutativo.
Gli elementi di uno spazio vettoriale si chiamano vettori.

\subsubsection*{Proprietà dell'elemento neutro}
Sia \(V\) uno spazio vettoriale su campo \(k\):
\begin{enumerate}[topsep=3pt, itemsep=0pt]
	\item \(0 \in k, \; v \in V \quad \Rightarrow \quad 0 \cdot v = \vec{0}\)
	\item \(\alpha \in k, \; \vec{0} \in V \quad \Rightarrow \quad \alpha \cdot \vec{0} = \vec{0}\)
\end{enumerate}

\subsubsection*{Proprietà dell'opposto}
Sia \(V\) uno spazio vettoriale su campo \(k\) e \(v \in V\), allora vale \(-1 \cdot v = -v\) e \(-v\) è l'opposto di \(v\). \\
Sia \(A\) un anello commutativo di uno spazio vettoriale (spazio vettoriale), allora l'opposto è unico.

\newpage


\section{Spazi vettoriali, sottospazi e vettori}
\subsection{Combinazione lineare di vettori}
Sia \(V\) uno spazio vettoriale su campo \(k\), dati \(v_1, \dots v_n \in V\) vettori e \(\alpha_1, \dots \alpha_n \in k\) scalari
allora possiamo costruire il vettore \(v = \alpha_1 v_1 + \alpha_2 v_2 + \dots + \alpha_n v_n\), chiamato combinazione lineare dei
vettori \(v_1, \dots v_n\).

\subsection{Vettori liearmente indipendenti}
Siano \(v_1, \dots v_n\) vettori di uno spazio vettoriale \(V\), i vettori sono linearmente indipendenti se l'unica soluzione di
\(\alpha_1 v_1 + \dots + \alpha_n v_n = \vec{0}\) è per \(\alpha_1 = \dots = \alpha_n = 0\).

Quando dei vettori non sono linearmente indipendenti, si dicono linearmente dipendenti ed è possibile scriverne uno come combinazione
lineare degli altri.

Se tra i vettori è presente il vettore nullo \(\vec{0}\), allora i vettori saranno linearmente dipendenti.

Se \(v_1, \dots v_n\) sono linearmente dipendenti e aggiungo altri vettori \(v_{n+1}, \dots v_m\), allora saranno ancora linearmente
dipendenti.

\subsection{Sottospazi vettoriali}
Sia \(V\) uno spazio vettoriale su campo \(k\), un sottospazio vettoriale di \(V\) è un sottoinsieme \(W\) di \(V\) che è sempre
sottospazio vettoriale secondo le stesse operazioni di \(V\).

Sia \(W\) un sottoinsieme di uno spazio vettoriale \(V\) (\(W \subseteq V\)), \(W\) è sottospazio vettoriale di \(V\) (\(W \leq V\))
se e solo se valgono le seguenti condizioni:
\begin{enumerate}[topsep=3pt, itemsep=0pt]
	\item \(W\) non è vuoto: \(W \neq \varnothing\)
	\item \(W\) è chiuso per la somma: \(w_1, w_2 \in W \quad \Rightarrow \quad w_1 + w_2 = w_3 \in W\)
	\item \(W\) è chiuso per il prodotto: \(\alpha \in k, \; w_1 \in W \quad \Rightarrow \quad \alpha w_1 = w_2 \in W\)
\end{enumerate}

Sia \(W \leq V\), allora \(\vec{0} \in W\), ovvero l'elemento nullo di \(V\) appartiene anche a \(W\).

Ogni spazio vettoriale \(V\) ha almeno due sottospazi vettoriali, ovvero \(W_1 = \{ \vec{0} \}\) e \(W_2 = V\), per cui
dato un generico sottospazio \(W\) di \(V\), vale \(\{ \vec{0} \} \leq W \leq V\).

\subsubsection*{Unione, somma e intersezione tra sottospazi vettoriali}
Siano \(U, W \leq V\) sottospazi di \(V\), spazio vettoriale su \(k\), vengono definite:
\begin{itemize}[topsep=3pt, itemsep=0pt]
	\item[-] \(U \cap W = \left\{ v \in U \land v \in W \right\}\) è l'intrsezione, si dimostra che \(U \cap W \leq V\)
	\item[-] \(U \cup W = \left\{ v \in U \lor v \in W \right\}\) è l'unione, che si dimostra non essere sottospazi
	\item[-] \(U + W = \left\{ u + w \; \text{t.c.} \; u \in U, \; w \in W\right\}\) è il più piccolo sottospazio vettoriale che
	contiene \(U\) e \(W\)
\end{itemize}

\subsubsection*{Sottospazio generato da vettori generatori}
Sia \(V\) uno spazio vettoriale su campo \(k\) e \(S \subseteq V\) un insieme di vettori di \(V\), allora il sottospazio vettoriale
generato da \(S\) è definito come il più piccolo sottospazio di \(V\) che contiene \(S\) e si indica \(L(S)\) o \(<S>\).

Sia \(S = \left\{ v_1, \dots v_n \right\}\), si dimostra che il sottospazio \(L(S) = \lambda_1 v_1 + \dots + \lambda_n v_n\) con
\(\lambda_1, \dots \lambda_n \in k\) è dato dalla combinazione lineare dei vettori di \(S\) chiamati vettori generatori.

Sia \(V\) spazio vettoriale su campo \(k\), un sottoinsieme \(S = \left\{ v_1, \dots v_n \right\}\) è detto insieme di generatori
di \(V\) se \(L(S) = V\), per cui ogni vettore di \(V\) si può scrivere come combinazione lineare dei vettori di \(S\).

\newpage

\subsection{Base di uno spazio vettoriale}
\begin{itemize}
	\item[-] Sia \(V\) uno spazio vettoriale su \(k\), una base di \(V\) è un sottoinsieme \(B\) di vettori che sono contemporaneamente
	generatori di \(V\) e linearmente indipendenti.
	
	\item[-] Uno stesso spazio vettoriale può avere più basi differenti.
	
	\item[-] Sia \(V\) spazio vettoriale con base \(\left\{ v_1, \dots v_n \right\}\), ogni vettore di \(V\) si scrive in modo
	unico come combinazione lineare dei vettori \(v_1, \dots v_n\).
	
	\item[-] Sia \(V\) spazio vettoriale e \(S = \left\{ v_1, \dots v_n \right\}\) sottoinsieme di \(V\), \(S\) è base di \(V\)
	se e solo se ogni vettore di \(V\) si scrive in modo unico come combinazione lineare dei vettori di \(S\).
	
	\item[-] Sia \(V\) spazio vettoriale con \(B = \left\{ v_1, \dots v_n \right\}\) base di \(V\) e sia \(v \in V\) vettore
	esprimibile come combinazione lineare dei vettori della base: \(v = \alpha_1 v_1 + \dots + \alpha_n v_n = \left( \begin{smallmatrix} \alpha_1 \\ \vdots \\ \alpha_n \end{smallmatrix} \right)^B\),
	con \(\alpha_1, \dots \alpha_n\) coordinate di \(v\) rispetto alla base \(B\).
	
	\item[-] Uno spazio vettoriale si dice finitamente generato se è generato da un numero finito di vettori.
	
	\item[-] Sia \(V\) spazio vettoriale con \(v_1, \dots v_n\) insieme di generatori di \(V\), sia \(w = \alpha_1 v_1 + \dots + \alpha_n v_n\)
	vettore di \(V\), allora \(w, v_1, \dots v_{n-1}\) è insieme di generatori di \(V\).
	
	\item[-] Sia \(V\) spazio vettoriale \(\left\{ v_1, \dots v_n \right\}\) insieme di generatori di \(V\) e \(\left\{ w_1, \dots w_r \right\}\)
	vettori linearmente indipendenti di \(V\), allora \(r \leq n\).
	
	\item[-] Tutte le basi di uno spazio vettoriale hanno lo stesso numero di elementi, per cui si definisce la dimensione di uno
	spazio vettoriale come il numero di elementi di una sua base e si indica con \(\dim V\). Per convenzione \(V = \left\{ \vec{0} \right\}\)
	ha base \(B = \varnothing\) e dimensione \(\dim V = 0\).
	
	\item[-] Sia \(V\) spazio vettoriale e \(S = \left\{ v_1 \dots v_n \right\}\) insieme di generatori, da \(S\) si può estrarre
	una base di \(V\) (tenendo solo vettori linearmente indipendenti), e vale \(\dim V \leq S\).
	
	\item[-] Sia \(V\) spazio vettoriale con \(dim V = n\) e \(S = \left\{ v_1, \dots v_n \right\}\) vettori linearmente indipendenti,
	allora \(S\) può essere completato ad una base di \(V\) (aggiungendo altri vettori linearmente indipendenti).
	
	\item[-] Sia \(V\) spazio vettoriale e siano \(R,S \subseteq V\), allora \(R \subseteq S \; \Rightarrow \; L(R) \leq L(S)\)
	
	\item[-] Sia \(V\) spazio vettoriale con \(\dim V = n\), allora le seguenti proprietà si implicano a vicenda:
	\begin{enumerate}
		\item \(v_1, \dots v_n\) sono linearmente indipendenti
		\item \(v_1, \dots v_n\) generano \(V\)
		\item \(\left\{ v_1, \dots v_n \right\}\) è base di \(V\)
	\end{enumerate}
	
	\item[-] Sia \(V\) uno spazio vettoriale e \(U, W\) sottospazi di \(V\) con \(U \cap W\) e \(U + W\) sottospazi di \(V\), allora
	\(\dim (U + W) = \dim U + \dim W - \dim (U \cap W)\) detta Formula di Grassmann. Nel caso particolare in cui \(U \cap W = \left\{ \vec{0} \right\}\)
	si ha \(\dim (U \cap W) = 0\) e \(\dim (U + W) = \dim U + \dim W\) e i due sottospazi \(U\) e \(W\) si dicono in somma diretta
	e si scrive \(U + W = U \oplus W\).

	\item[-] Un vettore di \(U \oplus W\) si scrive in modo unico nella forma \(u + w\) con \(u \in U\) e \(w \in W\).
\end{itemize}

\newpage


\section{Matrici}
riprendere parte matrici 2x2 (lasciata indietro)

\section{Funzioni e applicazioni lineari}
\dots

\end{document}