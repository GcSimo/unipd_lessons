\section{Analisi a tempo continuo}
\subsection{Sistemi lineari a tempo invariante - LTI}
\subsubsection*{Assunzioni}
Un sistema di dice lineare a tempo invariante (LTI) se:
\begin{itemize}
	\item i coefficienti non cambiano nel tempo
	\item vale il principio di sovrapposizione degli effetti
	\item si ha invarianza nel tempo, ovvero se a partire dalle stesse condizioni una certa causa \(u(t)\) ha sempre lo stesso
	effetto \(x(t)\) anche se viene ritardata \(u(t-t_0)\) di un tempo \(t_0\)
\end{itemize}

\subsubsection*{Sovrapposizione degli effetti}
Per il principio di sovrapposizione degli effetti, dato un sistema LTI \(\dot{x}(t) = Ax(t) + Bu(t)\):
\begin{itemize}
	\item se \(x'(t)\) è soluzione per le condizioni iniziali \(x'(0)\) e ingressi \(u'(t)\)
	\item se \(x''(t)\) è soluzione per le condizioni iniziali \(x''(0)\) e ingressi \(u''(t)\)
	\item se \(x'''(0) = \alpha x'(0) + \beta x''(0)\) e \(u'''(t) = \alpha u'(t) + u''(t)\)
	\[\text{allora si ottiene una nuova soluzione: } \qquad x'''(t) = \alpha x'(t) + x''(t)\]
\end{itemize}
Ciò significa che se si è in grado di scomporre le condizioni iniziali e gli ingressi in due parti \(x'(0),u'(t)\), \(x''(0),u''(t)\)
e si conoscono le soluzioni \(x'(t),x''(t)\) per i due sottoproblemi allora per trovare la soluzione completa è sufficiente
sommare le due soluzioni parziali.

\subsubsection*{Introduzione all'evoluzione libera e alla risposta forzata}
Si definiscono quindi due situazioni per cui risulta più facile calcolare le soluzioni:
\begin{itemize}
	\item \textbf{evoluzione libera}: con condizioni iniziali non nulle \(x(0) \neq 0\) e  ingressi nulli \(u(t) = 0\)
	\item \textbf{risposta forzata}: con condizioni iniziali nulle \(x(0) = 0\) e ingressi non nulli \(u(t) \neq 0\)
\end{itemize}
La soluzione generale del sistema con condizioni iniziali \(x(0)\) dell'evoluzione libera e con ingressi \(u(t)\) della risposta
forzata sarà data dalla somma della soluzione dell'evoluzione libera con quella della risposta forzata.

\newpage

\subsection{Evoluzione libera}
\subsubsection*{Introduzione}
L'evoluzione libera \(x_l(t)\) di un sistema LTI \(\dot{x}(t) = Ax(t) + Bu(t)\) è la soluzione parziale del sistema ottenuta
considerando solo le condizioni iniziali \(x(0) \neq 0\), ponendo gli ingressi a zero \(u(t) = 0\).

\subsubsection*{Calcolo dell'evoluzione libera}
Dato un sistema LTI \(\dot{x}(t) = Ax(t) + Bu(t)\) di cui si vuole analizzare l'evoluzione libera:
\begin{itemize}
	\item[1.] si pongono gli ingressi a zero \(u(t) = 0\) e si scelgono determinate condizioni iniziali \(x(0) \neq 0\), per
	cui l'equazione del sistema diventa \[\dot{x}(t) = Ax(t)\]
	\item[2.] è possibile esprimere il vettore \(x(t)\) come combinazione lineare degli autovettori di \(A\) ottenendo \(x(t) = c_1 v_1 + c_2 v_2 + \dots\)
	per cui l'equazione diventa \[\dot{x}(t) = A(c_i v_1 + c_2 v_2 + \dots)\]
	\item[3.] per la proprietà degli autovettori \(A v_i = \lambda_i v_i\), con \(\lambda_i\) autovalore associato, l'equazione
	diventa \[\dot{x}(t) = Ax(t) = c_1 \lambda_1 v_1 + c_2 \lambda_2 v_2 + \dots\]
	\item[4.] per linearità del sistema si studia ogni contributo indipendentemente e alla fine si sommano tutti i risultati
	ottenuti, la soluzione di un generico contributo (si pone \(x(t) = c_i v_i\)) vale
	\[\dot{x}(t) = c_i \lambda_i v_i \;\; \Leftrightarrow \;\; x_{l,i}(t) = c_i e^{\lambda_i t} v_i \;\; (= x_0 e^{\lambda_i t})\]
	\item[5.] dalla somma di tutti i contributi si ottiene la soluzione dell'evoluzione libera
	\[x_l(t) = \sum_{i=1}^{n} c_i e^{\lambda_i t} v_i \qquad \qquad \text{con } x(0) = \sum_{i=0}^{n} c_i v_i, \quad \begin{array}{l}
		\lambda_i, v_i = \text{autovalori e autovettori di } A \\
		e^{\lambda_i t} v_i =  \text{modo naturale associato a } \lambda_i
	\end{array}\]
\end{itemize}

\subsubsection*{Modi naturali reali}
Dato il modo naturale \(e^{\lambda_i t} v_i\) relativo all'autovalore \(\lambda_i\), se \(\lambda_i \in \mathbb{R}\), il modo
naturale viene detto reale e:
\begin{center}
	\begin{minipage}{0.5\textwidth}
		\begin{itemize}
			\item come traiettoria evolutiva nel diagramma delle fasi ha una retta che passa per il punto \(x(t)\) e ha come vettore
			direttore proprio l'autovettore del modo naturale reale
			\item come evoluzione nel tempo ha un andamento esponenziale che dipende da \(\lambda_i\):
			\begin{itemize}
				\item se \(\lambda_i > 0\) moto esponenziale divergente
				\item se \(\lambda_i = 0\) moto costante 
				\item se \(\lambda_i < 0\) moto esponenziale convergente
			\end{itemize}
			\item ha costante di tempo caratteristica del modo naturale reale: \(\displaystyle\tau_i = -\frac{1}{\lambda_i}\)
		\end{itemize}
	\end{minipage}
	\begin{minipage}{0.49\textwidth}
		\centering
		\includegraphics[width=0.8\textwidth]{immagini/evoluzione libera reale 1.png}
		\includegraphics[width=0.8\textwidth]{immagini/evoluzione libera reale 2.png}
	\end{minipage}
\end{center}
In conclusione l’evoluzione libera è la combinazione lineare di \(n\) modi naturali a cui corrispondono \(n\) traiettorie,
ciascuna delle quali evolve lungo il sottospazio unidimensionale determinato dal corrispondente autovettore \(v_i\).

\newpage

\subsubsection*{Modi naturali complessi}
Se dalla soluzione \(x_l(t)\) si ottengono due autovalori \(\lambda_{k1}, \lambda_{k2}\) complessi coniugati tali che
\(\lambda_{k1,k2} = \sigma_k \pm j\omega_k\) allora si avrà anche degli autovettore complessi coniugati \(v_{k1,k2} = v_{ka} \pm j v_{kb}\).
Il modo naturale complesso associato sarà della forma:
\[x(t) = m_k e^{\sigma_k t} \; [ \sin(\omega_k t + \varphi_k) v_{ka} + \cos (\omega_k t + \varphi_k) v_{kb}]\]
\[\text{con}\quad m_p = \sqrt{{c_{ka}}^2 + {c_{kb}}^2} \qquad x(0) = c_{ka} v_{ka} + c_{kb} v_{kb} \qquad
\varphi = \begin{cases} \arctan (c_{ka} / c_{kb}) & c_{kb} > 0 \\ \arctan (c_{ka} / c_{kb}) + \pi & c_{kb} < 0 \end{cases}\]
Analizzando il modo naturale complesso si osserva che:
\begin{center}
	\begin{minipage}{0.5\textwidth}
		\begin{itemize}
			\item come traiettoria evolutiva nel diagramma delle fasi ha una spirale ellittica che parte per il punto \(x(t)\)
			e che può convergere, divergere o rimanere costante in funzione di \(\sigma_k\)
			\item come evoluzione nel tempo ha un andamento esponenziale oscillatorio che dipende da \(\sigma_k\):
			\begin{itemize}
				\item se \(\sigma_k > 0\) moto oscillante divergente
				\item se \(\sigma_k = 0\) moto periodico 
				\item se \(\sigma_k < 0\) moto oscillante convergente
			\end{itemize}
			\item ha costante di tempo caratteristica del modo naturale complesso : \(\displaystyle\tau_k = -\frac{1}{\sigma_k}\)
		\end{itemize}
	\end{minipage}
	\begin{minipage}{0.49\textwidth}
		\centering
		\includegraphics[width=0.8\textwidth]{immagini/evoluzione libera complessa 1.png}
		\vspace{10pt}
		
		\includegraphics[width=0.9\textwidth]{immagini/evoluzione libera complessa 2.png}
	\end{minipage}
\end{center}

\subsubsection*{Evoluzione libera completa: modi naturali reali + complessi}
Mettendo insieme l'evoluzione libera costituita dai moti naturali reali e quella costituita dai moti naturali complessi si ottiene
la formula generale dell'evoluzione libera (considerando che \(A\) abbia \(n\) autovalori distinti):
\[x_l(t) = \sum_{i=1}^{\mu} c_i e^{\lambda_i t} v_i + \sum_{k=1}^{\nu} m_k e^{\sigma_k t} [ \sin(\omega_k t + \varphi_k) v_{ka} + \cos (\omega_k t + \varphi_k) v_{kb}]\]
\[\text{con} \qquad x(0) = \sum_{i=1}^{\mu} c_i v_i + \sum_{k=1}^{\nu} c_{ka} v_{ka} + c_{kb} v_{kb} \qquad n = \mu + 2\nu\]
In base al posizionamento degli autovalori nel piano complesso si hanno diversi moti evolutivi del sistema:
\begin{center}
	\includegraphics[width=0.6\textwidth]{immagini/evoluzione libera generale.png}
\end{center}

\subsubsection*{Evoluzione libera come esponenziale di una matrice}
L'esponenziale di una matrice è definito sfruttando lo sviluppo di Taylor dell'esponenziale nell'origine
\[e^A = \sum_{i=0}^{+\infty} \frac{A^k}{k!} = I + A + \frac{A^2}{2} + \dots\]
Si dimostra che l'espressione dell'evoluzione libera corrisponde all'esponenziale della matrice \(A\)
\[x_l(t) = \sum_{i=1}^{n} c_i e^{\lambda_i t} v_i = e^{At} x(0) \qquad \qquad \text{con} \;\; x(0) = \sum_{i=1}^{n} c_i v_i\]

\subsection{Risposta forzata}
\subsubsection*{Introduzione}
La risposta forzata \(x_f(t)\) di un sistema LTI \(\dot{x}(t) = Ax(t) + Bu(t)\) è la soluzione parziale del sistema ottenuta
considerando l'evoluzione del sistema di fronte agli ingressi non nulli \(u(t) \neq 0\), ponendo le condizioni
iniziali a zero \(x(0) = 0\).

\subsubsection*{Calcolo della risposta forzata}
Dato un sistema LTI \(\dot{x}(t) = Ax(t) + Bu(t)\) di cui si vuole analizzare la risposta forzata:
\begin{itemize}
	\item[1.] si pongono le condizioni iniziali a zero \(x(0) = 0\) e si analizza il sistema in funzione degli ingressi
	\(u(t) \neq 0\), per cui l'equazione del sistema rimane \[\dot{x}(t) = Ax(t) + Bu(t) \qquad \text{con} \quad x(0) = 0\]
	\item[2.] risolvendo l'equazione differenziale si ottiene che la formula per la risposta forzata vale
	\[x_f(t) = \int_{0}^{t} e^{A(t-\tau)} Bu(\tau) d\tau\]
	\item[3.] analizzando per un certo ingresso fissato \(u(t) = \bar{u}\) e ponendo \(\xi = (t - \tau)\) si ha:
	\[x_f(t) = \int_{0}^{t} e^{A(t-\tau)} Bu(\tau) d\tau = \int_{t}^{0} e^{A\xi} B\bar{u} -d\xi = \left(\int_{0}^{t} e^{A\xi} B d\xi\right) \bar{u}\]
	\item[4.] se la matrice \(A\) è invertibile si ottiene
	\[\int_{0}^{t} e^{A\xi} B d\xi = \left. A^{-1} e^{A\xi} B \right|_0^t = A^{-1} \left(e^{A\xi} - I\right) B \quad \rightarrow \quad x_f(t) = A^{-1} \left(e^{A\xi} - I\right) B \bar{u}\]
	\item[5.] se la matrice \(A\) non è invertibile si usa sovrapposizione degli effetti per ogni ingresso \(1, \dots, m\) (colonne
	della matrice \(B\)) e sarà necessario calcolare l'integrale, in caso è possibile notare che la risposta forzata corrisponde 
	all'evoluzione libera con condizioni iniziali \(B_i\)
	\[x_f(t) = \sum_{i=1}^{m} \left(\int_{0}^{t} e^{A\xi} B_i d\xi\right) \bar{u}_i = \sum_{i=1}^{m} \left(\int_{0}^{t} \left.x_l(t) \right|_{x(0)=B_i} d\xi\right) \bar{u}_i = \sum_{i=1}^{m} \left(\int_{0}^{t} \sum_{j=1}^{n} \left(c_j e^{\lambda_j t} v_j\right)  d\xi\right) \bar{u}_i\]
	\[\text{con} \qquad B_i = \sum_{j=1}^{n} c_j v_j\]
\end{itemize}

\newpage


\subsection{Convoluzione}
\subsubsection*{Risposta impulsiva}
La risposta impulsiva è una funzione matematica che descrive la risposta forzata di un sistema perturbato da un impulso unitario,
ovvero quando \(u(t) = \delta(t)\):
\[h(t) := e^{At}B \qquad\qquad \left. x_f(t) \right|_{u(t) = \delta(t)} = \int_{0}^{t} h(t-\tau) \, \delta(t) \, d\tau= h(t)\]

\subsubsection*{Risposta forzata come prodotto di convoluzione}
La risposta forzata può essere interpretata come un integrale della convoluzione (o prodotto di convoluzione), ovvero la
combinazione tra risposta impulsiva \(h(t)\) e l'ingresso \(u(t)\):
\[x_f(t) = (h * u)(t) := \int_{-\infty}^{+\infty} h(t - \tau) u(\tau) \, d\tau\]

\begin{itemize}
	\item l'impulso di Dirac è l'elemento neutro del prodotto di convoluzione \(x_f(t) = (h * \delta)(t) = h(t)\)
	\item poiché la risposta di un sistema LTI a un impulso di Dirac è la risposta impulsiva \(h(t)\), la risposta forzata a un
	ingresso \(u(t)\) è la somma pesata (tramite \(u(t)\)) delle risposte impulsive \(h(t)\) traslate nel tempo.
\end{itemize}

\subsection{Trasformata di Laplace}
\subsubsection*{Definizione}
La trasformata di Laplace mappa funzioni nel dominio del tempo in funzioni nel dominio dei numeri complessi permettendo di
semplificare i calcoli di analisi dei sistemi.
\[\lap : f \left(\begin{array}{l} \mathbb{R} \to \mathbb{R} \\ t \mapsto f(t) \end{array}\right) \mapsto F \left(\begin{array}{l} \mathbb{C} \to \mathbb{C} \\ s \mapsto F(s) \end{array}\right) \qquad F(s) = \lap[f(t)] := \int_{0^-}^{+\infty} f(t) e^{st} dt\]

\subsubsection*{Osservazioni}
\begin{itemize}
	\item si considera solo la trasformata di Laplace unilatera (solo per tempi positivi o nulli)
	\item il fattore \(e^{st}\) serve per far convergere l'integrale a \(t \to +\infty\)
	\item per rendere l'esponente \(st\) adimensionale, \(s\) ha dimensioni di una frequenza
	\item siccome è un integrale indefinito, l'integrale va trattato con un limite
	\item l'integrale non dipende da \(t\), ma solo dal parametro \(s\), per questo \(F\) dipende solo da \(s\)
	\item se l'integrale converge per un certo valore \(s_0\), allora convergerà per tutti i valori \(\Real(s) > \Real(s_0)\), si 
	definisce, quindi, l'ascissa di convergenza \(\sigma_c\) che separa il semipiano sinistro con \(s\) convergenti da quello
	destro con \(s\) convergenti
\end{itemize}

\subsubsection*{Proprietà}
\begin{itemize}
	\item unicità: se esiste la trasformata è unica e vale \(\lap[f(t)] = F(s) \qquad \lap^{-1}[F(s)] = f(t)\)
	\item linearità: \(\lap[\alpha f(t) + \beta g(t)] = \alpha\lap[f(t)] + \beta\lap[g(t)] = \alpha L(s) + \beta G(s)\)
	\item proprietà della derivata e dell'integrale: \(\lap\left[\frac{d f(t)}{dt}\right] = sF(s) - f(0^-) \qquad \lap\left[\int_0^t f(t) dt \right] = F(s)/s\)
	\item convoluzione: \(\lap[(f*h)(t)] = F(s)H(s)\)
	\item traslazione temporale: \(\lap[f(t-a)] = e^{-as} F(s)\)
	\item traslazione in frequenza: \(\lap[e^{at}f(t)] = F(s-a)\)
	\item teorema del valore iniziale: \(\displaystyle \lim_{t \to 0^+} f(t) = \lim_{s \to +\infty} sF(s)\)
	\item teorema del valore finale: \(\displaystyle \lim_{t \to +\infty} f(t) = \lim_{s \to 0} sF(s)\)
\end{itemize}

\subsubsection*{Tabelle per trasformata e antitrasformata}
\begin{center}
	\begin{tabularx}{\textwidth}{ C | C }
		\textbf{Funzione del tempo} & \textbf{Trasformata di Laplace} \\
		\toprule
		\(\delta(t)\) (impulso di Dirac) & 1 \\
		\midrule
		\(\delta_{-1}(t)\) (gradino unitario) & \(1/s\) \\
		\midrule
		\(\delta_{-2}(t)\) (rampa unitaria) & \(1/s^2\) \\
		\midrule
		\(e^{at}\) (esponenziale) & \(\dfrac{1}{s - a}\) \\
		\midrule
		\(\dfrac{t^{n-1}}{(n-1)!} e^{at}\) (esponenziale polinomiale) & \(\dfrac{1}{(s - a)^n}\) \\
		\midrule
		\(\sin(\omega t)\) (sinusoide) & \(\dfrac{\omega}{s^2 + \omega^2}\) \\
		\midrule
		\(\cos(\omega t)\) (cosinusoide) & \(\dfrac{s}{s^2 + \omega^2}\) \\
		\midrule
		\(\dfrac{1}{\omega_n \sqrt{1 - \zeta^2}} e^{-\zeta \omega_n t} \sin\big(\omega_n \sqrt{1 - \zeta^2} t \big)\) & 
		\(\dfrac{1}{s^2 + 2\zeta \omega_n s + \omega_n^2}\) (fattore trinomio) \\
		\midrule
		\(e^{-at} \cos(\omega t)\) & \(\dfrac{s + a}{(s + a)^2 + \omega^2}\) \\
		\midrule
		\(e^{-at} \sin(\omega t)\) & \(\dfrac{\omega}{(s + a)^2 + \omega^2}\)
	\end{tabularx}
\end{center}

\subsubsection*{Antitrasformata}
Definita la trasformata di Laplace come sopra, l'antitrasformata è definita come segue:
\[\lap[f(t)] = \int_{0^-}^{+\infty} f(t) e^{-st} dt \qquad\qquad \lap^{-1}[F(s)] = \frac{1}{2\pi j} \int_{\sigma-j\infty}^{\sigma+j\infty} F(s) e^{st} ds \qquad \text{con} \; \sigma > \sigma_c\]
Siccome l'integrale dell'antitrasformata è ostico da calcolare, si ricorre all'uso della tabella sopra. Di solito le funzioni
da antitrasformare sono un rapporto tra polinomi in \(s\) per cui si definiscono alcune regole e procedimenti ricondursi alle
forme in tabella:
\begin{itemize}
	\item[1.] si scrive la funzione come rapporto di polinomi \(\mathcal{N}(s)\) e \(\mathcal{D}(s)\)
	\item[2.] si scompone il denominatore esplicitandone le radici (reali o complesse), dette poli
\end{itemize}
\[F(s) = \frac{\mathcal{N}(s)}{\mathcal{D}(s)} = \frac{\mathcal{N}(s)}{(s-\lambda_1)^{\mu_1}(s-\lambda_2)^{\mu_2}\dots(s-\lambda_n)^{\mu_n}} = \dots\]

\subsubsection*{Antitrasformata per poli semplici}
\begin{itemize}
	\item[3.] se i poli hanno molteplicità unitaria, si può dividere la frazione in fratti semplici \(C_i \; / \; (s-\lambda_i)\)
	\[\dots = \frac{C_1}{s-\lambda_1} + \frac{C_2}{s-\lambda_2} + \dots + \frac{C_n}{s-\lambda_n} \qquad \text{con} \qquad C_i = \lim_{s \to \lambda_i} (s-\lambda_i) F(s)\]
	\item[4.] si antitrasformano i fratti semplici individualmente, in caso di poli reali le soluzioni sono esponenziali, in
	caso di poli complessi coniugati, le soluzioni si combinano ottenendo moti oscillatori
	\[\lap^{-1}\left[\frac{C_i}{s-\lambda_i}\right] = C_i e^{\lambda_i t} \qquad\quad C_i e^{\lambda_i t} + {C_i}^* e^{{\lambda_i}^* t} = 2e^{\sigma t} (a \cos(\omega t) - b \sin(\omega t)) \quad \begin{array}{l} C,C^* = a \pm jb \\[5pt] \lambda,\lambda^* = \sigma \pm j\omega \end{array}\]
\end{itemize}

\newpage

\subsubsection*{Antitrasformata per poli multipli}
\begin{itemize}
	\item[3.] se i poli hanno molteplicità maggiore di 1, si considera un residuo per ogni esponente da 1 a \(\mu_i\)
	\[\dots = \dots \frac{C_{i,0}}{(s-\lambda_i)} + \frac{C_{i,1}}{(s-\lambda_i)^2} + \dots + \frac{C_{i,\mu_i-1}}{(s-\lambda_i)^{\mu_i}} + \dots\]
	\begin{align*}
		C_{i,\mu_i-1} &= \lim_{s \to \lambda_i} F(s) (s-\lambda_i)^{\mu_i} &\qquad C_{i,\mu_i-2} &= \lim_{s \to \lambda_i} \left(F(s) - \frac{C_{i,\mu_i-1}}{(s-\lambda_i)^{\mu_i}} \right) (s-\lambda_i)^{\mu_i-1} \\
		\quad &\dots\quad  &\qquad C_{i,0} &= \lim_{s \to \lambda_i} \left(F(s) - \sum_{l=1}^{\mu_i-1} \frac{C_{i,l}}{(s-\lambda_i)^{l+1}} \right) (s-\lambda_i)
	\end{align*}
	\item[4.] si antitrasformano i fratti raggruppando quelli associati ad uno stesso autovalore:
	\[\lap^{-1}\left[ \sum_{l=0}^{\mu_i-1} \frac{C_{i,l}}{(s-\lambda_i)^{l+1}} \right] = \sum_{l=0}^{\mu_i-1} C_{i,l} \frac{t^l}{l!} e^{\lambda_i t}\]
\end{itemize}

\subsubsection*{Moti naturali di poli semplici}
\begin{center}
	\includegraphics[width=0.9\textwidth]{immagini/laplace poli semplici.png}
\end{center}

\subsubsection*{Moti naturali di poli multipli}
\begin{center}
	\includegraphics[width=0.9\textwidth]{immagini/laplace poli multipli.png}
\end{center}

\subsubsection*{Evoluzione libera con Laplace}
Studio dell'evoluzione libera attraverso la trasformata di Laplace
\[\dot{x}_l(t) = Ax_l(t) \;\;\stackrel{\lap}{\longrightarrow}\;\; sX_l(s) - x_l(0) = AX_l(s) \;\;\rightarrow\;\; sX_l(s)-AX_l(s) = x_l(0) \;\;\rightarrow\;\; (sI-A)X_l(s) = x_l(0)\]
\[(sI-A)X_l(s) = x_l(0) \;\;\rightarrow\;\; X_l(s) = (sI-A)^{-1} x_l(0) \;\;\stackrel{\lap^{-1}}{\longrightarrow}\;\; x_l(t) = e^{At} x_l(0)\]

\subsubsection*{Risposta forzata con Laplace}
Studio della risposta forzata attraverso la trasformata di Laplace
\[\dot{x}_f(t) = Ax_f(t) + Bu(t) \;\;\stackrel{\lap}{\longrightarrow}\;\; sX_f(s) - x_f(0) = AX_f(s) + BU(s) \;\;\rightarrow\;\; X_f(s) = (sI-A)^{-1}BU(s)\]
Si definisce la matrice di trasferimento tra ingressi e stato, che è anche la risposta impulsiva del sistema:
\[\frac{X_f(s)}{U(s)} = (sI-A)^{-1}B \qquad\qquad X_f(s) = H(s)U(s) \;\;\rightarrow\;\; \frac{X_f(s)}{U(s)} = H(s) = (sI-A)^{-1}B\]
Se si definiscono le uscite del sistema come \(y(t) = Cx(t) + Du(t)\), allora è possibile definire la matrice di trasferimento tra 
ingressi e uscite: \[\frac{Y(s)}{U(s)} = C(sI-A)^{-1}B + D\]

\subsubsection*{Risposta complessiva (libera + forzata) con Laplace}
Unendo l'evoluzione libera e la risposta forzata trasformate con Laplace si ottiene:
\[X(s) = X_l(s) + X_f(s) = (sI-A)^{-1} x_l(0) + (sI-A)^{-1}BU(s)\]

\subsection{Risposta in frequenza}
\subsubsection*{Risposta in frequenza in un sistema BIBO stabile}
L'analisi della risposta in frequenza consiste nell'analizzare come varia l'ampiezza e la fase tra un segnale in ingresso e un segnale
in uscita di un sistema BIBO-stabile. Si considerano solo sistemi BIBO-stabili in quanto, se perturbati con un segnale sinusoidale ad
una certa frequenza, restituiscono sempre un segnale sinusoidale alla stessa frequenza alterato di ampiezza e fase.

\subsubsection*{Analisi in frequenza e diagrammi di Bode}
Per l'analisi in frequenza si eseguono i seguenti passaggi:
\begin{itemize}
	\item[1.] si calcola la sola risposta forzata di un sistema BIBO stabile e si ottiene la risposta impulsiva del sistema trasformata
	con Laplace (o funzione di trasferimento): \(H(s) = X_f(s) / U(s)\)
	\item[2.] si studia la risposta in frequenza (in uscita) del sistema perturbato da una sinusoide (in ingresso) analizzando come
	varia l'ampiezza e la fase in funzione di \(\omega\):
	\[x_r(t) := \left|H(j\omega)\right| \sin (\omega t + \arg{\left\{H(j\omega)\right\}}) \qquad \text{ampiezza} = \left|H(j\omega)\right| \qquad \text{fase} = \arg{\left\{H(j\omega)\right\}}\]
	\item[3.] è possibile disegnare i grafici che si ottengono studiando ampiezza e fase in funzione della frequenza che prendono il
	nome di diagrammi di Bode; ampiezza \((20\log_{10}(H(j\omega)) = [dB])\) e frequenza \((\log_{10}(\omega))\) sono indicate in
	scala logaritmica per una migliore lettura
\end{itemize}

\subsubsection*{Frequenza o pulsazione di taglio}
Nel caso di sistemi BIBO che fungono da filtri (abbassano l'ampiezza di frequenze sopra o sotto una certa soglia) si definisce la
pulsazione di taglio quando la parte reale uguaglia la parte immaginaria e si ha un'attenuazione dell'ampiezza di \(-3 \; \text{dB}\)
o \(70\%\) e la fase si inverte di \(\pi/4\):
\[\left. H(j\omega) \right|_{\omega = \omega_\text{taglio}} = \frac{\sqrt{2}}{2} = 0.7 \qquad\qquad 20\log_{10}\left(\left. H(j\omega) \right|_{\omega = \omega_\text{taglio}}\right) = -3 \; \text{dB}\]

\newpage


\section{Analisi a tempo discreto}
\subsection{Parallelismo tra tempo continuo e tempo discreto}
\begin{center}
	\begin{tabularx}{\textwidth}{l | C | C}
		& \textbf{tempo continuo} & \textbf{tempo discreto} \\
		\toprule
		\multirow{2}{*}{equazioni} & \(\dot{x}(t) = Ax(t) + Bu(t)\) & \(x(k+1) = Ax(k) + Bu(k)\) \\
		& \(x(t) = x_l(t) + x_f(t)\) & \(x(k) = x_l(k) + x_f(k)\) \\
		\midrule
		condizioni iniziali & \(x_0 = c_1 v_1 + \dots + c_n v_n\) & \(x_0 = c_1 v_1 + \dots + c_n v_n\) \\
		\midrule
		modo naturale & \(e^{\lambda_i t} v_i\) & \({\lambda_i}^k v_i\) \\
		\midrule
		evoluzione & \(\lambda_i < 0\) convergente, \(\lambda_i = 0\) costante, \(\lambda_i > 0\) divergente & \(\left|\lambda_i\right| < 1\) convergente, \(\left|\lambda_i\right| = 1\) costante, \(\left|\lambda_i\right| > 1\) divergente, \(\lambda_i < 0\) alternati \\
		\midrule
		evoluzione libera & \(x_l(t) = \sum_{i=1}^{n} c_i e^{\lambda_i t} v_i = e^{At} x_0\) & \(x_l(k) = \sum_{i=1}^{n} c_i {\lambda_i}^k v_i = A^k x_0\) \\
		\midrule
		\multirow{3}{*}{risposta forzata} & \(x_f(t) = \int_{0}^{t} e^{A(t-\tau)}Bu(\tau) d\tau\) & \(x_f(k) = \sum_{h=0}^{k-1} A^{k-h-1} B u(h)\) \\
		& \(x_f(t) = \left(\int_{0}^{t} e^{A\xi}B d\xi\right) \bar{u}\) & \(x_f(t) = \left(\sum_{\xi=0}^{k-1} A^\xi B\right) \bar{u}\) \\
		& \(x_f(t) = \sum_{i=1}^{p} \left(\int_{0}^{t} e^{A\xi}[B]_i d\xi\right) \bar{u}_i\) & \(x_f(t) = \sum_{i=1}^{p} \left(\sum_{\xi=0}^{k-1} A^\xi [B]_i\right) \bar{u}_i\) \\
		\bottomrule
	\end{tabularx}
\end{center}

\subsubsection*{Andamenti delle evoluzioni (convergenti, divergenti, oscillanti, alterni)}
\begin{center}
	\begin{tabularx}{\textwidth}{C C}
		\midrule
		diagramma delle fasi in tempo continuo & diagramma delle fasi in tempo discreto
	\end{tabularx}
	\includegraphics[width=\textwidth]{immagini/tempo discreto 1.png}
	\vspace{10pt}

	\begin{tabularx}{\textwidth}{C C}
		\midrule
		modi naturali in tempo continuo & modi naturali in tempo discreto
	\end{tabularx}
	\includegraphics[width=\textwidth]{immagini/tempo discreto 2.png}
\end{center}

\subsection{Trasformata Zeta}
\subsubsection*{Definizione - notazione}
La trasformata Zeta mappa equazioni alle differenze in equazioni algebriche, permettendo di semplificare i calcoli nell'analisi dei
sistemi a tempo discreto.
\[\z : f \left(\begin{array}{l} \mathbb{Z} \to \mathbb{R} \\ k \mapsto f(k) \end{array}\right) \mapsto F \left(\begin{array}{l} \mathbb{Z} \to \mathbb{R} \\ z \mapsto F(z) \end{array}\right) \qquad F(z) = \z[f(k)] := \sum_{k=0}^{+\infty} x(k) z^{-k}\]

\subsubsection*{Osservazioni}
\begin{itemize}
	\item il termine \(z^{-k}\) corrisponde ad un ritardo temporale di \(k\) passi e serve per \say{posizionare} i valori del campione
	\(x(k)\) nella serie (es. \(x(k)=[4,2,0,5] \rightarrow X(z) = 4z^{-0} + 2z^{-1} + 0z^{-2} + 5z^{-3}\))
	\item è possibile osservare che la trasformata Zeta corrisponde alla trasformata di Laplace di un segnale continuo campionato
	idealmente ogni periodo \(T\)
	\begin{align*}
		x(k) \leftrightarrow x_q(t) &= x(0)\delta(t) + x(1)\delta(t-T) + x(2)\delta(t-2T) + x(3)\delta(t-T3) + \dots + x(k)\delta(t-kT) \\
		X(z) = \z[x(k)] &= x(0) + x(1)z^{-1} + x(2)z^{-2} + x(3)z^{-3} + \dots + x(k)z^{-k} \\
		X_q(s) = \lap[x_q(t)] &= x(0) + x(1)e^{-sT} + x(2)e^{-2sT} + x(3)e^{-3sT} + \dots + x(k)e^{-skT}
	\end{align*}
	\[z = e^{st} \qquad\qquad X_q(s) = \left. X(z) \right|_{z = e^{st}}\]
\end{itemize}

\subsubsection*{Proprietà}
\begin{itemize}
	\item unicità: se esiste la trasformata è unica e vale \(\z[x(k)] = X(z) \qquad \z^{-1}[X(z)] = x(k)\)
	\item linearità: \(\z[\alpha x(k) + \beta y(k)] = \alpha\z[x(k)] + \beta\z[y(k)] = \alpha L(z) + \beta Y(z)\)
	\item moltiplicazione per \(k\) e \(k^2\): \(\lap\left[kx(k)\right] = -z\frac{dX(z)}{dz} \qquad \lap\left[k^2 x(k)\right] = z\frac{dX(z)}{dz} + z^2\frac{d^2X(z)}{dz^2}\)
	\item ritardo temporale: \(\z[x(k-1)] = z^{-1} X(z)\)
	\item anticipo temporale: \(\z[x(k+1)] = zX(z)- zx(0)\)
	\item convoluzione: \(\z[(x_1*x_2)(k)] = X_1(z)X_2(z)\)
	\item moltiplicazione per esponenziale: \(\z[\lambda^kx(k)] = X(z/\lambda)\)
\end{itemize}

\subsubsection*{Tabelle per trasformata e antitrasformata}
\begin{center}
	\begin{tabularx}{\textwidth}{ c | C }
		\textbf{Funzione del tempo} & \textbf{Trasformata di Laplace} \\
		\toprule
		\(\delta(k)\) (impulso di Kronecker - unitario discreto) & 1 \\
		\midrule
		\(\delta(k-i)\) (impulso di Kronecker traslato) & \(z^{-i}\) \\
		\midrule
		\(\delta_{-1}(k)\) (gradino unitario discreto) & \(z/(z-1)\) \\
		\midrule
		\(\lambda^k\delta_{-1}(k)\) (successione esponenziale causale) & \(z/(z-\lambda)\) \\
		\midrule
		\(k\lambda^k\delta_{-1}(k)\) (\(k\) volte successione esponenziale causale) & \(z\lambda/(z-\lambda)^2\) \\
		\midrule
		\(k^2\lambda^k\delta_{-1}(k)\) (\(k^2\) volte successione esponenziale causale) & \(z\lambda(z+\lambda)/(z-\lambda)^3\) \\
		\midrule
		\(A \cos(\theta k + \phi)\, \delta_{-1}(k)\) (successone sinusoidale causale) & \(A \frac{z\left[z\cos(\phi)-\cos(\phi-\theta)\right]}{z^2 - 2z\cos\theta + 1}\)
	\end{tabularx}
\end{center}

\newpage

\subsubsection*{Calcolo della trasformata}
Per il calcolo della trasformata di una successione si sfrutta il ruolo di \(z^{-k}\) per la trasformata zeta
\begin{align*}
	x(k) = [4,2,0,5] \;\;&\rightarrow\;\; x(k) = 4\delta(k) + 2\delta(k-1) + 0\delta(k-2) + 5\delta(k-3) \;\;\rightarrow \\
	&\rightarrow\;\; X(z) = \sum_{k=0}^{\infty} x(k)z^{-k} = 4 + 2z^{-1} + 0z^{-2} + 5z^{-3} = \frac{4z^3+2z^2+5}{z^3}
\end{align*}

\subsubsection*{Calcolo dell'antitrasformata}
Si sfruttano tecniche analoghe alla antitrasformata di Laplace:
\begin{itemize}
	\item[1.] si scrive la funzione come rapporto irriducibile di polinomi 
	\item[2.] si scompone il rapporto in fratti semplici
	\item[3.] si antitrasformano i fratti semplici singolarmente come con Laplace
\end{itemize}

\subsubsection*{Evoluzione libera con trasformata Zeta}
Studio dell'evoluzione libera attraverso la trasformata Zeta
\[x_l(k) = Ax_l(k) \;\;\stackrel{\z}{\longrightarrow}\;\; zX_l(k) - zx_l(0) = AX_l(z) \;\;\rightarrow\;\; zX_l(z)-AX_l(z) = zx_l(0)\]
\[zX_l(z)-AX_l(z) = zx_l(0) \;\;\rightarrow\;\; (zI-A)X_l(z) = zx_l(0) \;\;\rightarrow\;\; X_l(z) = (zI-A)^{-1} zx_l(0)\]

\subsubsection*{Risposta forzata con trasformata Zeta}
Studio della risposta forzata attraverso la trasformata zeta
\[x_f(k+1) = Ax_f(k) + Bu(k) \;\;\stackrel{\z}{\longrightarrow}\;\; zX_f(z) - zx_f(0) = AX_f(z) + BU(z) \;\;\rightarrow\;\; X_f(z) = (zI-A)^{-1}BU(z)\]
