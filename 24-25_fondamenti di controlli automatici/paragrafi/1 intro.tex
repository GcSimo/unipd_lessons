\section{Introduzione ai controlli automatici}
\begin{itemize}
	\item \textbf{automatica}: studia, a livello teorico, sistemi e soluzioni in grado di regolarsi automaticamente per raggiungere un obiettivo
	\item \textbf{controlli automatici}: strumenti necessari per utilizzare l'automatica in applicazioni pratiche
	\item \textbf{automazione}: applicazione dell'automatica
\end{itemize}

\section{Nozioni di segnali}
\subsection{Concetto di segnale}
\begin{itemize}
	\item \textbf{segnale}: grandezza che evolve in funzione di una o più variabili indipendenti a cui è associata una informazione
	di una qualche natura
	\item \textbf{segnale fisico}: segnale che rappresenta la grandezza fisica di un'informazione ed è legato alla realtà, sono
	misurati attraverso sensori e in genere rappresentano le entrate o le uscite
	\item \textbf{segnale matematico}: segnale che modella segnali fisici, sono ottenuti da funzioni matematiche e vengono analizzati
	per conoscere l'evoluzione del sistema
\end{itemize}

\subsection{Classificazione dei segnali matematici}
\subsubsection*{Dominio - tempo continuo e discreto}
Il dominio è l'insieme dei valori assunti dalla variabile indipendente. Un segnale con una singola variabile indipendente è detto segnale
monodimensionare, uno con più variabili indipendenti è detto segnale multidimensionale. In base al dominio il segnale può essere:
\begin{itemize}
	\item a tempo continuo \(D \subseteq \mathbb{R}\)
	\item a tempo discreto \(D \subseteq \mathbb{Z}\)
\end{itemize}


\subsubsection*{Codominio - segnale analogico e digitale}
Il codominio è l'insieme dei valori assunti dalla variabile dipendente. Un segnale con singola variabile dipendente è detto segnale
scalare in quanto assume valori scalari, uno con più variabili dipendenti è detto segnale vettoriale in quanto assume valori vettoriali.
In base al codominio il segnale può essere:
\begin{itemize}
	\item analogico (ampiezza continua) \(I \subseteq \mathbb{R}\)
	\item digitale (ampiezza discreta) \(I \subseteq \mathbb{Z}\)
\end{itemize}

\subsubsection*{Tabella riassuntiva}
\begin{center}
	\includegraphics[width=0.8\textwidth]{immagini/classificazione segnali.png}
\end{center}
Un segnale digitale a tempo continuo è anche detto segnale quantizzato, mentre un segnale digitale a tempo discreto è anche detto
segnale numerico.

\subsubsection*{Altre proprietà dei segnali}
\begin{itemize}
	\item \textbf{segnali pari}: simmetrici rispetto all'asse delle ascisse, \(x(t) = x(-t)\)
	\item \textbf{segnali dispari}: simmetrici rispetto all'origine, \(x(t) = -x(-t)\)
	\item \textbf{segnali causali}: segnali nulli per tempo negativo, \(x(t) = 0 \;\; \forall t < 0\)
	\item \textbf{segnali periodici}: segnali che si ripetono ogni periodo \(T\), \(x(t) = x(t+T)\)
\end{itemize}

\subsection{Impulso di Dirac}
L'impulso di Dirac \(\delta_0(t)\) è definito come la derivata del gradino unitario, ovvero è una distribuzione che esibisce la proprietà di
avere integrale unitario su tutto il dominio. È rappresentata graficamente come una freccia verticale e può essere vista come un
rettangolo alto e molto stretto di area unitaria.

\begin{center}
	\begin{minipage}{0.45\textwidth}
		\[\delta_0(t) := \frac{d}{dt}\delta_{-1}(t) \qquad \int_{-\infty}^{+\infty} \delta_0(t) = 1\]
	\end{minipage}
	\begin{minipage}{0.2\textwidth}
		\centering
		\includegraphics[width=\textwidth]{immagini/dirac1.png}
	\end{minipage}
	\begin{minipage}{0.05\textwidth}
		\centering
		\(\Leftrightarrow\)
	\end{minipage}
	\begin{minipage}{0.2\textwidth}
		\centering
		\includegraphics[width=\textwidth]{immagini/dirac2.png}
	\end{minipage}
\end{center}

\subsection{Segnali canonici causali a tempo continuo}
\begin{center}
	\begin{tabularx}{\textwidth}{l | l | X}
		\textbf{tipo di segnale} & \textbf{formula matematica} & \textbf{grafico} \\
		\toprule
		impulso di Dirac & \(\delta_0(t)\) & \includegraphics[width=0.25\textwidth]{immagini/dirac1.png} \\
		\midrule
		gradino unitario & \(\delta_{-1}(t) = \begin{cases} 0 & t < 0 \\ 1 & t \geq 0 \end{cases}\) & \includegraphics[width=0.25\textwidth]{immagini/gradino.png} \\
		\midrule
		rampa & \(\delta_{-2}(t) = \begin{cases} 0 & t < 0 \\ t & t \geq 0 \end{cases}\) & \includegraphics[width=0.25\textwidth]{immagini/rampa.png} \\
		\midrule
		rampa parabolica & \(\delta_{-3}(t) = \begin{cases} 0 & t < 0 \\ t^2/2 & t \geq 0 \end{cases}\) & \includegraphics[width=0.25\textwidth]{immagini/rampa parabolica.png} \\
	\end{tabularx}
\end{center}
Si osserva che per i segnali causali elencati sopra vale la seguente proprietà:
\[\delta_0(t) \quad \begin{array}{c} \stackrel{\int}{\longrightarrow} \\ \stackrel{\longleftarrow}{^d\!/\!_{dt}} \end{array} \quad
\delta_{-1}(t) \quad \begin{array}{c} \stackrel{\int}{\longrightarrow} \\ \stackrel{\longleftarrow}{^d\!/\!_{dt}} \end{array} \quad
\delta_{-2}(t) \quad \begin{array}{c} \stackrel{\int}{\longrightarrow} \\ \stackrel{\longleftarrow}{^d\!/\!_{dt}} \end{array} \quad
\delta_{-3}(t)\]

\subsubsection*{Altri segnali canonici (non causali) a tempo continuo}
\begin{center}
	\begin{tabularx}{\textwidth}{l | l | X}
		\textbf{tipo di segnale} & \textbf{formula matematica} & \textbf{grafico} \\
		\toprule
		segnali sinusoidali & \(\begin{aligned} x(t) &= A \cos(\omega t + \varphi) \\ \omega &= 2 \pi f \quad f = \, ^1\!/\!_T \end{aligned}\) & \includegraphics[width=0.3\textwidth]{immagini/sinusoide.png} \\
		\midrule
		\parbox{3.5cm}{segnale esponenziale \\ monotono crescente} & \(x(t) = e^{\sigma t} \quad \sigma > 0\) & \includegraphics[width=0.3\textwidth]{immagini/esponenziale crescente.png} \\
		\midrule
		\parbox{3.5cm}{segnale esponenziale \\ monotono decrescente} & \(x(t) = e^{\sigma t} \quad \sigma < 0\) & \includegraphics[width=0.3\textwidth]{immagini/esponenziale decrescente.png} \\
		\midrule
		\parbox{3.5cm}{segnale esponenziale \\ costante} & \(x(t) = e^{\sigma t} \quad \sigma = 0\) & \includegraphics[width=0.3\textwidth]{immagini/costante.png} \\
		\midrule
		\parbox{3.5cm}{segnale sinusoidale \\ crescente in modo \\ esponenziale} & \(\begin{aligned} x(t) = e^{\sigma t} &A \cos(\omega t + \varphi) \\ \sigma &> 0 \end{aligned}\) & \includegraphics[width=0.3\textwidth]{immagini/sinusoide crescente esponenzialmente.png} \\
		\midrule
		\parbox{3.5cm}{segnale sinusoidale \\ decrescente in modo \\ esponenziale} & \(\begin{aligned} x(t) = e^{\sigma t} &A \cos(\omega t + \varphi) \\ \sigma &< 0 \end{aligned}\) & \includegraphics[width=0.3\textwidth]{immagini/sinusoide decrescente esponenzialmente.png} \\
		\midrule
		\parbox{3.5cm}{segnale sinusoidale \\ \say{costante in modo \\ esponenziale}} & \(\begin{aligned} x(t) = e^{\sigma t} &A \cos(\omega t + \varphi) \\ \sigma &= 0 \end{aligned}\) & \includegraphics[width=0.25\textwidth]{immagini/sinusoide costante.png}
	\end{tabularx}
\end{center}

\subsubsection*{Considerazioni sui segnali sinusoidali}
Un segnale sinusoidale decrescente/crescente in modo esponenziale può essere rappresentato attraverso un numero complesso sfruttando
l'esponenziale di un numero complesso e la formula di Eulero \[x(t) = e^{st} = e^{(\sigma + j\omega)t} = e^{\sigma t} \cos(\omega t) + j e^{\sigma t} \sin(\omega t) = e^{\sigma t} A \cos(\omega t + \varphi)\]
I sistemi dinamici lineari hanno, come soluzioni, forme esponenziali e sinusoidali crescenti/decrescenti in modo esponenziale.

\subsubsection*{Considerazioni a tempo discreto}
Il \textit{Delta di Dirac} viene chiamato impulso unitario o \textit{Delta di Kronecker}, per il resto si usano sommatorie al posto
degli integrali e qualche denominazione differente che abbiamo trascurato.


\newpage

\section{Nozioni di sistemi}
\subsection{Concetto di sistema}
\begin{itemize}
	\item un sistema è un oggetto o un insieme di oggetti (elementi, fenomeni, progetti, componenti e sottosistemi) con relazioni
	organizzate (interazioni, interfacce, entrate, uscite) tra le unità interne o componenti che compongono un insieme unificato
	\item è delimitato da un confine ed è circondato ed influenzato dall'ambiente
	\item è descritto dai suoi confini, dalla sua struttura e dal suo scopo e si esprime nel suo funzionamento
\end{itemize}

\subsection{Parti di un sistema}
\begin{itemize}
	\item \textbf{componente}: parte irriducibile o un aggregato di parti, noto anche come sottosistema
	\item \textbf{connessione tra componenti}: relazione tra la funzione di un componente e le funzioni di altri componenti
	\item \textbf{confine - contorno}: separazione tra un componente e l'altro all'interno del sistema o separazione tra sistema
	e ambiente esterno
	\item \textbf{scopo}: obiettivo del sistema
	\item \textbf{ambiente}: tutto ciò che è esterno al sistema
	\item \textbf{interfacce}: punti in cui il sistema entra in contatto con l'ambiente esterno
	\item \textbf{ingresso}: segnali di input dall'ambiente esterno
	\item \textbf{uscita}: segnale di output o risultato restituito all'ambiente esterno
	\item \textbf{vincoli}: limiti fisici, strutturali, ... che il sistema deve sopportare
\end{itemize}

\subsection{Sistema dinamico}
Un sistema dinamico è un sistema che evolve nel tempo ed è composto da:
\begin{itemize}
	\item \textbf{stato}: insieme di variabili che descrivono completamente il sistema
	\item \textbf{ingressi}: variabili che influenzano l'evoluzione dello stato
	\item \textbf{uscite}: variabili che descrivono lo stato (parzialmente o completamente)
	\item \textbf{leggi}: formule matematiche che descrivono l'evoluzione del sistema e le relazioni tra input/output
\end{itemize}

\subsection{Sistema dinamico lineare}
Un sistema dinamico si dice lineare se vale il principio di sovrapposizione degli effetti, ovvero se il risultato finale è una
ottenuto tramite combinazione lineare degli eventi agenti sul sistema, indipendentemente dall'ordine con cui si verificano.
