\section{Proprietà strutturali}
\subsection{Stabilità}
\subsubsection*{Equazione dell'equilibrio}
\begin{itemize}
	\item l'equilibrio di un sistema a tempo continuo per ingressi fissati \(u(t) = u_e\):
	\[\dot{x}(t) = Ax(t) + Bu(t) = 0 \;\; \rightarrow \;\; x_e = A^{-1} Bu_e\]
	\item l'equilibrio di un sistema a tempo discreto per ingressi fissati \(u(k) = u_e\):
	\[x(k+1) = Ax(k) + Bu(k) = x(k) \;\; \rightarrow \;\; x_e = (I-A)^{-1} Bu_e\]
	\item l'origine dello spazio di stato è sempre punto di equilibrio del sistema
\end{itemize}

\subsubsection*{Classificazione dei punti di equilibrio}
\begin{itemize}
	\item si definisce un punto di equilibrio \(x_e\):
	\begin{itemize}
		\item \textbf{asintoticamente stabile}: se l'evoluzione libera per qualsiasi stato iniziale \(x_0\) tende a \(x_e\)
		sul lungo periodo
		\item \textbf{instabile}: se esiste uno stato iniziale \(x_0\) per cui l'evoluzione libera tende ad allontanarsi
		definitivamente da \(x_e\) sul lungo periodo
		\item \textbf{marginalmente stabile}: se non è né asintoticamente stabile, né marginalmente stabile, in genere è
		dovuto alla presenza di un polo nullo di molteplicità unitaria
		\item \textbf{instabilità debole}: se è presente un polo nullo di molteplicità maggiore di 1
	\end{itemize}
	\item in un sistema lineare, tutti i punti di equilibrio di tale sistema sono caratterizzati allo stesso modo per cui è possibile
	classificare un sistema come asintoticamente stabile, instabile, marginalmente stabile o instabilità debole, in base a come
	si caratterizza un generico punto di equilibrio (ad esempio l'origine)
\end{itemize}
\begin{center}
	\begin{tabularx}{\textwidth}{C | C | C}
		\textbf{punto di equilibrio} & \textbf{tempo continuo} & \textbf{tempo discreto} \\
		\toprule
		asintoticamente stabile & \(\Real(\lambda) < 0 \quad \forall \lambda\) & \(\left|\Real(\lambda)\right| < 1 \quad \forall \lambda\) \\
		\midrule
		instabile & \(\exists \lambda : \Real(\lambda) > 0\) & \(\exists \lambda : \left|\Real(\lambda)\right| > 1\) \\
		\midrule
		marginalmente stabile & \(\exists! \lambda : \Real(\lambda) = 0\) & \(\exists! \lambda : \left|\Real(\lambda)\right| = 1\) \\
	\end{tabularx}
\end{center}

\subsubsection*{Criterio di Cartesio, di Routh e corrispondenza a tempo discreto}
Per individuare se un sistema è stabile, basta vedere se nelle soluzioni del polinomio caratteristico sono presenti inversioni
di segno. Se non sono presenti inversioni, vuol dire che tutte le soluzioni (o poli) sono negative e il sistema è stabile, se
è presente anche solo una inversione, vuol dire che una soluzione ha segno positivo e il sistema è instabile.

Per individuare le inversioni in un polinomio caratteristico di secondo grado si usa la regola di Cartesio: il numero di radici
reali positive è uguale al numero di variazioni di segno tra coefficienti consecutivi non nulli del polinomio. Ovvero se non ci
sono inversioni nel segno dei coefficienti, tutte i poli sono negativi e il sistema è stabile (o marginalmente stabile).

Nel caso di un polinomio caratteristico di grado superiore a 2, si utilizza la tabella di Routh, per i sistemi a tempo continuo.
Il criterio di Routh-Hurwitz è una condizione necessaria e sufficiente per verificare che tutte le radici del polinomio
caratteristico sono strettamente negative. Se tale condizione è verificata, si dice che il polinomio è Hurwitziano.

Per il tempo discreto esisterebbe il criterio di Jury, soltanto che noi approfittiamo della trasformazione bilineare dal piano
\(z\) al piano \(s\) per poter applicare il criterio di Routh-Hurwitz: \(\lambda=\frac{1+\omega}{1-\omega} \quad \omega=\frac{\lambda-1}{\lambda+1}\)
Il polinomio caratteristico a tempo discreto \(p(\lambda)\) ha tutte le radici all'interno del disco di raggio unitario se e
solo se \(q(\omega)\) ha tutte le radici a parte reale negativa (se è Hurwitziano).

Il polinomio caratteristico compare al denominatore di \((sI-A)^{-1}\) o di \((zI-A)^{-1}\) nelle analisi dei sistemi nei domini
di Laplace e di \(z\).

\newpage

\subsubsection*{BIBO stabilità e BIBS stabilità}
\begin{itemize}
	\item un sistema si dice BIBS-stabile (Bounded Input - Bounded State) se in corrispondenza di un ingresso limitato,
	l'evoluzione dello stato è limitata, ovvero se il denominatore della funzione di trasferimento ingresso-stato è
	Hurwitziano/Jury-stabile:
	\[T(s) = \frac{X(s)}{U(s)} = (sI-A)^{-1}B = \frac{\text{adj}(sI-A)}{\det{(sI-A)}}B \;\; \rightarrow \;\; \det{(sI-A)} = p_a(s) = \text{Hurwitziano}\]
	\[T(z) = \frac{X(z)}{U(z)} = (zI-A)^{-1}B = \frac{\text{adj}(zI-A)}{\det{(zI-A)}}B \;\; \rightarrow \;\; \det{(zI-A)} = p_a(z) = \text{Jury-stabile}\]
	\item un sistema di dice BIBO-stabile (Bounded Input - Bounded Output) se in corrispondenza di un ingresso limitato,
	le evoluzioni delle uscite del sistema sono limitate, ovvero se il denominatore della funzione di trasferimento
	ingresso-uscite è Hurwitziano/Jury-stabile:
	\[G(s) = \frac{Y(s)}{U(s)} = C(sI-A)^{-1}B + D = C\frac{\text{adj}(sI-A)}{\det{(sI-A)}}B + D \;\; \rightarrow \;\; \det{(sI-A)} = p_a(s) = \text{Hurwitziano}\]
	\[G(z) = \frac{Y(z)}{U(z)} = C(zI-A)^{-1}B + D = C\frac{\text{adj}(zI-A)}{\det{(zI-A)}}B + D \;\; \rightarrow \;\; \det{(zI-A)} = p_a(z) = \text{Jury-stabile}\]
	\item \textbf{teorema (evoluzione libera in tempo continuo)}: il sistema è asintoticamente stabile se e solo se tutte le radici
	del denominatore hanno parte reale negativa. Il sistema è marginalmente stabile se le radici del denominatore hanno parte reale
	negativa e ne esiste qualcuna con parte reale nulla ma di molteplicità unitaria. In tutti gli altri casi il sistema è debolmente
	instabile o instabile
	\item \textbf{teorema (evoluzione forzata in tempo continuo)}: il sistema è BIBO stabile se e solo se tutte le radici del
	denominatore della funzione di trasferimento G(s), hanno parte reale negativa (cioè il polinomio è di Hurwitz)
	\item \textbf{teorema (evoluzione libera in tempo discreto)}: il sistema è asintoticamente stabile se e solo se tutte le radici
	del denominatore hanno modulo inferiore a 1. Il sistema è marginalmente stabile se le radici del denominatore hanno modulo minore
	di 1 e ne esiste qualcuna con modulo unitario ma di molteplicità unitaria. In tutti gli altri casi il sistema è debolmente
	instabile o instabile
	\item \textbf{teorema (evoluzione forzata in tempo discreto)}: il sistema è BIBO stabile se e solo se tutte le radici del
	denominatore della funzione di trasferimento G(s), hanno modulo inferiore di 1 (cioè il polinomio è Jury-stabile)
	\item ciò che cambia tra l'evoluzione libera e quella forzata è che si potrebbero avere delle cancellazioni zero-polo quando
	si moltiplica l'inversa della matrice \((sI-A)^{-1}\) per l'uscita \(U(s)\), per cui si potrebbero avere poli che compaiono
	nell'evoluzione libera, ma che non compaiono nella risposta forzata
	\item si osserva che gli zeri dei denominatori sono detti poli della funzione di trasferimento e \(G(s)\), \(G(z)\) sono
	rispettivamente la trasformata di Laplace e la trasformata \(Z\) delle risposte impulsive a tempo continuo e tempo discreto
	\item se il sistema è asintoticamente stabile (e quindi BIBS-BIBO), è possibile applicare il teorema del valore finale per
	calcolare lo stato di equilibrio per \(t \to \infty\)
\end{itemize}

\newpage

\subsection{Raggiungibilità}
\begin{itemize}
	\item la raggiungibilità è una proprietà che descrive le possibilità di modificare lo stato \(x(t)\) del sistema a partire
	da un particolare stato iniziale \(x0\) agendo opportunamente sull’ingresso \(u(t)\)
	\item un sistema è raggiungibile (ovvero se non esistono stati non raggiungibili) se e solo se il rango della matrice di
	raggiungibilità (detta matrice di Kalman) è pari a \(n\), ovvero alla dimensione dello spazio di stato dove la matrice di
	raggiungibilità è: \[\mathcal{R} := [B, AB, \dots, A^{n-1}B] \qquad\qquad \text{rank} \mathcal{R} = n \;\; \Leftrightarrow \;\; \text{sistema raggiungibile}\]
\end{itemize}

\subsection{Controllabilità}
\begin{itemize}
	\item la controllabilità è la proprietà che descrive la possibilità di portare lo stato del sistema da un qualsiasi stato
	iniziale ad uno stato finale desiderato, in un tempo finito, mediante un opportuno segnale di ingresso \(u\)
	\item un sistema LTI è controllabile (tutti gli stati possono essere controllati) se è raggiungibile, infatti sono proprietà
	equivalenti per sistemi LTI, ma non in generale
\end{itemize}

\subsection{Osservabilità}
\begin{itemize}
	\item l'osservabilità descrive la possibilità di stimare lo stato iniziale del sistema mediante la misura dell'uscita \(y\) e
	dell'ingresso \(u\) per un certo istante di tempo \(t\)
	\item un sistema è osservabile (non esistono stati non osservabili) se e solo se il rango della matrice di osservabilità è pari
	a \(n\), cioè alla dimensione dello spazio di stato, la matrice di osservabilità è definita come:
	\[\mathcal{O}:= \left[\begin{matrix}
		C \\ CA \\ \vdots \\ C A^{n-1}
	\end{matrix}\right] \qquad\qquad \text{rank} \mathcal{O} = n \;\; \Leftrightarrow \;\; \text{sistema osservabile}\]
	osservazione: un sistema, per essere osservabile, deve aver definito delle uscite \(y(t) = Cx(t) + D\)
\end{itemize}

\newpage

\section{Controlli}
\subsection{Struttura dei controlli}
\subsubsection*{Controllo in catena aperta (azione diretta, open-loop, feed-forward)}
\begin{itemize}
	\item si costruisce un sistema con un blocco controllore \(C\) in serie al blocco del sistema \(G\) da controllare
	\item si stabilisce un desiderio o setpoint \(r(t)\) come ingresso del controllore, che produrrà un certo ingresso \(u(t)\)
	opportuno per il sistema da controllare, affinché l'uscita \(y(t) = r(t)\)
	\[Y(s) = R(s)C(s)G(s) = R(s) \;\; \rightarrow \;\; T(s) = \frac{Y(s)}{R(s)} = \frac{R(s)}{R(s)} = 1 \;\; \rightarrow \;\; C(s) = G^{-1}(s)\]
	\item il controllo in questo tipo è molto instabile in quanto non si possono correggere eventuali errori/disturbi
\end{itemize}

\subsubsection*{Controllo in retroazione (feedback) dall'uscita}
\begin{itemize}
	\item si costruisce un sistema a catena aperta in cui è presente un ramo che dalle uscite viene sottratto al desiderio \(r(t)\)
	in modo da ottenere uno scostamento dal desiderio \(e(t)\)
	\[Y(s) = \left[R(s)-Y(s)\right] C(s) G(s) \;\; \rightarrow \;\; W_\text{riferimento}(s) = \frac{Y(s)}{R(s)} = \frac{C(s)G(s)}{1+C(s)G(s)}\]
	\[W_\text{errore}(s) = \frac{E(s)}{R(s)} = \frac{1}{1+C(s)G(s)} \qquad\quad W_\text{disturbo}(s) = \frac{Y(s)}{D(s)} = \frac{1}{1+C(s)G(s)}\]
	\item si osserva che per \(C(s)G(s) \gg 1\) si ha \(W_r(t) \approx 1\) (l'uscita insegue l'ingresso), \(W_e(t) \approx 0\)
	(errore trascurabile rispetto al desiderio), \(W_d(s) \approx 0\) (effetto dei disturbi trascurabile)
	\item il controllo di questo tipo è approccio più complicato (richiede sensori di misura di \(y(t)\)), ma è più robusto a
	incertezze e/o disturbi esterni
\end{itemize}

\subsubsection*{Osservazione su zeri e poli nei controlli in retroazione}
Definita \(L(s) := C(s)G(s)\) si osserva che la funzione di trasferimento diventa \(W_r(s) = \frac{L(s)}{1+L(s)}\). Analizzando il
numeratore e denominatore della funzione \(W_r(s)\) si nota che:
\begin{itemize}
	\item la retroazione non cambia la posizione degli zeri rispetto ad un sistema a catena aperta
	\item la retroazione cambia la posizione dei poli (solo quelli raggiungibili e osservabili) rispetto ad un sistema a catena aperta
	\item se è presente un polo \say{instabile} in \(G(s)\), bisogna stare attenti a non cancellarlo con uno zero di \(C(s)\) in quanto
	nella teoria si ottiene un sistema stabile, ma nella pratica non è detto che la semplificazione avvenga in maniera esatta e si potrebbe
	avere un sistema instabile
\end{itemize}

\subsection{Analisi delle specifiche dinamiche durante il transitorio}
\subsubsection*{Sistemi elementari di primo ordine}
Dato un sistema elementare di primo ordine con \(\displaystyle G(s) = \frac{1}{s-p_1} = \frac{K_1}{T_1s + 1}\)
\begin{itemize}
	\item la costante di tempo \(T_1 := -1/p_1\) associata ad un polo caratterizza il comportamento dinamico nel transitorio,
	maggiore è \(T_1\), maggiore sarà la durata del transitorio e viceversa
	\item il guadagno statico \(K_1 := T_1\) associato ad un polo, caratterizza il comportamento asintotico del sistema, maggiore
	è \(K_1\), maggiore sarà il valore asintotico a cui tenderà il sistema a regime
\end{itemize}
In caso di presenza di uno zero al numeratore \(\displaystyle G(s) = \frac{s-z_1}{s-p_1} = \frac{\hat{T}_1s + 1}{T_1s + 1} = \alpha + \frac{1-\alpha}{T_1s+1}\)
con \(\alpha = \hat{T_1}/T_1\) si nota che variando la costante di tempo dello zero \(\hat{T}_1\) (ovvero variandone la posizione)
cambia la condizione iniziale del sistema, ma non il transitorio o il valore asintotico.

\subsubsection*{Sistemi elementari di secondo ordine}
Dato un sistema elementare del secondo ordine nella forma \(\displaystyle G(s) = \frac{{\omega_n}^2}{(s-p_1)(s-p_2)}\) si definiscono
le seguenti grandezze riferite al sistema
\begin{itemize}
	\item \(p_{1,2} = \sigma \pm j\omega = \omega_n(-cos\varphi \pm j \sin\varphi)\)
	\item \(\omega_n = \sqrt{\sigma^2+\omega^2}\) pulsazione naturale
	\item \(\xi=\cos\varphi=-\sigma/\omega_n\) fattore di smorzamento (\(\xi=0\) per immaginari puri, \(\xi=1\) per reali coincidenti)
\end{itemize}
Si definiscono inoltre le seguenti grandezze per l'analisi della funzione di trasferimento:
\begin{itemize}
	\item \textbf{overshoot} (massima sovraelongazione): \(M_p = e^{-\pi\xi/\sqrt{1-\xi^2}} \times 100\%\)\\
	essendo difficile da calcolare si utilizza una tabella con percentuale di massima elongazione in funzione di \(\xi = \cos\varphi\),
	imporre un overshoot massimo corrisponde ad imporre un \(\xi^*\) massimo, ovvero un \(\varphi^*\) massimo, in questo modo si
	individua una regione del piano complesso di tutti i possibili poli \(s_i\) con \(\varphi_i \leq \varphi^*\) che soddisfano la
	condizione imposta
	\item \textbf{settling time} (tempo di assestamento): \({T_s}_\text{5\%} \approx 3/\xi\omega_n\) \\
	imporre un tempo di assestamento consiste nel definire un vincolo alla parte reale dei poli \(s_i\): \(\sigma_i = -\xi\omega_n \leq -3/{T_s}^*\)
	ovvero si devono trovare a sinistra della retta verticale \(\sigma^* = -3/{T_s}^*\) nel piano complesso
	\item \textbf{rise time} (tempo di salita): \(T_r \approx 1.8/\omega_n\) \\
	imporre un tempo di salita consiste nell'imporre un \({\omega_n}^*\) minimo, ovvero si individuano sul piano tutti i possibili poli
	complessi con modulo maggiore di \({\omega_n}^*\) (esterni al cerchio di raggio \({\omega_n}^*\))
\end{itemize}

\subsubsection*{Sistemi di ordine superiore al secondo}
Per i sistemi di ordine superiore al secondo, ci si può ricondurre ad un sistema di secondo ordine facendo delle opportune semplificazioni:
\begin{itemize}
	\item si eliminano i poli (\(\Real(\lambda) < 0\)) molto vicini tra loro
	\item si eliminano i poli con parte reale molto negativa, ovvero quelli che danno origine a comportamenti molto veloci e che non
	influiscono sulla risposta dominante del sistema
	\item in tutto questo va tenuto costante il guadagno statico del sistema (\(\lim_{s \to 0} G(s)\))
\end{itemize}

\subsection{Fedeltà alla risposta - analisi delle specifiche a regime}
L'analisi della fedeltà alla risposta consiste nel studiare il sistema affinché raggiunga e il setpoint (problema di regolazione),
segua un riferimento variabile (tracking) e resista all’effetto di disturbi e incertezze nel modello (robustezza). Una volta raggiunta
la fedeltà desiderata si può lavorare per renderlo prestante (risposta veloce e precisa).

Si osserva che l'errore a regime di un sistema ad anello chiuso dipende dal tipo di sistema e dal tipo di ingresso:
\begin{itemize}
	\item \textbf{sistemi di tipo 0}: sistemi con nessun polo nell'origine, hanno errore costante \(e_0 = 1/(1+L(0))\) per ingresso
	di ordine 0 (gradino) e errore divergente per un ingresso di ordine superiore
	\item \textbf{sistemi di tipo 1}: sistemi con 1 polo nell'origine, hanno errore nullo per ingresso di ordine nullo, errore
	costante \(e_0 = 1/L(0)\) per ingresso di ordine 1 (rampa) e errore divergente per ingressi di ordine superiore
	\item \textbf{sistemi di tipo 2}: sistemi con 2 poli nell'origine, hanno errore nullo per ingressi di ordine \(<2\), errore
	costante \(e_0 = 1/L(0)\) per ingresso di ordine 2 (rampa parabolica) e errore divergente per un ingresso di ordine superiore
\end{itemize}

\newpage

\subsection{Controllori standard P, PI}
Definito un sistema ad anello chiuso (feedback), l'uscita \(u(t)\) del controllore \(C(s)\) corrispondente all'entrata del
sistema \(G(s)\),per un controllore PID vale:
\[u(t) = u_\text{proporzionale}(t) + u_\text{integrale}(t) + u_\text{derivativa}(t) = k_pe(t) + k_i \int_{0}^{t}e^{\tau}d\tau + k_d\frac{de(t)}{dt}\]

\subsubsection*{Controllore proporzionale}
Un controllore proporzionale consiste in un blocco controllore la cui uscita \(u(t)\) dipende solo proporzionalmente dall'errore
\(e(t)\) (dal presente), per cui si hanno le seguenti relazioni:
\[C(s) = k_p \qquad L(s)=k_pG(s) \qquad W_r(s) = \frac{k_p n_G(s)}{d_G(s)+k_pn_G(s)} \qquad W_e(s) = \frac{1}{1+k_pG(s)}\]
Si osserva che:
\begin{itemize}
	\item affinché il sistema completo sia stabile, il polinomio \(d_g(s)+k_pn_G(s)\) deve essere Hurwitziano
	\item l'errore a regime diminuisce all'aumentare di \(k_p\): \(\displaystyle\lim_{s \to 0} sE(s) = 1/(1+k_pG(0))\), ma per un
	ingresso costante non si annulla mai
\end{itemize}

\subsubsection*{Controllore proporzionale integrativo}
Un controllore proporzionale consiste in un blocco controllore la cui uscita \(u(t)\) dipende proporzionalmente dall'errore
\(e(t)\) (dal presente) e dall'integrale dell'errore nel tempo (dal passato), per cui si hanno le seguenti relazioni:
\[C(s) = k_p + \frac{k_i}{s} = \frac{sk_p + k_i}{s} \qquad L(s)=\frac{sk_p + k_i}{s}G(s) \qquad W_r(s) = \frac{(sk_p + k_i) n_G(s)}{sd_G(s)+(sk_p + k_i)n_G(s)}\]
\[W_e(s) = \frac{s}{s+(sk_p+k_i)G(s)}\]
Si osserva che:
\begin{itemize}
	\item affinché il sistema completo sia stabile, il polinomio \(sd_G(s)+(sk_p + k_i)n_G(s)\) deve essere Hurwitziano
	\item l'errore a regime si annulla per un ingresso costante: \(\displaystyle\lim_{s \to 0} sE(s) = 0 \cdot \frac{1}{(0+(0k_p+k_i)G(0)} = 0\)
\end{itemize}

\subsubsection*{Ruolo degli zeri nel controllo}
Dato un sistema con di secondo grado con risposta impulsiva:
\[H(s) = \frac{{\omega_n}^2(1+\hat{T}s)}{s^2 + 2\xi\omega_ns + {\omega_n}^2} \qquad \hat{T} = \text{tempo caratteristico dello zero}\]
\begin{itemize}
	\item \textbf{zero a parte reale positiva}: uno zero a parte reale positiva \(\hat{T} < 0\) provoca una sottoelongazione
	proporzionale allo zero, ovvero il sistema inizialmente evolve in direzione opposta al regime finale
	\item \textbf{zero a parte reale negativa a destra dei poli}: uno zero a parte reale negativa \(T_\text{polo} > \hat{T} > 0\),
	ma più vicino allo zero dei poli provoca una sovraelongazione extra proporzionale allo zero e, in generale, provoca una
	risposta più nervosa
	\item \textbf{quasi cancellazione zero-polo}: uno zero prossimo ad un polo con parte reale positiva \(T_\text{polo} \approx \hat{T} > 0\)
	provoca un andamento simile ad un sistema di primo ordine, se il polo è a destra \(T_\text{polo} < \hat{T}\) si arriva a
	regime più lentamente, viceversa se il polo si trova a sinistra \(T_\text{polo} > \hat{T}\) si arriva a regime più velocemente
\end{itemize}

\subsubsection*{Riassunto evoluzioni per controllori P vs PI}
\begin{center}
	\begin{tabularx}{\textwidth}{C | C | C | C | C | C}
		\textbf{parameter} & \textbf{rise time} & \textbf{overshoot} & \textbf{settling time} & \textbf{steady-state error} & \textbf{stability} \\
		\toprule
		\(k_p \;\; \uparrow\) & \(\searrow\) & \(\nearrow\) & \(\leadsto\) & \(\searrow\) & \(\searrow\) \\
		\midrule
		\(k_i \;\; \uparrow\) & \(\searrow\) & \(\nearrow\) & \(\nearrow\) & 0 & \(\searrow\)
	\end{tabularx}
\end{center}

\section{Luogo delle radici - LDR}
Il luogo delle radici è la rappresentazione sul piano complesso della posizione delle radici di un polinomio \(p_k(s)\) al variare
del parametro \(k\). Permette di scegliere facilmente il parametro \(k\) per posizionare i poli secondo le specifiche da rispettare.

\subsubsection*{Osservazioni}
\begin{itemize}
	\item[0.] si assume \(k \geq 0\) con \(p_k(s)\) di grado \(n\)
	\item[1.] la funzione che associa il polinomio \(p(k) = D(s) +kN(s)\) alle sue radici è continua
	\item[2.] ogni radice forma, al variare di \(k\) una curva continua nel piano complesso
	\item[3.] per cui il luogo delle radici ha \(n\) rami (uno per ogni radice)
	\item[4.] i rami si intersecano eventualmente per i valori di \(k\) in cui \(p_k(s)\) ha radici di molteplicità multipla
	\item[5.] il ldr si può tracciare per semplici regole per via grafica
	\item[6.] per \(k=0\) il ldr coincide con i poli di \(G(s)\)
	\item[7.] per \(k \to +\infty\) si ha che \(m\) rami tendono agli zeri di \(G(s)\), mentre i rimanenti \(m-n\) tendono a \(+\infty\)
	\item[8.] siccome le radici sono coppie di coniugati, il luogo è simmetrico per l'asse reale
\end{itemize}

\subsubsection*{Disegno del luogo delle radici}
\begin{itemize}
	\item[0.] si scrive \(G(s)\) in forma di Evans (fattorizzo numeratore e denominatore)
	\item[1.] il luogo delle radici è composto da \(n\) rami continui in \(\mathcal{C}\)
	\item[2.] i rami si originano dai poli di \(G(s)\), ogni polo avrà tanti rami quanto la sua molteplicità
	\item[3.] i rami terminano negli zeri di \(G(s)\), ogni zero avrà tanti rami quanto la sua molteplicità
	\item[4.] il luogo è simmetrico per l'asse reale
	\item[5.] appartengono al ldr, solo i punti dell'asse reale che lasciano a destra un numero dispari di zeri e poli, gli altri
	punti dell'asse reale che fanno parte del ldr sono al più altri zeri o poli
	\item[6.] se \(n-m\) è dispari, allora uno dei rami che tende a \(\infty\), vi tende lungo il semiasse reale negativo
	\item[7.] gli \(n-m\) rami che tendono a \(\infty\), ci vanno lungo \(n-m\) asintoti che si intersecano sullo stesso punto
	\(\sigma_c\) e dividono l'angolo giro in \(n-m\) settori:
	\begin{itemize}
		\item \(\sigma_c = \frac{\sum^n p_i - \sum^m z_i}{n-m} \;\; \in \;\; \mathbb{R}\)
		\item se \(n-m\) è dispari, uno degli asintoti è il semiasse reale negativo
		\item se \(n-m\) è pari, il semiasse reale negativo è la bisettrice di uno dei settori
	\end{itemize}
	\item[8.] è possibile usare il criterio di Routh per determinare per quali valori di \(k\) il luogo attraversa l'asse immaginario
	o un altro generico asse verticale con ascisse \(\sigma^*\) traslando il sistema \(\hat{s} = s - \sigma^*\)
	\item[9.] dato \(\hat{s} \in\) ldr, per trovare il \(k\) associato basta fare: \(D(\hat{s}) + kN(\hat{s}) \;\;\rightarrow\;\; k = -D(\hat{s})/N(\hat{s})\)
	\item[10.] per trovare i punti multipli (o intersezioni dei rami) basta porre la condizione che la derivata si annulla, in
	quanto una radice \(\hat{s}\) di molteplicità superiore a 1 annulla sia \(p_k(s)\) che \(p_k'(s)\) per cui si ottiene
	\(p_k'(s) = D'(\hat{s})-D(\hat{s})/N(\hat{s})\cdot N'(\hat{s}) \;\;\rightarrow\;\; D'(\hat{s})N(\hat{s})=D(\hat{s})N'(\hat{s})\)
	e risolvendo tale equazione si ottengono i punti multipli al finito, oltre che ai poli, zeri multipli e ai punti multipli a \(\infty\)
	\item[11.] le tangenti ai rami che escono per un punto multiplo dividono l'angolo giro in settori uguali, in analogo anche
	le tangenti dei rami entranti dividono l'angolo giro in settori uguali, tali che ogni ramo entrante è la bisettrice di un
	settore delimitato da due rami uscenti
\end{itemize}