\documentclass[a4paper]{article}
\usepackage[utf8]{inputenc} % standard unicode
\usepackage[italian]{babel} % corretta sillabazione in italiano
\usepackage{geometry} % per impostare margini e layout pagina
%\usepackage{layout} % per impostare il layout pagina
\usepackage{amssymb} % per l'ambiente matematico
\usepackage{amsmath} % per l'ambiente matematico
\usepackage{amsthm} % per l'ambiente matematico (simbolo qed alla fine delle dimostrazioni)
\usepackage{multirow} % per celle che si espandono su più righe
\usepackage{tabularx} % per tabelle con larghezza flessibile
\usepackage{booktabs} % per linee orizzontali tabelle
\usepackage{hyperref} % per collegamenti

\geometry{a4paper,left=25mm, right=25mm, bottom=25mm, top=30mm}

\renewcommand\tabularxcolumn[1]{m{#1}}

\newcommand\restr[2]{{% we make the whole thing an ordinary symbol
	\left.\kern-\nulldelimiterspace % automatically resize the bar with \right
	#1 % the function
	\vphantom{\big|} % pretend it's a little taller at normal size
	\right|_{#2} % this is the delimiter
	}}

\newcommand\dom{\text{dom}}
\newcommand\settcosh{\text{settcosh}}
\newcommand\settsinh{\text{settsinh}}
\newcommand\setttanh{\text{setttanh}}

\title{Appunti di analisi 1}
\author{Giacomo Simonetto}
\date{Primo semestre 2023-24}

\begin{document}

% -------------------------------------- Copertina e indice ---------------------------------------
\maketitle
\begin{abstract}
	Appunti del corso di Analisi 1 della facoltà di Ingegneria Informatica dell'Università di Padova.
\end{abstract}

\newpage

\tableofcontents

\newpage

% --------------------------------- Cenni di teoria degli insiemi ---------------------------------
\section{Cenni di teoria degli insiemi}
\subsection{Notazioni e definizioni di base}
\begin{align*}
	\mathbb{N} &= \left\{ \textit{ numeri naturali } \right\} = \left\{ \; 0, \; 1, \; 2, \; 3, \dots \right\} \\
	\mathbb{Z} &= \left\{ \textit{ numeri interi } \right\} = \left\{ \; 0, \; +1, \; -1, \; +2, \; -2, \dots \right\} \\
	\mathbb{Q} &= \left\{ \textit{ numeri razionali } \right\} = \left\{ \; \frac{p}{q} \textit{ t.c. } p,q \in \mathbb{N}, \; q \neq 0 \right\} \\
	\mathbb{R} &= \left\{ \textit{ numeri reali } \right\} \\
	\mathbb{C} &= \left\{ \textit{ numeri complessi } \right\}
\end{align*}

\begin{center}
	\begin{tabular}{c l}
		\(a \in A\) & a (elemento) appartiene ad A (insieme) \\
		\midrule
		\(A \subseteq B\) & A è un sottoinsieme di B \\
		\midrule
		\multirow{2}{*}{\(A = B\)} & gli insiemi A e B hanno gli stessi elementi \\
		& \(A = B \Leftrightarrow A \subseteq B \land B \subseteq A\)
	\end{tabular}
\end{center}


\subsection{Operazioni}
\begin{center}
	\begin{tabularx}{\textwidth}{l c X}
		\textbf{unione} & \(A \cup B\) & insieme costituito dagli elementi di A e dagli elementi di B \\
		\midrule
		\textbf{intersezione} & \(A \cap B\) & insieme costituito dagli elementi che appartengono sia ad A che a B \\
		\midrule
		\textbf{differenza} & \(A \; \backslash \; B\) & insieme costituito dagli elementi di A che non apaprtengono anche a B \\
		\midrule
		\multirow{2}{*}{\textbf{prodotto catesiano}} & 	\multirow{2}{*}{\(A \times B\)} & insieme delle coppie ordinate di elementi formate da un elmeento di A e da un elmeneto di B \\
		& & \(A \times B = \{ (a, b) \textit{ t.c. } a \in A \land b \in B \}\)
	\end{tabularx}
\end{center}


\subsection{Quantificatori}
\begin{center}
	\begin{tabular}{l c l}
		\textbf{quantificatore universale} & \(\forall\) & per ogni \\
		\midrule
		\textbf{quantificatore esistenziale} & \(\exists\) & esiste \\
		\midrule
		\textbf{quantificatore esistenziale unico} & \(\exists!\) & esiste ed è unico
	\end{tabular}
\end{center}


\subsection{Negazione di una proposizione}
Data una proposizione \(R\), la sua negazione \(\urcorner R\) è una proposizione \(Q\) che è vera se e solo se \(R\) è falsa.

\begin{center}
	\begin{tabular}{c c}
		proposizione \(R\) & proposizione \(Q = \urcorner R\) \\
		\toprule
		\(\forall x \in A : P(x)\) & \(\exists x \in A : \urcorner P(x)\) \\
		\midrule
		\(\exists x \in A : P(x)\) & \(\forall x \in A : \urcorner P(x)\) \\
	\end{tabular}
\end{center}

\newpage

% ------------------------------------------ Sommatorie -------------------------------------------
\section{Sommatorie}
\subsection{Definizione}
\[\sum_{i = n_0} ^ {n} x_i \;\; := \;\; x_{n_0} + x_{n_0+1} + x_{n_0+2} + \dots + x_n\] per \(n_0, n \in \mathbb{Z}\) con \(n_0 \leq n\)


\subsection{Proprietà}
\begin{align*}
	\text{proprietà 1 (distributiva)} \quad & c \cdot \sum_{i = n_0} ^ {n} x_i = \sum_{i = n_0} ^ {n} c \cdot x_i \;\; \; \forall c \in \mathbb{R} \\
	\text{proprietà 2} \quad & \sum_{i = n_0} ^ {n} x_i + y_i = \sum_{i = n_0} ^ {n} x_i + \sum_{i = n_0} ^ {n} y_i \\
	\text{proprietà 3.1} \quad & \sum_{i = n_0} ^ {n + m} x_i = \sum_{i = n_0} ^ {n} x_i + \sum_{i = n + 1} ^ {n + m} x_i \\
	\text{proprietà 3.2 per \(m = 0\)} \quad & \sum_{i = n_0} ^ {n} x_i = \sum_{i = n_0} ^ {n-1} x_i + x_n \\
	\text{proprietà 4 (sostituzione del pedice)} \quad & \sum_{i = n_0} ^ {n} x_i = \sum_{j = n_0 + m} ^ {n + m} x_j \\
	\text{proprietà 5 (variable muta)} \quad & \sum_{i = n_0} ^ {n} x_i = \sum_{j = n_0} ^ {n} x_j = \sum_{k = n_0} ^ {n} x_k
\end{align*}


\subsection{Principio di induzione}
\begin{itemize}
	\item[P:] Siano:
	\begin{itemize}
		\item \(n, n_0 \in \mathbb{N} \; t.c. \; n \geq n_0\)
		\item \(P(n)\) una proposizione ben definita
	\end{itemize}
	\item[H:] supponiamo che:
	\begin{itemize}
		\item \(P(n_0)\) sia vera
		\item \(\forall n \geq n_0\), se \(P(n)\) è vera, allora lo è anche \(P(n-1)\)
	\end{itemize}
	\item[T:] allora \(P(n)\) è vera \(\forall n \geq n_0\)
	\item[Dim:] Dimostrazione per assurdo \\
	Supponiamo che la tesi sia falsa: \(\exists \overline{n} \geq n_0 \;\; t.c. \;\; P(n)\) è falsa
	\begin{itemize}
		\item Se \(\overline{n} = n_0\), per cui \(P(n_0)\) è falsa
		\item Se \(\overline{n} > n_0\), per ipotesi anche \(P(\overline{n} - 1), P(\overline{n} - 2), P(\overline{n} - 3), \dots P(n_0)\) sono false
	\end{itemize}
	Questo va contro l'ipotesi iniziale che \(P(n_0)\) sia vera. \\
	Per cui \(\nexists \overline{n} \geq n_0 \;\; t.c. \;\; P(n)\) sia falsa e \(P(n)\) è vera \(\forall n \geq n_0\) \qed
\end{itemize}

\newpage

% ------------------------------------------ Fattoriali -------------------------------------------
\section{Fattoriali}
\subsection{Definizione}
\begin{equation*}
	n! :=
	\begin{cases}
		1 & n = 0 \\
		n \cdot (n-1) \cdot (n-2) \cdot \; \dots \; \cdot 1 & n > 0 
	\end{cases}
\end{equation*}
per \(n \in \mathbb{N}\) e \(n \geq 0\) \\
Numero di riordinamenti di una famiglia di n elementi

\subsection{Proprietà}
\[(n + 1)! = n! \cdot (n + 1)\]

% ------------------------------------ Coefficienti binomiali -------------------------------------
\section{Coefficienti binomiali}
\[\binom{n}{k} := \frac{n!}{k! \cdot (n - k)!}\]
per \(n, k \in \mathbb{R}\) \\
Significato geometrico \dots

\subsection{Proprietà}
\begin{align*}
	\text{proprietà 1.1} \quad & \binom{n}{n} = 1 \\
	\text{proprietà 1.2} \quad & \binom{n}{0} = 1 \\
	\text{proprietà 2} \quad & \binom{n}{k} = \binom{n}{n-k} \\
	\text{proprietà 3} \quad & \binom{n}{k} + \binom{n}{k+1} = \binom{n+1}{k+1} \\
	\text{proprietà 4 (binomio di Newton)} \quad & (a + b) ^ n = \sum_{k = 0} ^ {n} \binom{n}{k} \cdot a ^ k \cdot b ^ {n - k}
\end{align*}

\subsection{Dimosrtrazione binomio di Newton}
\dots

\newpage

% --------------------------------------- Numeri razionali ----------------------------------------
\section{Numeri razionali}
\subsection{Definizione}
\[\mathbb{Q} = \left\{ \textit{ numeri razionali } \right\} = \left\{ \; \frac{p}{q} \textit{ t.c. } p,q \in \mathbb{N}, \; q \neq 0 \right\}\]

\subsection{Proprietà di campo}
Quando un insieme soddisfa le seguenti proprietà, è detto campo. L'insieme \(\mathbb{Q}\) è un campo.

\subsubsection*{Proprietà della somma}
\begin{itemize}
	\item[S1] proprietà commutativa \(a + b =  b + a\)
	\item[S2] proprietà associativa \((a + b) + c = a + (b + c)\)
	\item[S3] \(\exists!\) elemento neutro indicato con \(0\) t.c. \(a + 0 = a\)
	\item[S4] \(\exists\) elemento opposto indicato con \(-a\) t.c. \(a + (-a) = 0\)
\end{itemize}

\subsubsection*{Proprietà del prodotto}
\begin{itemize}
	\item[P1] proprietà commutativa \(a \cdot b =  b \cdot a\)
	\item[P2] proprietà associativa \((a \cdot b) \cdot c = a \cdot (b \cdot c)\)
	\item[P3] proprietà distributiva \((a + b) \cdot c  = a \cdot c + b \cdot c\)
	\item[P4] \(\exists!\) elemento neutro indicato con \(1\) t.c. \(a \cdot 1 = a\)
	\item[P5] \(\exists!\) elemento reciproco indicato con \(\displaystyle \frac{1}{a}\) o \(a ^ {-1}\) t.c. \(a \cdot \displaystyle \frac{1}{a} = 1\)
\end{itemize}

\subsection{Relazione d'ordine}
Dato un insieme \(A\), una relazione d'ordine \((a, b)\) (es. \(a \leq b\)) su \(A\) è un sottoinsieme \(R = A \times A\) tale che:
\begin{itemize}
	\item[O1] \((a, a') \in R, \forall a \in A\)
	\item[O2] Se \((a, a') \in R\) e \((a', a'') \in R\) allora \((a, a'') \in R\)
	\item[O3] Se \((a, a') \in R\) e \((a', a) \in R\) allora \(a = a'\)
	\item[O4] Se \(\forall a,a' \in A\) vale \((a, a') \in R\) o \((a', a) \in R\), ovvero quando presi due elementi è sempre possibile stabilire una relazione d'ordine valida 
\end{itemize}
Quando un campo soddisfa le prime tre condizioni, allora viene detto ordinato.
Quando un campo soddisfa anche la quarta condizione, allora viene detto totalmente ordinato. \\
L'insieme \(\mathbb{Q}\) è un campo totalmente ordinato, per cui è possibile rappresentarne gli elementi in una retta.


\subsection{Discrezione dei numeri razionali}
L'insieme \(\mathbb{Q}\) è discreto: data una retta, non tutti i punti di tale retta appartengono a \(\mathbb{Q}\).
Ad esempio il punto \(\sqrt{2} \notin \mathbb{Q}\): \(\nexists x \in \mathbb{Q}\) t.c. \(x^2 = 2\).
\begin{itemize}
	\item[H:] \(x \in \mathbb{Q}\) e \(\mathbb{Q}\) è l'insieme dei numeri razionali
	\item[T:] \(\nexists x\) t.c. \(x^2 = 2\)
	\item[Dim:] supponiamo per assurdo che \(\exists x \in \mathbb{Q}\) t.c. \(x^2 = 2\) per cui \(x = \displaystyle \frac{p}{q}\), \(p,q \in \mathbb{Z}, q \neq 0\) e \(p,q\) primi tra loro \\
	per cui \(x^2 = 2 \Rightarrow \displaystyle \frac{p^2}{q^2} = 2 \Rightarrow p^2 = 2 \cdot q^2 \Rightarrow p\) è pari \(\Rightarrow \exists \overline{p}\) t.c. \(p = 2 \cdot \overline{p}\) \\
	per cui \(2 = \displaystyle \frac{p^2}{q^2} = \frac{(2 \cdot \overline{p})^2}{q^2} = 4 \cdot \frac{p^2}{q^2} \Rightarrow q^2 = 2 \cdot \overline{p}^2 \Rightarrow q\) è pari \\
	ma \(p\) e \(q\) sono stati assunti primi tra loro, per cui l'ipotesi che \(\exists x \in \mathbb{Q}\) t.c. \(x^2 = 2\) è errata \qed
\end{itemize}

\newpage

% ----------------------------------------- Numeri reali ------------------------------------------
\section{Numeri reali}
\subsection{Definizione}
L'insieme \(\mathbb{R}\) è composto da elementi (detti numeri reali) definiti come allineamenti decimali che possono essere: \\
- limitati es. \(5,347\) \\
- illimitati periodici es. \(6,\overline{2}\) \\
- illimitati non periodici es. \(\pi\) o \(\sqrt{2}\) \\
Questa estensione dell'insieme \(\mathbb{Q}\) serve per poter risolvere \(x^2 = 2\).


\subsection{Proprietà}
Su \(\mathbb{R}\) si possono estendere le proprietà di somma, prodotto e ordinamento di \(\mathbb{Q}\), per cui anche \(\mathbb{R}\) è un insieme totalmente ordinato.


\subsection{Teorema di completezza pt.1}
\begin{itemize}
	\item[H:] Dati \(A, B \subseteq \mathbb{R}\) tali che \(\forall a \in A\) e \(\forall b \in B\), \(a \leq b\)
	\item[T:] \(\exists c \in \mathbb{R}\) t.c. \(a \leq c \leq b, \forall a \in A\) e \(\forall b \in B\) dove \(c\) è detto elemento separatore di \(A\) e \(B\)
\end{itemize}
In altre parole, presi \(A, B \subseteq \mathbb{R}\) tali che \(\forall a \in A\) e \(\forall b \in B\), \(a \leq b\) è sempre possibile trovare l'elemento separatore tra i due insiemi. \\
Questo teorema vale solo in \(\mathbb{R}\) e non in \(\mathbb{Q}\): \\
dati \(A=\left\{ x \geq 0 \; t.c. \; x^2 \leq 2 \right\}\) e \(B=\left\{ x \geq 0 \; t.c. \; x^2 \geq 2 \right\}\), \(\exists c = \sqrt{2} \in \mathbb{R}\)


\subsection{Intervalli}
Dato che \(\mathbb{R}\) è un sistema completo, si può parlare di intervalli.
\begin{itemize}
	\item intervalli limitati
	\begin{align*}
		\left(a, b \right) &= \left]a, b \right[ = \left\{x \in R, a < x < b \right\} \\
		\left(a, b \right] &= \left]a, b \right] = \left\{x \in R, a < x \leq b \right\} \\
		\left[a, b \right) &= \left[a, b \right[ = \left\{x \in R, a \leq x < b \right\} \\
		\left[a, b \right] &= \left[a, b \right] = \left\{x \in R, a \leq x \leq b \right\}
	\end{align*}
	\item intervalli illimitati
	\begin{align*}
		\left(a, +\infty \right) &= \left] a, +\infty \right[ = \left\{x \in R, x > a \right\} \\
		\left[a, +\infty \right) &= \left[ a, +\infty \right[ = \left\{x \in R, x \geq a \right\} \\
		\left(-\infty, b \right) &= \left] -\infty, b \right[ = \left\{x \in R, x < b \right\} \\
		\left(-\infty, b \right] &= \left] -\infty, b \right] = \left\{x \in R, x \leq b \right\}
	\end{align*}
\end{itemize}
Da notare che \(+\infty, -\infty \notin \mathbb{R}\), per cui è stato definito \(\mathbb{R}^*\) o \(\overline{\mathbb{R}} = \mathbb{R} \cup \left\{ -\infty, +\infty \right\}\)

\newpage


% -------------------------------------------- Modulo ---------------------------------------------
\section{Modulo}
\subsection{Definizione}
\begin{equation*}
	\left| a \right| :=
	\begin{cases}
		a & a \geq 0 \\
		-a & a < 0
	\end{cases}
\end{equation*}
per \(a \in \mathbb{R}\)

\subsection{Proprietà}
\begin{enumerate}
	\item \(\left| a \right| \leq M \Leftrightarrow -M \leq a \leq M\) per \(m \geq 0\)
	\item \(\left| a \right| \geq M \Leftrightarrow a \leq -M\) o \(a \geq M\) per \(m \geq 0\)
	\item \(- \left| a \right| \leq a \leq \left| a \right|\)
\end{enumerate}

\subsection{Disuguaglianza triangolare pt.1}
\begin{enumerate}
	\item \(\left| a + b \right| \leq \left| a \right| + \left| b \right|\)
	\item \(\left| \left| a \right| - \left| b \right| \right| \leq \left| a - b \right|\), cioè
	\(- \left| a - b \right| \leq \left| a \right| - \left| b \right| \leq \left| a - b \right|\)
\end{enumerate}
blablabla

\newpage

% ------------------------ Insiemi limitati superiormente e inferiormente -------------------------
\section{Insiemi limitati e illimitati}
\subsection{Insiemi limitati e illimitati}
Sia \(A \subseteq \mathbb{R}, A \neq \emptyset\): \\
\(A\) è \textbf{limitato superiormente} se \(\exists M \in \mathbb{R}\) t.c. \(a \leq M, \forall a \in A\). \\
\(A\) è \textbf{illimitato superiormente} se \(\nexists M \in \mathbb{R}\) t.c. \(a \leq M, \forall a \in A\). \\
\(A\) è \textbf{limitato inferiormente} se \(\exists N \in \mathbb{R}\) t.c. \(a \geq n, \forall a \in A\). \\
\(A\) è \textbf{illimitato inferiormente} se \(\nexists n \in \mathbb{R}\) t.c. \(a \geq n, \forall a \in A\). \\
\(A\) è un insieme limitato se è limitato superiormente e inferiormente,\\cioè se \(\exists N \in \mathbb{R}\) t.c. \(a \leq \left| N \right|, \forall a \in A\).


\subsection{Maggioranti e minoranti}
Un tale numero \(M\) che limita \(A\) superiormente è detto \textbf{maggiorante} di \(A\). \\
Se \(A\) non è limitato superiormente, non ha maggioranti. \\
Un tale numero \(N\) che limita \(A\) inferiormente è detto \textbf{minorante} di \(A\). \\
Se \(A\) non è limitato inferiormente, non ha minoranti.


\subsection{Massimi e minimi}
Sia \(m\) un maggiorante di \(A\), se \(m \in A\) allora \(m\) è detto \textbf{massimo} di \(A\). \\
Sia \(n\) un minorante di \(A\), se \(n \in A\) allora \(n\) è detto \textbf{minimo} di \(A\).
	

\subsection{Estremi superiori e inferiori}
Sia \(A \subset \mathbb{R}, A \neq \emptyset\) e \(A\) è superiormente limitato, un numero \(S\) detto \(\sup A\) è detto \textbf{estremo superiore} di \(A\) quando è il minimo dei maggioranti di \(A\).
\begin{align*}
	\tag{definizione}
	S = \sup A &\text{ se } S = \min \left\{ \text{ maggioranti di } A \right\} \\
	\tag{caratterizzazione}
	S = \sup A &\text{ se }
	\begin{cases}
		S \text{ è maggiorante di } A \\
		\forall \varepsilon > 0, \exists a \in A \text{ t.c. } a < S - \varepsilon \quad \left( \nexists \text{ maggiorante più piccolo} \right)
	\end{cases}
\end{align*}

Sia \(A \subset \mathbb{R}, A \neq \emptyset\) e \(A\) è inferiormente limitato, un numero \(I\) detto \(\inf A\) è detto \textbf{estremo inferiore} di \(A\) quando è il massimo dei minoranti di \(A\).
\begin{align*}
	\tag{definizione}
	I = \inf A &\text{ se } I = \max \left\{ \text{ minoranti di } A \right\} \\
	\tag{caratterizzazione}
	I = \inf A &\text{ se }
	\begin{cases}
		S \text{ è minorante di } A \\
		\forall \varepsilon > 0, \exists a \in A \text{ t.c. } a < I + \varepsilon \quad \left( \nexists \text{ minorante più grande} \right)
	\end{cases}
\end{align*}


\subsection{Teorema di unicità dell'esistenza di max, min, sup, inf}
Se \(\max A\) esiste, allora è unico. (dim. per assurdo) \\
Se \(\min A\) esiste, allora è unico. (dim. per assurdo) \\
Se \(\sup A\) esiste, allora è unico. (dim. unicità del minimo dei maggioranti) \\
Se \(\inf A\) esiste, allora è unico. (dim. unicità del massimo dei minoranti)


\subsection{Corrispondenza tra sup e max, inf e min}
Se \(\exists \sup A\) e \(\sup A \in A\), allora \(\exists \max A\) e \(\sup A = \max A\). \\
Se \(\exists \inf A\) e \(\inf A \in A\), allora \(\exists \min A\) e \(\inf A = \min A\).


\subsection{Teorema di completezza pt.2}
Se \(A\) è superiormente limitato, allora \(A\) ammette un estremo superiore in \(\mathbb{R}\).
Se \(A\) è inferiormente limitato, allora \(A\) ammette un estremo inferiore in \(\mathbb{R}\).


\newpage

% ---------------------------------- Potenze, radici, logaritmi -----------------------------------
\section{Potenze, radici, logaritmi}
\subsection{Potenze intere}
Sia \(\alpha \in \mathbb{R}, p \in \mathbb{Z}\):
\[ \alpha ^ p := 
\begin{cases}
	\alpha \cdot \alpha \cdot \; \dots \; \cdot \alpha \; \left( \text{per p volte} \right) & p > 0 \\
	1 & p = 0, \alpha \neq 0 \\
	\frac{1}{ \alpha \cdot \alpha \cdot \; \dots \; \cdot \alpha \; \left( \text{per -p volte} \right) } & p < 0, \alpha \neq 0 \\
\end{cases}
\]

\subsection{Esistenza e unicità delle radici intere}
Sia \(y \in \mathbb{R}, y \geq 0, n \in \mathbb{N} \; \backslash \left\{ 0 \right\} \), allora \(\exists! r \in \mathbb{R}\) t.c. \(r ^ n = y\).\\
\(r = \sqrt[n]{y} \) è chiamata radice ennesima di y

\subsection{Potenze razionali (o radici)}
Sia \(a \in \mathbb{R}, p, q \in \mathbb{Z}, q > 0\):
\[ a ^ \frac{p}{q} := 
\begin{cases}
	\sqrt[q]{a ^ p} = \left( a ^ p \right) ^ \frac{1}{q} & a \neq 0 \text{ o } p \neq 0 \\
	1 & a \neq 0, p = 0 \\
	0 & a = 0, p > 0
\end{cases}
\]

\subsection{Potenze reali (o esponenziali)}
Sia \(a,r \in \mathbb{R}, a \geq 0\):
\[ a ^ r := 
\begin{cases}
	\sup  \left\{a ^ s \text{ t.c. } s \leq r, s \in \mathbb{Q} \right\} & a \neq 0, r > 0 \\
	\frac{1}{a ^ {-r}} & a \neq 0, r < 0 \\
	1 & a \neq 0, r = 0 \\
	0 & a = 0, r \neq 0 \\
\end{cases}
\]

\subsection{Logaritmi}
Siano \(a,b \in \mathbb{R}, a > 0, a \neq 1, b > 0\), allora \(\exists!  x \in \mathbb{R}\) t.c. \(a ^ x = b\) con
\[ x = \log_a b = 
\begin{cases}
	\sup  \left\{ r \in \mathbb{R} \text{ t.c. } a ^ r \leq b \right\} & a > 1 \\
	\sup  \left\{ r \in \mathbb{R} \text{ t.c. } a ^ r \geq b \right\} & 0 < a < 1
\end{cases}
\]

\subsection{Proprietà dei logaritmi}
\begin{align*}
	\text{proprietà 1} \quad & \log_a a = 1 \\
	\text{proprietà 2} \quad & \log_a 1 = 0 \\
	\text{proprietà 3} \quad & \log_a a^c = c \\
	\text{proprietà 4} \quad & \log_a x \cdot y = \log_a x + \log_a y \\
	\text{proprietà 5.1 (potenza)} \quad & \log_a x^\alpha = \alpha \cdot \log_a x\\
	\text{proprietà 5.2 (caso \(\alpha = -1\))} \quad & \log_a \frac{1}{x} = - \log_a x \\
	\text{proprietà 6} \quad & \log_a \frac{x}{y} = \log_a x - \log_a y\\
	\text{proprietà 7.1 (cambio di base)} \quad & \log_a b = \log_a c \cdot \log_c b \\
	\text{proprietà 7.2 (caso \(c = \frac{1}{a}\))} \quad & \log_a b = - \log_\frac{1}{a} b
\end{align*}

\newpage

% --------------------------------------- Numeri complessi ----------------------------------------
\section{Numeri complessi}
\subsection{Definizione e forma algebrica}
Per risolvere equazioni del tipo \(x^2 + 1  = 0\) è necessario introdurre un nuovo insieme definito con \(\mathbb{C}\) definito come
\[\mathbb{C} = \left\{ x + i y \text{ t.c. } x,y \in \mathbb{R} \text{ con \(i =\) unità immaginaria} \right\}\]
Sia \(z \in \mathbb{C}\):
\begin{align*}
	z = x + i y \quad & \text{forma algebrica di \(z \in \mathbb{C}\)} \\
	x = \Re(z) \quad & \text{parte reale di z} \\
	y = \Im(z) \quad & \text{parte immaginaria di z}
\end{align*}
Si osserva che se \(\Im(z) = 0 \Leftrightarrow y = 0 \Leftrightarrow z = x \Leftrightarrow z \in \mathbb{R}\) \\
per cui \(\mathbb{R} \subset \mathbb{C}\) e \(\mathbb{R} = \left\{ z \in \mathbb{C} \text{ t.c. } \Im(z) = 0 \right\}\)


\subsection{Proprietà}
\begin{align*}
	z_1 + z_2 &= (x_1 + x_2) + i (y_1 + y_2) \\
	z_1 \cdot z_2 &= (x_1 \cdot x_2 - y_1 \cdot y_2) + i (x_1 \cdot y_2 + x_2 \cdot y_1)\\
	0 + i0 &= 0 \text{ è l'elemento neutro della somma} \\
	1 + i0 &= 1 \text{ è l'elemento unitario (neutro del prodotto)} \\
	-z &= (-x)) + i(-y) \text{ opposto di } z = x + iy \\
	z^{-1} = \frac{1}{z} &= \frac{x}{x^2 + y^2} - i \cdot \frac{y}{x^2 + y^2} \text{ inverso di } z
\end{align*}
Definite queste proprietà, l'insieme \(\mathbb{C}\) è un campo. Dato che non è possibile stabilire una relazione d'ordine \(\mathbb{C}\) non è un campo ordinato e tantomento totalmente ordinato.


\subsection{Coniugato e proprietà}
Sia \(z \in \mathbb{C}, z = x + i y\) con \(x,y \in \mathbb{R}\), il suo coniugato è \(\overline{z} = x - iy\).
\begin{align*}
	\text{parte reale} \quad & \Re(\overline{z}) = \Re(z) \\
	\text{parte immaginaria} \quad & \Im(\overline{z}) = -\Im(z) \\
	\text{coniugato in } \mathbb{R} \quad & z = \overline{z} \Leftrightarrow z \in \mathbb{R} \\
	\text{somma} \quad & z_1 + \overline{z_1} = 2 x_1 \\
	\text{differenza} \quad & z_1 - \overline{z_1} = 2 i y_1 \\
	\\
	\text{somma} \quad & \overline{z_1 + z_2} = \overline{z_1} + \overline{z_2} \\
	\text{prodotto} \quad & \overline{z_1 \cdot z_2} = \overline{z_1} \cdot \overline{z_2} \\
	\text{quoziente} \quad & \overline{\left( \frac{z_1}{z_2} \right)} = \frac{\overline{z_1}}{\overline{z_2}} \\
	\text{doppio coniugato} \quad & \overline{\left( \overline{z} \right)} = z \\
\end{align*}


\subsection{Piano di Gauss}
I numeri complessi possono essere rappresentati in un piano cartesiano, chiamato piano di Gauss, secondo le loro coordinate \(\left( x; y \right) = \left( \Re(z); \Im(z) \right)\). \\
Se \(z \in \mathbb{R}\) il punto corrispondente sul piano giace sull'asse \(x\). \\
Inoltre due numeri complessi coniugati sono simmetrici rispetto all'asse \(x\).


\subsection{Modulo e proprietà}
Il modulo di un numero complesso è la distanza del punto dall'origine sul piano di Gauss.
\[ \left| z \right| = \sqrt{x^2 + y^2}\]
\begin{align*}
	\text{modulo in } \mathbb{R} \quad & \left| z \right| = \sqrt{x^2 + 0^2} = \left| x \right| \\
	\text{rel. d'oridne} \quad & \left| z \right| \geq 0 \quad \\
	\text{coniugato} \quad & \left| z \right| = \left| \overline{z} \right| \\
	\text{prodotto} \quad & \left| z_1 \cdot z_2 \right| = \left| z_1 \right| \cdot \left| z_2 \right| \\
	\text{inverso} \quad & \left| \frac{1}{z} \right| = \frac{1}{\left| z \right|} \\
	\text{quoziente} \quad & \left| \frac{z_1}{z_2} \right| = \frac{\left| z_1 \right|}{\left| z_2 \right|} 
\end{align*}


\subsection{Disuguaglianza triangolare pt.2}
\begin{align*}
	\left| z_1 + z_2 \right| &\leq \left| z_1 \right| + \left| z_2 \right| \\
	\left| \left| z_1 \right| - \left| z_2 \right| \right| &\leq \left| z_1 - z_2 \right|
\end{align*}

\(\left| z_1 + z_2 \right|\) corrisponde al "vettore" ottenuto dalla somma tra del "vettore" \(\left| z_1 \right|\) e il "vettore" \(\left| z_2 \right|\).
Graficamente si forma un triangolo con lati \(z_1\), \(z_2\) e \(z_1 + z_2\), per cui la prima disuguaglianza "garantisce" che il triangolo non sia degenere.

Analogamente per la seconda disuguaglianza, dove al posto della somma, c'è la differenza.

Dimostrazione \dots si quadra e si sviluppa il modulo \dots per la seconda si impiega la prima \dots


\subsection{Forma trigonometrica}
Un numero complesso \(z\) può essere rappresentato secondo sue coordinate polari:
\[z := \rho \cdot \left( \cos \vartheta + i \sin \vartheta \right)\]
con
\begin{align*}
	\rho &= \text{ modulo di } z \text{, ovvero la distanza tra } z \text{ e l'origine} \\
	&= \left| z \right| = \sqrt{x^2 + y^2} \\
	\\
	\vartheta &= \text{ argomento di } z \text{, ovvero l'angolo tra l'asse } x \text{ e il modulo di } z\\
	&= \arg(z) = 
	\begin{cases}
		\cos \vartheta = \frac{x}{\rho} \\
		\sin \vartheta = \frac{y}{\rho}
	\end{cases}
\end{align*}

Si nota che l'argomento di un numero complesso è determinato anche per multipli di \(2 \pi\), per cui è definito argomento principale
di \(z\): \(Arg(z)\) l'unico valore per \(\arg(z)\) nell'intervallo \(\left( -\pi; \pi \right]\). \\
Inoltre \(\arg(z)\) non è definito per \(z = 0\).

Il coniugato di \(z = \rho \left( \cos \vartheta + i \sin \vartheta \right)\) è \(\overline{z} = \rho \left( \cos \vartheta - i \sin \vartheta \right)\)


\subsubsection*{Proprietà dell'argomento}
Siano \(z_1, z_2 \in \mathbb{C} \; \backslash \left\{ 0 \right\}\):
\begin{align*}
	\arg \left( z_1 \cdot z_2 \right) &= \arg(z_1) + \arg(z_2) \\
	\arg \left( \frac{z_1}{z_2} \right) &= \arg(z_1) - \arg(z_2)
\end{align*}


\subsubsection*{Formule di De Moivre e potenze con la forma trigonometrica}
Siano \(z, z_1, z_2 \in \mathbb{C} \; \backslash \left\{ 0 \right\}\):
\begin{align*}
	z_1 \cdot z_2 &= 
	\begin{cases}
		\text{ modulo } = \left| z_1 \right| \cdot \left| z_2 \right| = \rho_1 \cdot \rho_2 \\
		\text{ argomento } = \arg(z_1) + \arg(z_2) = \vartheta_1 + \vartheta_2
	\end{cases} \\
	\frac{z_1}{z_2} &= 
	\begin{cases}
		\text{ modulo } = \frac{\left| z_1 \right|}{\left| z_2 \right|} = \frac{\rho_1}{\rho_2}\\
		\text{ argomento } = \arg(z_1) - \arg(z_2) = \vartheta_1 - \vartheta_2
	\end{cases} \\
	z ^ n &= 
	\begin{cases}
		\text{ modulo } = \left| z \right| ^ n = \rho ^ n \\
		\text{ argomento } = \arg(z) \cdot n = n \vartheta
	\end{cases}
\end{align*}


\subsection{Forma esponenziale}
Un numero complesso \(z\) può essere rappresentato in forma esponenziale:
\[z := \rho \cdot e ^ {i \vartheta}\]
con
\begin{align*}
	\rho &= \text{ modulo di } z \\
	e^{i\vartheta} &= \cos \vartheta + i \sin \vartheta \qquad \text{ (Fomula di Eulero) } \\
	\\
	\cos \vartheta &= \frac{e ^ {i \vartheta} + e ^ {-i \vartheta}}{2} \\
	\sin \vartheta &= \frac{e ^ {i \vartheta} - e ^ {-i \vartheta}}{2i}
\end{align*}
Il coniugato di  \(z = \rho e ^ {i \vartheta}\) è \(\overline{z} = \rho e ^ {-i \vartheta}\) \\
Se \(\vartheta = 0 \to e ^ i = 1\) \\
Se \(\vartheta = \pi \to e ^ {i\pi} = -1\)

\subsubsection*{Proprietà di \(e ^ {i \vartheta}\)}
Siano \(z, z_1, z_2 \in \mathbb{C}\)
\begin{align*}
	e^z &=
	\begin{cases}
		\text{modulo } e^x \\
		\text{argomento } y
	\end{cases} \\
	e ^ z &\neq 0 \quad \forall z \in \mathbb{C} \\
	e ^ {z_1} \cdot e ^ {z_2} &= e ^ {z_1 + z_2} \\
	\frac{e ^ {z_1}}{e ^ {z_2}} &= e ^ {z_1 - z_2}
\end{align*}

\subsubsection*{Prodotti, quozienti e potenze con la forma esponenziale}
Siano \(z, z_1, z_2 \in \mathbb{C} \; \backslash \left\{ 0 \right\}\):
\begin{align*}
	z_1 \cdot z_2 &= \rho_1 \rho_2 \cdot e ^ {i \left( \vartheta_1 + \vartheta_2 \right)} \\
	\frac{z_1}{z_2} &= \frac{\rho_1}{\rho_2} \cdot e ^ {i \left( \vartheta_1 + \vartheta_2 \right)} \\
	z ^ n &= \rho ^ n \cdot e ^ {i n \vartheta}
\end{align*}

\newpage


% --------------------------------- Equazioni e disequazioni in C ---------------------------------
\section{Equazioni e disequazioni in C}
%\subsection{Equazioni in C}
%Alcuni consigli su come risolvere le equazioni in \(\mathbb{C}\) \dots

\subsection{Teorema fondamentale dell'algebra}
\subsubsection*{Teorema fondamentale dell'algebra}
\begin{itemize}
	\item[H:] Sia \(n \in \mathbb{N} \; \backslash \left\{ 0 \right\}\) e \(P(z) = a_n z^n + a_{n-1} z^{n-1} + a_{n-2} z^{n-2} + \; \dots \; + a_0\) \\
	con \(a_n \neq 0\), \(a_j \in \mathbb{C} \; \forall j \in \left[0; n\right]\) 
	\item[T:] Allora esiste almento una radice di \(P\), cioè una soluzione dell'equazione \(P(z) = 0\), con \(z \in \mathbb{C}\).
	La soluzione \(z\) è chiamata zero di \(P\).
\end{itemize}

\subsubsection*{Molteplicità di una soluzione}
\begin{itemize}
	\item[H:] Sia \(P(z)\) come sopra e \(z_0 \in \mathbb{C}\) t.c. \(P(z_0) = 0\)
	\item[T:] \(z_0\) è uno zero con molteplicità \(k \in \mathbb{N}\) se \(\exists Q(x)\) di grado \(n-k\) t.c. \\
	\(P(z) = \left( z - z_0 \right) ^ k \cdot Q(z)\) e \(Q(z_0) \neq 0\)
\end{itemize}

\subsubsection*{Numero di soluzioni di un polinomio di grado \(n\)}
\begin{itemize}
	\item[H:] Sia \(P(z)\) come sopra (di grado \(n\))
	\item[T:] \(P(z) = 0\) ha esattamente \(n\) soluzioni se contate con la propria molteplicità
	\item[Dim:] applicando il teorema fondamentale dell'algebra a \(P(z)\) si ottiene che \\
	\(\exists z_0 \in \mathbb{C}\) t.c. \(P(z_0) = 0\), per cui \(\exists Q(z)\) t.c. \(P(z) = \left( z - z_0 \right) \cdot Q(z)\)\\
	riapplicando il teorema \(n\) volte si ottiene che \\
	\(P(z) = \left( z - z_0 \right) \cdot \left( z - z_1 \right) \cdot \left( z - z_2 \right) \cdot \; \dots \; \cdot \left( z - z_n \right)\) \\
	sono state, così, trovate \(n\) soluzioni \(z_0, z_1, \dots, z_n\) di molteplicità 1 \qed
\end{itemize}

\subsubsection*{Soluzioni complesse coniugate per polinomi reali}
\begin{itemize}
	\item[P:] Sia \(P(x)\) un polinomio di grado \(n\) a coefficienti reali
	\item[H:] se \(z_0 \in \mathbb{C}\) una soluzione di \(P(x)\)
	\item[T:] allora anche \(\overline{z_0}\) è una soluzione di \(P(x)\)
	\item[Dim:] Sia \(P(z_0) = 0\) con \(z_0 \in \mathbb{C}\)
	\begin{align*}
		P(\overline{z_0}) &= a_n \overline{z_0} ^ n + a_{n-1} \overline{z_0} ^ {n-1} + a_{n-2} \overline{z_0} ^ {n-2} + \; \dots \; + a_0 \\
		&= a_n \overline{{z_0} ^ n} + a_{n-1} \overline{{z_0} ^ {n-1}} + a_{n-2} \overline{{z_0} ^ {n-2}} + \; \dots \; + a_0 \\
		&= \overline{a_n} \overline{{z_0} ^ n} + \overline{a_{n-1}} \overline{{z_0} ^ {n-1}} + \overline{a_{n-2}} \overline{{z_0} ^ {n-2}} + \; \dots \; + \overline{a_0} \\
		&= \overline{a_n {z_0} ^ n} + \overline{a_{n-1} {z_0} ^ {n-1}} + \overline{a_{n-2} {z_0} ^ {n-2}} + \; \dots \; + \overline{a_0} \\
		&= \overline{a_n {z_0} ^ n + a_{n-1} {z_0} ^ {n-1} + a_{n-2} {z_0} ^ {n-2} + \; \dots \; + a_0} \\
		&= \overline{P(z_0)} \\
		&= \overline{0} \\
		P(\overline{z_0}) &= 0
	\end{align*}
	\qed
\end{itemize}

\subsubsection*{Numero di soluzioni complesse e reali}
\begin{itemize}
	\item[H\(_1\):] Sia \(P(x)\) un polinomio di grado \(n\) a coefficienti reali
	\item[T\(_1\):] Le radici con parte immaginaria non nulla sono pari e a due a due l'una coniugata dell'altra
	\item[H\(_2\):] Sia \(P(x)\) un polinomio di grado dispari a coefficienti reali
	\item[T\(_2\):] Il polinomio \(P(x)\) ha almeno una soluzione reale
\end{itemize}

\subsection{Teorema di decomposizione di polinomi}
\begin{itemize}
	\item[H:] Sia \(P(x)\) un polinomio di grado \(n\) a coefficienti reali
	\item[T:] \(P\) può essre scomposto in:
	\begin{multline*}
		P(z) = \left( z - x_1 \right) ^ {k_1} \cdot \left( z - x_2 \right) ^ {k_2} \cdot \; \dots \; \left( z - x_l \right) ^ {k_l} \cdot \left( z^2 + A_1 z + B_1 \right) ^ {j_1} \cdot \\
		\cdot \left( z^2 + A_2 z + B_2 \right) ^ {j_2} \cdot \; \dots \; \cdot \left( z^2 + A_m z + B_m \right) ^ {j_m} \cdot  C
	\end{multline*}
	con:
	\begin{itemize}
		\item \(x_1, x_2, \dots x_l \in \mathbb{R}\) radici reali del polinomio
		\item \(k_1, k,2 \dots k_l \in \mathbb{N}\) molteplicità delle radici reali
		\item \(\left( z^2 + A_m z + B_m \right) = \left( z - z_m \right) \cdot \left( z - \overline{z_m} \right)\) radici complesse coniugate
		\item \(j_1, j,2 \dots j_l \in \mathbb{N}\) molteplicità delle radici complesse coniugate
		\item \(A_1, A_2, \dots A_m, B_1, B_2, \dots B_m, C \in \mathbb{R}\)
	\end{itemize}
\end{itemize}

\subsection{Radici n-esime}
Siano \(z, w \in \mathbb{C}\) e \(n \in \mathbb{N}\) tali che \(z ^ n = w\)
\begin{align*}
	z ^ n = w &\Leftrightarrow \left( \rho e ^ {i \vartheta} \right) ^ n = r e ^ {i \varphi} \\
	&\Leftrightarrow \rho ^ n e ^ {i n \vartheta} = r e ^ {i \varphi} \\
	&\Leftrightarrow
	\begin{cases}
		\rho ^ n = r \\
		n \vartheta = \varphi + 2 k \pi \quad \text{ con } k \in \left\{ 0, 1, \dots n-1 \right\}
	\end{cases} \\
	&\Leftrightarrow
	\begin{cases}
		\rho = \sqrt[n]{r} \\
		\vartheta = \frac{\varphi + 2k\pi}{n} \quad \text{ con } k \in \left\{ 0, 1, \dots n-1 \right\}
	\end{cases}
\end{align*}

In questo modo si ottengono \(n\) valori di \(\vartheta\) al variare di \(k\), ovvero \(n\) soluzioni come previsto dal teorema fondamentale dell'algebra.

Le soluzioni rappresentate nel piano di Gauss vengono disposte in una circonferenza di raggio pari al modulo \(\rho = \sqrt[n]{r}\) distribuite
a distanza angolare pari a \(\frac{2 \pi}{n}\) con un angolo di sfasamento rispetto all'asse \(x\) di \(\frac{\varphi}{n}\). 
Congiungendo le soluzioni si ottiene un poligono regolare (es. 6 soluzioni \(\to\) esagono regolare).

\newpage

%\subsection{Equazioni di secondo grado in C}
%\subsection{Disequazioni in C}
%\newpage

% ------------------------------------------- Funzioni --------------------------------------------
\section{Funzioni}
%\subsection{Definizioni}
\begin{center}
	\begin{tabularx}{\textwidth}{l X}
		\toprule
		funzione & dati due insiemi \(X\), \(Y\) t.c. \(X,Y \neq \varnothing\), una funzione \(y = f(x)\) è una relazione che associa ad ogni elemento \(x \in X\) un unico elemento \(y \in Y\) \\
		\midrule
		dominio & insieme \(X\) \\
		\midrule
		codominio & insisme \(Y\) \\
		\midrule
		\multirow{2}{*}{immagine} & sottoinsieme di \(Y\) definito come \(Im(f) = f(A)\)\\
		& \(f(A) = \left\{ y \in Y  \text{ t.c. } \exists x \in A \text{ con } f(x) = y, A \subseteq X \right\}\) \\
		\midrule
		\multirow{2}{*}{controimmagine} & sottoinsieme di \(X\) definito come \\
		& \(f(B) ^ {-1} = \left\{ x \in X  \text{ t.c. } \exists y \in B \text{ con } f(x) = y, B \subseteq Y \right\}\) \\
		\midrule
		\multirow{2}{*}{grafico} & sottoinsieme di \(X \times Y\) definito come \\
		& \(G(f) = \left\{ \left( x; y \right) \in X \times Y \text{ t.c. } y = f(x) \right\}\) \\
		\bottomrule
	\end{tabularx}
\end{center}
\begin{center}
	\begin{tabularx}{\textwidth}{l X}
		\toprule
		\(f\) a variabili reali & se il dominio \(X \subseteq R\) \\
		\midrule
		\(f\) a valori reali & se il codominio \(Y \subseteq R\) \\
		\midrule
		\(f\) parte intera & \(f(x) = \left[ x \right]\) definta in \(\mathbb{R}\) come il più grande numero intero \(\leq x\) \\
		\midrule
		\(f\) parte frazionaria & \(f(x) = x- \left[ x \right]\)\\
		\midrule
		funzione di Dirichlet & \[f(x) = \begin{cases}
			1 & x \in \mathbb{Q} \\
			0 & x \in \mathbb{R} \; \backslash \; \mathbb{Q}
		\end{cases}\] \\
		\bottomrule
	\end{tabularx}
\end{center}
\begin{center}
	\begin{tabularx}{\textwidth}{l X}
		\toprule
		\(f\) pari & \(f\) è pari se \(\forall x \in \dom f\) e \(-x \in \dom f\), allora \(f(x) = f(-x)\) \\
		\midrule
		\(f\) dispari & \(f\) è dispari se \(\forall x \in \dom f\) e \(-x \in \dom f\), allora \(-f(x) = f(-x)\)\\
		\midrule
		\multirow{2}{*}{\(f\) iniettiva} & \(f:X \to Y\) è iniettiva se \(\forall y \in Y\) \(\exists\)al più un \(x \in X\) t.c. \(f(x) = y\), equivalentemente: \\
		& \(x_1 \neq x_2 \Rightarrow f(x_1) \neq f(x_2)\) e \(f(x_1) = f(x_2) \Rightarrow x_1 = x_2\) \\
		\midrule
		\(f\) suriettiva & \(f:X \to Y\) è suriettiva se \(\forall y \in Y\) \(\exists x \in X\) t.c. \(f(x) = y\) \\
		\midrule
		\(f\) bigettiva & \(f:X \to Y\) è bigettiva se è sia iniettiva che suriettiva \\
		\midrule
		\(f\) invertibile & \(f:X \to Y\) è invertibile se è bigettiva \\
		\midrule
		\multirow{3}{*}{\(f\) inversa} & \(f ^ {-1}:f(X) \subseteq Y \to X\) t.c. \(y \mapsto f ^ {-1}\) ovvero l'unico \(x \in X\) t.c. \(f(x) = y\) \\
		& alcune funzioni inverse \dots \\
		& proprietà pari dispari delle inverse \dots \\
		\midrule
		\(f\) composta & siano \(f:X \to Y\), \(g:V \to Z\) con \(f(X) \cap V \neq \varnothing\)
		e \(\overline{X} \subseteq X\) t.c. \(\overline{X} := \left\{ x \in X \text{ t.c. } f(x) \in V \right\}\),
		la funzione composta di \(f\) con \(g\) è definita come \(g \circ f : \overline{X} \to Z; x \mapsto g(f(x))\) \\
		\midrule
		\(f\) ristretta & \(\restr{f}{A} = A \subseteq X \to Y\) t.c. \(x \mapsto f(x)\) \\
		\midrule
		\multirow{2}{*}{\(f\) periodica} & \(f\) è periodica se \(f(x+T) = f(x), \forall x \in \dom f\) \\
		& per \(T \in R, T > 0, x + T \in \dom f\)\\
		\bottomrule
	\end{tabularx}
	\begin{tabularx}{\textwidth}{l X}
		\toprule
		\(f\) monotona crescente & se \(\forall x_1, x_2 \in \dom f\), \(x_1 < x_2\) si ha \(f(x_1) \leq f(x_2)\) \\
		\midrule
		\(f\) monotona decrescente & se \(\forall x_1, x_2 \in \dom f\), \(x_1 < x_2\) si ha \(f(x_1) \geq f(x_2)\) \\
		\midrule
		\(f\) strett. crescente & se \(\forall x_1, x_2 \in \dom f\), \(x_1 < x_2\) si ha \(f(x_1) < f(x_2)\) \\
		\midrule
		\(f\) strett. decrescente & se \(\forall x_1, x_2 \in \dom f\), \(x_1 < x_2\) si ha \(f(x_1) > f(x_2)\) \\
		\midrule
		\(f\) limitata superiormente & se \(Imf\) è limitata superiormente, ovvero se \(\exists M \in \mathbb{R}\) t.c. \(f(x) \leq M\), \(\forall x \in \dom f\) \\
		\midrule
		\(f\) limitata inferiormente & se \(Imf\) è limitata inferiormente, ovvero se \(\exists m \in \mathbb{R}\) t.c. \(f(x) \geq m\), \(\forall x \in \dom f\) \\
		\midrule
		\(f\) limitata & se \(Imf\) è limitata superiormente e inferiormente: \(\exists m, M \in \mathbb{R}\) t.c. \(m \leq f(x) \leq M\), \(\forall x \in \dom f\) \\
		\midrule
		maggioranti, minoranti & un maggiorante, minorante, massimo, minimo, \\
		massimi, minini, estemi & estremo inferiore, estremo superiore di \(f\) è \\
		sup. e inf. di \(f\) & definito come magg\dots di \(Im f\) \\
		\bottomrule
	\end{tabularx}
\end{center}
%\subsection{Funzione inversa}

\newpage

% ------------------------------------- Funzioni Iperboliche --------------------------------------
\section{Funzioni iperboliche}
\subsection{Coseno iperbolico}
\subsubsection*{Definizione}
Il coseno iperbolico è una unzione definita come:
\[\cosh : \quad
\begin{aligned}
	\mathbb{R} \quad &\to \quad \mathbb{R} \\
	x \quad &\mapsto \quad \frac{e^x + e^-x}{2}
\end{aligned}
\]

\subsubsection*{Proprietà}
Il coseno iperbolico è pari, decrescente in \(\left( - \infty, 0 \right]\) e decrescente in \(\left[ 0, + \infty \right)\). Non è una funzione iniettiva. \\
Somma: \(\cosh \left( x + y \right) = \cosh x \cosh y + \sinh x \sinh y\) \\
Differenza: \(\cosh \left( x - y \right) = \cosh x \cosh y - \sinh x \sinh y\)

\subsubsection*{Funzione inversa}
La funzione inversa del coseno iperbolico, definita come settore coseno iperbolico, è definita come:
\[\settcosh : \quad
\begin{aligned}
	\left[ 1, + \infty \right) \quad &\to \quad \left[ 0, + \infty \right) \\
	x \quad &\mapsto \quad \ln \left( x + \sqrt{ x^2 - 1 } \right)
\end{aligned}
\]
Da notare che siccome il \(\cosh\) non è invertibile, è necessario restringere la funzione a \(\restr{\cosh}{\left[ 0, + \infty \right)}\)
che ha come dominio \(\dom = \left[ 0, + \infty \right)\) e come immagine \(Im = \cosh \left( \left[ 0, + \infty \right) \right) = \left[ 1, + \infty \right)\).

\subsection{Seno iperbolico}
\subsubsection*{Definizione}
Il seno iperbolico è una unzione definita come:
\[\sinh : \quad
\begin{aligned}
	\mathbb{R} \quad &\to \quad \mathbb{R} \\
	x \quad &\mapsto \quad \frac{e^x - e^-x}{2}
\end{aligned}
\]

\subsubsection*{Proprietà}
Il seno iperbolico è dispari, sempre crescente, inoltre è una funzione iniettiva e suriettiva. \\
Somma: \(\sinh \left( x + y \right) = \sinh x \cosh y + \sinh y \cosh x\) \\
Differenza: \(\sinh \left( x - y \right) = \sinh x \cosh y - \sinh y \cosh x\)

\subsubsection*{Funzione inversa}
La funzione inversa del seno iperbolico, definita come settore seno iperbolico, è definita come:
\[\settsinh : \quad
\begin{aligned}
	\mathbb{R} \quad &\to \quad \mathbb{R} \\
	x \quad &\mapsto \quad \ln \left( x + \sqrt{ x^2 + 1 } \right)
\end{aligned}
\]

\subsection{Relazione fondamentale}
\[\cosh^2 x - \sinh^2 x = 1\]

\subsection{Tangente iperbolica}
\subsubsection*{Definizione}
Il coseno iperbolico è una unzione definita come
\[\tanh : \quad
\begin{aligned}
	\mathbb{R} \quad &\to \quad \mathbb{R} \\
	x \quad &\mapsto \quad \frac{e^x - e^-x}{e^x + e^-x}
\end{aligned}	
\]

\subsubsection*{Proprietà}
La tangente iperbolica è dispari, sempre crescente, inoltre è sia iniettiva che suriettiva, con immagine \(\left( -1, 1 \right)\).

\subsubsection*{Funzione inversa}
La funzione inversa del coseno iperbolico, definita come settore coseno iperbolico, è definita come:
\[\setttanh : \quad
\begin{aligned}
	\left( -1, 1 \right) \quad &\to \quad \mathbb{R} \\
	x \quad &\mapsto \quad \frac{1}{2} \ln \left( \frac{1 + x}{1 - x} \right)
\end{aligned}
\]

\newpage

% -------------------------------------------- Limiti ---------------------------------------------
\section{Limiti}
\subsection{Intorni}
\subsubsection*{Definizione}
Sia \(r \in \mathbb{R}^*\), allora un intorno "sferico" centrato in \(r\) è un intervallo aperto definito come:
\begin{align*}
	\left( r - \varepsilon, r + \varepsilon \right) \quad & \text{ se } r \in \mathbb{R} \\
	\left( M, +\infty \right) \quad & \text{ se } r = + \infty\\
	\left( -\infty, N \right) \quad & \text{ se } r = - \infty
\end{align*}

\subsubsection*{Proprietà}
\begin{itemize}
	\item[P\(_1\):] Sia \(r \in \mathbb{R}^*\) e siano \(U_1\) e \(U_2\) due intorni di \(r\), allora \(U_1 \cap U_2\) è ancora un intorno di \(r\).
	\item[Dim\(_1\):] consideriamo solo il caso per cui \(r \in \mathbb{R}\) (con \(r \neq \pm \infty\)) \\
	Siano \(U_1 = \left( r - \varepsilon_1, r + \varepsilon_1 \right)\) con \(\varepsilon_1 > 0\) oppure \(U_1 = \mathbb{R}\) \\
	e \(U_2 = \left( r - \varepsilon_2, r + \varepsilon_2 \right)\) con \(\varepsilon_2 > 0\) oppure \(U_2 = \mathbb{R}\) \\
	Prendiamo \(\varepsilon = \min \left\{ \varepsilon_1 , \varepsilon_2 \right\}\) e abbiamo che \(U_1 \cap U_2 = \left( r - \varepsilon, r + \varepsilon \right)\) che è un intorno di \(r\). \qed
	
	\item[P\(_2\):] \textbf{Proprietà di separazione}: \(\forall r_1, r_2 \in \mathbb{R}^*\) con \(r_1 \neq r_2\) esistono \(U_1\) e \(U_2\) intorni
	rispettivamente di \(r_1\) e \(r_2\) tali che \(U_1 \cap U_2 = \varnothing\).
	\item[Dim\(_2\):] consideriamo solo il caso per cui \(r_1, r_2 \in \mathbb{R}\) (con \(r_1, r_2 \neq \pm \infty\)) \\
	Assumiamo \(r_1 < r_2\) senza perdita di generalità (possono essere invertiti) e vogliamo dimostrare che esistono due intorni \(U_1, U_2\) t.c. \(U_1 \cap U_2 = \varnothing\) \\
	Scegliamo \(\varepsilon_1 = \varepsilon_2 = \displaystyle \frac{r_2 - r_1}{2}\) \\
	per cui \(\forall x_1 \in (r_1 - \varepsilon_1 , r_1 + \varepsilon_1)\) e \(\forall x_2 \in (r_2 - \varepsilon_2 , r_2 + \varepsilon_2)\) \\
	per cui \(x_1 < r_1 + \displaystyle \frac{r_2 - r_1}{2} = \frac{r_2 + r_1}{2} = r_2 - \frac{r_2 - r_1}{2} < x_2\) \\
	per cui \(\left( r_1 - \varepsilon_1, r_1 + \varepsilon_1 \right) \cap \left( r_2 - \varepsilon_2, r_2 + \varepsilon_2 \right) = \varnothing\) \qed
\end{itemize}


\subsection{Punti di accumulazione}
Sia \(A \subseteq \mathbb{R}\) con \(A \neq \varnothing\), \(r \in \mathbb{R}^*\) è punto di accumulazione di \(A\) quando:
\[\forall \text{ intorno } U \text{ di } r \text{ si ha che } A \cap U \; \backslash \left\{ r \right\} \neq \varnothing\]
In altre parole ogni intorno di \(r\) deve contenere (almeno) un elemento di \(A\) che non sia \(r\) stesso. Un punto di accumulazione di \(A\) può non appartenere all'insieme \(A\).
\begin{align*}
	\text{ se } r \in \mathbb{R} \quad & \forall \varepsilon > 0 \; \exists x \in A, x \neq r \text{ t.c. } \left| x - r \right| < \varepsilon \text{ ovvero } x \in \left( r - \varepsilon, r + \varepsilon \right) \\
	\text{ se } r = + \infty \quad & \forall M \in \mathbb{R} \; \exists x \in A \text{ t.c. } x > M \text{ ovvero che } A \text{ non è limitato superiormente } \\
	\text{ se } r = - \infty \quad & \forall N \in \mathbb{R} \; \exists x \in A \text{ t.c. } x < N \text{ ovvero che } A \text{ non è limitato inferiormente }
\end{align*}


\subsection{Punti isolati}
Sia \(A \subseteq \mathbb{R}\) con \(A \neq \varnothing\), \(r \in \mathbb{R}^*\) è punto isolato di \(A\) se non è un punto di accumulazione,
ovvero se \(\exists\) un intorno \(U\) di \(r\) t.c. \(U \cap A = \left\{ r \right\}\). \\
Un punto isolato di \(A\) deve necessariamente appartenere all'insieme \(A\), inolte \(\pm \infty\) non possono essere punti isolati, ma soltanto punti di accumulazione.

\newpage


\subsection{Intorni e proprietà vere definitivamente}
Sia \(f: \dom f \to \mathbb{R}\), \(x_0 \in \mathbb{R}^*\) punto di accumulazione di \(\dom f\) e \(P\) proprietà definita sul \(\dom f\),
allora \(f\) soddisfa \(P\) definitivamente per \(x \to x_0\) se \(\exists U\) intorno di \(x_0\) t.c. \(\forall x \in \dom f \cap U \; \backslash \left\{ x_0 \right\}\) \(f(x)\) verifica \(P\).
\subsubsection*{Osservazioni}
\begin{itemize}
	\item[O1:] Non è detto che \(P\) sia verificata in \(x_0\) (infatti \(x_0\) potrebbe \(\notin \dom f\)).
	\item[O2:] Basta che esista un intorno per cui \(P\) sia verificata, altrimenti se \(P\) è verificata per ogni intorno,
	ovvero \(P\) vale \(f\forall x \in \dom f \; \backslash \left\{ x_0 \right\}\), non significa che \(P\) sia verificata vicino a \(x_0\).
\end{itemize}


\subsection{Limite di una funzione}
\subsubsection*{Definizione}
Sia \(f: \dom f \to \mathbb{R}\), \(x_0 \in \mathbb{R}^*\) punto di accumulazione di \(\dom f\) e \(l \in \mathbb{R}^*\), allora:
\[\lim_{x \to x_0} f(x) = l \quad \text{quando}\]
\begin{itemize}
	\item[I:] \(\forall U\) intorno di \(l\) \(\exists V\) intorno di \(x_0\) t.c. \(f(x) \in U, \forall x \in \dom f \cap V \; \backslash \left\{ x_0 \right\}\)
	\item[II:] \(\forall U\) intorno di \(l\) \(\exists V\) intorno di \(x_0\) t.c. \(\forall x \in \dom f \cap V \; \backslash \left\{ x_0 \right\} \Rightarrow f(x) \in U\)
	\item[III:] \(\forall U\) intorno di \(l\) \(f(x) \in U\) è verificata definitivamente per \(x \to x_0\)
\end{itemize}
Usando la definizione di intorno si ottiene che:
%\begin{enumerate}
%	\item \(x_0 \in \mathbb{R}\) e \(l \in \mathbb{R}\)
%		\[\forall \varepsilon > 0 \; \exists \delta > 0 \text{ t.c. } \left| f(x) - l \right| < \varepsilon \; \forall x \in \dom f \cap \left( x_0 - \delta, x_0 + \delta \right) \backslash \left\{ x_0 \right\}\]
%	\item \(x_0 \in \mathbb{R}\) e \(l = - \infty\)
%		\[\forall M \in \mathbb{R} \; \exists \delta > 0 \text{ t.c. } f(x) > M \; \forall x \in \dom f \cap \left( x_0 - \delta, x_0 + \delta \right) \backslash \left\{ x_0 \right\}\]
%	\item \(x_0 \in \mathbb{R}\) e \(l = + \infty\)
%		\[\forall N \in \mathbb{R} \; \exists \delta > 0 \text{ t.c. } f(x) < N \; \forall x \in \dom f \cap \left( x_0 - \delta, x_0 + \delta \right) \backslash \left\{ x_0 \right\}\]
%	\item \(x_0 = + \infty\) e \(l \in \mathbb{R}\)
%		\[\forall \varepsilon > 0 \; \exists M \text{ t.c. } \left| f(x) - l \right| < \varepsilon \; \forall x \in \dom f \cap \left( M, + \infty \right) \backslash \left\{ x_0 \right\}\]
%	\item \(x_0 = + \infty\) e \(l = - \infty\)
%		\[\forall M \in \mathbb{R} \; \exists \overline{M} \text{ t.c. } f(x) > M \; \forall x \in \dom f \cap \left( \overline{M}, + \infty \right) \backslash \left\{ x_0 \right\}\]
%	\item \(x_0 = + \infty\) e \(l = + \infty\)
%		\[\forall N \in \mathbb{R} \; \exists M \text{ t.c. } f(x) < N \; \forall x \in \dom f \cap \left( M, + \infty \right) \backslash \left\{ x_0 \right\}\]
%	\item \(x_0 = - \infty\) e \(l \in \mathbb{R}\)
%		\[\forall \varepsilon > 0 \; \exists N \text{ t.c. } \left| f(x) - l \right| < \varepsilon \; \forall x \in \dom f \cap \left( - \infty, N \right) \backslash \left\{ x_0 \right\}\]
%	\item \(x_0 = - \infty\) e \(l = - \infty\)
%		\[\forall M \in \mathbb{R} \; \exists N \text{ t.c. } f(x) > M \; \forall x \in \dom f \cap \left( - \infty, N \right) \backslash \left\{ x_0 \right\}\]
%	\item \(x_0 = - \infty\) e \(l = + \infty\)
%		\[\forall N \in \mathbb{R} \; \exists \overline{N} \text{ t.c. } f(x) < N \; \forall x \in \dom f \cap \left( - \infty, \overline{N} \right) \backslash \left\{ x_0 \right\}\]
%\end{enumerate}
\begin{align*}
		\forall \varepsilon > 0 \; \exists \delta > 0 \text{ t.c. } \left| f(x) - l \right| < \varepsilon \; \forall x \in \dom f \cap \left( x_0 - \delta, x_0 + \delta \right) \backslash \left\{ x_0 \right\} \quad & x_0, l \in \mathbb{R} \\
		\forall M \in \mathbb{R} \; \exists \delta > 0 \text{ t.c. } f(x) > M \; \forall x \in \dom f \cap \left( x_0 - \delta, x_0 + \delta \right) \backslash \left\{ x_0 \right\} \quad & x_0 \in \mathbb{R}, l = + \infty \\
		\forall N \in \mathbb{R} \; \exists \delta > 0 \text{ t.c. } f(x) < N \; \forall x \in \dom f \cap \left( x_0 - \delta, x_0 + \delta \right) \backslash \left\{ x_0 \right\} \quad & x_0 \in \mathbb{R}, l = - \infty \\
		\\
		\forall \varepsilon > 0 \; \exists M \text{ t.c. } \left| f(x) - l \right| < \varepsilon \; \forall x \in \dom f \cap \left( M, + \infty \right) \backslash \left\{ x_0 \right\} \quad & x_0 = + \infty, l \in \mathbb{R} \\
		\forall M \in \mathbb{R} \; \exists \overline{M} \text{ t.c. } f(x) > M \; \forall x \in \dom f \cap \left( \overline{M}, + \infty \right) \backslash \left\{ x_0 \right\} \quad & x_0 = + \infty, l = + \infty \\
		\forall N \in \mathbb{R} \; \exists M \text{ t.c. } f(x) < N \; \forall x \in \dom f \cap \left( M, + \infty \right) \backslash \left\{ x_0 \right\} \quad & x_0 = + \infty, l = - \infty \\
		\\
		\forall \varepsilon > 0 \; \exists N \text{ t.c. } \left| f(x) - l \right| < \varepsilon \; \forall x \in \dom f \cap \left( - \infty, N \right) \backslash \left\{ x_0 \right\} \quad & x_0 = - \infty, l \in \mathbb{R} \\
		\forall M \in \mathbb{R} \; \exists N \text{ t.c. } f(x) > M \; \forall x \in \dom f \cap \left( - \infty, N \right) \backslash \left\{ x_0 \right\} \quad & x_0 = - \infty, l = + \infty \\
		\forall N \in \mathbb{R} \; \exists \overline{N} \text{ t.c. } f(x) < N \; \forall x \in \dom f \cap \left( - \infty, \overline{N} \right) \backslash \left\{ x_0 \right\} \quad & x_0 = - \infty, l = - \infty
\end{align*}


\subsubsection*{Limiti definiti e indefiniti}
Per \(\displaystyle \lim_{x \to x_0} f(x) = l\):
\begin{center}
	\begin{tabular}{l l}
		se \(l \in \mathbb{R}\) & il limite esiste ed è finito \\
		se \(l = \pm \infty\) & il limite esiste ed è infinito \\
		se \(l = 0\) & la funzione è infinitesima \\
		se \(l\) non è definibile & \multirow{2}{*}{il limite non esiste ed è indefinito} \\
		univocamente &
	\end{tabular}
\end{center}


\subsection*{Osservazioni}
\(x_0\) punto di accumulazione di \(\dom f\) non assicura che \(x_0 \in \dom f\), infatti se \(x_0 \notin \dom f\) non ha senso \(f(x_0)\) e
se \(x_0 \in \dom f\) il valore di \(f(x_0)\) non influenza il limite, in quanto si esclude il valore di \(x_0\).

Graficamente, la definizione di limite significa scegliere un certo "errore" \(\varepsilon\) lungo l'asse \(y\) e trovare un intorno di \(x_0\) lungo l'asse \(x\) per cui
preso qualsiasi punto nell'intervallo (escluso \(x_0\)), si ha che i valori assunti dalla funzione differiscono da un valore \(l\) al più di \(\varepsilon\).

\newpage


\subsection{Teorema di unicità del limite}
Se il limite esiste, è unico.
\begin{itemize}
	\item[P:] Dati \(f: \dom f \to \mathbb{R}\) e \(x_0\) punto di accumulazione di \(\dom f\),
	\item[H:] se valgono \(\displaystyle \lim_{x \to x_0} f(x) = l_1\) e \(\displaystyle \lim_{x \to x_0} f(x) = l_2\),
	\item[T:] allora \(l_1 = l_2\).
	\item[Dim:] Per assurdo supponiamo che \(l_1 \neq l_2\) \\
	dalla proprietà di separazione degli intorni \(\exists V_1\) e \(V_2\) intorni di \(l_1\) e \(l_2\) tali che \(V_1 \cap V_2 = \varnothing\) \\
	dalle definizioni: \(\lim_{x \to x_0} f(x) = l_1 \Rightarrow \exists U_1\) intorno di \(x_0\) t.c. \(f(x) \in V_1\) \(\forall x \in U_1 \cap \dom f \; \backslash \left\{ x_0 \right\}\) \\
	\(\lim_{x \to x_0} f(x) = l_2 \Rightarrow \exists U_2\) intorno di \(x_0\) t.c. \(f(x) \in V_2\) \(\forall x \in U_2 \cap \dom f \; \backslash \left\{ x_0 \right\}\) \\
	si considera \(U = U_1 \cap U_2\) per cui \(\forall x \in U \cap \dom f \; \backslash \left\{ x_0 \right\}\) vale \(f(x) \in V_1 \cap V_2 = \varnothing\) \\
	cioè \(\nexists x \in U \cap \dom f \; \backslash \left\{ x_0 \right\}\) che è in contaddizione con il fatto che \(x_0\) è un punto di accumulazione di \(\dom f\),
	per cui l'ipotesi che \(l_1 \neq l_2\) è sbagliata. \qed	
\end{itemize}


\subsection{Limite finito implica locale limitatezza}
\begin{itemize}
	\item[H:] Se \(\displaystyle \lim_{x \to x_0} f(x) = l \in \mathbb{R}\), cioè \(l \neq \pm \infty\),
	\item[T:] allora \(\exists U\) intorno di \(x_0\) e \(N \in \mathbb{R}\) t.c. \(\forall x \in U \cap \dom f \; \backslash \left\{ x_0 \right\}\) vale \(\left| f(x) - l \right| \leq N\).
	\item[Dim:] Dalla definizione di limite con \(\varepsilon = 1\) abbiamo che \(\exists U\) intorno di \(x_0\) t.c. \(\forall x \in U \cap \dom f \; \backslash \left\{ x_0 \right\}\) vale: \\
	\(\left| f(x) - l \right| < 1 \Rightarrow \left| f(x) \right| < \left| l \right| + 1\).	Scegliendo \(N = \left| f(x) \right| < \left| l \right| + 1\) si ottiene la tesi. \qed 
\end{itemize}


\subsection{Limite destro e limite sinistro}
\subsubsection*{Punti di accumulazione destro e sinistro}
Sia \(A \subset \mathbb{R}\), \(A = \varnothing\), un punto \(r \in \mathbb{R}\) è \textbf{punto di accumulazione destro} di \(A\) quando \(r\) è punto di accumulazione di \(A \cap \left( r, + \infty \right)\). \\
Sia \(A \subset \mathbb{R}\), \(A = \varnothing\), un punto \(r \in \mathbb{R}\) è \textbf{punto di accumulazione sinistro} di \(A\) quando \(r\) è punto di accumulazione di \(A \cap \left( - \infty, r \right)\). \\
Un punto di accumulazione destro o sinistro è necessariamente anche un punto di accumulazione,
un punto di accumulazione è anche punto di accumulazione destro oppure sinistro, non è detto che sia entrambi.

\subsubsection*{Intorni destro e sinistro}
Un \textbf{intorno destro} di \(r \in \mathbb{R}\) è un insieme della forma \(\left( r, r + \delta \right)\). \\
Un \textbf{intorno sinistro} di \(r \in \mathbb{R}\) è un insieme della forma \(\left( r - \delta, r \right)\).

\subsubsection*{Limiti destro e sinistro}
Sia \(f: \dom f \to \mathbb{R}\), \(x_0 \in \mathbb{R}^*\) punto di accumulazione destro di \(\dom f\) e \(l \in \mathbb{R}^*\), allora il \textbf{limite destro} è definito come:
\[\lim_{x \to {x_0}^+} f(x) = l \quad \Leftrightarrow \quad \lim_{x \to {x_0}} \restr{f(x)}{\left( x_0, + \infty \right)} = l \quad \text{cioè quando:}\]
\begin{itemize}
	\item[I:] \(\forall U\) intorno di \(l\) \(\exists V\) intorno destro di \(x_0\) t.c. \(f(x) \in U, \forall x \in \dom f \cap V\)
	\item[II:] \(\forall U\) intorno di \(l\) \(\exists \delta > 0\) t.c. \(\forall x \in \dom f \cap \left( x_0, x_0 + \delta \right) \Rightarrow f(x) \in U\)
\end{itemize}

Sia \(f: \dom f \to \mathbb{R}\), \(x_0 \in \mathbb{R}^*\) punto di accumulazione sinistro di \(\dom f\) e \(l \in \mathbb{R}^*\), allora il \textbf{limite sinistro} è definito come:
\[\lim_{x \to {x_0}^-} f(x) = l \quad \Leftrightarrow \quad \lim_{x \to {x_0}} \restr{f(x)}{\left( - \infty, x_0 \right)} = l \quad \text{cioè quando:}\]
\begin{itemize}
	\item[I:] \(\forall U\) intorno di \(l\) \(\exists V\) intorno sinistro di \(x_0\) t.c. \(f(x) \in U, \forall x \in \dom f \cap V\)
	\item[II:] \(\forall U\) intorno di \(l\) \(\exists \delta > 0\) t.c. \(\forall x \in \dom f \cap \left( x_0 - \delta, x_0 \right) \Rightarrow f(x) \in U\)
\end{itemize}

\subsubsection*{Teorema di unicità del limite destro e sinistro}
Se il limite destro esiste, è unico. \\
Se il limite sinistro esiste, è unico. 

I due teoremi si dimostrano in quando limite destro e limite sinistro sono limiti di funzioni ristrette e in quanto limiti, se esistono sono unici. \qed

\subsubsection*{Teorema della relazione tra limite e limiti destro e sinistro}
\begin{itemize}
	\item[P:] Sia \(x_0\) punto di accumulazione sia destro che sinistro di \(\dom f\),
	\item[H \(\Leftrightarrow\) T:] \(\displaystyle \lim_{x \to x_0} f(x) = l \in \mathbb{R}^* \quad \Leftrightarrow \quad
	\begin{cases}
		\displaystyle \lim_{x \to {x_0}^+} f(x) = l \\
		\displaystyle \lim_{x \to {x_0}^-} f(x) = l
	\end{cases}
	\)

	\item[Dim \(\Rightarrow\):] Dall'ipotesi si ha che \(\forall U\) intorno di \(l\) \(\exists V\) intorno di \(x_0\) t.c. \(\forall x \in V \cap \dom f \; \backslash \left\{ x_0 \right\} \Rightarrow \left| f(x) - l \right| < \varepsilon\) \\
	Considero come intorno destro \(V+ = V \cap \left( x_0, + \infty \right) \) e come intorno sinistro \(V^- = V \cap \left( -\infty, x_0 \right)\), \\
	per cui \(\forall U\) \(\exists V^+\) t.c. \(\forall x \in V^+ \cap \dom f \Rightarrow \left| f(x) - l \right| < \varepsilon\) è valida in quanto \(V^+ \subset V\) e anche \\
	\(\forall U\) \(\exists V^-\) t.c. \(\forall x \in V^- \cap \dom f \Rightarrow \left| f(x) - l \right| < \varepsilon\) è valida in quanto \(V^- \subset V\), \\
	ovvero \(\displaystyle \lim_{x \to {x_0}^+} f(x) = l\) e \(\displaystyle \lim_{x \to {x_0}^-} f(x) = l\)
	
	\item[Dim \(\Leftarrow\):] Dall'ipotesi si ha che \(\forall U\) intorno di \(l\) \(\exists \delta_1 > 0\) t.c. \(\forall x \in \left( x_0, x_0 + \delta_1 \right) \cap \dom f \Rightarrow \left| f(x) - l \right| < \varepsilon\) \\
	e che \(\forall U\) intorno di \(l\) \(\exists \delta_2 > 0\) t.c. \(\forall x \in \left( x_0 - \delta_2, x_0 \right) \cap \dom f \Rightarrow \left| f(x) - l \right| < \varepsilon\) \\
	scelgo \(\delta = \min \left\{ \delta_1, \delta_2 \right\}\) per cui \(\forall U\) \(\exists \delta > 0\) t.c. \(\forall x \in \left( x_0 - \delta, x_0 + \delta \right) \cap \dom f \; \backslash \left\{ x_0 \right\} \Rightarrow \left| f(x) - l \right| < \varepsilon\), \\
	ovvero \(\displaystyle \lim_{x \to x_0} f(x) = l\) \qed
\end{itemize}
Il teorema della relazione di unicità del limite destro e sinistro si utilizza per:
\begin{itemize}
	\item[I:] dimostrare l'inesistenza di un limite, se \(\displaystyle \lim_{x \to {x_0}^- } f(x) \neq \lim_{x \to {x_0}^+} f(x)\)
	\item[II:] semplificare il calcolo del limite dove \(f\), funzione definita per casi, cambia forma
\end{itemize}

\subsection{Relazione tra limite e modulo}
\begin{itemize}
	\item[P:] Sia \(x_0\) punto di accumulazione di \(\dom f\)
	\item[H\(_1 \Leftrightarrow\) T\(_1\):] \(\displaystyle \lim_{x \to x_0} f(x) = 0 \quad \Leftrightarrow \quad \lim_{x \to x_0} \left| f(x) \right| = 0\)
	\item[Dim\(_1\):] osservando che \(\left| f(x) - 0 \right| = \left| f(x) \right| = \left| \left| f(x) \right| - 0 \right|\) e che \(\dom \left| f \right| = \dom f\)
	\begin{align*}
		\lim_{x \to x_0} f(x) = 0 \quad &\Leftrightarrow \quad \forall \varepsilon > 0 \; \exists U \text{ intorno di } x_0 \text{ t.c. } \left| f(x) - 0 \right| < \varepsilon \; \forall x \in \dom f \cap U \; \backslash \left\{ x_0 \right\} \\
		&\Leftrightarrow \quad \forall \varepsilon > 0 \; \exists U \text{ intorno di } x_0 \text{ t.c. } \left| \left| f(x) \right| - 0 \right| < \varepsilon \; \forall x \in \dom \left| f \right| \cap U \; \backslash \left\{ x_0 \right\} \\
		&\Leftrightarrow \quad \lim_{x \to x_0} \left| f(x) \right| = 0
	\end{align*} \qed
	
	\item[H\(_2 \Rightarrow\) - T\(_2\):] \(\displaystyle \lim_{x \to x_0} f(x) = l \quad \Rightarrow \quad \lim_{x \to x_0} \left| f(x) \right| = \left| l \right|\)
	\item[Dim\(_2\):] Considerando il caso per cui \(l \in \mathbb{R}\) \\
	Dall'ipotesi si ottiene che \(\forall \varepsilon > 0 \; \exists V\) intorno di \(x_0\) t.c. \(\left| f(x) - l \right| < \varepsilon \; \forall x \in \dom f \cap V \; \backslash \left\{ x_0 \right\}\) \\
	si osserva che \(\left| \left| f(x) \right| - \left| l \right| \right| \leq \left| f(x) - l \right| < \varepsilon\) per la disuguaglianza triangolare \\
	per cui per proprietà transitiva \(\left| \left| f(x) \right| - \left| l \right| \right| < \varepsilon\), inoltre \(\dom \left| f \right| = \dom f\)\\
	per cui vale che \( \forall \varepsilon > 0 \; \exists \overline{V}\) intorno di \(x_0\) t.c. \(\left| \left| f(x) \right| - \left| l \right| \right| < \varepsilon \; \forall x \in \dom \left| f \right| \cap \overline{V} \; \backslash \left\{ x_0 \right\}\) \\
	ovvero \(\displaystyle \lim_{x \to x_0} \left| f(x) \right| = \left| l \right|\) \qed
\end{itemize}
Si osserva che nel secondo teorema vale solo \(\Rightarrow\) e non anche \(\Leftarrow\), come nel primo (\(\Leftrightarrow\)). 
\newpage


\subsection{Teorema di permanenza del segno}
\begin{itemize}
	\item[P:] Sia \(\displaystyle \lim_{x \to x_0} f(x) = l\) con \(x_0, l \in \mathbb{R}^*\)
	\item[H\(_1\):] se \(l \in \left( 0, + \infty \right)\)
	\item[T\(_1\):] allora \(f\) è definitivamente strettamente positiva per \(x \to x_0\)
	\item[Dim\(_1\):] Nel caso in cui \(l = + \infty\) si ottiene che \(\displaystyle \lim_{x \to x_0} f(x) = + \infty\) \\
	Dalla def. di limite si ottiene che \(\forall M \in \mathbb{R}\) \(\exists U\) intorno di \(x_0\) t.c. \(f(x) > M\) \(\forall x \in \dom f \cap U \; \backslash \left\{ x_0 \right\}\) \\
	scegiendo \(M = 0\) si ottiene che \(f(x) > 0\) \(\forall x \in \dom f \cap U \; \backslash \left\{ x_0 \right\}\)
	\item[] Nel caso in cui \(l = \left( 0, + \infty \right)\) si ottiene che \(\displaystyle \lim_{x \to x_0} f(x) = l\) con \(l > 0\) \\
	Dalla def. di limite si ottiene che \(\forall \varepsilon > 0\) \(\exists U\) intorno di \(x_0\) t.c. \(\left| f(x) - l \right| < \varepsilon\) \(\forall x \in \dom f \cap U \; \backslash \left\{ x_0 \right\}\) \\
	scegiendo \(\varepsilon = l\) si ottiene che \(\left| f(x) - l \right| < l \quad \Leftrightarrow \quad -l < f(x) - l < l \quad \Leftrightarrow \quad 0 < f(x) < 2l \quad \Leftrightarrow \\
	\Leftrightarrow f(x) > 0\) \(\forall x \in \dom f \cap U \; \backslash \left\{ x_0 \right\}\), ovvero la tesi. \qed

	\item[H\(_2\):] se \(l \in \left( - \infty, 0 \right)\)
	\item[T\(_2\):] allora \(f\) è definitivamente strettamente negativa per \(x \to x_0\)
	\item[Dim\(_2\):] analogamente alla Dim\(_1\). \qed
\end{itemize}

\subsubsection*{Corollario o II versione del teorema di permanenza del segno}
\begin{itemize}
	\item[P:] Sia \(\displaystyle \lim_{x \to x_0} f(x) = l\) con \(x_0, l \in \mathbb{R}^*\)
	\item[H\(_1\):] se \(f\) è definitivamente positiva \(\left( \geq 0 \right)\) per \(x \to x_0\)
	\item[T\(_1\):] allora \(l \geq 0\)
	\item[Dim\(_1\):] Supponiamo per assurdo che \(l < 0\) \\
	Per il teorema di permanenza del segno \(f(x) < 0\) definitivamente per \(x \to x_0\), ma questo è in contraddizione con l'ipotesi,
	per cui \(l\) deve necessariamente essere \(\geq 0\). \qed

	\item[H\(_2\):] se \(f\) è definitivamente negativa \(\left( \leq 0 \right)\) per \(x \to x_0\)
	\item[T\(_2\):] allora \(l \leq 0\)
	\item[Dim\(_2\):] analogamente alla Dim\(_1\). \qed
\end{itemize}
Da osserva che il teorema diventa falso se si sostiuisce \(\geq\) o \(\leq\) al posto di \(>\) o \(<\), dato che quando se \(l = 0\) non è possibile dedurre nessuna delle due conclusioni del teorema. \\
Inoltre il corollario diventa falso quando si sostiuisce \(>\) o \(<\) al posto di \(\geq\) o \(\leq\), per esempio \(f(x) = x^2 > 0\) definitivamente per \(x \to x_0 = 0\), ma \(\displaystyle \lim_{x \to 0} f(x) = l = 0 \ngtr 0\).

\newpage


\subsection{Teorema dei due carabinieri}
\subsubsection*{Caso limitato con \(l \in \mathbb{R}\)}
\begin{itemize}
	\item[P:] Siano \(f\), \(g\), \(h\), tre funzioni definite in \(X \subset \mathbb{R}\) e sia \(x_0\) punto di accumulazione di \(X\),
	\item[H:] se \(f(x) \leq g(x) \leq h(x)\) definitivamente per \(x \to x_0\) e \(\displaystyle \lim_{x \to x_0} f(x) = \lim_{x \to x_0} h(x) = l \in \mathbb{R}\)
	\item[T:] allora \(\displaystyle \lim_{x \to x_0} g(x) = l\)
	\item[Dim:] Scelto un \(\varepsilon > 0\) arbitrario, applicato alle definizioni dei seguenti limiti \(\displaystyle \lim_{x \to x_0} f(x) = l\) e \(\displaystyle \lim_{x \to x_0} h(x) = l\) \\
	si ottengono due intorni \(U_f\) e \(U_h\) per cui \(\begin{aligned}
		\left| f(x) - l \right| < \varepsilon \quad &\forall x \in U_f \cap X \backslash \left\{ x_0 \right\} \\
		\left| h(x) - l \right| < \varepsilon \quad &\forall x \in U_h \cap X \backslash \left\{ x_0 \right\}
	\end{aligned}\), \\ 
	mentre dalla prima ipotesi si ottiene che \(f(x) \leq g(x) \leq h(x)\) \(\forall x \in V\) intorno di \(x_0\) con \(V \subseteq X\) \\
	Scelto un intorno \(U = U_f \cap U_h \cap V\) si ha che \(\forall x \in U \cap X \backslash \left\{ x_0 \right\}\) valgono \\
	\(l - \varepsilon < f(x) \leq g(x) \leq h(x) < l + \varepsilon\), ovvero \(l - \varepsilon < g(x) < l + \varepsilon \Leftrightarrow \left| g(x) - l \right| < \varepsilon\) \(\forall x \in U \cap X \; \backslash \left\{ x_0 \right\}\)
	cioè la definizione di \(\displaystyle \lim_{x \to x_0} g(x) = l\) \qed
\end{itemize}

\subsubsection*{Caso illimitato con \(l = \pm \infty\)}
\begin{itemize}
	\item[P:] Siano \(f\), \(g\), due funzioni definite in \(X \subset \mathbb{R}\) e sia \(x_0\) punto di accumulazione di \(X\),
	\item[H\(_1\):] se \(f(x) \leq g(x)\) definitivamente per \(x \to x_0\) e \(\displaystyle \lim_{x \to x_0} f(x) = + \infty\)
	\item[T\(_1\):] allora \(\displaystyle \lim_{x \to x_0} g(x) = + \infty\)
	\item[H\(_2\):] se \(f(x) \geq g(x)\) definitivamente per \(x \to x_0\) e \(\displaystyle \lim_{x \to x_0} f(x) = - \infty\)
	\item[T\(_2\):] allora \(\displaystyle \lim_{x \to x_0} g(x) = - \infty\)
	\item[Dim:] Analoga a quella per il caso limitato\qed
\end{itemize}

Sia la dimostrazione per il caso limitato, sia quella per i casi illimitati, valgono anche per il limite destro e sinistro

\subsubsection*{Disuguaglianze di funzioni trigonometriche}
\begin{itemize}
	\item[H\(_1\):] per \(x \in \left( 0, \displaystyle \frac{\pi}{2} \right)\)
	\item[T\(_1\):] \(0 < \sin x \leq x \leq \tan x\)
	\item[Dim\(_1\):] Disegnando un arco di circonferenza di raggio \(r = 1\) con centro sull'origine \(O\) e chiamando \(P\) un punto sulla circonferenza, \(H\) la sua proiezione sull'asse \(x\),
	\(Q\) il punto di intersezione del semiasse positivo \(x\) con la circonferenza e \(R\), l'interesezione tra la perpendicolare a \(x\) per \(Q\) e la retta \(OP\): \\
	Si osserva che il tr. \(\overset{\triangle}{\text{OPH}}\) è contenuto nel settore circolare \(\overset{\smallfrown}{\text{OPQ}}\) che è contenuto nel tr. \(\overset{\triangle}{\text{ORQ}}\), per cui
	\(0 < A_{\overset{\triangle}{\text{OPH}}} \leq A_{\overset{\smallfrown}{\text{OPQ}}} \leq A_{\overset{\triangle}{\text{ORQ}}} \quad \Leftrightarrow
	\quad 0 < \frac{r \cdot \sin x}{2} \leq \frac{r ^ 2 \cdot x }{2} \leq \frac{r \cdot \tan x}{2} \quad \Leftrightarrow 0 < \sin x \leq x \leq \tan x \) \qed
	
	\item[H\(_2\):] per \(x \in \left( \displaystyle - \frac{\pi}{2}, 0 \right)\)
	\item[T\(_2\):] \(0 > \sin x \geq x \geq \tan x\)
	\item[Dim\(_2\):] si parte dalla disuguaglianza precedente, dove al posto di \(x \in \left( \displaystyle - \frac{\pi}{2}, 0 \right)\) viene posto \(-x = \left( 0, \displaystyle + \frac{\pi}{2} \right)\): \\
	\(0 < \sin \left( -x \right) \leq -x \leq \tan \left( -x \right) \;\; \Leftrightarrow \;\; 0 < -\sin x \leq -x \leq -\tan x \;\; \Leftrightarrow \;\; 0 > \sin x \geq x \geq \tan x\) \qed
\end{itemize}

\newpage


\subsection{Teorema sull'algebra dei limiti}
\subsubsection*{Caso limitato}
\begin{itemize}
	\item[P:] Siano \(f\), \(g\), due funzioni definite in \(X \subset \mathbb{R}\) e sia \(x_0\) punto di accumulazione di \(X\),
	\item[H:] se \(\displaystyle \lim_{x \to x_0} f(x) = l_f \in \mathbb{R}\) e \(\displaystyle \lim_{x \to x_0} g(x) = l_g \in \mathbb{R}\)
	\item[T\(_1\):] \(\displaystyle \lim_{x \to x_0} \left( f(x) \cdot g(x) \right) = l_f \cdot l_g\), ovvero limite del prodotto è il prodott dei limiti
	\item[T\(_2\):] \(\forall \alpha, \beta \in \mathbb{R} \quad \displaystyle \lim_{x \to x_0} \alpha f(x) + \beta g(x) = \alpha l_f + \beta l_g\), ovvero limite della somma è la somma dei limiti
	\item[T\(_3\):] \(\left( \text{se } l_g \neq 0 \right) \quad \displaystyle \lim_{x \to x_0} \frac{f(x)}{g(x)} = \frac{l_f}{l_g}\), ovvero limite del rapporto è il rapporto dei limiti
	\item[Dim\(_1\):] La tesi vuole che \(\forall \varepsilon > 0\) \(\exists V\) intorno di \(x_0\) t.c. \(\left| f(x)g(x) - l_f l_g \right| < \varepsilon\) \(\forall x \in X \cap V \; \backslash \left\{ x_0 \right\}\)
	\begin{align*}
		\left| f(x) g(x) - l_f l_g \right| &= \left| f(x) g(x) -f(x) l_g - f(x) l_g - l_f l_g\right| \\
		&= \left| f(x) \left( g(x) - l_g \right) - l_g \left( f(x) - l_f \right) \right| \\
		& \leq \left| f(x) \right| \cdot \left| g(x) - l_g \right| + \left| l_g \right| \cdot \left| f(x) - l_f \right| \qquad \text{per disuguaglianza triangolare}
	\end{align*}
	Per il teorema \textit{Limite finito implica locale limitatezza} abbiamo che \(\exists V_f\) intorno di \(x_0\), \(\exists M > 0\) \\ 
	t.c. \(\left| f(x) \right| < M\) \(\forall x \in X \cap V_f \; \backslash \left\{ x_0 \right\}\), per cui dato \(\overline{M} = \max \left\{ M, l_g \right\}\) abbiamo che:
	\[\left| f(x) g(x) - l_f l_g \right| \leq \overline{M} \cdot \left| g(x) - l_g \right| + \overline{M} \cdot \left| f(x) - l_f \right|\]
	per la def. di \(\displaystyle \lim_{x \to x_0} f(x) = l_f\), \(\forall \varepsilon_1 > 0\) \(\exists \overline{V_f}\) intorno di \(x_0\) t.c. \(\left| f(x) - l_f \right| < \varepsilon_1\) \(\forall x \in X \cap \overline{V_f} \; \backslash \left\{ x_0 \right\}\) \\
	per la def. di \(\displaystyle \lim_{x \to x_0} g(x) = l_g\), \(\forall \varepsilon_2 > 0\) \(\exists V_g\) intorno di \(x_0\) t.c. \(\left| g(x) - l_g \right| < \varepsilon_2\) \(\forall x \in X \cap V_g \; \backslash \left\{ x_0 \right\}\) \\
	scegliendo \(\varepsilon_1 = \varepsilon_2 = \frac{\varepsilon}{2 \overline{M}}\) si ha che:
	\[\left| f(x) g(x) - l_f l_g \right| \leq \overline{M} \cdot \frac{\varepsilon}{2 \overline{M}} + \overline{M} \cdot \frac{\varepsilon}{2 \overline{M}} = \varepsilon\]
	verificato \(\forall x \in X \cap V_f \cap \overline{V_f} \cap V_g\) \qed
	\item[Dim\(_2\):] 
	\begin{align*}
		\lim_{x \to x_0} \alpha f(x) + \beta g(x) &= \lim_{x \to x_0} \alpha f(x) + \lim_{x \to x_0} \beta g(x) \\
		&= \alpha \lim_{x \to x_0} f(x) + \beta \lim_{x \to x_0} g(x) \quad \text{ per T}_1 \\
		&= \alpha l_f + \beta l_g \quad \text{per ipotesi}
	\end{align*} \qed
	\item[Dim\(_3\):] considerando \(h(x) = \frac{1}{g(x)}\), si ottiene la T\(_1\) \qed
\end{itemize}

\newpage

\subsubsection*{Caso illimitato}
\begin{itemize}
	\item[P:] Siano \(f\), \(g\), due funzioni definite in \(X \subset \mathbb{R}\) e sia \(x_0\) punto di accumulazione di \(X\),
	\item[H\(_1\):] se \(\displaystyle \lim_{x \to x_0} f(x) = 0\) e \(g(x)\) definitivamente limitata per \(x \to x_0\), \(\displaystyle \lim_{x \to x_0} g(x)\) potrebbe anche non esistere
	\item[T\(_1\):] \(\displaystyle \lim_{x \to x_0} f(x) \cdot g(x) = 0\)
	\item[H\(_{2}\):] se \(\displaystyle \lim_{x \to x_0} f(x) = + \infty\) e \(g(x)\) definitivamente limitata per \(x \to x_0\)
	\item[T\(_{2}\):] \(\displaystyle \lim_{x \to x_0} f(x) + g(x) = + \infty\)
	\item[H\(_{3}\):] se \(\displaystyle \lim_{x \to x_0} f(x) = + \infty\) e \(g(x)\) definitivamente strettamente positiva per \(x \to x_0\)
	\item[T\(_{3}\):] \(\displaystyle \lim_{x \to x_0} f(x) \cdot g(x) = + \infty\)
	\item[H\(_{4}\):] se \(\displaystyle \lim_{x \to x_0} f(x) = + \infty\) e \(g(x)\) strettamente positiva e superiormente limitata definitivamente per \(x \to x_0\)
	\item[T\(_{4}\):] \(\displaystyle \lim_{x \to x_0} \frac{f(x)}{g(x)} = + \infty\)
	\item[Dim:] saltata
\end{itemize}
Per i casi 2, 3 e 4 in cui \(f(x) = - \infty\), bisogna invertire il segno del risultato del limite. Analogamente quando \(g(x) < 0\) nei punti 3 e 4.

\subsection{Teorema del confronto}
\begin{itemize}
	\item[P:] Siano \(f\), \(g\), due funzioni definite in \(X \subset \mathbb{R}\) e sia \(x_0\) punto di accumulazione di \(X\),
	\item[H:] se \(\displaystyle \lim_{x \to x_0} f(x) = l_f \in \mathbb{R}^*\) e \(\displaystyle \lim_{x \to x_0} g(x) = l_g \in \mathbb{R}^*\) e \(f(x) \leq g(x)\) definitivamente per \(x \to x_0\)
	\item[T:] \(l_f \leq l_g\) definitivamente per \(x \to x_0\)
	\item[Dim:] per \(l_f = -\infty\), \(l_g = + \infty\), \(l_f = l_g = + \infty\), o per \(l_f = l_g = - \infty\) la tesi è verificata, negli altri casi: \\
	definiamo \(h(x) = g(x) - f(x)\) t.c. \(h(x) \geq 0\) per ipotesi, definitivamente per \(x \to x_0\) \\
	Per il teorema di permamenza del segno: \(\displaystyle 0 \leq \lim_{x \to x_0} h(x) = \lim_{x \to x_0} g(x) - \lim_{x \to x_0} f(x) = l_g - l_f\) \\
	per cui \(0 \leq l_g - l_f\) ovvero la tesi \qed
\end{itemize}
Si osserva che il teorema vale soltanto con il \(leq\) e non con il \(<\), in quando se \(f(x) = 0\) e \(g(x) = x^2\),
\(f(x) < g(x)\) è definitivamente verificata per \(x \to 0\), ma \(\displaystyle \lim_{x \to 0} f(x) \nless \lim_{x \to 0} g(x)\)

\newpage

\subsection{Limiti delle funzioni composte}
\begin{itemize}
	\item[P:] Siano \(f: \dom f \to \mathbb{R}\) e \(g: \dom g \to \mathbb{R}\) due funzioni, 
	\(x_0\) punto di accumulazione di \(\dom f\) \\ e \(y_0 \in \mathbb{R}\) punto di accumulazione di \(\dom g\)
	\item[H\(_1\):] Se \(f(\dom f) \subset \dom g\), ovvero \(g \circ f : \dom f \to \mathbb{R}\) è definita,
	\item[H\(_2\):] \(\displaystyle \lim_{x \to x_0} f(x) = y_0\)
	\item[H\(_3\):] \(\displaystyle \lim_{x \to x_0} g(y) = l \in \mathbb{R}\)
	\item[H\(_4\):] \(f(x) \neq y_0\) per \(x \in \dom f\)
	\item[T:] \(\displaystyle \lim_{x \to x_0} g(f(x)) = l\)
	\item[Dim:] per H\(_2\): \(\forall \eta > 0\) \(\exists \delta > 0\) t.c. \(0 < \left| x - x_0 \right| < \delta \Rightarrow \left| f(x) - y_0 \right| < \eta\) \\
	per H\(_3\): \(\forall \varepsilon > 0\) \(\exists \eta > 0\) t.c. \(0 < \left| y - y_0 \right| < \eta \Rightarrow \left| g(y) - l \right| < \varepsilon\) \\
	per H\(_4\): \(\left| f(x) - y_0 \right| \neq 0 \) \\
	concatendando H\(_2\) e H\(_3\): \(\forall \varepsilon > 0\) \(\exists \delta > 0\) t.c. \(0 < \left| x - x_0 \right| < \delta \Rightarrow \left| g(f(x)) - l \right| < \varepsilon\) \\
	ovvero \(\displaystyle \lim_{x \to x_0} g(f(x)) = l\) \qed
\end{itemize}

\subsection{Limiti delle funzioni monotone}
\begin{itemize}
	\item[P:] Sia \(f: \left(a,b\right) \to \mathbb{R}\), con \(-\infty \leq a < b \leq + \infty\) e 
	\(a, b\) punti di accumulazione sinistro e destro dell'intervallo \(\left(a,b\right)\)
	\item[H\(_1\):] Se \(f\) è monotona crescente
	\item[T\(_1\):] allora \(\displaystyle \lim_{x \to a^+} f(x)) = \inf f(x)\) e \(\displaystyle \lim_{x \to b^-} f(x)) = \sup f(x)\) con \(x \in \dom f\)
	\item[H\(_2\):] Se \(f\) è monotona decrescente
	\item[T\(_2\):] allora \(\displaystyle \lim_{x \to a^+} f(x)) = \sup f(x)\) e \(\displaystyle \lim_{x \to b^-} f(x)) = \inf f(x)\) con \(x \in \dom f\)
	\item[Dim\(_1\):] supponiamo \(f(x)\) monotona crescente e limitata superiormente \\
	dalla def. di \(\sup\): \(\forall \varepsilon > 0\) \(\exists \overline{x} \in \dom f\) t.c. \(L - \varepsilon < f(x)\) e \(f(x) \leq L\) \(\forall x \in \dom f\) dove \(L\) è il \(\sup\) di \(f\) \\
	dalla def. di monotonia: \(\forall x, \overline{x} \in \dom f\) con \(x \geq \overline{x}\) si ha che \(f(x) \geq f(\overline{x})\) \\
	unendo le due def.: \(\forall \varepsilon > 0\) \(\exists \overline{x} \in \dom f\) t.c. \(\forall x \geq \overline{x}\) ho \(L - \varepsilon \leq f(\overline{x}) \leq f(x) (\leq L + \varepsilon)\) \\
	ovvero che \(\forall \varepsilon > 0\) \(\exists \overline{x} \in \dom f\) t.c. \(\forall x \in \left( \overline{x}, b \right)\) ho \(\left| f(x) - L \right| < \varepsilon\) \\
	cioè \(\displaystyle \lim_{x \to b^-} f(x) = L = \sup f\) \\
	Analogo per \(\inf f\) \qed
	\item[Dim\(_2\):] analogo alla Dim\(_1\) \qed
\end{itemize}

\newpage

\subsection{Funzioni continue e continuità}
\subsubsection*{DEfinizione di continuità}
Sia \(f: \dom f \to \mathbb{R}\), \(f(x)\) è continua in \(x_0 \in \dom f \Leftrightarrow \displaystyle \lim_{x \to x_0} f(x) = f(x_0)\). \\
Si osserva che se \(f: \dom f \in \mathbb{R}\) e \(g: \dom f \to \mathbb{R}\) continue in \(x_0\) punto di accumulazione \(\dom f\) \\ 
con \(\displaystyle \lim_{x \to x_0} f(x) = y_0\) e \(g \circ f\) è ben definita, allora \(\displaystyle \lim_{x \to x_0} g(f(x)) = g\left( \lim_{x \to x_0} f(x) \right) = g(y_0)\)

\subsubsection*{Funzioni continue in ogni punto del dominio}
\begin{enumerate}
	\item \(x^n\) con \(n \in \mathbb{N}\) e \(\dom = \mathbb{R}\)
	\item \(\left| x \right| ^ \alpha\) con \(\alpha \in \mathbb{R} \; \backslash \left\{ 0 \right\}\) e \(\dom = \mathbb{R}\)
	\item \(a^x\) con \(a \in \mathbb{R}, a > 0\) e \(\dom = \mathbb{R}\)
	\item \(\log_a n\) con \(a \in \mathbb{R}, a > 0, a \neq 1\) e \(\dom = \left(0, + \infty \right)\)
	\item \(\sin x, \cos x\) con \(\dom = \mathbb{R}\)
\end{enumerate}
Per le dimostrazioni vedere gli appunti

\subsection{Limiti a \(\pm \infty\)}
\begin{enumerate}
	\item \(\displaystyle \lim_{x \to + \infty} x ^ \alpha =
	\begin{cases}
		+ \infty & \text{se } \alpha > 0 \\
		1 & \text{se } \alpha = 0 \\
		0 & \text{se } \alpha < 0
	\end{cases}\)
	\item \(\displaystyle \lim_{x \to + \infty} a ^ x =
	\begin{cases}
		+ \infty & \text{se } a > 1 \\
		1 & \text{se } a = 1 \\
		0 & \text{se } 0 < a < 1
	\end{cases}\)
	\item \(\displaystyle \lim_{x \to + \infty} \log_a x =
	\begin{cases}
		+ \infty & \text{se } a > 1 \\
		- \infty & \text{se } 0 < a < 1
	\end{cases}\)
	\item \(\displaystyle \lim_{x \to 0} \log_a x =
	\begin{cases}
		- \infty & \text{se } a > 1 \\
		+ \infty & \text{se } 0 < a < 1
	\end{cases}\)
\end{enumerate}
Per le dimostrazioni vedere gli appunti

\subsection{Gerarchie degli infiniti}
\begin{itemize}
	\item[T\(_1\):] \(\displaystyle \lim_{x \to + \infty} \frac{a^x}{x^\alpha} = + \infty \qquad \forall a > 1, \alpha > 0\)
	\item[T\(_2\):] \(\displaystyle \lim_{x \to + \infty} \frac{x^\alpha}{\left( \log_a x \right) ^ \beta} = + \infty \qquad \forall a > 1, \alpha > 0, \beta > 0\)
	\item[Dim:] vedere appunti \qed
\end{itemize}

\newpage

\subsection{Numero di Nepero e alcuni limiti notevoli}
\subsubsection*{Definizione}
Il numero di Nepero \(e\) è definito come \(\displaystyle e = \lim_{x \to \pm \infty} \left( 1 + \frac{1}{x} \right) ^ x\)
con \(e = 2,71828\) ed \(e \in \mathbb{R} \; \backslash \mathbb{Q}\)

\subsubsection*{Altri limiti notevoli derivati dalla definizione del numero di Nepero}
\begin{enumerate}
	\item \(\displaystyle \lim_{x \to \pm \infty} \left(1 + \frac{\alpha}{x} \right) ^ x = e ^ \alpha \quad \forall \alpha \in \mathbb{R}\)
	\item \(\displaystyle \lim_{x \to 0} \left( 1 + \alpha x \right) ^ \frac{1}{x} = e ^ \alpha \quad \forall \alpha \in \mathbb{R}\)
	\item \(\displaystyle \lim_{x \to 0} \frac{\log \left( 1 + x \right)}{x} = 1\)
	\item \(\displaystyle \lim_{x \to 0} \frac{ e ^ x - 1 }{x} = 1\)
	\item \(\displaystyle \lim_{x \to 0} \frac{\sinh x}{x} = 1\)
\end{enumerate}

\subsection{Confronti asintotici}
Siano \(f\) e \(g\) due funzioni definite su \(X \subseteq R\) e \(x_0\) punto di accumulazione di \(X\). Le due funzioni sono asintotiche per \(x \to x_0\),
\(f \sim g\) per \(x \to x_0\) quando:
\begin{enumerate}
	\item sono entrambe \(\neq 0\) definitivamente per \(x \to x_0\)
	\item \(\displaystyle \lim_{x \to x_0} \frac{f(x)}{g(x)} = 1\)
\end{enumerate}

\subsubsection*{Proprietà}
\begin{itemize}
	\item[P1:] se \(f \sim g\) per \(x \to x_0\), allora \(\lim_{x \to x_0} f(x)\) e \(\lim_{x \to x_0} f(x)\) o non esistono o esistono e coincidono
	\item[P2:] se \(f \sim g\) per \(x \to x_0\) e \(g \sim h\) per \(x \to x_0\), allora \(f \sim h\) per \(x \to x_0\)
	\item[P3:] se \(f \sim f'\), \(g \sim g'\) e \(h \sim h'\) per \(x \to x_0\), allroa \(\displaystyle \frac{f \cdot g}{h} \sim \frac{f' \cdot g'}{h'}\) per \(x \to x_0\)
\end{itemize}

\subsubsection*{Esempi di funzioni asintotiche}
\begin{enumerate}
	\item \(\displaystyle \lim_{x \to 0} \frac{\sin x}{x} = 1 \quad \Rightarrow \quad \sin x \sim x \text{ per } x \to 0\)
	\item \(\displaystyle \lim_{x \to 0} \frac{\tan x}{x} = 1 \quad \Rightarrow \quad \tan x \sim x \text{ per } x \to 0\)
	\item \(\displaystyle \lim_{x \to 0} \frac{\log (1 + x)}{x} = 1 \quad \Rightarrow \quad \log (1 + x) \sim x \text{ per } x \to 0\)
	\item \(\displaystyle \lim_{x \to 0} \frac{e ^ x - 1}{x} = 1 \quad \Rightarrow \quad e ^ x - 1 \sim x \text{ per } x \to 0\)
	\item \(\displaystyle \lim_{x \to 0} \frac{1 - \cos x}{x^2} = \frac{1}{2} \quad \Rightarrow \quad 1 - \cos x \sim \frac{x^2}{2} \text{ per } x \to 0\)
\end{enumerate}

\newpage

\subsection{Simboli di Landau - \textit{o} piccoli}
Siano \(f\) e \(g\) due funzioni definite su \(X \subseteq R\) e \(x_0\) punto di accumulazione di \(X\), sia \(g\) definitivamente \(\neq 0\) per \(x \to x_0\),
\(f(x) = o(g(x))\), ovvero \(f\) è un \(o\) piccolo di \(g\), quando \(\displaystyle \lim_{x \to x_0} \frac{f(x)}{g(x)} = 0\). \\
Si osserva che se \(g(x) = 1 \; \forall x \in \mathbb{R}\) si ha che \(f(x) = o(1) \Leftrightarrow \displaystyle \lim_{x \to x_0} f(x) = 0\)

\subsubsection*{Proprietà degli \(o\) piccoli}
\begin{itemize}
	\item[P1:] \(o(g(x)) \pm o(g(x)) = o(g(x))\) per \(x \to x_0\)
	\item[P2:] \(o(g(x)) \cdot o(g(x)) = o(g^2(x))\) per \(x \to x_0\)
	\item[P3:] \(\varphi(x) \cdot o(g(x)) = o(\varphi (x) \cdot g(x))\) per \(x \to x_0\) e \(\varphi(x) \neq 0\) definitivamente
	\item[P4.1:] \(\varphi(x) \cdot o(g(x)) = o(g(x))\) per \(x \to x_0\) e \(\varphi(x)\) è definitivamente limitata
	\item[P4.2:] \(c \cdot o(g(x)) = o(g(x))\) per \(x \to x_0\)
	\item[P5:] \(\left| o(g(x)) \right| ^ \alpha = o(\left| g(x) \right| ^ \alpha)\) per \(x \to x_0\)
\end{itemize}

\subsubsection*{Legame tra asintoticità e \(o\) piccoli}
Siano \(f\) e \(g\) due funzioni definite su \(X \subseteq R\) e \(x_0\) punto di accumulazione di \(X\), \(f\), \(g\) definitivamente \(\neq 0\) per \(x \to x_0\), allora:
\begin{align*}
	lim_{x \to x_0} \frac{f(x)}{g(x)} = l \in \mathbb{R} \; \backslash 0 \quad &\Leftrightarrow \quad f(x) \sim l \cdot g(x) \text{ per } x \to x_0 \\
	&\Leftrightarrow \quad f(x) = l \cdot g(x) + o(g(x)) \text{ per } x \to x_0
\end{align*}

\subsubsection*{Teorema del cambio di variabili negli sviluppi}
\begin{itemize}
	\item[P:] Siano \(f\), \(\varphi\) e \(\varphi_1\) tre funzioni t.c. \(\varphi \circ f\) e \(\varphi_1 \circ f\) siano definite sullo stesso
	insieme \(X \subseteq \mathbb{R}\) e sia \(x_0\) punto di accumulazione di \(X\), se:
	\item[H:]
	\begin{enumerate}
		\item \(\varphi(y) = \varphi_1(y) + o(\varphi_1(y))\) per \(y \to y_0\)
		\item \(\displaystyle \lim_{x \to x_0} f(x) = y_0\)
		\item \(f(x) \neq y_0\) per \(x \to x_0\) o \(\varphi(y_0) = \varphi_1(y_0)\)
	\end{enumerate}
	\item[T:] \(\varphi(f(x)) = \varphi_1(f(x)) + o(\varphi_1(f(x)))\) per \(x \to x_0\)
\end{itemize}

\subsubsection*{Esempi di cambio di variabile negli sviluppi}
Sappiamo che \(\sin y = y + o(y)\) per \(y \to y_0 = 0\)
\begin{itemize}
	\item per \(y = f(x) = x^2\), si osserva che \(\displaystyle \lim_{x \to 0} f(x) = 0 = y_0\) si ottiene che \\ \(\sin(x^2) = x^2 + o(x^2)\) per \(x \to 0\)
	\item per \(y = f(x) = x^3 - 1\), si osserva che \(\displaystyle \lim_{x \to 1} f(x) = 0 = y_0\) si ottiene che \\ \(\sin(x^3 - 1) = x^3 - 1 + o(x^3 - 1)\) per \(x \to 1\)
\end{itemize}

\subsubsection*{Teorema di sostituzione degli infiniti e infinitesimi}
\begin{itemize}
	\item[P:] Siano \(f, f_1, g, g_1\) quattro funzioni definite in \(X \subseteq \mathbb{R}\) ed \(x_0\) un punto di accumulazione di \(X\),
	assumiamo che le funzioni siano tutte \(\neq 0\) definitivamente per \(x \to x_0\)
	\item[H:] se \(f(x) = f_1(x) + o(f_1(x))\) e \(g(x) = g_1(x) + o(g_1(x))\) per \(x \to x_0\)
	\item[T:] allora \(\displaystyle \lim_{x \to x_0} \frac{f(x)}{g(x)}\) e \(\displaystyle \lim_{x \to x_0} \frac{f_1(x)}{g_1(x)}\) hanno lo stesso comportamento,
	ovvero o non esistono entrambi, o esistono e sono coincidenti
\end{itemize}
Questo teorema si utilizza per risolvere i limiti in cui compaiono rapporti tra polinomi.

\subsection{Ordini di infinito e infinitesimo}
\subsubsection*{Ordini di infinito}
Siano \(f\) e \(g\) due funzioni definite su \(X \subseteq R\) e \(x_0\) punto di accumulazione di \(X\), siano \(\displaystyle \lim_{x \to x_0} f(x) = \infty\)
e \(\displaystyle \lim_{x \to x_0} g(x) = \infty\), consideriamo \(\displaystyle l := \lim_{x \to x_0} \frac{f(x)}{g(x)}\):
\begin{itemize}
	\item se il limite esiste e \(l \in \mathbb{R} \; \backslash 0\), allora \(f\) e \(g\) sono infiniti dello steso ordine per \(x \to x_0\)
	\item se il limite esiste e \(l = 0\), allora \(f\) è un infinito di ordine minore di \(g\) per \(x \to x_0\)
	\item se il limite esiste e \(l = \infty\), allora \(f\) è un infinito di ordine maggiore di \(g\) per \(x \to x_0\)
	\item se il limite non esiste, \(f\) e \(g\) sono infiniti non confrontabili per \(x \to x_0\)
\end{itemize}

\subsubsection*{Ordini di infinitesimo}
Siano \(f\) e \(g\) due funzioni definite su \(X \subseteq R\) e \(x_0\) punto di accumulazione di \(X\), siano \(\displaystyle \lim_{x \to x_0} f(x) = 0\)
e \(\displaystyle \lim_{x \to x_0} g(x) = 0\), con \(f(x)\) e \(g(x)\) \(\neq 0\) definitivamente, consideriamo \(\displaystyle l := \lim_{x \to x_0} \frac{f(x)}{g(x)}\):
\begin{itemize}
	\item se il limite esiste e \(l \in \mathbb{R} \; \backslash 0\), allora \(f\) e \(g\) sono infinitesimi dello steso ordine per \(x \to x_0\)
	\item se il limite esiste e \(l = 0\), allora \(f\) è un infinitesimo di ordine maggiore di \(g\) per \(x \to x_0\)
	\item se il limite esiste e \(l = \infty\), allora \(f\) è un infinitesimo di ordine minore di \(g\) per \(x \to x_0\)
	\item se il limite non esiste, \(f\) e \(g\) sono infinitesimi non confrontabili per \(x \to x_0\)
\end{itemize}

\subsubsection*{Definizione dell'ordine di infinito o infinitesimo}
Per definire l'ordine di infinito/infinitesimo di una funzione, la si deve confrontare con una classe si funzioni campione definita
come \(f(x) = \left| x - x_0 \right| ^ \alpha\), con \(\alpha \in \mathbb{R}\), se \(x_0 \in \mathbb{R}\)

Si osserva che per \(x \to x_0\), se \(\alpha > 0\), allora \(f(x) \to 0\), mentre se \(\alpha < 0\), allora \(f(x) \to \infty\)

\subsubsection*{Definizione dell'ordine di infinito}
Sia \(f\) una funzione definita su \(X \subseteq \mathbb{R}\) ed \(x_0\) punto di accumulazione di \(X\), sia \(\displaystyle \lim_{x \to x_0} f(x) = \infty\), allora:
\begin{itemize}
	\item se \(x_0 \in \mathbb{R}\) e \(f\) è dello stesso ordine di infinito di \(\left| x - x_0 \right| ^ {-\alpha}\), con \(\alpha > 0\),
	allora \(f\) è un infinito di ordine \(\alpha\)
	\item se \(x_0 = \pm \infty\) e \(f\) è dello stesso ordine di infinito di \(\left| x \right| ^ \alpha\), con \(\alpha > 0\),
	allora \(f\) è un infinito di ordine \(\alpha\)
\end{itemize}

\subsubsection*{Definizione dell'ordine di infinitesimo}
Sia \(f\) una funzione definita su \(X \subseteq \mathbb{R}\) ed \(x_0\) punto di accumulazione di \(X\), sia \(\displaystyle \lim_{x \to x_0} f(x) = 0\)
ed \(f \neq 0\) definitivamente per \(x \to x_0\), allora:
\begin{itemize}
	\item se \(x_0 \in \mathbb{R}\) e \(f\) è dello stesso ordine di infinitesimo di \(\left| x - x_0 \right| ^ \alpha\), con \(\alpha > 0\),
	allora \(f\) è un infinitesimo di ordine \(\alpha\)
	\item se \(x_0 = \pm \infty\) e \(f\) è dello stesso ordine di infinitesimo di \(\left| x \right| ^ {-\alpha}\), con \(\alpha > 0\),
	allora \(f\) è un infinitesimo di ordine \(\alpha\)
\end{itemize}

\newpage


\section{Succesioni}
\subsubsection*{Definizione}
Sia \(A \subseteq \mathbb{N}\) illimitato, una successione (a valori reali) è una funzione da \(A\) in \(\mathbb{R}\). \\
Una successione \(a : A \to \mathbb{R}\) si indica come \(a_n\) o \(\left\{ a_n \right\}_{n \in A}\).

\subsubsection*{Limiti di successioni}
Dato che il dominio delle successioni è costituito da punti isolati, l'unico punto di accumulazione, di cui ha senso 
calcolarne il limite è \(+\infty\). Quando si indica "definitivamente" riferito ad una successione, si intende "definitivamente
per \(n \to +\infty\).

\subsubsection*{Carattere di una successione}
Data \(\left\{ a_n \right\}_{n \in \mathbb{N}}\) successione a valori reali con dominio \(\mathbb{N}\) e \(\displaystyle \lim_{n \to +\infty} a_n = l \in \mathbb{R}^*\), se:
\begin{itemize}
	\item \(\exists \; l \in \mathbb{R}\), la succesione \(a_n\) è convergente
	\item \(\exists \; l = 0\), la succesione \(a_n\) è in particolare infinitesima
	\item \(\exists \; l = \pm \infty\), la succesione \(a_n\) è divergente
	\item \(\nexists \; l\), la succesione \(a_n\) è irregolare
\end{itemize}

\subsection{Sottosuccessioni}
\subsubsection*{Definizione}
Data \(\left\{ a_n \right\}_{n \in \mathbb{N}}\), una sottosuccesione (o successione estratta) di \(a_n\) è una successione della forma \(\left\{ a_{n_k} \right\}_{k \in \mathbb{N}}\)
con \(n_k : \mathbb{N} \to \mathbb{N}\) successione strettamente crescente.

\subsubsection*{Convergenza con sottosuccessioni}
\begin{itemize}
	\item[P:] Sia \(\left\{ a_n \right\}_{n \in \mathbb{N}}\) una successione
	\item[H/T:] \(\displaystyle \lim_{n \to \infty} a_n = l \in \mathbb{R}^* \quad \Leftrightarrow \quad \forall \left\{ a_{n_k} \right\}_{k \in \mathbb{N}}\) si ha \(\displaystyle \lim_{k \to \infty} a_{n_k} = l\)
\end{itemize}
Una successione ha come limite \(l \in \mathbb{R}^*\), se ogni sottosuccessione ha come limite \(l\). \\
Una successione è irregolare (non ha limite), se esistono due sottosuccessioni che hanno limite diverso.

\subsubsection*{Teorema Bolzano - Weierstrass}
Se una succesione \(\left\{ a_n \right\}_{n \in \mathbb{N}}\) è limitata, allora ha una sottosuccesione convergente.

\subsubsection*{Teorema ponte}
\begin{itemize}
	\item[P:] Sia \(f: X \to \mathbb{R}\) una funzione con \(X \subseteq \mathbb{R}\) e sia \(x_0\) punto di accumulazione di \(X\)
	\item[H/T:] \(\displaystyle \lim_{x \to x_0} f(x) = l \in \mathbb{R}^* \quad \Leftrightarrow \quad \forall \left\{ a_n \right\}_{n \in \mathbb{N}}\) con \(\displaystyle a_n \in X, \lim_{n \to +\infty} a_n \neq x_0 \;\) si ha \(\; \displaystyle \lim_{k \to \infty} f(a_n) = l\)
\end{itemize}
Il limite per \(x_0\) di una funzione vale \(l \in \mathbb{R}^*\) se e solo se per ogni successione, il limite della successione vale \(x_0\) e limite di \(f(a_n)\) è \(l\). \\
Se esistono due successioni \(a_n\) e \(b_n\), per cui il limite di \(f(a_n)\) e quello di \(f(b_n)\) sono diversi, allora non esiste il limite per \(f(x_0)\).

\subsection{Formula di Stirling per n!}
\(\forall \; n \in \mathbb{N}, n \geq 1 \quad \; \exists \; a_n \in \left( 0, 1 \right)\) t.c. \(\displaystyle n! = \frac{n^n}{e^n} \cdot \sqrt{2 \pi n} \cdot e ^ {\frac{a_n}{12n}}\) \\
per \(n \to \infty \quad \Leftrightarrow \quad n! \sim \frac{n^n}{e^n} \cdot \sqrt{2 \pi n}\)

\section{Serie}
\subsubsection*{Definizione}
Data la successione \(a_k \in \mathbb{R}\), con \(k \in \mathbb{N}\), la somma parziale n-esima dei primi \(n\) termini della successione,
definita come \(\displaystyle S_n = \sum_{k = 1}^{n} a_k = a_1 + a_2 + a_3 + \dots + a_n\).

La successione delle somme parziali \(\left\{ S_n \right\}_{n \in \mathbb{N}}\) è chiamata serie di termine generale \(a_k\).

\subsubsection*{Limiti di serie}
Analogo discorso per le successioni, essendo il dominio costituito da punti isolati, l'unico punto di accumulazione è \(+ \infty\)

\subsection{Studio della convergenza e divergenza}
\subsubsection*{Carattere di una serie}
Data una serie \(S_n\) di termine generale \(a_k\) e \(\displaystyle \lim_{n \to +\infty} S_n = l \in \mathbb{N}^*\), se:
\begin{itemize}
	\item \(\exists \; l \in \mathbb{R}\), la serie \(S_n\) è convergente
	\item \(\exists \; l = + \infty\), la serie \(S_n\) è divergente a \(+ \infty\)
	\item \(\exists \; l = - \infty\), la serie \(S_n\) è divergente a \(- \infty\)
	\item \(\nexists \; l\), la serie \(s_n\) è irregolare
\end{itemize}

\subsubsection*{Comportamento primi termini}
Il comportamento di una serie non dipende dai primi termini.

\subsubsection*{Condizione necessaria per convergenza}
Se una serie è convergente, allora l'ultimo elemento è 0, ovvero \(\displaystyle \lim_{k \to +\infty} a_k = 0\)

\subsubsection*{Condizione sufficiente per convergenza / convergenza assoluta}
Sia una serie \(\displaystyle S_n = \sum_{k = 0}^{n} a_k\), se \(\displaystyle S_n = \sum_{k = 0}^{n} \left| a_k \right|\) converge, allora la serie converge assolutamente. \\
Se una serie converge assolutamente, converge anche semplicemente.

\subsection{Serie a termini positivi (conv o diverge a \(+ \infty\))}
Una serie è a termini positivi se ogni termine è \(\geq 0\) \\ 
Una serie a termini postivi converge o doverge a \(+ \infty\)

\subsubsection*{Confronto}
Siano \(\displaystyle \sum_{k = 1}^{n} a_k\), \(\displaystyle \sum_{k = 1}^{n} b_k\) serie a termini positivi, se \(a_k < bk\) definitivamente, allora:
\begin{itemize}
	\item se \( \sum_{k = 1}^{n} b_k\) converge, converge anche \(\sum_{k = 1}^{n} a_k\)
	\item se \(\sum_{k = 1}^{n} a_k\) diverge, diverge anche \(\sum_{k = 1}^{n} b_k\)
\end{itemize}

\subsubsection*{Confronto asintotico}
Siano \(\displaystyle \sum_{k = 1}^{n} a_k\), \(\displaystyle \sum_{k = 1}^{n} b_k\) serie a termini positivi, e \(\displaystyle \lim_{k \to +\infty} \frac{a_k}{b_k} = l\), allora:
\begin{itemize}
	\item se \(l = \mathbb{R} \; \backslash 0\), sono asintotiche: se una converge, l'altra converge, se una diverge, l'altra diverge
	\item se \(l = 0\), allora \(a_k < b_k\) definitivamente, allora se \(\sum_{k = 1}^{n} b_k\) converge, converge anche \(\sum_{k = 1}^{n} a_k\)
	\item se \(l = +\infty\), allora \(a_k > b_k\), allora se \(\sum_{k = 1}^{n} b_k\) diverge, allora diverge anche \(\sum_{k = 1}^{n} a_k\)
\end{itemize}

\subsubsection*{Criterio condensazione}
Sia \(S_n\) serie a termini positivi e \(a_k\) successione decrescente, allora \(S_{a_k}\) e \(\displaystyle S_{2^k \cdot a_{2^k}}\) hanno lo stesso comportamento

\subsubsection*{Rapporto}
Sia \(S_n\) serie a termini positivi:
\begin{itemize}
	\item se \(\displaystyle \frac{a_{k + 1}}{a_k} < 1\) definitivamente, la serie converge
	\item se \(\displaystyle \frac{a_{k + 1}}{a_k} \geq 1\) definitivamente, la serie diverge a \(+ \infty\)
\end{itemize}

\subsubsection*{Rapporto asintotico}
Sia \(S_n\) serie a termini positivi e \(\displaystyle \lim_{k \to +\infty} \frac{a_{k+1}}{a_k} = l \in \left[0, +\infty \right) \cup +\infty \)
\begin{itemize}
	\item se \(l < 1\) la serie converge
	\item se \(l > 1\) la serie diverge
	\item se \(l = 1\) non si può dire nulla
\end{itemize}

\subsubsection*{Criterio radice}
Sia \(S_n\) serie a termini positivi:
\begin{itemize}
	\item se \(\sqrt[k]{a_k} < 1\) definitivamente, la serie converge
	\item se \(\sqrt[k]{a_k} \geq 1\) definitivamente, la serie diverge a \(+ \infty\)
\end{itemize}

\subsubsection*{Criterio radice asintotica}
Sia \(S_n\) serie a termini positivi e \(\displaystyle \lim_{k \to +\infty} \sqrt[k]{a_k} = l \in \left[0, +\infty \right) \cup +\infty \)
\begin{itemize}
	\item se \(l < 1\) la serie converge
	\item se \(l > 1\) la serie diverge
	\item se \(l = 1\) non si può dire nulla
\end{itemize}

\subsubsection*{Ponte tra criterio della radice e criterio del rapporto}
Se esiste il limite del rapporto, esiste limite della radice e coincidono, ma non il viceversa, quindi:
\begin{itemize}
	\item se il lim del rapporto = 1, allora il lim della radice = 1 e non si può concludere nulla
	\item se il lim della radice = 1, allora il lim del rapporto = 1 o non esiste e non si può concludere nulla
\end{itemize}

\newpage

\subsection{Esempi di serie convergenti e divergenti}
\subsubsection*{Serie geometrica}
\(\displaystyle S_n = \sum_{k = 0}^{n} r ^ k = \frac{1-r^{n+1}}{1-r}\), con ragione \(r > 0\)
\begin{itemize}
	\item se \(0 < r < 1 \quad \rightarrow \quad\) convergente a \(\frac{1}{1-r}\)
	\item se \(r \geq 1 \quad \rightarrow \quad\) divergente a \(+ \infty\)
\end{itemize}

\subsubsection*{Serie armonica}
\(\displaystyle S_n = \sum_{k = 1}^{n} \frac{1}{k^a}\)
\begin{itemize}
	\item per \(a > 1\) la serie converge
	\item per \(a \leq 1\) la serie diverge a \(+ \infty\)
\end{itemize}

\subsubsection*{Serie armonica generalizzata}
\(\displaystyle S_n = \sum_{k = 1}^{n} \frac{1}{k^a \cdot \log^b k}\)
\begin{itemize}
	\item per \(a > 1\) la serie converge
	\item per \(a < 1\) la serie diverge a \(+ \infty\)
	\item per \(a = 1\), \(b > 1\) la serie converge
	\item per \(a = 1\), \(b \leq 1\) diverge a \(+ \infty\)
\end{itemize}

\subsubsection*{Serie esponenziale}
\(\displaystyle S_n = \sum_{k = 1}^{n} \frac{x^k}{k!}\) con parametro \(x \in \mathbb{R}\)
\begin{itemize}
	\item per \(x > 0\) la serie ha termini positivi
	\item per \(x < 0\) la serie ha terminidi di segno alterni
	\item per \(x = 0\) si ha la forma \(0^0\), da definire
\end{itemize}
La serie è assolutamente convergente e uguale a \(e^x\)

\subsubsection*{Serie a segno alterno}
\(\displaystyle S_n = \sum_{k = 1}^{n} (-1)^k * a_k\) con \(a_k > 0\)
\begin{itemize}
	\item per \(k\) pari, il teminine ha segno positivo
	\item per \(k\) dispari, in termine ha negativo
\end{itemize}

\subsubsection*{Criterio di Leibniz}
Se \(\displaystyle \lim_{k \to +\infty} a_k = 0\) (cond. necess.) e \(a_k\) definitivamente decrescente, la serie a segno alterno è convergente.


\end{document}
