\documentclass[a4paper]{article}
\usepackage[utf8]{inputenc} % standard unicode
\usepackage[italian]{babel} % corretta sillabazione in italiano
\usepackage{geometry} % per impostare margini e layout pagina
\usepackage{amssymb} % per l'ambiente matematico
\usepackage{amsmath} % per l'ambiente matematico
\usepackage{enumitem} % per elenchi puntati
\usepackage{multirow} % per celle che si espandono su più righe
\usepackage{tabularx} % per tabelle con larghezza flessibile
\usepackage{booktabs} % per linee orizzontali tabelle
\usepackage{hyperref} % per collegamenti
\usepackage{graphicx} % per immagini
\usepackage{listings} % per codice
\usepackage{xcolor} % per colori nel codice
\usepackage{dirtytalk} % per le ""

% per margini
\geometry{a4paper,left=25mm, right=25mm, bottom=25mm, top=30mm}

% per centrare testo nelle tabelleX
\renewcommand\tabularxcolumn[1]{m{#1}}

% percorso delle immagini da inserire
\graphicspath{ {./ } }

% parte funzione reale e parte immaginaria
\newcommand\Real{\text{Re}}
\newcommand\Img{\text{Im}}

% versori
\newcommand\ux{\vec{u}_x}
\newcommand\uy{\vec{u}_y}
\newcommand\uz{\vec{u}_z}
\newcommand\ur{\vec{u}_r}
\newcommand\uv{\vec{u}_v}
\newcommand\ug{\vec{u}_\gamma}
\newcommand\uper{\vec{u}_\perp}
\newcommand\upar{\vec{u}_\parallel}

% derivata
\newcommand\dt{\frac{d}{dt}\,}
\newcommand\dtau{\frac{d}{d\tau}\,}
\newcommand\dts{\frac{d^2}{dt^2}\,}

% modulo vettore
\newcommand\vmod[1]{\left|\left|{#1}\right|\right|}

\title{Appunti di Fisica 1}
\author{Giacomo Simonetto}
\date{Secondo semetre 2023-24}

\begin{document}

% -------------------------------------- Copertina e indice ---------------------------------------
\maketitle
\begin{abstract}
	Appunti del corso di Fisica 1 - (Meccanica e termodinamica) della facoltà di Ingegneria Informatica dell'Università di Padova.
\end{abstract}

\newpage

\tableofcontents

\newpage

% ----------------------------------------- introduzione ------------------------------------------
\section{Introduzione}
\subsection{Interazioni fondamentali}
Le interazioni (o forze) fondamentali sono:
\begin{enumerate}[topsep=3pt, itemsep=0pt]
	\item forza gravitazionale: scoperta per prima nel 1600 circa da Galileo
	\item forza elettromagnetica: scoperta nel 1800
	\item forza debole: legata ai costituenti degli atomi (radioattività)
	\item forza forte: legata ai costituenti degli atomi (quark)
\end{enumerate}
Si sta cercando un legame tra la forza elettromagnetica e quella debole (forza elettrodebole) e una teoria che lega le forze
elettromagnetica, debole e forte (teoria delle forze unificate). La forza gravitazione è considerata particolare in quanto:
\begin{itemize}[topsep=3pt, itemsep=0pt]
	\item[-] è molto meno intensa delle altre
	\item[-] ha solo "carica" positiva (non esiste massa negativa)
	\item[-] spazio e forza gravitazionale non possono essere sepatati
	\item[-] non si è ancora riusciti a comprenderla quantisticamente
\end{itemize}

% ---------------------------------------- punto materiale ----------------------------------------
\section{Il punto materiale}
\subsection{Introduzione al punto materiale}
È una finzione matematica in quanto non esiste nella realtà, ma serve come approssimazione. Non ha estensione, ma ha una massa
\(m\) ed è possibile determinarne la posizione. In un sistema di riferimento (cartesiano con 3 assi), la posizione è data dal
vettore \(\vec{r_0} = \left(x_0, y_0, z_0\right)\).

\subsection{Grandezze elementari}

massa: \(m\), l'unità di misura è \(\left[m\right] = kg\)

posizione: \(\vec{r_0} = \left(x_0, y_0, z_0\right)\), con unità di misura \(\left[x_0\right] = \left[y_0\right] = \left[z_0\right] = m\)

tempo: \(t\), con unità di misura \(\left[t\right] = s\)

\newpage


\subsection{Punto materiale in movimento - cinematica}
\subsubsection*{Posizione}
La posizione nello spazio di un punto sono le coordinate del punto un sistema di riferimeno cartesiano di 3 assi.
\[\vec{r}(t) = \left(x_0(t), y_0(t), z_0(t)\right) \qquad \left[x_0\right] = \left[y_0\right] = \left[z_0\right] = m\]

\subsubsection*{Velocità}
La velocità è lo spazio percorso in un tempo piccolo. \[\vec{v}(t) = \dt \vec{r}(t) \qquad \left[v\right] = \frac{m}{s}\]
Per ottenere la posizione dalla velocità: \[\vec{r}(t) = \vec{r_0} + \int_{t0}^{t1} \vec{v}(\tau) \, d\tau\]

\subsubsection*{Accelerazione}
L'accelerazione è la variazione della velocità nel tempo. \[\vec{a}(t) = \dt \vec{v}(t) = \frac{d^2}{dx^2} \, \vec{r}(t) \qquad \left[a\right] = \frac{m}{s^2}\]
Per ottenere la velocità dall'accelerazione: \[\vec{v}(t) = \vec{v_0} + \int_{t0}^{t1} \vec{a}(\tau) \, d\tau\]
Per ottenere la posizione dall'accelerazione: \[\vec{r}(t) = \vec{r_0} + \int_{t0}^{t1} \left( \vec{v_0}(\tau) + \int_{t0}^{t1} \vec{a}(\tau) \, d\tau \right) \, d\tau \quad \left(= \vec{r_0} + \vec{v_0}(t-t_0) + \iint_{t_0}^{t_1} \vec{a}(\tau) \, d\tau^2 \right)\]

\subsubsection*{Moto uniformemente accelerato}
Moto con accelerazione costante, le leggi orarie sono:
\[ \begin{cases}
	\vec{r}(t) = \vec{r_0} + \vec{v_0}(t-t_0) + \frac{1}{2}\vec{a}(t-t_0)^2 \\
	\vec{v}(t) = \vec{v_0} + \vec{a}(t-t_0)
\end{cases} \]
Per convenzione si sceglie \(t_0 = 0\):
\[ \begin{cases}
	\vec{r}(t) = \vec{r_0} + \vec{v_0}t + \frac{1}{2}\vec{a}t^2 \\
	\vec{v}(t) = \vec{v_0} + \vec{a}t
\end{cases} \]

Si osserva che per trovare \(\vec{r}(t)\) a partire dall'accelerazione è necessario conoscere i due dati iniziali \(\vec{r_0}\)
(posizione) e \(\vec{v_0}\) (velocità), in quanto sono stati fatti due integrali nel calcolo.

Ogni vettore \(\vec{r}(t)\), \(\vec{v}(t)\) e \(\vec{a}(t)\) può essere scomposto nelle tre componenti \(x, y, z\) degli assi cartesiani
ottenendo tre equazioni del moto, una per ogni asse.

Esempi di applicazioni:
\begin{itemize}
	\item[-] caduta di un grave (da fermo e con moto orizzontale)
	\item[-] moto di due automobili sulla stessa retta
	\item[-] moto di un proiettile (con angolo inziale \(\theta\) rispetto al suolo)
\end{itemize}

\newpage

\subsection{Moto armonico semplice in una dimensione}
\begin{align*}
	x(t) &= A \cos (\omega t + \varphi_0) = A \cos (\omega (t - t_0)) \qquad \text{con } -\omega t_0 = \varphi_0 \\
	v(t) &= - A \omega \sin (\omega t + \varphi_0) \\
	a(t) &= -A \omega^2 \cos (\omega t + \varphi_0) = -\omega^2 x(t)
\end{align*}
\begin{itemize}[topsep=3pt, itemsep=0pt]
	\item[-] \(A\) ampiezza del moto, \(\left[A\right] = m\)
	\item[-] \(\omega\) velocità angolare, \(\left[\omega\right] = \frac{rad}{s}\)
	\item[-] \(\varphi_0\) sfasamento iniziale, \(\left[\varphi_0\right] = rad\)
	\item[-] si osserva che \(\left[A\right] = m\), \(\left[A\omega\right] = \frac{m}{s}\), \(\left[A\omega^2\right] = \frac{m}{s^2}\)
\end{itemize}

\subsection{Moto circolare uniforme sul piano xy}
\begin{align*}
	\vec{r}(t) &= \left(A \cos (\omega t + \varphi_0), \; A \sin (\omega t + \varphi_0), \; 0\right) \\
	\vec{v}(t) &= \left(-A \omega \sin (\omega t + \varphi_0), \; A \omega \cos (\omega t + \varphi_0), \; 0 \right) \\
	\vec{a}(t) &= \left(-A \omega^2 \cos (\omega t + \varphi_0), \; -A \omega^2 \sin (\omega t + \varphi_0), \; 0 \right) = -\omega^2 \vec{r}(t)
\end{align*}
\begin{itemize}[topsep=3pt, itemsep=0pt]
	\item[-] il vettore velocità è tangente alla circonferenza e perpendicolare al raggio
	\item[-] il vettore accelerazione è perpendicolare a \(\vec{v}\), opposto a \(\vec{r}\) e diretto verso il centro
	\item[-] l'accelerazione del moto è chiamata accelerazione centripeta
\end{itemize}

\subsection{Moto vario}
\begin{itemize}[topsep=3pt, itemsep=0pt]
	\item[-] la posizione è data da \(\vec{r}(t)\)
	\item[-] la velocità è data da \(\displaystyle \vec{v}(t) = \lim_{\Delta t \to 0} \frac{\vec{r}(t+\Delta t) - \vec{r}(t)}{\Delta t} = \dt \vec{r}(t)\)
	\item[-] l'accelerazione è data da \(\displaystyle \vec{a}(t) = \lim_{\Delta t \to 0} \frac{\vec{v}(t+\Delta t) - \vec{v}(t)}{\Delta t} = \dt \vec{v}(t) = \dts \vec{r}(t)\)
	\item[-] la velocità è tangente alla traiettoria, ma non è detto che sia perpendicolare al vettore \(\vec{r}\)
\end{itemize}

\newpage


\section{Funzioni goniometriche (ripasso e proprietà)}
\subsection{Sviluppi di Taylor}
\[\sin \theta = \theta - \frac{\theta^3}{3!} + o(\theta^5) \qquad \qquad \cos \theta = 1 - \frac{\theta^2}{2!} + o(\theta^4)\]
\begin{itemize}[topsep=3pt, itemsep=0pt]
	\item[-] le formule valgono solo se \(\theta\) è un numero puro (non posso ad esempio sommare \(m\) e \(m^2\)).
	\item[-] \(\left[\theta\right] = rad\), si misura in radianti (numero puro), un radiante è il rapporto tra la lunghezza dell'arco
	di circonferenza che sottende un angolo \(\theta\) e il raggio della circonferenza.
\end{itemize}

\subsection{Formule di Eulero}
\[\sin \theta = \frac{e^{i\theta}-e^{-i\theta}}{2i} \qquad \qquad \cos \theta = \frac{e^{i\theta}+e^{-i\theta}}{2}\]
\begin{itemize}[topsep=3pt, itemsep=0pt]
	\item[-] le formule valgono solo se \(\Img (\sin \theta) = \Img (\cos \theta) = 0\) per \(\theta \in \mathbb{Q}\): \\
	ricordando che \(\displaystyle \Real (z) = \frac{z + z^*}{2}\), \(\displaystyle \Img (z) = \frac{z - z^*}{2i}\), si ha: \\
	\(\displaystyle \Img (\cos \theta) = \frac{1}{2i} \cdot \left(\frac{e^{i\theta}+e^{-i\theta}}{2} - \frac{e^{-i\theta}+e^{i\theta}}{2}\right) = \frac{0}{2i} = 0\) \\
	\(\displaystyle \Img (\sin \theta) = \frac{1}{2i} \cdot \left(\frac{e^{i\theta}-e^{-i\theta}}{2i} - \frac{e^{-i\theta}-e^{i\theta}}{2i}\right) = \frac{0}{2i} = 0\)
	\item[-] si osserva che \(\left|\cos \theta\right| \leq 1\), \(\left|\sin \theta\right| \leq 1\): \\
	\(\displaystyle \left|\cos \theta\right| = \left|\frac{e^{i\theta)}+e^{-i\theta}}{2}\right| = \frac{1}{2}\left|e^{i\theta} + e^{-i\theta}\right| \leq \frac{1}{2}\left|1 \cdot e^{i\theta}\right| + \left|1 \cdot e^{-i\theta}\right|= \frac{1}{2} (1+1) = 1\) \\
	\(\displaystyle \left|\sin \theta\right| = \dots\)
\end{itemize}

\subsection{Derivate}
\[\frac{d}{d\theta} \sin \theta = \cos \theta \qquad \frac{d}{d\theta} \cos \theta = -\sin \theta\]
\[\frac{d^2}{d\theta^2} \sin \theta = -\sin \theta \qquad \frac{d^2}{d\theta^2} \cos \theta = -\cos \theta\]

\newpage


\section{Vettori e versori}
\subsection{Definizione}
Un vettore è un \say{segmento orientato}, cioè definito da 3 proprietà: lunghezza, direzione e verso. Un vettore si indica con
lettere minuscole come \(\vec{a}\), \(\vec{b}\), \dots Questa definizione ci permette di essere indipendenti dal sistema di 
coordinate di riferimento.

La lunghezza di un vettore è chiamata norma o modulo e si indica \(\vmod{\vec{a}}\)

\subsection{Prodotto per uno scalare}
Dati \(\vec{a}\) vettore e \(\lambda\) scalare (numero reale), allora \(\vec{b} = \lambda \vec{a}\) è un vettore tale che:
\begin{itemize}[topsep=3pt, itemsep=0pt]
	\item[-] se \(\lambda > 0\), \(\vec{b}\) ha stessa direzione e verso di \(\vec{a}\), con lunghezza \(\lambda\) volte quella di \(\vec{a}\)
	\item[-] se \(\lambda < 0\), \(\vec{b}\) ha stessa direzione e verso opposto di \(\vec{a}\), con lunghezza \(-\lambda\) volte quella di \(\vec{a}\)
	\item[-] se \(\lambda = 0\), \(\vec{b}\) è vettore nullo \(\vec{0}\)
\end{itemize}

\subsection{Somma di vettori}
Dati due vettori \(\vec{a}\), \(\vec{b}\) la loro somma è un vettore \(\vec{c} = \vec{a} + \vec{b}\) definita dalla regola del
parallelogramma. La somma ha le seguenti proprietà:
\begin{itemize}[topsep=3pt, itemsep=0pt]
	\item[-] \(\vec{a} + \vec{b} = \vec{b} + \vec{a}\)
	\item[-] \((\vec{a} + \vec{b}) + \vec{c} = \vec{a} + (\vec{b} + \vec{c})\)
	\item[-] \(\lambda (\vec{a} + \vec{b}) = \lambda \vec{a} + \lambda \vec{b}\)
	\item[-] \(\vec{a} (\lambda_1 + \lambda_2) = \vec{a} \lambda_1 + \vec{a} \lambda_2\)
	\item[-] \(\vec{a} - \vec{b} = \vec{a} + (-1)\vec{b}\)
	\item[-] \(\vec{a} - \vec{a} = \vec{a}(1-1) = \vec{0}\)
\end{itemize}	


\subsection{Versori}
\begin{itemize}[topsep=3pt, itemsep=0pt]
	\item[-] i versori sono vettori unitari (con lunghezza 1).
	\item[-] sono definiti come \(\vec{u_a} = \frac{1}{\vmod{\vec{a}}} \cdot \vec{a}\).
	\item[-] una terna di assi è definita da 3 versori \(\ux\), \(\uy\), \(\uz\).
	\item[-] dato un vettore \(\vec{a} = \left(a_x, a_y, a_z\right)\) si può esprimere come \(\vec{a} = a_x \ux + a_y \uy + a_z \uz\)
\end{itemize}

\subsection{Prodotto scalare}
Dati due vettori \(\vec{a}\) e \(\vec{b}\) che formano un angolo \(\theta\) misurato in senso antiorario, il prodotto scalare tra
due vettori è uno scalare definito come \(\vec{a} \cdot \vec{b} = \vmod{\vec{a}} \vmod{\vec{b}} \cos \theta\)
con le seguenti proprietà:
\begin{itemize}[topsep=3pt, itemsep=0pt]
	\item[-] \(\vec{a} \cdot \vec{b} = \vec{a} \cdot \vec{b}\)
	\item[-] \((\lambda \vec{a}) \cdot \vec{b} = \lambda (\vec{a} \cdot \vec{b})\)
	\item[-] \((\vec{a} + \vec{b}) \cdot \vec{c} = \vec{a} \cdot \vec{c} + \vec{b} \cdot \vec{c}\)
\end{itemize}

\subsubsection*{Prodotto scalare tra versori}
\begin{itemize}[topsep=3pt, itemsep=0pt]
	\item[-] \(\ux \cdot \ux = \uy \cdot \uy = \uz \cdot \uz = 1\)
	\item[-] \(\ux \cdot \uy = \uy \cdot \uz = \uz \cdot \ux = 0\)
\end{itemize}

\subsubsection*{Prodotto scalare per componenti}
\begin{itemize}[topsep=3pt, itemsep=0pt]
	\item[-] \(\vec{a} \cdot \vec{b} = \left(a_x \ux + a_y \uy + a_z \uz\right) \cdot \left(b_x \ux + b_y \uy + b_z \uz\right) = a_x b_x + a_y b_y + a_z b_z\)
	\item[-] \(\vec{a} \cdot \vec{a} = \vmod{\vec{a}}^2 = a_x^2 + a_y^2 + a_z^2 \quad \Rightarrow \quad \vmod{\vec{a}} = \sqrt{a_x^2 + a_y^2 + a_z^2}\)
\end{itemize}

\subsection{Prodotto vettore}
Dati due vettori \(\vec{a}\) e \(\vec{b}\) con angolo \(\theta\) misurato in senso antiorario, il prodotto vettore tra due vettori
è un vettore definito come \(\vec{c} = \vec{a} \times \vec{b} = \vmod{\vec{a}} \vmod{\vec{b}} \sin \theta \; \vec{u_c}\)
con \(\vec{u_c}\) versore perpendicolare al piano di \(\vec{a}\) e \(\vec{b}\).
\begin{itemize}[topsep=3pt, itemsep=0pt]
	\item[-] se due vettori sono paralleli, \(\vec{a} \times \vec{b} = \vec{0}\)
	\item[-] l'orientamento di \(\vec{u_c}\) è una scelta convenzionale secondo la \say{regola della mano destra}
	\item[-] \(\vec{a} \times \vec{b} \neq \vec{b} \times \vec{a} \quad \Rightarrow \quad \vec{a} \times \vec{b} = - \vec{b} \times \vec{a}\)
	\item[-] \((\lambda \vec{a}) \times \vec{b} = \lambda (\vec{a} \times \vec{b})\)
	\item[-] \((\vec{a} + \vec{b}) \times \vec{c} = \vec{a} \times \vec{c} + \vec{b} \times \vec{c}\)
	\item[-] \(\vec{c} \times (\vec{a} + \vec{b}) = \vec{c} \times \vec{a} + \vec{c} \times \vec{b}\)
	\item[-] \((\vec{a} \times \vec{b}) \times \vec{c} \neq \vec{a} \times (\vec{b} \times \vec{c})\)
\end{itemize}

\subsubsection*{Prodotto vettore tra versori}
\begin{itemize}[topsep=3pt, itemsep=0pt]
	\item[-] \(\ux \times \ux = \uy \times \uy = \uz \times \uz = 0\)
	\item[-] \(\ux \times \uy = \uz\)
	\item[-] \(\uy \times \uz = \ux\)
	\item[-] \(\uz \times \ux = \uy\)
\end{itemize}

\subsection{Derivata di vettore}
\begin{itemize}
	\item[-] Dato un vettore \(\vec{a}(t) = (a_x(t) \; \ux + a_y(t) \; \uy + a_z(t) \; \uz)\), la sua derivata è definita come \\
	\(\displaystyle \dt \vec{a}(t) = \left(\dt a_x(t) \; \ux + \dt a_y(t) \; \uy + \dt a_z(t) \; \uz\right)\)
	\item[-] Dato un versore \(\vec{u}(t)\), la sua derivata è definita come \(\displaystyle \dt \vec{u}(t) = \frac{d \, \theta(t)}{dt} \, \uper(t)\)
	con \(\uper(t)\) versore perpendicolare a \(\vec{u}(t)\) e con \(\theta\) angolo spazzato da \(\vec{u}(t)\) in \(\Delta t\) piccolo.
\end{itemize}
Sia \(\vec{r}(t)\) vettore qualsiasi, la sua derivata è definita come
\[\dt \vec{r}(t) = \frac{d \vmod{\vec{r}(t)}}{dt} \; \vec{u_r}(t) + \vmod{\vec{r}(t)} \frac{d \, \theta(t)}{dt} \; \uper(t)\]

\newpage


\section{Dinamica del punto materiale}
\subsection{Prima legge della dinamica - principio di inerzia}
Se su un corpo non agisce alcuna forza, allora questo si muove a velocità costante \(\vec{v}\).

\subsection{Seconda legge della dinamica - legge di Newton}
Definita \(\vec{p} = m \vec{v}\) la quantità di moto, ovvero la variazione della quantità di moto rispetto al tempo è pari alla
risultante di tutte le forze che agiscono sul corpo:
\[\vec{F}(t) = \dt \vec{p}(t) \quad \Rightarrow \quad \vec{F}(t) = m \, \vec{a}(t)\]

\begin{itemize}[topsep=3pt, itemsep=0pt]
	\item[-] la quantità di moto ha unità di misura \(\displaystyle \left[\vec{p}\right] = kg \cdot \frac{m}{s}\)
	\item[-] la forza ha come unità di misura \(\displaystyle \left[\vec{F}\right] = kg \cdot \frac{m}{s^2} = N\)
	\item[-] se \(\vec{F} = \vec{0}\) allora \(m \, \vec{a} = 0 \; \Rightarrow \; \vec{a} = 0 \; \Rightarrow \; \vec{v}\) costante
	\item[-] la massa rappresenta un ostacolo al moto (più precisamente alla variazione della velocità), per questo viene chiamata massa inerziale.
	\item[-] la forza si ricava sperimentalmente e in generale ha la forma \(\vec{F}(t) = \vec{F}(\vec{r}(t), \vec{v}(t), t)\), ovvero può dipendere
	dalla posizione, dalla velocità o dal tempo
\end{itemize}

\subsection{Terza legge della dinamica - principio di azione-reazione}
Se un corpo \(A\) esercita una forza \(\vec{F}_{AB}\) su un corpo \(B\), allora \(B\) esercita una forza \(\vec{F}_{BA} = -\vec{F}_{AB}\) su \(A\)

\subsection{Sistemi di riferimento}
Dati due sistemi di riferimento \(O\) e \(O'\), le equazioni del moto di un punto \(P\) rispetto a \(O'\) sono:
\begin{itemize}[topsep=3pt, itemsep=0pt]
	\item[-] \(\vec{r'}(t) = \vec{r}(t) + \vec{r}_{OO'}(t)\)
	\item[-] \(\vec{v'}(t) = \vec{v}(t) + \dt \vec{r}_{OO'}(t)\)
	\item[-] \(\vec{a'}(t) = \vec{a}(t) + \dts \vec{r}_{OO'}(t)\)
\end{itemize}

\subsubsection*{Sistemi di riferimento inerziali}
Due sistemi di riferimento inerziali sono tali se si muovono con velocità costante (con accelerazione nulla) uno rispetto all'altro.
Dati due sistemi di riferimento inerziali \(O\) e \(O'\) tali per cui in \(O\) vale \(\vec{F} = m \; \vec{a}\), in \(O'\) vale \(\vec{F'} = m \; \vec{a'}\)
allora vale anche \(\vec{F} = \vec{F'}\) in quanto \(\vec{a} = \vec{a'}\), cioè \(\displaystyle \dts \vec{r}_{OO'}(t) = 0\).

\newpage


\subsection{Forza gravitazionale universale}
La forza gravitazionale tra un corpo \(A\) di massa \(m_A\) e un corpo \(B\) di massa \(m_B\) vale:
\[\vec{F}_{AB} = - G_N \frac{m_A \cdot m_B}{\vmod{\vec{r}_{AB}}^2} \, \vec{u}_{AB}\]

\begin{itemize}[topsep=3pt, itemsep=0pt]
	\item[-] il segno \(-\) indica che è attrattiva
	\item[-] \(\vmod{\vec{r}_{AB}}\) è la distanza tra i due corpi
	\item[-] \(\displaystyle G_N \approx 6.67 \cdot 10^{-11} \; \frac{N \cdot m^2}{kg^2}\) è la costante di gravitazione universale di Newton
\end{itemize}

\subsubsection*{Massa gravitazionale}
Le masse nella formula vengono dette masse gravitazionali in quanto sono considerate respondabili del moto di attrazione gravitazionale.
Nel teorema della relatività generale verrà dimostrato che la massa inerziale e gravitazionale corrispondono.

\subsubsection*{Accelerazione gravitazionale e forza peso}
Per il corpo \(A\) vale \(\displaystyle \vmod{\vec{a}_A} = -G_N \frac{m_B}{\vmod{\vec{r}_{AB}}^2}\)
%\begin{align*}
%	m_A \, \vec{a}_A &= -G_N \frac{m_A \cdot m_B}{\vmod{\vec{r}_{AB}}^2} \, \vec{u}_{AB} \\
%	m_A \, \vmod{\vec{a}_A} &= G_N \frac{m_A \cdot m_B}{\vmod{\vec{r}_{AB}}^2} \, \vmod{-\vec{u}_{AB}} \\
%	\vmod{\vec{a}_A} &= -G_N \frac{m_B}{\vmod{\vec{r}_{AB}}^2}
%\end{align*}

\begin{itemize}[topsep=3pt, itemsep=0pt]
	\item[-] l'accelerazione gravitazionale dipende solo dalla massa del corpo che attrae
	\item[-] sulla superficie terrestre vale \(\displaystyle g \approx 9.8 \; \frac{m}{s^2}\)
	\item[-] la forza di attrazione che esercita la Terra su un oggetto di massa \(m\) è detta forza peso \(\vec{F}_p = m \, g \, \uz\).
\end{itemize}

\subsection{Forza elastica}
La forza elastica esercita da una molla su un corpo di massa \(m\) è definita come:
\[\vec{F}_{el} = -k \, \Delta x \, \ux\]
\begin{itemize}[topsep=3pt, itemsep=0pt]
	\item[-] definendo \(x=0\) il punto in cui la molla è a riposo per cui \(\vec{F}_{el} = 0\), si ha \(\vec{F}_{el} = -k \, x \, \ux\)
	\item[-] è una forza di richiamo: tende a riportare la molla alla posizione di riposo, infatti agisce in verso opposto al vettore
	posizione per la presenza del segno \(-\)
	\item[-] è una forza universale: non fa parte delle forze elementari, ma è la forma di tutte le forze di richiamo (localmente a \(x = 0\))
\end{itemize}

\subsection{Forza di reazione vincolare}
La forza di reazione vincolare è esercitata da una superfice su un corpo appoggiato ad essa. È perpendicolare alla superficie e
opposta alla somma delle forze entranti, definita come:
\[\vec{F}_\perp = \vmod{\vec{F}_{TOT,\perp}} \uper = -\vec{F}_{TOT,\perp}\]

\subsection{Forza di attrito radente}
Forza di attrito esercitata dallo strisciamento di un corpo sulla superficie definita come:
\[\vec{F}_\text{attrito radente} = \begin{cases}
	-\vec{F}_{\parallel} &\text{se} \; \vmod{\vec{F}_\parallel} < \mu_s \vmod{\vec{F}_\perp}\\
	-\mu_d \, \vmod{\vec{F}_\perp} \, \uv &\text{altrimenti}
\end{cases}\]

\begin{itemize}[topsep=3pt, itemsep=0pt]
	\item[-] \(\mu_s\) numero puro, coefficiente di attrito statico (per \(\vec{v}=0\))
	\item[-] \(\mu_d\) numero puro, coefficiente di attrito dinamico
	\item[-] \(\uv\) versore con direzione e verso di \(\vec{v}\)
	\item[-] generalmente \(\mu_d < \mu_s\)
\end{itemize}

\subsection{Forza di attrito viscoso}
Forza di attrito esercitata da un fluido su un corpo in movimento, definita come:
\[\vec{F}_\text{attrito viscoso} = - b \, \vec{v}\]
\begin{itemize}[topsep=3pt, itemsep=0pt]
	\item[-] \(b\) coefficiente di attrito viscoso, \(\left[b\right] = \frac{kg}{s}\)
	\item[-] direzione opposta alla velocità e modulo proporzionale ad essa
	\item[-] in presenza di solo attrito viscoso lungo \(\ux\) le equazioni del moto diventano:
	\[v_x(t) = v_0 \, e^{-\frac{b}{m}t} \qquad x(t) = x_0 + \frac{m \, v_0}{b} \, \left(1-e^{-\frac{b}{m}t}\right)\]
	il grafico della velocità è un'esponenziale con esponente decrescente, quello della posizione è un'esponenziale capovolta con asintoto orizzontale \(x_{\infty} = x_0 + \frac{m \, v_0}{b}\)
\end{itemize}

\newpage


\section{Trovare le equazioni del moto}
\subsection{Trovare le soluzioni del moto per una forza generica in 1D}
Data una generica forza \(\vec{F}_x\) lungo \(\ux\), l'equazione del moto dalla la seconda legge della dinamica è:
\[m \, a_x(t) = F_x(x(t), v(t), t) = -k \, x(t) - b \, v_x(t) + f(t) \quad \Rightarrow \quad a_x(t) = -\frac{k \, x(t)}{m} - \frac{b \, v_x(t)}{m} + \frac{f(t)}{m}\]
\begin{itemize}[topsep=3pt, itemsep=0pt]
	\item[-] \(-k \, x(t)\) è la componente dipendente dalla posizione con \(k\) costante elastica
	\item[-] \(-b \, v_x(t)\) è la componente dipendente dalla velocità con \(b\) costante
	\item[-] \(f(t)\) è la componente dipendente dal tempo, può essere costante (es. \(mg\)) o periodica (es. risonanza)
\end{itemize}

Dall'equazione sopra si deriva l'equazione differenzale \eqref{diff} e la sua omogenea associata \eqref{omog}:
\begin{equation}
	\label{diff}
	\left(\dts + \frac{b}{m} \dt + \frac{k}{m}\right) x(t) = \frac{f(t)}{m}
\end{equation}
\begin{equation}
	\label{omog}
	\left(\dts + \frac{b}{m} \dt + \frac{k}{m}\right) x(t) = 0
\end{equation}

Le soluzioni di \eqref{diff} si scrivono nella forma:
\[X(t) = \lambda_1 x_1(t) + \lambda_2 x_2(t) + x_s(t)\]
\begin{itemize}[topsep=3pt, itemsep=0pt]
	\item[-] \(x_1(t)\) e \(x_2(t)\) sono soluzioni generiche della omogenea associata
	\item[-] \(x_s(t)\) è una soluzione particolare della differenzale
	\item[-] \(\lambda_1\) e \(\lambda_2\) sono parametri liberi da determinare in funzione delle condizioni iniziali
\end{itemize}

\subsection{Piano inclinato}
\subsubsection*{Piano inclinato liscio}
\begin{itemize}[topsep=3pt, itemsep=0pt]
	\item[-] le forze presenti sono \(\vec{F}_P = -m \, g \, \uz\) forza peso e \(\vec{F}_R = \vmod{\vec{F}_\perp} \uper\) forza di reazione vincolare
	\item[-] si scompongono le forze lungo la direzione \(\parallel\) e \(\perp\) al piano e si applica la seconda legge della dinamica
	\item[-] lungo \(\upar\) si ha \(a_\parallel = 0\), lungo \(\uper\) si ha \(a_\perp = g \, \sin \theta\)
	\item[-] il moto è uniformemente accelerato con accelerazione minore di \(g\)
	\item[-] per trovare le equazioni del moto in funzione dell'altezza si utilizza il teorema di conservazione dell'energia meccanica:
	\(v(h_1) = \sqrt{{v_0}^2 + 2g(h_1 - h_0)})\)
\end{itemize}

\subsubsection*{Piano inclinato scabro}
\begin{itemize}[topsep=3pt, itemsep=0pt]
	\item[-] le forze presenti sono \(\vec{F}_P = -m \, g \, \uz\), \(\vec{F}_R = \vmod{\vec{F}_\perp} \uper\), \(\vec{F}_A\) forza di attrito
	\item[-] si scompongono le forze lungo la direzione \(\parallel\) e \(\perp\) al piano e si applica la seconda legge della dinamica
	\item[-] per \(\mu_s < \tan \theta\) il corpo non si muove
	\item[-] lungo \(\upar\) si ha \(a_\parallel = 0\), lungo \(\uper\) si ha \(a_\perp = g \, \sin \theta \, (1-\mu_d \cot \theta)\)
	\item[-] il moto è uniformemente accelerato con accelerazione minore di \(g\)
	\item[-] per trovare le equazioni del moto in funzione dell'altezza si utilizza il teorema di conservazione dell'energia meccanica:
	\(v(h_1) = \begin{cases}
		\sqrt{{v_0}^2 + 2g(h_1 - h_0)(1-\mu_d \cot \theta)}) &\text{per la discesa} \\
		\sqrt{{v_0}^2 + 2g(h_1 - h_0)(1+\mu_d \cot \theta)}) &\text{per la salita}
	\end{cases}\)
\end{itemize}

\newpage

\subsection{Oscillatore armonico semplice}
Moto di un corpo attaccatto all'estremità di una molla con costante elastica \(k\) che si muove lungo una direzione (\(\ux\)),
con \(x=0\) posizione di riposo della molla. La forza che agisce sul corpo è la forza elastica e dipende solo dalla posizione:
\[m \, a_x = -k \, x(t) \quad \Rightarrow \quad \dts x(t) + \frac{k}{m} x(t) = 0\]
L'equazione del moto che soddisfa la differenziale sopra corrisponde a quella del moto armonico:
\[x(t) = A \cos (\omega t + \varphi_0) \qquad v(t) = -A \omega \sin (\omega t + \varphi_0) \qquad a(t) = -A \omega^2 \cos (\omega t + \varphi_0)\]
\begin{itemize}[topsep=3pt, itemsep=0pt]
	\item[-] \(A\) ampiezza del moto, \(\left[A\right] = m\)
	\item[-] \(\omega\) velocità angolare, \(\left[\omega\right] = \frac{rad}{s}\), \(\omega^2 = \frac{k}{m}\)
	\item[-] \(\varphi_0\) sfasamento iniziale, \(\left[\varphi_0\right] = rad\)
\end{itemize}

L'equazione del moto dipende dai due parametri \(A\) e \(\varphi_0\) da definire in base alle condizioni iniziali, ad esempio
si possono esprimere in funzione delle condizioni iniziali \(x_0\) e \(v_0\):
\[x_0 = A \cos \varphi_0 \qquad v_0 = -A \omega \sin \varphi_0 \qquad \tan \varphi_0 = -\frac{v_0}{\omega \, x_0} \qquad A = \sqrt{{x_0}^2 + \left(\frac{v_0}{\omega}\right)^2}\]

L'equazione si può riscrivere come composizione di funzioni goniometriche esplicitando i due parametri dipendenti dalle condizioni iniziali:
\[x(t) = A_1 \cos (\omega t) + A_2 \sin (\omega t) \quad v(t) = -A_1 \omega \sin (\omega t) + A_2 \omega \cos (\omega t) \quad a(t) = -A_1 \omega^2 \cos (\omega t) - A_2 \omega^2 \sin (\omega t)\]
\begin{itemize}[topsep=3pt, itemsep=0pt]
	\item[-] \(A_1\) prima ampiezza del moto, \(\left[A_1\right] = m\), con \(A_1 = A \cos \varphi_0\)
	\item[-] \(A_2\) seconda ampiezza del moto, \(\left[A_2\right] = m\), con \(A_2 = -A \sin \varphi_0\)
\end{itemize}

\subsection{Oscillatore armonico con l'azione della forza peso}
Stessa situazione precedente, con moto lungo \(\uz\), considerando anche la forza peso del corpo. Le forze che agiscono sono
quella elastica e la forza peso e dipendono ancora soltanto dalla posizione:
\[m \, a_z = -k \, z(t) - m \, g \quad \Rightarrow \quad \dts z(t) + \frac{k}{m} z(t) = -g\]
L'equazione del moto che soddisfa la differenziale sopra corrisponde a quella del moto armonico al netto di una costante
dovuta alla presenza della forza peso:
\[z(t) = A \cos (\omega t + \varphi_0) + \Delta l \qquad \Rightarrow \qquad \tilde{z}(t) = z(t) - \Delta l = A \cos (\omega t + \varphi_0)\]
\begin{itemize}[topsep=3pt, itemsep=0pt]
	\item[-] \(A\) ampiezza del moto, \(\left[A\right] = m\)
	\item[-] \(\omega\) velocità angolare, \(\left[\omega\right] = \frac{rad}{s}\)
	\item[-] \(\varphi_0\) sfasamento iniziale, \(\left[\varphi_0\right] = rad\)
	\item[-] \(\displaystyle \Delta l = - \frac{m \, g}{k}\) spostamento della posizione di equilibrio dovuta alla forza peso
	\item[-] \(\tilde{z}(t) = z(t) - \Delta l\) equazione della posizione introdotta traslando il punto di riposo della molla di \(\Delta l\) affiché \(\tilde{z}(t) = 0\) per \(\displaystyle z(t) = -\frac{m \, g}{k}\)
\end{itemize}

\newpage

\subsection{Oscillatore armonico smorzato (con forza peso e attrito)}
Analoga situazione precedente, con l'introduzione della forza di attrito viscoso con coefficiente di attrito \(b\). L'equazione
delle forze in gioco è:
\begin{multline*}
	m \, a_z = -k \, z(t) - m \, g - b \, v(t) \quad \Rightarrow \quad m \, a_z = -k \, \tilde{z}(t) - b \, v(t) \quad \Rightarrow \quad \dts \tilde{z}(t) + \frac{b}{m} \dt \tilde{z}(t) + \frac{k}{m} \tilde{z}(t) = 0 \\
	\Rightarrow \quad \dts \tilde{z}(t) + 2 \gamma \dt \tilde{z}(t) + \omega^2 \tilde{z}(t) = 0
\end{multline*}
\begin{itemize}[topsep=3pt, itemsep=0pt]
	\item[-] \(\displaystyle 2 \gamma = \frac{b}{m} > 0\) fattore di smorzamento (approfondito dopo)
	\item[-] \(\displaystyle \omega^2 = \frac{k}{m} > 0\) velocità oscillazione (approfondito dopo)
\end{itemize}

L'equazione che soddisfa la soluzione è del tipo \(z_0 \, e^{\lambda t}\) con \(\lambda_{1,2} = -\gamma \pm \sqrt{\gamma^2-\omega^2}\) ed
in base alla condizione di esistenza della radice si considerano i casi \(\gamma > \omega\) (1) e \(\gamma < \omega\) (2), :
\begin{enumerate}
	\item per \(\gamma > \omega\) la radice esiste nei reali e si ha:
	\[\tilde{z}(t) = z_1 \, e^{\left(-\gamma + \sqrt{\gamma^2 - \omega^2}\right) \, t} + z_2 \,  e^{\left(-\gamma - \sqrt{\gamma^2 - \omega^2}\right) \, t}\]
	\begin{itemize}[topsep=3pt, itemsep=0pt]
		\item[-] \(z_1, z_2\) parametri da determinare in base alla situazione iniziale, per \(t_0 = 0 \quad \Rightarrow \quad \tilde{z}(0) = z_1 + z_2\)
		\item[-] la componente \(z_2 \, e^{\lambda_2 t}\) va a 0 più velocemente rispetto a \(z_1 \, e^{\lambda_1 t}\), per cui
		il comportamento prevalente dipende dal secondo contributo
		\item[-] il grafico spazio-tempo è un'esponenziale con esponenti negativi crescenti, può essere simmetrica rispetto
		all'asse orizzontale in base al segno di \(\tilde{z}(0)\)
	\end{itemize}
	\item per \(\gamma < \omega\) la radice ha soluzioni complesse coniugate \(\lambda_{1,2} = -\gamma \pm i \, \sqrt{\left|\gamma^2 - \omega^2\right|}\):
	\setcounter{equation}{0}
	\begin{equation}
		\label{eq1}
		\tilde{z}(t) = z_1 \, e^{-\gamma t} \, e^{i \, \sqrt{\left|\gamma^2 - \omega^2\right|} \, t} + z_2 \, e^{-\gamma t} \, e^{-i \, \sqrt{\left|\gamma^2 - \omega^2\right|} \, t} =
		e^{-\gamma t} \left(z_1 \, e^{i \, \sqrt{\left|\gamma^2 + \omega^2\right|}} + z_2 \, e^{-i \, \sqrt{\left|\gamma^2 + \omega^2\right|}}\right)
	\end{equation}
	\begin{equation}
		\label{eq2}
		\begin{cases}
			\tilde{z}(t) = A \, e^{-\gamma t} \left( \frac{e^{i \, \sqrt{\left|\gamma^2 - \omega^2\right|} \, t} + e^{-i \, \sqrt{\left|\gamma^2 - \omega^2\right|} \, t}}{2}\right) = A \, e^{-\gamma t} \cos \left(\sqrt{\left|\gamma^2 + \omega^2\right|} \, t\right) &\text{per} \; z_1 = z_2 = \frac{A}{2} \\
			\tilde{z}(t) = B \, e^{-\gamma t} \left( \frac{e^{i \, \sqrt{\left|\gamma^2 - \omega^2\right|} \, t} - e^{-i \, \sqrt{\left|\gamma^2 - \omega^2\right|} \, t}}{2i}\right) = B \, e^{-\gamma t} \sin \left(\sqrt{\left|\gamma^2 + \omega^2\right|} \, t\right) &\text{per} \; z_1 = - z_2 = \frac{B}{2}
		\end{cases}
	\end{equation}
	\begin{align}
		\label{eq3}	\tilde{z}(t) &= e^{-\gamma t} \left( A \cos \left(\sqrt{\left|\gamma^2 - \omega^2\right|} \, t\right) + B \sin \left( \sqrt{\left|\gamma^2 - \omega^2\right|} \, t \right) \right) &\text{per} \; z_1 = \frac{A}{2} + \frac{B}{2i}, \; \; z_2 = \frac{A}{2} - \frac{B}{2i} \\
		\label{eq4}	\tilde{z}(t) &= e^{-\gamma t} \, C \cos \left(\sqrt{\left|\gamma^2 - \omega^2\right|} \, t\right) &\text{per} \; A = C \cos \varphi_0, \; \; B = C \sin \varphi_0
	\end{align}
	\begin{itemize}[topsep=3pt, itemsep=0pt]
		\item[-] eq. \eqref{eq1} si ottiene sostituendo le soluzioni \(\gamma_{1,2}\) per i parametri \(z_{1,2}\)
		\item[-] eq. \eqref{eq2} si ottiene in funzione dei parametri \(A\) e \(B\): gli esponenziali complessi si riconducono alle formule di Eulero per seni e coseni
		\item[-] eq. \eqref{eq3} si uniscono le equazioni precedenti con \(A\) e \(B\) al posto di \(z_{1,2}\)
		\item[-] eq. \eqref{eq4} si utilizza \(C\) e \(\varphi_0\) al posto di \(A\) e \(B\)
		\item[-] si osserva che il moto è armonico, esponenzialmente smorzato, in quanto l'ampiezza di riduce esponenzialmente
		\item[-] il periodo di oscillazione vale \(\displaystyle T = \frac{2 \pi}{\sqrt{\left|\gamma^2 - \omega^2\right|}}\)
	\end{itemize}
	\item il caso \(\gamma = \omega\) non viene trattato perché non capiterà mai nella realtà
\end{enumerate}

\newpage


\subsection{Risonanza e oscillatore armonico}
È un fenomeno fisico per cui un corpo si muove con una specifica legge oraria (lungo una direzione) dovuta a:
\begin{itemize}[topsep=3pt, itemsep=0pt]
	\item[-] forza di richiamo (es. elastica) \(F_{el}(\tilde{z}(t)) = -k \, \tilde{z}(t)\)
	\item[-] attrito o smorzamento \(F_{attr}(v(t)) = -b \, v(t)\)
	\item[-] forza esterna periodica, detta forzante \(F_{ext}(t) = F_0 \sin (\Omega t)\)
\end{itemize}
La somma delle forze è \(F = -k \, \tilde{z}(t) - b \, v(t) + F_0 \sin (\Omega t)\) per cui:
\[m \, a_z(t) = -k \, \tilde{z}(t) - b \, v(t) + F_0 \sin (\Omega t) \quad \Rightarrow \quad \left(\dts + 2\gamma \dt + \omega^2\right) \tilde{z}(t) = \frac{F_0}{m} \sin (\Omega t)\]
La soluzione è una combinazione lineare di \(z_s(t) = A \sin (\omega t + \Phi)\) soluzione particolare e
\(B \, e^{\lambda_1 t} + C \, e^{\lambda_2 t}\) soluzioni dell'omogenea associata:
\[\tilde{z}(t) = A \sin (\Omega t + \Phi) + B \, e^{\lambda_1 t} + C \, e^{\lambda_2 t}\]
\[\tan \Phi = - \frac{2 \, \gamma \, \Omega}{\omega^2 - \Omega^2} \qquad A = \frac{F_0}{m \, \sqrt{(\omega^2 + \Omega^2)^2 + 4 \gamma^2 \Omega^2}} \qquad \lambda_{1,2} = -\gamma \pm \sqrt{\gamma^2 - \omega^2} \qquad 2\gamma = \frac{b}{m} \quad \omega^2 = \frac{k}{m}\]
\begin{itemize}[topsep=3pt, itemsep=0pt]
	\item[-] \(A, \Phi\) parametri fissati in termini di \(\omega^2\), \(\gamma\), \(\frac{F_0}{m}\), \(\Omega\)
	\item[-] \(B, C\) determinati in funzione di \(z_0\) e \(v_0\)
	\item[-] per \(\displaystyle \lim_{t \to +\infty} B \, e^{\lambda_1 t} + C \, e^{\lambda_2 t} = 0\) per cui il moto è determinato
	soltanto da \(\tilde{z}(t) = A \sin (\Omega t + \Phi)\) e dipende solamente da \(A\) e \(\Phi\)
	\item[-] studiando \(A\) per \(\frac{F_0}{m}\) fissato:
	\begin{itemize}[topsep=3pt, itemsep=0pt]
		\item[-] per \(\gamma\) (smorzamento) grande si ha \(\displaystyle A \approx \frac{F_0}{m (2 \gamma \Omega + \text{errore})}\)
		per cui il grafico (\(A(\Omega)\) in funzione di \(\Omega\)) rappresenta un ramo di iperbole: se la frequeza del forzante
		è bassa, lo sarà anche l'ampiezza del movimento, viceversa se la frequenza è alta, l'ampiezza sarà minore
		\item[-] per \(\gamma\) piccolo si ha sempre un ramo di iperbole, con un punto di massimo quando \(\omega = \Omega\) ovvero
		quando il denominatore è prossimo allo 0: se la frequenza del forzante è vicina a quella del moto armonico, il moto armonico
		e il moto generato dal forzante vanno in interferenza costruttiva (risonanza) e \(\displaystyle A_{max} \approx \frac{F_0}{2 \, m \, \gamma \, \omega}\)
	\end{itemize}
	\item[-] studiando \(\Phi\):
	\begin{itemize}[topsep=3pt, itemsep=0pt]
		\item[-] se \(\Omega \approx \omega\) si ha \(\tan \Phi \approx \infty\) per cui \(\Phi \approx 90^\circ\), cioè quando la
		frequenza del forzante e del moto armonico sono simili si ha ampiezza massima \(A_{max}\) e lo sfasamento tra il forzante e
		il moto è di circa 90°, ovvero si ha quadratura di fase
		\item[-] se \(\Omega \gg \omega\) si ha \(\tan \Phi \approx \frac{2 \gamma}{\Omega}\) per cui \(\Phi \approx 0^\circ\), cioè
		quando la frequenza del forzante è maggiore di quella del moto, lo sfasamento tra forzante e moto è circa 0° e il moto si
		dice in fase
	\end{itemize}
	\item[-] quando si parla di grandezze piccole, si indende trascurabili rispetto alle altre (per ordine di grandezza circa 10/100)
\end{itemize}

\newpage

\subsection{Velocità limite}
Corpo in caduta libera soggetto alla forza peso e alla forza di attrito viscoso con equazione delle forze:
\[m \, a_z(t) = -m \, g -b \, v_z \qquad \Rightarrow \qquad \dt v_z(t) + \frac{b}{m} v_z(t) = -g\]
Combinando la soluzione dell'omogenea \(\displaystyle v_z(t) = v_0 \, e^{- \frac{b}{m} t}\) e la soluzione particolare della
differenziale \(\displaystyle {v_z}_s(t) = -\frac{m g}{b}\) (e integrandole per trovare la posizione) si ottengolo le equazioni:
\begin{align*}
	v_z(t) &= -\frac{mg}{b} + v_0 \, e^{-\frac{b}{m}t} \\
	z(t) &= z_0 -\frac{mg}{b}t + v_0 \frac{m}{b} \left(1-e^{-\frac{b}{m}t}\right)
\end{align*}
\begin{itemize}[topsep=3pt, itemsep=0pt]
	\item[-] studiando il moto per \(t \to +\infty\) si osserva che \(\displaystyle v_\infty(t) = v_\text{limite} = -\frac{mg}{b}\)
	cioè si ha una velocità limite costante e la posizione diventa direttamente proporzionale al tempo \(\displaystyle z_\infty(t) = z_0 + v_0 \frac{m}{b} - \frac{mg}{b} t\)
	\item[-] il tempo di arrivare a velocità prossime a quella limite è detto transiente
\end{itemize}

\subsection{Fili in tensione}
Applicando una forza ad un filo ideale (inestensibile e di massa trascurabile), tale forza si propaga lungo il filo e si scarica
sull'altra estremità del filo. Le forze possono cambiare direzione se il filo cambia direzione.
\subsubsection*{Macchina di Atwood}
Nell'esempio della macchina di Atwood (detta carrucola per gli amici) con due corpi \(A\) e \(B\) con masse \(m_1\) e \(m_2\)
l'equazione delle forze lungo \(\uz\) sono:
\[m_1 \, a_1(t) = -m_1 g + T \qquad m_2 \, a_2(t) = -m_2 g + T \]
\begin{itemize}[topsep=3pt, itemsep=0pt]
	\item[-] \(T\) è la tensione applicata al filo
	\item[-] \(a_1(t) = -a_2(t) = a(t)\) è la stessa per entrambi i corpi (a meno di un segno) in quanto legati dal filo
	\item[-] il	segno è dovuto al fatto che uno si muoverà verso l'alto, l'altro verso il basso ed è stato scelto per convenzione
	positivo per il corpo \(A\)
\end{itemize}
Risolvendo le equazioni, si ottiene un moto uniformemente accelerato con accelerazione \(a(t)\) costante che vale
\(\displaystyle a_z = g \, \frac{m_2 - m_1}{m_2 + m_1}\) per cui le accelerazioni \(a_1\), \(a_2\) e la tensione \(T\) valgono:
\[a_1(t) = g \, \frac{m_2 - m_1}{m_2 + m_1} \; \uz \qquad a_2(t) = g \, \frac{m_1 - m_2}{m_2 + m_1} \; \uz \qquad T = 2g \, \frac{m_1 m_2}{m_1 + m_2}\]
Si osserva le soluzioni sono simmetriche per i due corpi, infatti il problema è simmetrico anche nella realtà

\subsubsection*{Moto circolare con filo}
Dato un oggetto che si muove di moto circolare, vincolato al centro da un filo ideale, si hanno:
\[\vec{r}(t) = R \; \ur(t) \quad \vec{v}(t) = R \, \omega(t) \; \uper(t) \quad \vec{a}(t) = R \, \alpha(t) \; \uper() - R \, \omega^2(t) \; \ur(t) \quad \; \omega(t) = \dt \theta(t), \; \alpha(t) = \dts \theta(t)\]
La tensione del filo è data dal prodotto della massa per l'accelerazione centripeta:
\[\vec{T} = m \, \vec{a}(t) = R \, \alpha(t) \; \uper(t) - R \, \omega^2(t) \; \ur(t) \qquad \text{per} \; \alpha(t) = 0 \quad \Rightarrow \quad \vec{T} = - m \, R \, \omega^2 \; \ur(t)\] 
\begin{itemize}[topsep=3pt, itemsep=0pt]
	\item[-] \(\alpha(t)\) accelerazione angolare, \(\omega(t)\) velocità angolare, \(R\) raggio del moto
	\item[-] \(\ur\) versore con stesso orientamento di \(R\), \(\uper\) versore perpendicolare a \(\ur\)
	\item[-] il segno \(-\) in \(\vec{T}\) è dato dal fatto che la tensione tiene l'oggetto vincolato al centro
\end{itemize}
\newpage

\subsection{Pendolo semplice}
Corpo di massa \(m\) attaccato ad un filo ideale che oscilla sul piano \(\ux - \uz\) ha equazioni del moto:
\[m \, \vec{a}(t) = - mg \; \uz - T \; \ur(t) \quad \Rightarrow \quad \begin{cases}
	\displaystyle mR  \, \dts \theta(t) = -mg \sin (\theta(t)) &\text{lungo} \; \uper(t) \\
	\displaystyle -mR \, \omega^2 = mg \cos (\theta(t)) + T &\text{lungo} \; \ur(t)
\end{cases}\]
\begin{itemize}[topsep=3pt, itemsep=0pt]
	\item[-] la formula sopra usa (dal moto circolare) \(\vec{a}(t) = R \, \dts \theta(t) \; \uper(t) - R \, \omega^2 \; \ur(t)\)
	\item[-] dalla seconda equazione si ottiene \(\vec{T} = -mg \cos \theta(t) \; \ur(t) - mR \, \omega^2 \; \ur(t)\)
	\item[-] la risultante delle forze lungo \(\uper\) ha la forma di una forza di richiamo \(F = -k f(\theta(t))\), anche se non
	dipende linearmente dall'angolo \(\theta\)
	\item[-] per ottenere le equazioni del moto, è necessario risolvere la differenziale
\end{itemize}
Non è possibile risolvere la differenziale in termini di funzioni elementari, ma si osserva che per angoli piccoli (\(\theta < 1/4 \; \text{rad} \approx 15^\circ\)),
dallo sviluppo di Taylor si ha \(\sin \theta(t) = \theta(t) + \!\) errore trascurabile, per cui:
\[mR  \, \dts \theta(t) = -mg \sin (\theta(t)) \quad \stackrel{\theta \; \text{piccoli}}{\Rightarrow} \quad \dts \theta(t) = -\frac{g}{R} \theta(t) \quad \Rightarrow \quad \theta(t) = \theta_\text{max} \cos(\Omega t + \varphi_0)\]
\begin{itemize}[topsep=3pt, itemsep=0pt]
	\item[-] \(\theta_\text{max}\) ampiezza massima del moto
	\item[-] \(\varphi_0\) è lo sfasamento iniziale delle oscillazioni
	\item[-] \(\Omega\) frequenza delle piccole oscillazioni, \(\displaystyle \Omega^2 = \frac{g}{R}\)
	\item[-] il periodo del pendolo delle piccole oscillazioni vale \(\displaystyle T = \frac{2\pi}{\Omega}\)
	\item[-] \(\omega \; _\text{velocità angolare dal moto circolare} = \dt \theta(t) \neq \Omega \; _\text{frequenza piccole oscillazioni dal moto armonico} = \sqrt{\frac{g}{R}}\)
\end{itemize}

\subsection{Pendolo conico}
Corpo appeso ad un filo ideale che si muove di moto circolare (pendolo lungo una circonferenza)
\[m \, a_z(t) \; \uz + mR \dts \theta(t) \uper(t) - mR\left(\dt \theta(t)\right)^2 \; \ur(t) = - mg \, \uz + T \cos \alpha \, \uz - T \sin \alpha \; \ur(t) \]
Scomponendo lungo le componenti si ha:
\begin{itemize}[topsep=3pt, itemsep=0pt]
	\item[-] lungo \(\uz\): \(\displaystyle \quad m \, a_z = -mg + T \cos \alpha \qquad \text{per ipotesi} \; v_z = a_z = 0 \quad \Rightarrow \quad T = \frac{mg}{\cos \alpha}\)
	\item[-] lungo \(\uper\): \(\displaystyle \quad mR \, \dts \theta(t) = 0 \quad \Rightarrow \quad \dts \theta(t) = 0 \quad \Rightarrow \quad \text{il corpo ha velocità angolare costante}\)
	\item[-] lungo \(\ur\): \(\displaystyle \quad -mR \omega^2 = -T \sin \alpha \quad \Rightarrow \quad \omega^2 = \frac{T \sin \alpha}{mR} = \frac{\tan \alpha}{R}\)
\end{itemize}
Il moto è circolare uniforme con velocità angolare costante \(\displaystyle \omega = \sqrt{\frac{T \sin \alpha}{mR}} = \sqrt{\frac{\tan \alpha}{R}}\) \\
L'equazione della posizione è \(\displaystyle \vec{r}(t) = - l \cos \alpha \; \uz + R \cos (\omega t + \varphi_0) \; \ux + R \sin (\omega t + \varphi_0) \; \uy\)
\begin{itemize}[topsep=3pt, itemsep=0pt]
	\item[-] \(l\) lunghezza del filo
	\item[-] \(\alpha\) angolo tra il filo e la perpendicolare
	\item[-] \(\omega\) velocità angolare
	\item[-] \(R\) raggio del moto \(R = l \sin \alpha\)
\end{itemize}

\newpage

\subsection{Curve sopraelevate (paraboliche)}
Corpo che si muove in moto circolare su una guida circolare a parabolica con inclinazione \(\theta\) dall'orizzontale (il corpo
non cambia quota durante il moto):
\[m \, \vec{a} = \vec{F}_\text{peso} + \vec{F}_\text{vincolare} \quad \Rightarrow \quad - m \, a_\text{centripeta} \; \ur = - mg \; \uz + F_V \cos \theta \; \uz + F_V \sin \theta \; \ur\]
Scomponendo lungo le componenti si ha:
\begin{itemize}[topsep=3pt, itemsep=0pt]
	\item[-] lungo \(\uz\): \(\displaystyle \quad 0 = - mg + F_V \cos \theta \quad \Rightarrow \quad F_V = \frac{mg}{\cos \theta}\)
	\item[-] lungo \(\ur\): \(\displaystyle \quad m \, a_\text{centripeta} = F_V \sin \theta \quad a_\text{centripeta} = g \tan \theta\)
\end{itemize}
Dal moto circolare uniforme \(\displaystyle a_\text{centripeta} = \frac{m v^2}{R} \quad \Rightarrow \quad \tan \theta = \frac{v^2}{Rg}, \quad v = \sqrt{Rg \tan \theta}\)

Un corpo, per procedere in moto circolare uniforme, deve avere una forza centripeta che lo fa rimanere in traiettoria e non lo fa
partire per la tangente. Questa forza può essere (lista incompleta):
\begin{itemize}[topsep=3pt, itemsep=0pt]
	\item[-] una tensione (dovuta ad un filo)
	\item[-] la forza di attrito dell'oggetto con il suolo (ad esempio degli pneumatici con l'asfalto)
	\item[-] la reazione vincolare del piano (se l'oggetto corre su una guida circolare)
\end{itemize}

\newpage

\section{Lavoro ed energia}
\subsection{Definizione}
Per un punto materiale che compie una traiettoria \(\Gamma = \left\{ \vec{\gamma}(\tau), \; \tau \in \left[ t_0, t_1 \right] \right\}\)
soggetto ad una generica forza \(\displaystyle \vec{F} = \vec{F}\left( \vec{\gamma}(\tau), \; \dtau \vec{\gamma}(\tau), \; \tau \right)\), il lavoro di \(\vec{F}\)
lungo \(\Gamma\) è:
\[W_{\vec{F}} [\Gamma] = \int_{t_0}^{t_1} \vec{F} \left( \vec{\gamma}(\tau), \; \dtau \vec{\gamma}(\tau), \; \tau \right) \cdot \dtau \vec{\gamma}(\tau) \; d \tau\]
Il lavoro ha le seguenti proprietà:
\begin{enumerate}
	\item[0.1.] il lavoro è un numero e non un vettore \(W_{\vec{F}} [\Gamma] \in \mathbb{R}\)
	\item[0.2.] \(\vec{\gamma}(\tau)\) è una curva generica e non necessariamente la legge oraria del corpo per \(\vec{F}\), se
	però vale che \(\displaystyle m \, \dts \vec{\gamma}(t) = \vec{F}\left( \vec{\gamma}(t), \; \dt \vec{\gamma}(t), \; t \right)\)
	allora \(\vec{\gamma}(t)\) è la legge oraria e la chiameremo \(\vec{r}(t)\)
	\item Se una forza \(\vec{F}\) è la somma di due forze \(\vec{F}_1\) e \(\vec{F}_2\), allora \(W_{\vec{F}}[\Gamma] = W_{\vec{F}_1}[\Gamma] + W_{\vec{F}_2}[\Gamma]\)
	\item Se la traiettoria \(\Gamma\) è concatenazione di curve \(\Gamma_1\) e \(\Gamma_2\) (\(\Gamma = \Gamma_2 \circ \Gamma_1\)),
	allora \(W_{\vec{F}}[\Gamma] = W_{\vec{F}}[\Gamma_1] + W_{\vec{F}}[\Gamma_2]\)
	\item Se \(\vec{F}\) non dipende dal tempo \(\displaystyle \vec{F} = \vec{F}\left( \vec{\gamma}(\tau), \; \dtau \vec{\gamma}(\tau) \right)\),
	allora \(W_{\vec{F}}[\Gamma]\) dipende solo dalla traiettoria \(\left\{ \vec{\gamma}(\tau), \; \tau \in \left[ t_0, t_1 \right] \right\}\), ma
	non dalla parametrizzazione della curva (ovvero dal modo di percorrenza come velocità, accelerazione, \dots)
	\item Se \(\vec{\gamma}\) è la traiettoria \(\vec{r}(t)\), allora \(\displaystyle W_{\vec{F}}[{\vec{r}(t)}] = \Delta E_k = E_k(\vec{v}_1) - E_k(\vec{v}_0)\)
	con \(\displaystyle E_k = \frac{1}{2} m \vmod{\vec{v}}^2\), ovvero il lavoro è uguale alla variazione di energia cinetica \(E_k\)
	\item il lavoro e l'energia cinetica si misurano in \([W] = [E_k] = N \cdot m = J\) (joule)
\end{enumerate}

\subsection{Lavoro della forza peso}
Sia \(\vec{F}_P = -mg \; \uz\) e sia \(\vec{\gamma}(\tau)\) una curva da \(\vec{r}_0 = (x_0, y_0, z_0)\) a \(\vec{r}_1 = (x_1, y_1, z_1)\),
il lavoro della forza peso lungo \(\Gamma\) è definito come:
\[W_{\vec{F}_P}[\Gamma] = \int_{t_0}^{t_1} -mg \; \uz \cdot \dtau \vec{\gamma}(\tau) \; d\tau = - mg \int_{t_0}^{t_1} \dtau \gamma_z(\tau) \; d\tau = -mg \int_{z_0}^{z_1} d\gamma_z = -mg (z_1 - z_0)\]

\subsection{Lavoro della forza elastica}
Sia \(\vec{F}_{el} = - kr \ur\) (in generale \(\vec{F}_{el} = - k (r - r_0) \ur\)), il lavoro della forza elastica lungo \(\Gamma\) è:
\begin{align*}
	W_{\vec{F}_P}[\Gamma] &= \int_{t_0}^{t_1} -kr \; \ur(\tau) \cdot \dtau \vec{\gamma}(\tau) \; d\tau = - k \int_{t_0}^{t_1} \gamma_r \ur \cdot \left( \dtau \gamma_r(\tau) \; \ur + \gamma_r(\tau) \dtau \theta(\tau) \; \uper \right) \; d\tau = \\
	&= -k \int_{t_0}^{t_1} \gamma_r(\tau) \, \dtau \gamma_r(\tau) \; d\tau = -k \int_{r_0}^{r_1} \gamma_r \; d\gamma_r = -k \left[ \frac{1}{2} \, {\gamma_r}^2 \right]_{r_0}^{r_1} = -\frac{k}{2} \; \left( {r_1}^2 - {r_0}^2 \right)
\end{align*}
Si osserva che \(\vec{r} = r \; \ur = \gamma_r \; \ur = \vec{\gamma}\) e che \(\displaystyle \dtau \vec{\gamma}(\tau) = \dtau \gamma_r(\tau) \; \ur + \gamma_r(\tau) \dtau \theta(\tau) \; \uper\)

\newpage

\subsection{Lavoro della forza di reazione vincolare}
Sia \(\vec{F}_R = F_R \uz\) reazione vincolare su un piano \((x,y)\) e \(\vec{\gamma}(\tau) = \gamma_x(\tau) \ux + \gamma_y(\tau) \uy\),
il lavoro della forza vincolare è:
\[W_{\vec{F}_R}[\Gamma] = \int_{t_0}^{t_1} F_R \; \uz \cdot \left( \dtau \gamma_x(\tau) \; \ux + \dtau \gamma_y(\tau) \; \uy \right) = 0 \qquad \qquad (\uz \cdot \ux = \uz \cdot \uy = 0)\]

\subsection{Lavoro della forza di attrito radente}
Sia \(\vec{F}_A = -\mu_d F_\perp \; \uv\) forza di attrito dinamico, il lavoro compiuto dalla forza è:
\begin{align*}
	W_{\vec{F}_A}[\Gamma] &= \int_{t_0}^{t_1} -\mu_d F_\perp \; \uv \; \cdot \; \dtau \vec{\gamma}(\tau) \; d\tau = -\mu_d F_\perp \int_{t_0}^{t_1} \uv \; \cdot \; \dtau \vmod{\vec{\gamma}(\tau)} \; \uv \; d\tau = -\mu_d F_\perp \int_{t_0}^{t_1} \vmod{\vec{\gamma}(\tau)} d\tau = \\
	&= -\mu_d F_\perp \mathcal{L}(\Gamma)
\end{align*}
Si osserva che \(\displaystyle \dtau \vec{\gamma}(\tau) = \vec{v}\), per cui \(\uv = \vec{u}_\frac{d \, \vec{\gamma}(\tau)}{d\tau}\),
inoltre \(\mathcal{L}(\Gamma)\) è la lunghezza della curva \(\Gamma\), per cui il lavoro dipende dal percorso compiuto tra i due
estremi \(r_0\) e \(r_1\).

\subsection{Lavoro di una generica forza centrale}
Data una forza centrale \(\vec{F}_C = - f(\vmod{\vec{\gamma}}) \ug\), (es. forza di gravitazione universale, forza elastica, \dots),
il lavoro è definito come:
\begin{align*}
	W_{\vec{F}_C}[\Gamma] &= \int_{t_0}^{t_1} f(\vmod{\vec{\gamma}(\tau)}) \; \ug \; \cdot \; \left( \dtau \vmod{\vec{\gamma}(\tau)} \; \ug + \vmod{\vec{\gamma}(\tau)} \dt \theta(\tau) \; \uper \right) \; d\tau = \\
	&= \int_{t_0}^{t_1} f(\vmod{\vec{\gamma}(\tau)}) \; \dtau \vmod{\vec{\gamma}(\tau)} \; d\tau = \int_{\vmod{\vec{\gamma}(t_0)}}^{\vmod{\vec{\gamma}(t_1)}} f(\vmod{\vec{\gamma}}) \; d\vmod{\vec{\gamma}} = \\
	&= \left[ V(\vmod{\vec{\gamma}}) \right]_{\vmod{\vec{\gamma}_i}}^{\vmod{\vec{\gamma}_f}} = V(\vmod{\vec{\gamma}_f}) - V(\vmod{\vec{\gamma}_i})
\end{align*}
Per forza elastica \(\displaystyle V(\vmod{\vec{\gamma}}) = -\frac{k}{2} \vmod{\vec{\gamma}}^2\), per la forza gravitazionale
\(\displaystyle V(\vmod{\vec{\gamma}}) = -G\frac{m_1 \, m_2}{\vmod{\vec{\gamma}}}\).

\subsection{Lavoro lungo un circuito chiuso}
Sia \(\Gamma\) un circuito chiuso costituito dalla concatenazione di due curve \(\Gamma_1\) e \(\Gamma_2\), il lavoro lungo il
circuito chiuso di una forza \(\vec{F}\) vale:
\[W_{\vec{F}}[\Gamma] = W_{\vec{F}}[\Gamma_1] + W_{\vec{F}}[\Gamma_2]\]

Se la forza dipende solo dalla posizione iniziale e finale, il lavoro diventa:
\[W_{\vec{F}}[\Gamma] = W_{\vec{F}}[\Gamma_1] + W_{\vec{F}}[\Gamma_2] = \int_{t_0}^{t_1}f(\tau) d\tau + \int_{t_1}^{t_0}f(\tau) d\tau = 0\]

Per indicare il lavoro lungo un circuito chiuso si utilizza l'integrale:
\[W_{\vec{F}}[\Gamma_\text{chiuso}] = \oint \vec{F}\left(\vec{\gamma}(\tau), \; \dtau \vec{\gamma}(\tau), \; \tau\right) \cdot \, \dtau \vec{\gamma}(\tau) \; d\tau\]

Per le forze che dipendono solo dalla posizione iniziale e finale, si ha:
\[W_{\vec{F}}[\Gamma_\text{chiuso}] = \oint \vec{F}\left(\vec{\gamma}(\tau), \; \dtau \vec{\gamma}(\tau), \; \tau\right) \cdot \, \dtau \vec{\gamma}(\tau) \; d\tau = 0\]

\newpage

\section{Forze conservative e non conservative}
\subsection{Definizione}
\subsection{Energia potenziale}
\subsubsection*{Potenziali da ricordare}
\subsubsection*{Potenziali in 1D}
\subsection{Teorema dell'energia cinetica}
\subsection{Applicazioni del lavoro ed energia}
\subsection{Potenza}
\subsection{Gradiente di una forza}

\section{Quantità conservate}
\subsection{Lavoro}
\subsection{Impulso}
\subsection{Momento angolare}
\subsection{Applicazioni del momento angolare}

\section{Trasformazioni tra sistemi di riferimento}
\subsection{Posizione}
\subsection{Velocità}
\subsection{Accelerazione}
\subsection{Forze e forze apparenti}
\subsection{Esperimento di Guglielmini}

\end{document}
