\documentclass[a4paper]{article}
\usepackage[utf8]{inputenc} % standard unicode
\usepackage[italian]{babel} % corretta sillabazione in italiano
\usepackage{geometry} % per impostare margini e layout pagina
\usepackage{amssymb} % per l'ambiente matematico
\usepackage{amsmath} % per l'ambiente matematico
\usepackage{enumitem} % per elenchi puntati
\usepackage{multirow} % per celle che si espandono su più righe
\usepackage{tabularx} % per tabelle con larghezza flessibile
\usepackage{booktabs} % per linee orizzontali tabelle
\usepackage{hyperref} % per collegamenti
\usepackage{graphicx} % per immagini
\usepackage{listings} % per codice
\usepackage{xcolor} % per colori nel codice
\usepackage{dirtytalk} % per le ""

% per margini
\geometry{a4paper,left=25mm, right=25mm, bottom=25mm, top=30mm}

% per centrare testo nelle tabelleX
\renewcommand\tabularxcolumn[1]{m{#1}}

% percorso delle immagini da inserire
\graphicspath{ {./ } }

% parte funzione reale e parte immaginaria
\newcommand\Real{\text{Re}}
\newcommand\Img{\text{Im}}

\title{Appunti di Fisica 1}
\author{Giacomo Simonetto}
\date{Secondo semetre 2023-24}

\begin{document}

% -------------------------------------- Copertina e indice ---------------------------------------
\maketitle
\begin{abstract}
	Appunti del corso di Fisica 1 - (Meccanica e termodinamica) della facoltà di Ingegneria Informatica dell'Università di Padova.
\end{abstract}

\newpage

\tableofcontents

\newpage

% ----------------------------------------- introduzione ------------------------------------------
\section{Introduzione}
\subsection{Interazioni fondamentali}
Le interazioni (o forze) fondamentali sono:
\begin{enumerate}[topsep=3pt, itemsep=0pt]
	\item forza gravitazionale: scoperta per prima nel 1600 circa da Galileo
	\item forza elettromagnetica: scoperta nel 1800
	\item forza debole: legata ai costituenti degli atomi (radioattività)
	\item forza forte: legata ai costituenti degli atomi (quark)
\end{enumerate}
Si sta cercando un legame tra la forza elettromagnetica e quella debole (forza elettrodebole) e una teoria che lega le forze
elettromagnetica, debole e forte (teoria delle forze unificate). La forza gravitazione è considerata particolare in quanto:
\begin{itemize}[topsep=3pt, itemsep=0pt]
	\item[-] è molto meno intensa delle altre
	\item[-] ha solo "carica" positiva (non esiste massa negativa)
	\item[-] spazio e forza gravitazionale non possono essere sepatati
	\item[-] non si è ancora riusciti a comprenderla quantisticamente
\end{itemize}

% ---------------------------------------- punto materiale ----------------------------------------
\section{Il punto materiale}
\subsection{Introduzione al punto materiale}
È una finzione matematica in quanto non esiste nella realtà, ma serve come approssimazione. Non ha estensione, ma ha una massa
\(m\) ed è possibile determinarne la posizione. In un sistema di riferimento (cartesiano con 3 assi), la posizione è data dal
vettore \(\vec{r_0} = \left(x_0, y_0, z_0\right)\).

\subsection{Grandezze elementari}

massa: \(m\), l'unità di misura è \(\left[m\right] = kg\)

posizione: \(\vec{r_0} = \left(x_0, y_0, z_0\right)\), con unità di misura \(\left[x_0\right] = \left[y_0\right] = \left[z_0\right] = m\)

tempo: \(t\), con unità di misura \(\left[t\right] = s\)

\newpage


\subsection{Punto materiale in movimento - cinematica}
\subsubsection*{Posizione}
La posizione nello spazio di un punto sono le coordinate del punto un sistema di riferimeno cartesiano di 3 assi.
\[\vec{r}(t) = \left(x_0(t), y_0(t), z_0(t)\right) \qquad \left[x_0\right] = \left[y_0\right] = \left[z_0\right] = m\]

\subsubsection*{Velocità}
La velocità è lo spazio percorso in un tempo piccolo. \[\vec{v}(t) = \frac{d}{dx} \vec{r}(t) \qquad \left[v\right] = \frac{m}{s}\]
Per ottenere la posizione dalla velocità: \[\vec{r}(t) = \vec{r_0} + \int_{t0}^{t1} \vec{v}(\tau) \, d\tau\]

\subsubsection*{Accelerazione}
L'accelerazione è la variazione della velocità nel tempo. \[\vec{a}(t) = \frac{d}{dx} \vec{v}(t) = \frac{d^2}{dx^2} \vec{r}(t) \qquad \left[a\right] = \frac{m}{s^2}\]
Per ottenere la velocità dall'accelerazione: \[\vec{v}(t) = \vec{v_0} + \int_{t0}^{t1} \vec{a}(\tau) \, d\tau\]
Per ottenere la posizione dall'accelerazione: \[\vec{r}(t) = \vec{r_0} + \int_{t0}^{t1} \left( \vec{v_0}(\tau) + \int_{t0}^{t1} \vec{a}(\tau) \, d\tau \right) \, d\tau \quad \left(= \vec{r_0} + \vec{v_0}(t-t_0) + \iint_{t_0}^{t_1} \vec{a}(\tau) \, d\tau^2 \right)\]

\subsubsection*{Moto uniformemente accelerato}
Moto con accelerazione costante, le leggi orarie sono:
\[ \begin{cases}
	\vec{r}(t) = \vec{r_0} + \vec{v_0}(t-t_0) + \frac{1}{2}\vec{a}(t-t_0)^2 \\
	\vec{v}(t) = \vec{v_0} + \vec{a}(t-t_0)
\end{cases} \]
Per convenzione si sceglie \(t_0 = 0\):
\[ \begin{cases}
	\vec{r}(t) = \vec{r_0} + \vec{v_0}t + \frac{1}{2}\vec{a}t^2 \\
	\vec{v}(t) = \vec{v_0} + \vec{a}t
\end{cases} \]

Si osserva che per trovare \(\vec{r}(t)\) a partire dall'accelerazione è necessario conoscere i due dati iniziali \(\vec{r_0}\)
(posizione) e \(\vec{v_0}\) (velocità), in quanto sono stati fatti due integrali nel calcolo.

Ogni vettore \(\vec{r}(t)\), \(\vec{v}(t)\) e \(\vec{a}(t)\) può essere scomposto nelle tre componenti \(x, y, z\) degli assi cartesiani
ottenendo tre equazioni del moto, una per ogni asse.

Esempi di applicazioni:
\begin{itemize}
	\item[-] caduta di un grave (da fermo e con moto orizzontale)
	\item[-] moto di due automobili sulla stessa retta
	\item[-] moto di un proiettile (con angolo inziale \(\theta\) rispetto al suolo)
\end{itemize}

\newpage

\subsection{Moto armonico semplice in una dimensione}
\begin{align*}
	x(t) &= A \sin (\omega t + \varphi_0) = A \cdot \sin (\omega (t - t_0)) \qquad \text{con } -\omega t_0 = \varphi_0 \\
	v(t) &= A \omega \cos (\omega t + \varphi_0) \\
	a(t) &= -A \omega^2 \sin (\omega t + \varphi_0) = -\omega^2 x(t)
\end{align*}
\begin{itemize}[topsep=3pt, itemsep=0pt]
	\item[-] \(A\) ampiezza del moto, \(\left[A\right] = m\)
	\item[-] \(\omega\) velocità angolare, \(\left[\omega\right] = \frac{rad}{s}\)
	\item[-] \(\varphi_0\) sfasamento iniziale, \(\left[\varphi_0\right] = rad\)
	\item[-] si osserva che \(\left[A\right] = m\), \(\left[A\omega\right] = \frac{m}{s}\), \(\left[A\omega^2\right] = \frac{m}{s^2}\)
\end{itemize}

\subsection{Moto circolare uniforme sul piano xy}
\begin{align*}
	\vec{r}(t) &= \left(A \cos (\omega t + \varphi_0), \; A \sin (\omega t + \varphi_0), \; 0\right) \\
	\vec{v}(t) &= \left(-A \omega \sin (\omega t + \varphi_0), \; A \omega \cos (\omega t + \varphi_0), \; 0 \right) \\
	\vec{a}(t) &= \left(-A \omega^2 \cos (\omega t + \varphi_0), \; -A \omega^2 \sin (\omega t + \varphi_0), \; 0 \right) = -\omega^2 \vec{r}(t)
\end{align*}
\begin{itemize}[topsep=3pt, itemsep=0pt]
	\item[-] il vettore velocità è tangente alla circonferenza e perpendicolare al raggio
	\item[-] il vettore accelerazione è perpendicolare a \(\vec{v}\), opposto a \(\vec{r}\) e diretto verso il centro
	\item[-] l'accelerazione del moto è chiamata accelerazione centripeta
\end{itemize}

\subsection{Moto vario}
\begin{itemize}[topsep=3pt, itemsep=0pt]
	\item[-] la posizione è data da \(\vec{r}(t)\)
	\item[-] la velocità è data da \(\displaystyle \vec{v}(t) = \lim_{\Delta t \to 0} \frac{\vec{r}(t+\Delta t) - \vec{r}(t)}{\Delta t} = \frac{d}{dt} \vec{r}(t)\)
	\item[-] l'accelerazione è data da \(\displaystyle \vec{a}(t) = \lim_{\Delta t \to 0} \frac{\vec{v}(t+\Delta t) - \vec{v}(t)}{\Delta t} = \frac{d}{dt} \vec{v}(t) = \frac{d^2}{dt^2} \vec{r}(t)\)
	\item[-] la velocità è tangente alla traiettoria, ma non è detto che sia perpendicolare al vettore \(\vec{r}\)
\end{itemize}

\newpage


\section{Funzioni goniometriche (ripasso e proprietà)}
\subsection{Sviluppi di Taylor}
\[\sin \theta = \theta - \frac{\theta^3}{3!} + o(\theta^5) \qquad \qquad \cos \theta = 1 - \frac{\theta^2}{2!} + o(\theta^4)\]
\begin{itemize}[topsep=3pt, itemsep=0pt]
	\item[-] le formule valgono solo se \(\theta\) è un numero puro (non posso ad esempio sommare \(m\) e \(m^2\)).
	\item[-] \(\left[\theta\right] = rad\), si misura in radianti (numero puro), un radiante è il rapporto tra la lunghezza dell'arco
	di circonferenza che sottende un angolo \(\theta\) e il raggio della circonferenza.
\end{itemize}

\subsection{Formule di Eulero}
\[\sin \theta = \frac{e^{i\theta}-e^{-i\theta}}{2i} \qquad \qquad \cos \theta = \frac{e^{i\theta}+e^{-i\theta}}{2}\]
\begin{itemize}[topsep=3pt, itemsep=0pt]
	\item[-] le formule valgono solo se \(\Img (\sin \theta) = \Img (\cos \theta) = 0\) per \(\theta \in \mathbb{Q}\): \\
	ricordando che \(\displaystyle \Real (z) = \frac{z + z^*}{2}\), \(\displaystyle \Img (z) = \frac{z - z^*}{2i}\), si ha: \\
	\(\displaystyle \Img (\cos \theta) = \frac{1}{2i} \cdot \left(\frac{e^{i\theta}+e^{-i\theta}}{2} - \frac{e^{-i\theta}+e^{i\theta}}{2}\right) = \frac{0}{2i} = 0\) \\
	\(\displaystyle \Img (\sin \theta) = \frac{1}{2i} \cdot \left(\frac{e^{i\theta}-e^{-i\theta}}{2i} - \frac{e^{-i\theta}-e^{i\theta}}{2i}\right) = \frac{0}{2i} = 0\)
	\item[-] si osserva che \(\left|\cos \theta\right| \leq 1\), \(\left|\sin \theta\right| \leq 1\): \\
	\(\displaystyle \left|\cos \theta\right| = \left|\frac{e^{i\theta)}+e^{-i\theta}}{2}\right| = \frac{1}{2}\left|e^{i\theta} + e^{-i\theta}\right| \leq \frac{1}{2}\left|1 \cdot e^{i\theta}\right| + \left|1 \cdot e^{-i\theta}\right|= \frac{1}{2} (1+1) = 1\) \\
	\(\displaystyle \left|\sin \theta\right| = \dots\)
\end{itemize}

\subsection{Derivate}
\[\frac{d}{d\theta} \sin \theta = \cos \theta \qquad \frac{d}{d\theta} \cos \theta = -\sin \theta\]
\[\frac{d^2}{d\theta^2} \sin \theta = -\sin \theta \qquad \frac{d^2}{d\theta^2} \cos \theta = -\cos \theta\]

\newpage


\section{Vettori e versori}
\subsection{Definizione}
Un vettore è un \say{segmento orientato}, cioè definito da 3 proprietà: lunghezza, direzione e verso. Un vettore si indica con
lettere minuscole come \(\vec{a}\), \(\vec{b}\), \dots Questa definizione ci permette di essere indipendenti dal sistema di 
coordinate di riferimento.

La lungheza di un vettore è chiamata norma o modulo e si indica \(\left|\left| \vec{a} \right|\right|\)

\subsection{Prodotto per uno scalare}
Dati \(\vec{a}\) vettore e \(\lambda\) scalare (numero reale), allora \(\vec{b} = \lambda \vec{a}\) è un vettore tale che:
\begin{itemize}[topsep=3pt, itemsep=0pt]
	\item[-] se \(\lambda > 0\), \(\vec{b}\) ha stessa direzione e verso di \(\vec{a}\), con lunghezza \(\lambda\) volte quella di \(\vec{a}\)
	\item[-] se \(\lambda < 0\), \(\vec{b}\) ha stessa direzione e verso opposto di \(\vec{a}\), con lunghezza \(-\lambda\) volte quella di \(\vec{a}\)
	\item[-] se \(\lambda = 0\), \(\vec{b}\) è vettore nullo \(\vec{0}\)
\end{itemize}

\subsection{Somma di vettori}
Dati due vettori \(\vec{a}\), \(\vec{b}\) la loro somma è un vettore \(\vec{c} = \vec{a} + \vec{b}\) definita dalla regola del
parallelogramma. La somma ha le seguenti proprietà:
\begin{itemize}[topsep=3pt, itemsep=0pt]
	\item[-] \(\vec{a} + \vec{b} = \vec{b} + \vec{a}\)
	\item[-] \((\vec{a} + \vec{b}) + \vec{c} = \vec{a} + (\vec{b} + \vec{c})\)
	\item[-] \(\lambda (\vec{a} + \vec{b}) = \lambda \vec{a} + \lambda \vec{b}\)
	\item[-] \(\vec{a} (\lambda_1 + \lambda_2) = \vec{a} \lambda_1 + \vec{a} \lambda_2\)
	\item[-] \(\vec{a} - \vec{b} = \vec{a} + (-1)\vec{b}\)
	\item[-] \(\vec{a} - \vec{a} = \vec{a}(1-1) = \vec{0}\)
\end{itemize}	


\subsection{Versori}
\begin{itemize}[topsep=3pt, itemsep=0pt]
	\item[-] i versori sono vettori unitari (con lunghezza 1).
	\item[-] sono definiti come \(\vec{u_a} = \frac{1}{\left|\left|\vec{a}\right|\right|} \cdot \vec{a}\).
	\item[-] una terna di assi è definita da 3 versori \(\vec{u_x}\), \(\vec{u_y}\), \(\vec{u_z}\).
	\item[-] dato un vettore \(\vec{a} = \left(a_x, a_y, a_z\right)\) si può esprimere come \(\vec{a} = a_x \vec{u_x} + a_y \vec{u_y} + a_z \vec{u_z}\)
\end{itemize}

\subsection{Prodotto scalare}
\subsection{Prodotto vettore}
\subsection{Derivata di vettore}

\newpage


\section{Dinamica del punto materiale}
\subsection{Prima legge della dinamica}
\subsection{Seconda legge della dinamica}
\subsection{Terza legge della dinamica}
\subsection{Sistemi di riferimento inerziali}
\subsection{Forza gravitazionale universale}
\subsection{Forza elastica}
\subsection{Forza di reazione vincolare}
\subsection{Forza di attrito radente}
\subsection{Forza di attrito viscoso}

\section{Trovare le equazioni del moto}
\subsection{Piano inclinato}
\subsection{Oscillatore armonico semplice}
\subsection{Trovare le soluzioni del moto per una forza generica in 1D}
\subsection{Oscillatore armonico con l'azione della forza peso}
\subsection{Oscillatore armonico smorzato (con forza peso e attrito)}
\subsection{Risonanza e oscillatore armonico}
\subsection{Velocità limite}
\subsection{Fili in tensione}
\subsection{Pendolo semplice}
\subsection{Pendolo conico}
\subsection{Curve sopraelevate (paraboliche)}

\section{Lavoro ed energia}
\subsection{Introduzione}
\subsection{Lavoro della forza peso}
\subsection{Lavoro della forza peso}
\subsection{Lavoro della forza elastica}
\subsection{Lavoro della forza di reazione vincolare}
\subsection{Lavoro della forza di attrito radente}
\subsection{Segno del lavoro}

\newpage

\section{Forze conservative e non conservative}
\subsection{Introduzione}
\subsection{Energia potenziale}
\subsubsection*{Potenziali da ricordare}
\subsubsection*{Potenziali in 1D}
\subsection{Teorema dell'energia cinetica}
\subsection{Applicazioni del lavoro ed energia}
\subsection{Potenza}
\subsection{Gradiente di una forza}

\section{Quantità conservate}
\subsection{Lavoro}
\subsection{Impulso}
\subsection{Momento angolare}
\subsection{Applicazioni del momento angolare}

\section{Trasformazioni tra sistemi di riferimento}
\subsection{Posizione}
\subsection{Velocità}
\subsection{Accelerazione}
\subsection{Forze e forze apparenti}
\subsection{Esperimento di Guglielmini}

\end{document}
