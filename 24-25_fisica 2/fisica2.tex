\documentclass[a4paper]{article}
\usepackage[utf8]{inputenc} % standard unicode
\usepackage[italian]{babel} % corretta sillabazione in italiano
\usepackage{geometry} % per impostare margini e layout pagina
\usepackage{amssymb} % per l'ambiente matematico
\usepackage{amsmath} % per l'ambiente matematico
\usepackage{enumitem} % per elenchi puntati
\usepackage{multirow} % per celle che si espandono su più righe
\usepackage{tabularx} % per tabelle con larghezza flessibile
\usepackage{booktabs} % per linee orizzontali tabelle
\usepackage{hyperref} % per collegamenti
\usepackage{graphicx} % per immagini
\usepackage{listings} % per codice
\usepackage{xcolor} % per colori nel codice
\usepackage{dirtytalk} % per le ""

% per margini
\geometry{a4paper,left=25mm, right=25mm, bottom=25mm, top=30mm}

% per centrare testo nelle tabelleX
\renewcommand\tabularxcolumn[1]{m{#1}}

% percorso delle immagini da inserire
\graphicspath{ {./ } }

% parte funzione reale e parte immaginaria
\newcommand\Real{\text{Re}}
\newcommand\Img{\text{Im}}

% versori
\newcommand\ux{\vec{u}_x}
\newcommand\uy{\vec{u}_y}
\newcommand\uz{\vec{u}_z}
\newcommand\uxp{\vec{u}_{x'}}
\newcommand\uyp{\vec{u}_{y'}}
\newcommand\uzp{\vec{u}_{z'}}
\newcommand\ur{\vec{u}_r}
\newcommand\uv{\vec{u}_v}
\newcommand\un{\vec{u}_n}
\newcommand\ug{\vec{u}_\gamma}
\newcommand\uper{\vec{u}_\perp}
\newcommand\upar{\vec{u}_\parallel}
\newcommand\nab{\vec{\nabla}} % nabla

% forza generica
\newcommand\ft{\vec{F}\left(\vec{\gamma}(t), \; \dt \vec{\gamma}(t), \; t\right)}
\newcommand\ftau{\vec{F}\left(\vec{\gamma}(\tau), \; \dtau \vec{\gamma}(\tau), \; \tau\right)}

% derivata
\newcommand\dt{\frac{d}{dt}\,}
\newcommand\dtau{\frac{d}{d\tau}\,}
\newcommand\dts{\frac{d^2}{dt^2}\,}

% modulo vettore
\newcommand\vmod[1]{\left|\left|{#1}\right|\right|}

\title{Appunti di Elementi di Fisica 2}
\author{Giacomo Simonetto}
\date{Secondo semetre 2023-24}

\begin{document}

% -------------------------------------- Copertina e indice ---------------------------------------
\maketitle
\begin{abstract}
	Appunti del corso di Elementi di Fisica 2 - (Elettromagnetismo) della facoltà di Ingegneria Informatica dell'Università di Padova.
\end{abstract}

\newpage

\tableofcontents

\newpage

\section{Elettromagnetismo}
\subsection{Introduzione}
L'elettromagnetismo è una teoria descrittiva costruita su solide basi sperimentali, ovvero non si utilizzano principi primi, ma
si parte dalle osservazioni di fenomeni. I principali scienziati che si sono occupati del fenomeno sono Coulomb, Ampère, Faraday
e Maxwell. La teoria classica dell'elettromagnetismo è stata usata come base per la relatività e le correzioni quantistiche sono
inssignificanti per distanze inferiori a \(10^{-12} \; m\), ovvero 100 volte più piccole dell'atomo.

\subsection{Carica elettrica}
\subsubsection*{Definizione}
La carica elettrica è una grandezza fisica collegata al bilanciamento di elettroni e protoni nella materia. Si misura in Coulomb
e ha la proprietà poter essere sia positiva che negativa. Corpi di cariche con segno opposto si attraggono, corpi di cariche con
stesso segno si respingono.

\subsubsection*{Legge di conservazione della carica}
In un sistema isolato, la carica totale non cambia.Nei decadimenti radioattivi, quando viene emesso un elettrone, si emette anche
un positrone (per bilanciare la carica). L'universo appare come una miscela bilanciata di cariche elettriche.

\subsubsection*{Quantizzazione della carica}
La misura della carica è costituita da multipli della carica elementare \(e = 1.6022 \cdot 10^{-19} C\). Le particelle elementari
cariche hanno tutte la stessa carica \(e\):
\begin{center}
	\begin{tabular}{c c c}
		\textbf{Particella} & \textbf{Carica} & \textbf{Massa} \\
		\toprule
		Elettrone & \(-1.6022 \cdot 10^{-19} C\) & \(9.1094 \cdot 10^{-31} kg\) \\
		\midrule
		Protone & \(1.6022 \cdot 10^{-19} C\) & \(1.6726 \cdot 10^{27} kg\) \\
		\bottomrule
	\end{tabular}
\end{center}

\subsection{Legge di Coulomb}
Due cariche elettriche si respingono con una forza proporzionale al prodotto delle intensità delle cariche e inversamente
proporzionale al quadrato della loro distanza.
\[\vec{F}_{21} = \frac{1}{4 \pi \varepsilon_0} \cdot \frac{q_1 \, q_2}{r^2} \cdot \hat{r}_{2,1} \qquad \qquad \begin{matrix}
	\vec{F}_{21} \rightarrow \vec{F}_\text{ della particella 1 sulla particella 2}  \qquad \qquad \qquad \qquad \qquad \;\;\;\; \\
	\varepsilon_0 \rightarrow \text{costante dielettrica nel vuoto}, \varepsilon_0 = 8,854 \cdot 10^{-12} \frac{C^2}{Nm ^2}
\end{matrix}\]
La forza è newtoniana \(\vec{F}_{21} = -\vec{F}_{12}\) ed è additiva \(\vec{F}_{tot} = \sum_{i=2}^{\# \textbf{cariche}} \vec{F}_{1i}\). \\
In un atomo, la forza di Coulomb tra elettroni e protoni è molto più forte della forza gravitazionale, per cui quest'ultima è
trascurabile. Nell'universo, invece, si ha prevalentemente forza gravitazionale siccome i corpi celesti sono genericamente neutri.

\subsection{Esperimenti di elettrostatica}
\subsubsection*{Primo esperimento: bacchette di plexiglass e ebanite}
Caricando per strofinio le bacchette di ebanite e di plexiglass, si nota che due bacchette di uno stesso materiale si respingono,
mentre due bacchette di uno stesso materiale si attraggono. Il plexiglass si carica negativamente, l'ebanite si carica positivamente.

\subsubsection*{Secondo esperimento: elettroscopio a foglie}
Quando si avvicina una bacchetta carica al pomello dell'elettroscopio, le foglie si allontanano per fenomeno dell'induzione
elettrostatica, ovvero si verifica una redistribuzione delle cariche all'interno di un conduttore in presenza di un altro corpo
carico nelle vicinanze.

\subsection{Campo elettrico}
\subsubsection*{Definizione}
Il campo elettrico si definisce come \(\vec{E} := \vec{F}_{tot,0}/q_0\) dove \(q_0\) è detta carica di prova e \(\vec{F}_0\) è
la forza totale agente sulla carica di prova.
\[\vec{E}(x,y,z) = \sum_{i} \frac{1}{4 \pi \varepsilon_0} \; \frac{q_i}{r_{0i}^2} \; \hat{r}_{0i} \qquad \qquad \vec{E} = \vec{F}_{tot,0}/q_0 \qquad \vec{F}_{tot,0} = q_0 \vec{E} \qquad \qquad \vec{E} = \left[ \frac{N}{C} \right] = \left[ \frac{V}{M} \right]\]
Siccome la forza è additiva, anche il campo elettrico è additivo.

\subsubsection*{Linee del campo elettrostatico}
Le linee del campo elettrostatico permettono una rappresentazione grafica complessiva del campo elettrico nello spazio. Le linee
di campo sono curve con le seguenti caratteristiche:
\begin{itemize}[topsep=3pt, itemsep=0pt]
	\item[-] in ogni punto sono tangenti ai vettori del campo
	\item[-] il campo è più intenso dove c'è maggiore densità
	\item[-] le linee non si incrociano mai
	\item[-] si originano dalle cariche positive e si chiudono nelle cariche negative, se è presente solo una carica, si chiudono
	all'infinito.
\end{itemize}

\subsection{Distribuzione continua di carica}
\subsubsection*{Tipologie di distribuzione}
Il campo elettrico generato da una distribuzione continua di cariche vale:
\[\vec{E}(x,y,z) = \frac{1}{4 \pi \varepsilon_0} \int \limits_\text{distribuzione} \frac{dq}{r^2} \; \hat{u}\]
\begin{itemize}[topsep=3pt, itemsep=-7pt]
	\item[-] \textbf{Distribuzione lineare}: descritta da \(\lambda\) che descrive la carica per unità di lunghezza \(dl\).
	\[dq = \lambda dl \quad \rightarrow \quad \vec{E}(x,y,z) = \frac{1}{4 \pi \varepsilon_0} \int_L \frac{\lambda dl}{r^2} \; \hat{u}\]
	\item[-] \textbf{Distribuzione superficiale}: descritta da \(\sigma\) che descrive la carica per unità di superficie \(d\Sigma\).
	\[dq = \sigma d\Sigma \quad \rightarrow \quad \vec{E}(x,y,z) = \frac{1}{4 \pi \varepsilon_0} \int_\Sigma \frac{\sigma d\Sigma}{r^2} \; \hat{u}\]
	\item[-] \textbf{Distribuzione volumetrica}: descritta da \(\rho\) che descrive la carica per unità di volume \(d\tau\).
	\[dq = \rho d\tau \quad \rightarrow \quad \vec{E}(x,y,z) = \frac{1}{4 \pi \varepsilon_0} \int_V \frac{\rho d\tau}{r^2} \; \hat{u}\]
\end{itemize}

\subsubsection*{Distribuzioni da sapere}
\begin{itemize}[topsep=3pt, itemsep=0pt]
	\item[-] campo generato lungo l'asse di un anello di raggio \(R\) con carica \(q\) uniformemente distribuita ad una distanza
	\(x\) dal centro dell'anello: \(\displaystyle \vec{E} = \frac{\lambda}{2 \varepsilon_0} \; \frac{R \, x}{(R^2 + x^2)^{3/2}} \ux\)
	\item[-] campo generato lungo l'asse di un disco di raggio \(R\) con carica \(q\) uniformemente distribuita ad una distanza
	\(x\) dal centro del disco: \(\displaystyle \vec{E} = \pm \frac{\sigma}{2 \varepsilon_0} \; \left(1- \frac{|x|}{\sqrt{R^2 + x^2}}\right) \ux\)
	\item[-] campo generato da un piano infinito con carica uniformemente distribuita con fattore \(\sigma\) ad una distanza \(x\):
	\(\displaystyle \vec{E} = \pm \frac{\sigma}{2 \varepsilon_0}\ux\)
\end{itemize}

\newpage

\subsection{Lavoro della forza elettrostatica}
Il lavoro compiuto dalla forza elettrostatica lungo un cammino \(AB\) è:
\[W_{AB} = \int_A^B \vec{F} \cdot d\vec{s} = \int_A^B q \vec{E} \cdot d\vec{s} = q \int_A^B \vec{E} \cdot d\vec{s} \quad (= - q (V_B - V_A))\]
Si osserva che la forza è conservativa, ovvero è indipendente dal cammino svolto e dipende solo dal punto iniziale \(A\) e dal
punto finale \(B\). La circuitazione del campo elettrostatico è sempre 0: \(\displaystyle \oint \vec{E} \cdot d\vec{s} = 0\)

\subsection{Potenziale elettrostatico ed energia potenziale elettrostatica}
\subsubsection*{Potenziale elettrostatico}
Siccome la forza è conservativa è possibile definire un potenziale elettrostatico:
\[V_B - V_A = - \int_A^B \vec{E} \cdot d\vec{s} \qquad \qquad V(r) = -\int_{+\infty}^r \vec{E} \cdot d\vec{s}, \text{ con } V(\infty) = 0 \qquad \qquad V = \left[V\right] = \left[ \frac{J}{C} \right]\]

\begin{itemize}
	\item[-] Il lavoro della forza elettrostatica diventa \(\displaystyle W_{AB} = q \int_A^B \vec{E} \cdot d\vec{s} = - q (V_B - V_A)\)
	\item[-] Il potenziale e il campo elettrico sono legati dalla relazione: \(\displaystyle E = - \nab V\)
\end{itemize}

\subsubsection*{Energia potenziale elettrostatica}
\begin{itemize}
	\item[-] L'energia potenziale elettrostatica di una carica \(q\) è definita come il lavoro compiuto dal campo elettrico \(E\) per spostare
	la carica da un punto \(r\) all'infinito: \(\displaystyle U_e = - W_{\infty,r} =  q \int_r^{\infty} \vec{E} \cdot d\vec{s} = qV\)
	
	\item[-] L'energia potenziale di una composizione di cariche è la somma dei lavori compiuti per spostare le varie cariche da \(+\infty\)
	alla loro rispettiva posizione: \(\displaystyle U_e = W_{ext} = \sum_{j>i} \frac{1}{4 \pi \varepsilon_0} \; \frac{q_i q_j}{r_{ij}} = \frac{1}{2} \sum_{i \neq j} \frac{1}{4 \pi \varepsilon_0} \; \frac{q_i q_j}{r_{ij}}\)
\end{itemize}

\subsubsection*{Legge di conservazione dell'energia}
Durante il moto di una partricella l'energia totale si conserva:
\[E_{TOT} = E_K + U_g + U_e = \frac{1}{2} mv^2 + mgh + qV = \text{costante}\]

\subsubsection*{Potenziali da sapere}
\begin{itemize}[topsep=3pt, itemsep=0pt]
	\item[-] potenziale generato da una distribuzione è \(\displaystyle V_{tot}(r) = -\int_{\infty}^r \vec{E} \cdot d\vec{s} = \frac{1}{4 \pi \varepsilon_0} \int \limits_\text{distribuzione} \frac{dq}{r}\)
	\item[-] potenziale generato da una carica ad una distanza \(r\) è \(\displaystyle V(r) = \frac{1}{4 \pi \varepsilon_0} \frac{q}{r}\)
	\item[-] potenziale lungo l'asse di un anello di raggio \(R\) con carica \(q\) uniformemente distribuita ad una distanza
	\(x\) dal centro dell'anello: \(\displaystyle V(x) = \frac{\lambda R}{2 \varepsilon_0 \; \sqrt{R^2 + x^2}}\)
	\item[-] potenziale lungo l'asse di un disco di raggio \(R\) con carica \(q\) uniformemente distribuita ad una distanza
	\(x\) dal centro del disco: \(\displaystyle V(x) = \pm \frac{\sigma}{2 \varepsilon_0} \; \left(\sqrt{R^2 + x^2} - x\right) \ux\)
	\item[-] potenziale generato da un piano infinito con carica uniformemente distribuita con fattore \(\sigma\) ad una distanza \(x\):
	\(\displaystyle V(x) = \pm \frac{\sigma}{2 \varepsilon_0} x\)
\end{itemize}

\subsection{Superfici equipotenziali}
Le superfici potenziali sono gli insiemi di punti in cui il potenziale è costante. Vengono utilizzate per dare una rappresentazione
grafica al potenziale in ogni punto del piano. Non forniscono l'intensità del campo, per un punto passa un'unica superficie e le
linee di forza sono ortogonali alle superfici.

\subsection{Dipolo elettrico}
\subsubsection*{Potenziale}
Il dipolo elettrico è costituito da due cariche puntiformi \(+q\) e \(-q\) a distanza \(a\). Il potenziale in un generico punto
\(P\) a distanza \(r_1\) da \(+q\) e \(r_2\) da \(-q\) vale 
\[V(P) = \frac{q}{4 \pi \varepsilon_0} \left(\frac{1}{r_1} - \frac{1}{r_2}\right) = \frac{q}{4 \pi \varepsilon_0} \frac{r_2 - r_1}{r_1 r_2}\]
Per \(r \gg a\) e definito l'angolo \(\theta\) tra \(\vec{r}\) e \(\vec{a}\) con \(\vec{a}\) diretto da \(-q\) a \(+q\) si ha:
\[V(P) = \frac{q \, a \cos\theta}{4 \pi \varepsilon_0 \, r^2} = \frac{q \, \vec{a} \cdot \hat{r}}{4 \pi \varepsilon_0 \, r^2}\]

\subsubsection*{Momento del dipolo elettrico}
Viene definito il momento del dipolo elettrico:
\[\vec{p} = q \, \vec{a} \qquad \rightarrow \qquad V(P) = \frac{\vec{p} \cdot \hat{r}}{4 \pi \varepsilon_0 \, r^2}\]

\subsubsection*{Campo elettrico generato da un dipolo}
Applicando la relazione \(E = - \nab V\) si ha che il campo generato da un dipolo è: 
\[\vec{E} = \frac{p}{4 \pi \varepsilon_0 \, r^3} (2 \cos \theta \; \hat{r} + \sin \theta \; \hat{\theta})\]

\begin{itemize}[topsep=3pt, itemsep=0pt]
	\item[-] lungo l'asse del dipolo (\(\theta = 0, \pi\)) il campo è \(\displaystyle \vec{E} = \frac{2 \vec{p}}{4 \pi \varepsilon_0 r^3}\)
	\item[-] lungo la perpendicolare all'asse del dipolo (\(\theta = \pi/2, 3\pi/2\)) il campo è \(\displaystyle \vec{E} = -\frac{\vec{p}}{4 \pi \varepsilon_0 r^3}\)
\end{itemize}

\subsubsection*{Dinamica del dipolo magnetico}
Si considera un dipolo immerso in un campo elettrico:
\begin{itemize}[topsep=3pt, itemsep=0pt]
	\item[-] sulle cariche agiscono due forze uguali e opposte \(\vec{F}_1 = -q\vec{E}\) e \(\vec{F}_2 = q\vec{E}\), la risultante
	delle forze è nulla per cui il centro di massa non si muove
	\item[-] il momento delle forze è \(\vec{M} = \vec{p} \times \vec{E} = -p E \sin \theta \hat{z}\), il dipolo ruota sotto azione
	del campo elettrico in modo da allineare il suo asse con l'orientamento del campo elettrico
\end{itemize}

\subsubsection*{Multipolo}
Generalizzando una distribuzione di cariche complessivamente neutra, si ha che il potenziale in un punto a distanza \(r\) con per
\(r \gg a\) dalla distribuzione è:
\[V(r) = \frac{q}{4 \pi \varepsilon_0 r} + \frac{\vec{p} \cdot \hat{r}}{4 \pi \varepsilon_0 r^2} + \frac{\hat{r} \cdot Q \hat{r}}{4 \pi \varepsilon_0 r^3} + \dots\]
\[\frac{q}{4 \pi \varepsilon_0 r} \rightarrow \text{carica singola} \qquad \frac{\vec{p} \cdot \hat{r}}{4 \pi \varepsilon_0 r^2} \rightarrow \text{dipolo} \qquad \frac{\hat{r} \cdot Q \hat{r}}{4 \pi \varepsilon_0 r^3} \rightarrow \text{quadrupolo} \qquad \dots\]

\newpage

\subsection{Flusso del campo elettrostatico}
Il flusso del campo elettrico attraverso una superficie infinitesima \(d\Sigma\) con vettore normale \(\un\) è:
\[d\Phi(\vec{E}) := \vec{E} \cdot \un \, d\Sigma\]
Si definisce quindi il flusso attraverso una superficie chiusa \(\Sigma\):
\[\Phi(\vec{E}) := \oint \vec{E} \cdot \un \, d\Sigma\]
Se il flusso è positivo, si dice che il flusso è uscente, se il flusso è negativo, il flusso è entrante.

\subsection{Teorema di Gauss}
\subsubsection*{Enunciato}
Il teorema di Gauss enuncia che il flusso del campo elettrico attraverso una superficie chiusa è pari al rapporto tra la carica
totale interna alla superficie e la costante \(\varepsilon_0\).
\[\Phi(\vec{E}) := \oint \vec{E} \cdot \un \, d\Sigma = \frac{q_\text{tot interna}}{\varepsilon_0}\]

\subsubsection*{Applicazioni}
\begin{itemize}[topsep=3pt, itemsep=0pt]
	\item[-] sfera di raggio \(R\) e densità superficiale uniforme \(\sigma\): \(\displaystyle \vec{E}_\text{ext} = \frac{R^2 \, \sigma}{r^2 \, \varepsilon_0} \; \hat{r} \qquad \vec{E}_\text{int} = 0\)
	\item[-] sfera di raggio \(R\) e densità volumetrica uniforme \(\tau\): \(\displaystyle \vec{E}_\text{ext} = \frac{R^3 \, \sigma}{3 r^2 \, \varepsilon_0} \; \hat{r} \qquad \vec{E}_\text{int} = \frac{\sigma \, r}{3 \, \varepsilon_0}\)
	\item[-] piano infinito con densità superficiale uniforme \(\sigma\): \(\displaystyle \vec{E} = \pm \frac{\sigma}{2 \varepsilon_0} \un\)
	\item[-] filo rettilineo con carica uniforme \(\lambda\): \(\displaystyle \vec{E} = \frac{\lambda}{2 \pi r \varepsilon_0} \hat{r}\)
\end{itemize}

\subsection{Forme integrali, forme differenziali, equazione di Poisson}
\begin{center}
	\def\arraystretch{2.5}
	\begin{tabular}{c c c}
		& \textbf{forme integrali} & \textbf{forme differenziabili} \\
		\toprule
		\textbf{potenziale} & \(\displaystyle \qquad V(r) = - \int_\infty^r \vec{E} \cdot d\vec{s} \qquad\) & \(\displaystyle \vec{E} = - \nab V\) \\[7pt]
		\hline
		\textbf{flusso} & \(\displaystyle \qquad \Phi(\vec{E}) = \oint \vec{E} \cdot d\vec{\Sigma} \qquad\) & \(\displaystyle \text{div} \, \vec{E} = \nab \cdot \vec{E} = \frac{\rho}{\varepsilon_0}\) \\[7pt]
		\hline
		\textbf{circuitazione} & \(\displaystyle \qquad \oint \vec{E} \cdot d\vec{s} \qquad\) & \(\displaystyle \text{rot} \, \vec{E} = \nab \times \vec{E} = \vec{0}\) \\[7pt]
		\bottomrule
		\bottomrule
		\textbf{eq. di Poisson} & \multicolumn{2}{c}{\(\displaystyle \nab^2 V = \nab \cdot \nab V = - \frac{\rho}{\varepsilon_0}\)} \\
		\bottomrule
	\end{tabular}
\end{center}
Le seguenti tre proprietà sono equivalenti:
\[\vec{E} \; \text{capmo conservativo} \quad \Leftrightarrow \quad \int_C \vec{E} \cdot d\vec{s} \; \text{indipendente da} \; s \quad \Leftrightarrow \quad \vec{E} = - \nab V \quad \Leftrightarrow \quad \nab \times \vec{E} = \vec{0}\]

\newpage

\subsection{Conduttori in equilibrio}
\subsubsection*{Definizione}
I conduttori sono materiali con determinate proprietà che permettono la mobilità dei portatori di carica (il moto delle cariche
che li costituiscono).

\subsubsection*{Campo elettrico all'interno del conduttore}
In condizioni statiche e di equilibrio, non c'è movimento delle cariche e si ha che \(\vec{E} = 0\), ovvero non c'è forza che
mette in moto le cariche. Per questo motivo all'interno del conduttore (se non c'è movimento di cariche), il campo elettrico
interno è nullo \(\vec{E}_\text{int} = 0\) indipendentemente dal campo esterno. Questo implica che:
\begin{itemize}[topsep=3pt, itemsep=0pt]
	\item[-] l'eccesso di carica si trova solo sulla superficie del conduttore
	\item[-] il potenziale è costante sul conduttore (è una superficie equipotenziale)
	\item[-] il campo elettrico nelle vicinanze della superficie è \(\vec{E} = \sigma / \varepsilon_0 \; \un\)
	\item[-] la densità di carica superficiale è maggiore dove il raggio di curvatura è minore (effetto ago)
\end{itemize}

\subsubsection*{Conduttore cavo}
Un conduttore cavo è un coduttore al cui interno è presente una cavità. Valgono le seguenti proprietà:
\begin{itemize}[topsep=3pt, itemsep=0pt]
	\item[-] il campo all'interno della cavità è nullo (T. di Gauss)
	\item[-] il campo all'interno del conduttore è nullo (proprietà conduttore)
	\item[-] sulle pareti delle cavità non ci sono cariche elettriche (altrimenti ci si contraddice con i punti sopra)
	\item[-] la carica si distribuisce solo e soltanto sulla superficie esterna (conseguenza)
	\item[-] il campo all'interno del conduttore e delle cavità è nullo (conseguenza)
	\item[-] il potenziale è costante all'interno del conduttore e delle cavità (conseguenza)
	\item[-] il conduttore cavo costituisce uno schermo elettrostatico perfetto
\end{itemize}

\subsubsection*{Induzione completa}
Un conduttore carico si trova all'interno della cavità di un conduttore cavo. Si hanno le seguenti proprietà:
\begin{itemize}[topsep=3pt, itemsep=0pt]
	\item[-] all'interno del conduttore interno e del conduttore esterno il campo è nullo, nella cavità il campo è generato dalla
	carica superficiale del conduttore interno
	\item[-] la carica del conduttore interno induce la presenza di una carica uguale e opposta sulla superficie della cavità nel
	conduttore esterno che a sua volta induce una carica uguale alla prima sulla superficie esterna del conduttore esterno
	\item[-] l'oggetto si comporta come se ci fosse soltalto la carica distribuita sulla superficie esterna del conduttore esterno,
	ovvero si ha un perfetto schermo elettrostatico.
\end{itemize}

\newpage

\subsection{Condensatori}
\subsubsection*{Struttura}
Il condensatore è un componente elettronico costituito da due conduttori (armature) separati da un mezzo isolante (dielettrico).
È in grado di immagazzinare energia elettrostatica attraverso l'accumulo delle cariche sulle armature e di rilasciarla. Si
definisce la capacità di un condensatore come il rapporto tra la carica sulle armature e la differenza di potenziale:
\[C = \frac{q}{\Delta V} \qquad \qquad q = C \, \Delta V \qquad \qquad \Delta V = \frac{q}{C} \qquad \qquad [C] = \frac{C_\text{oulomb}}{V_\text{olt}} = F_\text{araday}\]

\subsubsection*{Condensatore sferico}
Un condesatore sferico è costituito da una sfera di raggio \(R_1\) contenuta all'interno della cavità di una sfera cava con raggio
della cavità \(R_2\). Le due sfere costituiscono le due armature con distanza \(h = R_2 - R_1\). Nello spazio tra le armature si ha:
\[\vec{E}(r) = \frac{q}{4 \pi \varepsilon_0 \, r^2} \; \hat{r} \qquad \quad \Delta V = \frac{q}{4 \pi \varepsilon_0} \, \left(\frac{1}{R_1} - \frac{1}{R_2}\right) \qquad C = \varepsilon_0 \, \frac{4 \pi R_1 R_2}{h} \quad C_{R_2 \approx R_1} \approx \varepsilon_0 \, \frac{4 \pi R^2}{h} = \frac{\varepsilon_0 \, \Sigma}{h}\]

\subsubsection*{Condensatore cilindrico}
Un condensatore cilindrico è costituito da un cilindro interno di raggio \(R_1\) circondato da un cilindro cavo esterno con raggio
della cavità \(R_2\). Le armature sono costituite dai due cilindri concentrici con \(h = R_2 - R_1\) e altezza \(d\). Tra le armature vale:
\[\vec{E} = \frac{q}{2 \pi r d \, \varepsilon_0} \; \hat{r} \qquad \quad \Delta V = \frac{q}{2 \pi d \, \varepsilon_0} \, \log \left(\frac{R_2}{R_1}\right) \qquad C = \varepsilon_0 \, \frac{2 \pi d}{\log \left(R_2 / R_1\right)} \quad C_{R_2 \approx R_1} \approx \varepsilon_0 \, \frac{2 \pi d R}{h} = \frac{\varepsilon_0 \, \Sigma}{h}\]

\subsubsection*{Condensatore piano}
Un condensatore piano è costituito da due piani (infiniti) che costituiscono le due armature ad una distanza costante \(h\). Tra le armature si ha:
\[\vec{E} = \frac{\sigma}{\varepsilon_0} \; \un \qquad \qquad \Delta V = \frac{\sigma \, h}{\varepsilon_0} \qquad \qquad C = - \frac{\varepsilon_0 \, \Sigma}{h}\]

\subsubsection*{Condensatori in serie}
La carica sulle armature è costante per tutti i condensatori \(q_\text{tot} = q_1 = q_2 = \dots = q_n\) \\
La differenza di potenziale totale è la somma delle ddp di ogni condensatore \(\Delta V_\text{tot} = \Delta V_1 + \Delta V_2 + \dots + \Delta V_n\) \\
La capacità complessiva di \(n\) condensatori collegati in serie è:
\[\frac{1}{C_\text{tot}} = \frac{1}{C_1} + \frac{1}{C_2} + \dots + \frac{1}{C_n}\]

\subsubsection*{Condensatori in parallelo}
La differenza di potenziale è costante per tutti i condensatori \(\Delta V_\text{tot} = \Delta V_1 = \Delta V_2 = \dots = \Delta V_n\) \\
La carica totale è la somma delle cariche di ogni condensatore \(q_\text{tot} = q_1 + q_2 + \dots + q_n\) \\
La capacità complessiva di \(n\) condensatori collegati in parallelo è:
\[C_\text{tot} = C_1 + C_2 + \dots + C_n\]

\newpage

\subsection{Energia del campo elettrostatico}
\subsubsection*{Definizione}
L'energia del campo elettrostatico (generato da un condensatore) è pari al lavoro necessario a caricare il condensatore che genera
il campo:
\begin{align*}
	dW &= \Delta V dq = \frac{q}{C} dq \quad \rightarrow \quad W = \int_0^{q} \frac{q'}{C} \; dq' = \frac{1}{2} \frac{q^2}{C} = \frac{1}{2} C \Delta V^2 = \frac{1}{2} q \Delta V \\
	U_e &= W = \frac{1}{2} C \Delta V^2 = \frac{\varepsilon_0}{2} \, E^2 \, \Sigma \, h = \frac{\varepsilon_0}{2} \, E^2 \, \tau \quad \text{con} \; C = \frac{\varepsilon_0 \, \Sigma}{h}, \; \Delta V = E \, h, \; \tau = \Sigma \, h
\end{align*}
Si ottiene che l'energia del campo elettrostatico è:
\[dU_e = \frac{\varepsilon_0}{2} \, E^2 \, d\tau \qquad U_e = \frac{\varepsilon_0}{2} \int \limits_\text{distribuzione} E^2 \; d\tau\]

\subsection{Dielettrici}
\subsubsection*{Definizione}
Un dielettrico è un materiale isolante.

\subsubsection*{Condensatore di Epino e costante dielettrica}
Il condensatore di Epino è un condensatore in cui è possibile modificare la distanza tra le armature e inserire materiali
dielettrici tra le armature. Si osserva che:
\begin{itemize}[topsep=3pt, itemsep=0pt]
	\item[-] allontanando le armature, aumenta la differenza di potenziale e diminuisce la capacità, viceversa avvicinando le
	armature diminuisce la differenza di potenziale e aumenta la capacità
	\item[-] se viene inserito un conduttore tra le armature:
	\begin{itemize}[topsep=0pt, itemsep=0pt]
		\item[-] si ha induzione eletrostatica completa
		\item[-] il campo all'interno del conduttore è nullo
		\item[-] il potenziale si riduce, come se lo spessore del conduttore fa ridurre l'\(h\) e di conseguenza diminuisce
		il potenziale e aumenta la capacità
	\end{itemize}
	\item[-] se viene inserito un dielettrico tra le armature:
	\begin{itemize}[topsep=0pt, itemsep=0pt]
		\item[-] si ha un fenomeno di polarizzazione del dielettrico
		\item[-] il potenziale si riduce perché varia la costate dielettrica del mezzo
	\end{itemize}
\end{itemize}

\subsubsection*{Costante dielettrica relativa}
Viene definita la costante dielettrica relativa \(\kappa\) del dielettrico come il rapporto tra la differenza di potenziale in
assenza e  in presenza del dielettrico tra le armature. Si definisce anche la suscettività elettrica del dielettrico \(\chi\):
\[\kappa := \frac{V_0}{V_\kappa} > 1 \qquad \qquad \chi = \kappa - 1 > 0\]
Si ha che:
\[E_\kappa = \frac{V_\kappa}{h} = \frac{V_0}{\kappa h} = \frac{E_0}{\kappa} = \frac{\sigma_0}{\kappa \varepsilon_0} = \frac{\sigma_0}{\varepsilon_\kappa} = \frac{\sigma_0}{\varepsilon_0} - \frac{\sigma_p}{\varepsilon_0} = \frac{\sigma_0}{\varepsilon_0} - \frac{\chi \sigma_0}{\kappa \varepsilon_0} \quad
\text{con} \; \sigma_p = \frac{\kappa-1}{\kappa} \sigma_0 = \frac{\chi}{\chi+1}\sigma_0\]
Si osserva all'interno del dielettrico si crea un campo con modulo \(\displaystyle E_\text{int} = \frac{\chi \sigma_0}{\kappa \varepsilon_0}\)
che si oppone al campo esterno. Questo è dovuto alla polarizzazione del dielettrico che fa \say{accumulare} cariche alle estremità
del materiale isolante. \\[10pt]
La costante dielettrica del mezzo (\(\neq\) vuoto) diventa: \[\varepsilon_\kappa = \kappa \,  \varepsilon_0 > \varepsilon_0\]

\subsubsection*{Polarizzazione del dielettrico}
In presenza di un campo elettrico esterno, all'interno del dielettrico il nucleo degli atomi si sposta in direzione concorde
con il campo elettrico, mentre gli elettroni si concentrano nella parte opposta. In questo modo di formano tanti dipoli con
momento \(p_a = Z_\text{\# protoni} \cdot e_\text{carica elem.} \cdot \chi_\text{spostamento del nucleo}\), come nei materiali
costituiti da molecole polari (es. acqua). Si definisce il vettore di polarizzazione:
\[\vec{p} = \varepsilon_0 \, \chi_\text{suscettività} \, \vec{E} \qquad \qquad \vec{p} = \frac{\vec{p}_\text{tot}}{\tau} = \frac{N}{\tau} \langle \vec{p} \rangle  = n \langle \vec{p} \rangle  \qquad n = \text{\# dipoli per volume}\]
In assenza di campo elettrico esterno, il momento di dipolo totale è nullo. In presenza di un campo elettrico esterno, i momenti
si allineano e formano un momento totale risultante non nullo e concorde con il campo esterno. Lungo 

\subsubsection*{Capacità di condensatori con dielettrico}
La formula della capacità di condensatori con un dielettrico tra le armature aumenta di un fattore \(\kappa\):
\[C_\kappa = \frac{q}{\Delta V_\kappa} = \kappa \, \frac{q}{\Delta V_0} = \kappa \, C_0\]

\subsection{Materiali conduttori, modello di Drude e corrente elettrica}
\subsubsection*{Moto degli elettroni - elettroni liberi}
Nei materiali conduttori, la presenza di elettroni liberi permette la formazione di correnti di elettroni che si spostano da una
zona con potenziale minore ad un'altra con potenziale maggiore (verso opposto al campo elettrico).

\subsubsection*{Modello di Drude}
Nel modello di Drude di immagina che gli elettroni liberi viaggino in moto disordinato rimbalzando tra i cationi del conduttore.
\begin{itemize}[topsep=3pt, itemsep=0pt]
	\item[-] In assenza di campo elettrico, la direzione dopo gli urti è causale e la velocità media degli elettroni è nulla
	\(v_m = \frac{1}{N} \, \sum v_i = 0\), con \(\tau\) tempo medio tra gli urti
	\item[-] In presenza di un campo elettrico, gli elettroni subiscono una accelerazione dovuta alla forza elettrica
	\(\vec{a} = \frac{\vec{F}}{m} = - \frac{e \vec{E}}{m}\) per cui avranno una velocità di deriva \(v_d = v_m + a t = - e \, \vec{E} \, \tau / m \approx \text{costante}\)
\end{itemize}

\subsubsection*{Corrente elettrica e densità di corrente elettrica}
È possibile definire la corrente elettrica come la quantità di carica che attraversa una determinata superficie in un'unità di tempo:
\[i = \lim_{\Delta t \to 0} \frac{\Delta q}{\Delta t} = \frac{dq}{dt}\]
Sapendo che \(\Delta q = n \, (-e) \, v_d \, \Delta t \, \Sigma \cos \Theta\) con \(n =\) \# elettroni per unità di volume, si ha
\(di = n \, v_d \, d\Sigma \cos \theta\).
Si può definire la densità di corrente (corrente per unità di superficie):
\[\vec{j} = n (-e) \vec{v_d} \qquad \qquad i = j \Sigma, \quad j = i / \Sigma\]

\newpage

\subsection{Legge di Ohm}
Unendo le formule della densità di corrente \(\vec{j}\) e della velocità di deriva \(v_d\) si ha:
\[\vec{j} = \frac{n e^2 \tau}{m} \vec{E} \quad \rightarrow \quad \begin{matrix}
	\vec{j} = \sigma \vec{E} \;\; \text{con} \; \sigma = \frac{n e^2 \tau}{m} \;\; \text{conduttività del materiale} \\[10pt]
	\vec{E} = \rho \vec{j} \;\; \text{con} \; \rho = \frac{1}{\sigma} \;\; \text{resistività del materiale} \qquad \quad \,\,
\end{matrix}\]
In situazioni stazionarie la corrente è costante in ogni sezione del conduttore, per cui è possibile riscrivere la formula sopra
in funzione della corrente \(i\):
\[\vec{E} = \rho \vec{j} \quad \rightarrow \quad \vec{E} = \frac{\rho}{\Sigma} \vec{i}\]
Riscrivendo il campo attraverso il potenziale \(V = E \, h\) si ottiene
\[\vec{E} = \frac{\rho}{\Sigma} \vec{i} \quad \rightarrow \quad V = \frac{\rho \, h}{\Sigma} i  \quad \rightarrow \quad V = R \, i \;\; \text{con} \; R = \frac{\rho \, h}{\Sigma} \;\; \text{resistenza del materiale}\]

\subsection{Resistenze}
La resistenza è un componente elettronico costituito da un materiale in grado di ostacolare/attenuare l'intensità della corrente
che circola in un circuito. La relazione tra \(i\), \(V\), \(R\) è data dalla legge di Ohm \[V = R \, i\]
Il valore della resistenza varia in base alla temperatura seconodo la legge: \(R = R_{20} (1 + \alpha \Delta T)\), con la costante
\(\alpha\) propria di ogni materiale.

\subsubsection*{Resistenze in serie}
L'intensità di corrente è costante su tutte le resistenze \(i_{tot} = i_1 = i_2 = \dots = i_n\) \\
La differenza di potenziale totale è la somma delle ddp \(\Delta V_\text{tot} = \Delta V_1 + \Delta V_2 + \dots + \Delta V_n\) \\
La resistenza complessiva di \(n\) resistenze collegate in serie è:
\[R_\text{tot} = R_1 + R_2 + \dots + R_n\]

\subsubsection*{Resistenze in parallelo}
L'intensità di corrente è è la somma delle correnti di ogni resistenza \(i_{tot} = i_1 + i_2 + \dots + i_n\) \\
La differenza di potenziale è costante per tutte le resistenze \(\Delta V_\text{tot} = \Delta V_1 = \Delta V_2 = \dots = \Delta V_n\) \\
La resistenza complessiva di \(n\) resistenze collegate in parallelo è:
\[\frac{1}{R_\text{tot}} = \frac{1}{R_1}+ \frac{1}{R_2} + \dots + \frac{1}{R_n}\]

\subsubsection*{Potenza dissipata da una resistenza}
La potenza dissipata da una resistenza sottoforma di calore per effetto Joule è:
\[P_\text{dissipata} = R \, i^2 = V \, i = \frac{V^2}{R}\]

\newpage

\subsection{Generatori di forza elettromotrice}
\subsubsection*{Definizione}
Un generatore di forza elettromotrice è un dispositivo in grado di mantenere una differenza di potenziale costante per un determinato
periodo di tempo. Il primo generatore inventato è stato la pila di Alessandro Volta (pila voltaica) che trasforma energia chimica in
energia elettrica.

\subsubsection*{Forza elettromotrice}
In un circuito con corrente che circola, il campo elettrico è costituito da un campo conservativo \(E\) e da un altro non conservativo
\(E*\) originato da un generatore (generatore chimico, induzione elettromagnetica). La circuitazione sul circuito è pari alla
differenza di potenziale creata dal generatore di forza elettromotrice:
\[\mathcal{E}_\text{FEM} := \oint \vec{E} \cdot d\vec{s} = V_A - V_B = \Delta V = R \, i\]
Nei generatori reali è presente una resistenza interna \(r\) al generatore, per cui si ha:
\[\mathcal{E} - r \, i = R \, i \quad \rightarrow \quad \Delta V = (R + r) i\]

\subsubsection*{Potenza erogata - dissipata}
La potenza erogata da un generatore è: \(\qquad P_\text{erogata} = \mathcal{E} i\) \\
Sostituendo \(\mathcal{E} = (R + r) \, i\) si ottiene che la potenza erogata viene interamente dissipata dalla resistenza:
\[P_\text{erogata} = \mathcal{E} i = R_\text{tot} i^2 = P_\text{dissipata}\]

\subsection{Circuiti e leggi di Kirchhoff}
\subsubsection*{Leggi di Kirchhoff}
\begin{itemize}
	\item[-] \textbf{Legge dei nodi}: \\
	La somma algebrica delle correnti che confluiscono in un nodo è nulla, le correnti entranti hanno segno	positivo, le correnti
	uscenti hanno segno negativo, \[\sum_k i_k = 0\]
	\item[-] \textbf{Legge delle maglie}: \\
	La soma algebrica delle forze elettromotrici presenti nei rami è uguale alla somma dei prodotti delle differenze di potenziale
	ai capi dei resistori nei rami. Fissato un verso di percorrenza della maglia, se la corrente è concorde avrà segno positivo
	altrimenti ha segno negativo. La sorgente di FEM ha segno positivo se è percorsa da polo negativo al positivo, altrimenti avrà
	segno negativo. \[\sum_k \mathcal{E}_k = \sum_k R_k i_k\]
\end{itemize}

\newpage

\subsubsection*{Circuito RC in carica}
\begin{itemize}
	\item[-] in un circuito RC aperto con condensatore carico si ha che il condensatore ha immagazzinato \(U_e = {q_0}^2 / 2C\) e
	ha una differenza di potenziale tra le armature \(V_0 = q_0/C\)
	\item[-] alla chiusura del circuito inizia a circolare corrente \(i = - dq/dt\) per annullare le cariche presenti nelle armature
	del condensatore
	\item[-] dalle equazioni sopra e dalle leggi di Kirchhoff si ottiene che:
	\begin{align*}
		&1. \quad V_C = \frac{q}{C} = V_R = R \, i \;\;\rightarrow\;\; i = \frac{q}{RC} \\
		&2. \quad i = - \frac{dq}{dt} = \frac{q}{RC} \;\;\rightarrow\;\; \frac{dq}{q} = \frac{dt}{RC} \\
		&3. \quad \int_q^0 \frac{dq}{q} = - \int_0^t \frac{dt}{RC} \;\;\rightarrow\;\; \ln(q/q_0) = -\frac{t}{RC} \;\;\stackrel{\tau = RC}{\rightarrow}\;\; q(t) = q_0 \, e^{-t/\tau} \\
		&4. \quad V(t) = \frac{q(t)}{C} = \frac{q_0}{C} e^{-t/\tau} = V_0 \, e^{t/\tau} = V_\text{max} \, e^{t/\tau} \\
		&5. \quad i(t) = -\frac{dq}{dt} = \frac{q_0}{RC}e^{-t/\tau} = i_\text{max} \, e^{-t/\tau} \qquad\qquad\qquad\qquad\qquad\qquad\qquad\qquad\qquad
	\end{align*}
	\item[-] per cui in un condensatore in scarica si hanno le seguenti equazioni
	\[q(t) = q_0 \, e^{-t/\tau} \qquad \qquad V(t) = V_0 \, e^{-t/\tau} \qquad \qquad i(t) = \frac{q_0}{RC} \, e^{-t/\tau} \qquad \qquad \tau = RC\]
	\item[-] l'energia immgazzinata dal condensatore viene tutta dissipata dalla resistenza
\end{itemize}

\subsubsection*{Circuito RC in scarica}
\begin{itemize}
	\item[-] in un circuito RC aperto con condensatore scarico e con un generatore di FEM, l'energia immagazinata dal
	condensatore è nulla
	\item[-] alla chiusura del circuito inizia a circolare corrente \(i = dq/dt\) che deposita cariche sulle armature del
	condensatore, accumulando energia, fino a \(q_0 = C \mathcal{E}\)
	\item[-] dalle equazioni sopra e dalle leggi di Kirchhoff si ottiene che:
	\begin{align*}
		&1. \quad \mathcal{E} = V_R(t) + V_C(t) \;\;\rightarrow\;\; \mathcal{E} = Ri(t) + \frac{q(t)}{C} \\
		&2. \quad i = \frac{dq}{dt} \;\;\rightarrow\;\; R \frac{dq}{dt} = \mathcal{E} -\frac{q}{C} \;\;\rightarrow\;\; \frac{dq}{q - C\mathcal{E}} = - \frac{dt}{RC} \\
		&3. \quad \frac{dq}{q - C\mathcal{E}} = - \frac{dt}{RC} \;\;\rightarrow\;\; \int_0^q \frac{dq}{q - C\mathcal{E}} = -\int_0^t \frac{dt}{RC} \;\;\rightarrow\;\; \ln \frac{q - C\mathcal{E}}{-C\mathcal{E}} = -\frac{t}{RC} \\
		&4. \quad \ln \frac{q - C\mathcal{E}}{-C\mathcal{E}} = -\frac{t}{RC}  \;\;\rightarrow\;\; q(t) = C \mathcal{E} (1-e^{-t/\tau}) = q_\text{tot} (1-e^{-t/\tau}) \\
		&5. \quad V(t) = \frac{q(t)}{C} = \mathcal{E} (1-e^{-t/\tau}) =  V_\text{max} (1-e^{-t/\tau}) \\
		&6. \quad i = \frac{dq}{dt} = \frac{\mathcal{E}}{R} (1-e^{-t/\tau}) = i_\text{max} (1-e^{-t/\tau}) \qquad\qquad\qquad\qquad\qquad\qquad\qquad\qquad
	\end{align*}
	\item[-] per cui in un condensatore in scarica si hanno le seguenti equazioni
	\[q(t) = C \mathcal{E} \, (1-e^{-t/\tau}) \qquad \qquad V(t) = \mathcal{E} \, (1-e^{-t/\tau}) \qquad \qquad i(t) = \frac{\mathcal{E}}{R} \, (1-e^{-t/\tau}) \qquad \qquad \tau = RC\]
	\item[-] l'energia fornita dal generatore viene per metà immagazzinata nel condensatore e per metà dissipata nella resistenza
	\item[-] si immagina che esista una corrente di spostamento tra le armature del condenstore, oltre a quella di conduzione che
	circola all'interno del circuito:
	\[i_\text{spostamento} = \varepsilon_0 \dt \Phi(E) \qquad i = i_\text{conduzione} + i_\text{spostamento} = i_\text{conduzione} + \varepsilon_0 \dt \Phi(E)\]
\end{itemize}

\end{document}
