\documentclass[a4paper]{article}
\usepackage[utf8]{inputenc} % standard unicode
\usepackage[italian]{babel} % corretta sillabazione in italiano
\usepackage{geometry} % per impostare margini e layout pagina
\usepackage{amssymb} % per l'ambiente matematico
\usepackage{amsmath} % per l'ambiente matematico
\usepackage{enumitem} % per elenchi puntati
\usepackage{multirow} % per celle che si espandono su più righe
\usepackage{tabularx} % per tabelle con larghezza flessibile
\usepackage{booktabs} % per linee orizzontali tabelle
\usepackage{hyperref} % per collegamenti
\usepackage{graphicx} % per immagini
\usepackage{listings} % per codice
\usepackage{xcolor} % per colori nel codice
\usepackage{dirtytalk} % per le ""

% per margini
\geometry{a4paper,left=25mm, right=25mm, bottom=25mm, top=30mm}

% per centrare testo nelle tabelleX
\renewcommand\tabularxcolumn[1]{m{#1}}

% percorso delle immagini da inserire
\graphicspath{ {./ } }

% parte funzione reale e parte immaginaria
\newcommand\Real{\text{Re}}
\newcommand\Img{\text{Im}}

% versori
\newcommand\ux{\vec{u}_x}
\newcommand\uy{\vec{u}_y}
\newcommand\uz{\vec{u}_z}
\newcommand\uxp{\vec{u}_{x'}}
\newcommand\uyp{\vec{u}_{y'}}
\newcommand\uzp{\vec{u}_{z'}}
\newcommand\ur{\vec{u}_r}
\newcommand\uv{\vec{u}_v}
\newcommand\un{\vec{u}_n}
\newcommand\ug{\vec{u}_\gamma}
\newcommand\uper{\vec{u}_\perp}
\newcommand\upar{\vec{u}_\parallel}
\newcommand\nab{\vec{\nabla}} % nabla

% forza generica
\newcommand\ft{\vec{F}\left(\vec{\gamma}(t), \; \dt \vec{\gamma}(t), \; t\right)}
\newcommand\ftau{\vec{F}\left(\vec{\gamma}(\tau), \; \dtau \vec{\gamma}(\tau), \; \tau\right)}

% derivata
\newcommand\dt{\frac{d}{dt}\,}
\newcommand\dtau{\frac{d}{d\tau}\,}
\newcommand\dts{\frac{d^2}{dt^2}\,}

% modulo vettore
\newcommand\vmod[1]{\left|\left|{#1}\right|\right|}

\title{Appunti di Elementi di Fisica 2}
\author{Giacomo Simonetto}
\date{Secondo semestre 2023-24}

\begin{document}

% -------------------------------------- Copertina e indice ---------------------------------------
\maketitle
\begin{abstract}
	Appunti del corso di Elementi di Fisica 2 - (Elettromagnetismo) della facoltà di Ingegneria Informatica dell'Università di Padova.
\end{abstract}

\newpage

\tableofcontents

\newpage

\section{Elettrostatica ed elettrodinamica}
\subsection{Introduzione}
L'elettromagnetismo è una teoria descrittiva costruita su solide basi sperimentali, ovvero non si utilizzano principi primi, ma
si parte dalle osservazioni di fenomeni. I principali scienziati che si sono occupati del fenomeno sono Coulomb, Ampère, Faraday
e Maxwell. La teoria classica dell'elettromagnetismo è stata usata come base per la relatività e le correzioni quantistiche sono
insignificanti per distanze inferiori a \(10^{-12} \; m\), ovvero 100 volte più piccole dell'atomo.

\subsection{Carica elettrica}
\subsubsection*{Definizione}
La carica elettrica è una grandezza fisica collegata al bilanciamento di elettroni e protoni nella materia. Si misura in Coulomb
e ha la proprietà poter essere sia positiva che negativa. Corpi di cariche con segno opposto si attraggono, corpi di cariche con
stesso segno si respingono.

\subsubsection*{Legge di conservazione della carica}
In un sistema isolato, la carica totale non cambia.Nei decadimenti radioattivi, quando viene emesso un elettrone, si emette anche
un positrone (per bilanciare la carica). L'universo appare come una miscela bilanciata di cariche elettriche.

\subsubsection*{Quantizzazione della carica}
La misura della carica è costituita da multipli della carica elementare \(e = 1.6022 \cdot 10^{-19} C\). Le particelle elementari
cariche hanno tutte la stessa carica \(e\):
\begin{center}
	\begin{tabular}{c c c}
		\textbf{Particella} & \textbf{Carica} & \textbf{Massa} \\
		\toprule
		Elettrone & \(-1.6022 \cdot 10^{-19} C\) & \(9.1094 \cdot 10^{-31} kg\) \\
		\midrule
		Protone & \(1.6022 \cdot 10^{-19} C\) & \(1.6726 \cdot 10^{27} kg\) \\
		\bottomrule
	\end{tabular}
\end{center}

\subsection{Legge di Coulomb}
Due cariche elettriche si respingono con una forza proporzionale al prodotto delle intensità delle cariche e inversamente
proporzionale al quadrato della loro distanza.
\[\vec{F}_{21} = \frac{1}{4 \pi \varepsilon_0} \cdot \frac{q_1 \, q_2}{r^2} \cdot \hat{r}_{2,1} \qquad \qquad \begin{matrix}
	\vec{F}_{21} \rightarrow \vec{F}_\text{ della particella 1 sulla particella 2}  \qquad \qquad \qquad \qquad \qquad \;\;\;\; \\
	\varepsilon_0 \rightarrow \text{costante dielettrica nel vuoto}, \varepsilon_0 = 8,854 \cdot 10^{-12} \frac{C^2}{Nm ^2}
\end{matrix}\]
La forza è newtoniana \(\vec{F}_{21} = -\vec{F}_{12}\) ed è additiva \(\vec{F}_{tot} = \sum_{i=2}^{\# \textbf{cariche}} \vec{F}_{1i}\). \\
In un atomo, la forza di Coulomb tra elettroni e protoni è molto più forte della forza gravitazionale, per cui quest'ultima è
trascurabile. Nell'universo, invece, si ha prevalentemente forza gravitazionale siccome i corpi celesti sono genericamente neutri.

\subsection{Esperimenti di elettrostatica}
\subsubsection*{Primo esperimento: bacchette di plexiglas ed ebanite}
Caricando per strofinio le bacchette di ebanite e di plexiglas, si nota che due bacchette di uno stesso materiale si respingono,
mentre due bacchette di uno stesso materiale si attraggono. Il plexiglas si carica negativamente, l'ebanite si carica positivamente.

\subsubsection*{Secondo esperimento: elettroscopio a foglie}
Quando si avvicina una bacchetta carica al pomello dell'elettroscopio, le foglie si allontanano per fenomeno dell'induzione
elettrostatica, ovvero si verifica una redistribuzione delle cariche all'interno di un conduttore in presenza di un altro corpo
carico nelle vicinanze.

\subsection{Campo elettrico}
\subsubsection*{Definizione}
Il campo elettrico si definisce come \(\vec{E} := \vec{F}_{tot,0}/q_0\) dove \(q_0\) è detta carica di prova e \(\vec{F}_0\) è
la forza totale agente sulla carica di prova.
\[\vec{E}(x,y,z) = \sum_{i} \frac{1}{4 \pi \varepsilon_0} \; \frac{q_i}{r_{0i}^2} \; \hat{r}_{0i} \qquad \qquad \vec{E} = \vec{F}_{tot,0}/q_0 \qquad \vec{F}_{tot,0} = q_0 \vec{E} \qquad \qquad \vec{E} = \left[ \frac{N}{C} \right] = \left[ \frac{V}{M} \right]\]
Siccome la forza è additiva, anche il campo elettrico è additivo.

\subsubsection*{Linee del campo elettrostatico}
Le linee del campo elettrostatico permettono una rappresentazione grafica complessiva del campo elettrico nello spazio. Le linee
di campo sono curve con le seguenti caratteristiche:
\begin{itemize}[topsep=3pt, itemsep=0pt]
	\item[-] in ogni punto sono tangenti ai vettori del campo
	\item[-] il campo è più intenso dove c'è maggiore densità
	\item[-] le linee non si incrociano mai
	\item[-] si originano dalle cariche positive e si chiudono nelle cariche negative, se è presente solo una carica, si chiudono
	all'infinito.
\end{itemize}

\subsection{Distribuzione continua di carica}
\subsubsection*{Tipologie di distribuzione}
Il campo elettrico generato da una distribuzione continua di cariche vale:
\[\vec{E}(x,y,z) = \frac{1}{4 \pi \varepsilon_0} \int \limits_\text{distribuzione} \frac{dq}{r^2} \; \hat{u}\]
\begin{itemize}[topsep=3pt, itemsep=-7pt]
	\item[-] \textbf{Distribuzione lineare}: descritta da \(\lambda\) che descrive la carica per unità di lunghezza \(dl\).
	\[dq = \lambda dl \quad \rightarrow \quad \vec{E}(x,y,z) = \frac{1}{4 \pi \varepsilon_0} \int_L \frac{\lambda dl}{r^2} \; \hat{u}\]
	\item[-] \textbf{Distribuzione superficiale}: descritta da \(\sigma\) che descrive la carica per unità di superficie \(d\Sigma\).
	\[dq = \sigma d\Sigma \quad \rightarrow \quad \vec{E}(x,y,z) = \frac{1}{4 \pi \varepsilon_0} \int_\Sigma \frac{\sigma d\Sigma}{r^2} \; \hat{u}\]
	\item[-] \textbf{Distribuzione volumetrica}: descritta da \(\rho\) che descrive la carica per unità di volume \(d\tau\).
	\[dq = \rho d\tau \quad \rightarrow \quad \vec{E}(x,y,z) = \frac{1}{4 \pi \varepsilon_0} \int_V \frac{\rho d\tau}{r^2} \; \hat{u}\]
\end{itemize}

\subsubsection*{Distribuzioni da sapere}
\begin{itemize}[topsep=3pt, itemsep=0pt]
	\item[-] campo generato lungo l'asse di un anello di raggio \(R\) con carica \(q\) uniformemente distribuita ad una distanza
	\(x\) dal centro dell'anello: \(\displaystyle \vec{E} = \frac{\lambda}{2 \varepsilon_0} \; \frac{R \, x}{(R^2 + x^2)^{3/2}} \ux\)
	\item[-] campo generato lungo l'asse di un disco di raggio \(R\) con carica \(q\) uniformemente distribuita ad una distanza
	\(x\) dal centro del disco: \(\displaystyle \vec{E} = \pm \frac{\sigma}{2 \varepsilon_0} \; \left(1- \frac{|x|}{\sqrt{R^2 + x^2}}\right) \ux\)
	\item[-] campo generato da un piano infinito con carica uniformemente distribuita con fattore \(\sigma\) ad una distanza \(x\):
	\(\displaystyle \vec{E} = \pm \frac{\sigma}{2 \varepsilon_0}\ux\)
\end{itemize}

\newpage

\subsection{Lavoro della forza elettrostatica}
Il lavoro compiuto dalla forza elettrostatica lungo un cammino \(AB\) è:
\[W_{AB} = \int_A^B \vec{F} \cdot d\vec{s} = \int_A^B q \vec{E} \cdot d\vec{s} = q \int_A^B \vec{E} \cdot d\vec{s} \quad (= - q (V_B - V_A))\]
Si osserva che la forza è conservativa, ovvero è indipendente dal cammino svolto e dipende solo dal punto iniziale \(A\) e dal
punto finale \(B\). La circuitazione del campo elettrostatico è sempre 0: \(\displaystyle \oint \vec{E} \cdot d\vec{s} = 0\)

\subsection{Potenziale elettrostatico ed energia potenziale elettrostatica}
\subsubsection*{Potenziale elettrostatico}
Siccome la forza è conservativa è possibile definire un potenziale elettrostatico:
\[V_B - V_A = - \int_A^B \vec{E} \cdot d\vec{s} \qquad \qquad V(r) = -\int_{+\infty}^r \vec{E} \cdot d\vec{s}, \text{ con } V(\infty) = 0 \qquad \qquad V = \left[V\right] = \left[ \frac{J}{C} \right]\]

\begin{itemize}
	\item[-] Il lavoro della forza elettrostatica diventa \(\displaystyle W_{AB} = q \int_A^B \vec{E} \cdot d\vec{s} = - q (V_B - V_A)\)
	\item[-] Il potenziale e il campo elettrico sono legati dalla relazione: \(\displaystyle E = - \nab V\)
\end{itemize}

\subsubsection*{Energia potenziale elettrostatica}
\begin{itemize}
	\item[-] L'energia potenziale elettrostatica di una carica \(q\) è definita come il lavoro compiuto dal campo elettrico \(E\) per spostare
	la carica da un punto \(r\) all'infinito: \(\displaystyle U_e = - W_{\infty,r} =  q \int_r^{\infty} \vec{E} \cdot d\vec{s} = qV\)
	
	\item[-] L'energia potenziale di una composizione di cariche è la somma dei lavori compiuti per spostare le varie cariche da \(+\infty\)
	alla loro rispettiva posizione: \(\displaystyle U_e = W_{ext} = \sum_{j>i} \frac{1}{4 \pi \varepsilon_0} \; \frac{q_i q_j}{r_{ij}} = \frac{1}{2} \sum_{i \neq j} \frac{1}{4 \pi \varepsilon_0} \; \frac{q_i q_j}{r_{ij}}\)
\end{itemize}

\subsubsection*{Legge di conservazione dell'energia}
Durante il moto di una particella l'energia totale si conserva:
\[E_{TOT} = E_K + U_g + U_e = \frac{1}{2} mv^2 + mgh + qV = \text{costante}\]

\subsubsection*{Potenziali da sapere}
\begin{itemize}[topsep=3pt, itemsep=0pt]
	\item[-] potenziale generato da una distribuzione è \(\displaystyle V_{tot}(r) = -\int_{\infty}^r \vec{E} \cdot d\vec{s} = \frac{1}{4 \pi \varepsilon_0} \int \limits_\text{distribuzione} \frac{dq}{r}\)
	\item[-] potenziale generato da una carica ad una distanza \(r\) è \(\displaystyle V(r) = \frac{1}{4 \pi \varepsilon_0} \frac{q}{r}\)
	\item[-] potenziale lungo l'asse di un anello di raggio \(R\) con carica \(q\) uniformemente distribuita ad una distanza
	\(x\) dal centro dell'anello: \(\displaystyle V(x) = \frac{\lambda R}{2 \varepsilon_0 \; \sqrt{R^2 + x^2}}\)
	\item[-] potenziale lungo l'asse di un disco di raggio \(R\) con carica \(q\) uniformemente distribuita ad una distanza
	\(x\) dal centro del disco: \(\displaystyle V(x) = \pm \frac{\sigma}{2 \varepsilon_0} \; \left(\sqrt{R^2 + x^2} - x\right) \ux\)
	\item[-] potenziale generato da un piano infinito con carica uniformemente distribuita con fattore \(\sigma\) ad una distanza \(x\):
	\(\displaystyle V(x) = \pm \frac{\sigma}{2 \varepsilon_0} x\)
\end{itemize}

\subsection{Superfici equipotenziali}
Le superfici potenziali sono gli insiemi di punti in cui il potenziale è costante. Vengono utilizzate per dare una rappresentazione
grafica al potenziale in ogni punto del piano. Non forniscono l'intensità del campo, per un punto passa un'unica superficie e le
linee di forza sono ortogonali alle superfici.

\subsection{Dipolo elettrico}
\subsubsection*{Potenziale}
Il dipolo elettrico è costituito da due cariche puntiformi \(+q\) e \(-q\) a distanza \(a\). Il potenziale in un generico punto
\(P\) a distanza \(r_1\) da \(+q\) e \(r_2\) da \(-q\) vale 
\[V(P) = \frac{q}{4 \pi \varepsilon_0} \left(\frac{1}{r_1} - \frac{1}{r_2}\right) = \frac{q}{4 \pi \varepsilon_0} \frac{r_2 - r_1}{r_1 r_2}\]
Per \(r \gg a\) e definito l'angolo \(\theta\) tra \(\vec{r}\) e \(\vec{a}\) con \(\vec{a}\) diretto da \(-q\) a \(+q\) si ha:
\[V(P) = \frac{q \, a \cos\theta}{4 \pi \varepsilon_0 \, r^2} = \frac{q \, \vec{a} \cdot \hat{r}}{4 \pi \varepsilon_0 \, r^2}\]

\subsubsection*{Momento del dipolo elettrico}
Viene definito il momento del dipolo elettrico:
\[\vec{p} = q \, \vec{a} \qquad \rightarrow \qquad V(P) = \frac{\vec{p} \cdot \hat{r}}{4 \pi \varepsilon_0 \, r^2}\]

\subsubsection*{Campo elettrico generato da un dipolo}
Applicando la relazione \(E = - \nab V\) si ha che il campo generato da un dipolo è: 
\[\vec{E} = \frac{p}{4 \pi \varepsilon_0 \, r^3} (2 \cos \theta \; \hat{r} + \sin \theta \; \hat{\theta})\]

\begin{itemize}[topsep=3pt, itemsep=0pt]
	\item[-] lungo l'asse del dipolo (\(\theta = 0, \pi\)) il campo è \(\displaystyle \vec{E} = \frac{2 \vec{p}}{4 \pi \varepsilon_0 r^3}\)
	\item[-] lungo la perpendicolare all'asse del dipolo (\(\theta = \pi/2, 3\pi/2\)) il campo è \(\displaystyle \vec{E} = -\frac{\vec{p}}{4 \pi \varepsilon_0 r^3}\)
\end{itemize}

\subsubsection*{Dinamica del dipolo magnetico}
Si considera un dipolo immerso in un campo elettrico:
\begin{itemize}[topsep=3pt, itemsep=0pt]
	\item[-] sulle cariche agiscono due forze uguali e opposte \(\vec{F}_1 = -q\vec{E}\) e \(\vec{F}_2 = q\vec{E}\), la risultante
	delle forze è nulla per cui il centro di massa non si muove
	\item[-] il momento delle forze è \(\vec{M} = \vec{p} \times \vec{E} = -p E \sin \theta \hat{z}\), il dipolo ruota sotto azione
	del campo elettrico in modo da allineare il suo asse con l'orientamento del campo elettrico
\end{itemize}

\subsubsection*{Multipolo}
Generalizzando una distribuzione di cariche complessivamente neutra, si ha che il potenziale in un punto a distanza \(r\) con per
\(r \gg a\) dalla distribuzione è:
\[V(r) = \frac{q}{4 \pi \varepsilon_0 r} + \frac{\vec{p} \cdot \hat{r}}{4 \pi \varepsilon_0 r^2} + \frac{\hat{r} \cdot Q \hat{r}}{4 \pi \varepsilon_0 r^3} + \dots\]
\[\frac{q}{4 \pi \varepsilon_0 r} \rightarrow \text{carica singola} \qquad \frac{\vec{p} \cdot \hat{r}}{4 \pi \varepsilon_0 r^2} \rightarrow \text{dipolo} \qquad \frac{\hat{r} \cdot Q \hat{r}}{4 \pi \varepsilon_0 r^3} \rightarrow \text{quadrupolo} \qquad \dots\]

\newpage

\subsection{Flusso del campo elettrostatico}
Il flusso del campo elettrico attraverso una superficie infinitesima \(d\Sigma\) con vettore normale \(\un\) è:
\[d\Phi(\vec{E}) := \vec{E} \cdot \un \, d\Sigma\]
Si definisce quindi il flusso attraverso una superficie chiusa \(\Sigma\):
\[\Phi(\vec{E}) := \oint \vec{E} \cdot \un \, d\Sigma\]
Se il flusso è positivo, si dice che il flusso è uscente, se il flusso è negativo, il flusso è entrante.

\subsection{Teorema di Gauss}
\subsubsection*{Enunciato}
Il teorema di Gauss enuncia che il flusso del campo elettrico attraverso una superficie chiusa è pari al rapporto tra la carica
totale interna alla superficie e la costante \(\varepsilon_0\).
\[\Phi(\vec{E}) := \oint \vec{E} \cdot \un \, d\Sigma = \frac{q_\text{tot interna}}{\varepsilon_0}\]

\subsubsection*{Applicazioni}
\begin{itemize}[topsep=3pt, itemsep=0pt]
	\item[-] sfera di raggio \(R\) e densità superficiale uniforme \(\sigma\): \(\displaystyle \vec{E}_\text{ext} = \frac{R^2 \, \sigma}{r^2 \, \varepsilon_0} \; \hat{r} \qquad \vec{E}_\text{int} = 0\)
	\item[-] sfera di raggio \(R\) e densità volumetrica uniforme \(\tau\): \(\displaystyle \vec{E}_\text{ext} = \frac{R^3 \, \sigma}{3 r^2 \, \varepsilon_0} \; \hat{r} \qquad \vec{E}_\text{int} = \frac{\sigma \, r}{3 \, \varepsilon_0}\)
	\item[-] piano infinito con densità superficiale uniforme \(\sigma\): \(\displaystyle \vec{E} = \pm \frac{\sigma}{2 \varepsilon_0} \un\)
	\item[-] filo rettilineo con carica uniforme \(\lambda\): \(\displaystyle \vec{E} = \frac{\lambda}{2 \pi r \varepsilon_0} \hat{r}\)
\end{itemize}

\subsection{Forme integrali, forme differenziali, equazione di Poisson}
\begin{center}
	\def\arraystretch{2.5}
	\begin{tabular}{c c c}
		& \textbf{forme integrali} & \textbf{forme differenziabili} \\
		\toprule
		\textbf{potenziale} & \(\displaystyle \qquad V(r) = - \int_\infty^r \vec{E} \cdot d\vec{s} \qquad\) & \(\displaystyle \vec{E} = - \nab V\) \\[7pt]
		\hline
		\textbf{flusso} & \(\displaystyle \qquad \Phi(\vec{E}) = \oint \vec{E} \cdot d\vec{\Sigma} \qquad\) & \(\displaystyle \text{div} \, \vec{E} = \nab \cdot \vec{E} = \frac{\rho}{\varepsilon_0}\) \\[7pt]
		\hline
		\textbf{circuitazione} & \(\displaystyle \qquad \oint \vec{E} \cdot d\vec{s} \qquad\) & \(\displaystyle \text{rot} \, \vec{E} = \nab \times \vec{E} = \vec{0}\) \\[7pt]
		\bottomrule
		\bottomrule
		\textbf{eq. di Poisson} & \multicolumn{2}{c}{\(\displaystyle \nab^2 V = \nab \cdot \nab V = - \frac{\rho}{\varepsilon_0}\)} \\
		\bottomrule
	\end{tabular}
\end{center}
Le seguenti tre proprietà sono equivalenti:
\[\vec{E} \; \text{campo conservativo} \quad \Leftrightarrow \quad \int_C \vec{E} \cdot d\vec{s} \; \text{indipendente da} \; s \quad \Leftrightarrow \quad \vec{E} = - \nab V \quad \Leftrightarrow \quad \nab \times \vec{E} = \vec{0}\]

\newpage

\subsection{Conduttori in equilibrio}
\subsubsection*{Definizione}
I conduttori sono materiali con determinate proprietà che permettono la mobilità dei portatori di carica (il moto delle cariche
che li costituiscono).

\subsubsection*{Campo elettrico all'interno del conduttore}
In condizioni statiche e di equilibrio, non c'è movimento delle cariche e si ha che \(\vec{E} = 0\), ovvero non c'è forza che
mette in moto le cariche. Per questo motivo all'interno del conduttore (se non c'è movimento di cariche), il campo elettrico
interno è nullo \(\vec{E}_\text{int} = 0\) indipendentemente dal campo esterno. Questo implica che:
\begin{itemize}[topsep=3pt, itemsep=0pt]
	\item[-] l'eccesso di carica si trova solo sulla superficie del conduttore
	\item[-] il potenziale è costante sul conduttore (è una superficie equipotenziale)
	\item[-] il campo elettrico nelle vicinanze della superficie è \(\vec{E} = \sigma / \varepsilon_0 \; \un\)
	\item[-] la densità di carica superficiale è maggiore dove il raggio di curvatura è minore (effetto ago)
\end{itemize}

\subsubsection*{Conduttore cavo}
Un conduttore cavo è un conduttore al cui interno è presente una cavità. Valgono le seguenti proprietà:
\begin{itemize}[topsep=3pt, itemsep=0pt]
	\item[-] il campo all'interno della cavità è nullo (T. di Gauss)
	\item[-] il campo all'interno del conduttore è nullo (proprietà conduttore)
	\item[-] sulle pareti delle cavità non ci sono cariche elettriche (altrimenti ci si contraddice con i punti sopra)
	\item[-] la carica si distribuisce solo e soltanto sulla superficie esterna (conseguenza)
	\item[-] il campo all'interno del conduttore e delle cavità è nullo (conseguenza)
	\item[-] il potenziale è costante all'interno del conduttore e delle cavità (conseguenza)
	\item[-] il conduttore cavo costituisce uno schermo elettrostatico perfetto
\end{itemize}

\subsubsection*{Induzione completa}
Un conduttore carico si trova all'interno della cavità di un conduttore cavo. Si hanno le seguenti proprietà:
\begin{itemize}[topsep=3pt, itemsep=0pt]
	\item[-] all'interno del conduttore interno e del conduttore esterno il campo è nullo, nella cavità il campo è generato dalla
	carica superficiale del conduttore interno
	\item[-] la carica del conduttore interno induce la presenza di una carica uguale e opposta sulla superficie della cavità nel
	conduttore esterno che a sua volta induce una carica uguale alla prima sulla superficie esterna del conduttore esterno
	\item[-] l'oggetto si comporta come se ci fosse soltanto la carica distribuita sulla superficie esterna del conduttore esterno,
	ovvero si ha un perfetto schermo elettrostatico.
\end{itemize}

\newpage

\subsection{Condensatori}
\subsubsection*{Struttura}
Il condensatore è un componente elettronico costituito da due conduttori (armature) separati da un mezzo isolante (dielettrico).
È in grado d'immagazzinare energia elettrostatica attraverso l'accumulo delle cariche sulle armature e di rilasciarla. Si
definisce la capacità di un condensatore come il rapporto tra la carica sulle armature e la differenza di potenziale:
\[C = \frac{q}{\Delta V} \qquad \qquad q = C \, \Delta V \qquad \qquad \Delta V = \frac{q}{C} \qquad \qquad [C] = \frac{C_\text{oulomb}}{V_\text{olt}} = F_\text{araday}\]

\subsubsection*{Condensatore sferico}
Un condensatore sferico è costituito da una sfera di raggio \(R_1\) contenuta all'interno della cavità di una sfera cava con raggio
della cavità \(R_2\). Le due sfere costituiscono le due armature con distanza \(h = R_2 - R_1\). Nello spazio tra le armature si ha:
\[\vec{E}(r) = \frac{q}{4 \pi \varepsilon_0 \, r^2} \; \hat{r} \qquad \quad \Delta V = \frac{q}{4 \pi \varepsilon_0} \, \left(\frac{1}{R_1} - \frac{1}{R_2}\right) \qquad C = \varepsilon_0 \, \frac{4 \pi R_1 R_2}{h} \quad C_{R_2 \approx R_1} \approx \varepsilon_0 \, \frac{4 \pi R^2}{h} = \frac{\varepsilon_0 \, \Sigma}{h}\]

\subsubsection*{Condensatore cilindrico}
Un condensatore cilindrico è costituito da un cilindro interno di raggio \(R_1\) circondato da un cilindro cavo esterno con raggio
della cavità \(R_2\). Le armature sono costituite dai due cilindri concentrici con \(h = R_2 - R_1\) e altezza \(d\). Tra le armature vale:
\[\vec{E} = \frac{q}{2 \pi r d \, \varepsilon_0} \; \hat{r} \qquad \quad \Delta V = \frac{q}{2 \pi d \, \varepsilon_0} \, \log \left(\frac{R_2}{R_1}\right) \qquad C = \varepsilon_0 \, \frac{2 \pi d}{\log \left(R_2 / R_1\right)} \quad C_{R_2 \approx R_1} \approx \varepsilon_0 \, \frac{2 \pi d R}{h} = \frac{\varepsilon_0 \, \Sigma}{h}\]

\subsubsection*{Condensatore piano}
Un condensatore piano è costituito da due piani (infiniti) che costituiscono le due armature ad una distanza costante \(h\). Tra le armature si ha:
\[\vec{E} = \frac{\sigma}{\varepsilon_0} \; \un \qquad \qquad \Delta V = \frac{\sigma \, h}{\varepsilon_0} \qquad \qquad C = - \frac{\varepsilon_0 \, \Sigma}{h}\]

\subsubsection*{Condensatori in serie}
La carica sulle armature è costante per tutti i condensatori \(q_\text{tot} = q_1 = q_2 = \dots = q_n\) \\
La differenza di potenziale totale è la somma delle ddp di ogni condensatore \(\Delta V_\text{tot} = \Delta V_1 + \Delta V_2 + \dots + \Delta V_n\) \\
La capacità complessiva di \(n\) condensatori collegati in serie è:
\[\frac{1}{C_\text{tot}} = \frac{1}{C_1} + \frac{1}{C_2} + \dots + \frac{1}{C_n}\]

\subsubsection*{Condensatori in parallelo}
La differenza di potenziale è costante per tutti i condensatori \(\Delta V_\text{tot} = \Delta V_1 = \Delta V_2 = \dots = \Delta V_n\) \\
La carica totale è la somma delle cariche di ogni condensatore \(q_\text{tot} = q_1 + q_2 + \dots + q_n\) \\
La capacità complessiva di \(n\) condensatori collegati in parallelo è:
\[C_\text{tot} = C_1 + C_2 + \dots + C_n\]

\newpage

\subsection{Energia del campo elettrostatico}
\subsubsection*{Definizione}
L'energia del campo elettrostatico (generato da un condensatore) è pari al lavoro necessario a caricare il condensatore che genera
il campo:
\begin{align*}
	dW &= \Delta V dq = \frac{q}{C} dq \quad \rightarrow \quad W = \int_0^{q} \frac{q'}{C} \; dq' = \frac{1}{2} \frac{q^2}{C} = \frac{1}{2} C \Delta V^2 = \frac{1}{2} q \Delta V \\
	U_e &= W = \frac{1}{2} C \Delta V^2 = \frac{\varepsilon_0}{2} \, E^2 \, \Sigma \, h = \frac{\varepsilon_0}{2} \, E^2 \, \tau \quad \text{con} \; C = \frac{\varepsilon_0 \, \Sigma}{h}, \; \Delta V = E \, h, \; \tau = \Sigma \, h
\end{align*}
Si ottiene che l'energia del campo elettrostatico è:
\[dU_e = \frac{\varepsilon_0}{2} \, E^2 \, d\tau \qquad U_e = \frac{\varepsilon_0}{2} \int \limits_\text{distribuzione} E^2 \; d\tau\]

\subsection{Dielettrici}
\subsubsection*{Definizione}
Un dielettrico è un materiale isolante.

\subsubsection*{Condensatore di Epino e costante dielettrica}
Il condensatore di Epino è un condensatore in cui è possibile modificare la distanza tra le armature e inserire materiali
dielettrici tra le armature. Si osserva che:
\begin{itemize}[topsep=3pt, itemsep=0pt]
	\item[-] allontanando le armature, aumenta la differenza di potenziale e diminuisce la capacità, viceversa avvicinando le
	armature diminuisce la differenza di potenziale e aumenta la capacità
	\item[-] se viene inserito un conduttore tra le armature:
	\begin{itemize}[topsep=0pt, itemsep=0pt]
		\item[-] si ha induzione elettrostatica completa
		\item[-] il campo all'interno del conduttore è nullo
		\item[-] il potenziale si riduce, come se lo spessore del conduttore fa ridurre l'\(h\) e di conseguenza diminuisce
		il potenziale e aumenta la capacità
	\end{itemize}
	\item[-] se viene inserito un dielettrico tra le armature:
	\begin{itemize}[topsep=0pt, itemsep=0pt]
		\item[-] si ha un fenomeno di polarizzazione del dielettrico
		\item[-] il potenziale si riduce perché varia la costate dielettrica del mezzo
	\end{itemize}
\end{itemize}

\subsubsection*{Costante dielettrica relativa}
Viene definita la costante dielettrica relativa \(\kappa\) del dielettrico come il rapporto tra la differenza di potenziale in
assenza e  in presenza del dielettrico tra le armature. Si definisce anche la suscettività elettrica del dielettrico \(\chi\):
\[\kappa := \frac{V_0}{V_\kappa} > 1 \qquad \qquad \chi = \kappa - 1 > 0\]
Si ha che:
\[E_\kappa = \frac{V_\kappa}{h} = \frac{V_0}{\kappa h} = \frac{E_0}{\kappa} = \frac{\sigma_0}{\kappa \varepsilon_0} = \frac{\sigma_0}{\varepsilon_\kappa} = \frac{\sigma_0}{\varepsilon_0} - \frac{\sigma_p}{\varepsilon_0} = \frac{\sigma_0}{\varepsilon_0} - \frac{\chi \sigma_0}{\kappa \varepsilon_0} \quad
\text{con} \; \sigma_p = \frac{\kappa-1}{\kappa} \sigma_0 = \frac{\chi}{\chi+1}\sigma_0\]
Si osserva all'interno del dielettrico si crea un campo con modulo \(\displaystyle E_\text{int} = \frac{\chi \sigma_0}{\kappa \varepsilon_0}\)
che si oppone al campo esterno. Questo è dovuto alla polarizzazione del dielettrico che fa \say{accumulare} cariche alle estremità
del materiale isolante. \\[10pt]
La costante dielettrica del mezzo (\(\neq\) vuoto) diventa: \[\varepsilon_\kappa = \kappa \,  \varepsilon_0 > \varepsilon_0\]

\subsubsection*{Polarizzazione del dielettrico}
In presenza di un campo elettrico esterno, all'interno del dielettrico il nucleo degli atomi si sposta in direzione concorde
con il campo elettrico, mentre gli elettroni si concentrano nella parte opposta. In questo modo di formano tanti dipoli con
momento \(p_a = Z_\text{\# protoni} \cdot e_\text{carica elem.} \cdot \chi_\text{spostamento del nucleo}\), come nei materiali
costituiti da molecole polari (es. acqua). Si definisce il vettore di polarizzazione:
\[\vec{p} = \varepsilon_0 \, \chi_\text{suscettività} \, \vec{E} \qquad \qquad \vec{p} = \frac{\vec{p}_\text{tot}}{\tau} = \frac{N}{\tau} \langle \vec{p} \rangle  = n \langle \vec{p} \rangle  \qquad n = \text{\# dipoli per volume}\]
In assenza di campo elettrico esterno, il momento di dipolo totale è nullo. In presenza di un campo elettrico esterno, i momenti
si allineano e formano un momento totale risultante non nullo e concorde con il campo esterno. Lungo 

\subsubsection*{Capacità di condensatori con dielettrico}
La formula della capacità di condensatori con un dielettrico tra le armature aumenta di un fattore \(\kappa\):
\[C_\kappa = \frac{q}{\Delta V_\kappa} = \kappa \, \frac{q}{\Delta V_0} = \kappa \, C_0\]

\subsection{Materiali conduttori, modello di Drude e corrente elettrica}
\subsubsection*{Moto degli elettroni - elettroni liberi}
Nei materiali conduttori, la presenza di elettroni liberi permette la formazione di correnti di elettroni che si spostano da una
zona con potenziale minore ad un'altra con potenziale maggiore (verso opposto al campo elettrico).

\subsubsection*{Modello di Drude}
Nel modello di Drude di immagina che gli elettroni liberi viaggino in moto disordinato rimbalzando tra i cationi del conduttore.
\begin{itemize}[topsep=3pt, itemsep=0pt]
	\item[-] In assenza di campo elettrico, la direzione dopo gli urti è causale e la velocità media degli elettroni è nulla
	\(v_m = \frac{1}{N} \, \sum v_i = 0\), con \(\tau\) tempo medio tra gli urti
	\item[-] In presenza di un campo elettrico, gli elettroni subiscono una accelerazione dovuta alla forza elettrica
	\(\vec{a} = \frac{\vec{F}}{m} = - \frac{e \vec{E}}{m}\) per cui avranno una velocità di deriva \(v_d = v_m + a t = - e \, \vec{E} \, \tau / m \approx \text{costante}\)
\end{itemize}

\subsubsection*{Corrente elettrica e densità di corrente elettrica}
È possibile definire la corrente elettrica come la quantità di carica che attraversa una determinata superficie in un'unità di tempo:
\[i = \lim_{\Delta t \to 0} \frac{\Delta q}{\Delta t} = \frac{dq}{dt}\]
Sapendo che \(\Delta q = n \, (-e) \, v_d \, \Delta t \, \Sigma \cos \Theta\) con \(n =\) \# elettroni per unità di volume, si ha
\(di = n \, v_d \, d\Sigma \cos \theta\).
Si può definire la densità di corrente (corrente per unità di superficie):
\[\vec{j} = n (-e) \vec{v_d} \qquad \qquad i = j \Sigma, \quad j = i / \Sigma\]

\newpage

\subsection{Legge di Ohm}
Unendo le formule della densità di corrente \(\vec{j}\) e della velocità di deriva \(v_d\) si ha:
\[\vec{j} = \frac{n e^2 \tau}{m} \vec{E} \quad \rightarrow \quad \begin{matrix}
	\vec{j} = \sigma \vec{E} \;\; \text{con} \; \sigma = \frac{n e^2 \tau}{m} \;\; \text{conduttività del materiale} \\[10pt]
	\vec{E} = \rho \vec{j} \;\; \text{con} \; \rho = \frac{1}{\sigma} \;\; \text{resistività del materiale} \qquad \quad \,\,
\end{matrix}\]
In situazioni stazionarie la corrente è costante in ogni sezione del conduttore, per cui è possibile riscrivere la formula sopra
in funzione della corrente \(i\):
\[\vec{E} = \rho \vec{j} \quad \rightarrow \quad \vec{E} = \frac{\rho}{\Sigma} \vec{i}\]
Riscrivendo il campo attraverso il potenziale \(V = E \, h\) si ottiene
\[\vec{E} = \frac{\rho}{\Sigma} \vec{i} \quad \rightarrow \quad V = \frac{\rho \, h}{\Sigma} i  \quad \rightarrow \quad V = R \, i \;\; \text{con} \; R = \frac{\rho \, h}{\Sigma} \;\; \text{resistenza del materiale}\]

\subsection{Resistenze}
La resistenza è un componente elettronico costituito da un materiale in grado di ostacolare/attenuare l'intensità della corrente
che circola in un circuito. La relazione tra \(i\), \(V\), \(R\) è data dalla legge di Ohm \[V = R \, i\]
Il valore della resistenza varia in base alla temperatura secondo la legge: \(R = R_{20} (1 + \alpha \Delta T)\), con la costante
\(\alpha\) propria di ogni materiale.

\subsubsection*{Resistenze in serie}
L'intensità di corrente è costante su tutte le resistenze \(i_{tot} = i_1 = i_2 = \dots = i_n\) \\
La differenza di potenziale totale è la somma delle ddp \(\Delta V_\text{tot} = \Delta V_1 + \Delta V_2 + \dots + \Delta V_n\) \\
La resistenza complessiva di \(n\) resistenze collegate in serie è:
\[R_\text{tot} = R_1 + R_2 + \dots + R_n\]

\subsubsection*{Resistenze in parallelo}
L'intensità di corrente è è la somma delle correnti di ogni resistenza \(i_{tot} = i_1 + i_2 + \dots + i_n\) \\
La differenza di potenziale è costante per tutte le resistenze \(\Delta V_\text{tot} = \Delta V_1 = \Delta V_2 = \dots = \Delta V_n\) \\
La resistenza complessiva di \(n\) resistenze collegate in parallelo è:
\[\frac{1}{R_\text{tot}} = \frac{1}{R_1}+ \frac{1}{R_2} + \dots + \frac{1}{R_n}\]

\subsubsection*{Potenza dissipata da una resistenza}
La potenza dissipata da una resistenza sottoforma di calore per effetto Joule è:
\[P_\text{dissipata} = R \, i^2 = V \, i = \frac{V^2}{R}\]

\newpage

\subsection{Generatori di forza elettromotrice}
\subsubsection*{Definizione}
Un generatore di forza elettromotrice è un dispositivo in grado di mantenere una differenza di potenziale costante per un determinato
periodo di tempo. Il primo generatore inventato è stato la pila di Alessandro Volta (pila voltaica) che trasforma energia chimica in
energia elettrica.

\subsubsection*{Forza elettromotrice}
In un circuito con corrente che circola, il campo elettrico è costituito da un campo conservativo \(E\) e da un altro non conservativo
\(E*\) originato da un generatore (generatore chimico, induzione elettromagnetica). La circuitazione sul circuito è pari alla
differenza di potenziale creata dal generatore di forza elettromotrice:
\[\mathcal{E}_\text{FEM} := \oint \vec{E} \cdot d\vec{s} = V_A - V_B = \Delta V = R \, i\]
Nei generatori reali è presente una resistenza interna \(r\) al generatore, per cui si ha:
\[\mathcal{E} - r \, i = R \, i \quad \rightarrow \quad \Delta V = (R + r) i\]

\subsubsection*{Potenza erogata - dissipata}
La potenza erogata da un generatore è: \(\qquad P_\text{erogata} = \mathcal{E} i\) \\
Sostituendo \(\mathcal{E} = (R + r) \, i\) si ottiene che la potenza erogata viene interamente dissipata dalla resistenza:
\[P_\text{erogata} = \mathcal{E} i = R_\text{tot} i^2 = P_\text{dissipata}\]

\subsection{Circuiti e leggi di Kirchhoff}
\subsubsection*{Leggi di Kirchhoff}
\begin{itemize}
	\item[-] \textbf{Legge dei nodi}: \\
	La somma algebrica delle correnti che confluiscono in un nodo è nulla, le correnti entranti hanno segno	positivo, le correnti
	uscenti hanno segno negativo, \[\sum_k i_k = 0\]
	\item[-] \textbf{Legge delle maglie}: \\
	La soma algebrica delle forze elettromotrici presenti nei rami è uguale alla somma dei prodotti delle differenze di potenziale
	ai capi dei resistori nei rami. Fissato un verso di percorrenza della maglia, se la corrente è concorde avrà segno positivo
	altrimenti ha segno negativo. La sorgente di FEM ha segno positivo se è percorsa da polo negativo al positivo, altrimenti avrà
	segno negativo. \[\sum_k \mathcal{E}_k = \sum_k R_k i_k\]
\end{itemize}

\newpage

\subsubsection*{Circuito RC in carica}
\begin{itemize}[topsep=3pt, itemsep=0pt]
	\item[-] in un circuito RC aperto con condensatore carico si ha che il condensatore ha immagazzinato \(U_e = {q_0}^2 / 2C\) e
	ha una differenza di potenziale tra le armature \(V_0 = q_0/C\)
	\item[-] alla chiusura del circuito inizia a circolare corrente \(i = - dq/dt\) per annullare le cariche presenti nelle armature
	del condensatore
	\item[-] dalle equazioni sopra e dalle leggi di Kirchhoff si ottiene che:
	\begin{align*}
		&1. \quad V_C = \frac{q}{C} = V_R = R \, i \;\;\rightarrow\;\; i = \frac{q}{RC} \\
		&2. \quad i = - \frac{dq}{dt} = \frac{q}{RC} \;\;\rightarrow\;\; \frac{dq}{q} = \frac{dt}{RC} \\
		&3. \quad \int_q^0 \frac{dq}{q} = - \int_0^t \frac{dt}{RC} \;\;\rightarrow\;\; \ln(q/q_0) = -\frac{t}{RC} \;\;\stackrel{\tau = RC}{\rightarrow}\;\; q(t) = q_0 \, e^{-t/\tau} \\
		&4. \quad V(t) = \frac{q(t)}{C} = \frac{q_0}{C} e^{-t/\tau} = V_0 \, e^{t/\tau} = V_\text{max} \, e^{t/\tau} \\
		&5. \quad i(t) = -\frac{dq}{dt} = \frac{q_0}{RC}e^{-t/\tau} = i_\text{max} \, e^{-t/\tau} \qquad\qquad\qquad\qquad\qquad\qquad\qquad\qquad\qquad
	\end{align*}
	\item[-] per cui in un condensatore in scarica si hanno le seguenti equazioni
	\[q(t) = q_0 \, e^{-t/\tau} \qquad \qquad V(t) = V_0 \, e^{-t/\tau} \qquad \qquad i(t) = \frac{q_0}{RC} \, e^{-t/\tau} \qquad \qquad \tau = RC\]
	\item[-] l'energia immagazzinata dal condensatore viene tutta dissipata dalla resistenza
\end{itemize}

\subsubsection*{Circuito RC in scarica}
\begin{itemize}[topsep=3pt, itemsep=0pt]
	\item[-] in un circuito RC aperto con condensatore scarico e con un generatore di FEM, l'energia immagazzinata dal
	condensatore è nulla
	\item[-] alla chiusura del circuito inizia a circolare corrente \(i = dq/dt\) che deposita cariche sulle armature del
	condensatore, accumulando energia, fino a \(q_0 = C \mathcal{E}\)
	\item[-] dalle equazioni sopra e dalle leggi di Kirchhoff si ottiene che:
	\begin{align*}
		&1. \quad \mathcal{E} = V_R(t) + V_C(t) \;\;\rightarrow\;\; \mathcal{E} = Ri(t) + \frac{q(t)}{C} \\
		&2. \quad i = \frac{dq}{dt} \;\;\rightarrow\;\; R \frac{dq}{dt} = \mathcal{E} -\frac{q}{C} \;\;\rightarrow\;\; \frac{dq}{q - C\mathcal{E}} = - \frac{dt}{RC} \\
		&3. \quad \frac{dq}{q - C\mathcal{E}} = - \frac{dt}{RC} \;\;\rightarrow\;\; \int_0^q \frac{dq}{q - C\mathcal{E}} = -\int_0^t \frac{dt}{RC} \;\;\rightarrow\;\; \ln \frac{q - C\mathcal{E}}{-C\mathcal{E}} = -\frac{t}{RC} \\
		&4. \quad \ln \frac{q - C\mathcal{E}}{-C\mathcal{E}} = -\frac{t}{RC}  \;\;\rightarrow\;\; q(t) = C \mathcal{E} (1-e^{-t/\tau}) = q_\text{tot} (1-e^{-t/\tau}) \\
		&5. \quad V(t) = \frac{q(t)}{C} = \mathcal{E} (1-e^{-t/\tau}) =  V_\text{max} (1-e^{-t/\tau}) \\
		&6. \quad i = \frac{dq}{dt} = \frac{\mathcal{E}}{R} (1-e^{-t/\tau}) = i_\text{max} (1-e^{-t/\tau}) \qquad\qquad\qquad\qquad\qquad\qquad\qquad\qquad
	\end{align*}
	\item[-] per cui in un condensatore in scarica si hanno le seguenti equazioni
	\[q(t) = C \mathcal{E} \, (1-e^{-t/\tau}) \qquad \qquad V(t) = \mathcal{E} \, (1-e^{-t/\tau}) \qquad \qquad i(t) = \frac{\mathcal{E}}{R} \, (1-e^{-t/\tau}) \qquad \qquad \tau = RC\]
	\item[-] l'energia fornita dal generatore viene per metà immagazzinata nel condensatore e per metà dissipata nella resistenza
	\item[-] si immagina che esista una corrente di spostamento tra le armature del condensatore, oltre a quella di conduzione che
	circola all'interno del circuito:
	\[i_\text{spostamento} = \varepsilon_0 \dt \Phi(E) \qquad i = i_\text{conduzione} + i_\text{spostamento} = i_\text{conduzione} + \varepsilon_0 \dt \Phi(E)\]
\end{itemize}

\newpage

\section{Magnetostatica}
\subsection{Esperimenti}
\subsubsection*{Esperimento di Gilbert}
Viene appesa una bacchetta (magnetica) al soffitto attraverso un filo. Se si avvicina l'estremità di un'altra bacchetta magnetica
all'estremità di quella appesa, o le due bacchette si avvicinano, o si respingono. Se si avvicina un'estremità al centro della
bacchetta, non succede nulla. Si conclude che esistono due cariche magnetiche distribuite alle estremità ed esiste un campo \(\vec{B}\)
che immerge lo spazio e agisce per iterazione a distanza.

\subsubsection*{Esperimento di Oërsted}
Viene preso un filo rettilineo. Il filo viene posto in posizione verticale e su un piano perpendicolare al filo viene depositata
della limatura di ferro. Si osserva che quando il filo è percorso da corrente, la limatura si dispone lungo delle circonferenze
concentriche (linee di campo). Si conclude che le correnti producono campo magnetico.

\subsection{Forza di Lorentz}
La forza agente su una carica \(q\) con velocità \(v\) immersa in un campo magnetico \(\vec{B}\) è
\[\vec{F} = q \vec{v} \times \vec{B}\]
La forza è perpendicolare alla direzione di moto e al campo magnetico, per cui il lavoro compiuto dalla forza di Lorentz è nullo
(\(\vec{F} \perp d\vec{s}\) e \(\vec{F} \cdot d\vec{s} = 0\)). L'energia totale della carica non cambia per cui non si hanno
variazioni di velocità, bensì solo di direzione: la carica si muoverà di moto rettilineo uniforme.
L'unità di misura del campo magnetico è \([B] = \frac{N_\text{ewton}}{A_\text{mpère} \, m_\text{etro}} = T_\text{esla}\)

\subsection{Seconda legge elementare di Laplace}
Immaginando un filo su cui scorre una corrente \(i\) con velocità \(v_d\), si ha:
\begin{align*}
	& d\vec{F} = - N_\text{elettroni} \; e_\text{carica} \; \vec{v}_d \times \vec{B} \qquad n_e = \frac{N_\text{elettroni}}{d\tau} = \frac{N}{\Sigma ds} \rightarrow N = n_e \, \Sigma \, ds \qquad \vec{j} = n \, q \, \vec{v}_d = \frac{i}{\Sigma} \, \un \\
	& d\vec{F} = - n_e \, e \, \Sigma \, ds \, \vec{v}_d \times \vec{B} = \Sigma \, ds \, \vec{j} \times \vec{B} = i \; d\vec{s} \times \vec{B} \\
	&\qquad\qquad\qquad d\vec{F} = i \; d\vec{s} \times \vec{B} \quad \vec{F} = i \int d\vec{s} \times \vec{B} \qquad \leftarrow \text{seconda legge di Laplace}
\end{align*}
Nel caso di un filo rettilineo la legge diventa \(\vec{F} = i \; B \; l \; \cos \theta\). La forza totale agente su un circuito
chiuso è nulla, per cui il centro di massa non si sposta (al massimo il circuito ruota).

\subsection{Dipolo magnetico}
\subsubsection*{Funzionamento}
Il dipolo magnetico è costituito da un generico magnete presente in natura. Nel nostro caso è una spira rettangolare in cui circola
corrente immersa in un campo magnetico. Si analizzano le forze e i momenti agenti sui lati del circuito:
\begin{itemize}[topsep=3pt, itemsep=0pt]
	\item[-] per quanto visto sopra \(F_\text{tot} = 0\) per cui il centro di massa non si sposta
	\item[-] i momenti delle forze agenti sui quattro lati si osserva che per i due lati orizzontali \(\vec{M} = 0\), mentre per i
	lati verticali si ha \(\vec{M} = \vec{r} \times \vec{F}_1 + \vec{r} \times \vec{F_2} = \frac{b}{2} \sin \theta (F_1 + F_2) \; \uz = \frac{b}{2} \sin \theta \; 2 i a B \; \uz = i \Sigma B \sin \theta \; \uz\)
	ovvero \(\vec{M}_\text{tot} = i \Sigma \un \times \vec{B}\)
	\item[-] da quanto osservato sopra il corpo ruota su se stesso per allineare la normale della superficie \(\un\) alla direzione
	del campo magnetico \(\vec{B}\)
\end{itemize}

\subsubsection*{Momento di dipolo magnetico}
Si definisce il momento di dipolo magnetico \(\vec{m} = i \Sigma \un\) per cui il momento è \(\vec{M} = \vec{m} \times \vec{B}\).
Inoltre si può calcolare:
\begin{itemize}[topsep=3pt, itemsep=0pt]
	\item[-] lavoro di rotazione: \(W = \int \vec{M} \cdot d\theta = - \int m B \sin \theta d\theta = mB (\cos \theta_1 - \cos \theta_0)\)
	\item[-] energia interna di rotazione: \(\Delta U = -W = mB (\cos \theta_0 - \cos \theta_1)\)
	\item[-] accelerazione angolare: \(\alpha = - d^2 \theta / dt^2\) (si osserva che descrive un moto armonico oscillante)
	\item[-] pulsazione delle piccole oscillazioni: \(\omega = \sqrt{\frac{i \sigma B}{I}} = \sqrt{\frac{mB}{I}}\)
	\item[-] periodo delle piccole oscillazioni: \(T = \frac{2 \pi}{\omega} = 2 \pi \sqrt{\frac{I}{mB}}\)
\end{itemize} 

\subsubsection*{Applicazioni}
\begin{itemize}[topsep=3pt, itemsep=0pt]
	\item[-] \textbf{galvanometro a bobina mobile}: bobina formata da spire immerse in un campo magnetico a cui è legato un
	indicatore. In base all'intensità di corrente fatta circolare all'interno della bobina, questa ruota e fa ruotare l'indicatore
	lungo una scala graduata. Attraverso questo strumento è possibile misurare l'intensità e la differenza di potenziale.
	\item[-] \textbf{effetto Hall}: si ha una sbarra conduttrice percorsa da corrente lungo il lato lungo, immersa in un campo
	magnetico perpendicolare al lato lungo. Sui portatori di carica agisce la forza di Lorentz \(\vec{F}_L = e \vec{v}_d \times \vec{B}\)
	che fa accumulare cariche ai margini dei conduttori, creando un campo magnetico interno \(\vec{E}_H \perp \vec{v}_d, \vec{B}\)
	chiamato Campo di Hall \(\vec{E}_H = \vec{v}_d \times \vec{B}\) a cui segue una differenza di potenziale
	\(\Delta V_H\) detta tensione di Hall che vale \(\Delta V_H = h E_H\). Misurando la tensione di Hall è possibile determinare
	l'intensità del campo magnetico o la velocità di deriva degli elettroni.
\end{itemize}

\subsection{Moto di particelle in campo magnetico}
\subsubsection*{Moto circolare uniforme con \(\vec{v}_d \perp \vec{B}\)}
La forza di Lorentz, agendo perpendicolarmente al moto delle cariche, agisce come una forza centripeta in un moto circolare uniforme
su un piano perpendicolare al campo magnetico. Si ottiene, quindi:
\begin{itemize}[topsep=3pt, itemsep=0pt]
	\item[-] \(F_L = F_C \;\;\rightarrow\;\; qvB = mv^2/r\) da cui si può ottenere \(r\), \(B\), \dots
	\item[-] \(\vec{F}_L = \vec{F}_C \;\;\rightarrow\;\; q \vec{v} \times \vec{B} = m \vec{\omega} \times (\vec{\omega} \times \vec{r}) \;\;\rightarrow\;\; \vec{\omega} = -q\vec{B}/m\)
	\item[-] \(\vec{v} = \vec{\omega} \times \vec{r} \rightarrow v = \omega r, \;\; a_c = \omega^2 r\) da cui si ricava \(\omega\), \(v\), \dots
	\item[-] \(\vec{\omega} = -q\vec{B}/m \qquad T = 2 \pi / \omega = 2\pi m / q B \qquad f = 1/T = qB/2\pi m\)
\end{itemize}
Il meno della \(\vec{\omega}\) è per convenzione: quando \(\vec{B}\) è verso l'alto, \(\vec{\omega}\) è verso il basso e la
rotazione è in senso orario, altrimenti il viceversa.

\subsubsection*{Moto elicoidale con \(\vec{v}_d \not\perp \vec{B}\)}
Si scompone la velocità in due componenti, \(\vec{v}_{d,\perp}\) e \(\vec{v}_{d,\parallel}\). La componente perpendicolare genera
un moto circolare uniforme come descritto sopra, quella perpendicolare agisce con moto rettilineo uniforme, creando un moto
complessivo elicoidale. Il passo dell'elica è \(P_\text{asso} = v_\parallel T = v \cos \theta \; 2 \pi m/qB\).

\subsubsection*{Spettrometro di massa}
Lo spettrometro di massa agisce facendo deviare le particelle (ionizzate) mentre attraversano una regione immersa da campo
magnetico a velocità controllata. Misurando il raggio di curvatura è possibile risalire alla massa delle particelle.

\subsubsection*{Selettore di velocità}
Il selettore di velocità è uno strumento che devia le particelle attraverso la forza di Lorentz e la forza del campo elettrico,
date dalla presenza del campo elettrico e magnetico tali che \(\vec{E} \perp \vec{B} \perp \vec{v}_d\). In questo modo le due
forze \(F_E\), \(F_L\) sono antagoniste e si equivalgono solo per un determinato valore di \(v\). Si ha, quindi, che solo le 
particelle con velocità \(v\) procederanno senza deviazione, mentre tutte le altre saranno deviate e di conseguenza bloccate.

\subsubsection*{Ciclotrone}
Il ciclotrone è un acceleratore di particelle che agisce accelerando le particelle mentre passano da una zona con potenziale
minore ad una con potenziale maggiore e allo stesso tempo vengono fatte ruotare per opera di un campo magnetico. È formato da
due semigusci cilindrici che fungono da elettroni. Ogni volta che la particella passa da un elettrodo all'altro, viene accelerata
per la differenza di potenziale tra i due elettrodi. Il campo magnetico mantiene la particella in moto circolare facendola entrare
ed uscire tra i due elettrodi. Il periodo d'inversione degli elettrodi è \(\tau = T/2 = \pi m/qB\), mentre la frequenza del
ciclotrone è \(\omega = qB/m\)

\subsection{Sorgenti del campo magnetico}
\subsubsection*{Prima legge elementare di Laplace}
Il campo magnetico generato da delle cariche in moto ad una distanza \(r\) vale:
\[d\vec{B} = k_m \frac{i d\vec{s} \times \ur}{r^2} \qquad \qquad k_m = 10^{-7} \frac{T m}{A} = 10^{-7} \frac{H}{m} \qquad \qquad H_\text{enry} = \frac{T m^2}{A}\]
Si definisce la permeabilità magnetica \(\mu_0 = 4 \pi k_m = 1.26 \cdot 10^{-6} H/m\) per cui la legge sopra diventa:
\[d\vec{B} = \frac{\mu_0}{4 \pi} \frac{i d\vec{s} \times \ur}{r^2}\]
Applicando la legge ad un circuito chiuso, si ottiene la legge di Ampère-Laplace:
\[\vec{B} = \frac{\mu_0}{4 \pi} i \oint \frac{d\vec{s} \times \ur}{r^2} \]

\subsubsection*{Campo magnetico di configurazioni particolari}
\begin{itemize}[topsep=3pt, itemsep=0pt]
	\item[-] \textbf{singola carica}: \(\displaystyle \vec{B}_q = \frac{\mu_0}{4 \pi} \frac{q \vec{v} \times \ur}{r^2}\)
	\item[-] \textbf{filo rettilineo - Legge di Biot-Savard}: le linee di campo sono circonferenze concentriche con verso
	determinato dalla mano destra, con modulo: \[B(r) = \frac{\mu_0 i}{2 \pi r}\]
	\item[-] \textbf{spira circolare}: il campo magnetico lungo l'asse generato da una spira circolare è
	\[\vec{B}(z) = \frac{\mu_0 i R^2}{2 \sqrt{R^2 + z^2}^3} \un\]
	\item[-] \textbf{dipolo}: analogo al dipolo elettrico, per un punto a distanza \(r\) con angolo \(\theta\)
	dall'asse del dipolo
	\[\vec{B} = \frac{\mu_0}{4 \pi} \frac{m}{r^3} \left(\begin{matrix} 2 \cos \theta \\ \sin \theta \\ 0 \end{matrix}\right)\]
	\item[-] \textbf{solenoide}: all'esterno è nullo, lungo l'asse interno per un punto a distanza \(x\) dal centro:
	\[B_{\text{lunghezza} \; d} = \frac{\mu_0 \lambda i}{2} \left[ \frac{d-2x}{\sqrt{(d-2x)^2+4R^2}} + \frac{d+2x}{\sqrt{(d+2x^2)^2 + 4R^2}} \right]\qquad \qquad B_\text{infinito} = \mu_0 \lambda i\]
\end{itemize}

\newpage

\subsection{Interazione tra fili percorsi da corrente}
\subsubsection*{Interazione tra fili paralleli}
Dati due fili paralleli percorsi da corrente con distanza \(d\) si ha che il primo genera un campo magnetico che agisce sul
secondo e viceversa. Se le due correnti hanno verso concorde, allora i fili si attraggono, se le due correnti sono discordi,
i fili si respingono.

\subsubsection*{Interazione tra fili perpendicolare}
Dati due fili perpendicolari percorsi da corrente si ha che la risultante delle forze è nulla, per cui non c'è spostamento,
in più si generano due momenti delle forze uguali ed opposte che inducono all'allineamento dei due fili. Se i fili sono legati
insieme, il momento risultante sarà nullo e non ci sarà nemmeno rotazione.

\subsubsection*{Elettrodinamometro assoluto}
L'elettrodinamometro assoluto è una bilancia a bracci uguali in cui da una parte si pone l'oggetto da pesare e dall'altra è
legata una spira (bobina) a cui viene fatta corrispondere un'altra spira appoggiata sul piano. In questo modo si ottiene una
configurazione di due fili infiniti paralleli percorsi da corrente. È possibile misurare la forza peso dell'oggetto misurando
la corrente sulle spire necessaria a tenere la bilancia in equilibrio.

\subsection{Magnetizzazione dei materiali}
\subsubsection*{Permeabilità magnetica}
Analogamente a quanto fatto con i dielettrici, si definisce la permeabilità magnetica relativa \(\kappa_m\) del mezzo come il
rapporto tra il campo generato da un solenoide vuoto e quello generato dal solenoide pieno e analogamente anche la permeabilità
magnetica assoluta \(\mu\): \[\kappa_m := \frac{B_\kappa}{B_0} \qquad\qquad\qquad \mu = \kappa_m \mu_0\]
In alcuni casi torna utile definire un campo ausiliario \(\vec{H} = \mu \vec{B}\) che elimina la dipendenza dal mezzo.

\subsubsection*{Suscettività magnetica}
Si definisce la suscettività magnetica del mezzo \(\chi_m\) tale per cui è possibile definire la variazione del campo:
\[\vec{B}_m = \vec{B}_\kappa - \vec{B}_0 = (\kappa_m - 1) B_0 = \chi_m B_0 \qquad\qquad \chi_m := \kappa_m - 1\]

\subsubsection*{Vettore di magnetizzazione}
Si definisce il vettore di magnetizzazione \(M\) dato dal contributo del materiale al campo magnetico:
\[M = \chi_m H \qquad \rightarrow \qquad \vec{B}_\kappa = \vec{B}_0 + \vec{B}_m = \vec{B}_0 + \chi_m \vec{B}_0 = \mu_0 \vec{H} + \mu_0 \chi_m \vec{H} = \mu_0(\vec{H} + \vec{M})\]

\subsubsection*{Correnti di magnetizzazione}
È possibile definire le correnti di magnetizzazione o correnti amperiane \(i_m\) tali che:
\[i_m := \chi_m i \qquad \rightarrow \qquad i_\text{tot} = i + i_m \qquad \rightarrow \qquad \vec{B} = \mu_0 \lambda (i + i_m)\]

\subsubsection*{Classificazione dei materiali}
\begin{itemize}[topsep=3pt, itemsep=0pt]
	\item[-] \textbf{materiali diamagnetici}: materiali che fanno diminuire il campo magnetico esterno, con \(\chi_m < 0\)
	\item[-] \textbf{materiali paramagnetici}: materiali che aumentano il campo magnetico esterno, con \(\chi_m > 0\)
	\item[-] \textbf{materiali ferromagnetici}: materiali che aumentano di molto il campo magnetico
\end{itemize}

\subsubsection*{Domini di Weiss}
Il comportamento dei materiali ferromagnetici si spiega attraverso i domini di Weiss. I domini sono sezioni del materiale con una
stessa polarizzazione magnetico, di solito i vettori di polarizzazione hanno direzione casuale, per cui complessivamente hanno 
un vettore di polarizzazione nullo. In presenza di campo magnetico esterno, i domini polarizzano lungo la direzione di quest'ultimo.

\subsubsection*{Ciclo di isteresi magnetica}
Una volta polarizzato, un materiale ferromagnetico conserva il suo stato di polarizzazione e non sarà più possibile depolarizzarlo
utilizzando solamente il campo magnetico. Nel grafico H-M si osserva la rappresentazione grafica dell'isteresi magnetica. L'area
del grafico indica il lavoro necessario a polarizzare i vari domini di Weiss per unità di volume e vale \(H \cdot B = \dots = E_\text{nergia}/L^3\).
Magneti permanenti hanno ciclo di isteresi molto grande, magneti deboli hanno ciclo di isteresi più stretto e vengono usati come
elettromagneti (che si polarizzano solo in presenza di campo magnetico).

\subsection{Legge di Ampère - circuitazione del campo magnetico}
\subsubsection*{Definizione}
Dati dei fili percorsi da corrente \(i_1, i_2, \dots\) e un cammino \(\mathcal{L}\) su cui calcolare la circuitazione del campo
magnetico generato dai fili si ha che:
\[\Gamma_\mathcal{L} (\vec{B}) = \oint_\mathcal{L} \vec{B} \cdot d\vec{s} = \mu_0 \cdot i_\text{concatenate} \qquad \qquad \nab \times \vec{B} = \mu_0 \; \vec{j}_\text{densità di corrente concatenata}\]

\subsubsection*{Campo in un solenoide toroidale}
Utilizzando il teorema di Ampère e scegliendo come cammino una circonferenza concentrica al toro e passante all'interno di esso,
si ottiene che il campo è analogo a quello di un solenoide infinito \(B = \mu_0 \lambda i\)

\subsubsection*{Legge di Ampère generalizzata}
In presenza di un mezzo diverso dal vuoto, si aveva definito la corrente amperiana \(i_m = i_\text{tot} - i = \chi_m i\). La
formula per la circuitazione del campo elettrico diventa:
\[\Gamma_\mathcal{L} (\vec{B}) = \oint_\mathcal{L} \vec{B} \cdot d\vec{s} = \mu_0 \cdot (i + i_m) \qquad\quad \oint_\mathcal{L} \vec{H} \cdot d\vec{s} = \mu_0 \cdot i \qquad\quad \oint_\mathcal{L} \vec{M} \cdot d\vec{s} = \mu_0 \cdot i_m \qquad\quad \nab \times \vec{H} = \vec{j}\]

\subsection{Legge di Gauss per il campo magnetico}
Si studia il flusso del campo magnetico attraverso una superficie chiusa \(\Sigma\). Si osserva che se il campo
entra in una superficie, prima o poi deve uscire, per cui si ha flusso nullo.
\[\Phi(\vec{B}) = \oint_\Sigma \vec{B} \cdot \un d\Sigma = 0 \qquad \qquad \nab \cdot \vec{B} = 0\]

\newpage

\section{Elettromagnetismo}
\subsection{Equazioni di Maxwell (senza variazione temporale)}
\begin{align*}
	&\oint \vec{E} \cdot d\vec{s} = 0 \qquad \nab \times \vec{E} = 0 &&\oint \vec{B} \cdot d\vec{s} = \mu_0 \; i \qquad \nab \times \vec{B} = \mu_0 \; \vec{j} \\[10pt]
	&\oint \vec{E} \cdot d\vec{\Sigma} = \frac{q_\text{int}}{\varepsilon_0} \qquad \nab \cdot \vec{E} = \frac{\rho_\text{int}}{\varepsilon_0} &&\oint \vec{B} \cdot d\vec{\Sigma} = 0 \qquad \nab \cdot \vec{B} = 0
\end{align*}

\subsection{Legge di Faraday-Neumann-Lenz}
Si osserva che in un circuito chiuso con amperometro a 0 centrale, quando si avvicina un magnete, circola corrente, quando si
ferma il magnete, la corrente smette di circolare. Si conclude che la corrente circola solo se c'è variazione di campo magnetico
e si osserva che il verso è opposto a quello della corrente che indurrebbe il campo: la corrente si oppone alla variazione di
flusso.
\[\mathcal{E}_\text{FEM indotta} = -\frac{d \Phi(\vec{B})}{dt} \qquad \rightarrow \qquad \oint_\mathcal{L} \vec{E}_i \cdot d\vec{s} = -\frac{\partial}{\partial t}\int_\Sigma \vec{B} \cdot d\vec{\Sigma} \qquad\qquad \nab \times \vec{E} = -\frac{d\vec{B}}{dt}\]

\subsubsection*{Correnti parassite - di Foucault - di eddy}
In un conduttore immerso in un campo magnetico variabile nel tempo, si formano delle correnti indotte all'interno di esso. Queste
correnti dissipano energia sotto forma di calore e vengono chiamate correnti parassite o correnti di Foucault o correnti di eddy
(vortice).

\subsection{Correnti alternate}
\subsubsection*{Alternatore}
Un alternatore è composto da una spira (bobina) circolare immersa in un campo magnetico che viene fatta ruotare con un momento
fornito dall'esterno con velocità angolare \(\omega\). Variando l'angolo tra il campo \(B\) e la normale alla superficie si ha
una variazione di flusso che induce una corrente indotta nella spira.
\begin{align*}
	&\quad \mathcal{E}_\text{FEM indotta} = - \frac{\partial \Phi(\vec{B})}{\partial t} = - B \Sigma \omega \sin (\omega t) = - \mathcal{E}_\text{ind max} \sin(\omega t) \qquad \mathcal{E}_\text{ind max} = B \Sigma \omega \\
	&\quad i_\text{indotta} = \frac{\mathcal{E}_\text{indotta}}{R} = i_\text{ind max} \sin(\omega t) \qquad i_\text{ind max} = \frac{B \Sigma \omega}{R} \\
	&\quad P_\text{otenza} = R \cdot i^2 = \frac{\mathcal{E}_\text{ind max}}{R} \sin (\omega t) \qquad P_\text{media} = \frac{1}{2} R \cdot {i_\text{ind max}}^2 \\
	&\quad \mathcal{E}_\text{eff} = \mathcal{E}_\text{ind max} \frac{\sqrt{2}}{2} \qquad i_\text{eff} = i_\text{ind max} \frac{\sqrt{2}}{2} \quad \rightarrow \quad P_\text{media} = \frac{{\mathcal{E}_\text{eff}}^2}{R} =  R {i_\text{eff}}^2
\end{align*}
Si osserva che forza elettromotrice e corrente sono in fase tra loro (\(\Delta\phi = 0\)). Per assorbire l'\(1/2\) della potenza
media si definisce la forza elettromotrice efficace e la corrente efficace.

\subsubsection*{Legge di Felici}
La carica che circola nel circuito tra due istanti \(t_1\) e \(t_2\) è pari a:
\[q = \frac{\Phi_{t_1}(\vec{B}) - \Phi_{t_2}(\vec{B})}{R}\]

\newpage

\subsection{Induttanza}
\subsubsection*{Induttanza}
Un'induttanza è un componente elettronico formato da una bobina su cui scorre corrente elettrica. In questa bobina possono avere
fenomeni di autoinduzione e di mutua induzione.

\subsubsection*{Autoinduzione}
Avviene quando un circuito genera un campo magnetico che induce una corrente indotta su sé stesso.
\[\vec{B} = \frac{\mu_0 \, i}{4 \pi} \oint \frac{d\vec{s} \times \ur}{r^2} \; \text{per Ampère-Laplace}, \quad \Phi(\vec{B}) = \int \vec{B} \cdot d\vec{\Sigma} \quad \rightarrow \quad \Phi(\vec{B}) = \int \left(\frac{\mu_0 \, i}{4 \pi} \oint \frac{d\vec{s} \times \ur}{r^2}\right) \cdot d\vec{\Sigma}\]
\[\Phi(\vec{B}) = L \, i \qquad\qquad L = \int \left(\frac{\mu_0}{4 \pi} \oint \frac{d\vec{s} \times \ur}{r^2}\right) \cdot d\vec{\Sigma}\]
Si definisce il coefficiente di autoinduzione (o autoinduttanza) \(L\) con unità di misura \(H_\text{enry}\). Siccome il fenomeno
genera correnti indotte, si ha che l'intensità di corrente che circola sul circuito dipenderà dal tempo ed è data dalla soluzione
di un'equazione differenziale.
\[\mathcal{E}_i = -\frac{d\Phi(\vec{B})}{dt} = -\frac{d(Li)}{dt} = -L \frac{di}{dt} \quad \rightarrow \quad i(t) = \frac{\mathcal{E}}{R} - \frac{A}{R} \; e^{-\frac{R}{L} t}\]

\subsubsection*{Extracorrenti di chiusura}
Nel caso di chiusura di un circuito RL all'istante \(t = 0\) si ha che \(i_0 = 0\) con \(i_{>0}\) crescente:
\[i(0) = 0 \quad\rightarrow\quad \frac{\mathcal{E}}{R} - \frac{A}{R} \; e^{-\frac{R}{L} \cdot\, 0} = 0 \quad\rightarrow\quad A = \mathcal{E}\]
Per cui alla chiusura del circuito si ha che l'intensità di corrente che circola è:
\[i(t) = \frac{\mathcal{E}}{R} (1-e^{-t/\tau}) = \frac{\mathcal{E}}{R} - \frac{\mathcal{E}}{R} e^{-t/\tau} \qquad \tau=\frac{L}{R} \leftarrow \text{t. caratteristico} \quad - \frac{\mathcal{E}}{R} e^{-t/\tau} \leftarrow \text{extracorrente di chiusura}\]

\subsubsection*{Extracorrenti di apertura}
Nel caso di apertura di un circuito RL all'istante \(t = 0\) si immagina d'inserire una resistenza \(R' \gg R\) in modo da poter
immaginare che si possa avere ancora corrente di scorrimento, ma allo stesso tempo si ha praticamente un circuito aperto per
l'alto valore della resistenza.
\[i(t) = \frac{1}{R + R'} \left(\mathcal{E} - A \; e^{-\frac{R+R'}{L} \, t}\right) = \frac{1}{R'} \left(\mathcal{E} - A \; e^{-\frac{R'}{L} \, t}\right) \qquad i(0) = \frac{\mathcal{E}}{R} \quad \rightarrow \quad A = \mathcal{E}\,\frac{R - R'}{R} \approx \mathcal{E} - \frac{R'}{R}\]
Per cui all'apertura del circuito si ha che l'intensità di corrente che circola è:
\[i(t) = \frac{\mathcal{E}}{R'}  + \frac{\mathcal{E}}{R} \; e^{-t/\tau'} \approx \frac{\mathcal{E}}{R} \, e^{-t/\tau'} \qquad \tau' = \frac{L}{R'} \leftarrow \text{t. caratteristico} \quad \frac{\mathcal{E}}{R} \, e^{-t/\tau'} \leftarrow \text{extracorrente di apertura}\]
L'extracorrente di apertura è quella che genera la scintilla quando si stacca la spina.

\subsubsection*{Autoinduttanze in solenoidi}
In un solenoide toroidale a sezione rettangolare di lati \(a\), \(b\) con \(N\) spire, il valore di autoinduttanza è
\[L = \frac{\mu_0 N^2 a}{2 \pi} \ln \left( 1+ \frac{b}{R} \right)\]

In un solenoide infinito di passo \(\lambda\), il valore di autoinduttanza per unità di lunghezza è:
\[L_{/l} = \mu_0 \lambda^2 \Sigma\]

\subsubsection*{Energia magnetica immagazzinata all'interno di un'induttanza}
È possibile definire l'energia magnetica immagazzinata all'interno di un'induttanza come il lavoro compiuto dal generatore nel
momento di chiusura e l'energia dissipata al momento di apertura del circuito.
Definendo il lavoro \(dW = \mathcal{E} dq = \mathcal{E} i \, dt = (R i + L \, di / dt) \, i \, dt = R i^2 dt + L i \, di\) si ha
che il lavoro complessivo al momento di chiusura del circuito è pari a:
\[W_\text{tot} = \frac{1}{2} Li^2 + R \, i_\infty^2 \, t_\infty \qquad \begin{matrix}
	^1\!/_2 \; Li^2 \leftarrow \text{energia immagazzinata nell'induttanza}\\[5pt]
	R \, i_\infty^2 \, t_\infty \leftarrow \text{energia dissipata dalla resistenza} \qquad\quad
\end{matrix}\]
Il l'energia dissipata al momento di apertura del circuito (puramente resistivo con \(R_{eq} = R'\)) è pari a:
\[W_\text{tot} = \frac{1}{2} Li^2 \quad \leftarrow \text{solamente energia immagazzinata nell'induttanza}\]
Si osserva, quindi, che l'induttanza immagazzina un'energia \(U_L\) ed è possibile definire l'energia del campo magnetico
(sostituendo per un opportuno \(L\)):
\[ \quad U_L = \frac{1}{2} Li^2 \qquad\qquad U_B = \frac{1}{2\mu_0} B^2 \tau \qquad \left(U_e = \frac{1}{2}\varepsilon_0 E^2 \tau\right)\]

\subsubsection*{Mutua induzione}
Si immagina di avere due circuiti RC tali per cui il campo generato dall'induttanza di un circuito investe completamente
l'induttanza dell'altro circuito e viceversa.
\[\begin{matrix}
	\Phi_{12} = \Phi_2(\vec{B}_1) = M_{12} \; i_1 \leftarrow \text{flusso del campo del circuito 1 agente sul circuito 2} \\
	\Phi_{21} = \Phi_1(\vec{B}_2) = M_{21} \; i_2 \leftarrow \text{flusso del campo del circuito 2 agente sul circuito 1}
\end{matrix}\]
Si dimostra che \(M_{12} = M_{21} = M\) ed è chiamato coefficiente di mutua induzione.\\
Si definisce l'energia immagazzinata dalla mutua induzione come il lavoro compiuto dal generatore di uno dei due circuiti
sull'altro, ottenendo:
\[W_m = U_m = M i_1 i_2 + \frac{1}{2} L_1 {i_1}^2 + \frac{1}{2} L_2 {i_2}^2 \qquad M i_1 i_2 \leftarrow \text{lavoro di mutua induzione} \]

\subsection{Circuiti a corrente alternata}
\subsubsection*{Circuito LC}
In un circuito LC si ha un condensatore carico, una induttanza e un interruttore inizialmente aperto. Alla chiusura del circuito,
il condensatore si scarica creando una corrente \(i\) e l'induttanza crea autoinduttanza.
\[q(t) = q_0 \cos (\omega t) \qquad i(t) = \omega \, q_0 \sin(\omega t) \qquad V_C(t) = \frac{q_0}{C} \cos(\omega t) \qquad V_L(t) =  -V_C(t) \qquad \omega = \frac{1}{\sqrt{LC}}\]
L'energia totale si conserva nel tempo e si distribuisce tra il condensatore e l'induttanza.
\[U_\text{tot} = U_C(t) + U_L(t) = \frac{1}{2} \frac{{q_0}^2}{C} \cos^2(\omega t) + \frac{1}{2} \frac{{q_0}^2}{C} \sin^2(\omega t)\]

\subsubsection*{Circuito RLC senza generatori}
Un circuito RLC senza generatore si comporta come un circuito LC con la presenza di un fattore di smorzamento.
\[i(t) = A e^{-\gamma t} \cos(\omega t + \theta) \qquad T_\text{pseudoperiodo} = \frac{2\pi}{\omega} = \frac{2\pi}{\sqrt{\omega_0^2 - \gamma^2}} \qquad \omega = \sqrt{\omega_0^2 - \gamma^2} \qquad \gamma \leftarrow \text{smorzamento}\]

\subsubsection*{Circuito R + AC}
In un circuito formato da una resistenza e un generatore di corrente alternata si ha che la corrente e la tensione sono in fase
tra di loro:
\[\mathcal{E}(t) = \mathcal{E}_0 \cos(\omega t) \qquad i(t) = \frac{\mathcal{E}(t)}{R} = \frac{\mathcal{E}_0}{R} \cos(\omega t) \qquad V_R(t) = R i_0 \cos(\omega t)\]

\subsubsection*{Circuito L + AC}
In un circuito formato da una induttanza e un generatore di corrente alternata si ha che la corrente e la tensione sono in ritardo
di \(\pi/2\) rispetto alla forza elettromotrice, ovvero c'è uno sfasamento di \(\pi/2\)
\[\mathcal{E}(t) = \mathcal{E}_0 \cos(\omega t) \qquad \mathcal{E} = L \frac{di}{dt} \;\;\rightarrow\;\; i(t) = \frac{\mathcal{E}_0}{L\omega} \cos \left( \omega t + \frac{\pi}{2} \right) \qquad V_L(t) = L \omega i_0 \cos \left( \omega t + \frac{\pi}{2} \right)\]
È possibile definire la reattanza dell'induttanza, ovvero la resistenza associata all'induttanza come \(Z_L\):
\[Z_L = L\omega \qquad i(t) = \frac{\mathcal{E}_0}{Z_L} \cos \left( \omega t + \frac{\pi}{2} \right) \qquad V_L(t) = Z_L i_0 \cos(\omega t + \frac{1}{2})\]

\subsubsection*{Circuito C + AC}
In un circuito formato da un condensatore e un generatore di corrente alternata si ha che la corrente e la tensione sono in
anticipo di \(\pi/2\) rispetto alla forza elettromotrice, per cui c'è uno sfasamento di \(\pi/2\)
\[\mathcal{E} = \mathcal{E}_0 \cos(\omega t) \qquad q = C \mathcal{E} \;\;\rightarrow\;\; i(t) = \frac{dq}{dt} = C \omega \mathcal{E}_0 \cos \left( \omega t + \frac{\pi}{2} \right) \qquad V_C(t) = \frac{q(t)}{C} = \frac{i_0}{\omega C} \cos \left( \omega t + \frac{\pi}{2} \right)\]
È possibile definire la reattanza del condensatore \(Z_C\):
\[Z_C = \frac{1}{C \omega} \qquad i(t) = \frac{\mathcal{E}}{Z_C} \cos \left( \omega t + \frac{\pi}{2} \right) \qquad V_C(t) = Z_C i_0 \cos \left( \omega t + \frac{\pi}{2} \right)\]

\subsubsection*{Circuito RL + AC}
In un circuito formato da una resistenza, un'induttanza e un alternatore si ha uno sfasamento \(\theta\) che dipende dai valori
di \(R\) e \(L\) e si definisce la reattanza del circuito \(Z_{RL}\)
\[Z_{RL} = \sqrt{R^2 + {Z_L}^2} = \sqrt{R^2 + \omega^2 L^2} \qquad \tan \theta = \frac{Z_L}{R} = \frac{\omega L}{R}\]

\subsubsection*{Circuito RC + AC}
In un circuito formato da una resistenza, un condensatore e un alternatore si ha uno sfasamento \(\theta\) che dipende dai valori
di \(R\) e \(C\) e si definisce la reattanza del circuito \(Z_{RC}\)
\[Z_{RC} = \sqrt{R^2 + {Z_C}^2} = \sqrt{R^2 + \frac{1}{\omega^2 C^2}} \qquad \tan \theta = -\frac{Z_C}{R} = -\frac{1}{RC\omega}\]

\subsubsection*{Circuito RLC con generatore}
In un circuito formato da una resistenza, un'induttanza un condensatore e un alternatore si ha uno sfasamento \(\theta\) che
dipende dai valori di \(R\), \(L\) e \(C\) e si definisce la reattanza del circuito \(Z\)
\[Z = \sqrt{R^2 + (Z_L-Z_C)^2} = \sqrt{R^2 + \left( \omega L - \frac{1}{\omega C} \right)^2} \qquad \tan \theta = \frac{Z_L - Z_C}{R} = \frac{\omega L - \frac{1}{\omega C}}{R}\]
Definendo \(\omega_0\) la pulsazione del circuito LC e \(\omega\) la pulsazione dell'alternatore, si ha risonanza per \(\omega = \omega_0\)
in particolare lo sfasamento è nullo. Si definisce l'ampiezza della risonanza \(\Delta \omega\) e il fattore di merito \(Q\).
\begin{align*}
	&\omega_0 = \frac{1}{\sqrt{LC}} \quad \omega = \omega_{res} = \omega_0 = \frac{1}{\sqrt{LC}} \;\; \rightarrow \;\; \tan \theta = 0 \\
	&\Delta \omega = \omega_2 - \omega_1 = \frac{R}{L} \quad \text{con} \; i_0(\omega_1) = i_0(\omega_2) = \frac{i_0(\omega_{res})}{\sqrt{2}}
	&Q := \frac{\omega_0}{\Delta \omega} = \omega_0 \frac{L}{R}
\end{align*}
L'energia del generatore è completamente dissipata dalla resistenza.

\newpage

\section{Onde elettromagnetiche}
\subsection{Legge di Ampère-Maxwell}
Dalla legge di Ampère sulla circuitazione del campo magnetico si ha:
\[\oint \vec{B} \cdot d\vec{s} = \mu_0(i_\text{conduzione} + i_\text{spostamento})\]
Immaginando di applicare il teorema per una circonferenza tra le armature del condensatore, si ottiene che:
\[i_\text{spost} = \frac{dq}{dt} = \dt CV = \dt \left(\frac{\varepsilon_0 \Sigma}{h} \; V\right) = \dt (\varepsilon_0 E \Sigma) = \varepsilon_0 \dt (\Sigma E) = \varepsilon_0 \frac{d \Phi(\vec{E})}{dt}\]
Per cui l'equazione di Ampère con la correzione di Maxwell diventa:
\[\oint \vec{B} \cdot d\vec{s} = \mu_0 \left( i_\text{conduzione} + \varepsilon_0 \frac{d \Phi(\vec{E})}{dt} \right)\]

\subsection{Leggi di Maxwell}
\subsubsection*{Equazioni di Maxwell}
Le equazioni di circuitazione e flusso dei campi \(\vec{E}\) e \(\vec{B}\) prendono il nome di equazioni di Maxwell:
\begin{align*}
	&1. \quad \Phi_\Sigma(\vec{E}) = \oint_\Sigma \vec{E} \cdot d\vec{\Sigma} = \frac{q}{\varepsilon_0} 
	&& \nab \cdot \vec{E} = \frac{\rho}{\varepsilon_0} \\
	&2. \quad \Phi_\Sigma(\vec{B}) = \oint_\Sigma \vec{B} \cdot d \vec{\Sigma} = 0 
	&& \nab \cdot \vec{B} = 0 \\
	&3. \quad \Gamma_\mathcal{L}(\vec{E}) = \oint_\mathcal{L} \vec{E} \cdot d\vec{s} = - \frac{d\Phi_\Sigma \vec{B}}{dt}
	&& \nab \times \vec{E} = - \frac{\partial \vec{B}}{\partial t} \\
	&4. \quad \Gamma_\mathcal{L}(\vec{B}) = \oint_\mathcal{L} \vec{B} \cdot d\vec{s} = \mu_0 (i_\text{concatenata} + \varepsilon_0 \frac{d \Phi_\Sigma \vec{E}}{dt})
	&& \nab \times \vec{B} = \mu_0 \vec{j} + \mu_0 \, \varepsilon_0 \frac{\partial \vec{E}}{\partial t}
\end{align*}

\subsubsection*{Equazioni di Maxwell in assenza di conduttori}
In assenza di conduttori, si hanno le seguenti due equazioni:
\begin{align*}
	&1. \quad \Phi_\Sigma(\vec{E}) = \oint_\Sigma \vec{E} \cdot d\vec{\Sigma} = 0
	&& \nab \cdot \vec{E} = 0\\
	&2. \quad \Phi_\Sigma(\vec{B}) = \oint_\Sigma \vec{B} \cdot d \vec{\Sigma} = 0 
	&& \nab \cdot \vec{B} = 0 \\
	&3. \quad \Gamma_\mathcal{L}(\vec{E}) = \oint_\mathcal{L} \vec{E} \cdot d\vec{s} = - \frac{d\Phi_\Sigma \vec{B}}{dt}
	&& \nab \times \vec{E} = - \frac{\partial \vec{B}}{\partial t} \\
	&4. \quad \Gamma_\mathcal{L}(\vec{B}) = \oint_\mathcal{L} \vec{B} \cdot d\vec{s} = \mu_0 \, \varepsilon_0 \frac{d \Phi_\Sigma \vec{E}}{dt} \qquad\qquad\qquad\;
	&& \nab \times \vec{B} = \mu_0 \, \varepsilon_0 \frac{\partial \vec{E}}{\partial t} \qquad\;\;\;
\end{align*}

\subsubsection*{Conseguenze delle equazioni di Maxwell}
\begin{itemize}[topsep=3pt, itemsep=0pt]
	\item[-] dalla prima si ottiene che esiste il monopolo elettrico, ovvero esistono cariche puntiformi
	\item[-] dalla seconda si ottiene che non esiste il monopolo magnetico
	\item[-] dalle forme locali si ottiene che \(\nab \cdot \vec{j} = - \partial \rho / \partial t\), ovvero che in assenza di
	densità di correnti (e di correnti) la carica si conserva
\end{itemize}

\newpage

\subsection{Onde piane}
\subsubsection*{Definizione}
Un'onda è la propagazione dell'energia e della quantità di moto in un mezzo o nel vuoto. Le onde meccaniche si propagano in un
mezzo (suono, vibrazione corda, onde mare, \dots), mentre le onde elettromagnetiche si propagano anche nel vuoto. 

\subsubsection*{Equazioni delle onde}
\begin{itemize}[topsep=3pt, itemsep=0pt]
	\item[-] l'equazione di un'onda generica è: \(f(\vec{s}, \vec{t}) = f(x,y,z,t)\)
	\item[-] l'equazione di un'onda piana che si propaga in un'unica direzione è: \(f(\vec{s}, \vec{t}) = f(z,t)\)
	\item[-] un'onda è descritta dall'equazione \(\displaystyle \frac{\partial^2 f}{\partial z^2} = \frac{1}{v^2} \, \frac{\partial^2 f}{\partial t^2}\)
	con \(v\) è la velocità di propagazione dell'onda
	\item[-] le onde piane armoniche sono onde del tipo \(f(z,t) = f_0 \sin(k(z-vt))\) o \(f(z,t) = f_0 \cos(k(z-vt))\) con:
	\begin{itemize}[topsep=0pt, itemsep=0pt]
		\item[-] \(k = 2\pi/\lambda\) costante per convertire i metri in radianti, onde per unità di spazio
		\item[-] \(\lambda = 2 \pi / k\) lunghezza d'onda
		\item[-] \(\omega = 2 \pi v / \lambda\) velocità angolare
		\item[-] \(f = \omega / 2 \pi\) frequenza 
		\item[-] \(T = 1/f\) periodo
		\item[-] vale la relazione \(\lambda f = v\) tra lunghezza d'onda, frequenza e velocità
	\end{itemize}
\end{itemize}

\subsection{Onde elettromagnetiche piane}
Le onde elettromagnetiche sono onde piane del tipo:
\[\vec{E}(z,t) = \left(\begin{matrix} E_{0x} \cos (kz - \omega t) \\ E_{0y} \cos (kz - \omega t) \\ E_{0z} \cos (kz - \omega t) \end{matrix}\right) \qquad \qquad
\vec{B}(z,t) = \left(\begin{matrix} B_{0x} \cos (kz - \omega t) \\ B_{0y} \cos (kz - \omega t) \\ B_{0z} \cos (kz - \omega t) \end{matrix}\right)\]
Dalle equazioni di Maxwell si ottiene che le onde si propagano solo in piani \(xy\) perpendicolari alla direzione di
propagazione \(z\), ovvero si ha che \(\vec{E}_z = \vec{B}_z = 0\):
\[\vec{E}(z,t) = \left(\begin{matrix} E_{0x} \cos (kz - \omega t) \\ E_{0y} \cos (kz - \omega t) \\ 0 \end{matrix}\right) \qquad \qquad
\vec{B}(z,t) = \left(\begin{matrix} B_{0x} \cos (kz - \omega t) \\ B_{0y} \cos (kz - \omega t) \\ 0 \end{matrix}\right)\]

\subsection{Velocità della luce}
Dalle equazioni di Maxwell e dalla equazione generale delle onde si ottiene che la velocità di propagazione è pari alla costante 
\(c\) identificata come velocità della luce:
\[v = c = \frac{1}{\mu_0 \, \varepsilon_0} \approx 3 \cdot 10^5 km/s\]
In un mezzo diverso dal vuoto si ha una velocità inferiore a \(c\) per un fattore dipendente dalla costante dielettrica del mezzo:
\[v_\kappa = \frac{1}{\kappa \, \mu_0 \, \varepsilon_0} = \frac{c}{\sqrt{\kappa}}\]

\newpage

\subsection{Proprietà delle onde elettromagnetiche piane}
\subsubsection*{Perpendicolarità tra campi}
Si osserva che dalle equazioni di Maxwell è possibile definire una relazione tra le onde del campo elettrico e magnetico:
\[\frac{\partial E_y}{\partial z} = \frac{\partial B_x}{\partial t} \qquad \frac{\partial E_x}{\partial z} = - \frac{\partial B_y}{ \partial t} \qquad\qquad
\frac{\partial E_x}{\partial t} = -\frac{1}{\mu_0 \, \varepsilon_0} \frac{\partial B_y}{\partial z} \qquad \frac{\partial E_y}{\partial t} = \frac{1}{\mu_0 \varepsilon_0} \frac{\partial B_y}{\partial z}\]
da cui si ottiene che i due campi sono perpendicolari tra di loro
\[B = \frac{E}{c} \qquad\qquad \vec{E} \cdot \vec{B} = 0 \qquad\qquad \vec{E} \times \vec{B} = \left( \begin{matrix} 0 \\ 0 \\ E^2/c \end{matrix} \right) = \left( \begin{matrix} 0 \\ 0 \\ cB^2 \end{matrix} \right)\]

\subsubsection*{Energia delle onde elettromagnetiche}
L'energia trasportata delle onde elettromagnetiche vale:
\[U_e = \frac{1}{2} \varepsilon_0 E^2 \quad U_m = \frac{1}{2\mu_0} B^2 \qquad U_\text{onda} = U_e + U_m = \frac{1}{2} \varepsilon_0 E^2 + \frac{1}{2\mu_0} B^2 = \varepsilon_0 E^2 = \frac{B^2}{\mu_0}\]

\subsubsection*{Vettore di Poynting}
Si definisce il vettore di Poynting \(\vec{S}\) tale per cui la potenza trasportata dall'onda è il flusso di tale vettore.
\[\vec{S} = \frac{1}{\mu_0} \vec{E} \times \vec{B} \qquad \qquad P_\text{otenza} = \int_\Sigma \vec{S} \cdot d\vec{\Sigma}\]

\subsubsection*{Intensità dell'onda}
Si definisce l'intensità dell'onda elettromagnetica \(I\) e l'ampiezza efficace \(E_{eff}\):
\[I := S_\text{medio nel tempo} = \frac{1}{2} \, \varepsilon_0 \, c \, {E_0}^2 = \varepsilon_0 \, c \, E_{eff}^2 \qquad\qquad E_{eff} = \frac{E_0}{\sqrt{2}}\]

\subsubsection*{Pressione di radiazione}
Si definisce la pressione di radiazione di un'onda elettromagnetica:
\[P = I/c = \frac{1}{2} \varepsilon_0 E^2 = \varepsilon_0 {E_{eff}}^2\]
Definendo \(\vec{p}\) la quantità di moto associata all'onda e \(\eta\) l'indice di riflessione dell'onda su una superficie
\(\Sigma\), la variazione totale di quantità di moto è: \(\Delta \vec{p} = -(1+\eta) \vec{p}\). La pressione di radiazione
esercitata su una superficie \(\Sigma\) da un'onda con riflessione \(\eta\) è:
\[P = (1 + \eta)I/c = (1+\eta) \varepsilon_0 \, {E_\text{eff}}^2\]

\subsection{Onde sferiche}
Per le onde sferiche si ha che \(E_0(r) \propto 1/r\), ovvero l'ampiezza dei campi è inversamente proporzionale al raggio della
distanza dalla sorgente. Valgono quindi:
\[E(r,t) = \frac{E_0}{r} \cos (kr -\omega t) \qquad B(r,t) = \frac{B_0}{r} \cos (kr -\omega t) \qquad S_m = I = \frac{1}{2} \, c \, \varepsilon_0 \, \frac{{E_0}^2}{r^2} \]

\subsection{Polarizzazione}
Si definisce la polarizzazione delle onde elettromagnetico usando il campo \(\vec{E}\) (il campo \(\vec{B}\) è \(\perp\)):
\begin{align*}
	&1. \quad \text{onda polarizzata linearmente}: &&\vec{E}(z,t) = \left(\begin{matrix} E_0 \cos (kz - \omega t) \cos \theta \\ E_0 \cos (kz - \omega t) \sin \theta \\ 0 \end{matrix}\right) \\
	&2. \quad \text{onda polarizzata circolarmente}: &&\vec{E}(z,t) = \left(\begin{matrix} E_0 \cos (kz - \omega t) \\ E_0 \cos (kz - \omega t) \\ 0 \end{matrix}\right) \\
	&3. \quad \text{onda polarizzata ellitticamente}: &&\vec{E}(z,t) = \left(\begin{matrix} E_0 \cos (kz - \omega t) \\ E_0 \cos (kz - \omega t) \\ 0 \end{matrix}\right) \qquad\qquad\qquad\qquad
\end{align*}

\subsection{Spettro elettromagnetico}
In base alla lunghezza d'onda, le onde elettromagnetiche si dividono in:
\begin{center}
	\begin{tabular}{l | c | c}
		\toprule
		Onde radio & \(\leq 250 \cdot 10^6\) Hz & 10 km - 10 cm \\
		\midrule
		Microonde & \(3 \cdot 10^9\) Hz - \(300 \cdot 10^9\) Hz & 10 cm - 1 mm \\
		\midrule
		Infrarossi & \(300 \cdot 10^9\) Hz - \(428 \cdot 10^{12}\) Hz & 1 mm - 700 nm \\
		\midrule
		Luce visibile & \(428 \cdot 10^{12}\) Hz - \(729\cdot 10^{12}\) Hz & 700 nm - 400 nm \\
		\midrule
		Ultravioletto & \(729\cdot 10^{12}\) Hz - \(30 \cdot 10^{15}\) Hz & 400 nm - 10 nm \\
		\midrule
		Raggi X & \(30 \cdot 10^{15}\) Hz - \(300 \cdot 10^{18}\) Hz & 10 nm - 1 pm \\
		\midrule
		Raggi gamma & \(\geq 300 \cdot 10^{15}\) Hz & \(\leq\) 1 pm \\
		\bottomrule
	\end{tabular}
\end{center}

\end{document}
