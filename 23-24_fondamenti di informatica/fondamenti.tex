\documentclass{article}
\usepackage[utf8]{inputenc}
\usepackage[italian]{babel} % corretta sillabazione in italiano
\usepackage{geometry} % per impostare margini e layout pagina
\usepackage{enumitem}
\usepackage{multirow} % per celle che si espandono su più righe
\usepackage{tabularx} % per tabelle con larghezza flessibile
\usepackage{booktabs} % per linee orizzontali tabelle
\usepackage{hyperref} % per collegamenti

\geometry{a4paper,left=25mm, right=25mm, bottom=25mm, top=30mm}

\renewcommand\tabularxcolumn[1]{m{#1}}

\title{Appunti di Fondamenti di Informatica}
\author{Giacomo Simonetto}
\date{Primo semetre 2023-24}

\begin{document}

% -------------------------------------- Copertina e indice ---------------------------------------
\maketitle
\begin{abstract}
	Appunti del corso di Fondamenti di Informatica della facoltà di Ingegneria Informatica dell'Università di Padova.
\end{abstract}

\newpage

\tableofcontents

\newpage

% -------------------------------------------- Storia ---------------------------------------------
\section{Storia dell'informatica}
I primi tentativi della ricerca di un linguaggio formale risalgono alla fine del 1800. Ci si chiede se la matematica
sia un sistema formale completo e se esiste un procedimento meccanico (passo-passo, finito) per dimostrare se una
proposizione sia vera o falsa.

Il primo tentavo di \textit{"formalizzazione della matematica"} viene svolto da David Hilbert, con cui si scopre che
la matematica possiede 23 problemi di formalizzazione chiamati \textit{"23 problemi di Hilbert"}.

La risposta alla prima domanda risale al 1931 quando Goedel, con il \textit{"teorema di incompletezza"} conferma che
la matematica non è un sistema formale.

Nel 1936 Church, Turing e Kleene elaborano dei formalismi meccanici tra cui la \textit{Macchina di Turing} e la
\textit{Tesi di Church-Turing} che sostiene che tutto ciò che è computabile è computabile dalla macchina di Turing
universale.

Nel 1943 si arriva a costruire l'\textit{ENIAC}, il primo computer (general purpose) della storia.

La capacità computazionale tra una macchina di Turing e un computer odierno è la stessa (ecceto per il
fatto che la macchina di Turing prevedeva uno spazio di archiviazione illimitato), cambia solo la velocità computazionale.
Entrambe le macchine risolvono gli stessi problemi, ovvero tutti quelli che si possono risolvere con un algoritmo.


% ------------------------------ Algoritmi e pensiero computazionale ------------------------------
\section{Algoritmo}
Un algoritmo è un metodo di risoluzione di un problema che:
\begin{itemize}[topsep=5pt, itemsep=0pt]
	\item[-] deve essere eseguibile
	\item[-] non deve essere ambiguo
	\item[-] deve concludersi in un numero finito di passi
\end{itemize}

\section{Computational Thinking}
Per computational thinking, o pensiero computazionale, si intende l'insieme delle abilità che permettono di astrarre
il problema e tradurlo in algoritmo. Comprende le tecniche di astrazione/risoluzione di problemi algoritmici tra
cui la decomposizione di problemi complessi e la modularità.

\newpage


% ------------------------------- Computer e Modello di Von Neumann -------------------------------
\section{Computer}
Per computer, o calcolatore, si intende un sistema di elaborazione e memorizzazione di informazioni che opera sotto
il controllo di un programma. È composto da hardware (parte fisica) e software (programmi e dati). I dati possono
essre di diverso tipo (immagini, testi, audio, video, \dots) e sono rappresentati elettricamente in 0 e 1.

Esistono diversi tipi di computer (workstation, smartphone, \dots) che possono svolgere diversi tipi di impieghi
(elettrodomestici, giochi, fotografie, \dots).


\section{Modello di Von Neumann}
Il modello di Von Neumann è una rappresentazione dell'archietttura di un elaboratore. Prevede la presenza di 4 blocchi
la CPU, la memoria primaria, la memoria secondaria e i dispositivi di I/O collegati insieme grazie al BUS.

Inoltre sono presenti due diversi flussi di informazioni: quello di dati è bidirezionale, mentre quello degli indirizzi
e dei segnali di controllo è unidirezionale con direzione CPU \(\rightarrow\) altri dispositivi.

\subsection{Central Processing Unit o CPU}
\subsubsection*{Compiti}
La Central Processing Unit ha il compito di:
\begin{itemize}[topsep=5pt, itemsep=0pt]
	\item[-] individuare ed eseguire le istruzioni
	\item[-] elaborare dati attraverso la ALU (Unità Logico Aritmetica)
	\item[-] reperire dati di input e restituire dati di output
\end{itemize}

\subsubsection*{Blocchi}
È costituita da tre blocchi:
\begin{center}	
	\begin{tabularx}{\textwidth}{r X}
		\textbf{ALU}: & anche chiamata \textit{Arithmetic Logical Unit}, risolve le espressioni logico - algebriche \\
		\midrule
		\textbf{registri}: & memoria temporanea per dati che devono essere subito elaborati \\
		\midrule
		\textbf{unità di controllo}: & costituita dal \textbf{Program Counter}, o \textit{PC}, che memorizza l'indirizzo dell'istruzione successiva da eseguire e dall'\textbf{Instruction Register}, o \textit{IR}, ovvero il registro che memorizza l'istruzione in esecuzione
	\end{tabularx}
\end{center}

\subsubsection*{Funzionamento}
La CPU ha funzionamento ciclico che si divide in tre fasi. La velocità di una CPU è espressa in cicli al secondo (dell'ordine dei GHz).
\begin{center}
	\begin{tabularx}{\textwidth}{c c X}
		1° fase & fetch & caricamento \dots \\
		\midrule
		2° fase & decode & decodifica \dots \\
		\midrule
		3° fase & execute & esecuzione \dots
	\end{tabularx}
\end{center}


\subsection{Memoria primaria}
\subsection{Memoria secondaria}
\subsection{Dispositivi di I/O}


% --------------------------------------- Sistemi operativi ---------------------------------------
\section{Sistemi operativi}
\subsection{Unix e Linux}


% ------------------------------ Rappresentazione delle informazioni ------------------------------
\section{Rappresentazione delle informazioni nei calcolatori}
\subsection{Sistema posizionale}
\subsection{Rappresentazione in modulo - segno}
\subsection{Rappresentazione in complemento a 2}
\subsection{Rappresentazione in virgola fissa}
\subsection{Rappresentazione in virgola mobile}

% ---------------------------------- Linguaggi di programmazione ----------------------------------
\section{Linguaggi di programmazione}

% --------------------------------------------- Java ----------------------------------------------
\section{Java}

\end{document}