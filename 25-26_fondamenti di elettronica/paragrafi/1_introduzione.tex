\section{Introduzione}

\subsection{Definizioni fondamentali}
\begin{itemize}
	\item \textbf{elettronica}: studia e realizza sistemi elettronici;
	\item \textbf{sistema elettronico}: è un insieme di componenti elettronici (sensori, circuiti e attuatori) che raccolgono
	informazioni dal mondo reale attraverso sensori, le elaborano attraverso circuiti elettronici e prendono decisioni o
	comandano azioni con degli attuatori;
	\item \textbf{segnale}: supporto fisico di natura qualunque (elettrica, acustica, ottica) a cui si associa un'informazione
	allo scopo di poterla trasferire da una sorgente ad un utilizzatore, può essere digitale (ampiezza e tempo discreti) o
	analogico (ampiezza e tempo continui);
	\item \textbf{sensore}: dispositivo che converte un segnale esterno (come temperatura, pressione, luce, suono) in una grandezza
	elettrica (come corrente o tensione);
	\item \textbf{circuito elettronico}: rete di componenti elettrici passivi (R, L, C) e attivi (diodi, transistor) che elaborano
	segnali elettrici (tensione e corrente). In base al tipo di segnale elaborato si distingue in:
	\begin{itemize}[topsep=0pt]
		\item \textbf{circuito analogico}: elabora segnali analogici;
		\item \textbf{circuito digitale}: elabora segnali digitali;
		\item \textbf{circuito misto}: opera in entrambi i domini del segnale.
	\end{itemize}
	Siccome i segnali provenienti dal mondo reale sono sempre analogici, in generale non esiste un sistema completamente digitale.
	Ogni sistema digitale, infatti, comprende un ADC (Analog-to-Digital Converter) in ingresso e un DAC (Digital-to-Analog Converter)
	in uscita.

	In base alla realizzazione fisica, un circuito elettronico si distingue in:
	\begin{itemize}[topsep=0pt]
		\item \textbf{circuito a elementi discreti}: realizzato con componenti costruiti separatamente che poi vengono montati su
		un supporto (breadboard, PCB) e collegati tra loro tramite fili o piste conduttive;
		\item \textbf{circuito integrato (IC)}: tutti i componenti sono miniaturizzati e vengono montati su un unico chip di silicio
		(es. microchip).
	\end{itemize}
	Un sistema elettronico completo è formato da circuiti integrati e componenti discreti montati in una scheda in cui sono
	realizzate le interconnessioni metalliche tra i terminali dei componenti
\end{itemize}

\subsection{Settori dell'elettronica}
\begin{itemize}
	\item \textbf{elettronica analogica}: progettazione e analisi di circuiti che elaborano segnali analogici;
	\item \textbf{elettronica digitale}: progettazione e analisi di circuiti che elaborano segnali digitali;
	\item \textbf{elettronica di consumo}: dispositivi elettronici per l'uso personale e domestico (computer, telefoni cellulari, televisori, elettrodomestici);
	\item \textbf{microelettronica}: progettazione e fabbricazione di componenti elettronici e circuiti integrati;
	\item \textbf{elettronica di potenza}: conversione e gestione dell'energia elettrica a diversi livelli (dal riscaldamento agli alimentatori per pc, cellulari o altri strumenti);
	\item \textbf{elettronica industriale}: sistemi elettronici per processi produttivi automatizzati;
	\item \textbf{telecomunicazioni}: sistemi per la trasmissione di dati (voce, video, file) attraverso dispositivi mobili o fissi;
	\item \textbf{biomedica}: sviluppo di apparecchiature elettroniche per la diagnostica, la cura e il monitoraggio della salute;
	\item \textbf{automotive}: sistemi per il controllo dei veicoli (dallo specchietto fino alla guida autonoma);
	\item \textbf{informatica}: dispositivi e sistemi elettronici per la gestione dei dati.
\end{itemize}

\newpage


\subsection{Richiamo di teoria dei circuiti}
\subsubsection*{Leggi di Kirchoff}
\begin{itemize}
	\item \textbf{Legge delle correnti (LKC)}: la somma delle correnti entranti in un nodo è uguale alla somma delle correnti uscenti.
	\item \textbf{Legge delle tensioni (LKT)}: la somma delle tensioni lungo una maglia è uguale a zero.
\end{itemize}

\subsubsection*{Elementi passivi}
\begin{center}
	\begin{tabular}{l c c c }
		\textbf{resistore (R)}: & \begin{circuitikz}[baseline=(current bounding box.center)]
			\draw (0,0) to[R] (2,0);
			\draw [->] (0.3,0.4) -- (1.7,0.4) node[midway, above] {\(i_R\)};
			\node at (0.3,-0.4) {+};
			\node at (1,-0.4) {\(v_R\)};
			\node at (1.7,-0.4) {--};
		\end{circuitikz} & \(\displaystyle \qquad v_R(t) = R \cdot i_R(t)\) & \(\displaystyle i_R(t) = \frac{v_R(t)}{R}\) \\
		\midrule
		
		\textbf{condensatore (C)}: & \begin{circuitikz}[baseline=(current bounding box.center)]
			\draw (0,0) to[C] (2,0);
			\draw [->] (0.2,0.5) -- (1.8,0.5) node[midway, above] {\(i_C\)};
			\node at (0.3,-0.6) {+};
			\node at (1,-0.7) {\(v_C\)};
			\node at (1.7,-0.6) {--};
		\end{circuitikz} & \(\displaystyle \qquad i_C(t) = C \frac{d v_C(t)}{d t}\) & \(\displaystyle v_C(t) = \frac{1}{C} \int_0^t i_C(t) \, dt + \frac{I_0}{C}\) \\
		\midrule
		
		\textbf{induttore (L)}: & \begin{circuitikz}[baseline=(current bounding box.center)]
			\draw (0,0) to[L] (2,0);
			\draw [->] (0.3,0.4) -- (1.7,0.4) node[midway, above] {\(i_L\)};
			\node at (0.3,-0.4) {+};
			\node at (1,-0.4) {\(v_L\)};
			\node at (1.7,-0.4) {--};
		\end{circuitikz} & \(\displaystyle \qquad v_L(t) = L \frac{d i_L(t)}{d t}\) & \(\displaystyle i_L(t) = \frac{1}{L} \int_0^t v_L(t) \,dt + \frac{V_0}{L}\)
	\end{tabular}
\end{center}

\subsubsection*{Elementi attivi}
\begin{center}
	\begin{tabular}{p{3.5cm} c p{8.6cm}}
		\textbf{generatore ideale di tensione} (GIT): &
		\begin{circuitikz}[baseline=(current bounding box.center), american voltages, american currents] \draw (0,0) to[V] (2,0); \end{circuitikz} &
		fornisce una tensione costante indipendentemente dalla corrente che lo attraversa \\[0.7cm]
		\midrule
		\textbf{generatore ideale di corrente} (GIC): &
		\begin{circuitikz}[baseline=(current bounding box.center), american voltages, american currents] \draw (0,0) to[I] (2,0); \end{circuitikz} &
		fornisce una corrente costante indipendentemente dalla tensione ai suoi capi \\[0.7cm]
		\midrule
		\textbf{diodi e transistor}: & \dots & componenti non lineari che verranno studiati successivamente.
	\end{tabular}
\end{center}

\subsubsection*{Principi di analisi dei circuiti}
\begin{itemize}
	\item \textbf{partitore di tensione}: due resistori in serie dividono la tensione in ingresso \(V_{in}\) in due tensioni
	\(V_{1}\) e \(V_{2}\) direttamente proporzionali alle resistenze (inversamente proporzionali alle conduttanze):
	\[ V_{1} = V_{in} \cdot \frac{R_1}{R_1 + R_2} \qquad\qquad V_{2} = V_{in} \cdot \frac{R_2}{R_1 + R_2}\]
	\item \textbf{partitore di corrente}: due resistori in parallelo dividono la corrente in ingresso \(I_{in}\) in due correnti
	\(I_{1}\) e \(I_{2}\) inversamente proporzionali alle resistenze (direttamente proporzionali alle conduttanze):
	\[ I_{1} = I_{in} \cdot \frac{R_2}{R_1 + R_2} \qquad\qquad I_{2} = I_{in} \cdot \frac{R_1}{R_1 + R_2}\]
	\item \textbf{sovrapposizione degli effetti}: dato un sistema lineare con \(C_1\), \(C_2\) possibili ingressi ed \(E_1\),
	\(E_2\) effetti prodotti in uscita dai due ingressi, se il sistema viene perturbato con un ingresso dato dalla composizione
	lineare dei due ingressi \(C = p_1 C_1 + p_2 C_2\) con \(p_1\) e \(p_2\) pesi dei due ingressi, l'effetto risultante in uscita
	sarà la composizione lineare dei due effetti \(E = p_1 E_1 + p_2 E_2\).

	In particolare in un circuito lineare con più generatori, la risposta (tensione o corrente) in un componente è uguale alla
	somma algebrica delle risposte dovute a ciascun generatore preso singolarmente, con gli altri generatori sostituiti dai loro
	rispettivi cortocircuiti (generatore di tensione ideale) o circuiti aperti (generatore di corrente ideale).
\end{itemize}

\subsubsection*{Potenziali, tensioni e nodi di riferimento}
\begin{itemize}
	\item Il potenziale elettrico è definito a meno di una costante, per cui anche la soluzione di una rete elettrica (data dai
	potenziali ai vari nodi) non è univoca, ma è definita a meno di una costante.
	\item Per rendere univoca la soluzione, si sceglie un nodo di riferimento a cui si assegna potenziale nullo e si calcolano
	i potenziali degli altri nodi rispetto a tale nodo.
	\item Le tensioni, invece, sono sempre definite univocamente come differenze di potenziale tra due nodi.
\end{itemize}

\vspace{0.2cm}\noindent
Esistono tre tipi di nodi di riferimento comunemente usati:
\begin{center}
	\begin{tabular}{m{3cm} c m{10.1cm}}
		nodo con potenziale di riferimento: &
		\begin{circuitikz} \draw (0, 0) node[sground]{}; \end{circuitikz} &
		nodo con il potenziale di riferimento scelto arbitrariamente a \(0V\) \\[0.3cm]
		\midrule
		nodo di massa: &
		\begin{circuitikz} \draw (0, 0) node[cground, scale=0.8]{}; \end{circuitikz}&
		nodo collegato al telaio metallico del dispositivo elettronico \\[0.3cm]
		\midrule
		nodo di terra: &
		\begin{circuitikz} \draw (0, 0) node[ground]{}; \end{circuitikz} &
		nodo collegato fisicamente alla terra tramite un conduttore metallico per motivi di sicurezza; di solito coincide con il nodo
		di massa; il potenziale di terra è molto stabile e indipendentemente dalle correnti che gli elettrodomestici prelevano o
		immettono in esso
	\end{tabular}
\end{center}

\subsubsection*{Rappresentazione elettronica di un circuito}
Un circuito elettronico può essere rappresentato in due modi equivalenti:
\begin{itemize}
	\item \textbf{notazione a maglie}: rappresentazione del circuito in maglie e nodi
	\item \textbf{notazione elettronica}: scelto il nodo di riferimento, tutti i terminali collegati a tale nodo sono marcati con il
	simbolo del nodo di riferimento, inoltre i nodi di cui si conosce già il potenziale (ad esempio quelli collegati a generatori di
	tensione ideali) sono marcati con il loro valore di potenziale.
\end{itemize}

\begin{center}
	\begin{minipage}{0.4\textwidth}
		\centering
		\begin{circuitikz}[american]
			% Definisco i nodi di riferimento (per chiarezza e per replicare la struttura)
			\coordinate (A) at (0.5, 3);   % Nodo superiore sinistro (sopra Vs)
			\coordinate (B) at (2, 3);   % Nodo superiore sopra R1
			\coordinate (C) at (4, 3);   % Nodo superiore sopra i_S
			\coordinate (D) at (2, 1.5); % Nodo centrale sotto R1
			\coordinate (E) at (4, 1.5); % Nodo centrale sotto i_S
			\coordinate (F) at (4, 0);   % Nodo inferiore destro (sotto R3)
			\coordinate (G) at (0.5, 0);   % Nodo inferiore sinistro (sotto Vs)
			\coordinate (H) at (2, 0);   % Nodo inferiore sotto C
	
			% Disegno il circuito
			\draw (A) to [V, l_=\(V_S\)] (G); % Generatore di tensione Vs
			\draw (G) -- (F); % Linea inferiore
			\draw (A) -- (C); % Linea superiore
			\draw (B) to [R, l_=\(R_1\), *-*] (D); % Resistenza R1
			\draw (D) to [C, *-*] (H); % Condensatore C
			\draw (C) to [I, l_=\(I_S\)] (E); % Generatore di corrente IS
			\draw (D) to [R, l_=\(R_2\), *-*] (E); % Resistenza R2
			\draw (E) to [R, l_=\(R_3\)] (F); % Resistenza R3
		\end{circuitikz}
	\end{minipage}
	\begin{minipage}{0.05\textwidth}
		\centering
		\(\longrightarrow\)
	\end{minipage}
	\begin{minipage}{0.3\textwidth}
		\centering
		\begin{circuitikz}[american]
			% Definisco i nodi di riferimento (per chiarezza e per replicare la struttura)
			\coordinate (B) at (2, 3);   % Nodo superiore sopra R1
			\coordinate (C) at (4, 3);   % Nodo superiore sopra i_S
			\coordinate (D) at (2, 1.5); % Nodo centrale sotto R1
			\coordinate (E) at (4, 1.5); % Nodo centrale sotto i_S
			\coordinate (F) at (4, 0);   % Nodo inferiore destro (sotto R3)
			\coordinate (H) at (2, 0);   % Nodo inferiore sotto C
	
			% Disegno il circuito
			\draw (B) to [R, l_=\(R_1\)] (D); % Resistenza R1
			\draw (D) to [C] (H); % Condensatore C
			\draw (C) to [I, l_=\(I_S\)] (E); % Generatore di corrente IS
			\draw (D) to [R, l_=\(R_2\), *-*] (E); % Resistenza R2
			\draw (E) to [R, l_=\(R_3\)] (F); % Resistenza R3
	
			% Nodo di riferimento a massa
			\draw (H) node[sground]{};
			\draw (F) node[sground]{};
			\draw (B) node[rground, rotate=180]{};
			\draw (C) node[rground, rotate=180]{};
			\node at (2, 3.7) {\(V_S\)};
			\node at (4, 3.7) {\(V_S\)};
		\end{circuitikz}
	\end{minipage}
\end{center}

\subsubsection*{Potenza ed energia}
Per definire la potenza e l'energia consumata da un componente, si definiscono:
\begin{itemize}
	\item \textbf{convenzione degli utilizzatori}: la corrente entra nel terminale positivo della tensione, se la potenza o l'energia è positiva, il componente assorbe energia.
	\item \textbf{convenzione dei produttori}: la corrente entra nel terminale negativo della tensione, se la potenza o l'energia erogata è positiva, il componente fornisce energia.
\end{itemize}
Si definiscono quindi:
\begin{itemize}
	\item \textbf{potenza istantanea}: \(\displaystyle p(t) = v(t) \cdot i(t)\) misurata in Watt [W] \(=\) [J/sec]
	\item \textbf{energia}: \(\displaystyle E = \int_{t_1}^{t_2} p(t) \, dt = \int_{t_1}^{t_2} v(t) \cdot i(t) \, dt\) misurata in Joule [J] \(=\) [W \(\cdot\) sec]
\end{itemize}

\newpage

\subsection{Reti in regime transitorio}
\subsubsection*{Introduzione}
\begin{itemize}
	\item una rete si dice in regime transitorio quando le variabili elettriche (tensione e corrente) variano nel tempo passando da uno stato iniziale a uno stato finale di equilibrio
	\item un esempio di reti in regime transitorio sono i circuiti in cui sono presenti componenti reattivi (condensatori e induttori) e interruttori che modificano la configurazione del circuito
	\item il transitorio è l'intervallo di tempo che impiegano le variabili elettriche per passare dallo stato iniziale allo stato finale di equilibrio
\end{itemize}

\subsubsection*{Componenti reattivi}
\begin{center}
	\begin{tabular}{l c c c }
		componente & schema & in regime stazionario & in regime transitorio \\
		\toprule
		\textbf{condensatore}: & \begin{circuitikz}[baseline=(current bounding box.center)]
			\draw (0,0) to[C] (2,0);
			\draw [->] (0.2,0.5) -- (1.8,0.5) node[midway, above] {\(i_C\)};
			\node at (0.3,-0.6) {+};
			\node at (1,-0.7) {\(v_C\)};
			\node at (1.7,-0.6) {--};
		\end{circuitikz} & circuito aperto \(i_C = 0\) & \(\displaystyle i_C(t) = C \frac{d v_C(t)}{d t}\) \\
		\midrule
		\textbf{induttore (L)}: & \begin{circuitikz}[baseline=(current bounding box.center)]
			\draw (0,0) to[L] (2,0);
			\draw [->] (0.3,0.4) -- (1.7,0.4) node[midway, above] {\(i_L\)};
			\node at (0.3,-0.4) {+};
			\node at (1,-0.4) {\(v_L\)};
			\node at (1.7,-0.4) {--};
		\end{circuitikz} & cortocircuito \(v_L = 0\) & \(\displaystyle v_L(t) = L \frac{d i_L(t)}{d t}\)
	\end{tabular}
\end{center}


\subsubsection*{Carica di un condensatore}
\begin{itemize}
	\item condizioni iniziali (\(t < 0\)): interruttore inizialmente aperto \(\quad \rightarrow \quad v_C(0) = 0, \quad i_C(0) = 0\)
	\item nel transitorio (\(t \geq 0\)): \(\displaystyle \;\; i_C(t) = C \frac{d v_C(t)}{d t}, \quad i_R = \frac{v_R(t)}{R}, \quad V_A = v_R(t) + v_C(t), \quad i_R(t) = i_C(t)\)
	\item dalla legge delle correnti si ottiene un'equazione differenziale del primo ordine:
	\[C \frac{dv_C(t)}{d t} = \frac{v_R(t)}{R} = \frac{V_A - v_C(t)}{R} \quad \rightarrow \quad \frac{dv_C(t)}{dt} = -\frac{v_C(t)}{RC} + \frac{V_A}{RC} \quad \rightarrow \quad v_C(t) = A \cdot e^{-\tfrac{t}{RC}} + B\]
	\item si sostituisce la soluzione generale nell'equazione differenziale e si impongono le condizioni iniziali:
	\[-\frac{1}{RC} A \cdot e^{-\tfrac{t}{RC}} = -\frac{1}{RC} A \cdot e^{-\tfrac{t}{RC}} - \frac{1}{RC} B + \frac{V_A}{RC} \quad \rightarrow \quad B = V_A\]
	\[v_C(0) = 0 \quad \rightarrow \quad A + B = 0 \quad \rightarrow \quad A = -B = -V_A \]
	\item si ottengono quindi le espressioni delle variabili elettriche durante il transitorio:
	\[v_C(t) = V_A -V_A \cdot e^{-\tfrac{t}{RC}} \qquad v_R(t) = V_A e^{-\tfrac{t}{RC}} \qquad i_C(t) = i_R(t) = \frac{V_A}{R} e^{-\tfrac{t}{RC}}\]
	\item l'istante in cui la tensione sul condensatore raggiunge metà del suo valore di regime è:
	\[\frac{V_A}{2} = V_A - V_A \cdot e^{-\tfrac{t_{1/2}}{RC}} \quad \rightarrow \quad t_{1/2} = \ln(2) \cdot RC \approx 0.69 RC\]
	\item analizzando il bilancio energetico del circuito si ottiene che metà dell'energia fornita dal generatore viene
	immagazzinata nel condensatore e metà viene dissipata dalla resistenza come calore:
	\[E_{V_A} = \int_{0}^{\infty} V_A \cdot i(t) dt = C \cdot {V_A}^2 \quad E_R = \int_{0}^{\infty} R \cdot i^2(t) dt = \frac{C \cdot {V_A}^2}{2} \quad E_C = \int_{0}^{\infty} v_C(t) \cdot i(t) dt = \frac{C \cdot {V_A}^2}{2}\]
\end{itemize}

\begin{center}
	\begin{minipage}{0.4\textwidth}
		\centering \begin{circuitikz}[american, scale=1.2]
			\coordinate (A) at (0, 2);   % Nodo superiore sinistro (sopra VA)
			\coordinate (B) at (0, 0);     % Nodo inferiore sinistro (sotto VA)
			\coordinate (C) at (3, 2);   % Nodo superiore destro (dopo R)
			\coordinate (D) at (3, 0);     % Nodo inferiore destro (sotto C)
			\coordinate (E) at (1.2, 2); % Nodo tra interruttore e resistore
			\coordinate (F) at (1.2, 0);   % Nodo tra generatore e condensatore
	
			\draw (A) to [V, l_=\(V_A\)] (B);
			\draw (A) to [ospst] (E); 
			\draw (E) to [R, l_=\(R\), v^=\(v_R\), i_>=\(i_R\)] (C);
			\draw (C) -- (D); 
			\draw (D) to [C, l_=\(C\), v^=\(v_C\), i_>=\(i_C\)] (F);
			\draw (F) -- (B);
		\end{circuitikz}	
	\end{minipage}
	\begin{minipage}{0.55\textwidth}
		\centering \includegraphics[width=0.8\textwidth]{immagini/1_intro/rc_carica.png}
	\end{minipage}
\end{center}

\vspace{0.5cm}

\subsubsection*{Scarica di un condensatore}
\begin{itemize}
	\item condizioni iniziali (\(t < 0\)): interruttore inizialmente aperto \(\quad \rightarrow \quad v_C(0) = V_A, \quad i_C(0) = 0\)
	\item nel transitorio (\(t \geq 0\)): \(\displaystyle \;\; i_C(t) = C \frac{d v_C(t)}{d t}, \quad i_R = \frac{v_R(t)}{R}, \quad v_R(t) = v_C(t), \quad i_R(t) + i_C(t) = 0\)
	\item dalla legge delle correnti si ottiene un'equazione differenziale del primo ordine:
	\[C \frac{dv_C(t)}{d t} = \frac{v_R(t)}{R} \quad \rightarrow \quad \frac{dv_C(t)}{dt} = -\frac{v_C(t)}{RC} \quad \rightarrow \quad v_C(t) = A \cdot e^{-\tfrac{t}{RC}}\]
	\item siccome l'equazione è omogenera, è sufficiente imporre le condizioni iniziali:
	\[v_C(0) = V_A \quad \rightarrow \quad A = V_A \]
	\item si ottengono quindi le espressioni delle variabili elettriche durante il transitorio:
	\[v_C(t) = v_R(t) = V_A \cdot e^{-\tfrac{t}{RC}} \qquad i_C(t) = -\frac{V_A}{R} e^{-\tfrac{t}{RC}} \qquad i_R(t) = \frac{V_A}{R} e^{-\tfrac{t}{RC}}\]
	\item l'istante in cui la tensione sul condensatore raggiunge metà del suo valore di regime è:
	\[\frac{V_A}{2} = V_A - V_A \cdot e^{-\tfrac{t_{1/2}}{RC}} \quad \rightarrow \quad t_{1/2} = \ln(2) \cdot RC \approx 0.69 RC\]
	\item analizzando il bilancio energetico del circuito si ottiene tutta l'energia immagazzinata nel condensatore viene dissipata
	dalla resistenza come calore e il condensatore rimane scarico alla fine del transitorio:
	\[E_R = \int_{0}^{\infty} R \cdot {i_R}^2(t) dt = \frac{C \cdot {V_A}^2}{2} \qquad\qquad E_C = \int_{0}^{\infty} v_C(t) \cdot i_C(t) dt = -\frac{C \cdot {V_A}^2}{2}\]
\end{itemize}

\begin{center}
	\begin{minipage}{0.4\textwidth}
		\centering \begin{circuitikz}[american, scale=1.2]
	    \coordinate (A) at (0, 0);
		\coordinate (B) at (0, 2);
		\coordinate (C) at (2, 2);
		\coordinate (D) at (2, 0);
	
	    \draw (B) to [R, l=\(R\), v=\(v_R\), i>=\(i_R\)] (A);
	    \draw (B) to [ospst] (C); 
	    \draw (D) to [C, l=\(C\), v<=\(v_C\), i<=\(i_C\)] (C);
		\draw (D) -- (A);
	\end{circuitikz}
	\end{minipage}
	\begin{minipage}{0.55\textwidth}
		\centering \includegraphics[width=0.8\textwidth]{immagini/1_intro/rc_scarica.png}
	\end{minipage}
\end{center}
