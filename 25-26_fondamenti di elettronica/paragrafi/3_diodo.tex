\section{Giunzione PN e diodi}
\subsection{Giunzione pn all'equilibrio}
\subsubsection*{Struttura base}
Una giunzione pn si ottiene unendo due regioni di semiconduttore drogate in modo diverso: una regione di tipo p (con eccesso di
lacune) e una regione di tipo n (con eccesso di elettroni).

\subsubsection*{Equilibrio tra diffusione e potenziale}
\begin{itemize}
	\item Quando le due regioni si uniscono, si forma un \textbf{gradiente di concentrazione} dei portatori di carica che induce
	uno spostamento di elettroni dalla regione n alla regione p e uno spostamento di lacune dalla regione p alla regione n;
	si forma in questo modo una corrente di diffusione dalla regione p alla regione n.
	\item Lo spostamento dei portatori induce la formazione di ioni fissi costituiti dagli atomi dei droganti: i donatori perdono
	il loro elettrone spaiato e diventano ioni con carica positiva nella regione n, mentre gli accettori catturano l'elettrone che
	gli mancava e diventano ioni con carica negativa nella regione p. Questi ioni fissi generano un campo elettrico e un
	\textbf{potenziale di giunzione}; si forma in questo modo anche una corrente di deriva dalla regione n alla regione p che si
	oppone alla corrente di diffusione.
	\item All'equilibrio le due correnti si bilanciano, ma rimane una regione in prossimità della giunzione sono presenti solo ioni
	fissi dei droganti per l'assenza di portatori di carica.
	\item Si formano in questo modo tre regioni:
	\begin{enumerate}[topsep=-2pt, itemsep=0pt]
		\item \textbf{regione di svuotamento} o \textbf{regione di carica spaziale} (RCS): zona in prossimità della giunzione
		priva di portatori di carica liberi (svuotamento) in cui sono presenti solo ioni fissi (carica spaziale);
		\item \textbf{regione quasi neutra} (RQN) \textbf{di tipo p}: zona lontana dalla giunzione che non risente della giunzione
		pn e mantiene le caratteristiche di un semiconduttore di tipo p;
		\item \textbf{regione quasi neutra} (RQN) \textbf{di tipo n}: zona lontana dalla giunzione che non risente della giunzione
		pn e mantiene le caratteristiche di un semiconduttore di tipo n.
	\end{enumerate}
\end{itemize}

\vspace{0.5cm}
\begin{center}
	\includegraphics[width=0.85\textwidth]{immagini/3_giunzioni_pn_e_diodi/pn_eq1.png}
\end{center}
\vspace{0.5cm}

\noindent
NOTA: i seguenti calcoli si riferiscono alle grandezze per unità di superficie di giunzione \(\Sigma_j\). Per cui per ottenere
ad esmepio la densità di carica elettrica effetiva è necessario moltiplicare la densità di carica elettrica per unità di superficie
utilizzata nei calcoli per la superficie di giunzione: \(\rho_{e\!f\!f} = \rho \cdot \Sigma_j\).

\newpage

\subsubsection*{Carica elettrica all'equilibrio}
La carica elettrica nelle regioni quasi neutre è nulla, siccome non vengono alterate le concentrazioni di portatori di carica
liberi (e il drogaggio non modifica la carica complessiva). Nella regione di svuotamento, invece, la carica elettrica è data
dalla somma delle cariche degli ioni fissi e dipende dalle concentrazioni di drogaggio \(N_A\) e \(N_D\):
\[\rho(x) = \begin{cases}
	-q N_A & -x_p \leq x \leq 0 \quad \text{(regione p)}\\
	+q N_D & 0 < x \leq x_n \quad \text{(regione n)}\\
\end{cases} \qquad \rightarrow \qquad Q_p = -q N_A x_p, \quad Q_n = +q N_D x_n\]
Siccome non ci sono stati scambi di cariche con l'esterno, la carica totale deve rimanere nulla:
\[Q_p + Q_n = 0 \quad \Rightarrow \quad N_A x_p = N_D x_n\]

\subsubsection*{Campo elettrico all'equilibrio}
Il campo elettrico nella regione di carica spaziale si calcola:
\[\frac{dE(x)}{dx} = -\frac{\rho(x)}{\varepsilon} = \begin{cases}
	-q N_A / \varepsilon & -x_p \leq x \leq 0 \\
	+q N_D / \varepsilon & 0 < x \leq x_n
\end{cases} \quad \Rightarrow \quad E(x) = \begin{cases}
	-q N_A (x + x_p) / \varepsilon & -x_p \leq x \leq 0 \\
	+q N_D (x_n - x) / \varepsilon & 0 < x \leq x_n
\end{cases}\]
Il campo elettrico è nullo nelle regioni quasi neutre e raggiunge il valore massimo in \(x = 0\):
\[E_{max} = E(0) = -\frac{q N_A x_p}{\varepsilon} = -\frac{q N_D x_n }{\varepsilon}\]

\subsubsection*{Potenziale elettrico all'equilibrio e potenziale di contatto}
La differenza di potenziale si calcola integrando il campo elettrico, per la relazione \(dV(x)/dx = -E(x)\). In particolare
si definiscono i potenziali nelle due regioni quasi neutre \(V_1\) in \(x = -x_p\) e \(V_2\) in \(x = x_n\) e si calcola il
potenziale intrinseco di giunzione o potenziale di contatto \(V_0 = V_2 - V_1\) tra le due estremità della regione di svuotamento ponendo \(V_1 = 0\):
\[V_0 = \int_{-x_p}^{x_n} E(x) \, dx = \frac{-E(0) \cdot (x_n + x_p)}{2}\]

\subsubsection*{Potenziale di contatto e concentrazioni di drogaggio}
Il potenziale di contatto può essere espresso in funzione delle concentrazioni di drogaggio \(N_A\) e \(N_D\) e delle concentrazioni
intrinseche di portatori di carica \(n_i\):
\[V_0 = V_2 - V_1 = V_T \ln \left(\frac{n_2}{n_1}\right) = V_T \ln{\left(\frac{N_A N_D}{{n_i}^2}\right)}\]

\subsubsection*{Schema riassuntivi per una giunzione pn in equilibrio}
\begin{minipage}{0.45\textwidth}
	\centering
	\includegraphics[width=0.9\textwidth]{immagini/3_giunzioni_pn_e_diodi/pn_eq_2.1.png}
\end{minipage}
\begin{minipage}{0.45\textwidth}
	\centering
	\includegraphics[width=0.95\textwidth]{immagini/3_giunzioni_pn_e_diodi/pn_eq_2.2.png}
\end{minipage}

\subsubsection*{Ampiezza della regione di svuotamento all'equilibrio}
L'ampiezza della regione di svuotamento \(W = x_n + x_p\) dipende dalle concentrazioni dei drogaggi e dal potenziale di contatto:
\[V_0 = \frac{-E(0) \cdot W}{2}, \quad x_n N_D = x_p N_A \quad \rightarrow \quad W = \frac{2 V_0}{-E(0)} = \frac{2 \varepsilon V_0}{q N_A x_p} = \sqrt{\frac{2 \varepsilon V_0}{q} \left(\frac{1}{N_A} + \frac{1}{N_D}\right)}\]
\[x_n = W\frac{N_A}{N_A + N_D} \qquad x_p = W\frac{N_D}{N_A + N_D}\]
Si osserva che l'ampiezza della regione di svuotamento è inversamente proporzionale alle concentrazioni di drogaggio, per cui
aumentando i drogaggi diminuisce l'ampiezza della regione di svuotamento. Inoltre la regione di svuotamento si ripartisce in maniera
inversamente proporzionale ai drogaggi, ovvero si allarga maggiormente nella regione meno drogata.

\subsubsection*{Regione pn con elettrodi metallici}
Una giunzione pn, per essere utilizzabile in un circuito, deve essere collegata alle due estremità a due elettrodi metallici.
Gli elettrodi, essendo buoni conduttori, inducono una locale ridistribuzione dei portatori di carica (spostamento di lacune
dalla regione p all'elettrodo e di elettroni dalla regione n all'elettrodo). Questo effetto ha le stesse dinamiche di una
giunzione pn, in particolare si formano due regioni di carica spaziale che inducono una differenza di potenziale tra gli
elettrodi (con potenziali \(V_1\) e \(V_2\)) e la giunzione che controbilanciano il potenziale (o meglio tensione) di contatto
\(V_0\). All'equilibrio si ha \(V_1 + V_0 + V_2 = 0\), per cui la differenza di potenziale degli estremi di una giunzione è nulla.

\begin{center}
	\includegraphics[width=0.7\textwidth]{immagini/3_giunzioni_pn_e_diodi/pn_eq3.png}
\end{center}

\newpage

\subsection{Giunzione pn polarizzata}
\subsubsection*{Polarizzazione diretta e inversa}
Si collega una giunzione pn ad un generatore di tensione \(V_A\) con polo positivo connesso alla regione p e polo negativo alla
regione n. In questo modo si ha una polarizzazione della giunzione che può essere:
\begin{itemize}
	\item \textbf{polarizzazione diretta} se \(V_A > 0\)
	\item \textbf{polarizzazione inversa} se \(V_A < 0\)
\end{itemize}
Inoltre si definisce il potenziale di riferimento come il potenziale della regione n (\(V_n = 0V\)). In questo modo la regione
p ha un potenziale \(V_p = -V_0\), uguale ed opposto al potenziale intrinseco o di contatto.

\subsubsection*{Giunzione in polarizzazione diretta}
In polarizzazione diretta la tensione applicata \(V_A > 0\) riduce la differenza di potenziale tra le due regioni quasi neutre,
ottenendo: \(V_p - V_n = -V_0 + V_A\). Di conseguenza il campo elettrico nella regione di svuotamento diminuisce in modulo e
l'ampiezza della regione di svuotamento si riduce:
\[W(V_0 - V_A) = \sqrt{\frac{2 \varepsilon (V_0 - V_A)}{q} \left(\frac{1}{N_A} + \frac{1}{N_D}\right)} < W(V_0)\]
Siccome il campo elettrico e il potenziale si riducono, la corrente di deriva diminuisce e prevale il fenomeno di diffusione 
che induce un flusso di elettroni dalla regione n alla regione p e di lacune dalla regione p alla regione n. Siccome la regione
n è ricca di elettroni e la regione p è ricca di lacune, questo flusso è detto flusso dei portatori maggioritari. Si induce in
questo modo una \textbf{elevata corrente di diffusione dei maggioritari} che attraversa la giunzione pn dalla regione p alla regione n.

\subsubsection*{Giunzione in polarizzazione inversa}
In polarizzazione inversa la tensione applicata \(V_A < 0\) aumenta la differenza di potenziale tra le due regioni quasi neutre,
ottenendo: \(V_p - V_n = -V_0 - |V_A|\). Di conseguenza il campo elettrico nella regione di svuotamento aumenta in modulo e
l'ampiezza della regione di svuotamento si allarga:
\[W(V_0 + |V_A|) = \sqrt{\frac{2 \varepsilon (V_0 + |V_A|)}{q} \left(\frac{1}{N_A} + \frac{1}{N_D}\right)} > W(V_0)\]
Siccome il campo elettrico e il potenziale aumentano, la corrente di deriva prevale sulla corrente di diffusione. Si ha in questo
modo un flusso di elettroni dalla regione p alla regione n e di lacune dalla regione n alla regione p. Siccome la regione
p è povera di elettroni e la regione n è povera di lacune, questo flusso è detto flusso dei portatori minoritari. Si induce in
questo modo una \textbf{debole corrente di deriva dei minoritari} che attraversa la giunzione pn dalla regione n alla regione p.
\[J = (\mu_n n + \mu_p p) \, qE \qquad \text{con} \; n \approx p \approx 0\]

\begin{center}
	\includegraphics[width=0.7\textwidth]{immagini/3_giunzioni_pn_e_diodi/pn_pol2.png}
\end{center}

\subsection{Giunzione pn polarizzata vista come diodo}
\subsubsection*{Relazione tensione-corrente in un diodo}
Si assegnano dei riferimenti ai due terminali della giunzione pn in modo da renderla schematizzabile e utilizzabile come diodo
in un circuito elettrico:
\begin{itemize}
	\item \textbf{anodo}: terminale positivo, collegato alla regione p, dove si assorbono gli elettroni
	\item \textbf{catodo}: terminale negativo, collegato alla regione n, dove si immettono gli elettroni
\end{itemize}
\begin{center}
	\includegraphics[width=0.55\textwidth]{immagini/3_giunzioni_pn_e_diodi/diodo_1.png}
\end{center}
Definiti i riferimenti di tensione e corrente in un diodo (o giunzione pn polarizzata), si può definire la relazione tensione-corrente
che lega la tensione applicata \(V_A\) alla corrente \(I\) che attraversa il diodo:
\[i_D = I_S \left(e^{\tfrac{v_D}{\eta V_T}} - 1\right)\]
\begin{itemize}
	\item \(i_D\) e \(v_D\) sono la corrente e la tensione nel diodo, con riferimento positivo dall'anodo al catodo;
	\item \(I_S\) è la corrente di saturazione inversa, ovvero la debole corrente che attraversa il diodo quando è collegato
	in polarizzazione inversa (tipicamente dell'ordine di qualche nA);
	\item \(\eta\) è il coefficiente di idealità del diodo, che dipende dal materiale e dal processo di fabbricazione (tipicamente
	compreso tra 1 e 2);
	\item \(V_T = k_B T / q\) è il potenziale termico (del valore di circa 25mV a temperatura ambiente).
\end{itemize}
\begin{minipage}{0.65\textwidth}
Analizzando la curva caratteristica del diodo si osserva che in polarizzazione diretta (\(v_D > 0\)) la corrente cresce
esponenzialmente con la tensione applicata, mentre in polarizzazione inversa (\(v_D < 0\)) la corrente si stabilizza ad un valore
negativo pari a \(-I_S\).
\end{minipage}
\begin{minipage}{0.34\textwidth}
	\centering
	\includegraphics[width=0.8\textwidth]{immagini/3_giunzioni_pn_e_diodi/diodo_2.png}
\end{minipage}

\subsubsection*{Capacità della giunzione in polarizzazione inversa}
Analizzando la carica elettrica (per unità di superficie) presente nella regione di carica spaziale di una giunzione pn in
polarizzazione inversa si osserva che:
\[\begin{array}{l}
	Q_n = q N_D x_n = q N_D \frac{N_A}{N_A + N_D} W = \sqrt{2 q \varepsilon (V_0 - V_A) \frac{N_A N_D}{N_A + N_D}} \\
	Q_p = q N_A x_p = q N_A \frac{N_D}{N_A + N_D} W = \sqrt{2 q \varepsilon (V_0 - V_A) \frac{N_A N_D}{N_A + N_D}}
\end{array} \qquad Q_n = -Q_p\]
Si osserva quindi che la carica dipende dalla tensione applicata \(V_A\). La giunzione pn in polarizzazione inversa equivale
ad un condensatore con capacità per unità di area non lineare data da:
\[C_{j} = \frac{dQ}{dV_A} = \sqrt{\frac{q \varepsilon}{2(V_0 - V_A)} \frac{N_A N_D}{N_A + N_D}} = \frac{\varepsilon}{W}\]
NOTA: per ottenere la reale capacità del diodo bisogna moltiplicare la capacità per unità di area \(C_j\) per la sezione
della giunzione \(\Sigma_j\): \[C = C_j \cdot \Sigma_j = \frac{\varepsilon}{W} \; \Sigma_j\]

\newpage

\subsubsection*{Coefficiente di idealità}
\begin{minipage}{0.65\textwidth}
	Facendo variare il coefficiente di idealità \(\eta\) tra 1 e 2 si osserva che avviene una traslazione orizzontale della curva
	caratteristica del diodo. Minore è il valore di \(\eta\), più la curva sale rapidamente in polarizzazione diretta. In genere
	si utilizza \(\eta = 1\) per correnti basse e \(\eta = 2\) per correnti elevate.
	\vspace{1.2cm}
\end{minipage}
\begin{minipage}{0.34\textwidth}
	\centering
	\includegraphics[width=0.77\textwidth]{immagini/3_giunzioni_pn_e_diodi/diodo_3.png}
\end{minipage}

\subsubsection*{Modello semplificato del diodo}
Si osserva che la curva caratteristica del diodo può essere approssimata con un modello semplificato definito in funzione della
tensione applicata \(v_D\) e di conseguenza della polarizzazione del diodo:
\begin{itemize}
	\item per \(v_D < V_{ON} \rightarrow i_D = 0\) il diodo è in interdizione e si comporta come un circuito aperto in condizioni
	stazionarie oppure come condensatore non lineare in condizioni non stazionarie;
	\item per \(v_D = V_{ON} \rightarrow i_D > 0\) il diodo è in conduzione e si comporta come un generatore ideale di tensione
	con tensione \(V_{ON}\).
\end{itemize}
La tensione \(V_{ON}\) è detta tensione di soglia del diodo e divide le due regioni di funzionamento. Tipicamente per un diodo
al silicio si assume \(V_{ON} = 0.7V\), mentre per un diodo al germanio si assume \(V_{ON} = 0.3V\).

\begin{center}
	\includegraphics[width=0.7\textwidth]{immagini/3_giunzioni_pn_e_diodi/diodo_4.png}
\end{center}

\vspace{0.3cm}

\subsection{Applicazioni speciali dei diodi}
\subsubsection*{Applicazioni generali}
I diodi sono componenti fondamentali in molti circuiti elettronici e trovano applicazione in diversi ambiti:
\begin{itemize}
	\item \textbf{raddrizzatori}: i diodi vengono utilizzati nei circuiti raddrizzatori per convertire la corrente alternata (AC)
	in corrente continua (DC), permettendo il funzionamento di dispositivi elettronici alimentati a corrente continua;
	\item \textbf{protezione da inversioni di polarità}: i diodi proteggono i circuiti elettronici da danni causati da inversioni
	accidentali di polarità della tensione di alimentazione;
	\item \textbf{limitatori di tensione}: i diodi limitano la tensione in un circuito, proteggendo i componenti sensibili
	da sovratensioni;
	\item \textbf{LED (Light Emitting Diode)}: i diodi LED emettono luce quando attraversati da corrente elettrica, trovando
	applicazione in display, indicatori luminosi e illuminazione;
	\item \textbf{fotorilevatore}: i diodi a semiconduttore possono essere utilizzati come sensori di luce, convertendo l'energia
	luminosa in corrente elettrica.
\end{itemize}

\newpage

\subsubsection*{Fotodiodo}
Il fotodiodo è una giunzione pn collegata in polarizzazione inversa con l'area di svuotamento esposta alla luce. Quando la luce
colpisce la regione di svuotamento, eccita gli elettroni che si liberano dai legami covalenti, generando coppie elettrone-lacuna
(fotogenerazione). I due portatori vengono separati dal campo elettrico presente nella regione, generando una corrente detta
fotocorrente. La fotocorrente è proporzionale all'intensità della luce incidente. Il fotodiodo viene utilizzato in applicazioni
come sensori di luce e telecomunicazioni ottiche.
\[i_\text{D} = I_S \left(e^{\tfrac{v_D}{\eta V_T}} - 1\right) - I_{PH} \qquad\qquad I_{PH} = R \cdot P_0 \qquad\qquad \begin{array}{l}
	I_{PH}: \text{ fotocorrente (A)} \\
	P_0: \text{ potenza ottica incidente (W)} \\
	R: \text{ responsività (A/W)}
\end{array}\]
La curva caratteristica risulta spostata verso il basso di un valore pari alla fotocorrente \(I_{PH}\). È possibile schematizzare
un fotodiodo come un diodo ideale in parallelo ad una sorgente di corrente pari a \(I_{PH}\).

\begin{center}
	\includegraphics[width=0.7\textwidth]{immagini/3_giunzioni_pn_e_diodi/diodo_5.png}
\end{center}

\subsubsection*{LED (Light Emitting Diode) o diodi a emissione luminosa}
I LED sono giunzioni pn in polarizzazione diretta con l'area di svuotamento \say{scoperta}. Quando una corrente attraversa il LED,
le lacune dalla regione p si ricombinano con gli elettroni dalla regione n nella regione di svuotamento, rilasciando energia sotto
forma di fotoni (emissione di luce).

La lunghezza d'onda della luce emessa dipende dalla differenza di energia tra la banda di conduzione e la banda di valenza
(energy gap) propria di ogni semiconduttore. Si utilizzano, infatti, semiconduttori diversi per ottenere colori diversi.

Un parametro importante è la tensione di accensione \(V_{ON}\) dei LED, ovvero la tensione a cui il LED inizia a emettere luce.
In genere è superiore alla tensione di soglia in quanto non basta fornire energia per permettere il passaggio della corrente, ma
è necessario fornire energia sufficiente per permettere l'emissione dei fotoni.
\[V_{ON} > \frac{hc}{\lambda q} \qquad \begin{array}{l}
	h: \text{ costante di Planck} (6.626 \times 10^{-34} \J\s) \\
	c: \text{ velocità della luce} (3.0 \times 10^{8} \m/\s) \\
	\lambda: \text{ lunghezza d'onda della luce emessa} \\
	q: \text{ carica dell'elettrone } (1.6 \times 10^{-19} \C)
\end{array} \quad \begin{array}{l l}
	\lambda = 620 \;\nm\; \text{(rosso)} & V_{ON} \approx 2.0 - 2.2 \V \\
	\lambda = 520 \;\nm\; \text{(giallo)} & V_{ON} \approx 2.1 - 2.2 \V \\
	\lambda = 510 \;\nm\; \text{(verde)} & V_{ON} \approx 2.5 - 3.3 \V \\
	\lambda = 470 \;\nm\; \text{(blu)} & V_{ON} \approx 3.2 - 3.3 \V
\end{array}\]
Non esiste nessun semiconduttore che emetta luce bianca: per creare un \say{LED bianco} si utilizza un LED blu con un rivestimento
di fosforo (giallo) che converte parte della luce blu in luce gialla. Dalla combinazione delle due luci si ottiene la luce bianca
percepita dall'occhio umano. In base alla quantità di fosforo utilizzata si possono ottenere diverse tonalità di bianco (caldo,
neutro, freddo).

\begin{center}
	\includegraphics[width=0.27\textwidth]{immagini/3_giunzioni_pn_e_diodi/diodo_6.png}
\end{center}
