\section{Clock e sincronizzazione}

\subsection{Distribuzione del clock}
\subsubsection*{Struttura della rete di distribuzione}
Il circuito di distribuzione del clock è costituito da un circuito di generazione del segnale di clock, una rete di
distribuzione globale ai vari moduli del circuito e una rete di distribuzione locale all'interno dei singoli moduli.

Durante la distribuzione possono provocarsi ritardi e sfasamenti, l'importante è che la fase relativa del segnale di clock
che arriva a tutti i moduli sia perfettamente sincronizzata, non importa il ritardo assoluto dall'origine.

\subsubsection*{Ritardo delle piste in interconnessione}
Per calcolare il ritardo delle piste di interconnessione si modella la pista come una serie di celle con una certa resistenza
(data dalla legge di Ohm) e una certa capacità parassita (data dalle capacità di area e dalle capacità di bordo). Calcolando
il tempo di ritardo della rete di Elmore così costruita si osserva che il modello RC equivale ad una lunga pista con tutta
la capacità condensata al centro della pista stessa.
\[R_{W,\text{pista}} = \rho \; \frac{L_\text{lunghezza}}{h_\text{altezza} \cdot W_\text{spessore}} = R_q \; \frac{L}{W} \qquad \text{con} \; R_q = \text{resistenza per quadro} [\Omega / q]\] 
\[C_{W,\text{pista}} = C_{W\!A} \cdot W \cdot L + C_{W\!B} \cdot 2L \qquad \text{con} \; \begin{array}{l}
	C_{W\!A} = \text{capacità di area} = [f\!F/\mu m^2] \\
	C_{W\!B} = \text{capacità di bordo} = [f\!F/\mu m]
\end{array}\]
\begin{align*}
	t_\text{pista} &= 0.69 \cdot R_{out} \cdot C_{out} \qquad\qquad\qquad\quad\; \text{circuito che pilotala pista} \\
	&\quad + 0.69 \cdot (R_{out} + R_w/2) \cdot C_{W,\text{pista}} \qquad \text{prima metà della pista con capacità al centro} \\
	&\quad + 0.69 \cdot (R_{out} + R_w) \cdot C_L \qquad\qquad\quad\, \text{seconda metà pista e carico pilotato}
\end{align*}

\subsubsection*{Errori nella propagazione del clock}
Le sorgenti di errori del clock possono essere:
\begin{itemize}
	\item \textbf{errori sistematici}: errori prevedibili e corretti in fase di progettazione, come carichi capacitivi diversi
	per ogni percorso di clock o lunghezze diverse delle piste di clock
	\item \textbf{errori casuali}: randomici e non prevedibili come sbavature dei processi di produzione
\end{itemize}
Gli errori si manifestano come:
\begin{itemize}
	\item \textbf{skew}: sfasamento costante nel tempo (e talvolta calcolabili e compensabili) tra due segnali di clock generati
	dalla stessa sorgente, dovuto a ritardi diversi nei percorsi di distribuzione
	\item \textbf{jitter}: variazione completamente casuali e che variano da ciclo a ciclo di clock, con media nulla e non
	prevedibili, per questo si definisce il valore di jitter massimo (in modulo)
\end{itemize}

\subsubsection*{Effetti del ritardo di propagazione del Clock}
Analizzando due segnali sfasati di uno skew \(t_\text{skew}\), lo sfasamento reale può variare nel range di
\(t_\text{skew} \pm 2t_\text{jitter}\) ad esempio se il segnale in anticipo viene anticipato di un jitter e il
segnale in ritardo viene ritardato di un jitter o viceversa. Il massimo e il minimo sfasamento tra due segnali
si indicano con:
\[\delta_{min} = t_\text{skew} - 2 t_\text{jitter} \qquad\qquad \delta_{max} = t_\text{skew} + 2 t_\text{jitter}\]
Tale sfasamento può violare i vincoli di temporizzazione dei circuiti sequenziali e la sincronizzazione tra moduli diversi:
in particolare uno sfasamento positivo nel clock potrebbe violare le condizioni su \(t_\text{hold}\), mentre uno sfasamento
negativo potrebbe non rispettare il \(t_\text{setup}\).
\begin{align*}
	t_\text{prop,reale} &= t_\text{prop,ideale} - t_\text{skew} + 2 t_\text{jitter} & & t_\text{skew} > 0 \;\; \text{se ritarda progressivamente} \\
	t_\text{cont,reale} &= t_\text{cont,ideale} - t_\text{skew} - 2 t_\text{jitter} & & t_\text{skew} < 0 \;\; \text{se anticipa progressivamente}
\end{align*}

\subsection{Generazione del clock e Phase Locked Loop - PLL}
\subsubsection*{Struttura del PLL}
Un PLL (o Anello ad Aggancio di Fase) è un circuito con retroazione che genera un segnale di clock ad una data frequenza
con elevata precisione. È composto da PFD, CP, VCO e due divisori di frequenza.

\subsubsection*{Voltage Controlled Oscillator (VCO)}
Il VCO è un oscillatore che genera un'onda quadra la cui frequenza è regolata da una tensione analogica in ingresso \(V_C\).
È costituito da un anello di N ``invertitori'' (raddrizzati) controllati in corrente in modo da variare il tempo \(t_p\) di
propagazione e quindi la frequenza di oscillazione complessiva: \(f_\text{out} = 1 \, / \, 2 N t_p\).

Un invertitore controllato in corrente è costituito da un invertitore CMOS collegato a \(V_{DD}\) e \(GND\) attraverso un PMOS
e un NMOS rispettivamente che funzionano da generatori ideali di corrente. I due MOS extra sono collegati gate-gate ad altri
due mos identici che formano un circuito di corrente speculare in cui il gate degli nmos è pilotato dalla tensione di controllo
\(V_C\) e il pmos ha drain e gate ponticellati per lavorare in saturazione a diodo. Si hanno le seguenti relazioni:
\[I = {k_n}' Z_n \frac{(V_C - V_{TN})^2}{2} \quad \text{oppure} \quad I = {k_n}' Z_n V_{DSATN} \left(V_C - V_{TN} - \frac{V_{DSATN}}{2}\right)\]
\[\Delta T_\text{ritardo invertitore} = \frac{C \cdot V_{DD}/2}{I} \qquad T_\text{periodo anello} = 2N \Delta T_\text{ritardo invertitore}\] 
Si osserva quindi che il tempo di ritardo \(t_p\) del singolo invertitore dipende dalla corrente \(I\) e di consegienza da
\(V_C\): aumentando \(V_C\) aumenta \(I\) e diminuisce \(t_p\), aumentando la frequenza di oscillazione.

Siccome gli invertitori controllati in corrente non godono di proprietà rigenerative, inoltre per ridurre il consumo statico
di correnti ``analogiche'' che pilotano gate, si inserisce un invertitore CMOS classico dopo ogni invertitore controllato in
corrente. Si aggiunge, infine, un altro invertitore CMOS classico da solo per rendere l'anello invertente.

\subsubsection*{Phase Frequency Detector (PFD)}
Il PFD è un circuito sequenziale asincrono che riceve in input due segnali di clock \(REF\) e \(IN\) e produce in uscita
due segnali di controllo \(U\!P\) e \(DOW\!N\) che indicano se il segnale \(IN\) è in anticipo o in ritardo rispetto a \(REF\).
\begin{itemize}
	\item se \(IN\) è in anticipo rispetto a \(REF\) in uscisa si ha \(UP = 0\) e \(DOWN = 1\)
	\item se \(IN\) è in ritardo rispetto a \(REF\) in uscisa si ha \(UP = 1\) e \(DOWN = 0\)
	\item se i due segnali sono sincronizzati in uscita si ha \(UP = 0\) e \(DOWN = 0\)
\end{itemize}
Il circuito è implementato da due flip flop con ingressi fissi a 1, controllati dai due segnali di clock in ingresso, inoltre
possiedono un segnale di reset asincrono che azzera entrambi i flip flop quando entrambi sono settati a 1. In questo modo
quando arriva il fronte di uno dei due segnali di clock, il flip flop corrispondente si setta a 1 e l'altro rimane a 0. Quando
arriva anche l'altro fronte di clock, il secondo flip flop si setta a 1 e il segnale di reset azzera entrambi i flip flop.

\subsubsection*{Charge Pump (CP)}
La Charge Pump è un circuito che riceve in ingresso i segnali \(U\!P\) e \(DOW\!N\) dal PFD e produce in uscita una tensione
di controllo \(V_C\) per il VCO. Il circuito è costituito da un invertitore CMOS che pilota una capacità \(C\). Quando
arriva un impulso sul segnale \(U\!P\), l'invertitore porta il nodo di uscita a \(V_{DD}\) e la capacità si carica, mentre
quando arriva un impulso sul segnale \(DOW\!N\), l'invertitore porta il nodo di uscita a \(GND\) e la capacità si scarica.

Per rendere il sistema stabile e limitare la velocità di caricamento/scaricamento della capacità, si utilizzano due mos
in regime di generatori ideali di corrente (come nel VCO) per limitare la corrente di carica e scarica della capacità.
Le tensioni che pilotano tali mos sono fissate in fase di progettazione.

\subsubsection*{Divisori di frequenza}
Nel PLL sono presenti anche due divisori di frequenza, uno in ingresso al PFD che divide la frequenza del segnale \(f_{REF}\)
di un fattore \(N\) e uno che collega il VCO al PFD (retroazione) che divide la frequenza del segnale \(f_{IN}\) di un fattore
\(M\). La PLL quindi lavora per mantenere la condizione:
\[\frac{f_{IN}}{M} = \frac{f_{REF}}{N} \qquad \Rightarrow \qquad f_{IN} = \frac{M}{N} \; f_{REF}\]
Si osserva che \(M\) e \(N\) possono essere solo multipli di 2, ma questo non ci interessa perché alla fine il rapporto
semplifica tale fattore 2 e si può ottenere qualsiasi rapporto di frequenze razionale.

\subsection{Struttura della rete di distribuzione del clock}
\subsubsection*{Struttura globale ad albero a H}
Per minimizzare lo skew tra i vari moduli dei circuito si utilizza la distribuzione del clock ad albero ad H, ovvero
una struttura ad albero in cui i rami originati da ogni biforcazione hanno la stessa lunghezza e di conseguenza lo
stesso tempo di ritardo RC. In prossimità dei moduli si aggiungono dei buffer (``dinamici'' in funzione del carico) per
ripristinare il segnale di clock e adattarlo al carico.

\subsubsection*{Struttura locale a griglia}
Per distribuire il clock all'interno dei moduli si utilizza una rete a griglia che forma una scacchiera di piste orizzontali
e verticali polarizzate contemporaneamente al segnale di clock, a cui sono collegati i vari elementi sincroni del modulo. In
questo modo si cerca di contenere lo skew locale evitando di avere numerose piste di clock con lunghezze diverse.

\subsubsection*{Origine di Skew e Jitter}
Le principali origini di skew e jitter nella distribuzione del clock sono:
\begin{itemize}
	\item variazioni del processo di fabbricazione: \\
	ad esempio imprecisioni sulle dimensioni \(W\) ed \(L\), sugli spessori dell'ossido e sui livelli di drogaggio che
	influenzano \(k_n\) e \(V_{TN}\) nei circuiti PLL e nei buffer di distribuzione del clock; tali erorri casuali sono
	costanti nel tempo (localmente ad un singolo chip) e di conseguenza si definiscono skew
	\item variazioni delle interconnessioni: \\
	ad esempio imprecisioni nello spessore del dielettrico, nella geometria delle piste e planarizzazione del chip che
	generano skew in quanto costanti nel tempo (localmente ad un singolo chip)
	\item variazioni della temperatura: \\
	un uso non omogeneo del chip può generare gradienti di temperatura che provocano variazioni nelle tensioni di soglia e
	nella mobilità dei portatori, tali variazioni cambiano nel tempo, ma per tempi infinitesimi sono costanti e generano skew
	\item variazioni della tensione di alimentazione: \\
	le cadute di tensione sulle linee di alimentazione possono variare nel tempo in funzione del carico dei moduli vicini
	generando skew (variazioni lente), oppure possono essere causate da rumore di commutazioni logiche che generano jitter,
	per ridurre tali jitter si aggiungono capacità di decoupling (disaccoppiamento) tra le linee di alimentazione in prossimità
	dei buffer del clock
\end{itemize}

\subsubsection*{Riduzione dello sfasamento nella distribuzione locale e DLL}
Per bilanciare il carico locale dei singoli moduli e ridurre lo skew dovuto ai carichi diersi sui buffer, si utilizzano 
circuiti chiamati Delay Locked Loop (DLL). Sono simili ai PLL, ma non possiedono divisori di frequenza in quanto si vuole
avere in uscita lo stesso segnale di clock in ingresso. Inoltre il VCO è sostituito da una catena di ritardo VCDL (Voltage
Controlled Delay Line) che riceve in ingresso (oltre alla tensione \(V_C\)) la frequenza originale e produce in uscita
lo stesso segnale ritardato di un tempo regolabile in base al carico del modulo. In questo modo si è in grado di compensare
le variazioni di carico locale e ridurre lo skew.
