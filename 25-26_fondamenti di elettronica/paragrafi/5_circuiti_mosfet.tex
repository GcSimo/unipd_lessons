\section{Circuiti con i MOSFET}
\subsection{MOSFET in serie a una resistenza}
\subsubsection*{Circuito NMOSFET e resistenza}
\begin{center}
	\begin{minipage}{0.3\textwidth}
		\centering \begin{circuitikz}[american]
			\ctikzset{tripoles/mos style=arrows}
		
			% =================================================================
			% 1. Nodi principali del circuito
			\coordinate (GND_left) at (0, 0);     % Massa a sinistra
			\coordinate (G_mosfet) at (2, 0);  % gate del MOSFET
		
			% =================================================================
			% 2. Circuito
			\draw (GND_left) node[sground]{}; % Simbolo di massa a sinistra
			\draw (G_mosfet) to [V, l_=\(V_G\)] (GND_left); % Generatore di tensione V_G
			\draw (G_mosfet) node[nmos, anchor=G](nmos){}; % NMOSFET
			\draw (nmos.S) node[sground]{}; % source del MOSFET
			\draw (nmos.D) -- ++(0, 0.2) to [R, l=\(R\), name=R1] ++(0, 1.5) node[rground, rotate=180, name=VDD]{}; % resistenza R e VDD
		
			% =================================================================
			% 3. Etichette e Vettori Corrente/Tensione
		
			% V_GS (Gate-Source Voltage)
			\node at ($(nmos.G)+(0, -0.2)$) {\(+\)};
			\node at ($(nmos.S)+(-0.25, 0.1)$) {\(-\)};
			\node at ($(nmos.G)+(0.2, -0.6)$) {\(V_{GS}\)}; % Testo V_GS sopra i segni
		
			% V_DS (Drain-Source Voltage)
			\node at ($(nmos.D)+(0.3, -0.2)$) {\(+\)};
			\node at ($(nmos.S)+(0.3, 0.2)$) {\(-\)};
			\node at ($(nmos.G)+(1.4, 0)$) {\(V_{DS}\)}; % Testo V_DS sopra i segni
		
			% V_R (Tensione sulla resistenza R)
			\node at ($(R1)+(0.4, 0.5)$) {\(+\)};
			\node at ($(R1)+(0.4, -0.5)$) {\(-\)};
			\node at ($(R1)+(0.5, 0)$) {\(V_R\)}; % Testo V_R sopra i segni
		
			% I_DS Corrente Drain-Source
			\node at ($(nmos.D)+(-0.5, -0.1)$) {\(I_{DS} \downarrow\)}; % Testo I_DS vicino al drain

			% VDD
			\node at ($(VDD)+(0, 0.6)$) {\(V_{DD}\)};
		\end{circuitikz}
	\end{minipage}
	\begin{minipage}{0.57\textwidth}
		Si costruisce un circuito con un NMOSFET e una resistenza come illustrato di lato, in cui:
		\begin{itemize}
			\item il terminale di drain è collegato a una tensione di alimentazione \(V_{DD}\) tramite una resistenza di carico \(R\);
			\item il terminale di source è collegato a massa;
			\item il terminale di gate è collegato a un generatore di tensione \(V_G\).
		\end{itemize}
		L'utilizzo di un NMOSFET al posto di un PMOSFET è perfettamente arbitrario e non cambia nulla nel procedimento di
		soluzione del circuito.
	\end{minipage}
\end{center}

\subsubsection*{Soluzione del circuito per via grafica}
La risoluzione del circuito consiste nel trovare la corrente \(I_{DS}\) e la tensione \(V_{DS}\) che soddisfano sia la
caratteristica del MOSFET sia la caratteristica della resistenza \(R\). Dalle LKC si impone \(I_{DS} = I_R\) e si risolve
l'equazione in funzione di \(V_{DS}\).

Un'alternativa alla risoluzione analitica è la risoluzione grafica, che consiste nel tracciare sullo stesso grafico la
caratteristica di uscita del MOSFET (\(I_{DS}\) in funzione di \(V_{DS}\)) e la caratteristica della resistenza \(R\)
sempre in funzione di \(V_{DS}\) e trovare il punto di intersezione tra le due curve.

\begin{center}
	\begin{minipage}{0.45\textwidth}
		\centering \includegraphics[width=0.9\textwidth]{immagini/5_circuiti_mosfet/circuiti_mos_1.png}
	\end{minipage}
	\begin{minipage}{0.45\textwidth}
		Un'alternativa alla risoluzione analitica è la risoluzione grafica, che consiste nel tracciare sullo stesso grafico la
		caratteristica di uscita del MOSFET (\(I_{DS}\) in funzione di \(V_{DS}\)) e la caratteristica della resistenza \(R\)
		sempre in funzione di \(V_{DS}\) e trovare il punto di intersezione tra le due curve.
	\end{minipage}
\end{center}



\subsubsection*{Modello a canale lungo}
Il modello a canale lungo del MOSFET prevede che il transistor operi in una delle tre regioni (interdizione, lineare o saturazione
per pinchoff) senza considerare gli effetti di modulazione di lunghezza di canale e saturazione di velocità. Per risolvere il
circuito si procede come segue:
\begin{enumerate}
	\item si determina se il MOSFET è \say{acceso} o \say{spento} confrontando \(V_{GS}\) con la tensione di soglia \(V_{TN}\):
	\begin{itemize}[topsep=0pt]
		\item se \(V_{GS} < V_{TN}\), il MOSFET è in interdizione e \(I_{DS} = 0\)
		\item se \(V_{GS} \geq V_{TN}\), il MOSFET è acceso e si procede al passo successivo
	\end{itemize}
	\item si ipotizza che il MOSFET sia in uno dei due regimi di conduzione (lineare o saturazione) e si risolve la rete ponendo
	\(I_{DS} = I_R\) (per le LKC), utilizzando la formula \(I_{DS}\) corrispondente all'ipotesi fatta e si risolve per \(V_{DS}\):
	\begin{itemize}[topsep=0pt]
		\item se si ipotizza il regime lineare: \(\displaystyle k_n V_{DS} \left(V_{GS} - V_{TN} - \frac{V_{DS}}{2}\right) = \frac{V_{DD} - V_{DS}}{R}\)
		\item se si ipotizza il regime di saturazione: \(\displaystyle \frac{k_n}{2} (V_{GS} - V_{TN})^2 = \frac{V_{DD} - V_{DS}}{R}\)
	\end{itemize}
	\item si verifica se l'ipotesi fatta sul regime di funzionamento è corretta:
	\begin{itemize}[topsep=0pt]
		\item se si è ipotizzato il regime lineare, si verifica che \(V_{DS} < V_{GS} - V_{TN}\);
		\item se si è ipotizzato il regime di saturazione, si verifica che \(V_{DS} \geq V_{GS} - V_{TN}\).
	\end{itemize}
	\item se l'ipotesi è corretta, si è trovata la soluzione; altrimenti si ripete il passo 2 con l'altra ipotesi.
\end{enumerate}

\subsubsection*{Modello a canale corto}
La soluzione del circuito con il modello a canale corto è analoga a quella con il modello a canale lungo, con la differenza che
si prevede anche l'effetto di modulazione di lunghezza di canale e la possibilità che il MOSFET lavori in saturazione di velocità.
In particolare si procede come segue:
\begin{enumerate}
	\item si determina se il MOSFET è \say{acceso} o \say{spento} come nel modello a canale lungo;
	\item si calcolano, se possibile, i tre valori di \(V_{MIN}\) (\(V_{DS}, V_{GS} - V_{TN}, V_{DSATN}\)) e se si conoscono già
	due dei valori (esempio \(V_{GS}-V_{TN}\) e \(V_{DSATN}\)), si può escludere a priori quello maggiore (ovvero quello che
	sicuramente non sarà il minimo), riducendo così il numero di ipotesi da fare;
	\item si ipotizza che il MOSFET sia in uno dei tre regimi di conduzione (lineare, saturazione per pinchoff o saturazione di
	velocità) escludendone, se possibile, uno come spiegato al passo precedente, e si risolve la rete come nel modello a canle lungo:
	\begin{itemize}[topsep=0pt]
		\item in regime lineare: \(\displaystyle k_n V_{DS} \left(V_{GS} - V_{TN} - \frac{V_{DS}}{2}\right) (1 + \lambda V_{DS}) = \frac{V_{DD} - V_{DS}}{R}\)
		\item in saturazione per pinchoff: \(\displaystyle \frac{k_n}{2} (V_{GS} - V_{TN})^2 (1 + \lambda V_{DS}) = \frac{V_{DD} - V_{DS}}{R}\)
		\item in saturazione di velocità: \(\displaystyle k_n V_{DSATN} \left(V_{GS} - V_{TN} - \frac{V_{DSATN}}{2}\right) (1 + \lambda V_{DS}) = \frac{V_{DD} - V_{DS}}{R}\)
	\end{itemize}
	\item si verifica se l'ipotesi fatta sul regime di funzionamento è corretta:
	\begin{itemize}[topsep=0pt]
		\item se si è ipotizzato il regime lineare: \(V_{DS} = \min \{V_{GS} - V_{TN}, V_{DSATN}, V_{DS}\}\);
		\item se si è ipotizzato il regime di sat. per pinchoff: \(V_{GS} - V_{TN} = \min \{V_{GS} - V_{TN}, V_{DSATN}, V_{DS}\}\);
		\item se si è ipotizzato il regime di sat. di velocità: \(V_{DSATN} = \min \{V_{GS} - V_{TN}, V_{DSATN}, V_{DS}\}\);
	\end{itemize}
	\item se l'ipotesi è corretta, si è trovata la soluzione; altrimenti si ripete il passo 3 con un'altra ipotesi.
\end{enumerate}

\subsection{MOSFET connesso a diodo}
\subsubsection*{Circuito di un NMOSFET connesso a diodo}
\begin{center}
	\begin{minipage}{0.3\textwidth}
		\centering \begin{circuitikz}[american]
			\ctikzset{tripoles/mos style=arrows}
		
			% =================================================================
			% 1. Nodi principali del circuito
			\coordinate (G_mosfet) at (0, 0);  % gate del MOSFET
		
			% =================================================================
			% 2. Circuito
			
			\draw (G_mosfet) node[nmos, anchor=G](nmos){}; % NMOSFET
			\draw (G_mosfet) -- ++(-0.2, 0) -- ++(0,1.1) -- ++(1.18,0);
			\draw (nmos.S) node[sground]{}; % source del MOSFET
			\draw (nmos.D) -- ++(0, 0.4) to [R, l=\(R\), name=R1] ++(0, 1.3) node[rground, rotate=180, name=VDD]{}; % resistenza R e VDD
		
			% =================================================================
			% 3. Etichette e Vettori Corrente/Tensione
		
			% V_GS (Gate-Source Voltage)
			\node at ($(nmos.G)+(0, -0.2)$) {\(+\)};
			\node at ($(nmos.S)+(-0.25, 0.1)$) {\(-\)};
			\node at ($(nmos.G)+(0.2, -0.6)$) {\(V_{GS}\)}; % Testo V_GS sopra i segni
		
			% V_DS (Drain-Source Voltage)
			\node at ($(nmos.D)+(0.3, -0.2)$) {\(+\)};
			\node at ($(nmos.S)+(0.3, 0.2)$) {\(-\)};
			\node at ($(nmos.G)+(1.4, 0)$) {\(V_{DS}\)}; % Testo V_DS sopra i segni
		
			% V_R (Tensione sulla resistenza R)
			\node at ($(R1)+(0.4, 0.5)$) {\(+\)};
			\node at ($(R1)+(0.4, -0.5)$) {\(-\)};
			\node at ($(R1)+(0.5, 0)$) {\(V_R\)}; % Testo V_R sopra i segni
		
			% I_DS Corrente Drain-Source
			\node at ($(nmos.D)+(-0.5, -0.1)$) {\(I_{DS} \downarrow\)}; % Testo I_DS vicino al drain

			% VDD
			\node at ($(VDD)+(0, 0.6)$) {\(V_{DD}\)};
		\end{circuitikz}
	\end{minipage}
	\begin{minipage}{0.57\textwidth}
		Si costruisce un circuito con un NMOSFET e una resistenza come illustrato di lato, in cui:
		\begin{itemize}
			\item il terminale di source è collegato a massa;
			\item il terminale di drain è collegato a una tensione di alimentazione \(V_{DD}\) tramite una resistenza di carico \(R\);
			\item il terminale di gate è cortocircuitato al terminale di drain e \(V_{GS} = V_{DS}\)
		\end{itemize}
		L'utilizzo di un NMOSFET al posto di un PMOSFET è perfettamente arbitrario e non cambia nulla nel procedimento di
		soluzione del circuito.
	\end{minipage}
\end{center}

\subsubsection*{Considerazioni sul regime di funzionamento e risoluzione grafica}
Si osserva che il MOSFET connesso in questo modo, se acceso, lavora in saturazione per pinchoff o in saturazione di velocità,
per cui si comporta in modo simile a un diodo ideale con soglia \(V_{TN}\):

\begin{center}
	\begin{minipage}{0.45\textwidth}
		\centering \includegraphics[width=0.9\textwidth]{immagini/5_circuiti_mosfet/circuiti_mos_2.png}
	\end{minipage}
	\begin{minipage}{0.54\textwidth}
		\begin{itemize}
			\item se \(V_{GS} < V_{TN}\), il MOSFET è spento con \(I_{DS} = 0\);
			\item se \(V_{GS} \geq V_{TN}\), il MOSFET è acceso e lavora in saturazione per pinchoff \(V_{DS} = V_{GS} > V_{GS} - V_{TN}\)
			oppure in sat. di velocità se \(V_{DS} = V_{GS} > V_{DSATN}\)
		\end{itemize}
		Si nota che la caratteristica di uscita del MOSFET in questo caso è simile a quella di un diodo con soglia \(V_{TN}\).
		\vspace{0pt}

		Come nel caso precedente, la risoluzione grafica consiste nel trovare le intersezioni delle due caratteristiche di uscita
		del MOSFET e della resistenza \(R\).
	\end{minipage}
\end{center}

\subsubsection*{Analisi del circuito}
Per risolvere il circuito si procede come segue:
\begin{enumerate}
	\item si determina se il MOSFET è \say{acceso} o \say{spento} confrontando \(V_{GS}\) con la tensione di soglia \(V_{TN}\):
	\begin{itemize}[topsep=0pt]
		\item se \(V_{GS} < V_{TN}\), il MOSFET è in interdizione e \(I_{DS} = 0\)
		\item se \(V_{GS} \geq V_{TN}\), il MOSFET è acceso e si procede al passo successivo
	\end{itemize}
	\item se si utilizza il modello a canale lungo, si utilizza \(I_{DS}\) in saturazione per pinchoff e si impone la condizione
	\(\displaystyle I_{DS} = I_R \;\;\rightarrow\;\; I_{DS} = \frac{k_n}{2} (V_{GS} - V_{TN})^2 = \frac{V_{DD} - V_{DS}}{R}\);
	non serve verificare il regime di funzionamento, in quanto il MOSFET connesso in questo modo lavora sempre in saturazione
	\item se si utilizza il modello a canale corto, bisogna ipotizzare il regime di saturazione (pinchoff o velocità) e risolvere
	il circuito utilizzando la formula corrispondente:
	\begin{itemize}[topsep=0pt]
		\item in saturazione per pinchoff: \(\displaystyle \frac{k_n}{2} (V_{GS} - V_{TN})^2 (1 + \lambda V_{DS}) = \frac{V_{DD} - V_{DS}}{R}\)
		\item in saturazione di velocità: \(\displaystyle k_n V_{DSATN} \left(V_{GS} - V_{TN} - \frac{V_{DSATN}}{2}\right) (1 + \lambda V_{DS}) = \frac{V_{DD} - V_{DS}}{R}\)
	\end{itemize}
	\item si verifica se l'ipotesi fatta sul regime di funzionamento è corretta:
	\begin{itemize}[topsep=0pt]
		\item se si è ipotizzato il regime di sat. per pinchoff: \(V_{GS} - V_{TN} = \min \{V_{GS} - V_{TN}, V_{DSATN}, V_{DS}\}\);
		\item se si è ipotizzato il regime di sat. di velocità: \(V_{DSATN} = \min \{V_{GS} - V_{TN}, V_{DSATN}, V_{DS}\}\);
	\end{itemize}
	\item se l'ipotesi è corretta, si è trovata la soluzione; altrimenti si ripete il passo 3 con un'altra ipotesi
\end{enumerate}

\subsection{MOSFET come generatore di corrente}
\subsubsection*{Circuito di un NMOSFET connesso come generatore di corrente}
Si costruisce un circuito con un NMOSFET e una resistenza come illustrato sotto, in cui:
\begin{itemize}
	\item il terminale di source è collegato a massa;
	\item il terminale di drain è collegato a una tensione di alimentazione \(V_{DD}\) tramite una resistenza di carico \(R\);
	\item il terminale di gate è cortocircuitato al terminale di source e \(V_{GS} = V_{REF}\)
\end{itemize}
Si suppone, inoltre, che il valore di \(V_{REF}\) sia tale da mantenere il MOSFET sempre acceso e in regime di saturazione per
pinchoff o in saturazione di velocità, ovvero che sia soddisfatta la condizione:
\[V_{DS} > \min \{V_{REF} - V_{TN}, V_{DSATN}\}\]

\begin{center}
	\begin{minipage}{0.4\textwidth}
		\centering \begin{circuitikz}[american]
			\ctikzset{tripoles/mos style=arrows}
			
			% =================================================================
			% 1. Nodi principali del circuito
			\coordinate (G_mosfet) at (0, 0);  % gate del MOSFET
			
			% =================================================================
			% 2. Circuito
			
			\draw (G_mosfet) node[nmos, anchor=G](nmos){}; % NMOSFET
			\draw (G_mosfet) -- ++(-0.8,0) to [V, l_=\(V_{REF}\)] ++(0,-1.2) -- ++(1.78,0);
			\draw (nmos.S) -- ++(0,-0.4) node[sground]{}; % source del MOSFET
			\draw (nmos.D) -- ++(0, 0.2) to [R, l=\(R\), name=R1] ++(0, 1.2) node[rground, rotate=180, name=VDD]{}; % resistenza R e VDD
			
			% =================================================================
			% 3. Etichette e Vettori Corrente/Tensione
			
			% V_GS (Gate-Source Voltage)
			\node at ($(nmos.G)+(0, -0.2)$) {\(+\)};
			\node at ($(nmos.S)+(-0.25, 0.1)$) {\(-\)};
			\node at ($(nmos.G)+(0.2, -0.6)$) {\(V_{GS}\)}; % Testo V_GS sopra i segni
			
			% V_DS (Drain-Source Voltage)
			\node at ($(nmos.D)+(0.3, -0.2)$) {\(+\)};
			\node at ($(nmos.S)+(0.3, 0.2)$) {\(-\)};
			\node at ($(nmos.G)+(1.4, 0)$) {\(V_{DS}\)}; % Testo V_DS sopra i segni
			
			% V_R (Tensione sulla resistenza R)
			\node at ($(R1)+(0.4, 0.5)$) {\(+\)};
			\node at ($(R1)+(0.4, -0.5)$) {\(-\)};
			\node at ($(R1)+(0.5, 0)$) {\(V_R\)}; % Testo V_R sopra i segni
			
			% I_DS Corrente Drain-Source
			\node at ($(nmos.D)+(-0.5, -0.1)$) {\(I_{DS} \downarrow\)}; % Testo I_DS vicino al drain
			
			% VDD
			\node at ($(VDD)+(0, 0.6)$) {\(V_{DD}\)};
		\end{circuitikz}
	\end{minipage}
	\begin{minipage}{0.1\textwidth}
		\centering
		\(\longrightarrow\)
	\end{minipage}
	\begin{minipage}{0.4\textwidth}
		\centering \begin{circuitikz}[american]
			% =================================================================
			% 1. Nodi principali del circuito
			\coordinate (gnd_2) at (0,0); % ground secondo circuito
			
			% =================================================================
			% 2. Circuito
			\draw (gnd_2) node[sground]{}; % ground secondo circuito
			\draw (gnd_2) -- ++(0, 0.2) -- ++(-0.7, 0) to [I, l=\(I_{DSAT}\), invert] ++(0,1.4) -- ++(0.7,0); % generatore di corrente I_DSAT
			\draw (gnd_2) ++(0, 0.2) -- ++(0.7,0) to [R, l=\(R_0\), name=R0] ++(0,1.4) -- ++(-0.7,0) -- ++(0,0.2) node[name=R0]{}; % resistenza R0
			\draw (R0) ++(0,-0.2) -- ++(0, 0.6) to [R, l=\(R\), name=R1] ++(0, 1.15) node[rground, rotate=180, name=VDD]{}; % resistenza R e VDD
			
			% =================================================================
			% 3. Etichette e Vettori Corrente/Tensione
			
			% V_DS (Drain-Source Voltage)
			\node at ($(R0)+(1.1, -0.4)$) {\(+\)};
			\node at ($(R0)+(1.1, -1.4)$) {\(-\)};
			\node at ($(R0)+(1.3, -0.95)$) {\(V_{DS}\)}; % Testo V_DS sopra i segni
			
			% V_R (Tensione sulla resistenza R)
			\node at ($(R1)+(0.4, 0.5)$) {\(+\)};
			\node at ($(R1)+(0.4, -0.5)$) {\(-\)};
			\node at ($(R1)+(0.5, 0)$) {\(V_R\)}; % Testo V_R sopra i segni
			
			% I_DS Corrente Drain-Source
			\node at ($(R0)+(-0.5, 0.1)$) {\(I_{DS} \downarrow\)}; % Testo I_DS vicino al drain
			
			% VDD
			\node at ($(VDD)+(0, 0.6)$) {\(V_{DD}\)};
		\end{circuitikz}
	\end{minipage}
\end{center}

\noindent
L'utilizzo di un NMOSFET al posto di un PMOSFET è perfettamente arbitrario e non cambia nulla nel procedimento di soluzione
del circuito.

\newpage

\subsubsection*{Considerazioni sul regime di funzionamento e risoluzione grafica}
Il MOSFET in saturazione, connesso in questo modo,  si può modellare con un generatore lineare di corrente secondo il teorema
di Norton, come illustrato sopra. In saturazione, infatti, la corrente \(I_{DS}\) è direttamente proporzionale a \(V_{DS}\)
tramite il parametro \(\lambda_n\):
\[I_{DS} = I_{DSAT}(1 + \lambda_n V_{DS}) = I_{DSAT} + \lambda_n I_{DSAT} V_{DS} = I_{DSAT} + R_0 V_{DS} \qquad \text{con} \; R_0 = \frac{1}{\lambda_n I_{DSAT}}\]
Si osserva che la corrente \(I_{DS}\) ha due contributi:
\begin{itemize}
	\item una corrente costante \(I_{DSAT}\) di saturazione del mosfet, corrispone alla corrente erogata dal generatore ideale
	di corrente per Norton;
	\item una corrente variabile \(R_0 V_{DS}\) che dipende linearmente dalla tensione \(V_{DS}\) e che può essere modellata
	con una resistenza \(R_0\) in parallelo con il generatore ideale di corrente.
\end{itemize}
Si nota che se \(\lambda = 0 \; \rightarrow \; I_{DS} = I_{DSAT} \; \rightarrow \; R_0 = \infty\), ovvero se non si
considera l'effetto di modulazione di lunghezza di canale, il MOSFET si comporta come un generatore ideale di corrente.

Il generatore lineare di corrente di Norton ha dei vincoli sul suo funzionamento, in quanto deve essere soddisfatta la condizione
che il MOSFET rimanga in saturazione per pinchoff o in saturazione di velocità:
\[V_{DS} > \min \{V_{REF} - V_{TN} \;_\text{sat. per pinchoff}, \;\; V_{DSATN} \;_\text{sat. di velocità}\}\]
Se questa condizione non è soddisfatta, il MOSFET esce dal regime di saturazione e il modello con il generatore lineare di corrente
non è più valido.

\subsubsection*{Analisi del circuito}
Per risolvere il circuito utilizzando il modello a canale corto si procede come segue:
\begin{enumerate}
	\item si impone che il mosfet sia acceso e che sia in saturazione (per pinchoff o di velocità) per cui si deve avere:
	\(V_{REF} > V_{TN}\) e \(V_{DS} > \min \{V_{REF} - V_{TN}, \; V_{DSATN}\}\)
	\item si ipotizza il regime di saturazione (pinchoff o velocità) e si calcola \(I_{DSAT}\) e \(R_0\):
	\begin{itemize}[topsep=0pt]
		\item in regime di saturazione per pinchoff: \(\displaystyle I_{DSAT} = \frac{k_n}{2} (V_{REF} - V_{TN})^2\)
		\item in regime di saturazione di velocità: \(\displaystyle I_{DSAT} = k_n V_{DSATN}\left(V_{REF} - V_{TN} - \frac{V_{DSATN}}{2}\right)\)
		\item a prescindere dal regime di saturazione \(R_0 = 1 / (\lambda_n I_{DSAT})\)
	\end{itemize}
	\item si applica il teorema di Norton, ovvero si sostituisce il MOSFET con il generatore lineare di	corrente e
	si risolve il circuito (es. per sovrapposizione degli effetti) in funzione di \(V_{DS}\):
	\[V_{DS} = V_{DD} \frac{R_0}{R + R_0} - I_{DSAT} R_0 \frac{R}{R + R_0}\]
	\item si verifica l'ipotesi sul regime di saturazione sia soddisfatta:
	\begin{itemize}
		\item in regime di saturazione per pinchoff: \(V_{DS} > V_{REF} - V_{TN}\)
		\item in regime di saturazione di velocità: \(V_{DS} > V_{DSATN}\)
	\end{itemize}
	\item se la condizione è soddisfatta, si è trovata la soluzione; altrimenti si deve ripetere il procedimento per l'altro
	regime di saturazione
\end{enumerate}

Il modello a canale lungo è un caso particolare del modello a canale corto con \(\lambda_n = 0\) e \(R_0 = \infty\), per cui
il MOSFET si può modellare come un generatore ideale di corrente con corrente \(I_{DSAT}\) data dal regime di saturazione per
pinchoff. Non prevedendo l'effetto di modulazione di lunghezza di canale e il fenomeno di saturazione di velocità, alla fine
è sufficiente verificare solo la condizione \(V_{DS} > V_{REF} - V_{TN}\).

\newpage

\subsection{MOSFET usato come interruttore}
\subsubsection*{Modellizzazione del MOSFET come interruttore}
Nell'elettronica digitale i MOSFET sono utilizzati come interruttori ON/OFF. È possibile, infatti, modellare un MOSFET come un
interruttore ideale pilotato dalla tensione di gate \(V_{GS}\) in serie ad una resistenza \(R_n\) per gli NMOS o \(R_p\)
per i PMOS. Si considerano, quindi, soltanto due regimi di funzionamento:
\begin{itemize}
	\item MOSFET \say{spento}: interruttore aperto, \(I_{DS} = 0\), per \(V_{GS} < V_{TN}\) (NMOS) o \(V_{GS} > V_{TP}\) (PMOS);
	\item MOSFET \say{acceso}: resistore di resistenza \(R_n\) (NMOS) o \(R_p\) (PMOS).
\end{itemize}
\begin{center}
	\centering \includegraphics[width=0.6\textwidth]{immagini/5_circuiti_mosfet/circuiti_mos_3.png}
\end{center}

\subsection{Scarica di un condensatore con NMOS}
\subsubsection*{Circuito di scarica}
\begin{center}
	\begin{minipage}{0.37\textwidth}
		\centering
		\begin{circuitikz}
			\ctikzset{tripoles/mos style=arrows}
			
			\draw (G_mosfet) node[nmos, anchor=G, rotate=-90](nmos){}; % NMOSFET
			\draw (nmos.S) -- ++(0,-0.2) node[sground]{}; % source del MOSFET
			\draw (nmos.D) to [C, l=\(C\), i_=\(I_C\)] ++(0, -1.2) ++(0,0.3) node[sground]{}; % condensatore C
			\draw (nmos.G) -- ++(-1.8,0) to [V, l_=\(V_{G}\quad\quad\), name=Vg] ++(0,-1.5) ++(0,0.3) node[sground]{}; % generatore di tensione V_G
	
			% generatore
			\node at ($(Vg)+(-0.2,-0.55)$) {\(\;_-\)};
			\node at ($(Vg)+(-0.2,0.6)$) {\(\;_+\)};
			\node at ($(Vg)+(-1.5,-0.4)$) {\(0V \rightarrow V_{DD}\)};
	
			% I_DS Corrente Drain-Source
			\node at ($(nmos.D)+(-0.8, -0.2)$) {\(\stackrel{\leftarrow}{I_{DS}}\)}; % Testo I_DS vicino al drain

			% V_C tensione condensatore
			\node at ($(nmos.D)+(0.2, -0.2)$) {\(\;_+\)};
			\node at ($(nmos.D)+(0.22, -1)$) {\(\;_-\)};

			% terminali del mosfet
			\node at ($(nmos.G)+(0.2, -0.1)$) {\(G\)};
			\node at ($(nmos.D)+(-0.25, 0.2)$) {\(D\)};
			\node at ($(nmos.S)+(0.25, 0.2)$) {\(S\)};
		\end{circuitikz}
	\end{minipage}
	\begin{minipage}{0.1\textwidth}
		\centering
		\(\longrightarrow\)
	\end{minipage}
	\begin{minipage}{0.2\textwidth}
		\centering
		\begin{circuitikz}
			\draw (0,0) node[sground]{}; % ground secondo circuito
			\draw (0,0) to[short,-*] ++(0.5, 0) -- ++(0.5, 0.3) to [open,-*] ++(0, -0.3) -- ++(0.5, 0) to[C, l=\(C\)] ++(0, -1.2) ++(0,0.3) node[sground]{}; % nodo superiore
		\end{circuitikz}

		per \(t < 0\)
	\end{minipage}
	\begin{minipage}{0.3\textwidth}
		\centering
		\begin{circuitikz}[european]
			\draw (0,0) node[sground]{}; % ground secondo circuito
			\draw (0,0) to[short,-*] ++(0.5, 0) -- ++(0.5, 0.05) to [open,-*] ++(0, -0.05) -- ++(0.5, 0) to[R, l=\(R\)] ++(1.5,0)  to[C, l=\(C\)] ++(0, -1.2) ++(0,0.3) node[sground]{}; % nodo superiore
		\end{circuitikz}

		per \(t \geq 0\)
	\end{minipage}
\end{center}

\subsubsection*{Equazioni del circuito per le leggi di Kirchhoff e la struttura della rete}
\[V_G(t) = \begin{cases} 0 V  & t < 0 \\ V_{DD} & t \geq 0 \end{cases} \quad\qquad V_C(t \leq 0) = V_{DD} \qquad
\begin{array}{l} V_{GS}(t) = V_G(t) \\[3pt] V_{DS}(t) = V_C(t) \end{array} \qquad I_{DS} + I_C(t) = 0\]

\subsubsection*{Analisi delle fasi del transitorio}
Si analizzano le variabili del circuito per \(t < 0\), \(t = 0\), \(t > 0\) e \(t \to \infty\):
\begin{itemize}
	\item per \(t < 0\): l'NMOS è spento poiché \(V_{GS} = 0 V < V_{TN}\) e il condensatore è carico a \(V_{DD}\);
	\item per \(t = 0\): l'NMOS si accende poiché \(V_{GS} = V_{DD} > V_{TN}\);
	\item per \(t > 0\): l'NMOS rimane acceso poiché \(V_{GS} = V_{DD} > V_{TN}\) e il condensatore si scarica;
	\item per \(t \to \infty\): a regime l'NMOS rimane acceso e il condensatore si scarica completamente.
\end{itemize}

\subsubsection*{Analisi grafica delle fasi del transitorio}
Si suppone, sotto certi dati, che il mosfet lavori soltanto in saturazione di velocità o in regime lineare.
Si osserva che la scarica del condensatore avviene in due fasi (con un punto intermedio \(t = t_1\)):
\begin{enumerate}
	\item per \(0 < t < t_1\), \(V_{DS} > V_{DSATN}\) il mosfet lavora in regime di saturazione di velocità e la corrente di
	scarica è costante pari a \(I_{DSATN}\):
	\item per \(t = t_1\), \(V_{DS} = V_{DSATN}\) il mosfet passa dal regime di saturazione di velocità a regime lineare:
	\item per \(t > t_1\), \(V_{DS} < V_{DSATN}\) il mosfet lavora in regime lineare e la corrente di scarica dipende dalla tensione \(V_{DS}(t) = V_C(t)\)
	e quindi dalla tensione del condensatore
\end{enumerate}

\begin{center}
	\begin{minipage}{0.48\textwidth}
		\centering \includegraphics[width=0.95\textwidth]{immagini/5_circuiti_mosfet/scarica_nmos_1.png}
	\end{minipage}
	\begin{minipage}{0.48\textwidth}
		\centering \includegraphics[width=0.95\textwidth]{immagini/5_circuiti_mosfet/scarica_nmos_2.png}
	\end{minipage}
\end{center}

\subsubsection*{Soluzione analitica del circuito}
Si analizza la prima fase di scarica del condensatore per \(0 < t < t_1\) (saturazione di velocità):
\[I_C(t) = C\frac{dV_C}{dt} \qquad\qquad I_{DSATN} = k_n' Z_n V_{DSATN} \left(V_{DD} - V_{TN} - \frac{V_{DSATN}}{2}\right)\]
\[I_C(t) = -I_{DSATN} \quad \rightarrow \quad V_C(t) = V_{DD} - \frac{I_{DSATN}}{C} t\]

\noindent
Si analizza il passaggio dalla prima alla seconda fase di per \(t = t_1\):
\[V_{GS}(t_1) = V_{DSATN} \quad \rightarrow \quad V_C(t_1) = V_{DSATN} \quad \rightarrow \quad t_1 = \frac{C}{I_{DSATN}} (V_{DD} - V_{DSATN})\]

\noindent
Si analizza la seconda fase di scarica del condensatore per \(t > t_1\) (regime lineare):
\[I_C(t) = C\frac{dV_C}{dt} \qquad\qquad I_{DS}(t) = k_n' Z_n V_C(t) \left(V_{DD} - V_{TN} - \frac{V_C(t)}{2}\right)\]
\[I_C(t) = -I_{DS}(t) \quad \rightarrow \quad V_C(t) = \frac{1}{A \, e^{\alpha (t - t_1)} + B}\]
\[\alpha = \frac{k_n' Z_n}{C} (V_{DD} - V_{TN}) \qquad A = \frac{1}{V_{DSATN}} - \frac{1}{2(V_{DD} - V_{TN})} \qquad B = \frac{1}{2(V_{DD} - V_{TN})}\]

\subsubsection*{Carica del condensatore a regime}
Per \(t \to \infty\), la tensione sul condensatore si riduce a zero e il condensatore si scarica completamente:
\[\lim_{t \to \infty} V_C(t) = 0 V\]

\subsubsection*{Tempo di dimezzamento di \(V_C(t)\)}
Il tempo di dimezzamento \(t_{1/2}\) è il tempo necessario affinché la tensione sul condensatore si riduca alla metà del suo
valore iniziale \(V_{DD}\), ovvero quando \(V_C(t_{1/2}) = V_{DD}/2\). Si osserva che tale valore viene assunto durante la prima
fase di scarica, in cui il mosfet lavora in saturazione di velocità. Si calcola, quindi, \(t_{1/2}\) come segue:
\[V_C(t_{1/2}) = \frac{V_{DD}}{2} \quad \rightarrow \quad t_{1/2} = \frac{C V_{DD}}{2 I_{DSATN}}\]

\newpage

\subsection{Carica di un condensatore con PMOS}
\subsubsection*{Circuito di carica}
\begin{center}
	\begin{minipage}{0.37\textwidth}
		\centering
		\begin{circuitikz}
			\ctikzset{tripoles/mos style=arrows}
			\ctikzset{tripoles/pmos style/emptycircle}

			\draw (G_mosfet) node[pmos, anchor=G, rotate=90, xscale=-1](pmos){}; % PMOSFET con source/drain invertiti
			\draw (pmos.S) -- ++(-0.1,0) node[rground, rotate=180, name=VDD]{}; % source del MOSFET
			\draw (pmos.D) to [C, l=\(C\), i_=\(I_C\)] ++(0, -1.2) ++(0,0.3) node[sground]{}; % condensatore C
			\draw (pmos.G) -- ++(-1.8,0) to [V, l_=\(V_{G}\quad\quad\), name=Vg] ++(0,-1.5) ++(0,0.3) node[sground]{}; % generatore di tensione V_G

			% generatore
			\node at ($(Vg)+(-0.2,-0.55)$) {\(\;_-\)};
			\node at ($(Vg)+(-0.2,0.6)$) {\(\;_+\)};
			\node at ($(Vg)+(-1.5,-0.4)$) {\(V_{DD} \rightarrow 0V\)};
	
			% I_DS Corrente Drain-Source
			\node at ($(pmos.D)+(-0.8, -0.2)$) {\(\stackrel{\rightarrow}{I_{DS}}\)}; % Testo I_DS vicino al drain
	
			% V_C Corrente condensatore
			\node at ($(pmos.D)+(0.2, -0.2)$) {\(\;_+\)};
			\node at ($(pmos.D)+(0.22, -1)$) {\(\;_-\)};

			% VDD
			\node at ($(VDD)+(0, 0.6)$) {\(V_{DD}\)};

			% terminali del mosfet
			\node at ($(nmos.G)+(0.2, -0.1)$) {\(G\)};
			\node at ($(nmos.D)+(-0.25, 0.2)$) {\(D\)};
			\node at ($(nmos.S)+(0.25, 0.2)$) {\(S\)};
		\end{circuitikz}
	\end{minipage}
	\begin{minipage}{0.1\textwidth}
		\centering
		\(\longrightarrow\)
	\end{minipage}
	\begin{minipage}{0.2\textwidth}
		\centering
		\begin{circuitikz}
			\draw (0,0) node[rground, rotate=180, name=VDD]{}; % ground secondo circuito
			\draw (0,0) to[short,-*] ++(0.5, 0) -- ++(0.5, 0.3) to [open,-*] ++(0, -0.3) -- ++(0.5, 0) to[C, l=\(C\)] ++(0, -1.2) ++(0,0.3) node[sground]{}; % nodo superiore
			\node at ($(VDD)+(0, 0.6)$) {\(V_{DD}\)};
		\end{circuitikz}

		per \(t < 0\)
	\end{minipage}
	\begin{minipage}{0.3\textwidth}
		\centering
		\begin{circuitikz}[european]
			\draw (0,0) node[rground, rotate=180, name=VDD]{}; % ground secondo circuito
			\draw (0,0) to[short,-*] ++(0.5, 0) -- ++(0.5, 0.05) to [open,-*] ++(0, -0.05) -- ++(0.5, 0) to[R, l=\(R\)] ++(1.5,0)  to[C, l=\(C\)] ++(0, -1.2) ++(0,0.3) node[sground]{}; % nodo superiore
			\node at ($(VDD)+(0, 0.6)$) {\(V_{DD}\)};
		\end{circuitikz}

		per \(t \geq 0\)
	\end{minipage}
\end{center}

\subsubsection*{Equazioni del circuito per le leggi di Kirchhoff e la struttura della rete}
\[V_G(t) = \begin{cases} V_{DD}  & t < 0 \\ 0 V & t \geq 0 \end{cases} \quad\qquad V_C(t \leq 0) = 0 V \qquad
\begin{array}{l} V_{GS}(t) = V_G(t) - V_{DD} \\[3pt] V_{DS}(t) = V_C(t) - V_{DD} \end{array} \qquad I_{DS} = I_C(t)\]

\subsubsection*{Analisi delle fasi del transitorio}
Si analizzano le variabili del circuito per \(t < 0\), \(t = 0\), \(t > 0\) e \(t \to \infty\):
\begin{itemize}
	\item per \(t < 0\): il PMOS è spento poiché \(V_{GS} = 0 V > V_{TP}\) e il condensatore è scarico con \(V_C = 0 V\);
	\item per \(t = 0\): il PMOS si accende poiché \(V_{GS} = - V_{DD} < V_{TP}\);
	\item per \(t > 0\): il PMOS rimane acceso poiché \(V_{GS} = - V_{DD} < V_{TP}\) e il condensatore si carica;
	\item per \(t \to \infty\): a regime il PMOS rimane acceso e il condensatore si carica completamente a \(V_{DD}\).
\end{itemize}

\subsubsection*{Analisi grafica delle fasi del transitorio}
Si suppone, sotto certi dati, che il mosfet lavori soltanto in saturazione di velocità o in regime lineare.
Si osserva che la scarica del condensatore avviene in due fasi (con un punto intermedio \(t = t_1\)):
\begin{enumerate}
	\item per \(0 < t < t_1\), \(V_{DS} < V_{DSATP}\) il mosfet lavora in regime di saturazione di velocità e la corrente di
	scarica è costante pari a \(I_{DSATP}\):
	\item per \(t = t_1\), \(V_{DS} = V_{DSATP}\) il mosfet passa dal regime di saturazione di velocità a regime lineare:
	\item per \(t > t_1\), \(V_{DS} > V_{DSATP}\) il mosfet lavora in regime lineare e la corrente di scarica dipende dalla tensione
	\(V_{DS}(t) = V_C(t) - V_{DD}\) e quindi dalla tensione del condensatore
\end{enumerate}

\begin{center}
	\begin{minipage}{0.48\textwidth}
		\centering \includegraphics[width=0.95\textwidth]{immagini/5_circuiti_mosfet/carica_pmos_1.png}
	\end{minipage}
	\begin{minipage}{0.48\textwidth}
		\centering \includegraphics[width=0.95\textwidth]{immagini/5_circuiti_mosfet/carica_pmos_2.png}
	\end{minipage}
\end{center}

\newpage

\subsubsection*{Soluzione analitica del circuito}
Si analizza la prima fase di scarica del condensatore per \(0 < t < t_1\) (saturazione di velocità):
\[I_C(t) = C\frac{dV_C}{dt} \qquad\qquad I_{DSATP} = k_p' Z_p V_{DSATP} \left(-V_{DD} - V_{TP} - \frac{V_{DSATP}}{2}\right)\]
\[I_C(t) = I_{DSATP} \quad \rightarrow \quad V_C(t) = \frac{I_{DSATP}}{C} t\]

\noindent
Si analizza il passaggio dalla prima alla seconda fase di per \(t = t_1\):
\[V_{DS}(t_1) = V_{DSATP} \quad \rightarrow \quad V_C(t_1) - V_{DD} = V_{DSATP} \quad \rightarrow \quad t_1 = \frac{C}{I_{DSATP}} (V_{DD} + V_{DSATP})\]

\noindent
Si analizza la seconda fase di scarica del condensatore per \(t > t_1\) (regime lineare):
\[I_C(t) = C\frac{dV_C}{dt} \qquad I_{DS}(t) = k_p' Z_p (V_C(t) - V_{DD}) \left(-V_{DD} - V_{TP} - \frac{V_C(t)-V_{DD}}{2}\right)\]
\[I_C(t) = I_{DS}(t) \quad \rightarrow \quad V_C(t) = \frac{1}{A \, e^{\alpha (t - t_1)} + B} + V_{DD}\]
\[\alpha = \frac{k_p' Z_p}{C} (V_{DD} + V_{TP}) \qquad A = \frac{1}{V_{DSATP}} + \frac{1}{2(V_{DD} + V_{TP})} \qquad B = -\frac{1}{2(V_{DD} + V_{TP})}\]

\subsubsection*{Carica del condensatore a regime}
Per \(t \to \infty\), la tensione sul condensatore raggiunge il valore \(V_{DD}\) e il condensatore si carica completamente:
\[\lim_{t \to \infty} V_C(t) = V_{DD}\]

\subsubsection*{Tempo di dimezzamento di \(V_C(t)\)}
Si analizza il tempo di dimezzamento \(t_{1/2}\) osservando che tale valore viene assunto durante la prima fase di carica, in
cui il mosfet lavora in saturazione di velocità:
\[V_C(t_{1/2}) = \frac{V_{DD}}{2} \quad \rightarrow \quad t_{1/2} = \frac{C V_{DD}}{2 I_{DSATP}}\]

\newpage

\subsection{Scarica di un condensatore con un PMOS}
\subsubsection*{Circuito di scarica}
\begin{center}
	\begin{minipage}{0.37\textwidth}
		\centering
		\begin{circuitikz}
			\ctikzset{tripoles/mos style=arrows}
			\ctikzset{tripoles/pmos style/emptycircle}

			\draw (G_mosfet) node[pmos, anchor=G, rotate=-90](pmos){}; % PMOSFET
			\draw (pmos.D) -- ++(0,-0.2) node[sground]{}; % source del MOSFET
			\draw (pmos.S) to [C, l=\(C\), i_=\(I_C\)] ++(0, -1.2) ++(0,0.3) node[sground]{}; % condensatore C
			\draw (pmos.G) -- ++(-1.8,0) to [V, l_=\(V_{G}\quad\quad\), name=Vg] ++(0,-1.5) ++(0,0.3) node[sground]{}; % generatore di tensione V_G
	
			% generatore
			\node at ($(Vg)+(-0.2,-0.55)$) {\(\;_-\)};
			\node at ($(Vg)+(-0.2,0.6)$) {\(\;_+\)};
			\node at ($(Vg)+(-1.5,-0.4)$) {\(V_{DD} \rightarrow 0V\)};
	
			% I_DS Corrente Drain-Source
			\node at ($(pmos.D)+(0.8, -0.2)$) {\(\stackrel{\leftarrow}{I_{DS}}\)}; % Testo I_DS vicino al drain
	
			% V_C Corrente condensatore
			\node at ($(pmos.S)+(0.2, -0.2)$) {\(\;_+\)};
			\node at ($(pmos.S)+(0.22, -1)$) {\(\;_-\)};

			% terminali del mosfet
			\node at ($(nmos.G)+(0.2, -0.1)$) {\(G\)};
			\node at ($(nmos.S)+(0.25, 0.2)$) {\(D\)};
			\node at ($(nmos.D)+(-0.25, 0.2)$) {\(S\)};
		\end{circuitikz}
	\end{minipage}
	\begin{minipage}{0.1\textwidth}
		\centering
		\(\longrightarrow\)
	\end{minipage}
	\begin{minipage}{0.2\textwidth}
		\centering
		\begin{circuitikz}
			\draw (0,0) node[sground]{}; % ground secondo circuito
			\draw (0,0) to[short,-*] ++(0.5, 0) -- ++(0.5, 0.3) to [open,-*] ++(0, -0.3) -- ++(0.5, 0) to[C, l=\(C\)] ++(0, -1.2) ++(0,0.3) node[sground]{}; % nodo superiore
		\end{circuitikz}

		per \(t < 0\)
	\end{minipage}
	\begin{minipage}{0.3\textwidth}
		\centering
		\begin{circuitikz}[european]
			\draw (0,0) node[sground]{}; % ground secondo circuito
			\draw (0,0) to[short,-*] ++(0.5, 0) -- ++(0.5, 0.05) to [open,-*] ++(0, -0.05) -- ++(0.5, 0) to[R, l=\(R\)] ++(1.5,0)  to[C, l=\(C\)] ++(0, -1.2) ++(0,0.3) node[sground]{}; % nodo superiore
		\end{circuitikz}

		per \(t \geq 0\)
	\end{minipage}
\end{center}

\subsubsection*{Equazioni del circuito per le leggi di Kirchhoff e la struttura della rete}
\[V_G(t) = \begin{cases} V_{DD} & t < 0 \\ 0 V & t \geq 0 \end{cases} \quad\qquad V_C(t \leq 0) = V_{DD} \qquad
\begin{array}{l} V_{GS}(t) = V_G(t) - V_C(t) \\[3pt] V_{DS}(t) = -V_C(t) \end{array} \qquad I_{DS} + I_C(t) = 0\]

\subsubsection*{Analisi delle fasi del transitorio}
Si analizzano le variabili del circuito per \(t < 0\), \(t = 0\), \(t > 0\) e \(t \to \infty\):
\begin{itemize}
	\item per \(t < 0\): il PMOS è spento poiché \(V_{GS}(t) = 0V > V_{TP}\) e il condensatore è carico a \(V_{DD}\);
	\item per \(t = 0\): il PMOS si accende poiché \(V_{GS}(t) = -V_{DD} < V_{TP}\);
	\item per \(t > 0\): il PMOS è acceso finché \(V_{GS}(t) < V_{TP} \;\; \rightarrow \;\; V_C(t) > -V_{TP}\)
	\item per \(t \rightarrow \infty\): a regime il PMOS si spegne per \(V_{GS}(t) = V_{TP} \; \rightarrow \; V_C(t) = -V_{TP}\)
	lasciando il condensatore carico con tensione finale \(V_C(t) = -V_{TP}\)
\end{itemize}

\subsubsection*{Analisi grafica delle fasi del transitorio}
Il PMOS è connesso a diodo (gate e drain entrambi a massa per \(t\geq 0\)), quindi lavora solo in saturazione.
Si osserva che la scarica del condensatore avviene in due fasi (con un punto intermedio \(t = t_1\)):
\begin{enumerate}
	\item per \(0 < t < t_1\), \(V_{GS} - V_{TP} < V_{DSATP}\) il mosfet lavora in regime di saturazione di velocità e
	la corrente di scarica è costante pari a \(I_{DSATP}\):
	\item per \(t = t_1\), \(V_{GS} - V_{TP} = V_{DSATP}\) il mosfet passa dal regime di saturazione per velocità a quello di
	saturazione per pinchoff:
	\item per \(t > t_1\), \(V_{GS} - V_{TP} > V_{DSATP}\) il mosfet lavora in saturazione per pinchoff e la corrente di scarica
	dipende da \(V_{GS}(t) - V_{TP} = -V_{C}(t) - V_{TP}\) e quindi anche dalla tensione del condensatore \(V_C(t)\)
\end{enumerate}

\begin{center}
	\begin{minipage}{0.48\textwidth}
		\centering \includegraphics[width=0.95\textwidth]{immagini/5_circuiti_mosfet/scarica_pmos_1.png}
	\end{minipage}
	\begin{minipage}{0.48\textwidth}
		\centering \includegraphics[width=0.95\textwidth]{immagini/5_circuiti_mosfet/scarica_pmos_2.png}
	\end{minipage}
\end{center}

\subsubsection*{Soluzione analitica del circuito}
Si analizza la prima fase di scarica del condensatore per \(0 < t < t_1\) (saturazione di velocità):
\[I_C(t) = C\frac{dV_C}{dt} \qquad\qquad I_{DSATP}(t) = k_p' Z_p V_{DSATP} \left(-V_{C}(t) - V_{TP} - \frac{V_{DSATP}}{2}\right)\]
\[I_C(t) = I_{DSATP}(t) \;\; \rightarrow \;\; V_C(t) = V_{DD} - \left(V_{TP} + \frac{V_{DSATP}}{2}\right) \left(1-e^{-\alpha t}\right) \quad \text{con} \; \alpha = -\frac{k_p' Z_p V_{DSATP}}{C}\]

\noindent
Si analizza il passaggio dalla prima alla seconda fase di per \(t = t_1\):
\[V_{GS}(t_1) - V_{TP} = V_{DSATP} \;\; \rightarrow \;\; V_C(t_1) = -V_{TP} - V_{DSATP} \;\; \rightarrow \;\; t_1 = \frac{1}{\alpha} \ln \left(\frac{2 V_{DD} + 2V_{TP} + V_{DSATP}}{-V_{DSATP}}\right)\]

\noindent
Si analizza la seconda fase di scarica del condensatore per \(t > t_1\) (saturazione per pinchoff):
\[I_C(t) = C\frac{dV_C}{dt} \qquad\qquad I_{DS}(t) = \frac{k_p' Z_p}{2} (V_{GS}(t) - V_{TP})^2 = \frac{k_p' Z_p}{2} (-V_C(t) - V_{TP})^2\]
\[I_C(t) = I_{DS}(t) \quad \rightarrow \quad V_C(t) = -V_{TP} + \frac{2CV_{DSATP}}{k_p' Z_p V_{DSATP}(t-t_1)-2C}\]

\subsubsection*{Valore di regime}
Si analizza il valore di regime della tensione sul condensatore \(V_C(t)\) per \(t \rightarrow \infty\):
\[\lim_{t \to \infty} V_C(t) = -V_{TP}\]
Si ha, quindi, che il condensatore si scarica fino a raggiungere la tensione di soglia del PMOS.

\subsubsection*{Tempo di dimezzamento di \(V_C(t)\)}
Si analizza il tempo di dimezzamento \(t_{1/2}\), osservando che tale valore viene assunto durante la seconda fase di scarica,
in cui il mosfet lavora in saturazione per pinchoff:
\[V_C(t_{1/2}) = \frac{V_{DD}}{2} \quad \rightarrow \quad t_{1/2} = t_1 - 2C \frac{V_{DD} + 2V_{TP} + 2V_{DSATP}}{k_p' Z_p V_{DSATP}(V_{DD} + 2V_{TP})}\]

\newpage

\subsection{Carica di un condensatore con un NMOS}
\subsubsection*{Circuito di carica}
\begin{center}
	\begin{minipage}{0.37\textwidth}
		\centering
		\begin{circuitikz}
			\ctikzset{tripoles/mos style=arrows}
			\ctikzset{tripoles/pmos style/emptycircle}

			\draw (G_mosfet) node[nmos, anchor=G, rotate=90, xscale=-1](nmos){}; % PMOSFET con source/drain invertiti
			\draw (nmos.D) -- ++(-0.1,0) node[rground, rotate=180, name=VDD]{}; % source del MOSFET
			\draw (nmos.S) to [C, l=\(C\), i_=\(I_C\)] ++(0, -1.2) ++(0,0.3) node[sground]{}; % condensatore C
			\draw (nmos.G) -- ++(-1.8,0) to [V, l_=\(V_{G}\quad\quad\), name=Vg] ++(0,-1.5) ++(0,0.3) node[sground]{}; % generatore di tensione V_G

			% generatore
			\node at ($(Vg)+(-0.2,-0.55)$) {\(\;_-\)};
			\node at ($(Vg)+(-0.2,0.6)$) {\(\;_+\)};
			\node at ($(Vg)+(-1.5,-0.4)$) {\(0V \rightarrow V_{DD}\)};
	
			% I_DS Corrente Drain-Source
			\node at ($(nmos.D)+(0.8, -0.2)$) {\(\stackrel{\rightarrow}{I_{DS}}\)}; % Testo I_DS vicino al drain
	
			% V_C Corrente condensatore
			\node at ($(nmos.S)+(0.2, -0.2)$) {\(\;_+\)};
			\node at ($(nmos.S)+(0.22, -1)$) {\(\;_-\)};

			% VDD
			\node at ($(VDD)+(0, 0.6)$) {\(V_{DD}\)};

			% terminali del mosfet
			\node at ($(nmos.G)+(0.2, -0.1)$) {\(G\)};
			\node at ($(nmos.D)+(0.25, 0.2)$) {\(D\)};
			\node at ($(nmos.S)+(-0.25, 0.2)$) {\(S\)};
		\end{circuitikz}
	\end{minipage}
	\begin{minipage}{0.1\textwidth}
		\centering
		\(\longrightarrow\)
	\end{minipage}
	\begin{minipage}{0.2\textwidth}
		\centering
		\begin{circuitikz}
			\draw (0,0) node[rground, rotate=180, name=VDD]{}; % ground secondo circuito
			\draw (0,0) to[short,-*] ++(0.5, 0) -- ++(0.5, 0.3) to [open,-*] ++(0, -0.3) -- ++(0.5, 0) to[C, l=\(C\)] ++(0, -1.2) ++(0,0.3) node[sground]{}; % nodo superiore
			\node at ($(VDD)+(0, 0.6)$) {\(V_{DD}\)};
		\end{circuitikz}

		per \(t < 0\)
	\end{minipage}
	\begin{minipage}{0.3\textwidth}
		\centering
		\begin{circuitikz}[european]
			\draw (0,0) node[rground, rotate=180, name=VDD]{}; % ground secondo circuito
			\draw (0,0) to[short,-*] ++(0.5, 0) -- ++(0.5, 0.05) to [open,-*] ++(0, -0.05) -- ++(0.5, 0) to[R, l=\(R\)] ++(1.5,0)  to[C, l=\(C\)] ++(0, -1.2) ++(0,0.3) node[sground]{}; % nodo superiore
			\node at ($(VDD)+(0, 0.6)$) {\(V_{DD}\)};
		\end{circuitikz}

		per \(t \geq 0\)
	\end{minipage}
\end{center}

\subsubsection*{Equazioni del circuito per le leggi di Kirchhoff e la struttura della rete}
\[V_G(t) = \begin{cases} 0 V & t < 0 \\ V_{DD} & t \geq 0 \end{cases} \quad\qquad V_C(t \leq 0) = 0 V \qquad
\begin{array}{l} V_{GS}(t) = V_G(t) - V_C(t) \\[3pt] V_{DS}(t) = V_{DD} - V_C(t) \end{array} \qquad I_{DS} = I_C(t)\]

\subsubsection*{Analisi delle fasi del transitorio}
Si analizzano le variabili del circuito per \(t < 0\), \(t = 0\), \(t > 0\) e \(t \to \infty\):
\begin{itemize}
	\item per \(t < 0\): l'NMOS è spento poiché \(V_{GS}(t) = 0V < V_{TN}\);
	\item per \(t = 0\): l'MOS si accende poiché \(V_{GS}(t) = V_{DD} > V_{TN}\);
	\item per \(t > 0\): l'MOS è acceso finché \(V_{GS}(t) > V_{TN} \;\; \rightarrow \;\; V_C(t) < V_{DD} -V_{TN}\)
	\item per \(t \rightarrow \infty\): a regime l'MOS si spegne quando \(V_{GS}(t) = V_{TN} \; \rightarrow \; V_C(t) = V_{DD} - V_{TN}\)
	lasciando il condensatore carico con tensione finale \(V_C(t) = V_{DD} - V_{TN}\)
\end{itemize}

\subsubsection*{Analisi grafica delle fasi del transitorio}
Il PMOS è connesso a diodo (gate e drain entrambi a \(V_{DD}\) per \(t\geq 0\)), quindi lavora solo in saturazione.
Si osserva che la carica del condensatore avviene in due fasi (con un punto intermedio \(t = t_1\)):
\begin{enumerate}
	\item per \(0 < t < t_1\), \(V_{GS} - V_{TN} > V_{DSATN}\) il mosfet lavora in regime di saturazione di velocità e
	la corrente di carica è costante pari a \(I_{DSATN}\):
	\item per \(t = t_1\), \(V_{GS} - V_{TN} = V_{DSATN}\) il mosfet passa dal regime di saturazione per velocità a quello di
	saturazione per pinchoff:
	\item per \(t > t_1\), \(V_{GS} - V_{TN} < V_{DSATN}\) il mosfet lavora in saturazione per pinchoff e la corrente di scarica
	dipende da \(V_{GS}(t) - V_{TN} = V_{DD} - V_{C}(t) - V_{TN}\), quindi anche dalla tensione del condensatore \(V_C(t)\)
\end{enumerate}

\begin{center}
	\begin{minipage}{0.48\textwidth}
		\centering \includegraphics[width=0.95\textwidth]{immagini/5_circuiti_mosfet/carica_nmos_1.png}
	\end{minipage}
	\begin{minipage}{0.48\textwidth}
		\centering \includegraphics[width=0.95\textwidth]{immagini/5_circuiti_mosfet/carica_nmos_2.png}
	\end{minipage}
\end{center}

\subsubsection*{Soluzione analitica del circuito}
Si analizza la prima fase di carica del condensatore per \(0 < t < t_1\) (saturazione di velocità):
\[I_C(t) = C\frac{dV_C}{dt} \qquad\qquad I_{DSATN}(t) = k_p' Z_p V_{DSATN} \left(V_{DD} - V_{C}(t) - V_{TN} - \frac{V_{DSATN}}{2}\right)\]
\[I_C(t) = I_{DSATN}(t) \;\; \rightarrow \;\; V_C(t) = \left(V_{DD} - V_{TN} - \frac{V_{DSATN}}{2}\right) \left(1-e^{-\alpha t}\right) \quad \text{con} \; \alpha = -\frac{k_p' Z_p V_{DSATN}}{C}\]

\noindent
Si analizza il passaggio dalla prima alla seconda fase di per \(t = t_1\):
\[V_{GS}(t_1) - V_{TN} = V_{DSATN} \;\; \rightarrow \;\; V_C(t_1) = V_{DD} -V_{TN} - V_{DSATN} \;\; \rightarrow \;\; t_1 = \frac{1}{\alpha} \ln \left(\frac{V_{DD} - V_{TN}}{V_{DSATN}} - 1\right)\]

\noindent
Si analizza la seconda fase di carica del condensatore per \(t > t_1\) (saturazione per pinchoff):
\[I_C(t) = C\frac{dV_C}{dt} \qquad\qquad I_{DS}(t) = \frac{k_p' Z_p}{2} (V_{GS}(t) - V_{TN})^2 = \frac{k_p' Z_p}{2} (V_{DD} - V_C(t) - V_{TN})^2\]
\[I_C(t) = I_{DS}(t) \quad \rightarrow \quad V_C(t) = V_{DD} - V_{TN} - \frac{2CV_{DSATN}}{k_p' Z_p V_{DSATN}(t-t_1)+2C}\]

\subsubsection*{Valore di regime}
Si analizza il valore di regime della tensione sul condensatore \(V_C(t)\) per \(t \rightarrow \infty\):
\[\lim_{t \to \infty} V_C(t) = V_{DD} - V_{TN}\]
Si ha, quindi, che il condensatore si carica fino a raggiungere la tensione finale \(V_C(t) = V_{DD} - V_{TN}\).

\subsubsection*{Tempo di dimezzamento di \(V_C(t)\)}
Si analizza il tempo di dimezzamento \(t_{1/2}\), osservando che tale valore viene assunto durante la prima fase di carica,
in cui il mosfet lavora in saturazione per velocità:
\[V_C(t_{1/2}) = \frac{V_{DD}}{2} \quad \rightarrow \quad t_{1/2} = \frac{1}{\alpha}  \ln \left(\frac{2V_{DD} - 2V_{TN} - V_{DSATN}}{V_{DD} - 2V_{TN} - V_{DSATN}}\right)\]

\newpage

\subsection{Confronto dei transitori per carica e scarica con NMOS e PMOS}
\subsubsection*{Dati}
Si scelgono i seguenti dati per il confronto dei transitori di carica e scarica con NMOS e PMOS:
\begin{center}
	\begin{tabular}{l | l l | l l}
		circuito & NMOS & & PMOS & \\
		\toprule
		\(C = 10 fF\) & \(k_n' = 125 \mu \A/\V^2\) & \(Z_n = 2\) & \(k_p' = 40 \mu \A/\V^2\) & \(Z_p = 2 \) \\[4pt]
		 \(V_{DD} = 2.5 V\) & \(V_{TN} = 0.5 V\) & \(V_{DSATN} = 0.6 V\) & \(V_{TP} = -0.5 V\) & \(V_{DSATP} = -0.8 V\)
	\end{tabular}
\end{center}

\subsubsection*{Confronto dei tempi caratteristici}
\begin{center}
	\begin{tabular}{l c c c}
		circuito & \(t_1\) & \(t_{1/2}\) & \(V_C(t \to \infty)\)\\
		\toprule
		scarica con NMOS & \(74.5 \, \ps\) & \(49 \, \ps\) & \(0 \V\) \\[4pt]
		carica con NMOS & \(115.6 \, \ps\) & \(88.6 \, \ps\) & \(V_{DD} - V_{TN} < V_{DD}\) \\[4pt]
		carica con PMOS & \(151.7 \, \ps\) & \(122 \, \ps\) & \(V_{DD}\) \\[4pt]
		scarica con PMOS & \(216.6 \, \ps\) & \(237 \, \ps\) & \(-V_{TP} > 0 \V\)
	\end{tabular}
\end{center}
Si nota che:
\begin{itemize}
	\item l'NMOS scarica totalmente il condensatore, ma lo carica parzialmente fino ad un valore inferiore a \(V_{DD}\),
	si dice che \textbf{l'NMOS trasmette bene il valore logico basso} (0V);
	\item il PMOS carica totalmente il condensatore, ma lo scarica parzialmente fino ad un valore superiore a \(0 \V\),
	si dice che \textbf{il PMOS trasmette bene il valore logico alto} (\(V_{DD}\));
	\item i tempi di dimezzamento per trasmettere il valore logico \say{cattivo} (carica con NMOS e scarica con PMOS) sono circa
	il doppio rispetto a quelli per trasmettere il valore logico \say{buono} (scarica con NMOS e carica con PMOS);
	\item il tempo di carica con PMOS è circa il doppio rispetto al tempo di scarica con NMOS, a parità del fattore di forma \(Z_n = Z_p\)
\end{itemize}

\subsubsection*{Effetto body nei circuiti di carica e scarica dei condensatori}
Spesso, nei circuiti digitali, i body degli NMOS sono collegati a massa e quelli dei PMOS a \(V_{DD}\), non necessariamente al loro
source. Nell'NMOS in scarica e nel PMOS in carica ciò non comporta variazioni, mentre nell'NMOS in carica e nel PMOS in scarica
si ha un aumento in modulo della soglia di tensione \(V_{TN}\) e \(V_{TP}\) rispettivamente, a causa dell'effetto body.
Ciò comporta un aumento dei tempi di carica e scarica dei condensatori e un \textbf{peggioramento ulteriore nella trasmissione
dei valori logici \say{cattivi}}.

\newpage

\subsection{Resistenza equivalente del MOSFET come interruttore}
Per facilitare l'analisi dei circuiti con MOSFET come interruttori, si può approssimare il comportamento del MOSFET acceso con
una resistenza equivalente \(R_{eq}\). In questo modo si linearizza il comportamento non lineare del MOSFET e si possono utilizzare
le tecniche di analisi dei circuiti lineari. In particolare le curve di scarica e carica diventano esponenziali con costante di
tempo \(\tau = R_{eq} C\).

\begin{center}
	\includegraphics[width=0.7\textwidth]{immagini/5_circuiti_mosfet/req_1.png}

	scarica con un NMOS \hspace{3cm} carica con un PMOS
\end{center}

\subsubsection*{Definizione di resistenza equivalente \(R_{eq}\)}

Si definisce la resistenza equivalente \(R_{eq}\) del MOSFET come interruttore acceso come la media della resistenza istantanea
agli estremi dell'intervallo di interesse, ovvero a \(t = 0\) e a \(t = t_{1/2}\), quando il MOSFET trasmette il valore logico
\say{buono}.

\begin{center}
	\begin{minipage}{0.35\textwidth}
		\centering \includegraphics[width=\textwidth]{immagini/5_circuiti_mosfet/req_2.png}
	\end{minipage}
	\begin{minipage}{0.6\textwidth}
		\[\text{NMOS:} \quad R_{n}(0) = \frac{V_{DD}}{I_{DSATN}}, \quad R_{n}(t_{1/2}) = \frac{V_{DD} / 2}{I_{DSATN}} \;\; \rightarrow \]
		\[\quad \rightarrow \;\; R_{n} = \frac{R_n(0) + R_n(t_{1/2})}{2} = \frac{3}{4} \frac{V_{DD}}{I_{DSATN}}\]
		\[\text{PMOS:} \quad R_{p}(0) = \frac{V_{DD}}{I_{DSATP}}, \quad R_{p}(t_{1/2}) = \frac{V_{DD} / 2}{I_{DSATP}} \;\; \rightarrow \]
		\[\quad \rightarrow \;\; R_{p} = \frac{R_p(0) + R_p(t_{1/2})}{2} = \frac{3}{4} \frac{V_{DD}}{I_{DSATP}}\]
	\end{minipage}
\end{center}

\noindent
NOTE:
\begin{itemize}
	\item nelle elaborazioni delle reti logiche digitali, si considera solo la prima parte del transitorio, ovvero fino al tempo
	di dimezzamento \(t_{1/2}\), in quanto è il momento in cui la tensione è vicina alla tensione di soglia logica, ovvero al
	valore in cui il segnale digitale cambia stato logico (da 0 a 1 o da 1 a 0);
	\item si calcola la resistenza solo nei casi in cui il MOSFET trasmette il valore logico \say{buono}, siccome è il comportamento
	più ricercato ed utilizzato nei circuiti digitali; più avanti viene anche approfondita la resistenza equivalente per valori logici
	\say{cattivi}.
\end{itemize}

\subsubsection*{Fattore di forma e resistenza equivalente}
La corrente di saturazione \(I_{DSAT}\) dipende dal fattore di forma \(Z\) del MOSFET e può esser riscritta come:
\[I_{DSATN} = Z_n V_{DSATN0} \qquad\qquad I_{DSATP} = Z_p V_{DSATP0}\]
Per cui la resistenza equivalente può essere riscritta in funzione delle costanti \(R_{n0}\) e \(R_{p0}\), ovvero le resistenze
equivalenti per fattore di forma unitario \(Z_n = Z_p = 1\).
\[R_{n} = \frac{R_{n0}}{Z_n} \;\; \text{con} \; R_{n0} = \frac{3}{4} \frac{V_{DD}}{V_{DSATN0}} \qquad\qquad R_{p} = \frac{R_{p0}}{Z_p} \;\; \text{con} \; R_{p0} = \frac{3}{4} \frac{V_{DD}}{V_{DSATP0}}\]
Si nota che la resistenza equivalente \(R_{eq}\) è inversamente proporzionale al fattore di forma \(Z\) del MOSFET, per cui
aumentando il fattore di forma \(Z\) del MOSFET, si riduce la resistenza equivalente \(R_{eq}\) e quindi si riduce anche
la costante di tempo \(\tau = R_{eq} C\) del circuito, migliorando le prestazioni del circuito.

Analizzando i valori tipici delle tecnologie CMOS si osserva che:
\[k_n' \approx 3k_p', \quad V_{TN} \approx -V_{TP}, \quad V_{DSATN} \approx -\frac{2}{3} V_{DSATP} \;\; \Rightarrow \;\; R_{n0} \approx 2R_{p0}\]
\[\frac{R_{n0}}{R_{p0}} = \frac{k_n' V_{DSATN} (V_{DD} - V_{TN} - \frac{V_{DSATN}}{2})}{k_p' V_{DSATP} (-V_{DD} - V_{TP} - \frac{V_{DSATP}}{2})} \approx \frac{k_n' V_{DSATN}}{-k_p' V_{DSATP}} \approx 3 \cdot \frac{2}{3} = 3\]

\subsubsection*{Resistenze equivalenti per valori logici \say{cattivi}}
Si osserva che per approssimare al meglio il transitorio dei mosfet nella trasmissione di valori logici \say{cattivi}, è necessario
raddoppiare le resistenze equivalenti calcolate in precedenza:
\[\text{NMOS (carica):} \quad R_{n, cattivo} = 2 R_{n,buono} \qquad\qquad \text{PMOS (scarica):} \quad R_{p, cattivo} = 2 R_{p,buono}\]

\begin{center}
	\centering \includegraphics[width=0.7\textwidth]{immagini/5_circuiti_mosfet/req_3.png}
\end{center}

\subsection{Reti di MOSFET e resistenza equivalente complessiva}
\subsubsection*{Serie di MOS}
La serie di MOSFET equivale ad un MOSFET equivalente con resistenza equivalente pari alla sommma delle resistenze equivalenti
dei singoli MOSFET e con fattore di forma pari al reciproco della somma dei recicproci dei fattori di forma dei singoli MOSFET:
\begin{center}
	\begin{minipage}{0.4\textwidth}
		\centering \includegraphics[width=0.7\textwidth]{immagini/5_circuiti_mosfet/serie_mos.png}
	\end{minipage}
	\begin{minipage}{0.55\textwidth}
		\begin{align*}
			R_{eq} &= R_1 + R_2 \\[5pt]
			\frac{1}{Z_{eq}} &= \frac{1}{Z_1} + \frac{1}{Z_2} \;\; \rightarrow \;\; Z_{eq} = \frac{Z_1 Z_2}{Z_1 + Z_2}
		\end{align*}
	\end{minipage}
\end{center}

\subsubsection*{Parallelo di MOS}
La parallelo di MOSFET equivale ad un MOSFET equivalente con resistenza equivalente pari al recicproco della somma dei recicproci
delle resistenze equivalenti dei singoli MOSFET e con fattore di forma pari alla somma dei fattori di forma dei singoli MOSFET:
\begin{center}
	\begin{minipage}{0.4\textwidth}
		\centering \includegraphics[width=0.7\textwidth]{immagini/5_circuiti_mosfet/parallelo_mos.png}
	\end{minipage}
	\begin{minipage}{0.55\textwidth}
		\begin{align*}
			\frac{1}{R_{eq}} &= \frac{1}{R_1} + \frac{1}{R_2} \;\; \rightarrow \;\; R_{eq} = \frac{R_1 R_2}{R_1 + R_2} \\[5pt]
			Z_{eq} &= Z_1 + Z_2
		\end{align*}
	\end{minipage}
\end{center}
