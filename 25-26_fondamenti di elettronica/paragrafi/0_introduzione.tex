\section{Introduzione}
\subsection{Settori dell'elettronica}
\begin{itemize}
	\item \textbf{elettronica analogica}: progettazione e analisi di circuiti che elaborano segnali analogici;
	\item \textbf{elettronica digitale}: progettazione e analisi di circuiti che elaborano segnali digitali;
	\item \textbf{elettronica di consumo}: dispositivi elettronici per l'uso personale e domestico (computer, telefoni cellulari, televisori, elettrodomestici);
	\item \textbf{microelettronica}: progettazione e fabbricazione di componenti elettronici e circuiti integrati;
	\item \textbf{elettronica di potenza}: conversione e gestione dell'energia elettrica a diversi livelli (dal riscaldamento agli alimentatori per pc, cellulari o altri strumenti);
	\item \textbf{elettronica industriale}: sistemi elettronici per processi produttivi automatizzati;
	\item \textbf{telecomunicazioni}: sistemi per la trasmissione di dati (voce, video, file) attraverso dispositivi mobili o fissi;
	\item \textbf{biomedica}: sviluppo di apparecchiature elettroniche per la diagnostica, la cura e il monitoraggio della salute;
	\item \textbf{automotive}: sistemi per il controllo dei veicoli (dallo specchietto fino alla guida autonoma);
	\item \textbf{informatica}: dispositivi e sistemi elettronici per la gestione dei dati.
\end{itemize}

\subsection{Definizioni fondamentali}
\begin{itemize}
	\item \textbf{elettronica}: studia e realizza sistemi elettronici;
	\item \textbf{sistema elettronico}: è un insieme di componenti elettronici (sensori, circuiti e attuatori) che raccolgono
	informazioni dal mondo reale attraverso sensori, le elaborano attraverso circuiti elettronici e prendono decisioni o
	comandano azioni con degli attuatori;
	\item \textbf{segnale}: supporto fisico di natura qualunque (elettrica, acustica, ottica) a cui si associa un'informazione
	allo scopo di poterla trasferire da una sorgente ad un utilizzatore, può essere digitale (ampiezza e tempo discreti) o
	analogico (ampiezza e tempo continui);
	\item \textbf{sensore}: dispositivo che converte un segnale esterno in una grandezza elettrica (corrente o tensione);
	\item \textbf{circuito elettronico}: rete di componenti elettrici passivi (R, L, C) e attivi (diodi, transistor) per
	l'elaborazione di segnali elettrici (tensione e corrente). In base al tipo di segnale elaborato si distingue in:
	\begin{itemize}[topsep=0pt]
		\item \textbf{circuito analogico}: elabora segnali analogici;
		\item \textbf{circuito digitale}: elabora segnali digitali;
		\item \textbf{circuito misto}: opera in entrambi i domini del segnale.
	\end{itemize}
	Siccome i segnali provenienti dal mondo reale sono sempre analogici, in generale non esiste un sistema completamente digitale,
	per cui ogni sistema digitale prevede un ADC (Analog-to-Digital Converter) in ingresso e un DAC (Digital-to-Analog Converter)
	in uscita.

	In base alla realizzazione fisica si distingue in:
	\begin{itemize}
		\item \textbf{circuito a elementi discreti}: realizzato con componenti singoli collegati tra loro (breadboard, circuiti
		stampati);
		\item \textbf{circuito integrato (IC)}: realizzato con componenti miniaturizzati su un unico chip di silicio (circuiti
		integrati, microchip).
	\end{itemize}
	Un sistema elettronico completo è formato da circuiti integrati e componenti discreti montati in una scheda in cui sono
	realizzate le interconnessioni metalliche tra i terminali dei componenti
\end{itemize}


\section{Richiamo di teoria dei circuiti}

