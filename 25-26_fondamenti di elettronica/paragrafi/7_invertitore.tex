\section{Invertitore CMOS}
\subsection{Schema circuitale}

\subsubsection*{Schema circuitale e simbolo logico}
L'invertitore CMOS (detto anche Complementary-MOS) è il circuito logico più semplice realizzabile con la tecnologia cmos.
Implementa la funzione logica NOT, ovvero l'operazione di negazione booleana. Lo schema circuitale dell'invertitore cmos
è riportato in figura (a sinistra), insieme al simbolo logico corrispondente (a destra).

\begin{center}
	\begin{minipage}{0.4\textwidth}
		\centering
		\begin{circuitikz}
			\ctikzset{tripoles/pmos style/emptycircle}
		
			\draw (0,-0.7) -- (0,0.7);
			\draw (-0.7,0) node[left] {\(V_{IN}\)} to[short, *-] (0,0);
		
			\draw (0,-0.7) node[nmos, anchor=G, scale=1.2] (nmos) {};
			\draw (0,0.7) node[pmos, anchor=G, scale=1.2] (pmos) {};
			\draw (nmos.S) -- ++(0,0.3) node[sground] {};
			\draw (pmos.S) -- ++(0,-0.3) node[rground, rotate=180, name=VDD]{};
		
			\draw (pmos.D) ++ (0,0.23) to[short, -*] ++(0.7,0) node[right] {\(V_{OUT}\)};
			\draw (pmos.D) ++ (-0.05,0.45) node[right] {\(_\text{D}\)};
			\draw (pmos.S) ++ (-0.05,-0.45) node[right] {\(_\text{S}\)};
			\draw (nmos.D) ++ (-0.05,-0.45) node[right] {\(_\text{D}\)};
			\draw (nmos.S) ++ (-0.05,0.45) node[right] {\(_\text{S}\)};
			\draw (nmos.centergap) ++ (1.2,0) node[left] {\(_\text{NMOS}\)};
			\draw (pmos.centergap) ++ (1.2,0) node[left] {\(_\text{PMOS}\)};
		
			\node at ($(VDD)+(0, 0.6)$) {\(V_{DD}\)};
		\end{circuitikz}
	\end{minipage}
	\begin{minipage}{0.4\textwidth}
		\centering
		\begin{circuitikz}
			\ctikzset{logic ports=ieee}
			\draw (0,0) node[not port, scale=0.8] (NOT) {} (NOT.in) node[left] {IN} (NOT.out) node[right] {OUT};
		\end{circuitikz}
	\end{minipage}	
\end{center}

\noindent
L'invertitore cmos è costituito da un transistor nmos e un transistor pmos collegati in serie tra \(V_{DD}\) e massa. L'ingresso
del circuito è collegato ai gate di entrambi i transistor, mentre l'uscita è prelevata dal nodo di connessione tra i due transistor.

\subsubsection*{Funzionamento}
Per comprendere meglio il funzionamento dell'invertitore cmos si sostituiscono i transistor (usati come interruttori) con la
serie di interruttore e resistenza equivalente, come mostrato in figura. Si distinguono due casi principali in base al valore
logico dell'ingresso \(V_{IN}\):

\begin{center}
	\begin{minipage}{0.3\textwidth}
		\centering
		\begin{circuitikz}[scale=0.75, transform shape]
			\draw (0.7,0) node[right] {\(V_{OUT}\)} to[short, *-] (0,0);
			\draw (0,-0.4) to[short, *-*] (0,0.4);
			\draw (-0.3,-0.4) -- (0,-0.9); 
			\draw (0,-0.9) to[R, l_=\(R_{n,eq}\), *-] (0,-2.5) ++(0,0.2) node[sground] {};
			\draw (0,0.4) -- (-0.05,0.9);
			\draw (0,0.9) to[R, l=\(R_{p,eq}\), *-] (0,2.5) ++(0,-0.2) node[rground, rotate=180, name=VDD]{};
			\node at ($(VDD)+(0, 0.6)$) {\(V_{DD}\)};
		\end{circuitikz}
	\end{minipage}
	\begin{minipage} {0.6\textwidth}
		Quando \(V_{IN} = 0 \V\), il pmos è acceso, mentre l'nmos è spento. L'uscita \(V_{OUT}\) è, quindi, collegata a
		\(V_{DD}\) tramite la resistenza equivalente \(R_p\) del pmos e si ha \(V_{OUT} = V_{DD}\).

		Il pmos porta l'uscita al livello logico alto (valore logico \say{buono} per pmos) ed è chiamato transistor di pull-up.
	\end{minipage}
\end{center}

\begin{center}
	\begin{minipage}{0.3\textwidth}
		\centering
		\begin{circuitikz}[scale=0.75, transform shape]
			\draw (0.7,0) node[right] {\(V_{OUT}\)} to[short, *-] (0,0);
			\draw (0,-0.4) to[short, *-*] (0,0.4);
			\draw (-0.05,-0.4) -- (0,-0.9); 
			\draw (0,-0.9) to[R, l_=\(R_{n,eq}\), *-] (0,-2.5) ++(0,0.2) node[sground] {};
			\draw (0,0.4) -- (-0.3,0.9);
			\draw (0,0.9) to[R, l=\(R_{p,eq}\), *-] (0,2.5) ++(0,-0.2) node[rground, rotate=180, name=VDD]{};
			\node at ($(VDD)+(0, 0.6)$) {\(V_{DD}\)};
		\end{circuitikz}
	\end{minipage}
	\begin{minipage} {0.6\textwidth}
		Quando \(V_{IN} = V_{DD}\), il pmos è spento, mentre l'nmos è acceso. L'uscita \(V_{OUT}\) è, quindi, collegata a
		massa tramite la resistenza equivalente \(R_n\) dell'nmos e si ha \(V_{OUT} = 0\).

		L'nmos porta l'uscita al livello logico basso (valore logico \say{buono} per nmos) ed è chiamato transistor di pull-down.
	\end{minipage}
\end{center}

\noindent
In entrambi i casi l'uscita \(V_{OUT}\) assume il valore logico opposto rispetto all'ingresso \(V_{IN}\), come previsto
dalla funzione logica NOT.

\newpage

\subsection{Caratteristica di trasferimento ingresso-uscita}
La caratteristica statica di trasferimento, o caratteristica di trasferimento ingresso-uscita (VTC, voltage transfer
characteristic) è la curva che descrive il valore di \(V_{OUT}\) in funzione di \(V_{IN}\).

\subsubsection*{Calcolo grafico della VTC}
Per ottenerla, si uguagliano
le correnti attraverso i due transistor secondo la legge di Kirchhoff delle correnti (LKC).
\[I_{DS,n}(V_{IN},V_{OUT}) = I_{DS,p}(V_{IN},V_{OUT})\]

Per evitare di fare conti, si risolve l'equazione graficamente, ovvero si tracciano le curve delle due correnti in funzione
di \(V_{OUT}\) per diversi valori di \(V_{IN}\). I punti di intersezione delle due curve rappresentano le soluzioni
dell'equazione, ovvero i valori di \(V_{OUT}\) corrispondenti a ciascun valore di \(V_{IN}\).

\begin{center}
	\begin{minipage}{0.4\textwidth}
		\centering \includegraphics[width=0.9\textwidth]{immagini/7_invertitore/vtc_1.png}
	\end{minipage}
	\begin{minipage}{0.1\textwidth}
		\centering \(\longrightarrow\)
	\end{minipage}
	\begin{minipage}{0.4\textwidth}
		\centering \includegraphics[width=0.9\textwidth]{immagini/7_invertitore/vtc_2.png}
	\end{minipage}
\end{center}

\subsubsection*{Regioni di funzionamento dei transistor}
Si osserva che per \(V_{IN} = 0 \V\) o \(V_{IN} = V_{DD}\) un solo transistor è acceso in regime lineare, mentre l'altro è spento.
Per valori intermedi si ha che:
\begin{itemize}
	\item il pmos passa da lineare a saturazione e infine si spegne al crescere di \(V_{IN}\)
	\item l'nmos da interdizione, passa da saturazione e infine lineare al crescere di \(V_{IN}\)
\end{itemize}

\subsubsection*{Punti importanti}
La pendenza della VTC (ovvero la derivata di \(V_{OUT}\) in funzione di \(V_{IN}\)) corrisponde al guadagno in tensione
dell'invertitore. Si definiscono i seguenti punti importanti della VTC:

\begin{center}
	\begin{minipage}{0.4\textwidth}
		\centering \includegraphics[width=0.9\textwidth]{immagini/7_invertitore/vtc_3.png}
	\end{minipage}
	\begin{minipage}{0.5\textwidth}
		P0 e P1 corrispondo ai punti con guadagno unitario. Tali punti delimitano la regione di livello logico indefinito,
		con guadagno \(>\) 1 e le regioni di livello logico definito (a sinistra e a destra), con guadagno \(< 1\).
		\vspace{5pt}

		Si osserva che in corrispondenza delle regioni di livello logico definito la pendenza della curva è molto bassa
		(guadagno \(<\) 1), per cui piccole variazioni di \(V_{IN}\) producono variazioni trascurabili di \(V_{OUT}\)
		(proprietà di rigenerazione del segnale).
		\vspace{5pt}

		Il pallino rosso in cui \(V_{OUT} = V_{IN}\) è il punto di soglia di commutazione logica dell'invertitore. La
		tensione di ingresso associata a questo punto è detta soglia di commutazione logica \(V_{M}\).
	\end{minipage}
\end{center}

\subsection{Soglia di commutazione logica}
\subsubsection*{Calcolo della tensione di commutazione logica}
Per calcolare la soglia di commutazione logica \(V_{M}\) si impone la condizione \(V_{OUT} = V_{IN} = V_{M}\), ovvero si
cortocircuitano l'ingresso e l'uscita dell'invertitore. Si ottiene, quindi, la seguente equazione che dipende solo da \(V_{M}\):
\[I_{DS,n}(V_{M}) = I_{DS,p}(V_{M})\]
Si ottiene \(V_{M}\) risolvendo l'equazione:
\[V_M = \frac{\displaystyle V_{TN} + \frac{V_{DSATN}}{2} + r \left(V_{DD} + V_{TP} + \frac{V_{DSATP}}{2}\right)}{1+r} \qquad \text{con} \; r = - \frac{Z_p}{Z_n} \frac{k_p' V_{DSATP}}{k_n' V_{DSATN}}\]

\subsubsection*{Relazione tra tensione di commutazione logica e rapporto dei fattori di forma}
Si osserva che \(V_{M}\) oltre a dipendere dai parametri fisici dei transistor e dal valore di \(V_{DD}\), dipende anche dal
rapporto tra i fattori di forma dei due transistor \(Z_p / Z_n\). Modificando tale rapporto è possibile regolare il valore di
\(V_{M}\).
In particolare per avere \(V_M = V_{DD}/2\) si ottiene:
\[V_M = \frac{V_{DD}}{2} \quad \rightarrow \quad r = \frac{\frac{V_{DD}}{2} - V_{TN} - \frac{V_{DSATN}}{2}}{\frac{V_{DD}}{2} + V_{TP} + \frac{V_{DSATP}}{2}} \quad \rightarrow \quad \frac{Z_p}{Z_n} = - r \frac{k_n' V_{DSATN}}{k_p' V_{DSATP}} \approx 3 - 3.5\]
Analizzando graficamente come varia \(V_{M}\) al variare del rapporto \(Z_p / Z_n\), si osserva che:
\begin{itemize}
	\item per \(Z_p > 3 Z_n\) aumenta \(V_{M}\), prevale il pmos e la VTC si sposta verso destra (il valore logico alto si mantiene più a lungo)
	\item per \(Z_p < 3 Z_n\) diminuisce \(V_{M}\), prevale l'nmos e la VTC si sposta verso sinistra (il valore logico basso si mantiene più a lungo)
	\item per \(Z_p = 3 Z_n\) si ha \(V_{M} \approx V_{DD}/2\) e i due transistor si bilanciano e la VTC è simmetrica
\end{itemize}
Infine si nota che la curva è molto piatta in corrispondenza della soglia di commutazione logica, il che implica che \(V_M\)
è poco sensibile alle variazioni dei fattori di forma dei transistor.

\begin{center}
	\begin{minipage}{0.4\textwidth}
		\centering \includegraphics[width=0.9\textwidth]{immagini/7_invertitore/soglia_1.png}

		\small{\(V_M\) al variare di \(Z_p / Z_n\), in scala logaritmica}
	\end{minipage}
	\begin{minipage}{0.55\textwidth}
		\centering \includegraphics[width=0.9\textwidth]{immagini/7_invertitore/soglia_2.png}

		\small{VTC al variare di \(Z_p / Z_n\)}
	\end{minipage}
\end{center}

\subsection{Tolleranza al rumore}
\subsubsection*{Tipi di disturbi in un circuito elettrico}
In un circuito elettrico possono essere presenti diversi tipi di disturbi che possono alterare il corretto funzionamento del
circuito stesso. I principali tipi di disturbi sono:
\begin{itemize}
	\item \textbf{accoppiamenti induttivi}: disturbi causati dalla mutua induzione di due conduttori vicini dovuta all'induttanza
	parassita delle piste di collegamento
	\item \textbf{accoppiamenti capacitivi}: disturbi causati dalla capacità parassita tra due conduttori vicini
	\item \textbf{rumore dell'alimentazione}: variazioni indesiderate della tensione di alimentazione del circuito dovuta ad
	esempio dalla caduta di tensione per le resistenze lungo le linee di alimentazione
\end{itemize}

\subsubsection*{Margine di immunità al rumore}
Il margine di immunità al rumore o NM (noise margin) rappresenta l'ampiezza massima del disturbo che è possibile avere durante
la trasmissione del segnale tra due porte logiche, senza che si verifichino errori di interpretazione del segnale logico. In base
al livello logico si distinguono due margini di immunità:
\begin{itemize}[topsep=0pt]
	\item NM per il livello logico alto: \(N\!M_H = V_{O\!H} - V_{I\!H}\) ovvero la
	differenza tra la minima tensione di uscita di un segnale a livello logico alto e la minima tensione di ingresso per
	riconoscere un segnale come livello logico alto;
	\item NM per il livello logico basso: \(N\!M_L = V_{I\!L} - V_{O\!L}\) ovvero la
	differenza tra la massima tensione di ingresso per riconoscere un segnale come livello logico basso e la massima tensione
	di uscita di un segnale a livello logico basso.
\end{itemize}
Il margine di immunità complessivo, ovvero il massimo disturbo tollerabile, corrisponde al margine di immunità più piccolo tra
i due livelli logici \(N\!M = \min(N\!M_H, N\!M_L)\).

\subsubsection*{Rigenerazione del segnale}
La rigenerazione del segnale è la capacità di un circuito logico di ripristinare i livelli logici di uscita a valori
vicini ai valori ideali \(V_{O\!H}\) e \(V_{O\!L}\), attenuando gli effetti dei disturbi e del rumore sul segnale.

Affinché una porta logica sia in grado di rigenerare il segnale, è necessario che il guadagno della VTC sia minore di 1 nelle
regioni di livello logico definito (VTC tendente all'orizzontale) e maggiore di 1 nella regione di livello logico
indefinito (VTC tendente al verticale).

\begin{center}
	\begin{minipage}{0.34\textwidth}
		\centering \includegraphics[width=\textwidth]{immagini/7_invertitore/noise_1.png}
	\end{minipage}
	\begin{minipage}{0.64\textwidth}
		\centering \includegraphics[width=\textwidth]{immagini/7_invertitore/noise_2.png}
	\end{minipage}
\end{center}

\subsubsection*{Ruolo della tensione di commutazione logica nella rigenerazione del segnale}
I margini di immunità al rumore per un invertitore cmos possono essere approssimati come:
\[N\!M_H \approx V_{DD} - V_{M} \qquad N\!M_L \approx V_{M} \qquad N\!M \approx \min(N\!M_H, N\!M_L)\]
Si osserva che per massimizzare il margine di immunità complessivo \(N\!M\) è necessario bilanciare i due margini di immunità, ovvero
impostare la soglia di commutazione logica \(V_{M}\) a metà della tensione di alimentazione \(V_{DD}/2\). Si ottiene così che
il dimensionamento ottimo dei transistor per massimizzare la tolleranza al rumore è \(Z_p \approx 3 Z_n\).

\subsection{Invertitore ideale}
Un invertitore ideale è un invertitore logico che interpreta \(V_{IN} < V_M\) come livello logico basso e \(V_{IN} > V_M\)
come livello logico alto, inoltre rigenera l'uscita esattamente ai valori ideali \(V_{O\!H} = V_{DD}\) e \(V_{O\!L} = 0 \V\).

\begin{center}
	\begin{minipage}{0.34\textwidth}
		\centering \includegraphics[width=\textwidth]{immagini/7_invertitore/invertitore_ideale.png}
	\end{minipage}
	\begin{minipage}{0.64\textwidth}
		La curva VTL di un invertitore ideale è una curva a gradino che passa per il punto di soglia di commutazione logica
		e la funzione di trasferimeno è una funzione definita a tratti:
		\[V_{OUT} = \begin{cases}
			V_{DD} \;\; \text{se} \; V_{IN} < V_{M} \\
			V_{M} \;\; \text{se} \; V_{IN} = V_{M} \\
			0 \V \;\; \text{se} \; V_{IN} > V_{M}
		\end{cases}\]

		I margini di immunità al rumore diventanto esattamente
		\[N\!M_H = V_{DD} - V_{M} \qquad N\!M_L = V_{M}\]
	\end{minipage}
\end{center}

\subsection{Tempo di ritardo}
\subsubsection*{Modello RC dell'invertitore}
Ad ogni terminale dei transistor sono associate delle capacità parassite. Queste capacità vanno caricate e scaricate durante la
commutazione del segnale, causando un ritardo temporale tra l'ingresso e l'uscita dell'invertitore. Si rappresenta, quindi,
l'invertitore con un modello RC equivalente in cui le capacità parassite sono evidenziate in rosso ed è stata aggiunta la
capacità di carico \(C_L\) collegata all'uscita del circuito:

\begin{center}
	\begin{minipage}{0.25\textwidth}
		\centering
		\begin{circuitikz}
			\ctikzset{tripoles/pmos style/emptycircle}
		
			\draw (0,-0.7) -- (0,0.7);
			\draw (-0.3,0) node[left] {\(V_{IN}\)} to[short, *-] (0,0);
		
			\draw (0,-0.7) node[nmos, anchor=G, scale=1.2] (nmos) {};
			\draw (0,0.7) node[pmos, anchor=G, scale=1.2] (pmos) {};
			\draw (nmos.S) -- ++(0,0.3) node[sground] {};
			\draw (pmos.S) -- ++(0,-0.3) node[rground, rotate=180, name=VDD]{};
		
			\draw (pmos.D) ++ (0,0.23) to[short, -*] ++(0.5,0) node[right] {\(V_{OUT}\)};
			\draw (pmos.D) ++ (-0.05,0.45) node[right] {\(_\text{D}\)};
			\draw (pmos.S) ++ (-0.05,-0.45) node[right] {\(_\text{S}\)};
			\draw (nmos.D) ++ (-0.05,-0.45) node[right] {\(_\text{D}\)};
			\draw (nmos.S) ++ (-0.05,0.45) node[right] {\(_\text{S}\)};
			\draw (nmos.centergap) ++ (1.2,0) node[left] {\(_\text{NMOS}\)};
			\draw (pmos.centergap) ++ (1.2,0) node[left] {\(_\text{PMOS}\)};
		
			\node at ($(VDD)+(0, 0.6)$) {\(V_{DD}\)};
		\end{circuitikz}
	\end{minipage}
	\begin{minipage}{0.05\textwidth}
		\centering \(\longrightarrow\)
	\end{minipage}
	\begin{minipage}{0.38\textwidth}
		\centering
		\begin{circuitikz}
			\ctikzset{bipoles/length=0.8cm}

			\draw (-2,0) node[left] {\(V_{IN}\)} to[short, *-] (-1.3,0);
			\draw [<->] (-0.3,-0.6) -- (-1.3,-0.6) -- (-1.3,0.6) -- (-0.3,0.6);

			\draw (2,0) node[right] {\(V_{OUT}\)} to[short, *-] (0,0);
			\draw (0,-0.4) to[short, *-*] (0,0.4);
			\draw (-0.3,-0.4) -- (0,-0.9); 
			\draw (0,-0.9) to[R, l_=\(R_{n}\), *-] (0,-2.5) node[sground] {};
			\draw (0,0.4) -- (-0.3,0.9);
			\draw (0,0.9) to[R, l=\(R_{p}\), *-] (0,2.5) node[rground, rotate=180, name=VDD]{};
			\node at ($(VDD)+(0, 0.5)$) {\(V_{DD}\)};

			\color{red}
			\draw (-0.9,-0.6) to[C, l_=\(\;_{C_{gn}}\)] ++(0,-0.7) ++(0,0.2) node[sground, color=black]{};
			\draw (-0.9,0.6) to[C, l=\(\;_{C_{gp}}\)] ++(0,0.7) ++(0,-0.2) node[rground, rotate=180, name=VDD, color=black]{};
			\node at ($(VDD)+(0, 0.4)$) [color=black] {\(\;_{V_{DD}}\)};

			\draw (0.7,0) to[C, l=\(\;_{C_{dn}}\)] ++(0,-0.7) ++(0,0.2) node[sground, color=black]{};
			\draw (0.7,0) to[C, l_=\(\;_{C_{dp}}\)] ++(0,0.7) ++(0,-0.2) node[rground, rotate=180, name=VDD, color=black]{};
			\node at ($(VDD)+(0, 0.4)$)[color=black] {\(\;_{V_{DD}}\)};

			\draw (0,-2.5) -- ++(0.7,0) to[C, l_=\(\;_{C_{sn}}\)] ++(0,0.7) ++(0,-0.2) node[sground, rotate=180, color=black]{};
			\draw (0,2.5) -- ++(0.7,0) to[C, l=\(\;_{C_{sp}}\)] ++(0,-0.7) ++(0,0.2) node[rground, name=VDD, color=black]{};
			\node at ($(VDD)+(0, -0.4)$) [color=black] {\(\;_{V_{DD}}\)};

			\draw (2,0) -- ++(0,-0.2) to[C, l=\(C_{L}\)] ++(0,-0.7) -- ++(0,-0.2) ++(0,0.2) node[sground, color=black]{};
		\end{circuitikz}
	\end{minipage}
	\begin{minipage}{0.05\textwidth}
		\centering \(\longrightarrow\)
	\end{minipage}
	\begin{minipage}{0.24\textwidth}
		\centering
		\begin{circuitikz}
			\ctikzset{bipoles/length=0.8cm}

			\draw (-1,0) node[left] {\(V_{IN}\)} to[short, *-] (-0.7,0);
			\draw [<->] (-0.3,-0.6) -- (-0.7,-0.6) -- (-0.7,0.6) -- (-0.3,0.6);

			\draw (0.7,0) node[right] {\(V_{OUT}\)} to[short, *-] (0,0);
			\draw (0,-0.4) to[short, *-*] (0,0.4);
			\draw (-0.3,-0.4) -- (0,-0.9); 
			\draw (0,-0.9) to[R, l_=\(R_{n}\), *-] (0,-2.5) ++(0,0.2) node[sground] {};
			\draw (0,0.4) -- (-0.3,0.9);
			\draw (0,0.9) to[R, l=\(R_{p}\), *-] (0,2.5) ++(0,-0.2) node[rground, rotate=180, name=VDD]{};
			\node at ($(VDD)+(0, 0.5)$) {\(V_{DD}\)};

			\color{red}
			\draw (0.7,0) -- ++(0,-0.2) to[C, l=\(C_{OUT}\)] ++(0,-0.7) -- ++(0,-0.2) ++(0,0.2) node[sground, color=black]{};
		\end{circuitikz}
	\end{minipage}	
\end{center}

\noindent
Analizzando le resistenze e le capacità del modello RC equivalente per i nodi di ingresso ed uscita si ha:
\[R_{IN} = \infty \qquad C_{IN} = C_{gn} + C_{dp} \qquad R_{OUT} =\frac{R_{n} + R_{p}}{2} \qquad C_{OUT} = C_{gp} + C_{dn} + C_{L}\]
NOTA: le capacità dei source sono cortocircuitate per cui sono ininfluenti, inoltre le capacità del pmos possono essere considerate
collegate tutte a massa invece che a \(V_{DD}\), come visto alla fine del paragrafo delle \hyperref[capacità_parassite_mosfet]{capacità parassite dei mosfet}.

\subsubsection*{Calcolo dei tempi di ritardo intrinseci}
Il tempo di ritardo è il tempo che impiega la porta di uscita a raggiungere il 50\% del valore finale dopo un cambio di stato
dell'ingresso. Il tempo di ritardo si dice intrinseco se il carico esterno è nullo (\(C_L = 0\)).
Si definiscono due tempi di ritardo intrinseci:
\begin{itemize}
	\item tempo di ritardo di salita \(t_{pLH0}\): tempo che impiega l'uscita a salire dal 50\% del livello logico basso al
	50\% del livello logico alto dopo un fronte di salita dell'ingresso
	\[V_{OUT} = \frac{V_{DD}}{2} \;\; \rightarrow \;\; V_{DD} - V_{DD} e ^{-\tfrac{t_{pLH0}}{R_{p}C_{OUT}}} = \frac{V_{DD}}{2} \;\; \rightarrow \;\; t_{pLH0} = \ln(2) \, R_p \, C_{OUT} \approx 0.69 \, R_p \, C_{OUT}\]
	\item tempo di ritardo di discesa \(t_{PHL0}\): tempo che impiega l'uscita a scendere dal 50\% del livello logico alto al
	50\% del livello logico basso dopo un fronte di discesa dell'ingresso
	\[V_{OUT} = \frac{V_{DD}}{2} \;\; \rightarrow \;\; V_{DD} e ^{-\tfrac{t_{pHL0}}{R_{n}C_{OUT}}} = \frac{V_{DD}}{2} \;\; \rightarrow \;\; t_{pHL0} = \ln(2) \, R_n \,  C_{OUT} \approx 0.69 \; R_n \, C_{OUT}\]
\end{itemize}

Si calcola il tempo di propagazione intrinseco medio come:
\[t_{p0} = \frac{t_{pLH0} + t_{pHL0}}{2} = \ln(2) \, \frac{R_n + R_p}{2} \, C_{OUT} \approx 0.69 \,\frac{R_n + R_p}{2} \, C_{OUT} = 0.69 \,R_{OUT} \, C_{OUT}\]

\newpage

\subsubsection*{Carico esterno e fan-out}
Quando l'invertitore guida un carico esterno \(C_L\), i tempi di ritardo aumentano in quanto la capacità totale da caricare
o scaricare è maggiore. Si definiscono il fan-out \(f\) e il coefficiente di carico \(\gamma\):
\[f = \frac{C_L}{C_{IN}} \qquad \gamma = \frac{C_{OUT}}{C_{IN}}\]
Si ottengono, quindi, i tempi di ritardo con carico esterno:
\[t_{pLH} = 0.69 \, R_p \, (C_{OUT} + C_L) = 0.69 \, R_p \, C_{OUT} + 0.69 \, R_p C_L = t_{pLH0} + 0.69 \, R_p C_L\]
\[t_{pHL} = 0.69 \, R_n \, (C_{OUT} + C_L) = 0.69 \, R_n \, C_{OUT} + 0.69 \, R_n C_L = t_{pHL0} + 0.69 \, R_n C_L\]
\[t_p = \frac{t_{pLH} + t_{pHL}}{2} = 0.69 \, R_{OUT} \, (C_{OUT} + C_L) = t_{p0} + 0.69 \, R_{OUT} C_L = t_{p0} \left(1 + \frac{f}{\gamma}\right)\]

\subsubsection*{Ottimizzazione del tempo di ritardo intrinseco}
Si osserva che il tempo di ritardo dipende da una serie di parametri tecnologici (specifici dei materiali) e da altri parametri
di progettazione che possono essere scelti liberamente. In particolare si può agire su:
\begin{itemize}
	\item il rapporto tra i fattori di forma dei transistor \(Z_p / Z_n\)
	\item il fattore di carico \(f\) che dipende dall'architettura del circuito logico
	\item la tensione di alimentazione \(V_{DD}\), difficile da modificare in quanto spesso imposta da vincoli esterni
\end{itemize}
Il tempo di ritardo può essere ottimizzato secondo i seguenti criteri:
\begin{itemize}
	\item ottimizzare il tempo massimo di ritardo \(\max (t_{pHL}, t_{pLH})\)
	\item ottimizzare il tempo di ritardo medio \(t_p\)
\end{itemize}
Si definiscono quindi i coefficienti \(\beta\) e \(\rho\) e si procede all'ottimizzazione scegliendo il valore ottimale di
\(\beta\) in base al criterio scelto:
\[\beta = \frac{Z_p}{Z_n} = \frac{W_p}{W_n} \qquad \rho = \frac{R_{p0}}{R_{n0}}\]
\[t_{pHL0} = 0.69 R_n C_{OUT} = 0.69 \frac{R_{n0}}{Z_n} C_{d0} W_n (1+\beta) = 0.69 R_{n0}C_{d0}L \left(1 + \beta\right)\]
\[t_{pLH0} = 0.69 R_p C_{OUT} = 0.69 \frac{\rho R_{n0}}{\beta Z_n} C_{d0} W_n (1+\beta) = 0.69 R_{n0}C_{d0}L \left(\frac{\rho}{\beta} + \rho\right)\]
\[t_{p0} = 0.69 \, \frac{R_{n0}C_{d0}L}{2}\left(1 + \beta + \rho + \frac{\rho}{\beta}\right)\]

\begin{center}
	\begin{minipage}{0.45\textwidth}
		\centering \includegraphics[width=0.9\textwidth]{immagini/7_invertitore/ritardo_1.png}
	\end{minipage}
	\begin{minipage}{0.5\textwidth}
		Ottimizzando il tempo massimo di ritardo (ovvero imponendo \(t_{pHL0} = t_{pLH0}\), punto blu in figura):
		\begin{align*}
			t_{pHL0} = t_{pLH0} \;\; &\rightarrow \;\; 1 + \beta = \frac{\rho}{\beta} + \rho\\
			&\rightarrow \;\; \beta = \rho \approx 2 - 2.5
		\end{align*}

		Ottimizzando il tempo medio di ritardo (ovvero trovando il minimo di \(t_{p0}\), punto rosso in figura):
		\begin{align*}
			\frac{d t_{p0}}{d \beta} = 0 \;\; &\rightarrow \;\; 1 - \frac{\rho}{\beta^2} = 0 \\
			&\rightarrow \;\; \beta = \sqrt{\rho} \approx 1.4 - 1.6
		\end{align*}
	\end{minipage}
\end{center}

\noindent
Si nota che il valore di \(\beta\) ottimizza anche l'affidabilità del circuito in quanto agisce sulla soglia di commutazione
logica \(V_{M}\) e quindi sui margini di immunità al rumore. Per avere affidabilità massima (ovvero \(V_M = V_{DD}/2\)) bisogna
scegliere \(\beta \approx 3\), ma siccome \(V_M\) reagisce molto poco alle variazioni di \(\beta\), si predilige scegliere
\(\beta\) in modo da ottimizzare il tempo di ritardo.

\subsubsection*{Dimensionamento dell'invertitore con carico esterno}
Fissato un certo \(\beta\) si calcola come ottimizzare il tempo di ritardo in funzione della capacità di carico esterna \(C_L\)
andando a dimensionare opportunamente i transistor. Si ottiene:
\[t_p = 0.69 R_{OUT} C_{OUT} \left(1 + \frac{C_L}{C_{OUT}}\right) = t_{p0} \left(1 + \frac{C_L}{C_{OUT}}\right) \quad \text{con } t_{p0} \text{ indipendente da } C_L\]
\[1 + \frac{C_L}{C_{OUT}} = 1+ \frac{C_L}{C_{d0} W_n (1 + \beta)} \approx 1 + C_L \frac{\text{costante}}{W_n} \qquad R_{OUT} = \frac{1}{2} \frac{R_{n0}}{Z_n} \left(1 + \frac{\rho}{\beta}\right) \approx \frac{\text{costante}}{Z_n}\]

\noindent
Si osserva che per minimizzare il tempo di ritardo si può aumentare la larghezza \(W_n\) (e proporzionalmente
anche \(W_p = \beta W_n\) e \(Z_n = \beta Z_p\)) in modo da ridurre sia \(R_{OUT}\) che il termine
\(\left(1 + \frac{C_L}{C_{OUT}}\right)\). Facendo così, però, si aumenta l'area occupata dal circuito e la capacità
parassita di uscita \(C_{OUT}\).

\begin{center}
	\begin{minipage}{0.45\textwidth}
		\centering \includegraphics[width=0.9\textwidth]{immagini/7_invertitore/ritardo_2.png}
	\end{minipage}
	\begin{minipage}{0.5\textwidth}
		Osservando il grafico di \(t_p\) in funzione di \(Z_n = W_n/L\) si osserva che:
		\begin{itemize}
			\item  diminuendo \(Z_n\) il tempo di ritardo aumenta a causa dell'aumento di \(R_{OUT}\) che limita
			la corrente di carica/scarica di \(C_{OUT} + C_L\)
			\item  aumentando \(Z_n\) il tempo di ritardo diminuisce inizialmente a causa della diminuzione di \(R_{OUT}\),
			poi si stabilizza attorno ad un valore minimo dovuto al fatto che l'invertitore deve anche \say{auto-caricarsi} la
			capacità parassita \(C_{OUT}\) che è ormai diventata significativa rispetto al carico \(C_L\)
		\end{itemize}
	\end{minipage}
\end{center}

\subsection{Consumo statico}
Il consumo statico è il consumo di potenza dell'invertitore quando l'ingresso è mantenuto costante ad un livello logico definito
(alto o basso). In questo caso uno dei due transistor è sempre spento per cui non c'è corrente di drenaggio tra i due terminali
di alimentazione \(V_{DD}\) e massa. L'unico contributo al consumo statico è dovuto alle correnti di sottosoglia dei due transistor,
alle correnti di perdita dell'ossido di gate e alla corrente inversa del diodo drain-substrato. Questi contributi sono però
trascurabili, per cui si può considerare che il consumo statico di un invertitore cmos sia praticamente nullo.

\subsection{Consumo dinamico}
\subsubsection*{Analisi del consumo dinamico per una porta logica generale}
Il consumo dinamico è il consumo di potenza di una generica porta logica durante la commutazione del segnale di uscita. La potenza
assorbita dalla porta logica viene utilizzata per la carica e scarica delle capacità parassite e del carico esterno. Si analizza
il consumo dinamico per entrambe le fasi di commutazione dell'uscita (LH e HL):
\begin{itemize}
	\item commutazione dell'uscita LH:
		\begin{align*}
			&\text{energia assorbita} & &E_{V_{DD}} = \int_0^T V_{DD} \; i_{DD}(t) \; dt = V_{DD} \int_{V_{OL}}^{V_{OH}} C \cdot dV = C V_{DD} (V_{OH} - V_{OL}) \\
			&\text{energia immagazzinata} & &E_{C} = \int_{V_{OL}}^{V_{OH}} V \cdot C \cdot dV = \frac{1}{2} C {V_{OH}}^2 - \frac{1}{2} C {V_{OL}}^2 \\
			&\text{energia dissipata} & &E_{diss} = E_{V_{DD}} - E_{C}
		\end{align*}
	\item commutazione dell'uscita HL:
		\begin{align*}
			&\text{energia assorbita} & &E_{V_{DD}} = 0  \qquad \text{il pmos è spento e non viene assorbita corrente da } V_{DD}\\
			&\text{energia immagazzinata} & &E_{C} = \int_{V_{OH}}^{V_{OL}} V \cdot C \cdot dV = \frac{1}{2} C {V_{OL}}^2 - \frac{1}{2} C {V_{OH}}^2 \\
			&\text{energia dissipata} & &E_{diss} = - E_{C}
		\end{align*}
\end{itemize}
Complessivamente si ottiene che l'energia totale assorbita dal generatore \(V_{DD}\) durante un ciclo di commutazione completo
(LH + HL) è pari all'energia dissipata e vale:
\[E_{V_{DD}, tot} = E_{diss, tot} = C \cdot V_{DD} (V_{OH} - V_{OL})\]
Definita la frequenza di commutazione \(f\) (numero di cicli di commutazione al secondo), si ottiene la potenza dinamica
dissipata dalla porta logica:
\[P_{DYN} = E_{diss, tot} \cdot f = C \cdot V_{DD} (V_{OH} - V_{OL}) \cdot f\]

\subsubsection*{Consumo dinamico dell'invertitore}
Applicando la formula generale del consumo dinamico all'invertitore cmos si ottiene:
\[P_{DYN, invertitore} = C_{OUT} \cdot {V_{DD}}^2 \cdot f\]

\subsubsection*{Cammino diretto}
Si osserva che durante la commutazione del segnale di uscita, per un breve intervallo di tempo, entrambi i transistor
possono essere contemporaneamente accesi, creando un cammino diretto tra \(V_{DD}\) e massa. Ciò si verifica quando la tensione
di ingresso attiva entrami i mosfet per legge \(V_{TN} < V_{IN} < V_{DD} - V_{TP}\) e il tempo di durata del cammino diretto
si indica con \(t_{cc}\). In questo intervallo di tempo si ha una corrente di cortocircuito \(I_{CC}\) che causa una dissipazione
di potenza addizionale e sottrae corrente alla carica/scarica delle capacità parassite e del carico esterno, allungando i tempi
di ritardo. Per questo motivo si vuole minimizzare la durata del cammino diretto durante la progettazione dell'invertitore.

\subsection{Oscillatore ad anello}
L'oscillatore ad anello è un circuito costituito da un numero dispari di invertitori collegati in cascata, in cui l'uscita
dell'ultimo invertitore è collegata all'ingresso del primo. In questo modo si crea un circuito ad anello chiuso che genera
un segnale che commuta periodicamente tra i livelli logici alto e basso ad una data frequenza.

NOTA1: per il corretto funzionamento dell'oscillatore è necessario che il numero di invertitori sia dispari e maggiore di 3:
con un invertitore si ha un circuito stabile alla soglia logica, con un numero pari di invertitori si ha un circuito che si
stabilizza ad uno dei due livelli logici senza commutare mai.

NOTA2: è necessario, inoltre, che il tempo di salita e discesa del segnale attraverso un singolo invertitore sia molto inferiore
al tempo di propagazione del segnale su tutto l'anello (metà del periodo di oscillazione), altrimenti il circuito non riesce a
commutare correttamente e si comporta come un singolo invertitore stabilizzato alla soglia logica.

Analizzando periodo, frequenza di oscillazione e consumo dinamico si ottiene:
\[T = 2 N t_p \qquad f = \frac{1}{T} = \frac{1}{2 N t_p} \qquad P_{DYN} = N \cdot C_{X} \, {V_{DD}}^2 \, f \qquad \text{per } N > t_{r,f} / t_p\]
dove \(N\) è il numero di invertitori nell'anello, \(t_p\) è il tempo di propagazione di un singolo invertitore e \(C_X\)
è la capacità di un nodo interno all'anello.

\begin{center}
	\begin{minipage}{0.55\textwidth}
		\centering \includegraphics[width=0.9\textwidth]{immagini/7_invertitore/oscillatore_1.png}

		\small{Oscillatore ad anello con 7 invertitori}
	\end{minipage}
	\begin{minipage}{0.4\textwidth}
		\centering \includegraphics[width=0.9\textwidth]{immagini/7_invertitore/oscillatore_2.png}

		\small{Segnale di uscita dell'oscillatore ad anello}
	\end{minipage}
\end{center}

\subsection{Buffer cmos}
\subsubsection*{Struttura generale}
Il buffer cmos è un circuito costituito da una serie di invertitori collegati in cascata, utilizzato per aumentare la capacità di
pilotaggio del segnale di uscita e migliorare le prestazioni del circuito logico.
\begin{center}
	\includegraphics[width=0.8\textwidth]{immagini/7_invertitore/buffer.png}
\end{center}

\noindent
Si suppone per ipotesi che \(L\), \(C_{d0}\) e \(C_{g0}\) siano uguali per tutti i transistor (pmos e nmos) di ogni invertitore.
Di conseguenza anche il parametro \(\gamma\) risulta essere uguale per tutti gli invertitori del buffer e indipendente dalle
dimensioni dei transistor.
\[\gamma_i = \frac{C_{OUT}}{C_{IN}} = \frac{C_{d0}}{C_{g0}L} \cdot \frac{W_p + W_n}{W_p + W_n} = \frac{C_{d0}}{C_{g0}L} = \text{costante per ogni invertitore}\]

\noindent
Il tempo di ritardo complessivo è dato dalla somma dei tempi di ritardo di ogni singolo invertitore:
\[t_{p,tot} = \sum_{i=1}^{N} t_{p,i} \qquad \text{con} \;\; t_{p,i} = t_{p0,i} \left(1 + \frac{f_i}{\gamma_i}\right)\]

\noindent
Le possibili ottimizzazioni per minimizzare il tempo di ritardo del buffer sono:
\begin{itemize}
	\item ottimizzare il tempo medio di ritardo intrinseco \(t_{p0}\)
	\item ottimizzare il numero di stadi \(N\)
	\item ottimizzare il fattore di carico \(f\) tra uno stadio e l'altro
\end{itemize}

\subsubsection*{Ottimizzazione di \(t_{p0}\) - dimensionamento del singolo invertitore}
Negli invertitori il segnale viene ripetutamente invertito e il tempo di ritardo complessivo è dato dalla somma dei tempi di
salita \(t_{pLH}\) e discesa \(t_{pHL}\) di ogni invertitore. Il modo migliore per ottimizzare il tempo complessivo è
minimizzare la somma dei tempi di ritardo \(t_{pLH}\) e \(t_{pHL}\), ovvero minimizzare il tempo di ritardo medio \(t_{p0}\)
di ogni stadio. Di conseguenza (per come è stato visto in precedenza) si deve scegliere il rapporto tra i fattori di forma
dei transistor come \(\beta = \sqrt{\rho}\). In questo modo tutti gli invertitori hanno lo stesso tempo di ritardo intrinseco
minimo: \[t_{p0} = 0.69 \, \frac{R_{n0}C_{d0}L}{2}\left(1 + \beta + \rho + \frac{\rho}{\beta}\right) \qquad R_{n0}, C_{d0}, L, \beta = \sqrt{\rho} \; \text{costanti per ogni invertitore}\]

\subsubsection*{Ottimizzazione di \(f\) - dimensionamento progressivo degli stadi}
Il fanout di un singolo invertitore dipende dal rapporto tra il fattore di forma dei suoi transistor e quello dell'invertitore successivo.
\[f_i = \frac{C_{IN, i+1}}{C_{IN, i}} = \frac{C_{g0} L (W_{p,i+1} + W_{n,i+1})}{C_{g0} L (W_{p,i} + W_{n,i})} = \frac{Z_{p,i+1} + Z_{n,i+1}}{Z_{p,i} + Z_{n,i}} = \frac{\beta Z_{n,i+1} + Z_{n,i+1}}{\beta Z_{n,i} + Z_{n,i}} = \frac{Z_{n,i+1}}{Z_{n,i}}\]
Si assume che tale rapporto sia costante per ogni stadio ottenendo così un dimensionamento progressivo dei transistor nei vari
invertitori tale per cui ogni invertitore ha dimensioni \(f\) volte maggiori del precedente e ogni nodo ha capacità \(f\) volte
maggiore del precedente:
\[\begin{array}{c c}
	Z_{n,i+1} = f \, Z_{n,i} = f^{i} \, Z_{n,1} & W_{n,i+1} = f \, W_{n,i} = f^{i} \, W_{n,1} \\
	Z_{p,i+1} = f \, Z_{p,i} = f^{i} \, Z_{p,1} & W_{p,i+1} = f \, W_{p,i} = f^{i} \, W_{p,1}
\end{array} \qquad C_{X,j} = f \, C_{X,j-1} = f^i \, C_{X,1}\]

\subsubsection*{Calcolo del valore di \(f\) ottimale}
Si osserva che tutti gli stadi hanno lo stesso tempo di ritardo \(t_{p,i}\), essendo \(t_{p0}\), \(\gamma\) e \(f\) costanti
per ogni invertitore, per cui è possibile calcolare il tempo di ritardo complessivo del buffer come:
\[t_{p,tot} = N \cdot t_p = N \cdot t_{p0} \left(1+\frac{f}{\gamma}\right) \qquad \text{per} \;\; t_{p} = t_{p0} \left(1+\frac{f}{\gamma}\right)\]
Calcolando il fanout totale del buffer si ottiene il numero di stadi necessari per raggiungere il carico esterno \(C_L\):
\[F = \frac{C_{L}}{C_{IN}} = \frac{C_{in,2}}{C_{in,1}} \frac{C_{in,3}}{C_{in,2}} \dots \frac{C_{in,N}}{C_{in,N-1}} = f_1 f_2 f_3 \dots f_N = f^N \quad \rightarrow \qquad N = \frac{\ln(F)}{\ln(f)}\]
Sostituendo il valore di \(N\) con l'espressione appena trovata, si ottiene un'espressione per il tempo di ritardo complessivo
con l'unica incognita \(f\). Per determinare il valore ottimale di \(f\) si cerca il tempo di ritardo è minimo, analizzando
la derivata rispetto a \(f\):
\[t_{p,tot} = \frac{\ln F}{\ln f} t_{p0} \left(1+\frac{f}{\gamma}\right) \qquad\qquad \frac{\partial t_{p,tot}}{\partial f} \;\; \rightarrow \;\; f = e^{1+\frac{\gamma}{f}}\]
L'equazione precedente non ammette una soluzione analitica, per cui si ricorre a metodi numerici per trovare il valore ottimale
di \(f\). Per \(\gamma = 1\) risulta \(f = 3.6\).

\subsubsection*{Calcolo del valore di \(N\), buffer invertente e non invertente}
Una volta determinato il valore di \(f\) e conoscendo il fanout totale \(F\) si può calcolare il numero di stadi necessari per
realizzare il buffer arrontondando il valore ottenuto all'intero più vicino:
\[N = \frac{\ln(F)}{\ln(f)}\]
Si osserva che se si vuole realizzare un buffer non invertente, è necessario avere un numero pari di stadi, mentre se si vuole
realizzare un buffer invertente, è necessario avere un numero dispari di stadi.

\subsubsection*{Consumo dinamico del buffer}
L'energia necessaria a far commutare l'uscita del buffer è usata per caricare e scaricare le capacità parassite di ogni
invertitore e la capacità di carico esterna \(C_L\). Per cui nella formula generale del consumo dinamico, al posto di \(C_{OUT}\)
si deve usare la somma delle capacità dei nodi interni più la capacità del nodo di uscita:
\[C_{tot} = \sum_{i=1}^{N-1} C_{X,i} + C_F \qquad \begin{array}{l l}
	C_{X,i} = C_{OUT,i} + C_{IN,i+1} & \text{capacità nodo interno i-esimo} \\
	C_F = C_{OUT,N} + C_L & \text{capacità nodo di uscita}
\end{array}\]
\[P_{DYN} = {C_{tot} \; V_{DD}}^2 \; f_\text{requenza}\]

\subsubsection*{Impieghi}
Il buffer cmos viene utilizzato per ridurre il tempo di ritardo quando si deve pilotare un carico esterno elevato. Alcuni esempi
di impiego del buffer sono:
\begin{itemize}
	\item pilotaggio di bus di dati o passaggio di segnali per piste lunghe con elevata capacità parassita
	\item pilotaggio di ingressi di circuiti logici con elevata capacità di ingresso
	\item pilotaggio da parte del clock di tutti i registri del computer
\end{itemize}
Siccome il buffer è costituito da più invertitori in cascata, il consumo dinamico è maggiore rispetto ad un singolo invertitore
ed inoltre occupa più area sul chip. Per questo motivo si cerca di utilizzare il buffer solo quando strettamente necessario.
