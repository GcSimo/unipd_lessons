\section{Condensatore MOS o CMOS}
\subsection{Struttura e funzionamento}
\subsubsection*{Struttura base}
Un condensatore MOS (Metal-Oxide-Semiconductor) è costituito da tre strati principali:
\begin{itemize}
	\item un metallo (Metal) che funge da elettrodo superiore detto \textbf{gate} (G), generalmente in polisilicio;
	\item un ossido (Oxide) che funge da dielettrico o \text{isolante}, di solito in biossido di silicio, SiO\(_2\);
	\item un semiconduttore (Semiconductor) che funge da elettrodo inferiore detto \textbf{substrato o body} (B),
	generalmente in silicio drogato di tipo p o n.
\end{itemize}

\subsubsection*{Funzionamento e proprietà}
\begin{itemize}
	\item Si identificano le dimensioni del dielettrico con \(L\) lunghezza, \(W\) larghezza e \(T_{ox}\) spessore.
	\item Si assume di collegare il substrato a massa (0 V) e applicare una tensione variabile al gate \(V_G\).
	\item La capacità del condensatore mos è data da: \(\displaystyle C_{ox} = \varepsilon \frac{W \ccdot L}{T_{ox}}\)
\end{itemize}

\begin{center}
	\includegraphics[width=0.5\textwidth]{immagini/4_mosfet/cmos_1.png}
\end{center}

\subsection{Condensatore mos con substrato di tipo p}
\subsubsection*{Tensione di gate negativa (\(V_G < 0\))}
Se \(V_G < 0\), il gate si carica negativamente, attirando le cariche positive (lacune) verso la superficie del semiconduttore
adiacente all'ossido, creando una \textbf{regione di accumulazione} di lacune.

\subsubsection*{Tensione di gate inferiore alla tensione di soglia (\(0 < V_G < V_{TN}\))}
Se \(0 < V_G < V_{TN}\), il gate si carica positivamente, creando una \textbf{regione di svuotamento} di lacune vicino alla
superficie del semiconduttore, lasciando dietro di sé ioni negativi fissi (atomi droganti). Si forma così una zona di carica
spaziale negativa, priva di portatori mobili.

\subsubsection*{Tensione di gate superiore alla tensione di soglia (\(V_G > V_{TN}\))}
Se \(V_G > V_{TN}\), il gate si carica ancora più positivamente, attirando elettroni verso la superficie del semiconduttore
adiacente all'ossido. Si crea una \textbf{regione di inversione} dove la concentrazione di elettroni supera quella delle lacune.
Si forma così un canale conduttivo di tipo n. La dimensione della regione di svuotamento rimane quasi costante, dopo aver
raggiunto il massimo per \(V_G = V_{TN}\), mentre la concentrazione di elettroni nella regione di inversione aumenta con \(V_G\).

\begin{center}
	\begin{minipage}{0.3\textwidth}
		\centering \includegraphics[width=0.85\textwidth]{immagini/4_mosfet/cmos_2.png}
		\small{Accumulazione per \(V_G < 0\)}
	\end{minipage}
	\begin{minipage}{0.3\textwidth}
		\centering \includegraphics[width=0.85\textwidth]{immagini/4_mosfet/cmos_3.png}
		\small{Svuotamento per \(0 < V_G < V_{TN}\)}
	\end{minipage}
	\begin{minipage}{0.3\textwidth}
		\centering \includegraphics[width=0.85\textwidth]{immagini/4_mosfet/cmos_4.png}
		\small{Inversione per \(V_G > V_{TN}\)}
	\end{minipage}
\end{center}

\subsection{Analisi del p-cmos in condizioni di svuotamento/inversione}
\subsubsection*{Densità di carica}
Analizzando la densità di carica \(\rho(x)\), il campo elettrico \(E(x)\) si ottengono le seguenti relazioni:
\[\rho(x) = \begin{cases}
	-q N_A & \text{per } -x_D < x < 0 \\
	0 & \text{altrimenti}
\end{cases}\]
\subsubsection*{Campo elettrico}
Dalla densità di carica si ricava il campo elettrico \(E(x)\) nella regione di svuotamento e nell'ossido, si noti che
c'è una discontinuità del campo elettrico all'interfaccia semiconduttore-ossido dovuta alla differenza di
permittività tra i due materiali, inoltre il campo elettrico nell'ossido è costante:
\[E(x) = \begin{cases}
	- q N_A (x + x_D) / \varepsilon_S & \text{per } -x_D < x < 0 \\
	- q N_A x_D / \varepsilon_{OX} & \text{per } 0 < x < t_{OX} \\
	0 & \text{altrimenti}
\end{cases} \qquad\qquad \begin{array}{l}
	E(0^-) = - \frac{q N_A x_D}{\varepsilon_S} \\[8pt]
	E(0^+) = - \frac{q N_A x_D}{\varepsilon_{OX}} \\[8pt]
	E_{OX} = E(0^+) = E(0^-) \frac{\varepsilon_S}{\varepsilon_{OX}}
\end{array}\]

\subsubsection*{Potenziale elettrico}
Si ottiene il potenziale nel substrato \(V_B\) (potenziale di riferimento), il potenziale all'interfaccia (nella giunzione tra
semiconduttore e ossido) \(V(0)\) e il potenziale al gate \(V_G\):
\[V_B = V(-x_D) = 0 \qquad\qquad V(0) = \frac{qN_A}{2\varepsilon_S} {x_D}^2 \quad\qquad V_G = V(t_{OX}) = \frac{qN_A}{2\varepsilon_S} {x_D}^2 + \frac{q N_A x_D}{\varepsilon_{OX}} t_{OX}\]

\subsubsection*{Concentrazioni dei portatori}
Si ricavano le concentrazioni dei portatori \(p_1\) e \(n_1\) nella regione neutra, lontano dall'interfaccia, e le concentrazioni
all'interfaccia \(p_2\), \(n_2\). All'interfaccia le concentrazioni variano esponenzialmente con \(V(0)\):
\[\begin{array}{l c l}
	p_1 = N_A & \quad & p_2 = p_1 \, e^{-\tfrac{V(0)}{V_T}} \\
	n_1 = n_i^2 / N_A & \quad & n_2 = n_1 \, e^{\tfrac{V(0)}{V_T}}
\end{array} \qquad \frac{n_2}{n_1} = \frac{p_1}{p_2} = e^{\tfrac{v_2-v_1}{V_T}} \qquad \frac{v_2 - v_1}{V_T} = \ln\frac{n_2}{n_1} = \ln\frac{p_1}{p_2}\]

\subsubsection*{Tensione di soglia}
Si definisce la \textbf{tensione di soglia} \(V_{TN}\) come differenza di potenziale tra gate e substrato \(V_G - V_B\) (pari
a \(V_G\)) per cui la concentrazione di elettroni all'interfaccia è uguale al numero di lacune nella regione neutra, ovvero
quando c'è inversione totale con \(n_2 = N_A\):
\[V(0) = \frac{qN_A}{2\varepsilon_S} {x_D}^2 = 2 V_T \ln\frac{N_A}{n_i}, \;\; x_D = \sqrt{\frac{4 \varepsilon_S V_T}{q N_A} \ln \frac{N_A}{n_i}} \; \rightarrow \; V_{TN} = 2 V_T \ln \frac{N_A}{n_i} + \frac{t_{OX}}{\varepsilon_{OX}} \sqrt{4\varepsilon_S q N_A V_T \ln \frac{N_A}{n_i}}\]
La tensione di soglia dipende, quindi, dallo spessore dell'ossido \(t_{OX}\), dalla concentrazione di drogaggio del substrato \(N_A\)
e dai materiali usati (tramite \(\varepsilon_{OX}\) e \(\varepsilon_S\)).

\subsubsection*{Carica elettrica e capacità}
La carica elettrica per unità di area immagazzinata nel condensatore mos è data dalla somma degli ioni fissi nella
regione di svuotamento e degli elettroni nella regione di inversione:
\[Q_{TOT} = Q_{RCS} + Q_{n} = Q \cdot V_G \qquad Q_{RCS} = C \cdot V_{TN} \qquad Q_{n} = C \cdot (V_G - V_{TN})\]

\subsubsection*{Rappresentazione grafica del comportamento del p-cmos}
\begin{center}
	\begin{minipage}{0.48\textwidth}
		\centering \includegraphics[width=0.9\textwidth]{immagini/4_mosfet/cmos_5.png}
	\end{minipage}
	\begin{minipage}{0.48\textwidth}
		\centering \includegraphics[width=0.9\textwidth]{immagini/4_mosfet/cmos_6.png}
	\end{minipage}
\end{center}
\vspace{0.5cm}

\subsection{cmos con substrato di tipo n e differenze rispetto al p-cmos}
\subsubsection*{Comportamento del cmos con substrato di tipo p}
Il funzionamento è analogo a quello del cmos con substrato di tipo p, ma con le polarità invertite:
\begin{itemize}
	\item per \(V_G > 0\), si crea una regione di accumulazione di elettroni.
	\item per \(0 > V_G > V_{TP}\), si crea una regione di svuotamento di elettroni.
	\item per \(V_G < V_{TP}\), si crea una regione di inversione con un canale conduttivo di tipo p.
\end{itemize}
\begin{center}
	\begin{minipage}{0.3\textwidth}
		\centering \includegraphics[width=0.85\textwidth]{immagini/4_mosfet/cmos_7.png}
		\small{Accumulazione per \(V_G > 0\)}
	\end{minipage}
	\begin{minipage}{0.3\textwidth}
		\centering \includegraphics[width=0.85\textwidth]{immagini/4_mosfet/cmos_8.png}
		\small{Svuotamento per \(0 > V_G > V_{TP}\)}
	\end{minipage}
	\begin{minipage}{0.3\textwidth}
		\centering \includegraphics[width=0.85\textwidth]{immagini/4_mosfet/cmos_9.png}
		\small{Inversione per \(V_G < V_{TP}\)}
	\end{minipage}
\end{center}

\vspace{0.5cm}
\subsubsection*{Rappresentazione aree di lavoro dei cmos di tipo p e n}
\begin{center}
	\includegraphics[width=0.8\textwidth]{immagini/4_mosfet/cmos_10.png}
\end{center}

\newpage

\section{Transistor MOSFET}
\subsection{Struttura generale e classificazione dei mosfet}
\subsubsection*{Introduzione}
Un transistor MOSFET (Metal-Oxide-Semiconductor Field-Effect Transistor) è un dispositivo a quattro terminali che sfrutta
un condensatore mos per controllare il flusso di corrente tra due terminali detti \textbf{source} (S) e \textbf{drain} (D)
tramite un potenziale applicato ad un terzo terminale detto \textbf{gate} (G). Il quarto terminale è il substrato o
\textbf{body} (B) e viene generalmente collegato al source o ad un potenziale di riferimento (massa o \(V_{D\!D}\)).
Di seguito una rappresentazione schematica di un n-mosfet:

\begin{center}
	\includegraphics[width=0.7\textwidth]{immagini/4_mosfet/mosfet_1.png}
\end{center}

\subsubsection*{Struttura fisica}
A livello fisico, un mosfet è costituito da un condensatore mos con gate e substrato affiancato da due regioni
pesantemente drogate di tipo opposto al substrato, dette source e drain, che fungono da terminali di ingresso e uscita.
Il dielettrico del condensatore mos è generalmente in diossido di silicio (SiO\(_2\)).

Ad ogni terminale è associato un potenziale elettrico e per ogni coppia di terminali si definisce la tensione e la corrente
tra i due nodi:
\begin{center}
	\begin{minipage}{0.34\textwidth}
		\begin{itemize}
			\item \(V_G\): potenziale del gate
			\item \(V_S\): potenziale del source
			\item \(V_D\): potenziale del drain
			\item \(V_B\): potenziale del substrato
		\end{itemize}
	\end{minipage}
	\begin{minipage}{0.65\textwidth}
		\begin{itemize}
			\item \(V_{XY} = V_X - V_Y\): tensione tra i nodi X e Y
			\item[] es. \(V_{GS} = V_G - V_S\) tensione tra gate e source
			\item \(I_{XY}\): corrente che entra nel nodo X e esce dal nodo Y
			\item[] es. \(I_{DS}\) corrente che entra nel drain e esce dal source 
		\end{itemize}
	\end{minipage}
\end{center}
Si definiscono inoltre le dimensioni fisiche del mosfet:
\begin{itemize}
	\item \(L\): lunghezza del canale tra source e drain
	\item \(W\): larghezza del canale tra source e drain
	\item \(t_{OX}\): spessore dell'ossido isolante tra gate e substrato
\end{itemize}

\subsubsection*{Classificazione}
In base al tipo di canale (e di conseguenza in base al tipo del substrato), i mosfet si classificano in:
\begin{itemize}
	\item \textbf{n-mosfet} o \textbf{mosfet a canale n}: \\ substrato di tipo p, source e drain di tipo n\(^+\), canale
	di tipo n con gli elettroni come portatori principali, il substrato p collegato al potenziale minore del circuito
	(massa o al source) e il source ha potenziale minore del drain
	\item \textbf{p-mosfet} o \textbf{mosfet a canale p}: \\ substrato di tipo n, source e drain di tipo p\(^+\), canale
	di tipo p con le lacune come portatori principali, il substrato n collegato al potenziale maggiore del circuito
	(\(V_{D\!D}\) o al source) e il source ha potenziale maggiore del drain
\end{itemize}


\subsection{Struttura di un n-mosfet e vincoli sui potenziali}
Si assume per convenzione che il terminale di source ha potenziale minore di quello di drain: \(V_S < V_D\) e, di conseguenza,
che la corrente scorra dal drain al source: \(I_{DS} > 0\).

\begin{center}
	\includegraphics[width=0.6\textwidth]{immagini/4_mosfet/mosfet_2.png}
\end{center}

\subsubsection*{Condizioni all'equilibrio (nessuna tensione applicata)}
In assenza di tensioni, tutti i potenziali sono nulli e non c'è corrente tra i terminali. In particolare si hanno due giunzioni
pn in equilibrio tra il substrato p e le regioni n\(^+\) del source e del drain. Siccome le regioni n\(^+\) sono pesantemente
drogate, la regione di svuotamento si estende quasi totalmente nel substrato p.

\subsubsection*{Vincoli di polarizzazione dei diodi e potenziale di substrato}
I due diodi con catodi collegati ai nodi source e drain e con anodo in comune nel substrato, devono rimanere in interdizione
per il corretto funzionamento del mosfet, si ottengono le seguenti condizioni:
\begin{itemize}
	\item \(V_{BS} \leq 0 \rightarrow V_S \geq V_B\) (diodo source-substrato in interdizione)
	\item \(V_{BD} \leq 0 \rightarrow V_D \geq V_B\) (diodo drain-substrato in interdizione)
\end{itemize}
Si ottiene che il substrato \(V_B\) deve essere il nodo a potenziale più basso \(V_D \geq V_S \geq V_B\). Si solito si
collega il substrato al potenziale minore dell'intero circuito (massa) \(V_B = 0\) oppure al source \(V_B = V_S\).

\textit{NOTA}: Non è possibile collegare il substrato al drain perché si violerebbe la condizione di interdizione del diodo
source-substrato in quanto \(V_S < V_D\) e quindi \(V_{BS} = V_B - V_S = V_D - V_S > 0\).

\subsubsection*{Potenziale e tensioni di gate}
Il potenziale di gate \(V_G\) controlla la tensione tra le armature del condensatore mos che si forma tra il gate e il substrato
e di conseguenza identifica l'area di lavoro del p-mosfet. La tensione tra le due armature è variabile lungo la lunghezza del
canale ed è compresa tra le tensioni di gate-source e gate-drain ai margini del canale:
\begin{itemize}
	\item in prossimità del source \(V_{C,\text{source}} = V_{GS} = V_G - V_S\)
	\item in prossimità del drain \(V_{C,\text{drain}} = V_{GD} = V_G - V_D\)
\end{itemize}
\textit{NOTA}: siccome \(V_D \geq V_S\), si ha \(V_{GS} \geq V_{GD}\)

\newpage

\subsection{Aree di lavoro di un n-mosfet}
\subsubsection*{Transistor n-mosfet spento o in interdizione per \(V_{GS} < V_{TN}\)}
La tensione tra le armature del condensatore mos è inferiore alla tensione di soglia \(V_{TN}\) sia in prossimità del source
(\(V_{GS} < V_{TN}\)) che in prossimità del drain (\(V_{GD} < V_{GS} < V_{TN}\)).  Il condensatore, quindi, è in regime di
svuotamento ed è presente nel substrato, in prossimità dell'ossido, un'area di svuotamento (senza portatori di carica) che
separa source e drain. Non essendoci cariche libere per condurre corrente tra drain e source, si ha corrente nulla \(I_{DS} = 0\)
e il transistor è spento o in interdizione.
\begin{center}
	\begin{minipage}{0.45\textwidth}
		\centering \includegraphics[width=0.9\textwidth]{immagini/4_mosfet/mosfet_3.png}
	\end{minipage}
	\begin{minipage}{0.5\textwidth}
		\begin{align*}
			V_{GS} &< V_{TN} & \text{source in interdizione} \\[4pt]
			V_{GD} &< V_{GS} < V_{TN} & \text{drain in interdizione}
		\end{align*}
		\[I_{DS} = 0\]
	\end{minipage}
\end{center}

\subsubsection*{Transistor n-mosfet in conduzione lineare (o triodo) per \(V_{DS} < V_{GS} - V_{TN}\)}
La tensione delle armature del condensatore mos è superiore alla tensione di soglia \(V_{TN}\) sia in prossimità del drain
(\(V_{GD} > V_{TN}\)) che in prossimità del source (\(V_{GS} > V_{GD} > V_{TN}\)). Il condensatore si trova in regime di
inversione e si forma un canale conduttivo di tipo n (di elettroni) che congiunge source e drain. Gli elettroni che
costituiscono il canale si muovono liberamente dal source al drain, permettendo il passaggio di corrente \(I_{DS} > 0\) dal
drain al source. Il transistor si comporta come una resistenza (dipendente da \(V_{GS}\)) tra drain e source, lineare per
\(V_{DS}\) piccoli o parabolica per \(V_{DS}\) grandi.
\begin{center}
	\begin{minipage}{0.45\textwidth}
		\centering \includegraphics[width=0.9\textwidth]{immagini/4_mosfet/mosfet_4.png}
	\end{minipage}
	\begin{minipage}{0.5\textwidth}
		\begin{align*}
			V_{GD} &> V_{TN} & \text{drain in conduzione} \\[4pt]
			V_{GS} &= V_{GD} + V_{DS} & \\
			V_{GS} &> V_{TN} + V_{DS} & \text{source in conduzione}
		\end{align*}
		\[I_{DS} = k_n V_{DS} \left( V_{GS} - V_{TN} - \frac{V_{DS}}{2} \right)\]
		\[k_n = k_n' \cdot Z_n \qquad k_n' = \mu_n \varepsilon / T_{OX} \qquad Z_n = W_n/L_n\]
	\end{minipage}
\end{center}


\subsubsection*{Transistor n-mosfet in saturazione per pinchoff per \(V_{DS} > V_{GS} - V_{TN}\)}
La tensione delle armature del condensatore mos è superiore alla tensione di soglia \(V_{TN}\) in prossimità del source
(\(V_{GS} > V_{TN}\)), ma inferiore alla tensione di soglia in prossimità del drain (\(V_{GD} < V_{TN}\)). Il condensatore
si trova in regime misto: inversione in prossimità del source e svuotamento in prossimità del drain e si forma un canale
conduttivo di tipo n che congiunge solo parzialmente source e drain. Nonostante ciò i portatori riescono a fluire lo stesso
nella RCS spinti dal campo elettrico tra source e drain. In questo modo si ha ugualmente il passaggio di una corrente
\(I_{DS} \neq 0\) costante rispetto a \(V_{DS}\) (a meno della modulazione di lunghezza di canale).
\begin{center}
	\begin{minipage}{0.45\textwidth}
		\centering \includegraphics[width=0.9\textwidth]{immagini/4_mosfet/mosfet_5.png}
	\end{minipage}
	\begin{minipage}{0.5\textwidth}
		\begin{align*}
			V_{GS} &> V_{TN} & \text{source in conduzione} \\[4pt]
			V_{GD} &< V_{TN} & \text{drain in interdizione}
		\end{align*}
		\vspace{-20pt}
		\begin{align*}
			V_{GD} < V_{TN} \;\; &\rightarrow \;\; V_{GS} - V_{DS} < V_{TN} \\
			&\rightarrow \;\; V_{DS} > V_{GS} - V_{TN}
		\end{align*}
		\[I_{DS} = \frac{k_n}{2} (V_{GS} - V_{TN})^2\]
		\[k_n = k_n' \cdot Z_n \qquad k_n' = \mu_n \varepsilon / T_{OX} \qquad Z_n = W_n/L_n\]
	\end{minipage}
\end{center}

\subsubsection*{Modulazione di lunghezza di canale con n-mosfet in saturazione per pinchoff}
Quando il mosfet entra in saturazione per pinchoff, la regione di strozzamento si sposta verso il source, riducendo la lunghezza
\(L\) del canale conduttivo. Questo fenomeno, chiamato modulazione di lunghezza di canale, provoca un lieve aumento della corrente
\(I_{DS}\) con l'aumentare di \(V_{DS}\) anche in saturazione. Si può modellare questo effetto aggiungendo alla corrente in
saturazione un termine correttivo che dipende dal coefficiente di modulazione di lunghezza \(\lambda\) determinato dalle
caratteristiche fisiche del mosfet:

\begin{center}
	\begin{minipage}{0.6\textwidth}
		\centering \includegraphics[width=0.9\textwidth]{immagini/4_mosfet/mosfet_6.png}
	\end{minipage}
	\begin{minipage}{0.35\textwidth}
		\[I_{DS} = \frac{k_n}{2} (V_{GS} - V_{TN})^2 (1 + \lambda V_{DS})\]
		\[\lambda V_{DS} = \frac{\Delta L}{L}\]
	\end{minipage}
\end{center}
Analizzando la continuità della corrente tra la regione lineare e la regione di saturazione con modulazione di lunghezza di canale
si osserva che matematicamente c'è una discontinuità che in natura non esiste. Per ovviare a questo problema si utilizza la
correzione di lunghezza di canale anche nella regione lineare (tratteggio blu nella figura superiore):
\begin{align*}
	I_{DS,\text{lin}} &= k_n V_{DS} \left( V_{GS} - V_{TN} - \frac{V_{DS}}{2} \right) (1 + \lambda V_{DS}) & \text{per} \; V_{DS} \ll V_{GS} - V_{TN} \\
	I_{DS,\text{sat}} &= \frac{k_n}{2} (V_{GS} - V_{TN})^2 (1 + \lambda V_{DS}) & \text{per} \; V_{DS} \gg V_{GS} - V_{TN}
\end{align*}

\subsubsection*{Transistor n-mosfet in saturazione di velocità per \(V_{DS} > V_{DSATN}\)}
I portatori di carica (in questo caso elettroni) si muovono nel canale spinti dal campo elettrico \(E = V_{DS} / L\) con velocità
\(v_n = \mu_n E = \mu_n V_{DS} / L\). La velocità è, quindi, proporzionale a \(V_{DS}\).
Tale velocità raggiunge un valore massimo costante detto velocità di saturazione \(v_\text{sat,n} = 8 \cdot 10^6 \cm/\s\) per
un certo campo elettrico critico \(E_{crit,n}\) e una certa tensione critica \(V_{DS,sat,n}\):
\[E_\text{crit,n} = \frac{v_{sat,n}}{\mu_n} = 1.3 \frac{\V}{\mu\m} \qquad V_{DSATN} = E_\text{crit,n} \cdot L = \frac{v_{sat,n}}{\mu_n}L\]
Per \(V_{DS} > V_{DSATN}\), si osserva che la velocità degli elettroni e di conseguenza anche la corrente \(I_{DS}\) rimangono
costanti. Si ha un fenomeno di saturazione anticipato, detto saturazione di velocità, quando ci si aspetterebbe che il mosfet
lavori in regime di conduzione lineare.
\begin{center}
	\begin{minipage}{0.38\textwidth}
		\centering \includegraphics[width=0.9\textwidth]{immagini/4_mosfet/mosfet_17.png}
	\end{minipage}
	\begin{minipage}{0.6\textwidth}
		La corrente di saturazione di velocità, tenendo conto anche della modulazione di lunghezza di canale, è data da:
		\begin{align*}
			I_{DS,sat,n} &= W C_{ox} v_{sat,n} (V_{GS} - V_{TN}) \\
			&= k_n V_{DSATN}\left(V_{GS} - V_{TN} - \frac{V_{DSATN}}{2}\right)(1 + \lambda V_{DS})	
		\end{align*}
	\end{minipage}
\end{center}

\newpage

\subsection{Curve caratteristiche di corrente-tensione di un n-mosfet}
La corrente \(I_{DS}\) dipende da due tensioni indipendenti \(V_{GS}\) e \(V_{DS}\). Si identificano due curve caratteristiche:
\begin{itemize}
	\item \textbf{caratteristica di uscita} con \(I_{DS}\) in funzione di \(V_{DS}\) per valori costanti di \(V_{GS}\)
	\item \textbf{caratteristica di trasferimento} o \textbf{transcaratteristica} con \(I_{DS}\) in funzione di \(V_{GS}\) per valori costanti di \(V_{DS}\)
\end{itemize}

\begin{center}
	\begin{minipage}{0.4\textwidth}
		\centering \includegraphics[width=0.9\textwidth]{immagini/4_mosfet/mosfet_7.png}
	\end{minipage}
	\begin{minipage}{0.55\textwidth}
		Per analizzare la curve caratteristiche si collega il mosfet ad un circuito di test con due generatori di tensione \(V_{GS}\)
		e \(V_{DS}\) e si misura la corrente \(I_{DS}\) che scorre tra drain e source. Per la caratteristica di uscita si mantiene
		\(V_{GS}\) costante e si varia \(V_{DS}\), mentre per la transcaratteristica si mantiene \(V_{DS}\) costante e si varia \(V_{GS}\).
	\end{minipage}
\end{center}

\subsubsection*{Caratteristica di uscita \(I_{DS}\) - \(V_{DS}\)}
Le tre aree di funzionamento del n-mosfet si riflettono nella caratteristica di uscita \(I_{DS}\) - \(V_{DS}\):
\begin{itemize}
	\item \textbf{interdizione} per \(V_{GS} < V_{TN}\): linea orizzontale coincidente con l'asse delle ascisse
	\item \textbf{conduzione lineare} per \(V_{DS} < V_{GS} - V_{TN}\): a sinistra della linea di saturazione
	\item \textbf{saturazione per pinchoff} per \(V_{DS} > V_{GS} - V_{TN}\): a destra della linea di saturazione
	\item \textbf{curva di saturazione}: separa la regione di funzionamento lineare e quella di saturazione, è costituita dai
	punti \((V_{DS}, I_{DS})\) che soddisfano l'equazione:
\end{itemize}

\begin{center}
	\begin{minipage}{0.45\textwidth}
		\centering \includegraphics[width=0.9\textwidth]{immagini/4_mosfet/mosfet_8.png}
	\end{minipage}
	\begin{minipage}{0.5\textwidth}
		\[I_{DS,sat} = \frac{k_n}{2} {V_{DS}}^2 \;\;\text{con}\;\; V_{DS} = V_{GS} - V_{TN}\]
		\vspace{10pt}

		\begin{itemize}
			\item aumentando \(V_{DS}\), il transistor passa da regime lineare a regime di saturazione per pinchoff
			\item aumentando \(V_{GS}\) (\(V_{G4} > V_{G3} > V_{G2} > V_{G1} > V_{TN}\)) la corrente \(I_{DS}\) aumenta
		\end{itemize}
		\vspace{20pt}
	\end{minipage}
\end{center}

\subsubsection*{Transcaratteristica \(I_{DS}\) - \(V_{GS}\)}
Le tre aree di funzionamento del n-mosfet si riflettono nella caratteristica di uscita \(I_{DS}\) - \(V_{GS}\):
\begin{itemize}
	\item \textbf{interdizione} per \(V_{GS} < V_{TN}\): linea orizzontale coincidente con l'asse delle ascisse
	\item \textbf{saturazione per pinchoff} per \(V_{TN} <  V_{GS} < V_{DS} + V_{TN}\) crescita quadratica
	\item \textbf{conduzione lineare} per \(V_{DS} + V_{TN}< V_{GS}\) crescita lineare
\end{itemize}

\begin{center}
	\begin{minipage}{0.45\textwidth}
		\centering \includegraphics[width=0.9\textwidth]{immagini/4_mosfet/mosfet_9.png}
	\end{minipage}
	\begin{minipage}{0.5\textwidth}
		\begin{itemize}
			\item aumentando \(V_{GS}\), il transistor passa da regime di interdizione a regime di saturazione per pinchoff
			e successivamente a regime lineare
			\item aumentando \(V_{DS}\) il confine tra regime di saturazione e regime lineare si sposta verso destra, inoltre
			aumenta la pendenza della retta in regime lineare
		\end{itemize}
		\vspace{20pt}
	\end{minipage}
\end{center}

\newpage

\subsection{Modello a canale corto di un n-mosfet}
\subsubsection*{Equazione generale per \(I_{DS}\)}
Il modello a canale corto tiene conto di tutti i fenomeni fisici che avvengono in un mosfet reale, tra cui la modulazione di
lunghezza di canale e la saturazione di velocità. Tutte le aree di funzionamento del n-mosfet si possono descrivere con un'unica
equazione per la corrente \(I_{DS}\):
\begin{itemize}
	\item per \(V_{GS} < V_{TN}\) mosfet in interdizione e \(I_{DS} = 0\)
	\item per \(V_{GS} > V_{TN}\) mosfet in conduzione con \(I_{DS}\) che dipende da \(V_{MIN}\):
	\begin{align*}
		I_{DS} &= k_n' Z_n V_{MIN} \left(V_{GS} - V_{TN} - \frac{V_{MIN}}{2}\right) (1 + \lambda_n V_{DS}) \qquad k_n' = \frac{\mu_n \varepsilon}{T_{OX}} \qquad Z_n = \frac{W_n}{L_n} \\
		V_{MIN} &= \min \left\{\begin{array}{l l}
			V_{DS} & \text{regime lineare} \\
			V_{GS} - V_{TN} & \text{saturazione per pinchoff} \\
			V_{DSATN} & \text{saturazione per velocità}
		\end{array} \right\}
	\end{align*}
\end{itemize}

\subsubsection*{Caratteristica di uscita del modello a canale corto}
Includendo anche gli effetti della saturazione di velocità, la caratteristica di uscita complessiva del modello a canale corto
del n-mosfet risulta come segue:
\begin{center}
	\includegraphics[width=0.7\textwidth]{immagini/4_mosfet/mosfet_18.png}
\end{center}

\begin{itemize}
	\item la regione di conduzione lineare e la regione di saturazione per pinchoff sono separate dalla curva di saturazione per pinchoff
	data dalla parabola \(I_{DS} = k_n / 2 \cdot {V_{DS}}^2\) con \(V_{DS} = V_{GS} - V_{TN}\)
	\item la regione di conduzione lineare e la regione di saturazione per velocità sono separate dalla linea verticale \(V_{DS} = V_{DSATN}\)
	\item la regione di saturazione per pinchoff e la regione di saturazione per velocità sono separate dalla linea orizzontale
	\(I_{DS} = k_n / 2 \cdot {V_{DS}}^2\) con \(V_{DS} = V_{DSATN}\)
\end{itemize}

\newpage

\subsection{Struttura di un p-mosfet e vincoli sui potenziali}
Il funzionamento di un mosfet a canale p (p-mosfet) con substrato di tipo n è analogo a quello del n-mosfet, ma con le polarità
invertite. Tutte le tensioni, infatti, sono negative. Si assume per convenzione che il terminale di source ha potenziale
maggiore di quello di drain (\(V_S > V_D\)) e, di conseguenza, che la corrente scorre dal source al drain rimanendo sempre
positiva (\(I_{DS} > 0\)).

\begin{center}
	\includegraphics[width=0.6\textwidth]{immagini/4_mosfet/mosfet_10.png}
\end{center}

\subsubsection*{Condizioni all'equilibrio}
In analogia al n-mosfet, all'equilibrio le due giunzioni pn tra substrato n e le regioni p\(^+\) del source e del drain sono in
equilibrio e la regione di svuotamento si estende quasi totalmente nel substrato n.

\subsubsection*{Vincoli di polarizzazione dei diodi e potenziale di substrato}
I due diodi con anodi collegati ai nodi source e drain e con catodo in comune nel substrato, devono rimanere in interdizione
per il corretto funzionamento del mosfet, si ottengono le seguenti condizioni:
\begin{itemize}
	\item \(V_{BS} \geq 0 \rightarrow V_S \leq V_B\) (diodo source-substrato in interdizione)
	\item \(V_{BD} \geq 0 \rightarrow V_D \leq V_B\) (diodo drain-substrato in interdizione)
\end{itemize}
Si ottiene che il substrato \(V_B\) deve essere il nodo a potenziale più alto \(V_D \leq V_S \leq V_B\). Di solito si
collega il substrato al potenziale maggiore dell'intero circuito (alimentazione) \(V_B = V_{DD}\) oppure al source \(V_B = V_S\).

\textit{NOTA}: Non è possibile collegare il substrato al drain perché si violerebbe la condizione di interdizione del diodo
source-substrato in quanto \(V_S > V_D\) e quindi \(V_{BS} = V_B - V_S = V_D - V_S < 0\).

\subsubsection*{Potenziale e tensioni di gate}
Analogamente al n-mosfet, il potenziale di gate \(V_G\) controlla l'area di lavoro del p-mosfet. La tensione tra le due armature
è variabile lungo la lunghezza del canale ed è compresa tra le tensioni di gate-source e gate-drain ai margini del canale:
\begin{itemize}
	\item in prossimità del source \(V_{C,\text{source}} = V_{GS} = V_G - V_S\)
	\item in prossimità del drain \(V_{C,\text{drain}} = V_{GD} = V_G - V_D\)
\end{itemize}
\textit{NOTA}: siccome \(0 > V_S \geq V_D\), si ha \(V_{GS} \leq V_{GD}\) (oppure \(|V_{GS}| \geq |V_{GD}|\))

\newpage

\subsection{Aree di lavoro di un p-mosfet}
Le aree di lavoro del p-mosfet sono analoghe a quelle del n-mosfet, ma con le polarità invertite:
\begin{itemize}
	\item \textbf{p-mosfet spento o in interdizione} per \(0 > V_{GS} > V_{TP}\): \\
	il condensatore è in regime di svuotamento e non c'è corrente tra drain e source \(I_{DS} = 0\)
	\item \textbf{p-mosfet in conduzione lineare} per \(V_{DS} > V_{GS} - V_{TP} \; \Leftrightarrow \; 0 > V_{TP} > V_{GD} > V_{GS}\): \\
	si ha una corrente \(I_{DS}\) tra source e drain data dal movimento delle lacune nel canale di tipo p
	\[I_{DS} = k_p V_{DS} \left(V_{GS} - V_{TP} - \frac{V_{DS}}{2}\right) \qquad k_p = k_p' \cdot Z_p \qquad k_p' = \mu_p \varepsilon / T_{OX} \qquad Z_p = W_p/Z_p\]
	\item \textbf{p-mosfet in saturazione per pinchoff} per \(V_{DS} < V_{GS} - V_{TP} \; \Leftrightarrow \; 0 > V_{GD} > V_{TP} > V_{GS}\): \\
	il condensatore è in regime misto e si ha una corrente \(I_{DS}\) costante rispetto a \(V_{DS}\):
	\[I_{DS} = \frac{k_p}{2} (V_{GS} - V_{TP})^2\]
	\item \textbf{Modulazione di lunghezza di canale}: \\
	Analoga a quella dell'n-mosfet, applicando il fattore correttivo si ottiene:
	\[I_{DS} = \frac{k_p}{2} (V_{GS} - V_{TP})^2 (1 + \lambda_p V_{DS}) \qquad \text{con} \; V_{DS} < 0, \; \lambda_p < 0\]
	\item \textbf{p-mosfet in saturazione di velocità} per \(V_{DS} < V_{DSATP}\): \\
	Analogo a quello dell' n-mosfet, si ha una corrente \(I_{DS}\) costante rispetto a \(V_{DS}\):
	\[I_{DS} = k_p V_{DSATP} \left(V_{GS} - V_{TP} - \frac{V_{DSATP}}{2}\right) (1 + \lambda_p V_{DS}) \qquad \text{con} \; V_{DSATP} < 0\]
	(alcuni valori tipici per p-mosfet: \(\mu_p = 200 \; \cm^2 / \V\s \quad v_{sat,p} \approx 4 \cdot 10^6 \; \cm/\s \quad E_C \approx 2 \; \V/\cm\))
\end{itemize}

\subsection{Curve caratteristiche di corrente-tensione di un p-mosfet}
Allo stesso modo dell'n-mosfet, si definiscono le curve caratteristiche di un p-mosfet, studiate attraverso un circuito simile:

\begin{center}
	\begin{minipage}{0.35\textwidth}
		\centering \includegraphics[width=\textwidth]{immagini/4_mosfet/mosfet_11.png}
	\end{minipage}
	\begin{minipage}{0.63\textwidth}
		\begin{itemize}
			\item \textbf{caratteristica di uscita} con \(I_{DS}\) in funzione di \(V_{DS}\) per valori costanti di \(V_{GS}\)
			\item \textbf{caratteristica di trasferimento} o \textbf{transcaratteristica} con \(I_{DS}\) in funzione di \(V_{GS}\) per valori costanti di \(V_{DS}\)
		\end{itemize}
	\end{minipage}
\end{center}

\subsubsection*{Caratteristica di uscita \(I_{DS}\) - \(V_{DS}\)}
Rispetto al n-mosfet, la caratteristica di uscita \(I_{DS}\) - \(V_{DS}\) del p-mosfet è speculare rispetto all'asse delle ordinate,
mantenendo invariate le forme delle curve e il posizionamento delle aree di funzionamento:
\begin{center}
	\begin{minipage}{0.4\textwidth}
		\centering \includegraphics[width=0.9\textwidth]{immagini/4_mosfet/mosfet_12.png}
	\end{minipage}
	\begin{minipage}{0.58\textwidth}
		\begin{itemize}
			\item \textbf{interdizione} per \(V_{GS} > 0\): linea orizzontale coincidente con l'asse delle ascisse
			\item \textbf{conduzione lineare} per \(V_{DS} > V_{GS} - V_{TP}\): a destra della linea di saturazione
			\item \textbf{saturazione per pinchoff} per \(V_{DS} < V_{GS} - V_{TP}\): a sinistra della linea di saturazione
			\item \textbf{curva di saturazione}: separa la regione di funzionamento lineare e quella di saturazione, è costituita
			dai punti in cui \(V_{DS} = V_{GS} - V_{TP}\)
		\end{itemize}
	\end{minipage}
\end{center}

\subsubsection*{Transcaratteristica \(I_{DS}\) - \(V_{GS}\)}
Rispetto al n-mosfet, la transcaratteristica \(I_{DS}\) - \(V_{GS}\) del p-mosfet è speculare rispetto all'asse delle ordinate,
mantenendo invariate le forme delle curve e il posizionamento delle aree di funzionamento:

\begin{center}
	\begin{minipage}{0.4\textwidth}
		\centering \includegraphics[width=0.9\textwidth]{immagini/4_mosfet/mosfet_13.png}
	\end{minipage}
	\begin{minipage}{0.58\textwidth}
		\begin{itemize}
			\item \textbf{interdizione} per \(V_{GS} > V_{TP}\): linea orizzontale coincidente con l'asse delle ascisse
			\item \textbf{saturazione per pinchoff} per \(V_{DS} + V_{TP} < V_{GS} < V_{TP}\): crescita quadratica
			\item \textbf{conduzione lineare} per \(V_{GS} < V_{DS} + V_{TP}\): crescita lineare
		\end{itemize}
		\vspace{20pt}
	\end{minipage}
\end{center}

\subsection{Modello a canale corto di un p-mosfet}
\subsubsection*{Equazione generale per \(I_{DS}\)}
Anche per il p-mosfet si può utilizzare un'unica equazione per la corrente \(I_{DS}\) che tiene conto di tutti i fenomeni fisici:
\begin{itemize}
	\item per \(V_{GS} > V_{TP}\) mosfet in interdizione e \(I_{DS} = 0\)
	\item per \(V_{GS} < V_{TP}\) mosfet in conduzione con \(I_{DS}\) che dipende da \(V_{MAX}\):
	\begin{align*}
		I_{DS} &= k_p' Z_p V_{MAX} \left(V_{GS} - V_{TP} - \frac{V_{MAX}}{2}\right) (1 + \lambda_p V_{DS}) \qquad k_p' = \frac{\mu_p \varepsilon}{T_{OX}} \qquad Z_p = \frac{W_p}{L_p} \\
		V_{MAX} &= \max \left\{\begin{array}{l l}
			V_{DS} & \text{regime lineare} \\
			V_{GS} - V_{TP} & \text{saturazione per pinchoff} \\
			V_{DSATP} & \text{saturazione per velocità}
		\end{array} \right\}
	\end{align*}
\end{itemize}

\subsubsection*{Caratteristica di uscita del modello a canale corto}
La caratteristica di uscita complessiva del modello a canale corto del p-mosfet risulta come segue:
\begin{center}
	\includegraphics[width=0.65\textwidth]{immagini/4_mosfet/mosfet_19.png}
\end{center}
\begin{itemize}
	\item la regione di conduzione lineare e la regione di saturazione per pinchoff sono separate dalla curva di saturazione per pinchoff
	data dalla parabola \(I_{DS} = k_p / 2 \cdot {V_{DS}}^2\) con \(V_{DS} = V_{GS} - V_{TP}\)
	\item la regione di conduzione lineare e la regione di saturazione per velocità sono separate dalla linea verticale \(V_{DS} = V_{DSATP}\)
	\item la regione di saturazione per pinchoff e la regione di saturazione per velocità sono separate dalla linea orizzontale
	\(I_{DS} = k_p / 2 \cdot {V_{DS}}^2\) con \(V_{DS} = V_{DSATP}\)
\end{itemize}

\newpage

\subsection{Simbologia e rappresentazione circuitale dei mosfet}
\subsubsection*{Simbologia a 4 terminali - con substrato}
Nelle simbologie a 4 terminali non è possibile identificare univocamente il drain e il source in quanto sono perfettamente
identici. Nelle rappresentazioni a freccia, la freccia sul terminale del substrato indica il tipo di mosfet ed è concorde
con il flusso di corrente nel diodo formato tra substrato e source/drain.

\begin{center}
	\begin{minipage}{0.45\textwidth}
		\centering
		\begin{minipage}{0.4\textwidth}
			\centering \includegraphics[width=0.8\textwidth]{immagini/4_mosfet/nmos_1.png}
			\small{n-mosfet}
		\end{minipage}
		\begin{minipage}{0.4\textwidth}
			\centering \includegraphics[width=0.8\textwidth]{immagini/4_mosfet/pmos_1.png}
			\small{p-mosfet}
		\end{minipage}
		\vspace{7pt}

		\small{rappresentazione a freccia}
	\end{minipage}
	\begin{minipage}{0.45\textwidth}
		\centering
		\begin{minipage}{0.4\textwidth}
			\centering \includegraphics[width=0.8\textwidth]{immagini/4_mosfet/nmos_2.png}
			\small{n-mosfet}
		\end{minipage}
		\begin{minipage}{0.4\textwidth}
			\centering \includegraphics[width=0.8\textwidth]{immagini/4_mosfet/pmos_2.png}
			\small{p-mosfet}
		\end{minipage}
		\vspace{7pt}

		\small{rappresentazione digitale}
	\end{minipage}
\end{center}

\subsubsection*{Simbologia a 3 terminali - senza substrato}
Nelle simbologie a 3 terminali, il substrato viene omesso in quanto collegato al source o ad un potenziale di riferimento.
Quando il substrato è collegato al source, il terminale source è identificato con la freccia concorde al flusso di corrente
tra source e drain. Quando il substrato è collegato ad un potenziale di riferimento, il terminale source non è
identificabile univocamente in quanto non ci sono freccie.

\begin{center}
	\begin{minipage}{0.45\textwidth}
		\centering
		\begin{minipage}{0.4\textwidth}
			\centering \includegraphics[width=0.8\textwidth]{immagini/4_mosfet/nmos_3.png}
			\small{n-mosfet}
		\end{minipage}
		\begin{minipage}{0.4\textwidth}
			\centering \includegraphics[width=0.8\textwidth]{immagini/4_mosfet/pmos_3.png}
			\small{p-mosfet}
		\end{minipage}
		\vspace{7pt}

		\small{substrato collegato al source}
	\end{minipage}
	\begin{minipage}{0.45\textwidth}
		\centering
		\begin{minipage}{0.4\textwidth}
			\centering \includegraphics[width=0.9\textwidth]{immagini/4_mosfet/nmos_4.png}
			\small{n-mosfet}
		\end{minipage}
		\begin{minipage}{0.4\textwidth}
			\centering \includegraphics[width=0.9\textwidth]{immagini/4_mosfet/pmos_4.png}
			\small{p-mosfet}
		\end{minipage}
		\vspace{7pt}

		\small{substrato collegato a potenziale di riferimento}
	\end{minipage}
\end{center}

\vspace{10pt}

\subsection{Struttura reale del mosfet}
Per ottimizzare le prestazioni, lo spazio e il processo produttivo vengono apportate alcune modifiche al nodo di substrato (B):
\begin{itemize}
	\item al posto di trovarsi sotto, il substrato viene realizzato sulla parte superiore del transistor a lato del 
	source o del drain per facilitare il processo di fabbricazione
	\item in prossimità dell'elettrodo, il substrato viene ulteriormente drogato dello stesso tipo del substrato (p\(^+\) in
	n-mosfet o n\(^+\) in p-mosfet) per ridurre la resistenza di contatto
\end{itemize}
Di seguito sono riportate le illustrazioni delle strutture reali di un n-mosfet e di un p-mosfet:

\begin{center}
	\includegraphics[width=0.8\textwidth]{immagini/4_mosfet/mosfet_14.png}
\end{center}

\newpage

\subsection{Effetto Body e variazione della tensione di soglia}
Quando il substrato non è collegato al source, ma ad un potenziale di riferimento (massa o alimentazione), la tensione
tra substrato e source \(V_{BS}\) può essere diversa da zero. Questa tensione fa variare la tensione di soglia del mosfet
secondo le relazioni (in base al tipo di mosfet):
\begin{align*}
	V_{TN} &= V_{TN0} + \gamma_n \left( \sqrt{V_{SB} + 2\phi_n} - \sqrt{2\phi_n} \right) \qquad& \gamma_n &= \frac{\sqrt{2q N_D \varepsilon_{Si}}}{C_{OX}} & \phi_n &= \frac{k_B T}{q} \ln \left( \frac{N_A}{n_i} \right) \\
	V_{TP} &= V_{TP0} - \gamma_p \left( \sqrt{V_{BS} + 2\phi_p} - \sqrt{2\phi_p} \right) \qquad& \gamma_p &= \frac{\sqrt{2q N_A \varepsilon_{Si}}}{C_{OX}} & \phi_p &= \frac{k_B T}{q} \ln \left( \frac{N_D}{n_i} \right)
\end{align*}
\begin{itemize}
	\item \(V_{TN0}\) e \(V_{TP0}\): tensione di soglia per \(V_{SB} = 0\)
	\item \(\gamma\) e \(\phi\): parametri legati al drogaggio e allo spessore dell'ossido
	\item \(C_{OX}\): capacità per unità di area dell'ossido
\end{itemize}
Rappresentando graficamente la variazione della tensione di soglia in funzione di \(V_{SB}\), si ottiene:
\begin{center}
	\begin{minipage}{0.48\textwidth}
		\centering \includegraphics[width=0.7\textwidth]{immagini/4_mosfet/mosfet_15.png}

		\small{Variazione di \(V_{TN}\) per n-mosfet}
	\end{minipage}
	\begin{minipage}{0.48\textwidth}
		\centering \includegraphics[width=0.75\textwidth]{immagini/4_mosfet/mosfet_16.png}

		\small{Variazione di \(V_{TP}\) per p-mosfet}
	\end{minipage}
\end{center}

\vspace{10pt}

\subsection{Corrente di sottosoglia}
La corrente di sottosoglia è una piccola corrente che scorre tra drain e source anche quando il mosfet è in interdizione
(\(V_{GS} < V_{T}\)). Questa corrente è molto debole e vale:
\begin{align*}
	I_{DS} &= I_{0n} e^{\frac{V_{GS} - V_{TN}}{n V_T}} \left( 1 - e^{-\frac{V_{DS}}{n V_T}} \right) & \text{per n-mosfet} \\
	I_{DS} &= I_{0p} e^{- \frac{V_{GS} - V_{TP}}{n V_T}} \left( 1 - e^{\frac{V_{DS}}{n V_T}} \right) & \text{per p-mosfet}
\end{align*}
Analizzando la transcaratteristica per un n-mosfet si osserva che la corrente di sottosoglia decresce esponenzialmente per
\(V_{GS} < V_{TN}\) (siccome le ordinate sono in scala logaritmica, la curva appare lineare):
\begin{center}
	\includegraphics[width=0.6\textwidth]{immagini/4_mosfet/mosfet_20.png}
\end{center}

\newpage

\subsection{Capacità parassite dei mosfet}
\label{capacità_parassite_mosfet}
Avendo numerose giunzioni pn che si vengono a formare, i mosfet presentano delle capacità parassite che influenzano il loro
funzionamento alle alte frequenze, come evidenziato in figura. Le capacità parassite sono indipendenti dal tipo di mosfet
(n-mosfet o p-mosfet), per cui le conclusioni ottenute valgono sia per un n-mosfet che per un p-mosfet.

\begin{center}
	\centering \includegraphics[width=0.45\textwidth]{immagini/4_mosfet/mosfet_21.png}
\end{center}

Le capacità parassite per ogni coppia di terminali sono:
\begin{itemize}
	\item \(C_{sb}\): capacità source-substrato \(C_{sb} \approx C_{j0} \cdot L_D \cdot W\)
	\item \(C_{db}\): capacità drain-substrato \(C_{db} \approx C_{j0} \cdot L_D \cdot W\)
	\item \(C_{gs}\): capacità gate-source \(C_{gs} \approx C_{gs0} \cdot W\)
	\item \(C_{gd}\): capacità gate-drain \(C_{gd} \approx C_{gd0} \cdot W\)
	\item \(C_{gc}\): capacità gate-substrato (del condensatore mos) \(C_{gc} = C_{OX} \cdot L \cdot W\)
\end{itemize}

\vspace{10pt}

\noindent
Le capacità parassite complessive di ogni nodo rispetto al substrato si calcolano come capacità equivalente della rete tra
il nodo in esame e il substrato, ottenuta cortocircuitando tutti gli altri nodi al substrato. Di seguito sono riportate le
espressioni delle capacità parassite totali:
\begin{align*}
	C_{source} &= C_{sb} + C_{gs} = (C_{j0} L_D + C_{gs0}) \cdot W = C_{s0} \cdot W & & \text{con} \; C_{s0} = C_{j0} L_D + C_{gs0} \\
	C_{drain} &= C_{db} + C_{gd} = (C_{j0} L_D + C_{gd0}) \cdot W = C_{d0} \cdot W & & \text{con} \; C_{d0} = C_{j0} L_D + C_{gd0} \\
	C_{gate} &= C_{gs} + C_{gd} + C_{gc} = (C_{gs0} + C_{gd0} + C_{OX} L) \cdot W \approx C_{g0} \cdot W L & &\text{per} \; C_{gs} + C_{gd} \ll C_{OX}L
\end{align*}

\noindent
NOTA: Dall'analisi dei circuiti RC in cui si hanno due condensatori \(C_1\) e \(C_2\) con un terminale in comune e l'altro collegato
rispettivamente a massa e a \(V_{DD}\), si osserva che la capacità equivalente tra i due nodi è data dalla somma delle due
capacità \(C_{eq} = C_1 + C_2\). Questo avviene anche se effettivamente i due condensatori non sono collegati in parallelo,
in quanto la corrente che arriva al nodo comune si divide tra i due condensatori in funzione della loro capacità ed è come
se fossero collegati in parallelo.
