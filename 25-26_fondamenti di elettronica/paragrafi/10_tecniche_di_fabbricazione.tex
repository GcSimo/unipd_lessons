\section{Fabbricazione dei circuiti integrati}
\subsection{Fasi di progettazione e costruzione}
\subsubsection*{Fasi di progettazione}
La fase di progettazione è svolta dal progettista del circuito integrato che ha conoscenze approfondite di elettronica e del
funzionamento dei componenti da realizzare. Le fasi di progettazione sono:
\begin{enumerate}
	\item definizione delle specifiche del circuito integrato
	\item progettazione dello schema a blocchi
	\item progettazione dello schema circuitale
	\item progettazione del layout
\end{enumerate}

\subsubsection*{Fasi di costruzione}
Le fasi di costruzione sono svolte in un impianto di fabbricazione (fab) da tecnici specializzati che non necessariamente hanno
conoscenze approfondite in elettronica. Le fasi di costruzione sono:
\begin{center}
	\begin{minipage}{0.55\textwidth}
		\begin{enumerate}
			\item fabbricazione del wafer (metodo di Czochralski)
		\end{enumerate}
	\end{minipage}
	\begin{minipage}{0.4\textwidth} \(\) \end{minipage}
	\vspace{-5pt}

	\color[gray]{0.75}\rule{0.9\textwidth}{0.4pt}\color{black}
	
	\begin{minipage}{0.4\textwidth}
		\begin{enumerate}[start=2]
			\item selezione delle regioni attive
			\item selezione del tipo di substrato
			\item ossido di gate ed elettrodo di gate
			\item diffusioni N+ e P+
		\end{enumerate}
	\end{minipage}
	\begin{minipage}{0.55\textwidth}
		\(\left. \vcenter{\hbox{\huge\bfseries\vrule height 2.2cm width 0pt}} \right\} \text{Front End of the Line - FEOL}\)
	\end{minipage}
	
	\color[gray]{0.75}\rule{0.9\textwidth}{0.4pt}\color{black}

	\begin{minipage}{0.4\textwidth}
		\begin{enumerate}[start=6]
			\item contatti e interconnessioni
			\item packaging
		\end{enumerate}
	\end{minipage}
	\begin{minipage}{0.55\textwidth}
		\(\left. \vcenter{\hbox{\huge\bfseries\vrule height 1.2cm width 0pt}} \right\} \text{Back End of the Line - BEOL}\)
	\end{minipage}
\end{center}

\noindent
Nelle spiegazioni successive si analizzeranno le varie fasi di produzione prendendo come modello la costruzione di un
invertitore CMOS.

\subsection{Fabbricazione del Wafer con metodo di Czochralski}
Attraverso il metodo di Czochralski si ottiene un lingotto di silicio monocristallino che viene successivamente tagliato in
fette sottili (wafer). Il processo prevede i seguenti passaggi:
\begin{enumerate}
	\item si fonde del silicio ad alta purezza in un crogiolo di quarzo, è possibile aggiungere elementi chimici (ad esempio
	arsenico o fosforo) in maniera controllata per il drogaggio di base
	\item si immerge di un seme di silicio monocristallino (barra di silicio puro) nella massa fusa e viene messo in rotazione
	attorno al suo asse verticale
	\item il seme viene estratto lentamente in modo che gli atomi di silicio fuso si solidificano e si dispongono naturalmente
	secondo la struttura cristallina del seme (per una proprietà propria del silicio), la velocità di estrazione controlla il
	diametro del lingotto
	\item il lingotto viene tagliato in fette sottili (wafer) tramite una sega a filo abrasivo di diamante
	\item i wafer vengono sottoposti a processi di lappatura e molatura per ottenere una superficie perfettamente piana che viene
	ricoperta con uno strato sottile di ossido di silicio per proteggerla
\end{enumerate}
\begin{center}
	\begin{minipage}{0.3\textwidth}
		\centering \includegraphics[width=0.8\textwidth]{immagini/10_tecniche_di_fabbricazione/wafer_1.png} \\
		\small{inserimento del seme nel crogiolo}
	\end{minipage}
	\begin{minipage}{0.3\textwidth}
		\centering \includegraphics[width=0.8\textwidth]{immagini/10_tecniche_di_fabbricazione/wafer_2.png} \\
		\small{estrazione e formazione del lingotto}
	\end{minipage}
	\begin{minipage}{0.3\textwidth}
		\centering \includegraphics[width=0.8\textwidth]{immagini/10_tecniche_di_fabbricazione/wafer_3.png} \\
		\small{taglio del lingotto in wafer}
	\end{minipage}
\end{center}

\subsection{Processo selettivo di costruzione del circuito integrato per litografia}
\subsubsection*{Divisione di un wafer in die}
Ogni wafer viene suddiviso a scacchiera in tante aree quadrate chiamate die (o chip). Ogni die compone un circuito integrato
completo che alla fine della lavorazione verrà tagliato e confezionato singolarmente. La dimensione dei die dipende dallo
spazio occupato dal circuito integrato e dal numero di difetti critici presenti nel wafer.

Un difetto critico è un difetto che rende inutilizzabile il circuito integrato. La probabilità di avere un difetto critico
aumenta con l'aumentare della superficie del die. Per questo motivo, per aumentare la resa di produzione, si tende a ridurre
la dimensione dei die.

\subsubsection*{Maschera}
Per identificare le aree del wafer che devono essere lavorate in ogni fase del processo di costruzione, si utilizza una maschera.
La maschera è una lastra di quarzo trasparente su cui sono incisi i disegni delle aree che devono essere lavorate.

\subsubsection*{Processo selettivo per litografia}
Per selezionare effettivamente le aree del wafer da lavorare e proteggere le altre si utilizza un processo selettivo per
litografia che prevede i seguenti passaggi:

\begin{center}
	\begin{minipage} {0.35\textwidth}
		\centering \includegraphics[width=\textwidth]{immagini/10_tecniche_di_fabbricazione/litografia.png}
	\end{minipage}
	\begin{minipage} {0.6\textwidth}
		\begin{enumerate}
			\item si ricopre il wafer con uno strato sottile di materiale fotosensibile chiamato fotoresist
		\end{enumerate}
		\vspace{18pt}
		\begin{enumerate}[start=2]
			\item si espone il wafer alla luce ultravioletta attraverso la maschera, le aree del fotoresist esposte alla luce
			cambiano le loro proprietà chimiche e diventano solubili in un apposito solvente
		\end{enumerate}
		\vspace{20pt}
		\begin{enumerate}[start=3]
			\item si sciolgono le aree del fotoresist diventate solubili, lasciando scoperte le aree del wafer che devono essere
			lavorate
		\end{enumerate}
		\vspace{12pt}
		\begin{enumerate}[start=4]
			\item si ottiene così il pattern desiderato sul wafer in cui le aree scoperte possono essere lavorate e il resto
			della superficie è protetta dal fotoresist
		\end{enumerate}
		\vspace{15pt}
		\begin{enumerate}[start=5]
			\item si esegue la lavorazione desiderata (ad esempio l'ossidazione o l'impiantazione ionica)
		\end{enumerate}
		\vspace{30pt}
		\begin{enumerate}[start=6]
			\item si rimuove il fotoresist rimanente con un altro solvente apposito
		\end{enumerate}
		\vspace{25pt}
		\begin{enumerate}[start=7]
			\item si ottiene così il wafer con le aree lavorate secondo il pattern desiderato
		\end{enumerate}
	\end{minipage}
\end{center}

\newpage
\subsection{Selezione delle regioni attive}
Il processo di selezione delle regioni attive consiste nel definire le regioni del wafer in cui verranno realizzati i singoli
mosfet e inserire delle barriere di ossido per isolare elettricamente le varie regioni. Il processo prevede i seguenti passaggi:

\begin{center}
	\begin{minipage} {0.35\textwidth}
		\centering \includegraphics[width=\textwidth]{immagini/10_tecniche_di_fabbricazione/regioni_attive.png}
	\end{minipage}
	\begin{minipage} {0.6\textwidth}
		\begin{enumerate}
			\item si deposita uno strato di nitruro di silicio (Si\(_3\)N\(_4\)) e si utilizza il processo di litografia selettiva
			per selezionare le aree in cui vanno inserite le barriere di ossido isolante
		\end{enumerate}
		\vspace{5pt}
		\begin{enumerate}[start=2]
			\item si rimuove il nitruro di silicio, l'ossido di silicio e una parte del substrato di silicio nelle aree
			selezionate tramite un attacco chimico con acido; per scavare verticalmente (e non anche lateralmente) le cavità,
			il solvente viene ionizzato e nebulizzato così, sotto l'azione di un campo elettrico verticale, gli ioni
			vengono direzionati e colpiscono il	wafer solo verticalmente senza intaccare le pareti laterali; questo processo
			è chiamato Reactive Ion Etching (RIE)
		\end{enumerate}
		\vspace{5pt}
		\begin{enumerate}[start=3]
			\item si rimuove il photoresist rimanente e si deposita l'ossido di isolamento (di bassa qualità siccome deve solo
			fungere da isolante) nelle regioni scoperte dal nitruro (ovvero nelle cavità scavate con il RIE) formando le
			Shallow Trench Isolation (STI)
		\end{enumerate}
		\vspace{5pt}
		\begin{enumerate}[start=4]
			\item si rimuove il nitruro di silicio e lo strato di ossido di silicio rimanenti, in modo da liberare la superficie di
			silicio puro del wafer
		\end{enumerate}
	\end{minipage}
\end{center}

\subsection{Selezione del tipo di substrato}
Il processo di selezione del tipo di substrato consiste nel creare un substrato di tipo P o N a seconda del tipo di mosfet
che si vuole realizzare all'interno delle aree attive definite nel punto precedente. Il processo prevede i seguenti passaggi:

\begin{center}
	\begin{minipage} {0.35\textwidth}
		\centering \includegraphics[width=\textwidth]{immagini/10_tecniche_di_fabbricazione/substrato.png}
	\end{minipage}
	\begin{minipage} {0.6\textwidth}
		\begin{enumerate}
			\item si utilizza il processo di litografia selettiva per selezionare le regioni attive in cui si vuole creare un
			substrato di tipo P (per la realizzazione di mosfet NMOS) o di tipo N (per la realizzazione di mosfet PMOS)
		\end{enumerate}
		\vspace{22pt}
		\begin{enumerate}[start=2]
			\item si bombardano le regioni scoperte con ioni droganti (ad esempio arsenico per il substrato di tipo P o fosforo
			per il substrato di tipo N) che penetrano nel silicio; questo processo è chiamato impiantazione ionica
		\end{enumerate}
		\vspace{5pt}
		\begin{enumerate}[start=3]
			\item si riscalda il wafer (annealing) per distribuire il drogante e farlo disporre correttamente nella struttura
			cristallina del silicio in modo da formare il substrato di tipo desiderato; questo è chiamato attivazione del drogaggio
		\end{enumerate}
		\vspace{22pt}
		\begin{enumerate}[start=4]
			\item si rimuove il fotoresist rimanente e si ripete tutto il processo di impiantazione ionica e attivazione del
			drogaggio con annealing per creare le aree di substrato del tipo opposto necessarie per la realizzazione dei mosfet
			complementari
		\end{enumerate}
	\end{minipage}
\end{center}

\subsection{Ossido di gate ed elettrodo di gate}
La fase successiva nella realizzazione dei mosfet è deposizione dell'ossido di gate e dell'elettrodo di gate in polisilicio.
Il processo prevede i seguenti passaggi:

\begin{center}
	\begin{minipage} {0.35\textwidth}
		\centering \includegraphics[width=\textwidth]{immagini/10_tecniche_di_fabbricazione/gate.png}
	\end{minipage}
	\begin{minipage} {0.6\textwidth}
		\begin{enumerate}
			\item si deposita uno strato sottile di ossido di silicio (SiO\(_2\)) di alta qualità (costituirà il dielettrico
			dei condensatori dei gate dei mosfet) su tutta la superficie del wafer, non richiede un processo selettivo in quanto
			lo strato di ossido ha spessore trascurabile rispetto alle STI
			\item si deposita uno strato di polisilicio (silicio policristallino con proprietà metalliche) sopra l'ossido di
			silicio per formare l'elettrodo di gate dei mosfet
			\item si utilizza il processo di litografia selettiva per selezionare le aree in cui si vogliono formare i gate dei
			mosfet e si elimina il polisilicio nelle aree scoperte perché non selezionate tramite un attacco chimico
		\end{enumerate}
	\end{minipage}
\end{center}

\vspace{2cm}

\subsection{Diffusioni N+ e P+}
Le diffusioni N+ e P+ servono a formare le regioni di source, drain e body dei mosfet. Il processo prevede i seguenti passaggi:

\begin{center}
	\begin{minipage} {0.35\textwidth}
		\centering \includegraphics[width=\textwidth]{immagini/10_tecniche_di_fabbricazione/diffusioni.png}
	\end{minipage}
	\begin{minipage} {0.6\textwidth}
		\begin{enumerate}
			\item si selezionano le aree in cui eseguire le diffusioni N+ (source e drain degli NMOS, body dei PMOS) tramite il
			processo di litografia selettiva
			\end{enumerate}
		\vspace{5pt}
		\begin{enumerate}[start=2]
			\item si bombarda il silicio con ioni di arsenico o fosforo attraverso il processo di impiantazione ionica con
			attivazione del drogaggio per annealing utilizzando per creare le regioni N+ nelle aree scoperte, infine si rimuove
			il fotoresist rimanente
			\end{enumerate}
		\vspace{5pt}
		\begin{enumerate}[start=3]
			\item si ripete il processo analogo di selezione delle aree, impiantazione ionica e attivazione del drogaggio per
			creare le regioni P+ (source e drain dei PMOS, body degli NMOS)
		\end{enumerate}
	\end{minipage}
\end{center}

\newpage

\subsection{Contatti e interconnessioni}
Dopo aver creato i mosfet, è necessario creare i terminali elettrici e le interconnessioni tra i vari componenti del circuito
integrato. Il processo prevede i seguenti passaggi:

\begin{center}
	\begin{minipage} {0.35\textwidth}
		\centering \includegraphics[width=\textwidth]{immagini/10_tecniche_di_fabbricazione/contatti.png}
	\end{minipage}
	\begin{minipage} {0.6\textwidth}
		\begin{enumerate}
			\item si deposita uno strato di ossido isolante (arancione) sopra tutta la superficie del wafer per isolare
			la struttura dei mosfet dalle piste metalliche di interconnessione costruite sopra
		\end{enumerate}
		\vspace{5pt}
		\begin{enumerate}[start=2]
			\item si utilizza un processo di litografia selettiva per forare l'ossido e raggiungere le aree di silicio dei
			terminali dei mosfet (source, drain, body e gate) dove vanno creati i contatti;
			\end{enumerate}
		\vspace{5pt}
		\begin{enumerate}[start=3]
			\item si deposita uno strato di metallo (ad esempio alluminio o rame) che riempie i fori sempre attraverso un
			processo di litografia selettiva in modo da formare i terminali del mosfet
		\end{enumerate}
		\vspace{5pt}
		\begin{enumerate}[start=4]
			\item si riveste l'intera superficie con un altro strato di ossido isolante (giallo) per separare i vari livelli di
			interconnessione
		\end{enumerate}
		\vspace{5pt}
		\begin{enumerate}[start=5]
			\item si riesegue l'intero processo di foratura e deposizione del metallo e deposizione del dielettrico più volte
			per creare i vari livelli di interconnessione necessari a collegare tra loro i vari componenti del circuito
		\end{enumerate}
	\end{minipage}
\end{center}

\vspace{1cm}

\subsection{Packaging}
Il processo di packaging consiste nel preparare il die per l'utilizzo esterno al fine di poterlo collegare ad altri circuiti
(ad esempio su una scheda madre) e proteggerlo da agenti esterni. Il processo prevede i seguenti passaggi:
\begin{enumerate}
	\item si taglia il wafer in singoli die tramite una sega a filo abrasivo di diamante
	\item si monta ogni die in un contenitore protettivo (package) che può essere di plastica o metallo
	\item si collegano i terminali del die (detti pad) ai terminali esterni del package tramite fili sottili di oro
\end{enumerate}
Alcuni esempi di package sono illustrai in figura a destra

\begin{center}
	\begin{minipage}{0.4\textwidth}
		\centering \includegraphics[width=0.9\textwidth]{immagini/10_tecniche_di_fabbricazione/packaging.png}

		\small{interconnessioni tra i pad del die e i terminali del package}
	\end{minipage}
	\begin{minipage}{0.55\textwidth}
		\centering
		\begin{tabular}{c c c}
			\includegraphics[width=0.25\textwidth]{immagini/10_tecniche_di_fabbricazione/th.png} &
			\includegraphics[width=0.27\textwidth]{immagini/10_tecniche_di_fabbricazione/smd.png} &
			\includegraphics[width=0.27\textwidth]{immagini/10_tecniche_di_fabbricazione/bga.png}\\
			\small{Through-Hole} & \small{Surface-Mount} & \small{Ball Grid Array} \\
		\end{tabular}
	\end{minipage}
\end{center}

\newpage

\subsection{Layout, regole di layout e sviluppo delle maschere}
\subsubsection*{Definizione del layout}
Il layout è la rappresentazione grafica in scala del circuito integrato che mostra la disposizione spaziale dei vari componenti
con le loro dimensioni e le interconnessioni tra di essi. Il layout è composto da una serie di maschere che verranno utilizzate
nel processo di costruzione del circuito integrato.

Il layout viene realizzato dal progettista del circuito integrato e viene poi usato dal tecnico di fabbricazione nelle varie
fasi di costruzione del circuito integrato. Funge da linguaggio comune tra progettista e tecnico di fabbricazione.

Ad ogni maschera è associta una fase del processo di costruzione del circuito integrato:
\begin{itemize}
	\item active: definizione delle regioni attive
	\item p-well/n-well: selezione del tipo di substrato
	\item n-diff/p-diff: diffusioni N+ e P+
	\item polysilicon: ossido di gate ed elettrodo di gate
	\item contact: deposizione del metallo per i terminali dei mosfet
	\item metal1, metal2, ... : deposizione del metallo per i vari livelli di interconnessione
	\item via 1-2, via 2-3, ... : foratura del dielettrico tra i vari livelli di interconnessione
\end{itemize}
\begin{center}
	\includegraphics[width=0.9\textwidth]{immagini/10_tecniche_di_fabbricazione/layout.png}

	\small{esempio di layout (maschere) per la realizzazione di un inverter CMOS, come illustrato nelle fasi di fabbricazione
	precedenti, a sinistra le FEOL e a destra le BEOL}
\end{center}

\subsubsection*{Regole di layout}
Affinché il layout possa essere effettivamente utilizzato per la costruzione del circuito integrato, deve rispettare una serie
di regole per garantire la corretta realizzazione fisica del circuito integrato. Le regole di layout sono dovute a:
\begin{itemize}
	\item lminima risoluzione e tolleranza della fotolitografia
	\item inevitabile disallineamento delle maschere
	\item imprecisioni del processo
	\item buon senso del progettista
\end{itemize}
Le regole di layout si dividono in:
\begin{itemize}
	\item \textbf{intra-layer}: regole che riguardano una singola maschera, ovvero le dimensioni e le distanze minime tra gli
	elementi di una singola maschera (per rispettare la risoluzione e le tolleranze del processo)
	\item \textbf{inter-layer}: regole che riguardano più maschere, ovvero le distanze minime che gli elementi di maschere
	diverse devono rispettare tra di loro (per contenere problemi di disallineamento)
\end{itemize}
