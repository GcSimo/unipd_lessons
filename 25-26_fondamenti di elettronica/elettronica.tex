\documentclass[a4paper]{article}
\usepackage[utf8]{inputenc} % standard unicode
\usepackage[italian]{babel} % corretta sillabazione in italiano
\usepackage{geometry} % per impostare margini e layout pagina
\usepackage{amssymb} % per l'ambiente matematico
\usepackage{amsmath} % per l'ambiente matematico
\usepackage{enumitem} % per elenchi puntati
\usepackage{multirow} % per celle che si espandono su più righe
\usepackage{tabularx} % per tabelle con larghezza flessibile
\usepackage{booktabs} % per linee orizzontali tabelle
\usepackage{hyperref} % per collegamenti
\usepackage{graphicx} % per immagini
\usepackage{multicol} % per pagina in colonne
\usepackage{dirtytalk} % per le ""

% per margini
\geometry{a4paper,left=25mm, right=25mm, bottom=25mm, top=30mm}

% per centrare testo nelle tabelleX
\renewcommand\tabularxcolumn[1]{m{#1}}
\newcolumntype{L}{>{\raggedright\arraybackslash}X}
\newcolumntype{C}{>{\centering\arraybackslash}X}
\newcolumntype{R}{>{\raggedleft\arraybackslash}X}

% per elenchi puntati
\setlist[itemize]{label=-, partopsep=0pt, topsep=3pt, itemsep=0pt}

% percorso delle immagini da inserire
\graphicspath{{./}}


% --- altro che si può eliminare ---

\title{Appunti di Fondamenti di elettronica}
\author{Giacomo Simonetto}
\date{Primo semestre 2025-26}

\begin{document}

\maketitle
\begin{abstract}
	Appunti del corso di Fondamenti di elettronica della facoltà di Ingegneria Informatica dell'Università di Padova.
\end{abstract}

\newpage

\tableofcontents

\newpage

\section{Introduzione}
\subsection{Settori dell'elettronica}
\begin{itemize}
	\item \textbf{elettronica analogica}: progettazione e analisi di circuiti che elaborano segnali analogici;
	\item \textbf{elettronica digitale}: progettazione e analisi di circuiti che elaborano segnali digitali;
	\item \textbf{elettronica di consumo}: dispositivi elettronici per l'uso personale e domestico (computer, telefoni cellulari, televisori, elettrodomestici);
	\item \textbf{microelettronica}: progettazione e fabbricazione di componenti elettronici e circuiti integrati;
	\item \textbf{elettronica di potenza}: conversione e gestione dell'energia elettrica a diversi livelli (dal riscaldamento agli alimentatori per pc, cellulari o altri strumenti);
	\item \textbf{elettronica industriale}: sistemi elettronici per processi produttivi automatizzati;
	\item \textbf{telecomunicazioni}: sistemi per la trasmissione di dati (voce, video, file) attraverso dispositivi mobili o fissi;
	\item \textbf{biomedica}: sviluppo di apparecchiature elettroniche per la diagnostica, la cura e il monitoraggio della salute;
	\item \textbf{automotive}: sistemi per il controllo dei veicoli (dallo specchietto fino alla guida autonoma);
	\item \textbf{informatica}: dispositivi e sistemi elettronici per la gestione dei dati.
\end{itemize}

\subsection{Definizioni fondamentali}
\begin{itemize}
	\item \textbf{elettronica}: studia e realizza sistemi elettronici;
	\item \textbf{sistema elettronico}: è un insieme di componenti elettronici (sensori, circuiti e attuatori) che raccolgono
	informazioni dal mondo reale attraverso sensori, le elaborano attraverso circuiti elettronici e prendono decisioni o
	comandano azioni con degli attuatori;
	\item \textbf{segnale}: supporto fisico di natura qualunque (elettrica, acustica, ottica) a cui si associa un'informazione
	allo scopo di poterla trasferire da una sorgente ad un utilizzatore, può essere digitale (ampiezza e tempo discreti) o
	analogico (ampiezza e tempo continui);
	\item \textbf{sensore}: dispositivo che converte un segnale esterno in una grandezza elettrica (corrente o tensione);
	\item \textbf{circuito elettronico}: rete di componenti elettrici passivi (R, L, C) e attivi (diodi, transistor) per
	l'elaborazione di segnali elettrici (tensione e corrente). In base al tipo di segnale elaborato si distingue in:
	\begin{itemize}[topsep=0pt]
		\item \textbf{circuito analogico}: elabora segnali analogici;
		\item \textbf{circuito digitale}: elabora segnali digitali;
		\item \textbf{circuito misto}: opera in entrambi i domini del segnale.
	\end{itemize}
	Siccome i segnali provenienti dal mondo reale sono sempre analogici, in generale non esiste un sistema completamente digitale,
	per cui ogni sistema digitale prevede un ADC (Analog-to-Digital Converter) in ingresso e un DAC (Digital-to-Analog Converter)
	in uscita.

	In base alla realizzazione fisica si distingue in:
	\begin{itemize}
		\item \textbf{circuito a elementi discreti}: realizzato con componenti singoli collegati tra loro (breadboard, circuiti
		stampati);
		\item \textbf{circuito integrato (IC)}: realizzato con componenti miniaturizzati su un unico chip di silicio (circuiti
		integrati, microchip).
	\end{itemize}
	Un sistema elettronico completo è formato da circuiti integrati e componenti discreti montati in una scheda in cui sono
	realizzate le interconnessioni metalliche tra i terminali dei componenti
\end{itemize}


\section{Richiamo di teoria dei circuiti}


\section{Semiconduttori}
\subsection{Classificazione e proprietà elettriche dei semiconduttori}
\subsubsection*{Classificazione dei materiali}
In base alla resisitività elettrica \(\rho\) dei materiali, questi si dividono in:
\begin{itemize}
	\item conduttori: \(\rho < 10^{-3} \Omega \cdot \m\)
	\item semiconduttori: \(10^{-3} \Omega \cdot \m < \rho < 10^{5} \Omega \cdot \m\)
	\item isolanti: \(\rho > 10^{5} \Omega \cdot \m\)
\end{itemize}

\subsubsection*{Classificazione chimica dei semiconduttori}
I semiconduttori sono composti dagli elementi chimici di transizione, come il silicio (Si) e il germanio (Ge). In base alla loro
composizione chimica si classificano in:
\begin{itemize}
	\item \textbf{semiconduttori a elemento singolo}: se sono formati da un solo elemento chimico, come silicio (Si) e germanio (Ge)
	\item \textbf{semiconduttori composti}: se sono formati da più elementi, come arseniuro di gallio (GaAs), fosfuro di indio (InP),
	nitruro di gallio (GaN), tellururo di cadmio (CdTe), ...
	\item \textbf{semiconduttori intrinseci}: se sono puri, cioè non contengono impurità
\end{itemize}

\subsubsection*{Struttura del silicio}
Il silicio (Si) ha 14 protoni, 14 neutroni e 14 elettroni. Possiede 4 elettroni nel guscio più esterno, detti elettroni di valenza.
Tali elettroni sono quelli che partecipano alla formazione dei legami chimici e di conseguenza il silicio è in grado di formare
4 legami covalenti con altri atomi. In un cristallo di silicio, ogni atomo di silicio condivide i suoi 4 elettroni di valenza
con 4 atomi di silicio vicini, formando così una struttura cristallina tetraedrica e periodica (simile a quella del carbonio).

\subsubsection*{Conducibilità elettica nei metalli, negli isolanti e nei semiconduttori}
Un materiale conduce corrente elettrica se possiede elettroni liberi in grado di muoversi all'interno del reticolo cristallino.
La differenza tra metalli, isolanti e semiconduttori risiede nel comportamento degli elettroni di valenza:
\begin{itemize}
	\item nei metalli, gli elettroni di legame sono condivisi tra più atomi e formano una nube di elettroni liberi che si muovono
	liberamente all'interno del reticolo cristallino, permettendo così la conduzione elettrica
	\item negli isolanti, gli elettroni di valenza sono fortemente vincolati ai loro atomi e non possono muoversi liberamente
	all'interno del reticolo cristallino, per cui sono scarsi conduttori di elettricità
	\item nei semiconduttori si hanno comportamenti simili agli isolanti, però i legami tra gli atomi sono più deboli e con poca
	energia è possibile rompere tali legami liberando gli elettroni di valenza, che possono così condurre corrente elettrica
\end{itemize}
L'energia necessaria a liberare un elettrone di valenza è detta \textbf{energy gap} (o band gap) e varia a seconda del materiale.
Di seguito una tabella con i valori di energy gap per alcuni materiali comuni:

\begin{center}
	\begin{tabular}{l c | l c}
		\toprule
		Materiale & Energy gap & Materiale & Energy gap \\
		\midrule
		Silicio (Si) & 1.124 eV & Germanio (Ge) & 0.66 eV \\
		Arseniuro di gallio (GaAs) & 1.42 eV & Nitruro di gallio (GaN) & 3.4 eV \\
		Fosfuro di indio (InP) & 1.35 eV & Seleniuro di cadmio (CdSe) & 1.74 eV \\
		Stagno (Sn) & 0.082 eV & Rame (Cu) & \say{0} eV \\
		\bottomrule
	\end{tabular}
\end{center}

\subsubsection*{Elettroni liberi e lacune}
Quando un elettrone di valenza acquisisce sufficiente energia per liberarsi dal legame con il suo atomo, esso diventa un elettrone
libero in grado di muoversi liberamente all'interno del reticolo cristallino. Il legame incompleto per la mancanza di un elettrone
è detto lacuna (o hole in inglese) e si comporta come una carica positiva mobile all'interno del reticolo cristallino.

La lacuna può essere colmata da un elettrone di valenza di un atomo vicino, che a sua volta lascia una nuova lacuna. In questo modo,
la lacuna sembra muoversi all'interno del reticolo cristallino, permettendo così la conduzione elettrica.

Gli elettroni liberi e e le lacune sono detti portatori di carica negativa per gli elettroni, positiva per le lacune.

\subsubsection*{Concentrazione di elettorni e lacune in un semiconduttore intrinseco}
In un semiconduttore intrinseco, la concentrazione di elettroni liberi \(n\) e la concentrazione di lacune \(p\) sono uguali e
per definizione si indicano con \(n_i\):
\[n = p = n_i \qquad n \cdot p = {n_i}^2\]
La concentrazione di portatori di carica in un semiconduttore intrinseco dipende dalla temperatura \(T\) e dall'energy gap \(E_g\)
del materiale, secondo la seguente formula, dove \(B\) è una costante dipendente dal materiale e \(k_B\) è la costante di Boltzmann.
\[n_i = BT^{3/2} e^{-E_g/2k_BT}\]
Si osserva che la concentrazione di portatori di carica aumenta all'aumentare della temperatura. Nel silicio a temperatura
ambiente (300 K), la concentrazione di portatori di carica è circa \(n_i = 1.45 \times 10^{10} \, \cm^{-3}\).

\subsection{Drogaggio dei semiconduttori}
Il drogaggio di un semiconduttore consiste nell'aggiunta di impurità al semiconduttore intrinseco per modificarne le proprietà
elettriche, senza alterare la struttura del reticolo. Le impurità sono atomi di elementi chimici con un numero di elettroni di
valenza diverso da quello del semiconduttore intrinseco che produrranno un eccesso di elettroni liberi o di lacune, migliorando
la conducibilità elettrica del materiale.

\subsubsection*{Drogaggio di tipo n}
Il drogaggio di tipo n si ottiene aggiungendo al silicio intrinseco atomi di un elemento chimico con 5 elettroni di valenza,
come il fosforo (P), l'arsenico (As) o l'antimonio (Sb). Questi atomi, facendo 4 legami covalenti con gli atomi di silicio vicini,
hanno un elettrone in più che non può essere utilizzato per il legame e diventa un elettrone libero. Per questo motivo sono
detti donatori.

\subsubsection*{Drogaggio di tipo p}
Il drogaggio di tipo p si ottiene aggiungendo al silicio intrinseco atomi di un elemento chimico con 3 elettroni di valenza,
come il boro (B), l'alluminio (Al) o il gallio (Ga). Questi atomi fanno sempre 4 legami covalenti con gli atomi di silicio vicini,
ma avendo un elettrone in meno, creano una lacuna. Per questo motivo sono detti accettori.

\subsubsection*{Equilibrio termodinamico e legge di azione di massa}
In un semiconduttore si hanno due processi opposti che avvengono contemporaneamente:
\begin{itemize}
	\item la \textbf{generazione} di coppie elettrone-lacuna, per cui si formano un elettrone libero e una lacuna dovuti alla
	rottura di un legame covalente; si indica con \(G = f_1(T)\) il tasso di generazione, ovvero il numero di coppie
	elettrone-lacuna generate per unità di volume e di tempo (\([\cm^{-3} \s^{-1}]\)), che dipende dalla temperatura \(T\)
	\item la \textbf{ricombinazione} di coppie elettrone-lacuna, in cui un elettrone libero si ricombina con una lacuna; si indica
	con \(R = n \ccdot p \ccdot f_2(T)\) il tasso di ricombinazione, ovvero il numero di coppie elettrone-lacuna che si
	ricombinano per unità di volume e di tempo (\([\cm^{-3} \s^{-1}]\)) e dipende dalle concentrazioni di elettroni \(n\) e di
	lacune \(p\) e dalla temperatura \(T\)
\end{itemize}

Quando il semiconduttore si trova a temperatura costante e senza sollecitazioni esterne, raggiunge uno stato di
\textbf{equilibrio termodinamico} in cui:
\begin{itemize}
	\item la concentrazione di elettroni \(n\) e la concentrazione di lacune \(p\) rimangono costanti nel tempo
	\item la velocità di generazione di coppie elettrone-lacuna è uguale alla velocità di ricombinazione
\end{itemize}
Imponendo l'equilibrio tra generazione e ricombinazione si ottiene la \textbf{legge di azione di massa}:
\[G = R \;\;\rightarrow\;\; f_1(T) = n \cdot p \cdot f_2(T) \;\;\rightarrow\;\; n \cdot p = \frac{f_1(T)}{f_2(T)} \;\;\rightarrow\;\; {n_i}^2 = n \cdot p\]

\subsubsection*{Concentrazione dei portatori in un semiconduttore drogato di tipo n}
In un semiconduttore drogato esclusivamente di tipo n si hanno le seguenti particelle cariche:
\begin{itemize}
	\item elettroni liberi \(n\) (intrinseci e donati dagli atomi di impurità)
	\item lacune \(p\) (intrinseche)
	\item ioni donatori \(N_D^+\) (atomi di impurità che hanno ceduto un elettrone libero)
\end{itemize}
Siccome il drogaggio non altera la carica complessiva del semiconduttore, si ha:
\[-q n +q p + q N_D = 0 \;\;\rightarrow\;\; n - p - N_D = 0 \;\;\rightarrow\;\; n - {n_i}^2/n - N_D = 0 \;\;\rightarrow\;\; n^2 - n N_D - {n_i}^2 = 0\]
\[n = \frac{N_D \pm \sqrt{{N_D}^2 + 4{n_i}^2}}{2} = N_D \frac{1+\sqrt{1 + 4{n_i}^2/{N_D}^2}}{2} \quad \stackrel{N_D \gg n_i}{\longrightarrow} \quad n \approx N_D, \quad p = \frac{{n_i}^2}{N_D}\]
La concentrazione di elettroni \(n\) è maggiore della concentrazione di lacune \(p\), per cui gli elettroni si definiscono
\textbf{portatori maggioritari}, mentre le lacune si definiscono \textbf{portatori minoritari}.

\subsubsection*{Concentrazione dei portatori in un semiconduttore drogato di tipo p}
In un semiconduttore drogato esclusivamente di tipo p si hanno le seguenti particelle cariche:
\begin{itemize}
	\item elettroni liberi \(n\) (intrinseci)
	\item lacune \(p\) (intrinseche e dovute agli atomi di impurità)
	\item ioni accettori \(N_A^-\) (atomi di impurità che hanno accettato un elettrone libero)
\end{itemize}
Siccome il drogaggio non altera la carica complessiva del semiconduttore, si ha:
\[-q n +q p - q N_A = 0 \;\;\rightarrow\;\; p - n - N_A = 0 \;\;\rightarrow\;\; p - {n_i}^2/p - N_A = 0 \;\;\rightarrow\;\; p^2 - p N_A - {n_i}^2 = 0\]
\[p = \frac{N_A \pm \sqrt{{N_A}^2 + 4{n_i}^2}}{2} = N_A \frac{1+\sqrt{1 + 4{n_i}^2/{N_A}^2}}{2} \quad \stackrel{N_A \gg n_i}{\longrightarrow} \quad p \approx N_A, \quad n = \frac{{n_i}^2}{N_A}\]
La concentrazione di lacune \(p\) è maggiore della concentrazione di elettroni \(n\), per cui le lacune si definiscono
\textbf{portatori maggioritari}, mentre gli elettroni si definiscono \textbf{portatori minoritari}.

\subsubsection*{Concentrazione dei portatori in un semiconduttore drogato sia di tipo n che di tipo p}
In un semiconduttore drogato sia di tipo n che di tipo p si hanno le seguenti particelle cariche:
\begin{itemize}
	\item elettroni liberi \(n\) (intrinseci e donati dagli atomi di impurità di tipo n)
	\item lacune \(p\) (intrinseche e dovute agli atomi di impurità di tipo p)
	\item ioni donatori \(N_D^+\) (atomi di impurità di tipo n che hanno ceduto un elettrone libero)
	\item ioni accettori \(N_A^-\) (atomi di impurità di tipo p che hanno accettato un elettrone libero)
\end{itemize}
Siccome il drogaggio non altera la carica complessiva del semiconduttore, si ha:
\[-q n +q p + q N_D - q N_A = 0 \;\;\rightarrow\;\; n - p - N_D + N_A = 0\]
Se \(N_D > N_A\) si comporta come un semiconduttore di tipo n con drogaggio netto \(N_D' = N_D - N_A\):
\[n^2 - n (N_D - N_A) - {n_i}^2 = 0 \quad \stackrel{N_D - N_A \gg n_i}{\longrightarrow} \quad n \approx N_D - N_A, \quad p = \frac{{n_i}^2}{N_D - N_A}\]
Viceversa se \(N_A > N_D\) si comporta come un semiconduttore di tipo p con drogaggio netto \(N_A' = N_A - N_D\):
\[p^2 - p (N_A - N_D) - {n_i}^2 = 0 \quad \stackrel{N_A - N_D \gg n_i}{\longrightarrow} \quad p \approx N_A - N_D, \quad n = \frac{{n_i}^2}{N_A - N_D}\]

\subsection{Corrente elettrica e conducibilità nei semiconduttori}
\subsubsection*{Cause del moto dei portatori di carica}
Lo spostamento dei portatori di carica nei semiconduttori è influenzata da tre meccanismi fisici:
\begin{itemize}
	\item \textbf{temperatura}: l'aumento della temperatura provoca un aumento di energia interna (e di conseguenza di energia
	cinetica) dei portatori di carica, che si muovono più velocemente all'interno del reticolo cristallino
	\item \textbf{campo elettrico}: un campo elettrico esercita una forza sui portatori di carica che di conseguenza induce
	un movimento ordinato dei portatori di carica, essendo questi ultimi carichi elettricamente
	\item \textbf{gradiente di concentrazione}: una differenza di concentrazione di portatori di carica in due regioni del
	semiconduttore provoca un flusso di portatori dalla regione a concentrazione maggiore verso la regione a concentrazione
	minore, questo fenomeno è detto principio di diffusione
\end{itemize}

\subsubsection*{Moto dei portatori di carica dovuto alla sola energia cinetica / agitazione termica}
In assenza di campo elettrico e di gradiente di concentrazione, i portatori di carica si muovono casualmente all'interno
del reticolo cristallino a causa della loro energia cinetica. Questo moto casuale produce uno spostamento medio nullo,
poiché i portatori di carica si muovono in tutte le direzioni con uguale probabilità.

\subsubsection*{Moto dei portatori di carica dovuto al campo elettrico}
Quando si applica un campo elettrico \(E\) al semiconduttore, i portatori di carica subiscono una forza \(F_E\) dovuta al campo
elettrico che induce un movimento ordinato dei portatori di carica:
\begin{itemize}
	\item gli elettroni liberi, essendo cariche negative, si muovono in direzione opposta al campo elettrico
	\item le lacune, essendo cariche positive, si muovono in direzione del campo elettrico
\end{itemize}
Il moto ordinato dei portatori di carica dovuto al campo elettrico si sovrappone al moto casuale dovuto all'agitazione
termica, producendo uno spostamento medio non nullo dei portatori di carica nella direzione del campo elettrico (per le lacune)
o in direzione opposta al campo elettrico (per gli elettroni).

La velocità di deriva dei portatori dipende linearmente al campo elettrico (approssimando per campi non troppo elevati):
\[\begin{cases}
	v_n = - \mu_n \ccdot E & \text{per gli elettroni } \rightarrow \text{ moto opposto a } E \\
	v_p = \mu_p \ccdot E & \text{per le lacune } \rightarrow \text{ moto concorde a } E
\end{cases}\]
con \(v_n\) e \(v_p\) velocità di deriva degli elettroni e delle lacune (\([\cm/\s]\)), \(E\) il campo elettrico (\([\V/\cm]\))
e \(\mu_n\) e \(\mu_p\) le mobilità degli elettroni e delle lacune rispettivamente (\([\cm^2/\V\s]\)). In generale
\(\mu_n / \mu_p \approx 3\) siccome gli elettroni si muovono più facilmente delle lacune. Negli esercizi si assume
\(\mu_n = 1000 \; \cm^2/\V\s\) e \(\mu_p = 300 \; \cm^2/\V\s\) per il silicio intrinseco a temperatura ambiente.

\subsubsection*{Corrente di deriva}
Analizzando la quantità di carica che attraversa una sezione di area \(A\) in un intervallo di tempo \(dt\), si ottiene la corrente di deriva:
\begin{align*}
	\#_\text{elettroni} &= n \ccdot A \ccdot dx &\rightarrow \quad I_n &= \frac{-q \ccdot \#_\text{elettr.}}{dt} = \frac{-q \ccdot n \ccdot A \ccdot dx}{dt} = -q \ccdot n \ccdot A \ccdot v_n &\rightarrow \quad {j_n}^\text{drift} &= \frac{I_n}{A} = -q \ccdot n \ccdot v_n \\
	\#_\text{lacune} &= p \ccdot A \ccdot dx  &\rightarrow \quad I_p &= \frac{q \ccdot \#_\text{lacune}}{dt} = \frac{q \ccdot p \ccdot A \ccdot dx}{dt} = q \ccdot p \ccdot A \ccdot v_p &\rightarrow \quad {j_p}^\text{drift} &= \frac{I_p}{A} = q \ccdot p \ccdot v_p	
\end{align*}
\[{J_\text{tot}}^\text{drift} = J_n + J_p = -q \ccdot n \ccdot v_n + q \ccdot p \ccdot v_p = q (n \mu_n + p \mu_p) E\]
Si nota quindi che la densità di corrente di deriva totale \(J_\text{drift,tot}\) è proporzionale e concorde al campo elettrico
\(E\), per una costante di proporzionalità detta resistività elettrica \(\rho\):
\[{j_\text{tot}}^\text{drift} = \frac{E}{\rho}, \quad \rho = \frac{1}{q(n \mu_n + p \mu_p)} \qquad \begin{cases}
	{j_\text{tot}}^\text{drift} = q N_D \mu_n E, \quad \rho = 1 / q N_D \mu_n & \text{se drogato di tipo n} \\
	{j_\text{tot}}^\text{drift} = q N_A \mu_p E, \quad \rho = 1 / q N_A \mu_p & \text{se drogato di tipo p} \\
\end{cases}\]

\subsubsection*{Moto dei portatori di carica dovuto al gradiente di concentrazione}
Quando si mettono in contatto due regioni di un semiconduttore con diversa concentrazione di portatori di carica, si crea
un gradiente di concentrazione che induce un flusso di portatori dalla regione a concentrazione maggiore verso la regione
a concentrazione minore. Questo fenomeno è detto principio di diffusione e si verifica in natura per tutte le particelle
libere di muoversi.

Per analizzare il moto dei portatori di carica dovuto al gradiente di concentrazione, si definisce il flusso per unità di areaù
in un intervallo di tempo \(dt\), misurato in \([\cm^{-2}\s^{-1}]\):
\[\phi(x) = - D \frac{dC(x)}{dx} \qquad \phi(x)_n = - D_n \frac{dn(x)}{dx} \qquad \phi(x)_p = - D_p \frac{dp(x)}{dx}\]
con \(C(x)\) la concentrazione di particelle in funzione della posizione \(x\) (\([\cm^{-3}]\)) e \(D\) il coefficiente
di diffusione (\([\cm^2/\s]\)). Il segno \say{\(-\)} indica che il flusso avviene in direzione opposta al gradiente.

\subsubsection*{Corrente di diffusione}
Analizzando la quantità di carica associata al flusso dei portatori si ottiene la corrente di diffusione:
\[{j_n}^\text{diff} = -q \phi(x)_n = + q D_n \frac{dn(x)}{dx} \qquad {j_p}^\text{diff} = + q \phi(x)_p = - q D_p \frac{dp(x)}{dx} \qquad [\C\,\cm^{-2}\s^{-1}] = [\A\,\cm^{-2}]\]
Si osserva che il verso della la densità di corrente di diffusione dipende dal tipo di portatore:
\begin{itemize}
	\item la corrente degli elettroni ha lo stesso verso della concentrazione (cariche negative)
	\item la corrente delle lacune ha verso opposto alla concentrazione (cariche positive)
\end{itemize}

\subsubsection*{Corrente totale in un semiconduttore}
La corrente totale in un semiconduttore è data dalla somma della corrente di deriva e della corrente di diffusione degli
elettroni e delle lacune:
\[j_n = {j_n}^\text{drift} + {j_n}^\text{diff} = -q n \mu_n E + q D_n \frac{dn(x)}{dx} \qquad\qquad j_p = {j_p}^\text{drift} + {j_p}^\text{diff} = q p \mu_p E - q D_p \frac{dp(x)}{dx}\]
Le costanti \(D_n\), \(D_p\), \(\mu_n\), \(\mu_p\) sono correlate tra loro dalla relazione di Einstein:
\[\frac{D_n}{\mu_n} = \frac{D_p}{\mu_p} = \frac{k_B T}{q} = V_T\]
dove \(k_B\) è la costante di Boltzmann, \(T\) la temperatura assoluta, \(q\) la carica elementare e \(V_T\) è il potenziale
termico. Si ottiene quindi la seguente espressione per la corrente totale:
\[j_n = q \mu_n \left(n E + V_T \frac{dn(x)}{dx}\right) \qquad\qquad j_p = q\mu_p \left(p E - V_T \frac{dp(x)}{dx}\right)\]

\subsection{Semiconduttori in equilibrio}
\subsubsection*{Correnti in un semiconduttore in equilibrio}
Quando un semiconduttore si trova in equilibrio termodinamico, la somma delle correnti di deriva e di diffusione
per ciascun tipo di portatore di carica è nulla (che implica corrente totale nulla):
\[j_n = {j_n}^\text{drift} + {j_n}^\text{diff} = 0 \qquad\qquad j_p = {j_p}^\text{drift} + {j_p}^\text{diff} = 0 \qquad \Rightarrow \qquad j_\text{tot} = j_n + j_p = 0\]

\subsubsection*{Gradienti di potenziale e concentrazione}
Se in un semiconduttore in equilibrio si ha un gradiente di concentrazione di portatori di carica (\(n_2 - n_1\) per gli elettroni
o \(p_2 - p_1\) per le lacune), si deve necessariamente avere un gradiente di potenziale \(v_2-v_1\) tale da bilanciare la corrente
di diffusione con la corrente di deriva, in modo che la corrente totale sia nulla. Si ottiene quindi la seguente relazione:
\[\frac{n_2}{n_1} = \frac{p_1}{p_2} = e^{\tfrac{(v_2-v_1)}{V_T}}\]
Viceversa se in un semiconduttore in equilibrio si ha un gradiente di potenziale \(v_2 - v_1\), si deve necessariamente avere un
gradiente di concentrazione di portatori di carica tale da bilanciare la corrente di deriva con la corrente di diffusione:
\[\frac{v_2-v_1}{V_T} = \ln\left(\frac{n_2}{n_1}\right) = \ln\left(\frac{p_1}{p_2}\right)\]
Le due relazioni precedenti sono equivalenti e si ottengono imponendo la condizione di equilibrio \(j_n = 0\) e \(j_p = 0\),
utilizzando \(E = - dV / dx\) e risolvendo le due equazioni differenziali ottenute.

\subsubsection*{Considerazioni sulle correnti, gradienti e potenziali in equilibrio}
In un semiconduttore in equilibrio termodinamico in cui è presente un gradiente di concentrazione di portatori di carica e un
gradiente di potenziale, valgono le seguenti considerazioni:

\begin{itemize}
	\item[1. -] \({j_n}^\text{drift}\) ha verso opposto al gradiente di potenziale e concorde con il campo elettrico
	\item \({j_p}^\text{drift}\) ha verso opposto al gradiente di potenziale e concorde con il campo elettrico
	\item i gradienti di potenziale per gli elettroni e per le lacune sono uguali tra loro, per cui \(j^\text{drift}\) ha
	complessivamente verso opposto al gradiente di potenziale e concorde con il campo elettrico
	\item[2. -] \({j_n}^\text{diff}\) ha verso concorde con il gradiente di concentrazione degli elettroni
	\item \({j_p}^\text{diff}\) ha verso opposto al gradiente di concentrazione delle lacune
	\item i gradienti di concentrazione degli elettroni e delle lacune sono opposti tra loro (per la legge di azione di massa),
	per cui \(j^\text{diff}\) ha complessivamente verso concorde al gradiente di concentrazione degli elettroni e verso opposto
	al gradiente di concentrazione delle lacune
	\item[3. -] siccome la corrente totale deve essere nulla, la corrente di deriva e la corrente di diffusione devono avere verso
	opposto da cui si conclude che:
	\item il gradiente di potenziale e il gradiente di concentrazione degli elettroni hanno lo stesso verso
	\item il gradiente di potenziale e il gradiente di concentrazione delle lacune hanno verso opposto
\end{itemize}

\begin{center}
	\includegraphics[width=\textwidth]{immagini/2_semiconduttori/correnti_equilibrio.png}
\end{center}

\include{paragrafi/2_diodo}
\include{paragrafi/3_mosfet}

\end{document}
