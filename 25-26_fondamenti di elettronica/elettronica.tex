\documentclass[a4paper]{article}
\usepackage[utf8]{inputenc} % standard unicode
\usepackage[italian]{babel} % corretta sillabazione in italiano
\usepackage{geometry} % per impostare margini e layout pagina
\usepackage{amssymb} % per l'ambiente matematico
\usepackage{amsmath} % per l'ambiente matematico
\usepackage{enumitem} % per elenchi puntati
\usepackage{multirow} % per celle che si espandono su più righe
\usepackage{tabularx} % per tabelle con larghezza flessibile
\usepackage{booktabs} % per linee orizzontali tabelle
\usepackage{hyperref} % per collegamenti
\usepackage{graphicx} % per immagini
\usepackage{multicol} % per pagina in colonne
\usepackage{dirtytalk} % per le ""
\usepackage{circuitikz} % per disegni di circuiti elettrici
\usepackage{calc} % per calcoli con le coordinate

% per margini
\geometry{a4paper,left=25mm, right=25mm, bottom=25mm, top=30mm}

% per centrare testo nelle tabelleX
\renewcommand\tabularxcolumn[1]{m{#1}}
\newcolumntype{L}{>{\raggedright\arraybackslash}X}
\newcolumntype{C}{>{\centering\arraybackslash}X}
\newcolumntype{R}{>{\raggedleft\arraybackslash}X}

% per elenchi puntati
\setlist[itemize]{label=-, partopsep=0pt, topsep=3pt, itemsep=0pt}
\setlist[enumerate]{partopsep=0pt, topsep=3pt, itemsep=0pt}

% percorso delle immagini da inserire
\graphicspath{{./}}

% unità di misura
\newcommand\m{\text{m}} % metri
\newcommand\cm{\text{cm}} % centimetri
\newcommand\mm{\text{mm}} % millimetri
\newcommand\nm{\text{nm}} % nanometri
\newcommand\s{\text{s}} % secondi
\newcommand\ps{\text{ps}} % picosecondi
\newcommand\V{\text{V}} % volt
\newcommand\A{\text{A}} % ampere
\newcommand\C{\text{C}} % coulomb
\newcommand\F{\text{F}} % farad
\newcommand\J{\text{J}} % joule
\newcommand\W{\text{W}} % watt

% cdot compatto
\newcommand\ccdot{\!\cdot\!}

% --- altro che si può eliminare ---


\title{Appunti di Fondamenti di elettronica}
\author{Giacomo Simonetto}
\date{Primo semestre 2025-26}

\begin{document}

\maketitle
\begin{abstract}
	Appunti del corso di Fondamenti di elettronica della facoltà di Ingegneria Informatica dell'Università di Padova.
\end{abstract}

\newpage

\tableofcontents

\newpage

%\section{Introduzione}

\subsection{Definizioni fondamentali}
\begin{itemize}
	\item \textbf{elettronica}: studia e realizza sistemi elettronici;
	\item \textbf{sistema elettronico}: è un insieme di componenti elettronici (sensori, circuiti e attuatori) che raccolgono
	informazioni dal mondo reale attraverso sensori, le elaborano attraverso circuiti elettronici e prendono decisioni o
	comandano azioni con degli attuatori;
	\item \textbf{segnale}: supporto fisico di natura qualunque (elettrica, acustica, ottica) a cui si associa un'informazione
	allo scopo di poterla trasferire da una sorgente ad un utilizzatore, può essere digitale (ampiezza e tempo discreti) o
	analogico (ampiezza e tempo continui);
	\item \textbf{sensore}: dispositivo che converte un segnale esterno (come temperatura, pressione, luce, suono) in una grandezza
	elettrica (come corrente o tensione);
	\item \textbf{circuito elettronico}: rete di componenti elettrici passivi (R, L, C) e attivi (diodi, transistor) che elaborano
	segnali elettrici (tensione e corrente). In base al tipo di segnale elaborato si distingue in:
	\begin{itemize}[topsep=0pt]
		\item \textbf{circuito analogico}: elabora segnali analogici;
		\item \textbf{circuito digitale}: elabora segnali digitali;
		\item \textbf{circuito misto}: opera in entrambi i domini del segnale.
	\end{itemize}
	Siccome i segnali provenienti dal mondo reale sono sempre analogici, in generale non esiste un sistema completamente digitale.
	Ogni sistema digitale, infatti, comprende un ADC (Analog-to-Digital Converter) in ingresso e un DAC (Digital-to-Analog Converter)
	in uscita.

	In base alla realizzazione fisica, un circuito elettronico si distingue in:
	\begin{itemize}[topsep=0pt]
		\item \textbf{circuito a elementi discreti}: realizzato con componenti costruiti separatamente che poi vengono montati su
		un supporto (breadboard, PCB) e collegati tra loro tramite fili o piste conduttive;
		\item \textbf{circuito integrato (IC)}: tutti i componenti sono miniaturizzati e vengono montati su un unico chip di silicio
		(es. microchip).
	\end{itemize}
	Un sistema elettronico completo è formato da circuiti integrati e componenti discreti montati in una scheda in cui sono
	realizzate le interconnessioni metalliche tra i terminali dei componenti
\end{itemize}

\subsection{Settori dell'elettronica}
\begin{itemize}
	\item \textbf{elettronica analogica}: progettazione e analisi di circuiti che elaborano segnali analogici;
	\item \textbf{elettronica digitale}: progettazione e analisi di circuiti che elaborano segnali digitali;
	\item \textbf{elettronica di consumo}: dispositivi elettronici per l'uso personale e domestico (computer, telefoni cellulari, televisori, elettrodomestici);
	\item \textbf{microelettronica}: progettazione e fabbricazione di componenti elettronici e circuiti integrati;
	\item \textbf{elettronica di potenza}: conversione e gestione dell'energia elettrica a diversi livelli (dal riscaldamento agli alimentatori per pc, cellulari o altri strumenti);
	\item \textbf{elettronica industriale}: sistemi elettronici per processi produttivi automatizzati;
	\item \textbf{telecomunicazioni}: sistemi per la trasmissione di dati (voce, video, file) attraverso dispositivi mobili o fissi;
	\item \textbf{biomedica}: sviluppo di apparecchiature elettroniche per la diagnostica, la cura e il monitoraggio della salute;
	\item \textbf{automotive}: sistemi per il controllo dei veicoli (dallo specchietto fino alla guida autonoma);
	\item \textbf{informatica}: dispositivi e sistemi elettronici per la gestione dei dati.
\end{itemize}

\newpage


\subsection{Richiamo di teoria dei circuiti}
\subsubsection*{Leggi di Kirchoff}
\begin{itemize}
	\item \textbf{Legge delle correnti (LKC)}: la somma delle correnti entranti in un nodo è uguale alla somma delle correnti uscenti.
	\item \textbf{Legge delle tensioni (LKT)}: la somma delle tensioni lungo una maglia è uguale a zero.
\end{itemize}

\subsubsection*{Elementi passivi}
\begin{center}
	\begin{tabular}{l c c c }
		\textbf{resistore (R)}: & \begin{circuitikz}[baseline=(current bounding box.center)]
			\draw (0,0) to[R] (2,0);
			\draw [->] (0.3,0.4) -- (1.7,0.4) node[midway, above] {\(i_R\)};
			\node at (0.3,-0.4) {+};
			\node at (1,-0.4) {\(v_R\)};
			\node at (1.7,-0.4) {--};
		\end{circuitikz} & \(\displaystyle \qquad v_R(t) = R \cdot i_R(t)\) & \(\displaystyle i_R(t) = \frac{v_R(t)}{R}\) \\
		\midrule
		
		\textbf{condensatore (C)}: & \begin{circuitikz}[baseline=(current bounding box.center)]
			\draw (0,0) to[C] (2,0);
			\draw [->] (0.2,0.5) -- (1.8,0.5) node[midway, above] {\(i_C\)};
			\node at (0.3,-0.6) {+};
			\node at (1,-0.7) {\(v_C\)};
			\node at (1.7,-0.6) {--};
		\end{circuitikz} & \(\displaystyle \qquad i_C(t) = C \frac{d v_C(t)}{d t}\) & \(\displaystyle v_C(t) = \frac{1}{C} \int_0^t i_C(t) \, dt + \frac{I_0}{C}\) \\
		\midrule
		
		\textbf{induttore (L)}: & \begin{circuitikz}[baseline=(current bounding box.center)]
			\draw (0,0) to[L] (2,0);
			\draw [->] (0.3,0.4) -- (1.7,0.4) node[midway, above] {\(i_L\)};
			\node at (0.3,-0.4) {+};
			\node at (1,-0.4) {\(v_L\)};
			\node at (1.7,-0.4) {--};
		\end{circuitikz} & \(\displaystyle \qquad v_L(t) = L \frac{d i_L(t)}{d t}\) & \(\displaystyle i_L(t) = \frac{1}{L} \int_0^t v_L(t) \,dt + \frac{V_0}{L}\)
	\end{tabular}
\end{center}

\subsubsection*{Elementi attivi}
\begin{center}
	\begin{tabular}{p{3.5cm} c p{8.6cm}}
		\textbf{generatore ideale di tensione} (GIT): &
		\begin{circuitikz}[baseline=(current bounding box.center), american voltages, american currents] \draw (0,0) to[V] (2,0); \end{circuitikz} &
		fornisce una tensione costante indipendentemente dalla corrente che lo attraversa \\[0.7cm]
		\midrule
		\textbf{generatore ideale di corrente} (GIC): &
		\begin{circuitikz}[baseline=(current bounding box.center), american voltages, american currents] \draw (0,0) to[I] (2,0); \end{circuitikz} &
		fornisce una corrente costante indipendentemente dalla tensione ai suoi capi \\[0.7cm]
		\midrule
		\textbf{diodi e transistor}: & \dots & componenti non lineari che verranno studiati successivamente.
	\end{tabular}
\end{center}

\subsubsection*{Principi di analisi dei circuiti}
\begin{itemize}
	\item \textbf{partitore di tensione}: due resistori in serie dividono la tensione in ingresso \(V_{in}\) in due tensioni
	\(V_{1}\) e \(V_{2}\) direttamente proporzionali alle resistenze (inversamente proporzionali alle conduttanze):
	\[ V_{1} = V_{in} \cdot \frac{R_1}{R_1 + R_2} \qquad\qquad V_{2} = V_{in} \cdot \frac{R_2}{R_1 + R_2}\]
	\item \textbf{partitore di corrente}: due resistori in parallelo dividono la corrente in ingresso \(I_{in}\) in due correnti
	\(I_{1}\) e \(I_{2}\) inversamente proporzionali alle resistenze (direttamente proporzionali alle conduttanze):
	\[ I_{1} = I_{in} \cdot \frac{R_2}{R_1 + R_2} \qquad\qquad I_{2} = I_{in} \cdot \frac{R_1}{R_1 + R_2}\]
	\item \textbf{sovrapposizione degli effetti}: dato un sistema lineare con \(C_1\), \(C_2\) possibili ingressi ed \(E_1\),
	\(E_2\) effetti prodotti in uscita dai due ingressi, se il sistema viene perturbato con un ingresso dato dalla composizione
	lineare dei due ingressi \(C = p_1 C_1 + p_2 C_2\) con \(p_1\) e \(p_2\) pesi dei due ingressi, l'effetto risultante in uscita
	sarà la composizione lineare dei due effetti \(E = p_1 E_1 + p_2 E_2\).

	In particolare in un circuito lineare con più generatori, la risposta (tensione o corrente) in un componente è uguale alla
	somma algebrica delle risposte dovute a ciascun generatore preso singolarmente, con gli altri generatori sostituiti dai loro
	rispettivi cortocircuiti (generatore di tensione ideale) o circuiti aperti (generatore di corrente ideale).
\end{itemize}

\subsubsection*{Potenziali, tensioni e nodi di riferimento}
\begin{itemize}
	\item Il potenziale elettrico è definito a meno di una costante, per cui anche la soluzione di una rete elettrica (data dai
	potenziali ai vari nodi) non è univoca, ma è definita a meno di una costante.
	\item Per rendere univoca la soluzione, si sceglie un nodo di riferimento a cui si assegna potenziale nullo e si calcolano
	i potenziali degli altri nodi rispetto a tale nodo.
	\item Le tensioni, invece, sono sempre definite univocamente come differenze di potenziale tra due nodi.
\end{itemize}

\vspace{0.2cm}\noindent
Esistono tre tipi di nodi di riferimento comunemente usati:
\begin{center}
	\begin{tabular}{m{3cm} c m{10.1cm}}
		nodo con potenziale di riferimento: &
		\begin{circuitikz} \draw (0, 0) node[sground]{}; \end{circuitikz} &
		nodo con il potenziale di riferimento scelto arbitrariamente a \(0V\) \\[0.3cm]
		\midrule
		nodo di massa: &
		\begin{circuitikz} \draw (0, 0) node[cground, scale=0.8]{}; \end{circuitikz}&
		nodo collegato al telaio metallico del dispositivo elettronico \\[0.3cm]
		\midrule
		nodo di terra: &
		\begin{circuitikz} \draw (0, 0) node[ground]{}; \end{circuitikz} &
		nodo collegato fisicamente alla terra tramite un conduttore metallico per motivi di sicurezza; di solito coincide con il nodo
		di massa; il potenziale di terra è molto stabile e indipendentemente dalle correnti che gli elettrodomestici prelevano o
		immettono in esso
	\end{tabular}
\end{center}

\subsubsection*{Rappresentazione elettronica di un circuito}
Un circuito elettronico può essere rappresentato in due modi equivalenti:
\begin{itemize}
	\item \textbf{notazione a maglie}: rappresentazione del circuito in maglie e nodi
	\item \textbf{notazione elettronica}: scelto il nodo di riferimento, tutti i terminali collegati a tale nodo sono marcati con il
	simbolo del nodo di riferimento, inoltre i nodi di cui si conosce già il potenziale (ad esempio quelli collegati a generatori di
	tensione ideali) sono marcati con il loro valore di potenziale.
\end{itemize}

\begin{center}
	\begin{minipage}{0.4\textwidth}
		\centering
		\begin{circuitikz}[american]
			% Definisco i nodi di riferimento (per chiarezza e per replicare la struttura)
			\coordinate (A) at (0.5, 3);   % Nodo superiore sinistro (sopra Vs)
			\coordinate (B) at (2, 3);   % Nodo superiore sopra R1
			\coordinate (C) at (4, 3);   % Nodo superiore sopra i_S
			\coordinate (D) at (2, 1.5); % Nodo centrale sotto R1
			\coordinate (E) at (4, 1.5); % Nodo centrale sotto i_S
			\coordinate (F) at (4, 0);   % Nodo inferiore destro (sotto R3)
			\coordinate (G) at (0.5, 0);   % Nodo inferiore sinistro (sotto Vs)
			\coordinate (H) at (2, 0);   % Nodo inferiore sotto C
	
			% Disegno il circuito
			\draw (A) to [V, l_=\(V_S\)] (G); % Generatore di tensione Vs
			\draw (G) -- (F); % Linea inferiore
			\draw (A) -- (C); % Linea superiore
			\draw (B) to [R, l_=\(R_1\), *-*] (D); % Resistenza R1
			\draw (D) to [C, *-*] (H); % Condensatore C
			\draw (C) to [I, l_=\(I_S\)] (E); % Generatore di corrente IS
			\draw (D) to [R, l_=\(R_2\), *-*] (E); % Resistenza R2
			\draw (E) to [R, l_=\(R_3\)] (F); % Resistenza R3
		\end{circuitikz}
	\end{minipage}
	\begin{minipage}{0.05\textwidth}
		\centering
		\(\longrightarrow\)
	\end{minipage}
	\begin{minipage}{0.3\textwidth}
		\centering
		\begin{circuitikz}[american]
			% Definisco i nodi di riferimento (per chiarezza e per replicare la struttura)
			\coordinate (B) at (2, 3);   % Nodo superiore sopra R1
			\coordinate (C) at (4, 3);   % Nodo superiore sopra i_S
			\coordinate (D) at (2, 1.5); % Nodo centrale sotto R1
			\coordinate (E) at (4, 1.5); % Nodo centrale sotto i_S
			\coordinate (F) at (4, 0);   % Nodo inferiore destro (sotto R3)
			\coordinate (H) at (2, 0);   % Nodo inferiore sotto C
	
			% Disegno il circuito
			\draw (B) to [R, l_=\(R_1\)] (D); % Resistenza R1
			\draw (D) to [C] (H); % Condensatore C
			\draw (C) to [I, l_=\(I_S\)] (E); % Generatore di corrente IS
			\draw (D) to [R, l_=\(R_2\), *-*] (E); % Resistenza R2
			\draw (E) to [R, l_=\(R_3\)] (F); % Resistenza R3
	
			% Nodo di riferimento a massa
			\draw (H) node[sground]{};
			\draw (F) node[sground]{};
			\draw (B) node[rground, rotate=180]{};
			\draw (C) node[rground, rotate=180]{};
			\node at (2, 3.7) {\(V_S\)};
			\node at (4, 3.7) {\(V_S\)};
		\end{circuitikz}
	\end{minipage}
\end{center}

\subsubsection*{Potenza ed energia}
Per definire la potenza e l'energia consumata da un componente, si definiscono:
\begin{itemize}
	\item \textbf{convenzione degli utilizzatori}: la corrente entra nel terminale positivo della tensione, se la potenza o l'energia è positiva, il componente assorbe energia.
	\item \textbf{convenzione dei produttori}: la corrente entra nel terminale negativo della tensione, se la potenza o l'energia erogata è positiva, il componente fornisce energia.
\end{itemize}
Si definiscono quindi:
\begin{itemize}
	\item \textbf{potenza istantanea}: \(\displaystyle p(t) = v(t) \cdot i(t)\) misurata in Watt [W] \(=\) [J/sec]
	\item \textbf{energia}: \(\displaystyle E = \int_{t_1}^{t_2} p(t) \, dt = \int_{t_1}^{t_2} v(t) \cdot i(t) \, dt\) misurata in Joule [J] \(=\) [W \(\cdot\) sec]
\end{itemize}

\newpage

\subsection{Reti in regime transitorio}
\subsubsection*{Introduzione}
\begin{itemize}
	\item una rete si dice in regime transitorio quando le variabili elettriche (tensione e corrente) variano nel tempo passando da uno stato iniziale a uno stato finale di equilibrio
	\item un esempio di reti in regime transitorio sono i circuiti in cui sono presenti componenti reattivi (condensatori e induttori) e interruttori che modificano la configurazione del circuito
	\item il transitorio è l'intervallo di tempo che impiegano le variabili elettriche per passare dallo stato iniziale allo stato finale di equilibrio
\end{itemize}

\subsubsection*{Componenti reattivi}
\begin{center}
	\begin{tabular}{l c c c }
		componente & schema & in regime stazionario & in regime transitorio \\
		\toprule
		\textbf{condensatore}: & \begin{circuitikz}[baseline=(current bounding box.center)]
			\draw (0,0) to[C] (2,0);
			\draw [->] (0.2,0.5) -- (1.8,0.5) node[midway, above] {\(i_C\)};
			\node at (0.3,-0.6) {+};
			\node at (1,-0.7) {\(v_C\)};
			\node at (1.7,-0.6) {--};
		\end{circuitikz} & circuito aperto \(i_C = 0\) & \(\displaystyle i_C(t) = C \frac{d v_C(t)}{d t}\) \\
		\midrule
		\textbf{induttore (L)}: & \begin{circuitikz}[baseline=(current bounding box.center)]
			\draw (0,0) to[L] (2,0);
			\draw [->] (0.3,0.4) -- (1.7,0.4) node[midway, above] {\(i_L\)};
			\node at (0.3,-0.4) {+};
			\node at (1,-0.4) {\(v_L\)};
			\node at (1.7,-0.4) {--};
		\end{circuitikz} & cortocircuito \(v_L = 0\) & \(\displaystyle v_L(t) = L \frac{d i_L(t)}{d t}\)
	\end{tabular}
\end{center}


\subsubsection*{Carica di un condensatore}
\begin{itemize}
	\item condizioni iniziali (\(t < 0\)): interruttore inizialmente aperto \(\quad \rightarrow \quad v_C(0) = 0, \quad i_C(0) = 0\)
	\item nel transitorio (\(t \geq 0\)): \(\displaystyle \;\; i_C(t) = C \frac{d v_C(t)}{d t}, \quad i_R = \frac{v_R(t)}{R}, \quad V_A = v_R(t) + v_C(t), \quad i_R(t) = i_C(t)\)
	\item dalla legge delle correnti si ottiene un'equazione differenziale del primo ordine:
	\[C \frac{dv_C(t)}{d t} = \frac{v_R(t)}{R} = \frac{V_A - v_C(t)}{R} \quad \rightarrow \quad \frac{dv_C(t)}{dt} = -\frac{v_C(t)}{RC} + \frac{V_A}{RC} \quad \rightarrow \quad v_C(t) = A \cdot e^{-\tfrac{t}{RC}} + B\]
	\item si sostituisce la soluzione generale nell'equazione differenziale e si impongono le condizioni iniziali:
	\[-\frac{1}{RC} A \cdot e^{-\tfrac{t}{RC}} = -\frac{1}{RC} A \cdot e^{-\tfrac{t}{RC}} - \frac{1}{RC} B + \frac{V_A}{RC} \quad \rightarrow \quad B = V_A\]
	\[v_C(0) = 0 \quad \rightarrow \quad A + B = 0 \quad \rightarrow \quad A = -B = -V_A \]
	\item si ottengono quindi le espressioni delle variabili elettriche durante il transitorio:
	\[v_C(t) = V_A -V_A \cdot e^{-\tfrac{t}{RC}} \qquad v_R(t) = V_A e^{-\tfrac{t}{RC}} \qquad i_C(t) = i_R(t) = \frac{V_A}{R} e^{-\tfrac{t}{RC}}\]
	\item l'istante in cui la tensione sul condensatore raggiunge metà del suo valore di regime è:
	\[\frac{V_A}{2} = V_A - V_A \cdot e^{-\tfrac{t_{1/2}}{RC}} \quad \rightarrow \quad t_{1/2} = \ln(2) \cdot RC \approx 0.69 RC\]
	\item analizzando il bilancio energetico del circuito si ottiene che metà dell'energia fornita dal generatore viene
	immagazzinata nel condensatore e metà viene dissipata dalla resistenza come calore:
	\[E_{V_A} = \int_{0}^{\infty} V_A \cdot i(t) dt = C \cdot {V_A}^2 \quad E_R = \int_{0}^{\infty} R \cdot i^2(t) dt = \frac{C \cdot {V_A}^2}{2} \quad E_C = \int_{0}^{\infty} v_C(t) \cdot i(t) dt = \frac{C \cdot {V_A}^2}{2}\]
\end{itemize}

\begin{center}
	\begin{minipage}{0.4\textwidth}
		\centering \begin{circuitikz}[american, scale=1.2]
			\coordinate (A) at (0, 2);   % Nodo superiore sinistro (sopra VA)
			\coordinate (B) at (0, 0);     % Nodo inferiore sinistro (sotto VA)
			\coordinate (C) at (3, 2);   % Nodo superiore destro (dopo R)
			\coordinate (D) at (3, 0);     % Nodo inferiore destro (sotto C)
			\coordinate (E) at (1.2, 2); % Nodo tra interruttore e resistore
			\coordinate (F) at (1.2, 0);   % Nodo tra generatore e condensatore
	
			\draw (A) to [V, l_=\(V_A\)] (B);
			\draw (A) to [ospst] (E); 
			\draw (E) to [R, l_=\(R\), v^=\(v_R\), i_>=\(i_R\)] (C);
			\draw (C) -- (D); 
			\draw (D) to [C, l_=\(C\), v^=\(v_C\), i_>=\(i_C\)] (F);
			\draw (F) -- (B);
		\end{circuitikz}	
	\end{minipage}
	\begin{minipage}{0.55\textwidth}
		\centering \includegraphics[width=0.8\textwidth]{immagini/1_intro/rc_carica.png}
	\end{minipage}
\end{center}

\vspace{0.5cm}

\subsubsection*{Scarica di un condensatore}
\begin{itemize}
	\item condizioni iniziali (\(t < 0\)): interruttore inizialmente aperto \(\quad \rightarrow \quad v_C(0) = V_A, \quad i_C(0) = 0\)
	\item nel transitorio (\(t \geq 0\)): \(\displaystyle \;\; i_C(t) = C \frac{d v_C(t)}{d t}, \quad i_R = \frac{v_R(t)}{R}, \quad v_R(t) = v_C(t), \quad i_R(t) + i_C(t) = 0\)
	\item dalla legge delle correnti si ottiene un'equazione differenziale del primo ordine:
	\[C \frac{dv_C(t)}{d t} = \frac{v_R(t)}{R} \quad \rightarrow \quad \frac{dv_C(t)}{dt} = -\frac{v_C(t)}{RC} \quad \rightarrow \quad v_C(t) = A \cdot e^{-\tfrac{t}{RC}}\]
	\item siccome l'equazione è omogenera, è sufficiente imporre le condizioni iniziali:
	\[v_C(0) = V_A \quad \rightarrow \quad A = V_A \]
	\item si ottengono quindi le espressioni delle variabili elettriche durante il transitorio:
	\[v_C(t) = v_R(t) = V_A \cdot e^{-\tfrac{t}{RC}} \qquad i_C(t) = -\frac{V_A}{R} e^{-\tfrac{t}{RC}} \qquad i_R(t) = \frac{V_A}{R} e^{-\tfrac{t}{RC}}\]
	\item l'istante in cui la tensione sul condensatore raggiunge metà del suo valore di regime è:
	\[\frac{V_A}{2} = V_A - V_A \cdot e^{-\tfrac{t_{1/2}}{RC}} \quad \rightarrow \quad t_{1/2} = \ln(2) \cdot RC \approx 0.69 RC\]
	\item analizzando il bilancio energetico del circuito si ottiene tutta l'energia immagazzinata nel condensatore viene dissipata
	dalla resistenza come calore e il condensatore rimane scarico alla fine del transitorio:
	\[E_R = \int_{0}^{\infty} R \cdot {i_R}^2(t) dt = \frac{C \cdot {V_A}^2}{2} \qquad\qquad E_C = \int_{0}^{\infty} v_C(t) \cdot i_C(t) dt = -\frac{C \cdot {V_A}^2}{2}\]
\end{itemize}

\begin{center}
	\begin{minipage}{0.4\textwidth}
		\centering \begin{circuitikz}[american, scale=1.2]
	    \coordinate (A) at (0, 0);
		\coordinate (B) at (0, 2);
		\coordinate (C) at (2, 2);
		\coordinate (D) at (2, 0);
	
	    \draw (B) to [R, l=\(R\), v=\(v_R\), i>=\(i_R\)] (A);
	    \draw (B) to [ospst] (C); 
	    \draw (D) to [C, l=\(C\), v<=\(v_C\), i<=\(i_C\)] (C);
		\draw (D) -- (A);
	\end{circuitikz}
	\end{minipage}
	\begin{minipage}{0.55\textwidth}
		\centering \includegraphics[width=0.8\textwidth]{immagini/1_intro/rc_scarica.png}
	\end{minipage}
\end{center}

%\include{paragrafi/2_semiconduttori}
%\section{Giunzione PN e diodi}
\subsection{Giunzione pn all'equilibrio}
\subsubsection*{Struttura base}
Una giunzione pn si ottiene unendo due regioni di semiconduttore drogate in modo diverso: una regione di tipo p (con eccesso di
lacune) e una regione di tipo n (con eccesso di elettroni).

\subsubsection*{Equilibrio tra diffusione e potenziale}
\begin{itemize}
	\item Quando le due regioni si uniscono, si forma un \textbf{gradiente di concentrazione} dei portatori di carica che induce
	uno spostamento di elettroni dalla regione n alla regione p e uno spostamento di lacune dalla regione p alla regione n;
	si forma in questo modo una corrente di diffusione dalla regione p alla regione n.
	\item Lo spostamento dei portatori induce la formazione di ioni fissi costituiti dagli atomi dei droganti: i donatori perdono
	il loro elettrone spaiato e diventano ioni con carica positiva nella regione n, mentre gli accettori catturano l'elettrone che
	gli mancava e diventano ioni con carica negativa nella regione p. Questi ioni fissi generano un campo elettrico e un
	\textbf{potenziale di giunzione}; si forma in questo modo anche una corrente di deriva dalla regione n alla regione p che si
	oppone alla corrente di diffusione.
	\item All'equilibrio le due correnti si bilanciano, ma rimane una regione in prossimità della giunzione sono presenti solo ioni
	fissi dei droganti per l'assenza di portatori di carica.
	\item Si formano in questo modo tre regioni:
	\begin{enumerate}[topsep=-2pt, itemsep=0pt]
		\item \textbf{regione di svuotamento} o \textbf{regione di carica spaziale} (RCS): zona in prossimità della giunzione
		priva di portatori di carica liberi (svuotamento) in cui sono presenti solo ioni fissi (carica spaziale);
		\item \textbf{regione quasi neutra} (RQN) \textbf{di tipo p}: zona lontana dalla giunzione che non risente della giunzione
		pn e mantiene le caratteristiche di un semiconduttore di tipo p;
		\item \textbf{regione quasi neutra} (RQN) \textbf{di tipo n}: zona lontana dalla giunzione che non risente della giunzione
		pn e mantiene le caratteristiche di un semiconduttore di tipo n.
	\end{enumerate}
\end{itemize}

\vspace{0.5cm}
\begin{center}
	\includegraphics[width=0.85\textwidth]{immagini/3_giunzioni_pn_e_diodi/pn_eq1.png}
\end{center}
\vspace{0.5cm}

\noindent
NOTA: i seguenti calcoli si riferiscono alle grandezze per unità di superficie di giunzione \(\Sigma_j\). Per cui per ottenere
ad esmepio la densità di carica elettrica effetiva è necessario moltiplicare la densità di carica elettrica per unità di superficie
utilizzata nei calcoli per la superficie di giunzione: \(\rho_{e\!f\!f} = \rho \cdot \Sigma_j\).

\newpage

\subsubsection*{Carica elettrica all'equilibrio}
La carica elettrica nelle regioni quasi neutre è nulla, siccome non vengono alterate le concentrazioni di portatori di carica
liberi (e il drogaggio non modifica la carica complessiva). Nella regione di svuotamento, invece, la carica elettrica è data
dalla somma delle cariche degli ioni fissi e dipende dalle concentrazioni di drogaggio \(N_A\) e \(N_D\):
\[\rho(x) = \begin{cases}
	-q N_A & -x_p \leq x \leq 0 \quad \text{(regione p)}\\
	+q N_D & 0 < x \leq x_n \quad \text{(regione n)}\\
\end{cases} \qquad \rightarrow \qquad Q_p = -q N_A x_p, \quad Q_n = +q N_D x_n\]
Siccome non ci sono stati scambi di cariche con l'esterno, la carica totale deve rimanere nulla:
\[Q_p + Q_n = 0 \quad \Rightarrow \quad N_A x_p = N_D x_n\]

\subsubsection*{Campo elettrico all'equilibrio}
Il campo elettrico nella regione di carica spaziale si calcola:
\[\frac{dE(x)}{dx} = -\frac{\rho(x)}{\varepsilon} = \begin{cases}
	-q N_A / \varepsilon & -x_p \leq x \leq 0 \\
	+q N_D / \varepsilon & 0 < x \leq x_n
\end{cases} \quad \Rightarrow \quad E(x) = \begin{cases}
	-q N_A (x + x_p) / \varepsilon & -x_p \leq x \leq 0 \\
	+q N_D (x_n - x) / \varepsilon & 0 < x \leq x_n
\end{cases}\]
Il campo elettrico è nullo nelle regioni quasi neutre e raggiunge il valore massimo in \(x = 0\):
\[E_{max} = E(0) = -\frac{q N_A x_p}{\varepsilon} = -\frac{q N_D x_n }{\varepsilon}\]

\subsubsection*{Potenziale elettrico all'equilibrio e potenziale di contatto}
La differenza di potenziale si calcola integrando il campo elettrico, per la relazione \(dV(x)/dx = -E(x)\). In particolare
si definiscono i potenziali nelle due regioni quasi neutre \(V_1\) in \(x = -x_p\) e \(V_2\) in \(x = x_n\) e si calcola il
potenziale intrinseco di giunzione o potenziale di contatto \(V_0 = V_2 - V_1\) tra le due estremità della regione di svuotamento ponendo \(V_1 = 0\):
\[V_0 = \int_{-x_p}^{x_n} E(x) \, dx = \frac{-E(0) \cdot (x_n + x_p)}{2}\]

\subsubsection*{Potenziale di contatto e concentrazioni di drogaggio}
Il potenziale di contatto può essere espresso in funzione delle concentrazioni di drogaggio \(N_A\) e \(N_D\) e delle concentrazioni
intrinseche di portatori di carica \(n_i\):
\[V_0 = V_2 - V_1 = V_T \ln \left(\frac{n_2}{n_1}\right) = V_T \ln{\left(\frac{N_A N_D}{{n_i}^2}\right)}\]

\subsubsection*{Schema riassuntivi per una giunzione pn in equilibrio}
\begin{minipage}{0.45\textwidth}
	\centering
	\includegraphics[width=0.9\textwidth]{immagini/3_giunzioni_pn_e_diodi/pn_eq_2.1.png}
\end{minipage}
\begin{minipage}{0.45\textwidth}
	\centering
	\includegraphics[width=0.95\textwidth]{immagini/3_giunzioni_pn_e_diodi/pn_eq_2.2.png}
\end{minipage}

\subsubsection*{Ampiezza della regione di svuotamento all'equilibrio}
L'ampiezza della regione di svuotamento \(W = x_n + x_p\) dipende dalle concentrazioni dei drogaggi e dal potenziale di contatto:
\[V_0 = \frac{-E(0) \cdot W}{2}, \quad x_n N_D = x_p N_A \quad \rightarrow \quad W = \frac{2 V_0}{-E(0)} = \frac{2 \varepsilon V_0}{q N_A x_p} = \sqrt{\frac{2 \varepsilon V_0}{q} \left(\frac{1}{N_A} + \frac{1}{N_D}\right)}\]
\[x_n = W\frac{N_A}{N_A + N_D} \qquad x_p = W\frac{N_D}{N_A + N_D}\]
Si osserva che l'ampiezza della regione di svuotamento è inversamente proporzionale alle concentrazioni di drogaggio, per cui
aumentando i drogaggi diminuisce l'ampiezza della regione di svuotamento. Inoltre la regione di svuotamento si ripartisce in maniera
inversamente proporzionale ai drogaggi, ovvero si allarga maggiormente nella regione meno drogata.

\subsubsection*{Regione pn con elettrodi metallici}
Una giunzione pn, per essere utilizzabile in un circuito, deve essere collegata alle due estremità a due elettrodi metallici.
Gli elettrodi, essendo buoni conduttori, inducono una locale ridistribuzione dei portatori di carica (spostamento di lacune
dalla regione p all'elettrodo e di elettroni dalla regione n all'elettrodo). Questo effetto ha le stesse dinamiche di una
giunzione pn, in particolare si formano due regioni di carica spaziale che inducono una differenza di potenziale tra gli
elettrodi (con potenziali \(V_1\) e \(V_2\)) e la giunzione che controbilanciano il potenziale (o meglio tensione) di contatto
\(V_0\). All'equilibrio si ha \(V_1 + V_0 + V_2 = 0\), per cui la differenza di potenziale degli estremi di una giunzione è nulla.

\begin{center}
	\includegraphics[width=0.7\textwidth]{immagini/3_giunzioni_pn_e_diodi/pn_eq3.png}
\end{center}

\newpage

\subsection{Giunzione pn polarizzata}
\subsubsection*{Polarizzazione diretta e inversa}
Si collega una giunzione pn ad un generatore di tensione \(V_A\) con polo positivo connesso alla regione p e polo negativo alla
regione n. In questo modo si ha una polarizzazione della giunzione che può essere:
\begin{itemize}
	\item \textbf{polarizzazione diretta} se \(V_A > 0\)
	\item \textbf{polarizzazione inversa} se \(V_A < 0\)
\end{itemize}
Inoltre si definisce il potenziale di riferimento come il potenziale della regione n (\(V_n = 0V\)). In questo modo la regione
p ha un potenziale \(V_p = -V_0\), uguale ed opposto al potenziale intrinseco o di contatto.

\subsubsection*{Giunzione in polarizzazione diretta}
In polarizzazione diretta la tensione applicata \(V_A > 0\) riduce la differenza di potenziale tra le due regioni quasi neutre,
ottenendo: \(V_p - V_n = -V_0 + V_A\). Di conseguenza il campo elettrico nella regione di svuotamento diminuisce in modulo e
l'ampiezza della regione di svuotamento si riduce:
\[W(V_0 - V_A) = \sqrt{\frac{2 \varepsilon (V_0 - V_A)}{q} \left(\frac{1}{N_A} + \frac{1}{N_D}\right)} < W(V_0)\]
Siccome il campo elettrico e il potenziale si riducono, la corrente di deriva diminuisce e prevale il fenomeno di diffusione 
che induce un flusso di elettroni dalla regione n alla regione p e di lacune dalla regione p alla regione n. Siccome la regione
n è ricca di elettroni e la regione p è ricca di lacune, questo flusso è detto flusso dei portatori maggioritari. Si induce in
questo modo una \textbf{elevata corrente di diffusione dei maggioritari} che attraversa la giunzione pn dalla regione p alla regione n.

\subsubsection*{Giunzione in polarizzazione inversa}
In polarizzazione inversa la tensione applicata \(V_A < 0\) aumenta la differenza di potenziale tra le due regioni quasi neutre,
ottenendo: \(V_p - V_n = -V_0 - |V_A|\). Di conseguenza il campo elettrico nella regione di svuotamento aumenta in modulo e
l'ampiezza della regione di svuotamento si allarga:
\[W(V_0 + |V_A|) = \sqrt{\frac{2 \varepsilon (V_0 + |V_A|)}{q} \left(\frac{1}{N_A} + \frac{1}{N_D}\right)} > W(V_0)\]
Siccome il campo elettrico e il potenziale aumentano, la corrente di deriva prevale sulla corrente di diffusione. Si ha in questo
modo un flusso di elettroni dalla regione p alla regione n e di lacune dalla regione n alla regione p. Siccome la regione
p è povera di elettroni e la regione n è povera di lacune, questo flusso è detto flusso dei portatori minoritari. Si induce in
questo modo una \textbf{debole corrente di deriva dei minoritari} che attraversa la giunzione pn dalla regione n alla regione p.
\[J = (\mu_n n + \mu_p p) \, qE \qquad \text{con} \; n \approx p \approx 0\]

\begin{center}
	\includegraphics[width=0.7\textwidth]{immagini/3_giunzioni_pn_e_diodi/pn_pol2.png}
\end{center}

\subsection{Giunzione pn polarizzata vista come diodo}
\subsubsection*{Relazione tensione-corrente in un diodo}
Si assegnano dei riferimenti ai due terminali della giunzione pn in modo da renderla schematizzabile e utilizzabile come diodo
in un circuito elettrico:
\begin{itemize}
	\item \textbf{anodo}: terminale positivo, collegato alla regione p, dove si assorbono gli elettroni
	\item \textbf{catodo}: terminale negativo, collegato alla regione n, dove si immettono gli elettroni
\end{itemize}
\begin{center}
	\includegraphics[width=0.55\textwidth]{immagini/3_giunzioni_pn_e_diodi/diodo_1.png}
\end{center}
Definiti i riferimenti di tensione e corrente in un diodo (o giunzione pn polarizzata), si può definire la relazione tensione-corrente
che lega la tensione applicata \(V_A\) alla corrente \(I\) che attraversa il diodo:
\[i_D = I_S \left(e^{\tfrac{v_D}{\eta V_T}} - 1\right)\]
\begin{itemize}
	\item \(i_D\) e \(v_D\) sono la corrente e la tensione nel diodo, con riferimento positivo dall'anodo al catodo;
	\item \(I_S\) è la corrente di saturazione inversa, ovvero la debole corrente che attraversa il diodo quando è collegato
	in polarizzazione inversa (tipicamente dell'ordine di qualche nA);
	\item \(\eta\) è il coefficiente di idealità del diodo, che dipende dal materiale e dal processo di fabbricazione (tipicamente
	compreso tra 1 e 2);
	\item \(V_T = k_B T / q\) è il potenziale termico (del valore di circa 25mV a temperatura ambiente).
\end{itemize}
\begin{minipage}{0.65\textwidth}
Analizzando la curva caratteristica del diodo si osserva che in polarizzazione diretta (\(v_D > 0\)) la corrente cresce
esponenzialmente con la tensione applicata, mentre in polarizzazione inversa (\(v_D < 0\)) la corrente si stabilizza ad un valore
negativo pari a \(-I_S\).
\end{minipage}
\begin{minipage}{0.34\textwidth}
	\centering
	\includegraphics[width=0.8\textwidth]{immagini/3_giunzioni_pn_e_diodi/diodo_2.png}
\end{minipage}

\subsubsection*{Capacità della giunzione in polarizzazione inversa}
Analizzando la carica elettrica (per unità di superficie) presente nella regione di carica spaziale di una giunzione pn in
polarizzazione inversa si osserva che:
\[\begin{array}{l}
	Q_n = q N_D x_n = q N_D \frac{N_A}{N_A + N_D} W = \sqrt{2 q \varepsilon (V_0 - V_A) \frac{N_A N_D}{N_A + N_D}} \\
	Q_p = q N_A x_p = q N_A \frac{N_D}{N_A + N_D} W = \sqrt{2 q \varepsilon (V_0 - V_A) \frac{N_A N_D}{N_A + N_D}}
\end{array} \qquad Q_n = -Q_p\]
Si osserva quindi che la carica dipende dalla tensione applicata \(V_A\). La giunzione pn in polarizzazione inversa equivale
ad un condensatore con capacità per unità di area non lineare data da:
\[C_{j} = \frac{dQ}{dV_A} = \sqrt{\frac{q \varepsilon}{2(V_0 - V_A)} \frac{N_A N_D}{N_A + N_D}} = \frac{\varepsilon}{W}\]
NOTA: per ottenere la reale capacità del diodo bisogna moltiplicare la capacità per unità di area \(C_j\) per la sezione
della giunzione \(\Sigma_j\): \[C = C_j \cdot \Sigma_j = \frac{\varepsilon}{W} \; \Sigma_j\]

\newpage

\subsubsection*{Coefficiente di idealità}
\begin{minipage}{0.65\textwidth}
	Facendo variare il coefficiente di idealità \(\eta\) tra 1 e 2 si osserva che avviene una traslazione orizzontale della curva
	caratteristica del diodo. Minore è il valore di \(\eta\), più la curva sale rapidamente in polarizzazione diretta. In genere
	si utilizza \(\eta = 1\) per correnti basse e \(\eta = 2\) per correnti elevate.
	\vspace{1.2cm}
\end{minipage}
\begin{minipage}{0.34\textwidth}
	\centering
	\includegraphics[width=0.77\textwidth]{immagini/3_giunzioni_pn_e_diodi/diodo_3.png}
\end{minipage}

\subsubsection*{Modello semplificato del diodo}
Si osserva che la curva caratteristica del diodo può essere approssimata con un modello semplificato definito in funzione della
tensione applicata \(v_D\) e di conseguenza della polarizzazione del diodo:
\begin{itemize}
	\item per \(v_D < V_{ON} \rightarrow i_D = 0\) il diodo è in interdizione e si comporta come un circuito aperto in condizioni
	stazionarie oppure come condensatore non lineare in condizioni non stazionarie;
	\item per \(v_D = V_{ON} \rightarrow i_D > 0\) il diodo è in conduzione e si comporta come un generatore ideale di tensione
	con tensione \(V_{ON}\).
\end{itemize}
La tensione \(V_{ON}\) è detta tensione di soglia del diodo e divide le due regioni di funzionamento. Tipicamente per un diodo
al silicio si assume \(V_{ON} = 0.7V\), mentre per un diodo al germanio si assume \(V_{ON} = 0.3V\).

\begin{center}
	\includegraphics[width=0.7\textwidth]{immagini/3_giunzioni_pn_e_diodi/diodo_4.png}
\end{center}

\vspace{0.3cm}

\subsection{Applicazioni speciali dei diodi}
\subsubsection*{Applicazioni generali}
I diodi sono componenti fondamentali in molti circuiti elettronici e trovano applicazione in diversi ambiti:
\begin{itemize}
	\item \textbf{raddrizzatori}: i diodi vengono utilizzati nei circuiti raddrizzatori per convertire la corrente alternata (AC)
	in corrente continua (DC), permettendo il funzionamento di dispositivi elettronici alimentati a corrente continua;
	\item \textbf{protezione da inversioni di polarità}: i diodi proteggono i circuiti elettronici da danni causati da inversioni
	accidentali di polarità della tensione di alimentazione;
	\item \textbf{limitatori di tensione}: i diodi limitano la tensione in un circuito, proteggendo i componenti sensibili
	da sovratensioni;
	\item \textbf{LED (Light Emitting Diode)}: i diodi LED emettono luce quando attraversati da corrente elettrica, trovando
	applicazione in display, indicatori luminosi e illuminazione;
	\item \textbf{fotorilevatore}: i diodi a semiconduttore possono essere utilizzati come sensori di luce, convertendo l'energia
	luminosa in corrente elettrica.
\end{itemize}

\newpage

\subsubsection*{Fotodiodo}
Il fotodiodo è una giunzione pn collegata in polarizzazione inversa con l'area di svuotamento esposta alla luce. Quando la luce
colpisce la regione di svuotamento, eccita gli elettroni che si liberano dai legami covalenti, generando coppie elettrone-lacuna
(fotogenerazione). I due portatori vengono separati dal campo elettrico presente nella regione, generando una corrente detta
fotocorrente. La fotocorrente è proporzionale all'intensità della luce incidente. Il fotodiodo viene utilizzato in applicazioni
come sensori di luce e telecomunicazioni ottiche.
\[i_\text{D} = I_S \left(e^{\tfrac{v_D}{\eta V_T}} - 1\right) - I_{PH} \qquad\qquad I_{PH} = R \cdot P_0 \qquad\qquad \begin{array}{l}
	I_{PH}: \text{ fotocorrente (A)} \\
	P_0: \text{ potenza ottica incidente (W)} \\
	R: \text{ responsività (A/W)}
\end{array}\]
La curva caratteristica risulta spostata verso il basso di un valore pari alla fotocorrente \(I_{PH}\). È possibile schematizzare
un fotodiodo come un diodo ideale in parallelo ad una sorgente di corrente pari a \(I_{PH}\).

\begin{center}
	\includegraphics[width=0.7\textwidth]{immagini/3_giunzioni_pn_e_diodi/diodo_5.png}
\end{center}

\subsubsection*{LED (Light Emitting Diode) o diodi a emissione luminosa}
I LED sono giunzioni pn in polarizzazione diretta con l'area di svuotamento \say{scoperta}. Quando una corrente attraversa il LED,
le lacune dalla regione p si ricombinano con gli elettroni dalla regione n nella regione di svuotamento, rilasciando energia sotto
forma di fotoni (emissione di luce).

La lunghezza d'onda della luce emessa dipende dalla differenza di energia tra la banda di conduzione e la banda di valenza
(energy gap) propria di ogni semiconduttore. Si utilizzano, infatti, semiconduttori diversi per ottenere colori diversi.

Un parametro importante è la tensione di accensione \(V_{ON}\) dei LED, ovvero la tensione a cui il LED inizia a emettere luce.
In genere è superiore alla tensione di soglia in quanto non basta fornire energia per permettere il passaggio della corrente, ma
è necessario fornire energia sufficiente per permettere l'emissione dei fotoni.
\[V_{ON} > \frac{hc}{\lambda q} \qquad \begin{array}{l}
	h: \text{ costante di Planck} (6.626 \times 10^{-34} \J\s) \\
	c: \text{ velocità della luce} (3.0 \times 10^{8} \m/\s) \\
	\lambda: \text{ lunghezza d'onda della luce emessa} \\
	q: \text{ carica dell'elettrone } (1.6 \times 10^{-19} \C)
\end{array} \quad \begin{array}{l l}
	\lambda = 620 \;\nm\; \text{(rosso)} & V_{ON} \approx 2.0 - 2.2 \V \\
	\lambda = 520 \;\nm\; \text{(giallo)} & V_{ON} \approx 2.1 - 2.2 \V \\
	\lambda = 510 \;\nm\; \text{(verde)} & V_{ON} \approx 2.5 - 3.3 \V \\
	\lambda = 470 \;\nm\; \text{(blu)} & V_{ON} \approx 3.2 - 3.3 \V
\end{array}\]
Non esiste nessun semiconduttore che emetta luce bianca: per creare un \say{LED bianco} si utilizza un LED blu con un rivestimento
di fosforo (giallo) che converte parte della luce blu in luce gialla. Dalla combinazione delle due luci si ottiene la luce bianca
percepita dall'occhio umano. In base alla quantità di fosforo utilizzata si possono ottenere diverse tonalità di bianco (caldo,
neutro, freddo).

\begin{center}
	\includegraphics[width=0.27\textwidth]{immagini/3_giunzioni_pn_e_diodi/diodo_6.png}
\end{center}

%\section{Condensatore MOS o CMOS}
\subsection{Struttura e funzionamento}
\subsubsection*{Struttura base}
Un condensatore MOS (Metal-Oxide-Semiconductor) è costituito da tre strati principali:
\begin{itemize}
	\item un metallo (Metal) che funge da elettrodo superiore detto \textbf{gate} (G), generalmente in polisilicio;
	\item un ossido (Oxide) che funge da dielettrico o \text{isolante}, di solito in biossido di silicio, SiO\(_2\);
	\item un semiconduttore (Semiconductor) che funge da elettrodo inferiore detto \textbf{substrato o body} (B),
	generalmente in silicio drogato di tipo p o n.
\end{itemize}

\subsubsection*{Funzionamento e proprietà}
\begin{itemize}
	\item Si identificano le dimensioni del dielettrico con \(L\) lunghezza, \(W\) larghezza e \(T_{ox}\) spessore.
	\item Si assume di collegare il substrato a massa (0 V) e applicare una tensione variabile al gate \(V_G\).
	\item La capacità del condensatore mos è data da: \(\displaystyle C_{ox} = \varepsilon \frac{W \ccdot L}{T_{ox}}\)
\end{itemize}

\begin{center}
	\includegraphics[width=0.5\textwidth]{immagini/4_mosfet/cmos_1.png}
\end{center}

\subsection{Condensatore mos con substrato di tipo p}
\subsubsection*{Tensione di gate negativa (\(V_G < 0\))}
Se \(V_G < 0\), il gate si carica negativamente, attirando le cariche positive (lacune) verso la superficie del semiconduttore
adiacente all'ossido, creando una \textbf{regione di accumulazione} di lacune.

\subsubsection*{Tensione di gate inferiore alla tensione di soglia (\(0 < V_G < V_{TN}\))}
Se \(0 < V_G < V_{TN}\), il gate si carica positivamente, creando una \textbf{regione di svuotamento} di lacune vicino alla
superficie del semiconduttore, lasciando dietro di sé ioni negativi fissi (atomi droganti). Si forma così una zona di carica
spaziale negativa, priva di portatori mobili.

\subsubsection*{Tensione di gate superiore alla tensione di soglia (\(V_G > V_{TN}\))}
Se \(V_G > V_{TN}\), il gate si carica ancora più positivamente, attirando elettroni verso la superficie del semiconduttore
adiacente all'ossido. Si crea una \textbf{regione di inversione} dove la concentrazione di elettroni supera quella delle lacune.
Si forma così un canale conduttivo di tipo n. La dimensione della regione di svuotamento rimane quasi costante, dopo aver
raggiunto il massimo per \(V_G = V_{TN}\), mentre la concentrazione di elettroni nella regione di inversione aumenta con \(V_G\).

\begin{center}
	\begin{minipage}{0.3\textwidth}
		\centering \includegraphics[width=0.85\textwidth]{immagini/4_mosfet/cmos_2.png}
		\small{Accumulazione per \(V_G < 0\)}
	\end{minipage}
	\begin{minipage}{0.3\textwidth}
		\centering \includegraphics[width=0.85\textwidth]{immagini/4_mosfet/cmos_3.png}
		\small{Svuotamento per \(0 < V_G < V_{TN}\)}
	\end{minipage}
	\begin{minipage}{0.3\textwidth}
		\centering \includegraphics[width=0.85\textwidth]{immagini/4_mosfet/cmos_4.png}
		\small{Inversione per \(V_G > V_{TN}\)}
	\end{minipage}
\end{center}

\subsection{Analisi del p-cmos in condizioni di svuotamento/inversione}
\subsubsection*{Densità di carica}
Analizzando la densità di carica \(\rho(x)\), il campo elettrico \(E(x)\) si ottengono le seguenti relazioni:
\[\rho(x) = \begin{cases}
	-q N_A & \text{per } -x_D < x < 0 \\
	0 & \text{altrimenti}
\end{cases}\]
\subsubsection*{Campo elettrico}
Dalla densità di carica si ricava il campo elettrico \(E(x)\) nella regione di svuotamento e nell'ossido, si noti che
c'è una discontinuità del campo elettrico all'interfaccia semiconduttore-ossido dovuta alla differenza di
permittività tra i due materiali, inoltre il campo elettrico nell'ossido è costante:
\[E(x) = \begin{cases}
	- q N_A (x + x_D) / \varepsilon_S & \text{per } -x_D < x < 0 \\
	- q N_A x_D / \varepsilon_{OX} & \text{per } 0 < x < t_{OX} \\
	0 & \text{altrimenti}
\end{cases} \qquad\qquad \begin{array}{l}
	E(0^-) = - \frac{q N_A x_D}{\varepsilon_S} \\[8pt]
	E(0^+) = - \frac{q N_A x_D}{\varepsilon_{OX}} \\[8pt]
	E_{OX} = E(0^+) = E(0^-) \frac{\varepsilon_S}{\varepsilon_{OX}}
\end{array}\]

\subsubsection*{Potenziale elettrico}
Si ottiene il potenziale nel substrato \(V_B\) (potenziale di riferimento), il potenziale all'interfaccia (nella giunzione tra
semiconduttore e ossido) \(V(0)\) e il potenziale al gate \(V_G\):
\[V_B = V(-x_D) = 0 \qquad\qquad V(0) = \frac{qN_A}{2\varepsilon_S} {x_D}^2 \quad\qquad V_G = V(t_{OX}) = \frac{qN_A}{2\varepsilon_S} {x_D}^2 + \frac{q N_A x_D}{\varepsilon_{OX}} t_{OX}\]

\subsubsection*{Concentrazioni dei portatori}
Si ricavano le concentrazioni dei portatori \(p_1\) e \(n_1\) nella regione neutra, lontano dall'interfaccia, e le concentrazioni
all'interfaccia \(p_2\), \(n_2\). All'interfaccia le concentrazioni variano esponenzialmente con \(V(0)\):
\[\begin{array}{l c l}
	p_1 = N_A & \quad & p_2 = p_1 \, e^{-\tfrac{V(0)}{V_T}} \\
	n_1 = n_i^2 / N_A & \quad & n_2 = n_1 \, e^{\tfrac{V(0)}{V_T}}
\end{array} \qquad \frac{n_2}{n_1} = \frac{p_1}{p_2} = e^{\tfrac{v_2-v_1}{V_T}} \qquad \frac{v_2 - v_1}{V_T} = \ln\frac{n_2}{n_1} = \ln\frac{p_1}{p_2}\]

\subsubsection*{Tensione di soglia}
Si definisce la \textbf{tensione di soglia} \(V_{TN}\) come differenza di potenziale tra gate e substrato \(V_G - V_B\) (pari
a \(V_G\)) per cui la concentrazione di elettroni all'interfaccia è uguale al numero di lacune nella regione neutra, ovvero
quando c'è inversione totale con \(n_2 = N_A\):
\[V(0) = \frac{qN_A}{2\varepsilon_S} {x_D}^2 = 2 V_T \ln\frac{N_A}{n_i}, \;\; x_D = \sqrt{\frac{4 \varepsilon_S V_T}{q N_A} \ln \frac{N_A}{n_i}} \; \rightarrow \; V_{TN} = 2 V_T \ln \frac{N_A}{n_i} + \frac{t_{OX}}{\varepsilon_{OX}} \sqrt{4\varepsilon_S q N_A V_T \ln \frac{N_A}{n_i}}\]
La tensione di soglia dipende, quindi, dallo spessore dell'ossido \(t_{OX}\), dalla concentrazione di drogaggio del substrato \(N_A\)
e dai materiali usati (tramite \(\varepsilon_{OX}\) e \(\varepsilon_S\)).

\subsubsection*{Carica elettrica e capacità}
La carica elettrica per unità di area immagazzinata nel condensatore mos è data dalla somma degli ioni fissi nella
regione di svuotamento e degli elettroni nella regione di inversione:
\[Q_{TOT} = Q_{RCS} + Q_{n} = Q \cdot V_G \qquad Q_{RCS} = C \cdot V_{TN} \qquad Q_{n} = C \cdot (V_G - V_{TN})\]

\subsubsection*{Rappresentazione grafica del comportamento del p-cmos}
\begin{center}
	\begin{minipage}{0.48\textwidth}
		\centering \includegraphics[width=0.9\textwidth]{immagini/4_mosfet/cmos_5.png}
	\end{minipage}
	\begin{minipage}{0.48\textwidth}
		\centering \includegraphics[width=0.9\textwidth]{immagini/4_mosfet/cmos_6.png}
	\end{minipage}
\end{center}
\vspace{0.5cm}

\subsection{cmos con substrato di tipo n e differenze rispetto al p-cmos}
\subsubsection*{Comportamento del cmos con substrato di tipo p}
Il funzionamento è analogo a quello del cmos con substrato di tipo p, ma con le polarità invertite:
\begin{itemize}
	\item per \(V_G > 0\), si crea una regione di accumulazione di elettroni.
	\item per \(0 > V_G > V_{TP}\), si crea una regione di svuotamento di elettroni.
	\item per \(V_G < V_{TP}\), si crea una regione di inversione con un canale conduttivo di tipo p.
\end{itemize}
\begin{center}
	\begin{minipage}{0.3\textwidth}
		\centering \includegraphics[width=0.85\textwidth]{immagini/4_mosfet/cmos_7.png}
		\small{Accumulazione per \(V_G > 0\)}
	\end{minipage}
	\begin{minipage}{0.3\textwidth}
		\centering \includegraphics[width=0.85\textwidth]{immagini/4_mosfet/cmos_8.png}
		\small{Svuotamento per \(0 > V_G > V_{TP}\)}
	\end{minipage}
	\begin{minipage}{0.3\textwidth}
		\centering \includegraphics[width=0.85\textwidth]{immagini/4_mosfet/cmos_9.png}
		\small{Inversione per \(V_G < V_{TP}\)}
	\end{minipage}
\end{center}

\vspace{0.5cm}
\subsubsection*{Rappresentazione aree di lavoro dei cmos di tipo p e n}
\begin{center}
	\includegraphics[width=0.8\textwidth]{immagini/4_mosfet/cmos_10.png}
\end{center}

\newpage

\section{Transistor MOSFET}
\subsection{Struttura generale e classificazione dei mosfet}
\subsubsection*{Introduzione}
Un transistor MOSFET (Metal-Oxide-Semiconductor Field-Effect Transistor) è un dispositivo a quattro terminali che sfrutta
un condensatore mos per controllare il flusso di corrente tra due terminali detti \textbf{source} (S) e \textbf{drain} (D)
tramite un potenziale applicato ad un terzo terminale detto \textbf{gate} (G). Il quarto terminale è il substrato o
\textbf{body} (B) e viene generalmente collegato al source o ad un potenziale di riferimento (massa o \(V_{D\!D}\)).
Di seguito una rappresentazione schematica di un n-mosfet:

\begin{center}
	\includegraphics[width=0.7\textwidth]{immagini/4_mosfet/mosfet_1.png}
\end{center}

\subsubsection*{Struttura fisica}
A livello fisico, un mosfet è costituito da un condensatore mos con gate e substrato affiancato da due regioni
pesantemente drogate di tipo opposto al substrato, dette source e drain, che fungono da terminali di ingresso e uscita.
Il dielettrico del condensatore mos è generalmente in diossido di silicio (SiO\(_2\)).

Ad ogni terminale è associato un potenziale elettrico e per ogni coppia di terminali si definisce la tensione e la corrente
tra i due nodi:
\begin{center}
	\begin{minipage}{0.34\textwidth}
		\begin{itemize}
			\item \(V_G\): potenziale del gate
			\item \(V_S\): potenziale del source
			\item \(V_D\): potenziale del drain
			\item \(V_B\): potenziale del substrato
		\end{itemize}
	\end{minipage}
	\begin{minipage}{0.65\textwidth}
		\begin{itemize}
			\item \(V_{XY} = V_X - V_Y\): tensione tra i nodi X e Y
			\item[] es. \(V_{GS} = V_G - V_S\) tensione tra gate e source
			\item \(I_{XY}\): corrente che entra nel nodo X e esce dal nodo Y
			\item[] es. \(I_{DS}\) corrente che entra nel drain e esce dal source 
		\end{itemize}
	\end{minipage}
\end{center}
Si definiscono inoltre le dimensioni fisiche del mosfet:
\begin{itemize}
	\item \(L\): lunghezza del canale tra source e drain
	\item \(W\): larghezza del canale tra source e drain
	\item \(t_{OX}\): spessore dell'ossido isolante tra gate e substrato
\end{itemize}

\subsubsection*{Classificazione}
In base al tipo di canale (e di conseguenza in base al tipo del substrato), i mosfet si classificano in:
\begin{itemize}
	\item \textbf{n-mosfet} o \textbf{mosfet a canale n}: \\ substrato di tipo p, source e drain di tipo n\(^+\), canale
	di tipo n con gli elettroni come portatori principali, il substrato p collegato al potenziale minore del circuito
	(massa o al source) e il source ha potenziale minore del drain
	\item \textbf{p-mosfet} o \textbf{mosfet a canale p}: \\ substrato di tipo n, source e drain di tipo p\(^+\), canale
	di tipo p con le lacune come portatori principali, il substrato n collegato al potenziale maggiore del circuito
	(\(V_{D\!D}\) o al source) e il source ha potenziale maggiore del drain
\end{itemize}


\subsection{Struttura di un n-mosfet e vincoli sui potenziali}
Si assume per convenzione che il terminale di source ha potenziale minore di quello di drain: \(V_S < V_D\) e, di conseguenza,
che la corrente scorra dal drain al source: \(I_{DS} > 0\).

\begin{center}
	\includegraphics[width=0.6\textwidth]{immagini/4_mosfet/mosfet_2.png}
\end{center}

\subsubsection*{Condizioni all'equilibrio (nessuna tensione applicata)}
In assenza di tensioni, tutti i potenziali sono nulli e non c'è corrente tra i terminali. In particolare si hanno due giunzioni
pn in equilibrio tra il substrato p e le regioni n\(^+\) del source e del drain. Siccome le regioni n\(^+\) sono pesantemente
drogate, la regione di svuotamento si estende quasi totalmente nel substrato p.

\subsubsection*{Vincoli di polarizzazione dei diodi e potenziale di substrato}
I due diodi con catodi collegati ai nodi source e drain e con anodo in comune nel substrato, devono rimanere in interdizione
per il corretto funzionamento del mosfet, si ottengono le seguenti condizioni:
\begin{itemize}
	\item \(V_{BS} \leq 0 \rightarrow V_S \geq V_B\) (diodo source-substrato in interdizione)
	\item \(V_{BD} \leq 0 \rightarrow V_D \geq V_B\) (diodo drain-substrato in interdizione)
\end{itemize}
Si ottiene che il substrato \(V_B\) deve essere il nodo a potenziale più basso \(V_D \geq V_S \geq V_B\). Si solito si
collega il substrato al potenziale minore dell'intero circuito (massa) \(V_B = 0\) oppure al source \(V_B = V_S\).

\textit{NOTA}: Non è possibile collegare il substrato al drain perché si violerebbe la condizione di interdizione del diodo
source-substrato in quanto \(V_S < V_D\) e quindi \(V_{BS} = V_B - V_S = V_D - V_S > 0\).

\subsubsection*{Potenziale e tensioni di gate}
Il potenziale di gate \(V_G\) controlla la tensione tra le armature del condensatore mos che si forma tra il gate e il substrato
e di conseguenza identifica l'area di lavoro del p-mosfet. La tensione tra le due armature è variabile lungo la lunghezza del
canale ed è compresa tra le tensioni di gate-source e gate-drain ai margini del canale:
\begin{itemize}
	\item in prossimità del source \(V_{C,\text{source}} = V_{GS} = V_G - V_S\)
	\item in prossimità del drain \(V_{C,\text{drain}} = V_{GD} = V_G - V_D\)
\end{itemize}
\textit{NOTA}: siccome \(V_D \geq V_S\), si ha \(V_{GS} \geq V_{GD}\)

\newpage

\subsection{Aree di lavoro di un n-mosfet}
\subsubsection*{Transistor n-mosfet spento o in interdizione per \(V_{GS} < V_{TN}\)}
La tensione tra le armature del condensatore mos è inferiore alla tensione di soglia \(V_{TN}\) sia in prossimità del source
(\(V_{GS} < V_{TN}\)) che in prossimità del drain (\(V_{GD} < V_{GS} < V_{TN}\)).  Il condensatore, quindi, è in regime di
svuotamento ed è presente nel substrato, in prossimità dell'ossido, un'area di svuotamento (senza portatori di carica) che
separa source e drain. Non essendoci cariche libere per condurre corrente tra drain e source, si ha corrente nulla \(I_{DS} = 0\)
e il transistor è spento o in interdizione.
\begin{center}
	\begin{minipage}{0.45\textwidth}
		\centering \includegraphics[width=0.9\textwidth]{immagini/4_mosfet/mosfet_3.png}
	\end{minipage}
	\begin{minipage}{0.5\textwidth}
		\begin{align*}
			V_{GS} &< V_{TN} & \text{source in interdizione} \\[4pt]
			V_{GD} &< V_{GS} < V_{TN} & \text{drain in interdizione}
		\end{align*}
		\[I_{DS} = 0\]
	\end{minipage}
\end{center}

\subsubsection*{Transistor n-mosfet in conduzione lineare (o triodo) per \(V_{DS} < V_{GS} - V_{TN}\)}
La tensione delle armature del condensatore mos è superiore alla tensione di soglia \(V_{TN}\) sia in prossimità del drain
(\(V_{GD} > V_{TN}\)) che in prossimità del source (\(V_{GS} > V_{GD} > V_{TN}\)). Il condensatore si trova in regime di
inversione e si forma un canale conduttivo di tipo n (di elettroni) che congiunge source e drain. Gli elettroni che
costituiscono il canale si muovono liberamente dal source al drain, permettendo il passaggio di corrente \(I_{DS} > 0\) dal
drain al source. Il transistor si comporta come una resistenza (dipendente da \(V_{GS}\)) tra drain e source, lineare per
\(V_{DS}\) piccoli o parabolica per \(V_{DS}\) grandi.
\begin{center}
	\begin{minipage}{0.45\textwidth}
		\centering \includegraphics[width=0.9\textwidth]{immagini/4_mosfet/mosfet_4.png}
	\end{minipage}
	\begin{minipage}{0.5\textwidth}
		\begin{align*}
			V_{GD} &> V_{TN} & \text{drain in conduzione} \\[4pt]
			V_{GS} &= V_{GD} + V_{DS} & \\
			V_{GS} &> V_{TN} + V_{DS} & \text{source in conduzione}
		\end{align*}
		\[I_{DS} = k_n V_{DS} \left( V_{GS} - V_{TN} - \frac{V_{DS}}{2} \right)\]
		\[k_n = k_n' \cdot Z_n \qquad k_n' = \mu_n \varepsilon / T_{OX} \qquad Z_n = W_n/L_n\]
	\end{minipage}
\end{center}


\subsubsection*{Transistor n-mosfet in saturazione per pinchoff per \(V_{DS} > V_{GS} - V_{TN}\)}
La tensione delle armature del condensatore mos è superiore alla tensione di soglia \(V_{TN}\) in prossimità del source
(\(V_{GS} > V_{TN}\)), ma inferiore alla tensione di soglia in prossimità del drain (\(V_{GD} < V_{TN}\)). Il condensatore
si trova in regime misto: inversione in prossimità del source e svuotamento in prossimità del drain e si forma un canale
conduttivo di tipo n che congiunge solo parzialmente source e drain. Nonostante ciò i portatori riescono a fluire lo stesso
nella RCS spinti dal campo elettrico tra source e drain. In questo modo si ha ugualmente il passaggio di una corrente
\(I_{DS} \neq 0\) costante rispetto a \(V_{DS}\) (a meno della modulazione di lunghezza di canale).
\begin{center}
	\begin{minipage}{0.45\textwidth}
		\centering \includegraphics[width=0.9\textwidth]{immagini/4_mosfet/mosfet_5.png}
	\end{minipage}
	\begin{minipage}{0.5\textwidth}
		\begin{align*}
			V_{GS} &> V_{TN} & \text{source in conduzione} \\[4pt]
			V_{GD} &< V_{TN} & \text{drain in interdizione}
		\end{align*}
		\vspace{-20pt}
		\begin{align*}
			V_{GD} < V_{TN} \;\; &\rightarrow \;\; V_{GS} - V_{DS} < V_{TN} \\
			&\rightarrow \;\; V_{DS} > V_{GS} - V_{TN}
		\end{align*}
		\[I_{DS} = \frac{k_n}{2} (V_{GS} - V_{TN})^2\]
		\[k_n = k_n' \cdot Z_n \qquad k_n' = \mu_n \varepsilon / T_{OX} \qquad Z_n = W_n/L_n\]
	\end{minipage}
\end{center}

\subsubsection*{Modulazione di lunghezza di canale con n-mosfet in saturazione per pinchoff}
Quando il mosfet entra in saturazione per pinchoff, la regione di strozzamento si sposta verso il source, riducendo la lunghezza
\(L\) del canale conduttivo. Questo fenomeno, chiamato modulazione di lunghezza di canale, provoca un lieve aumento della corrente
\(I_{DS}\) con l'aumentare di \(V_{DS}\) anche in saturazione. Si può modellare questo effetto aggiungendo alla corrente in
saturazione un termine correttivo che dipende dal coefficiente di modulazione di lunghezza \(\lambda\) determinato dalle
caratteristiche fisiche del mosfet:

\begin{center}
	\begin{minipage}{0.6\textwidth}
		\centering \includegraphics[width=0.9\textwidth]{immagini/4_mosfet/mosfet_6.png}
	\end{minipage}
	\begin{minipage}{0.35\textwidth}
		\[I_{DS} = \frac{k_n}{2} (V_{GS} - V_{TN})^2 (1 + \lambda V_{DS})\]
		\[\lambda V_{DS} = \frac{\Delta L}{L}\]
	\end{minipage}
\end{center}
Analizzando la continuità della corrente tra la regione lineare e la regione di saturazione con modulazione di lunghezza di canale
si osserva che matematicamente c'è una discontinuità che in natura non esiste. Per ovviare a questo problema si utilizza la
correzione di lunghezza di canale anche nella regione lineare (tratteggio blu nella figura superiore):
\begin{align*}
	I_{DS,\text{lin}} &= k_n V_{DS} \left( V_{GS} - V_{TN} - \frac{V_{DS}}{2} \right) (1 + \lambda V_{DS}) & \text{per} \; V_{DS} \ll V_{GS} - V_{TN} \\
	I_{DS,\text{sat}} &= \frac{k_n}{2} (V_{GS} - V_{TN})^2 (1 + \lambda V_{DS}) & \text{per} \; V_{DS} \gg V_{GS} - V_{TN}
\end{align*}

\subsubsection*{Transistor n-mosfet in saturazione di velocità per \(V_{DS} > V_{DSATN}\)}
I portatori di carica (in questo caso elettroni) si muovono nel canale spinti dal campo elettrico \(E = V_{DS} / L\) con velocità
\(v_n = \mu_n E = \mu_n V_{DS} / L\). La velocità è, quindi, proporzionale a \(V_{DS}\).
Tale velocità raggiunge un valore massimo costante detto velocità di saturazione \(v_\text{sat,n} = 8 \cdot 10^6 \cm/\s\) per
un certo campo elettrico critico \(E_{crit,n}\) e una certa tensione critica \(V_{DS,sat,n}\):
\[E_\text{crit,n} = \frac{v_{sat,n}}{\mu_n} = 1.3 \frac{\V}{\mu\m} \qquad V_{DSATN} = E_\text{crit,n} \cdot L = \frac{v_{sat,n}}{\mu_n}L\]
Per \(V_{DS} > V_{DSATN}\), si osserva che la velocità degli elettroni e di conseguenza anche la corrente \(I_{DS}\) rimangono
costanti. Si ha un fenomeno di saturazione anticipato, detto saturazione di velocità, quando ci si aspetterebbe che il mosfet
lavori in regime di conduzione lineare.
\begin{center}
	\begin{minipage}{0.38\textwidth}
		\centering \includegraphics[width=0.9\textwidth]{immagini/4_mosfet/mosfet_17.png}
	\end{minipage}
	\begin{minipage}{0.6\textwidth}
		La corrente di saturazione di velocità, tenendo conto anche della modulazione di lunghezza di canale, è data da:
		\begin{align*}
			I_{DS,sat,n} &= W C_{ox} v_{sat,n} (V_{GS} - V_{TN}) \\
			&= k_n V_{DSATN}\left(V_{GS} - V_{TN} - \frac{V_{DSATN}}{2}\right)(1 + \lambda V_{DS})	
		\end{align*}
	\end{minipage}
\end{center}

\newpage

\subsection{Curve caratteristiche di corrente-tensione di un n-mosfet}
La corrente \(I_{DS}\) dipende da due tensioni indipendenti \(V_{GS}\) e \(V_{DS}\). Si identificano due curve caratteristiche:
\begin{itemize}
	\item \textbf{caratteristica di uscita} con \(I_{DS}\) in funzione di \(V_{DS}\) per valori costanti di \(V_{GS}\)
	\item \textbf{caratteristica di trasferimento} o \textbf{transcaratteristica} con \(I_{DS}\) in funzione di \(V_{GS}\) per valori costanti di \(V_{DS}\)
\end{itemize}

\begin{center}
	\begin{minipage}{0.4\textwidth}
		\centering \includegraphics[width=0.9\textwidth]{immagini/4_mosfet/mosfet_7.png}
	\end{minipage}
	\begin{minipage}{0.55\textwidth}
		Per analizzare la curve caratteristiche si collega il mosfet ad un circuito di test con due generatori di tensione \(V_{GS}\)
		e \(V_{DS}\) e si misura la corrente \(I_{DS}\) che scorre tra drain e source. Per la caratteristica di uscita si mantiene
		\(V_{GS}\) costante e si varia \(V_{DS}\), mentre per la transcaratteristica si mantiene \(V_{DS}\) costante e si varia \(V_{GS}\).
	\end{minipage}
\end{center}

\subsubsection*{Caratteristica di uscita \(I_{DS}\) - \(V_{DS}\)}
Le tre aree di funzionamento del n-mosfet si riflettono nella caratteristica di uscita \(I_{DS}\) - \(V_{DS}\):
\begin{itemize}
	\item \textbf{interdizione} per \(V_{GS} < V_{TN}\): linea orizzontale coincidente con l'asse delle ascisse
	\item \textbf{conduzione lineare} per \(V_{DS} < V_{GS} - V_{TN}\): a sinistra della linea di saturazione
	\item \textbf{saturazione per pinchoff} per \(V_{DS} > V_{GS} - V_{TN}\): a destra della linea di saturazione
	\item \textbf{curva di saturazione}: separa la regione di funzionamento lineare e quella di saturazione, è costituita dai
	punti \((V_{DS}, I_{DS})\) che soddisfano l'equazione:
\end{itemize}

\begin{center}
	\begin{minipage}{0.45\textwidth}
		\centering \includegraphics[width=0.9\textwidth]{immagini/4_mosfet/mosfet_8.png}
	\end{minipage}
	\begin{minipage}{0.5\textwidth}
		\[I_{DS,sat} = \frac{k_n}{2} {V_{DS}}^2 \;\;\text{con}\;\; V_{DS} = V_{GS} - V_{TN}\]
		\vspace{10pt}

		\begin{itemize}
			\item aumentando \(V_{DS}\), il transistor passa da regime lineare a regime di saturazione per pinchoff
			\item aumentando \(V_{GS}\) (\(V_{G4} > V_{G3} > V_{G2} > V_{G1} > V_{TN}\)) la corrente \(I_{DS}\) aumenta
		\end{itemize}
		\vspace{20pt}
	\end{minipage}
\end{center}

\subsubsection*{Transcaratteristica \(I_{DS}\) - \(V_{GS}\)}
Le tre aree di funzionamento del n-mosfet si riflettono nella caratteristica di uscita \(I_{DS}\) - \(V_{GS}\):
\begin{itemize}
	\item \textbf{interdizione} per \(V_{GS} < V_{TN}\): linea orizzontale coincidente con l'asse delle ascisse
	\item \textbf{saturazione per pinchoff} per \(V_{TN} <  V_{GS} < V_{DS} + V_{TN}\) crescita quadratica
	\item \textbf{conduzione lineare} per \(V_{DS} + V_{TN}< V_{GS}\) crescita lineare
\end{itemize}

\begin{center}
	\begin{minipage}{0.45\textwidth}
		\centering \includegraphics[width=0.9\textwidth]{immagini/4_mosfet/mosfet_9.png}
	\end{minipage}
	\begin{minipage}{0.5\textwidth}
		\begin{itemize}
			\item aumentando \(V_{GS}\), il transistor passa da regime di interdizione a regime di saturazione per pinchoff
			e successivamente a regime lineare
			\item aumentando \(V_{DS}\) il confine tra regime di saturazione e regime lineare si sposta verso destra, inoltre
			aumenta la pendenza della retta in regime lineare
		\end{itemize}
		\vspace{20pt}
	\end{minipage}
\end{center}

\newpage

\subsection{Modello a canale corto di un n-mosfet}
\subsubsection*{Equazione generale per \(I_{DS}\)}
Il modello a canale corto tiene conto di tutti i fenomeni fisici che avvengono in un mosfet reale, tra cui la modulazione di
lunghezza di canale e la saturazione di velocità. Tutte le aree di funzionamento del n-mosfet si possono descrivere con un'unica
equazione per la corrente \(I_{DS}\):
\begin{itemize}
	\item per \(V_{GS} < V_{TN}\) mosfet in interdizione e \(I_{DS} = 0\)
	\item per \(V_{GS} > V_{TN}\) mosfet in conduzione con \(I_{DS}\) che dipende da \(V_{MIN}\):
	\begin{align*}
		I_{DS} &= k_n' Z_n V_{MIN} \left(V_{GS} - V_{TN} - \frac{V_{MIN}}{2}\right) (1 + \lambda_n V_{DS}) \qquad k_n' = \frac{\mu_n \varepsilon}{T_{OX}} \qquad Z_n = \frac{W_n}{L_n} \\
		V_{MIN} &= \min \left\{\begin{array}{l l}
			V_{DS} & \text{regime lineare} \\
			V_{GS} - V_{TN} & \text{saturazione per pinchoff} \\
			V_{DSATN} & \text{saturazione per velocità}
		\end{array} \right\}
	\end{align*}
\end{itemize}

\subsubsection*{Caratteristica di uscita del modello a canale corto}
Includendo anche gli effetti della saturazione di velocità, la caratteristica di uscita complessiva del modello a canale corto
del n-mosfet risulta come segue:
\begin{center}
	\includegraphics[width=0.7\textwidth]{immagini/4_mosfet/mosfet_18.png}
\end{center}

\begin{itemize}
	\item la regione di conduzione lineare e la regione di saturazione per pinchoff sono separate dalla curva di saturazione per pinchoff
	data dalla parabola \(I_{DS} = k_n / 2 \cdot {V_{DS}}^2\) con \(V_{DS} = V_{GS} - V_{TN}\)
	\item la regione di conduzione lineare e la regione di saturazione per velocità sono separate dalla linea verticale \(V_{DS} = V_{DSATN}\)
	\item la regione di saturazione per pinchoff e la regione di saturazione per velocità sono separate dalla linea orizzontale
	\(I_{DS} = k_n / 2 \cdot {V_{DS}}^2\) con \(V_{DS} = V_{DSATN}\)
\end{itemize}

\newpage

\subsection{Struttura di un p-mosfet e vincoli sui potenziali}
Il funzionamento di un mosfet a canale p (p-mosfet) con substrato di tipo n è analogo a quello del n-mosfet, ma con le polarità
invertite. Tutte le tensioni, infatti, sono negative. Si assume per convenzione che il terminale di source ha potenziale
maggiore di quello di drain (\(V_S > V_D\)) e, di conseguenza, che la corrente scorre dal source al drain rimanendo sempre
positiva (\(I_{DS} > 0\)).

\begin{center}
	\includegraphics[width=0.6\textwidth]{immagini/4_mosfet/mosfet_10.png}
\end{center}

\subsubsection*{Condizioni all'equilibrio}
In analogia al n-mosfet, all'equilibrio le due giunzioni pn tra substrato n e le regioni p\(^+\) del source e del drain sono in
equilibrio e la regione di svuotamento si estende quasi totalmente nel substrato n.

\subsubsection*{Vincoli di polarizzazione dei diodi e potenziale di substrato}
I due diodi con anodi collegati ai nodi source e drain e con catodo in comune nel substrato, devono rimanere in interdizione
per il corretto funzionamento del mosfet, si ottengono le seguenti condizioni:
\begin{itemize}
	\item \(V_{BS} \geq 0 \rightarrow V_S \leq V_B\) (diodo source-substrato in interdizione)
	\item \(V_{BD} \geq 0 \rightarrow V_D \leq V_B\) (diodo drain-substrato in interdizione)
\end{itemize}
Si ottiene che il substrato \(V_B\) deve essere il nodo a potenziale più alto \(V_D \leq V_S \leq V_B\). Di solito si
collega il substrato al potenziale maggiore dell'intero circuito (alimentazione) \(V_B = V_{DD}\) oppure al source \(V_B = V_S\).

\textit{NOTA}: Non è possibile collegare il substrato al drain perché si violerebbe la condizione di interdizione del diodo
source-substrato in quanto \(V_S > V_D\) e quindi \(V_{BS} = V_B - V_S = V_D - V_S < 0\).

\subsubsection*{Potenziale e tensioni di gate}
Analogamente al n-mosfet, il potenziale di gate \(V_G\) controlla l'area di lavoro del p-mosfet. La tensione tra le due armature
è variabile lungo la lunghezza del canale ed è compresa tra le tensioni di gate-source e gate-drain ai margini del canale:
\begin{itemize}
	\item in prossimità del source \(V_{C,\text{source}} = V_{GS} = V_G - V_S\)
	\item in prossimità del drain \(V_{C,\text{drain}} = V_{GD} = V_G - V_D\)
\end{itemize}
\textit{NOTA}: siccome \(0 > V_S \geq V_D\), si ha \(V_{GS} \leq V_{GD}\) (oppure \(|V_{GS}| \geq |V_{GD}|\))

\newpage

\subsection{Aree di lavoro di un p-mosfet}
Le aree di lavoro del p-mosfet sono analoghe a quelle del n-mosfet, ma con le polarità invertite:
\begin{itemize}
	\item \textbf{p-mosfet spento o in interdizione} per \(0 > V_{GS} > V_{TP}\): \\
	il condensatore è in regime di svuotamento e non c'è corrente tra drain e source \(I_{DS} = 0\)
	\item \textbf{p-mosfet in conduzione lineare} per \(V_{DS} > V_{GS} - V_{TP} \; \Leftrightarrow \; 0 > V_{TP} > V_{GD} > V_{GS}\): \\
	si ha una corrente \(I_{DS}\) tra source e drain data dal movimento delle lacune nel canale di tipo p
	\[I_{DS} = k_p V_{DS} \left(V_{GS} - V_{TP} - \frac{V_{DS}}{2}\right) \qquad k_p = k_p' \cdot Z_p \qquad k_p' = \mu_p \varepsilon / T_{OX} \qquad Z_p = W_p/Z_p\]
	\item \textbf{p-mosfet in saturazione per pinchoff} per \(V_{DS} < V_{GS} - V_{TP} \; \Leftrightarrow \; 0 > V_{GD} > V_{TP} > V_{GS}\): \\
	il condensatore è in regime misto e si ha una corrente \(I_{DS}\) costante rispetto a \(V_{DS}\):
	\[I_{DS} = \frac{k_p}{2} (V_{GS} - V_{TP})^2\]
	\item \textbf{Modulazione di lunghezza di canale}: \\
	Analoga a quella dell'n-mosfet, applicando il fattore correttivo si ottiene:
	\[I_{DS} = \frac{k_p}{2} (V_{GS} - V_{TP})^2 (1 + \lambda_p V_{DS}) \qquad \text{con} \; V_{DS} < 0, \; \lambda_p < 0\]
	\item \textbf{p-mosfet in saturazione di velocità} per \(V_{DS} < V_{DSATP}\): \\
	Analogo a quello dell' n-mosfet, si ha una corrente \(I_{DS}\) costante rispetto a \(V_{DS}\):
	\[I_{DS} = k_p V_{DSATP} \left(V_{GS} - V_{TP} - \frac{V_{DSATP}}{2}\right) (1 + \lambda_p V_{DS}) \qquad \text{con} \; V_{DSATP} < 0\]
	(alcuni valori tipici per p-mosfet: \(\mu_p = 200 \; \cm^2 / \V\s \quad v_{sat,p} \approx 4 \cdot 10^6 \; \cm/\s \quad E_C \approx 2 \; \V/\cm\))
\end{itemize}

\subsection{Curve caratteristiche di corrente-tensione di un p-mosfet}
Allo stesso modo dell'n-mosfet, si definiscono le curve caratteristiche di un p-mosfet, studiate attraverso un circuito simile:

\begin{center}
	\begin{minipage}{0.35\textwidth}
		\centering \includegraphics[width=\textwidth]{immagini/4_mosfet/mosfet_11.png}
	\end{minipage}
	\begin{minipage}{0.63\textwidth}
		\begin{itemize}
			\item \textbf{caratteristica di uscita} con \(I_{DS}\) in funzione di \(V_{DS}\) per valori costanti di \(V_{GS}\)
			\item \textbf{caratteristica di trasferimento} o \textbf{transcaratteristica} con \(I_{DS}\) in funzione di \(V_{GS}\) per valori costanti di \(V_{DS}\)
		\end{itemize}
	\end{minipage}
\end{center}

\subsubsection*{Caratteristica di uscita \(I_{DS}\) - \(V_{DS}\)}
Rispetto al n-mosfet, la caratteristica di uscita \(I_{DS}\) - \(V_{DS}\) del p-mosfet è speculare rispetto all'asse delle ordinate,
mantenendo invariate le forme delle curve e il posizionamento delle aree di funzionamento:
\begin{center}
	\begin{minipage}{0.4\textwidth}
		\centering \includegraphics[width=0.9\textwidth]{immagini/4_mosfet/mosfet_12.png}
	\end{minipage}
	\begin{minipage}{0.58\textwidth}
		\begin{itemize}
			\item \textbf{interdizione} per \(V_{GS} > 0\): linea orizzontale coincidente con l'asse delle ascisse
			\item \textbf{conduzione lineare} per \(V_{DS} > V_{GS} - V_{TP}\): a destra della linea di saturazione
			\item \textbf{saturazione per pinchoff} per \(V_{DS} < V_{GS} - V_{TP}\): a sinistra della linea di saturazione
			\item \textbf{curva di saturazione}: separa la regione di funzionamento lineare e quella di saturazione, è costituita
			dai punti in cui \(V_{DS} = V_{GS} - V_{TP}\)
		\end{itemize}
	\end{minipage}
\end{center}

\subsubsection*{Transcaratteristica \(I_{DS}\) - \(V_{GS}\)}
Rispetto al n-mosfet, la transcaratteristica \(I_{DS}\) - \(V_{GS}\) del p-mosfet è speculare rispetto all'asse delle ordinate,
mantenendo invariate le forme delle curve e il posizionamento delle aree di funzionamento:

\begin{center}
	\begin{minipage}{0.4\textwidth}
		\centering \includegraphics[width=0.9\textwidth]{immagini/4_mosfet/mosfet_13.png}
	\end{minipage}
	\begin{minipage}{0.58\textwidth}
		\begin{itemize}
			\item \textbf{interdizione} per \(V_{GS} > V_{TP}\): linea orizzontale coincidente con l'asse delle ascisse
			\item \textbf{saturazione per pinchoff} per \(V_{DS} + V_{TP} < V_{GS} < V_{TP}\): crescita quadratica
			\item \textbf{conduzione lineare} per \(V_{GS} < V_{DS} + V_{TP}\): crescita lineare
		\end{itemize}
		\vspace{20pt}
	\end{minipage}
\end{center}

\subsection{Modello a canale corto di un p-mosfet}
\subsubsection*{Equazione generale per \(I_{DS}\)}
Anche per il p-mosfet si può utilizzare un'unica equazione per la corrente \(I_{DS}\) che tiene conto di tutti i fenomeni fisici:
\begin{itemize}
	\item per \(V_{GS} > V_{TP}\) mosfet in interdizione e \(I_{DS} = 0\)
	\item per \(V_{GS} < V_{TP}\) mosfet in conduzione con \(I_{DS}\) che dipende da \(V_{MAX}\):
	\begin{align*}
		I_{DS} &= k_p' Z_p V_{MAX} \left(V_{GS} - V_{TP} - \frac{V_{MAX}}{2}\right) (1 + \lambda_p V_{DS}) \qquad k_p' = \frac{\mu_p \varepsilon}{T_{OX}} \qquad Z_p = \frac{W_p}{L_p} \\
		V_{MAX} &= \max \left\{\begin{array}{l l}
			V_{DS} & \text{regime lineare} \\
			V_{GS} - V_{TP} & \text{saturazione per pinchoff} \\
			V_{DSATP} & \text{saturazione per velocità}
		\end{array} \right\}
	\end{align*}
\end{itemize}

\subsubsection*{Caratteristica di uscita del modello a canale corto}
La caratteristica di uscita complessiva del modello a canale corto del p-mosfet risulta come segue:
\begin{center}
	\includegraphics[width=0.65\textwidth]{immagini/4_mosfet/mosfet_19.png}
\end{center}
\begin{itemize}
	\item la regione di conduzione lineare e la regione di saturazione per pinchoff sono separate dalla curva di saturazione per pinchoff
	data dalla parabola \(I_{DS} = k_p / 2 \cdot {V_{DS}}^2\) con \(V_{DS} = V_{GS} - V_{TP}\)
	\item la regione di conduzione lineare e la regione di saturazione per velocità sono separate dalla linea verticale \(V_{DS} = V_{DSATP}\)
	\item la regione di saturazione per pinchoff e la regione di saturazione per velocità sono separate dalla linea orizzontale
	\(I_{DS} = k_p / 2 \cdot {V_{DS}}^2\) con \(V_{DS} = V_{DSATP}\)
\end{itemize}

\newpage

\subsection{Simbologia e rappresentazione circuitale dei mosfet}
\subsubsection*{Simbologia a 4 terminali - con substrato}
Nelle simbologie a 4 terminali non è possibile identificare univocamente il drain e il source in quanto sono perfettamente
identici. Nelle rappresentazioni a freccia, la freccia sul terminale del substrato indica il tipo di mosfet ed è concorde
con il flusso di corrente nel diodo formato tra substrato e source/drain.

\begin{center}
	\begin{minipage}{0.45\textwidth}
		\centering
		\begin{minipage}{0.4\textwidth}
			\centering \includegraphics[width=0.8\textwidth]{immagini/4_mosfet/nmos_1.png}
			\small{n-mosfet}
		\end{minipage}
		\begin{minipage}{0.4\textwidth}
			\centering \includegraphics[width=0.8\textwidth]{immagini/4_mosfet/pmos_1.png}
			\small{p-mosfet}
		\end{minipage}
		\vspace{7pt}

		\small{rappresentazione a freccia}
	\end{minipage}
	\begin{minipage}{0.45\textwidth}
		\centering
		\begin{minipage}{0.4\textwidth}
			\centering \includegraphics[width=0.8\textwidth]{immagini/4_mosfet/nmos_2.png}
			\small{n-mosfet}
		\end{minipage}
		\begin{minipage}{0.4\textwidth}
			\centering \includegraphics[width=0.8\textwidth]{immagini/4_mosfet/pmos_2.png}
			\small{p-mosfet}
		\end{minipage}
		\vspace{7pt}

		\small{rappresentazione digitale}
	\end{minipage}
\end{center}

\subsubsection*{Simbologia a 3 terminali - senza substrato}
Nelle simbologie a 3 terminali, il substrato viene omesso in quanto collegato al source o ad un potenziale di riferimento.
Quando il substrato è collegato al source, il terminale source è identificato con la freccia concorde al flusso di corrente
tra source e drain. Quando il substrato è collegato ad un potenziale di riferimento, il terminale source non è
identificabile univocamente in quanto non ci sono freccie.

\begin{center}
	\begin{minipage}{0.45\textwidth}
		\centering
		\begin{minipage}{0.4\textwidth}
			\centering \includegraphics[width=0.8\textwidth]{immagini/4_mosfet/nmos_3.png}
			\small{n-mosfet}
		\end{minipage}
		\begin{minipage}{0.4\textwidth}
			\centering \includegraphics[width=0.8\textwidth]{immagini/4_mosfet/pmos_3.png}
			\small{p-mosfet}
		\end{minipage}
		\vspace{7pt}

		\small{substrato collegato al source}
	\end{minipage}
	\begin{minipage}{0.45\textwidth}
		\centering
		\begin{minipage}{0.4\textwidth}
			\centering \includegraphics[width=0.9\textwidth]{immagini/4_mosfet/nmos_4.png}
			\small{n-mosfet}
		\end{minipage}
		\begin{minipage}{0.4\textwidth}
			\centering \includegraphics[width=0.9\textwidth]{immagini/4_mosfet/pmos_4.png}
			\small{p-mosfet}
		\end{minipage}
		\vspace{7pt}

		\small{substrato collegato a potenziale di riferimento}
	\end{minipage}
\end{center}

\vspace{10pt}

\subsection{Struttura reale del mosfet}
Per ottimizzare le prestazioni, lo spazio e il processo produttivo vengono apportate alcune modifiche al nodo di substrato (B):
\begin{itemize}
	\item al posto di trovarsi sotto, il substrato viene realizzato sulla parte superiore del transistor a lato del 
	source o del drain per facilitare il processo di fabbricazione
	\item in prossimità dell'elettrodo, il substrato viene ulteriormente drogato dello stesso tipo del substrato (p\(^+\) in
	n-mosfet o n\(^+\) in p-mosfet) per ridurre la resistenza di contatto
\end{itemize}
Di seguito sono riportate le illustrazioni delle strutture reali di un n-mosfet e di un p-mosfet:

\begin{center}
	\includegraphics[width=0.8\textwidth]{immagini/4_mosfet/mosfet_14.png}
\end{center}

\newpage

\subsection{Effetto Body e variazione della tensione di soglia}
Quando il substrato non è collegato al source, ma ad un potenziale di riferimento (massa o alimentazione), la tensione
tra substrato e source \(V_{BS}\) può essere diversa da zero. Questa tensione fa variare la tensione di soglia del mosfet
secondo le relazioni (in base al tipo di mosfet):
\begin{align*}
	V_{TN} &= V_{TN0} + \gamma_n \left( \sqrt{V_{SB} + 2\phi_n} - \sqrt{2\phi_n} \right) \qquad& \gamma_n &= \frac{\sqrt{2q N_D \varepsilon_{Si}}}{C_{OX}} & \phi_n &= \frac{k_B T}{q} \ln \left( \frac{N_A}{n_i} \right) \\
	V_{TP} &= V_{TP0} - \gamma_p \left( \sqrt{V_{BS} + 2\phi_p} - \sqrt{2\phi_p} \right) \qquad& \gamma_p &= \frac{\sqrt{2q N_A \varepsilon_{Si}}}{C_{OX}} & \phi_p &= \frac{k_B T}{q} \ln \left( \frac{N_D}{n_i} \right)
\end{align*}
\begin{itemize}
	\item \(V_{TN0}\) e \(V_{TP0}\): tensione di soglia per \(V_{SB} = 0\)
	\item \(\gamma\) e \(\phi\): parametri legati al drogaggio e allo spessore dell'ossido
	\item \(C_{OX}\): capacità per unità di area dell'ossido
\end{itemize}
Rappresentando graficamente la variazione della tensione di soglia in funzione di \(V_{SB}\), si ottiene:
\begin{center}
	\begin{minipage}{0.48\textwidth}
		\centering \includegraphics[width=0.7\textwidth]{immagini/4_mosfet/mosfet_15.png}

		\small{Variazione di \(V_{TN}\) per n-mosfet}
	\end{minipage}
	\begin{minipage}{0.48\textwidth}
		\centering \includegraphics[width=0.75\textwidth]{immagini/4_mosfet/mosfet_16.png}

		\small{Variazione di \(V_{TP}\) per p-mosfet}
	\end{minipage}
\end{center}

\vspace{10pt}

\subsection{Corrente di sottosoglia}
La corrente di sottosoglia è una piccola corrente che scorre tra drain e source anche quando il mosfet è in interdizione
(\(V_{GS} < V_{T}\)). Questa corrente è molto debole e vale:
\begin{align*}
	I_{DS} &= I_{0n} e^{\frac{V_{GS} - V_{TN}}{n V_T}} \left( 1 - e^{-\frac{V_{DS}}{n V_T}} \right) & \text{per n-mosfet} \\
	I_{DS} &= I_{0p} e^{- \frac{V_{GS} - V_{TP}}{n V_T}} \left( 1 - e^{\frac{V_{DS}}{n V_T}} \right) & \text{per p-mosfet}
\end{align*}
Analizzando la transcaratteristica per un n-mosfet si osserva che la corrente di sottosoglia decresce esponenzialmente per
\(V_{GS} < V_{TN}\) (siccome le ordinate sono in scala logaritmica, la curva appare lineare):
\begin{center}
	\includegraphics[width=0.6\textwidth]{immagini/4_mosfet/mosfet_20.png}
\end{center}

\newpage

\subsection{Capacità parassite dei mosfet}
\label{capacità_parassite_mosfet}
Avendo numerose giunzioni pn che si vengono a formare, i mosfet presentano delle capacità parassite che influenzano il loro
funzionamento alle alte frequenze, come evidenziato in figura. Le capacità parassite sono indipendenti dal tipo di mosfet
(n-mosfet o p-mosfet), per cui le conclusioni ottenute valgono sia per un n-mosfet che per un p-mosfet.

\begin{center}
	\centering \includegraphics[width=0.45\textwidth]{immagini/4_mosfet/mosfet_21.png}
\end{center}

Le capacità parassite per ogni coppia di terminali sono:
\begin{itemize}
	\item \(C_{sb}\): capacità source-substrato \(C_{sb} \approx C_{j0} \cdot L_D \cdot W\)
	\item \(C_{db}\): capacità drain-substrato \(C_{db} \approx C_{j0} \cdot L_D \cdot W\)
	\item \(C_{gs}\): capacità gate-source \(C_{gs} \approx C_{gs0} \cdot W\)
	\item \(C_{gd}\): capacità gate-drain \(C_{gd} \approx C_{gd0} \cdot W\)
	\item \(C_{gc}\): capacità gate-substrato (del condensatore mos) \(C_{gc} = C_{OX} \cdot L \cdot W\)
\end{itemize}

\vspace{10pt}

\noindent
Le capacità parassite complessive di ogni nodo rispetto al substrato si calcolano come capacità equivalente della rete tra
il nodo in esame e il substrato, ottenuta cortocircuitando tutti gli altri nodi al substrato. Di seguito sono riportate le
espressioni delle capacità parassite totali:
\begin{align*}
	C_{source} &= C_{sb} + C_{gs} = (C_{j0} L_D + C_{gs0}) \cdot W = C_{s0} \cdot W & & \text{con} \; C_{s0} = C_{j0} L_D + C_{gs0} \\
	C_{drain} &= C_{db} + C_{gd} = (C_{j0} L_D + C_{gd0}) \cdot W = C_{d0} \cdot W & & \text{con} \; C_{d0} = C_{j0} L_D + C_{gd0} \\
	C_{gate} &= C_{gs} + C_{gd} + C_{gc} = (C_{gs0} + C_{gd0} + C_{OX} L) \cdot W \approx C_{g0} \cdot W L & &\text{per} \; C_{gs} + C_{gd} \ll C_{OX}L
\end{align*}

\noindent
NOTA: Dall'analisi dei circuiti RC in cui si hanno due condensatori \(C_1\) e \(C_2\) con un terminale in comune e l'altro collegato
rispettivamente a massa e a \(V_{DD}\), si osserva che la capacità equivalente tra i due nodi è data dalla somma delle due
capacità \(C_{eq} = C_1 + C_2\). Questo avviene anche se effettivamente i due condensatori non sono collegati in parallelo,
in quanto la corrente che arriva al nodo comune si divide tra i due condensatori in funzione della loro capacità ed è come
se fossero collegati in parallelo.

%\section{Circuiti con i MOSFET}
\subsection{MOSFET in serie a una resistenza}
\subsubsection*{Circuito NMOSFET e resistenza}
\begin{center}
	\begin{minipage}{0.3\textwidth}
		\centering \begin{circuitikz}[american]
			\ctikzset{tripoles/mos style=arrows}
		
			% =================================================================
			% 1. Nodi principali del circuito
			\coordinate (GND_left) at (0, 0);     % Massa a sinistra
			\coordinate (G_mosfet) at (2, 0);  % gate del MOSFET
		
			% =================================================================
			% 2. Circuito
			\draw (GND_left) node[sground]{}; % Simbolo di massa a sinistra
			\draw (G_mosfet) to [V, l_=\(V_G\)] (GND_left); % Generatore di tensione V_G
			\draw (G_mosfet) node[nmos, anchor=G](nmos){}; % NMOSFET
			\draw (nmos.S) node[sground]{}; % source del MOSFET
			\draw (nmos.D) -- ++(0, 0.2) to [R, l=\(R\), name=R1] ++(0, 1.5) node[rground, rotate=180, name=VDD]{}; % resistenza R e VDD
		
			% =================================================================
			% 3. Etichette e Vettori Corrente/Tensione
		
			% V_GS (Gate-Source Voltage)
			\node at ($(nmos.G)+(0, -0.2)$) {\(+\)};
			\node at ($(nmos.S)+(-0.25, 0.1)$) {\(-\)};
			\node at ($(nmos.G)+(0.2, -0.6)$) {\(V_{GS}\)}; % Testo V_GS sopra i segni
		
			% V_DS (Drain-Source Voltage)
			\node at ($(nmos.D)+(0.3, -0.2)$) {\(+\)};
			\node at ($(nmos.S)+(0.3, 0.2)$) {\(-\)};
			\node at ($(nmos.G)+(1.4, 0)$) {\(V_{DS}\)}; % Testo V_DS sopra i segni
		
			% V_R (Tensione sulla resistenza R)
			\node at ($(R1)+(0.4, 0.5)$) {\(+\)};
			\node at ($(R1)+(0.4, -0.5)$) {\(-\)};
			\node at ($(R1)+(0.5, 0)$) {\(V_R\)}; % Testo V_R sopra i segni
		
			% I_DS Corrente Drain-Source
			\node at ($(nmos.D)+(-0.5, -0.1)$) {\(I_{DS} \downarrow\)}; % Testo I_DS vicino al drain

			% VDD
			\node at ($(VDD)+(0, 0.6)$) {\(V_{DD}\)};
		\end{circuitikz}
	\end{minipage}
	\begin{minipage}{0.57\textwidth}
		Si costruisce un circuito con un NMOSFET e una resistenza come illustrato di lato, in cui:
		\begin{itemize}
			\item il terminale di drain è collegato a una tensione di alimentazione \(V_{DD}\) tramite una resistenza di carico \(R\);
			\item il terminale di source è collegato a massa;
			\item il terminale di gate è collegato a un generatore di tensione \(V_G\).
		\end{itemize}
		L'utilizzo di un NMOSFET al posto di un PMOSFET è perfettamente arbitrario e non cambia nulla nel procedimento di
		soluzione del circuito.
	\end{minipage}
\end{center}

\subsubsection*{Soluzione del circuito per via grafica}
La risoluzione del circuito consiste nel trovare la corrente \(I_{DS}\) e la tensione \(V_{DS}\) che soddisfano sia la
caratteristica del MOSFET sia la caratteristica della resistenza \(R\). Dalle LKC si impone \(I_{DS} = I_R\) e si risolve
l'equazione in funzione di \(V_{DS}\).

Un'alternativa alla risoluzione analitica è la risoluzione grafica, che consiste nel tracciare sullo stesso grafico la
caratteristica di uscita del MOSFET (\(I_{DS}\) in funzione di \(V_{DS}\)) e la caratteristica della resistenza \(R\)
sempre in funzione di \(V_{DS}\) e trovare il punto di intersezione tra le due curve.

\begin{center}
	\begin{minipage}{0.45\textwidth}
		\centering \includegraphics[width=0.9\textwidth]{immagini/5_circuiti_mosfet/circuiti_mos_1.png}
	\end{minipage}
	\begin{minipage}{0.45\textwidth}
		Un'alternativa alla risoluzione analitica è la risoluzione grafica, che consiste nel tracciare sullo stesso grafico la
		caratteristica di uscita del MOSFET (\(I_{DS}\) in funzione di \(V_{DS}\)) e la caratteristica della resistenza \(R\)
		sempre in funzione di \(V_{DS}\) e trovare il punto di intersezione tra le due curve.
	\end{minipage}
\end{center}



\subsubsection*{Modello a canale lungo}
Il modello a canale lungo del MOSFET prevede che il transistor operi in una delle tre regioni (interdizione, lineare o saturazione
per pinchoff) senza considerare gli effetti di modulazione di lunghezza di canale e saturazione di velocità. Per risolvere il
circuito si procede come segue:
\begin{enumerate}
	\item si determina se il MOSFET è \say{acceso} o \say{spento} confrontando \(V_{GS}\) con la tensione di soglia \(V_{TN}\):
	\begin{itemize}[topsep=0pt]
		\item se \(V_{GS} < V_{TN}\), il MOSFET è in interdizione e \(I_{DS} = 0\)
		\item se \(V_{GS} \geq V_{TN}\), il MOSFET è acceso e si procede al passo successivo
	\end{itemize}
	\item si ipotizza che il MOSFET sia in uno dei due regimi di conduzione (lineare o saturazione) e si risolve la rete ponendo
	\(I_{DS} = I_R\) (per le LKC), utilizzando la formula \(I_{DS}\) corrispondente all'ipotesi fatta e si risolve per \(V_{DS}\):
	\begin{itemize}[topsep=0pt]
		\item se si ipotizza il regime lineare: \(\displaystyle k_n V_{DS} \left(V_{GS} - V_{TN} - \frac{V_{DS}}{2}\right) = \frac{V_{DD} - V_{DS}}{R}\)
		\item se si ipotizza il regime di saturazione: \(\displaystyle \frac{k_n}{2} (V_{GS} - V_{TN})^2 = \frac{V_{DD} - V_{DS}}{R}\)
	\end{itemize}
	\item si verifica se l'ipotesi fatta sul regime di funzionamento è corretta:
	\begin{itemize}[topsep=0pt]
		\item se si è ipotizzato il regime lineare, si verifica che \(V_{DS} < V_{GS} - V_{TN}\);
		\item se si è ipotizzato il regime di saturazione, si verifica che \(V_{DS} \geq V_{GS} - V_{TN}\).
	\end{itemize}
	\item se l'ipotesi è corretta, si è trovata la soluzione; altrimenti si ripete il passo 2 con l'altra ipotesi.
\end{enumerate}

\subsubsection*{Modello a canale corto}
La soluzione del circuito con il modello a canale corto è analoga a quella con il modello a canale lungo, con la differenza che
si prevede anche l'effetto di modulazione di lunghezza di canale e la possibilità che il MOSFET lavori in saturazione di velocità.
In particolare si procede come segue:
\begin{enumerate}
	\item si determina se il MOSFET è \say{acceso} o \say{spento} come nel modello a canale lungo;
	\item si calcolano, se possibile, i tre valori di \(V_{MIN}\) (\(V_{DS}, V_{GS} - V_{TN}, V_{DSATN}\)) e se si conoscono già
	due dei valori (esempio \(V_{GS}-V_{TN}\) e \(V_{DSATN}\)), si può escludere a priori quello maggiore (ovvero quello che
	sicuramente non sarà il minimo), riducendo così il numero di ipotesi da fare;
	\item si ipotizza che il MOSFET sia in uno dei tre regimi di conduzione (lineare, saturazione per pinchoff o saturazione di
	velocità) escludendone, se possibile, uno come spiegato al passo precedente, e si risolve la rete come nel modello a canle lungo:
	\begin{itemize}[topsep=0pt]
		\item in regime lineare: \(\displaystyle k_n V_{DS} \left(V_{GS} - V_{TN} - \frac{V_{DS}}{2}\right) (1 + \lambda V_{DS}) = \frac{V_{DD} - V_{DS}}{R}\)
		\item in saturazione per pinchoff: \(\displaystyle \frac{k_n}{2} (V_{GS} - V_{TN})^2 (1 + \lambda V_{DS}) = \frac{V_{DD} - V_{DS}}{R}\)
		\item in saturazione di velocità: \(\displaystyle k_n V_{DSATN} \left(V_{GS} - V_{TN} - \frac{V_{DSATN}}{2}\right) (1 + \lambda V_{DS}) = \frac{V_{DD} - V_{DS}}{R}\)
	\end{itemize}
	\item si verifica se l'ipotesi fatta sul regime di funzionamento è corretta:
	\begin{itemize}[topsep=0pt]
		\item se si è ipotizzato il regime lineare: \(V_{DS} = \min \{V_{GS} - V_{TN}, V_{DSATN}, V_{DS}\}\);
		\item se si è ipotizzato il regime di sat. per pinchoff: \(V_{GS} - V_{TN} = \min \{V_{GS} - V_{TN}, V_{DSATN}, V_{DS}\}\);
		\item se si è ipotizzato il regime di sat. di velocità: \(V_{DSATN} = \min \{V_{GS} - V_{TN}, V_{DSATN}, V_{DS}\}\);
	\end{itemize}
	\item se l'ipotesi è corretta, si è trovata la soluzione; altrimenti si ripete il passo 3 con un'altra ipotesi.
\end{enumerate}

\subsection{MOSFET connesso a diodo}
\subsubsection*{Circuito di un NMOSFET connesso a diodo}
\begin{center}
	\begin{minipage}{0.3\textwidth}
		\centering \begin{circuitikz}[american]
			\ctikzset{tripoles/mos style=arrows}
		
			% =================================================================
			% 1. Nodi principali del circuito
			\coordinate (G_mosfet) at (0, 0);  % gate del MOSFET
		
			% =================================================================
			% 2. Circuito
			
			\draw (G_mosfet) node[nmos, anchor=G](nmos){}; % NMOSFET
			\draw (G_mosfet) -- ++(-0.2, 0) -- ++(0,1.1) -- ++(1.18,0);
			\draw (nmos.S) node[sground]{}; % source del MOSFET
			\draw (nmos.D) -- ++(0, 0.4) to [R, l=\(R\), name=R1] ++(0, 1.3) node[rground, rotate=180, name=VDD]{}; % resistenza R e VDD
		
			% =================================================================
			% 3. Etichette e Vettori Corrente/Tensione
		
			% V_GS (Gate-Source Voltage)
			\node at ($(nmos.G)+(0, -0.2)$) {\(+\)};
			\node at ($(nmos.S)+(-0.25, 0.1)$) {\(-\)};
			\node at ($(nmos.G)+(0.2, -0.6)$) {\(V_{GS}\)}; % Testo V_GS sopra i segni
		
			% V_DS (Drain-Source Voltage)
			\node at ($(nmos.D)+(0.3, -0.2)$) {\(+\)};
			\node at ($(nmos.S)+(0.3, 0.2)$) {\(-\)};
			\node at ($(nmos.G)+(1.4, 0)$) {\(V_{DS}\)}; % Testo V_DS sopra i segni
		
			% V_R (Tensione sulla resistenza R)
			\node at ($(R1)+(0.4, 0.5)$) {\(+\)};
			\node at ($(R1)+(0.4, -0.5)$) {\(-\)};
			\node at ($(R1)+(0.5, 0)$) {\(V_R\)}; % Testo V_R sopra i segni
		
			% I_DS Corrente Drain-Source
			\node at ($(nmos.D)+(-0.5, -0.1)$) {\(I_{DS} \downarrow\)}; % Testo I_DS vicino al drain

			% VDD
			\node at ($(VDD)+(0, 0.6)$) {\(V_{DD}\)};
		\end{circuitikz}
	\end{minipage}
	\begin{minipage}{0.57\textwidth}
		Si costruisce un circuito con un NMOSFET e una resistenza come illustrato di lato, in cui:
		\begin{itemize}
			\item il terminale di source è collegato a massa;
			\item il terminale di drain è collegato a una tensione di alimentazione \(V_{DD}\) tramite una resistenza di carico \(R\);
			\item il terminale di gate è cortocircuitato al terminale di drain e \(V_{GS} = V_{DS}\)
		\end{itemize}
		L'utilizzo di un NMOSFET al posto di un PMOSFET è perfettamente arbitrario e non cambia nulla nel procedimento di
		soluzione del circuito.
	\end{minipage}
\end{center}

\subsubsection*{Considerazioni sul regime di funzionamento e risoluzione grafica}
Si osserva che il MOSFET connesso in questo modo, se acceso, lavora in saturazione per pinchoff o in saturazione di velocità,
per cui si comporta in modo simile a un diodo ideale con soglia \(V_{TN}\):

\begin{center}
	\begin{minipage}{0.45\textwidth}
		\centering \includegraphics[width=0.9\textwidth]{immagini/5_circuiti_mosfet/circuiti_mos_2.png}
	\end{minipage}
	\begin{minipage}{0.54\textwidth}
		\begin{itemize}
			\item se \(V_{GS} < V_{TN}\), il MOSFET è spento con \(I_{DS} = 0\);
			\item se \(V_{GS} \geq V_{TN}\), il MOSFET è acceso e lavora in saturazione per pinchoff \(V_{DS} = V_{GS} > V_{GS} - V_{TN}\)
			oppure in sat. di velocità se \(V_{DS} = V_{GS} > V_{DSATN}\)
		\end{itemize}
		Si nota che la caratteristica di uscita del MOSFET in questo caso è simile a quella di un diodo con soglia \(V_{TN}\).
		\vspace{0pt}

		Come nel caso precedente, la risoluzione grafica consiste nel trovare le intersezioni delle due caratteristiche di uscita
		del MOSFET e della resistenza \(R\).
	\end{minipage}
\end{center}

\subsubsection*{Analisi del circuito}
Per risolvere il circuito si procede come segue:
\begin{enumerate}
	\item si determina se il MOSFET è \say{acceso} o \say{spento} confrontando \(V_{GS}\) con la tensione di soglia \(V_{TN}\):
	\begin{itemize}[topsep=0pt]
		\item se \(V_{GS} < V_{TN}\), il MOSFET è in interdizione e \(I_{DS} = 0\)
		\item se \(V_{GS} \geq V_{TN}\), il MOSFET è acceso e si procede al passo successivo
	\end{itemize}
	\item se si utilizza il modello a canale lungo, si utilizza \(I_{DS}\) in saturazione per pinchoff e si impone la condizione
	\(\displaystyle I_{DS} = I_R \;\;\rightarrow\;\; I_{DS} = \frac{k_n}{2} (V_{GS} - V_{TN})^2 = \frac{V_{DD} - V_{DS}}{R}\);
	non serve verificare il regime di funzionamento, in quanto il MOSFET connesso in questo modo lavora sempre in saturazione
	\item se si utilizza il modello a canale corto, bisogna ipotizzare il regime di saturazione (pinchoff o velocità) e risolvere
	il circuito utilizzando la formula corrispondente:
	\begin{itemize}[topsep=0pt]
		\item in saturazione per pinchoff: \(\displaystyle \frac{k_n}{2} (V_{GS} - V_{TN})^2 (1 + \lambda V_{DS}) = \frac{V_{DD} - V_{DS}}{R}\)
		\item in saturazione di velocità: \(\displaystyle k_n V_{DSATN} \left(V_{GS} - V_{TN} - \frac{V_{DSATN}}{2}\right) (1 + \lambda V_{DS}) = \frac{V_{DD} - V_{DS}}{R}\)
	\end{itemize}
	\item si verifica se l'ipotesi fatta sul regime di funzionamento è corretta:
	\begin{itemize}[topsep=0pt]
		\item se si è ipotizzato il regime di sat. per pinchoff: \(V_{GS} - V_{TN} = \min \{V_{GS} - V_{TN}, V_{DSATN}, V_{DS}\}\);
		\item se si è ipotizzato il regime di sat. di velocità: \(V_{DSATN} = \min \{V_{GS} - V_{TN}, V_{DSATN}, V_{DS}\}\);
	\end{itemize}
	\item se l'ipotesi è corretta, si è trovata la soluzione; altrimenti si ripete il passo 3 con un'altra ipotesi
\end{enumerate}

\subsection{MOSFET come generatore di corrente}
\subsubsection*{Circuito di un NMOSFET connesso come generatore di corrente}
Si costruisce un circuito con un NMOSFET e una resistenza come illustrato sotto, in cui:
\begin{itemize}
	\item il terminale di source è collegato a massa;
	\item il terminale di drain è collegato a una tensione di alimentazione \(V_{DD}\) tramite una resistenza di carico \(R\);
	\item il terminale di gate è cortocircuitato al terminale di source e \(V_{GS} = V_{REF}\)
\end{itemize}
Si suppone, inoltre, che il valore di \(V_{REF}\) sia tale da mantenere il MOSFET sempre acceso e in regime di saturazione per
pinchoff o in saturazione di velocità, ovvero che sia soddisfatta la condizione:
\[V_{DS} > \min \{V_{REF} - V_{TN}, V_{DSATN}\}\]

\begin{center}
	\begin{minipage}{0.4\textwidth}
		\centering \begin{circuitikz}[american]
			\ctikzset{tripoles/mos style=arrows}
			
			% =================================================================
			% 1. Nodi principali del circuito
			\coordinate (G_mosfet) at (0, 0);  % gate del MOSFET
			
			% =================================================================
			% 2. Circuito
			
			\draw (G_mosfet) node[nmos, anchor=G](nmos){}; % NMOSFET
			\draw (G_mosfet) -- ++(-0.8,0) to [V, l_=\(V_{REF}\)] ++(0,-1.2) -- ++(1.78,0);
			\draw (nmos.S) -- ++(0,-0.4) node[sground]{}; % source del MOSFET
			\draw (nmos.D) -- ++(0, 0.2) to [R, l=\(R\), name=R1] ++(0, 1.2) node[rground, rotate=180, name=VDD]{}; % resistenza R e VDD
			
			% =================================================================
			% 3. Etichette e Vettori Corrente/Tensione
			
			% V_GS (Gate-Source Voltage)
			\node at ($(nmos.G)+(0, -0.2)$) {\(+\)};
			\node at ($(nmos.S)+(-0.25, 0.1)$) {\(-\)};
			\node at ($(nmos.G)+(0.2, -0.6)$) {\(V_{GS}\)}; % Testo V_GS sopra i segni
			
			% V_DS (Drain-Source Voltage)
			\node at ($(nmos.D)+(0.3, -0.2)$) {\(+\)};
			\node at ($(nmos.S)+(0.3, 0.2)$) {\(-\)};
			\node at ($(nmos.G)+(1.4, 0)$) {\(V_{DS}\)}; % Testo V_DS sopra i segni
			
			% V_R (Tensione sulla resistenza R)
			\node at ($(R1)+(0.4, 0.5)$) {\(+\)};
			\node at ($(R1)+(0.4, -0.5)$) {\(-\)};
			\node at ($(R1)+(0.5, 0)$) {\(V_R\)}; % Testo V_R sopra i segni
			
			% I_DS Corrente Drain-Source
			\node at ($(nmos.D)+(-0.5, -0.1)$) {\(I_{DS} \downarrow\)}; % Testo I_DS vicino al drain
			
			% VDD
			\node at ($(VDD)+(0, 0.6)$) {\(V_{DD}\)};
		\end{circuitikz}
	\end{minipage}
	\begin{minipage}{0.1\textwidth}
		\centering
		\(\longrightarrow\)
	\end{minipage}
	\begin{minipage}{0.4\textwidth}
		\centering \begin{circuitikz}[american]
			% =================================================================
			% 1. Nodi principali del circuito
			\coordinate (gnd_2) at (0,0); % ground secondo circuito
			
			% =================================================================
			% 2. Circuito
			\draw (gnd_2) node[sground]{}; % ground secondo circuito
			\draw (gnd_2) -- ++(0, 0.2) -- ++(-0.7, 0) to [I, l=\(I_{DSAT}\), invert] ++(0,1.4) -- ++(0.7,0); % generatore di corrente I_DSAT
			\draw (gnd_2) ++(0, 0.2) -- ++(0.7,0) to [R, l=\(R_0\), name=R0] ++(0,1.4) -- ++(-0.7,0) -- ++(0,0.2) node[name=R0]{}; % resistenza R0
			\draw (R0) ++(0,-0.2) -- ++(0, 0.6) to [R, l=\(R\), name=R1] ++(0, 1.15) node[rground, rotate=180, name=VDD]{}; % resistenza R e VDD
			
			% =================================================================
			% 3. Etichette e Vettori Corrente/Tensione
			
			% V_DS (Drain-Source Voltage)
			\node at ($(R0)+(1.1, -0.4)$) {\(+\)};
			\node at ($(R0)+(1.1, -1.4)$) {\(-\)};
			\node at ($(R0)+(1.3, -0.95)$) {\(V_{DS}\)}; % Testo V_DS sopra i segni
			
			% V_R (Tensione sulla resistenza R)
			\node at ($(R1)+(0.4, 0.5)$) {\(+\)};
			\node at ($(R1)+(0.4, -0.5)$) {\(-\)};
			\node at ($(R1)+(0.5, 0)$) {\(V_R\)}; % Testo V_R sopra i segni
			
			% I_DS Corrente Drain-Source
			\node at ($(R0)+(-0.5, 0.1)$) {\(I_{DS} \downarrow\)}; % Testo I_DS vicino al drain
			
			% VDD
			\node at ($(VDD)+(0, 0.6)$) {\(V_{DD}\)};
		\end{circuitikz}
	\end{minipage}
\end{center}

\noindent
L'utilizzo di un NMOSFET al posto di un PMOSFET è perfettamente arbitrario e non cambia nulla nel procedimento di soluzione
del circuito.

\newpage

\subsubsection*{Considerazioni sul regime di funzionamento e risoluzione grafica}
Il MOSFET in saturazione, connesso in questo modo,  si può modellare con un generatore lineare di corrente secondo il teorema
di Norton, come illustrato sopra. In saturazione, infatti, la corrente \(I_{DS}\) è direttamente proporzionale a \(V_{DS}\)
tramite il parametro \(\lambda_n\):
\[I_{DS} = I_{DSAT}(1 + \lambda_n V_{DS}) = I_{DSAT} + \lambda_n I_{DSAT} V_{DS} = I_{DSAT} + R_0 V_{DS} \qquad \text{con} \; R_0 = \frac{1}{\lambda_n I_{DSAT}}\]
Si osserva che la corrente \(I_{DS}\) ha due contributi:
\begin{itemize}
	\item una corrente costante \(I_{DSAT}\) di saturazione del mosfet, corrispone alla corrente erogata dal generatore ideale
	di corrente per Norton;
	\item una corrente variabile \(R_0 V_{DS}\) che dipende linearmente dalla tensione \(V_{DS}\) e che può essere modellata
	con una resistenza \(R_0\) in parallelo con il generatore ideale di corrente.
\end{itemize}
Si nota che se \(\lambda = 0 \; \rightarrow \; I_{DS} = I_{DSAT} \; \rightarrow \; R_0 = \infty\), ovvero se non si
considera l'effetto di modulazione di lunghezza di canale, il MOSFET si comporta come un generatore ideale di corrente.

Il generatore lineare di corrente di Norton ha dei vincoli sul suo funzionamento, in quanto deve essere soddisfatta la condizione
che il MOSFET rimanga in saturazione per pinchoff o in saturazione di velocità:
\[V_{DS} > \min \{V_{REF} - V_{TN} \;_\text{sat. per pinchoff}, \;\; V_{DSATN} \;_\text{sat. di velocità}\}\]
Se questa condizione non è soddisfatta, il MOSFET esce dal regime di saturazione e il modello con il generatore lineare di corrente
non è più valido.

\subsubsection*{Analisi del circuito}
Per risolvere il circuito utilizzando il modello a canale corto si procede come segue:
\begin{enumerate}
	\item si impone che il mosfet sia acceso e che sia in saturazione (per pinchoff o di velocità) per cui si deve avere:
	\(V_{REF} > V_{TN}\) e \(V_{DS} > \min \{V_{REF} - V_{TN}, \; V_{DSATN}\}\)
	\item si ipotizza il regime di saturazione (pinchoff o velocità) e si calcola \(I_{DSAT}\) e \(R_0\):
	\begin{itemize}[topsep=0pt]
		\item in regime di saturazione per pinchoff: \(\displaystyle I_{DSAT} = \frac{k_n}{2} (V_{REF} - V_{TN})^2\)
		\item in regime di saturazione di velocità: \(\displaystyle I_{DSAT} = k_n V_{DSATN}\left(V_{REF} - V_{TN} - \frac{V_{DSATN}}{2}\right)\)
		\item a prescindere dal regime di saturazione \(R_0 = 1 / (\lambda_n I_{DSAT})\)
	\end{itemize}
	\item si applica il teorema di Norton, ovvero si sostituisce il MOSFET con il generatore lineare di	corrente e
	si risolve il circuito (es. per sovrapposizione degli effetti) in funzione di \(V_{DS}\):
	\[V_{DS} = V_{DD} \frac{R_0}{R + R_0} - I_{DSAT} R_0 \frac{R}{R + R_0}\]
	\item si verifica l'ipotesi sul regime di saturazione sia soddisfatta:
	\begin{itemize}
		\item in regime di saturazione per pinchoff: \(V_{DS} > V_{REF} - V_{TN}\)
		\item in regime di saturazione di velocità: \(V_{DS} > V_{DSATN}\)
	\end{itemize}
	\item se la condizione è soddisfatta, si è trovata la soluzione; altrimenti si deve ripetere il procedimento per l'altro
	regime di saturazione
\end{enumerate}

Il modello a canale lungo è un caso particolare del modello a canale corto con \(\lambda_n = 0\) e \(R_0 = \infty\), per cui
il MOSFET si può modellare come un generatore ideale di corrente con corrente \(I_{DSAT}\) data dal regime di saturazione per
pinchoff. Non prevedendo l'effetto di modulazione di lunghezza di canale e il fenomeno di saturazione di velocità, alla fine
è sufficiente verificare solo la condizione \(V_{DS} > V_{REF} - V_{TN}\).

\newpage

\subsection{MOSFET usato come interruttore}
\subsubsection*{Modellizzazione del MOSFET come interruttore}
Nell'elettronica digitale i MOSFET sono utilizzati come interruttori ON/OFF. È possibile, infatti, modellare un MOSFET come un
interruttore ideale pilotato dalla tensione di gate \(V_{GS}\) in serie ad una resistenza \(R_n\) per gli NMOS o \(R_p\)
per i PMOS. Si considerano, quindi, soltanto due regimi di funzionamento:
\begin{itemize}
	\item MOSFET \say{spento}: interruttore aperto, \(I_{DS} = 0\), per \(V_{GS} < V_{TN}\) (NMOS) o \(V_{GS} > V_{TP}\) (PMOS);
	\item MOSFET \say{acceso}: resistore di resistenza \(R_n\) (NMOS) o \(R_p\) (PMOS).
\end{itemize}
\begin{center}
	\centering \includegraphics[width=0.6\textwidth]{immagini/5_circuiti_mosfet/circuiti_mos_3.png}
\end{center}

\subsection{Scarica di un condensatore con NMOS}
\subsubsection*{Circuito di scarica}
\begin{center}
	\begin{minipage}{0.37\textwidth}
		\centering
		\begin{circuitikz}
			\ctikzset{tripoles/mos style=arrows}
			
			\draw (G_mosfet) node[nmos, anchor=G, rotate=-90](nmos){}; % NMOSFET
			\draw (nmos.S) -- ++(0,-0.2) node[sground]{}; % source del MOSFET
			\draw (nmos.D) to [C, l=\(C\), i_=\(I_C\)] ++(0, -1.2) ++(0,0.3) node[sground]{}; % condensatore C
			\draw (nmos.G) -- ++(-1.8,0) to [V, l_=\(V_{G}\quad\quad\), name=Vg] ++(0,-1.5) ++(0,0.3) node[sground]{}; % generatore di tensione V_G
	
			% generatore
			\node at ($(Vg)+(-0.2,-0.55)$) {\(\;_-\)};
			\node at ($(Vg)+(-0.2,0.6)$) {\(\;_+\)};
			\node at ($(Vg)+(-1.5,-0.4)$) {\(0V \rightarrow V_{DD}\)};
	
			% I_DS Corrente Drain-Source
			\node at ($(nmos.D)+(-0.8, -0.2)$) {\(\stackrel{\leftarrow}{I_{DS}}\)}; % Testo I_DS vicino al drain

			% V_C tensione condensatore
			\node at ($(nmos.D)+(0.2, -0.2)$) {\(\;_+\)};
			\node at ($(nmos.D)+(0.22, -1)$) {\(\;_-\)};

			% terminali del mosfet
			\node at ($(nmos.G)+(0.2, -0.1)$) {\(G\)};
			\node at ($(nmos.D)+(-0.25, 0.2)$) {\(D\)};
			\node at ($(nmos.S)+(0.25, 0.2)$) {\(S\)};
		\end{circuitikz}
	\end{minipage}
	\begin{minipage}{0.1\textwidth}
		\centering
		\(\longrightarrow\)
	\end{minipage}
	\begin{minipage}{0.2\textwidth}
		\centering
		\begin{circuitikz}
			\draw (0,0) node[sground]{}; % ground secondo circuito
			\draw (0,0) to[short,-*] ++(0.5, 0) -- ++(0.5, 0.3) to [open,-*] ++(0, -0.3) -- ++(0.5, 0) to[C, l=\(C\)] ++(0, -1.2) ++(0,0.3) node[sground]{}; % nodo superiore
		\end{circuitikz}

		per \(t < 0\)
	\end{minipage}
	\begin{minipage}{0.3\textwidth}
		\centering
		\begin{circuitikz}[european]
			\draw (0,0) node[sground]{}; % ground secondo circuito
			\draw (0,0) to[short,-*] ++(0.5, 0) -- ++(0.5, 0.05) to [open,-*] ++(0, -0.05) -- ++(0.5, 0) to[R, l=\(R\)] ++(1.5,0)  to[C, l=\(C\)] ++(0, -1.2) ++(0,0.3) node[sground]{}; % nodo superiore
		\end{circuitikz}

		per \(t \geq 0\)
	\end{minipage}
\end{center}

\subsubsection*{Equazioni del circuito per le leggi di Kirchhoff e la struttura della rete}
\[V_G(t) = \begin{cases} 0 V  & t < 0 \\ V_{DD} & t \geq 0 \end{cases} \quad\qquad V_C(t \leq 0) = V_{DD} \qquad
\begin{array}{l} V_{GS}(t) = V_G(t) \\[3pt] V_{DS}(t) = V_C(t) \end{array} \qquad I_{DS} + I_C(t) = 0\]

\subsubsection*{Analisi delle fasi del transitorio}
Si analizzano le variabili del circuito per \(t < 0\), \(t = 0\), \(t > 0\) e \(t \to \infty\):
\begin{itemize}
	\item per \(t < 0\): l'NMOS è spento poiché \(V_{GS} = 0 V < V_{TN}\) e il condensatore è carico a \(V_{DD}\);
	\item per \(t = 0\): l'NMOS si accende poiché \(V_{GS} = V_{DD} > V_{TN}\);
	\item per \(t > 0\): l'NMOS rimane acceso poiché \(V_{GS} = V_{DD} > V_{TN}\) e il condensatore si scarica;
	\item per \(t \to \infty\): a regime l'NMOS rimane acceso e il condensatore si scarica completamente.
\end{itemize}

\subsubsection*{Analisi grafica delle fasi del transitorio}
Si suppone, sotto certi dati, che il mosfet lavori soltanto in saturazione di velocità o in regime lineare.
Si osserva che la scarica del condensatore avviene in due fasi (con un punto intermedio \(t = t_1\)):
\begin{enumerate}
	\item per \(0 < t < t_1\), \(V_{DS} > V_{DSATN}\) il mosfet lavora in regime di saturazione di velocità e la corrente di
	scarica è costante pari a \(I_{DSATN}\):
	\item per \(t = t_1\), \(V_{DS} = V_{DSATN}\) il mosfet passa dal regime di saturazione di velocità a regime lineare:
	\item per \(t > t_1\), \(V_{DS} < V_{DSATN}\) il mosfet lavora in regime lineare e la corrente di scarica dipende dalla tensione \(V_{DS}(t) = V_C(t)\)
	e quindi dalla tensione del condensatore
\end{enumerate}

\begin{center}
	\begin{minipage}{0.48\textwidth}
		\centering \includegraphics[width=0.95\textwidth]{immagini/5_circuiti_mosfet/scarica_nmos_1.png}
	\end{minipage}
	\begin{minipage}{0.48\textwidth}
		\centering \includegraphics[width=0.95\textwidth]{immagini/5_circuiti_mosfet/scarica_nmos_2.png}
	\end{minipage}
\end{center}

\subsubsection*{Soluzione analitica del circuito}
Si analizza la prima fase di scarica del condensatore per \(0 < t < t_1\) (saturazione di velocità):
\[I_C(t) = C\frac{dV_C}{dt} \qquad\qquad I_{DSATN} = k_n' Z_n V_{DSATN} \left(V_{DD} - V_{TN} - \frac{V_{DSATN}}{2}\right)\]
\[I_C(t) = -I_{DSATN} \quad \rightarrow \quad V_C(t) = V_{DD} - \frac{I_{DSATN}}{C} t\]

\noindent
Si analizza il passaggio dalla prima alla seconda fase di per \(t = t_1\):
\[V_{GS}(t_1) = V_{DSATN} \quad \rightarrow \quad V_C(t_1) = V_{DSATN} \quad \rightarrow \quad t_1 = \frac{C}{I_{DSATN}} (V_{DD} - V_{DSATN})\]

\noindent
Si analizza la seconda fase di scarica del condensatore per \(t > t_1\) (regime lineare):
\[I_C(t) = C\frac{dV_C}{dt} \qquad\qquad I_{DS}(t) = k_n' Z_n V_C(t) \left(V_{DD} - V_{TN} - \frac{V_C(t)}{2}\right)\]
\[I_C(t) = -I_{DS}(t) \quad \rightarrow \quad V_C(t) = \frac{1}{A \, e^{\alpha (t - t_1)} + B}\]
\[\alpha = \frac{k_n' Z_n}{C} (V_{DD} - V_{TN}) \qquad A = \frac{1}{V_{DSATN}} - \frac{1}{2(V_{DD} - V_{TN})} \qquad B = \frac{1}{2(V_{DD} - V_{TN})}\]

\subsubsection*{Carica del condensatore a regime}
Per \(t \to \infty\), la tensione sul condensatore si riduce a zero e il condensatore si scarica completamente:
\[\lim_{t \to \infty} V_C(t) = 0 V\]

\subsubsection*{Tempo di dimezzamento di \(V_C(t)\)}
Il tempo di dimezzamento \(t_{1/2}\) è il tempo necessario affinché la tensione sul condensatore si riduca alla metà del suo
valore iniziale \(V_{DD}\), ovvero quando \(V_C(t_{1/2}) = V_{DD}/2\). Si osserva che tale valore viene assunto durante la prima
fase di scarica, in cui il mosfet lavora in saturazione di velocità. Si calcola, quindi, \(t_{1/2}\) come segue:
\[V_C(t_{1/2}) = \frac{V_{DD}}{2} \quad \rightarrow \quad t_{1/2} = \frac{C V_{DD}}{2 I_{DSATN}}\]

\newpage

\subsection{Carica di un condensatore con PMOS}
\subsubsection*{Circuito di carica}
\begin{center}
	\begin{minipage}{0.37\textwidth}
		\centering
		\begin{circuitikz}
			\ctikzset{tripoles/mos style=arrows}
			\ctikzset{tripoles/pmos style/emptycircle}

			\draw (G_mosfet) node[pmos, anchor=G, rotate=90, xscale=-1](pmos){}; % PMOSFET con source/drain invertiti
			\draw (pmos.S) -- ++(-0.1,0) node[rground, rotate=180, name=VDD]{}; % source del MOSFET
			\draw (pmos.D) to [C, l=\(C\), i_=\(I_C\)] ++(0, -1.2) ++(0,0.3) node[sground]{}; % condensatore C
			\draw (pmos.G) -- ++(-1.8,0) to [V, l_=\(V_{G}\quad\quad\), name=Vg] ++(0,-1.5) ++(0,0.3) node[sground]{}; % generatore di tensione V_G

			% generatore
			\node at ($(Vg)+(-0.2,-0.55)$) {\(\;_-\)};
			\node at ($(Vg)+(-0.2,0.6)$) {\(\;_+\)};
			\node at ($(Vg)+(-1.5,-0.4)$) {\(V_{DD} \rightarrow 0V\)};
	
			% I_DS Corrente Drain-Source
			\node at ($(pmos.D)+(-0.8, -0.2)$) {\(\stackrel{\rightarrow}{I_{DS}}\)}; % Testo I_DS vicino al drain
	
			% V_C Corrente condensatore
			\node at ($(pmos.D)+(0.2, -0.2)$) {\(\;_+\)};
			\node at ($(pmos.D)+(0.22, -1)$) {\(\;_-\)};

			% VDD
			\node at ($(VDD)+(0, 0.6)$) {\(V_{DD}\)};

			% terminali del mosfet
			\node at ($(nmos.G)+(0.2, -0.1)$) {\(G\)};
			\node at ($(nmos.D)+(-0.25, 0.2)$) {\(D\)};
			\node at ($(nmos.S)+(0.25, 0.2)$) {\(S\)};
		\end{circuitikz}
	\end{minipage}
	\begin{minipage}{0.1\textwidth}
		\centering
		\(\longrightarrow\)
	\end{minipage}
	\begin{minipage}{0.2\textwidth}
		\centering
		\begin{circuitikz}
			\draw (0,0) node[rground, rotate=180, name=VDD]{}; % ground secondo circuito
			\draw (0,0) to[short,-*] ++(0.5, 0) -- ++(0.5, 0.3) to [open,-*] ++(0, -0.3) -- ++(0.5, 0) to[C, l=\(C\)] ++(0, -1.2) ++(0,0.3) node[sground]{}; % nodo superiore
			\node at ($(VDD)+(0, 0.6)$) {\(V_{DD}\)};
		\end{circuitikz}

		per \(t < 0\)
	\end{minipage}
	\begin{minipage}{0.3\textwidth}
		\centering
		\begin{circuitikz}[european]
			\draw (0,0) node[rground, rotate=180, name=VDD]{}; % ground secondo circuito
			\draw (0,0) to[short,-*] ++(0.5, 0) -- ++(0.5, 0.05) to [open,-*] ++(0, -0.05) -- ++(0.5, 0) to[R, l=\(R\)] ++(1.5,0)  to[C, l=\(C\)] ++(0, -1.2) ++(0,0.3) node[sground]{}; % nodo superiore
			\node at ($(VDD)+(0, 0.6)$) {\(V_{DD}\)};
		\end{circuitikz}

		per \(t \geq 0\)
	\end{minipage}
\end{center}

\subsubsection*{Equazioni del circuito per le leggi di Kirchhoff e la struttura della rete}
\[V_G(t) = \begin{cases} V_{DD}  & t < 0 \\ 0 V & t \geq 0 \end{cases} \quad\qquad V_C(t \leq 0) = 0 V \qquad
\begin{array}{l} V_{GS}(t) = V_G(t) - V_{DD} \\[3pt] V_{DS}(t) = V_C(t) - V_{DD} \end{array} \qquad I_{DS} = I_C(t)\]

\subsubsection*{Analisi delle fasi del transitorio}
Si analizzano le variabili del circuito per \(t < 0\), \(t = 0\), \(t > 0\) e \(t \to \infty\):
\begin{itemize}
	\item per \(t < 0\): il PMOS è spento poiché \(V_{GS} = 0 V > V_{TP}\) e il condensatore è scarico con \(V_C = 0 V\);
	\item per \(t = 0\): il PMOS si accende poiché \(V_{GS} = - V_{DD} < V_{TP}\);
	\item per \(t > 0\): il PMOS rimane acceso poiché \(V_{GS} = - V_{DD} < V_{TP}\) e il condensatore si carica;
	\item per \(t \to \infty\): a regime il PMOS rimane acceso e il condensatore si carica completamente a \(V_{DD}\).
\end{itemize}

\subsubsection*{Analisi grafica delle fasi del transitorio}
Si suppone, sotto certi dati, che il mosfet lavori soltanto in saturazione di velocità o in regime lineare.
Si osserva che la scarica del condensatore avviene in due fasi (con un punto intermedio \(t = t_1\)):
\begin{enumerate}
	\item per \(0 < t < t_1\), \(V_{DS} < V_{DSATP}\) il mosfet lavora in regime di saturazione di velocità e la corrente di
	scarica è costante pari a \(I_{DSATP}\):
	\item per \(t = t_1\), \(V_{DS} = V_{DSATP}\) il mosfet passa dal regime di saturazione di velocità a regime lineare:
	\item per \(t > t_1\), \(V_{DS} > V_{DSATP}\) il mosfet lavora in regime lineare e la corrente di scarica dipende dalla tensione
	\(V_{DS}(t) = V_C(t) - V_{DD}\) e quindi dalla tensione del condensatore
\end{enumerate}

\begin{center}
	\begin{minipage}{0.48\textwidth}
		\centering \includegraphics[width=0.95\textwidth]{immagini/5_circuiti_mosfet/carica_pmos_1.png}
	\end{minipage}
	\begin{minipage}{0.48\textwidth}
		\centering \includegraphics[width=0.95\textwidth]{immagini/5_circuiti_mosfet/carica_pmos_2.png}
	\end{minipage}
\end{center}

\newpage

\subsubsection*{Soluzione analitica del circuito}
Si analizza la prima fase di scarica del condensatore per \(0 < t < t_1\) (saturazione di velocità):
\[I_C(t) = C\frac{dV_C}{dt} \qquad\qquad I_{DSATP} = k_p' Z_p V_{DSATP} \left(-V_{DD} - V_{TP} - \frac{V_{DSATP}}{2}\right)\]
\[I_C(t) = I_{DSATP} \quad \rightarrow \quad V_C(t) = \frac{I_{DSATP}}{C} t\]

\noindent
Si analizza il passaggio dalla prima alla seconda fase di per \(t = t_1\):
\[V_{DS}(t_1) = V_{DSATP} \quad \rightarrow \quad V_C(t_1) - V_{DD} = V_{DSATP} \quad \rightarrow \quad t_1 = \frac{C}{I_{DSATP}} (V_{DD} + V_{DSATP})\]

\noindent
Si analizza la seconda fase di scarica del condensatore per \(t > t_1\) (regime lineare):
\[I_C(t) = C\frac{dV_C}{dt} \qquad I_{DS}(t) = k_p' Z_p (V_C(t) - V_{DD}) \left(-V_{DD} - V_{TP} - \frac{V_C(t)-V_{DD}}{2}\right)\]
\[I_C(t) = I_{DS}(t) \quad \rightarrow \quad V_C(t) = \frac{1}{A \, e^{\alpha (t - t_1)} + B} + V_{DD}\]
\[\alpha = \frac{k_p' Z_p}{C} (V_{DD} + V_{TP}) \qquad A = \frac{1}{V_{DSATP}} + \frac{1}{2(V_{DD} + V_{TP})} \qquad B = -\frac{1}{2(V_{DD} + V_{TP})}\]

\subsubsection*{Carica del condensatore a regime}
Per \(t \to \infty\), la tensione sul condensatore raggiunge il valore \(V_{DD}\) e il condensatore si carica completamente:
\[\lim_{t \to \infty} V_C(t) = V_{DD}\]

\subsubsection*{Tempo di dimezzamento di \(V_C(t)\)}
Si analizza il tempo di dimezzamento \(t_{1/2}\) osservando che tale valore viene assunto durante la prima fase di carica, in
cui il mosfet lavora in saturazione di velocità:
\[V_C(t_{1/2}) = \frac{V_{DD}}{2} \quad \rightarrow \quad t_{1/2} = \frac{C V_{DD}}{2 I_{DSATP}}\]

\newpage

\subsection{Scarica di un condensatore con un PMOS}
\subsubsection*{Circuito di scarica}
\begin{center}
	\begin{minipage}{0.37\textwidth}
		\centering
		\begin{circuitikz}
			\ctikzset{tripoles/mos style=arrows}
			\ctikzset{tripoles/pmos style/emptycircle}

			\draw (G_mosfet) node[pmos, anchor=G, rotate=-90](pmos){}; % PMOSFET
			\draw (pmos.D) -- ++(0,-0.2) node[sground]{}; % source del MOSFET
			\draw (pmos.S) to [C, l=\(C\), i_=\(I_C\)] ++(0, -1.2) ++(0,0.3) node[sground]{}; % condensatore C
			\draw (pmos.G) -- ++(-1.8,0) to [V, l_=\(V_{G}\quad\quad\), name=Vg] ++(0,-1.5) ++(0,0.3) node[sground]{}; % generatore di tensione V_G
	
			% generatore
			\node at ($(Vg)+(-0.2,-0.55)$) {\(\;_-\)};
			\node at ($(Vg)+(-0.2,0.6)$) {\(\;_+\)};
			\node at ($(Vg)+(-1.5,-0.4)$) {\(V_{DD} \rightarrow 0V\)};
	
			% I_DS Corrente Drain-Source
			\node at ($(pmos.D)+(0.8, -0.2)$) {\(\stackrel{\leftarrow}{I_{DS}}\)}; % Testo I_DS vicino al drain
	
			% V_C Corrente condensatore
			\node at ($(pmos.S)+(0.2, -0.2)$) {\(\;_+\)};
			\node at ($(pmos.S)+(0.22, -1)$) {\(\;_-\)};

			% terminali del mosfet
			\node at ($(nmos.G)+(0.2, -0.1)$) {\(G\)};
			\node at ($(nmos.S)+(0.25, 0.2)$) {\(D\)};
			\node at ($(nmos.D)+(-0.25, 0.2)$) {\(S\)};
		\end{circuitikz}
	\end{minipage}
	\begin{minipage}{0.1\textwidth}
		\centering
		\(\longrightarrow\)
	\end{minipage}
	\begin{minipage}{0.2\textwidth}
		\centering
		\begin{circuitikz}
			\draw (0,0) node[sground]{}; % ground secondo circuito
			\draw (0,0) to[short,-*] ++(0.5, 0) -- ++(0.5, 0.3) to [open,-*] ++(0, -0.3) -- ++(0.5, 0) to[C, l=\(C\)] ++(0, -1.2) ++(0,0.3) node[sground]{}; % nodo superiore
		\end{circuitikz}

		per \(t < 0\)
	\end{minipage}
	\begin{minipage}{0.3\textwidth}
		\centering
		\begin{circuitikz}[european]
			\draw (0,0) node[sground]{}; % ground secondo circuito
			\draw (0,0) to[short,-*] ++(0.5, 0) -- ++(0.5, 0.05) to [open,-*] ++(0, -0.05) -- ++(0.5, 0) to[R, l=\(R\)] ++(1.5,0)  to[C, l=\(C\)] ++(0, -1.2) ++(0,0.3) node[sground]{}; % nodo superiore
		\end{circuitikz}

		per \(t \geq 0\)
	\end{minipage}
\end{center}

\subsubsection*{Equazioni del circuito per le leggi di Kirchhoff e la struttura della rete}
\[V_G(t) = \begin{cases} V_{DD} & t < 0 \\ 0 V & t \geq 0 \end{cases} \quad\qquad V_C(t \leq 0) = V_{DD} \qquad
\begin{array}{l} V_{GS}(t) = V_G(t) - V_C(t) \\[3pt] V_{DS}(t) = -V_C(t) \end{array} \qquad I_{DS} + I_C(t) = 0\]

\subsubsection*{Analisi delle fasi del transitorio}
Si analizzano le variabili del circuito per \(t < 0\), \(t = 0\), \(t > 0\) e \(t \to \infty\):
\begin{itemize}
	\item per \(t < 0\): il PMOS è spento poiché \(V_{GS}(t) = 0V > V_{TP}\) e il condensatore è carico a \(V_{DD}\);
	\item per \(t = 0\): il PMOS si accende poiché \(V_{GS}(t) = -V_{DD} < V_{TP}\);
	\item per \(t > 0\): il PMOS è acceso finché \(V_{GS}(t) < V_{TP} \;\; \rightarrow \;\; V_C(t) > -V_{TP}\)
	\item per \(t \rightarrow \infty\): a regime il PMOS si spegne per \(V_{GS}(t) = V_{TP} \; \rightarrow \; V_C(t) = -V_{TP}\)
	lasciando il condensatore carico con tensione finale \(V_C(t) = -V_{TP}\)
\end{itemize}

\subsubsection*{Analisi grafica delle fasi del transitorio}
Il PMOS è connesso a diodo (gate e drain entrambi a massa per \(t\geq 0\)), quindi lavora solo in saturazione.
Si osserva che la scarica del condensatore avviene in due fasi (con un punto intermedio \(t = t_1\)):
\begin{enumerate}
	\item per \(0 < t < t_1\), \(V_{GS} - V_{TP} < V_{DSATP}\) il mosfet lavora in regime di saturazione di velocità e
	la corrente di scarica è costante pari a \(I_{DSATP}\):
	\item per \(t = t_1\), \(V_{GS} - V_{TP} = V_{DSATP}\) il mosfet passa dal regime di saturazione per velocità a quello di
	saturazione per pinchoff:
	\item per \(t > t_1\), \(V_{GS} - V_{TP} > V_{DSATP}\) il mosfet lavora in saturazione per pinchoff e la corrente di scarica
	dipende da \(V_{GS}(t) - V_{TP} = -V_{C}(t) - V_{TP}\) e quindi anche dalla tensione del condensatore \(V_C(t)\)
\end{enumerate}

\begin{center}
	\begin{minipage}{0.48\textwidth}
		\centering \includegraphics[width=0.95\textwidth]{immagini/5_circuiti_mosfet/scarica_pmos_1.png}
	\end{minipage}
	\begin{minipage}{0.48\textwidth}
		\centering \includegraphics[width=0.95\textwidth]{immagini/5_circuiti_mosfet/scarica_pmos_2.png}
	\end{minipage}
\end{center}

\subsubsection*{Soluzione analitica del circuito}
Si analizza la prima fase di scarica del condensatore per \(0 < t < t_1\) (saturazione di velocità):
\[I_C(t) = C\frac{dV_C}{dt} \qquad\qquad I_{DSATP}(t) = k_p' Z_p V_{DSATP} \left(-V_{C}(t) - V_{TP} - \frac{V_{DSATP}}{2}\right)\]
\[I_C(t) = I_{DSATP}(t) \;\; \rightarrow \;\; V_C(t) = V_{DD} - \left(V_{TP} + \frac{V_{DSATP}}{2}\right) \left(1-e^{-\alpha t}\right) \quad \text{con} \; \alpha = -\frac{k_p' Z_p V_{DSATP}}{C}\]

\noindent
Si analizza il passaggio dalla prima alla seconda fase di per \(t = t_1\):
\[V_{GS}(t_1) - V_{TP} = V_{DSATP} \;\; \rightarrow \;\; V_C(t_1) = -V_{TP} - V_{DSATP} \;\; \rightarrow \;\; t_1 = \frac{1}{\alpha} \ln \left(\frac{2 V_{DD} + 2V_{TP} + V_{DSATP}}{-V_{DSATP}}\right)\]

\noindent
Si analizza la seconda fase di scarica del condensatore per \(t > t_1\) (saturazione per pinchoff):
\[I_C(t) = C\frac{dV_C}{dt} \qquad\qquad I_{DS}(t) = \frac{k_p' Z_p}{2} (V_{GS}(t) - V_{TP})^2 = \frac{k_p' Z_p}{2} (-V_C(t) - V_{TP})^2\]
\[I_C(t) = I_{DS}(t) \quad \rightarrow \quad V_C(t) = -V_{TP} + \frac{2CV_{DSATP}}{k_p' Z_p V_{DSATP}(t-t_1)-2C}\]

\subsubsection*{Valore di regime}
Si analizza il valore di regime della tensione sul condensatore \(V_C(t)\) per \(t \rightarrow \infty\):
\[\lim_{t \to \infty} V_C(t) = -V_{TP}\]
Si ha, quindi, che il condensatore si scarica fino a raggiungere la tensione di soglia del PMOS.

\subsubsection*{Tempo di dimezzamento di \(V_C(t)\)}
Si analizza il tempo di dimezzamento \(t_{1/2}\), osservando che tale valore viene assunto durante la seconda fase di scarica,
in cui il mosfet lavora in saturazione per pinchoff:
\[V_C(t_{1/2}) = \frac{V_{DD}}{2} \quad \rightarrow \quad t_{1/2} = t_1 - 2C \frac{V_{DD} + 2V_{TP} + 2V_{DSATP}}{k_p' Z_p V_{DSATP}(V_{DD} + 2V_{TP})}\]

\newpage

\subsection{Carica di un condensatore con un NMOS}
\subsubsection*{Circuito di carica}
\begin{center}
	\begin{minipage}{0.37\textwidth}
		\centering
		\begin{circuitikz}
			\ctikzset{tripoles/mos style=arrows}
			\ctikzset{tripoles/pmos style/emptycircle}

			\draw (G_mosfet) node[nmos, anchor=G, rotate=90, xscale=-1](nmos){}; % PMOSFET con source/drain invertiti
			\draw (nmos.D) -- ++(-0.1,0) node[rground, rotate=180, name=VDD]{}; % source del MOSFET
			\draw (nmos.S) to [C, l=\(C\), i_=\(I_C\)] ++(0, -1.2) ++(0,0.3) node[sground]{}; % condensatore C
			\draw (nmos.G) -- ++(-1.8,0) to [V, l_=\(V_{G}\quad\quad\), name=Vg] ++(0,-1.5) ++(0,0.3) node[sground]{}; % generatore di tensione V_G

			% generatore
			\node at ($(Vg)+(-0.2,-0.55)$) {\(\;_-\)};
			\node at ($(Vg)+(-0.2,0.6)$) {\(\;_+\)};
			\node at ($(Vg)+(-1.5,-0.4)$) {\(0V \rightarrow V_{DD}\)};
	
			% I_DS Corrente Drain-Source
			\node at ($(nmos.D)+(0.8, -0.2)$) {\(\stackrel{\rightarrow}{I_{DS}}\)}; % Testo I_DS vicino al drain
	
			% V_C Corrente condensatore
			\node at ($(nmos.S)+(0.2, -0.2)$) {\(\;_+\)};
			\node at ($(nmos.S)+(0.22, -1)$) {\(\;_-\)};

			% VDD
			\node at ($(VDD)+(0, 0.6)$) {\(V_{DD}\)};

			% terminali del mosfet
			\node at ($(nmos.G)+(0.2, -0.1)$) {\(G\)};
			\node at ($(nmos.D)+(0.25, 0.2)$) {\(D\)};
			\node at ($(nmos.S)+(-0.25, 0.2)$) {\(S\)};
		\end{circuitikz}
	\end{minipage}
	\begin{minipage}{0.1\textwidth}
		\centering
		\(\longrightarrow\)
	\end{minipage}
	\begin{minipage}{0.2\textwidth}
		\centering
		\begin{circuitikz}
			\draw (0,0) node[rground, rotate=180, name=VDD]{}; % ground secondo circuito
			\draw (0,0) to[short,-*] ++(0.5, 0) -- ++(0.5, 0.3) to [open,-*] ++(0, -0.3) -- ++(0.5, 0) to[C, l=\(C\)] ++(0, -1.2) ++(0,0.3) node[sground]{}; % nodo superiore
			\node at ($(VDD)+(0, 0.6)$) {\(V_{DD}\)};
		\end{circuitikz}

		per \(t < 0\)
	\end{minipage}
	\begin{minipage}{0.3\textwidth}
		\centering
		\begin{circuitikz}[european]
			\draw (0,0) node[rground, rotate=180, name=VDD]{}; % ground secondo circuito
			\draw (0,0) to[short,-*] ++(0.5, 0) -- ++(0.5, 0.05) to [open,-*] ++(0, -0.05) -- ++(0.5, 0) to[R, l=\(R\)] ++(1.5,0)  to[C, l=\(C\)] ++(0, -1.2) ++(0,0.3) node[sground]{}; % nodo superiore
			\node at ($(VDD)+(0, 0.6)$) {\(V_{DD}\)};
		\end{circuitikz}

		per \(t \geq 0\)
	\end{minipage}
\end{center}

\subsubsection*{Equazioni del circuito per le leggi di Kirchhoff e la struttura della rete}
\[V_G(t) = \begin{cases} 0 V & t < 0 \\ V_{DD} & t \geq 0 \end{cases} \quad\qquad V_C(t \leq 0) = 0 V \qquad
\begin{array}{l} V_{GS}(t) = V_G(t) - V_C(t) \\[3pt] V_{DS}(t) = V_{DD} - V_C(t) \end{array} \qquad I_{DS} = I_C(t)\]

\subsubsection*{Analisi delle fasi del transitorio}
Si analizzano le variabili del circuito per \(t < 0\), \(t = 0\), \(t > 0\) e \(t \to \infty\):
\begin{itemize}
	\item per \(t < 0\): l'NMOS è spento poiché \(V_{GS}(t) = 0V < V_{TN}\);
	\item per \(t = 0\): l'MOS si accende poiché \(V_{GS}(t) = V_{DD} > V_{TN}\);
	\item per \(t > 0\): l'MOS è acceso finché \(V_{GS}(t) > V_{TN} \;\; \rightarrow \;\; V_C(t) < V_{DD} -V_{TN}\)
	\item per \(t \rightarrow \infty\): a regime l'MOS si spegne quando \(V_{GS}(t) = V_{TN} \; \rightarrow \; V_C(t) = V_{DD} - V_{TN}\)
	lasciando il condensatore carico con tensione finale \(V_C(t) = V_{DD} - V_{TN}\)
\end{itemize}

\subsubsection*{Analisi grafica delle fasi del transitorio}
Il PMOS è connesso a diodo (gate e drain entrambi a \(V_{DD}\) per \(t\geq 0\)), quindi lavora solo in saturazione.
Si osserva che la carica del condensatore avviene in due fasi (con un punto intermedio \(t = t_1\)):
\begin{enumerate}
	\item per \(0 < t < t_1\), \(V_{GS} - V_{TN} > V_{DSATN}\) il mosfet lavora in regime di saturazione di velocità e
	la corrente di carica è costante pari a \(I_{DSATN}\):
	\item per \(t = t_1\), \(V_{GS} - V_{TN} = V_{DSATN}\) il mosfet passa dal regime di saturazione per velocità a quello di
	saturazione per pinchoff:
	\item per \(t > t_1\), \(V_{GS} - V_{TN} < V_{DSATN}\) il mosfet lavora in saturazione per pinchoff e la corrente di scarica
	dipende da \(V_{GS}(t) - V_{TN} = V_{DD} - V_{C}(t) - V_{TN}\), quindi anche dalla tensione del condensatore \(V_C(t)\)
\end{enumerate}

\begin{center}
	\begin{minipage}{0.48\textwidth}
		\centering \includegraphics[width=0.95\textwidth]{immagini/5_circuiti_mosfet/carica_nmos_1.png}
	\end{minipage}
	\begin{minipage}{0.48\textwidth}
		\centering \includegraphics[width=0.95\textwidth]{immagini/5_circuiti_mosfet/carica_nmos_2.png}
	\end{minipage}
\end{center}

\subsubsection*{Soluzione analitica del circuito}
Si analizza la prima fase di carica del condensatore per \(0 < t < t_1\) (saturazione di velocità):
\[I_C(t) = C\frac{dV_C}{dt} \qquad\qquad I_{DSATN}(t) = k_p' Z_p V_{DSATN} \left(V_{DD} - V_{C}(t) - V_{TN} - \frac{V_{DSATN}}{2}\right)\]
\[I_C(t) = I_{DSATN}(t) \;\; \rightarrow \;\; V_C(t) = \left(V_{DD} - V_{TN} - \frac{V_{DSATN}}{2}\right) \left(1-e^{-\alpha t}\right) \quad \text{con} \; \alpha = -\frac{k_p' Z_p V_{DSATN}}{C}\]

\noindent
Si analizza il passaggio dalla prima alla seconda fase di per \(t = t_1\):
\[V_{GS}(t_1) - V_{TN} = V_{DSATN} \;\; \rightarrow \;\; V_C(t_1) = V_{DD} -V_{TN} - V_{DSATN} \;\; \rightarrow \;\; t_1 = \frac{1}{\alpha} \ln \left(\frac{V_{DD} - V_{TN}}{V_{DSATN}} - 1\right)\]

\noindent
Si analizza la seconda fase di carica del condensatore per \(t > t_1\) (saturazione per pinchoff):
\[I_C(t) = C\frac{dV_C}{dt} \qquad\qquad I_{DS}(t) = \frac{k_p' Z_p}{2} (V_{GS}(t) - V_{TN})^2 = \frac{k_p' Z_p}{2} (V_{DD} - V_C(t) - V_{TN})^2\]
\[I_C(t) = I_{DS}(t) \quad \rightarrow \quad V_C(t) = V_{DD} - V_{TN} - \frac{2CV_{DSATN}}{k_p' Z_p V_{DSATN}(t-t_1)+2C}\]

\subsubsection*{Valore di regime}
Si analizza il valore di regime della tensione sul condensatore \(V_C(t)\) per \(t \rightarrow \infty\):
\[\lim_{t \to \infty} V_C(t) = V_{DD} - V_{TN}\]
Si ha, quindi, che il condensatore si carica fino a raggiungere la tensione finale \(V_C(t) = V_{DD} - V_{TN}\).

\subsubsection*{Tempo di dimezzamento di \(V_C(t)\)}
Si analizza il tempo di dimezzamento \(t_{1/2}\), osservando che tale valore viene assunto durante la prima fase di carica,
in cui il mosfet lavora in saturazione per velocità:
\[V_C(t_{1/2}) = \frac{V_{DD}}{2} \quad \rightarrow \quad t_{1/2} = \frac{1}{\alpha}  \ln \left(\frac{2V_{DD} - 2V_{TN} - V_{DSATN}}{V_{DD} - 2V_{TN} - V_{DSATN}}\right)\]

\newpage

\subsection{Confronto dei transitori per carica e scarica con NMOS e PMOS}
\subsubsection*{Dati}
Si scelgono i seguenti dati per il confronto dei transitori di carica e scarica con NMOS e PMOS:
\begin{center}
	\begin{tabular}{l | l l | l l}
		circuito & NMOS & & PMOS & \\
		\toprule
		\(C = 10 fF\) & \(k_n' = 125 \mu \A/\V^2\) & \(Z_n = 2\) & \(k_p' = 40 \mu \A/\V^2\) & \(Z_p = 2 \) \\[4pt]
		 \(V_{DD} = 2.5 V\) & \(V_{TN} = 0.5 V\) & \(V_{DSATN} = 0.6 V\) & \(V_{TP} = -0.5 V\) & \(V_{DSATP} = -0.8 V\)
	\end{tabular}
\end{center}

\subsubsection*{Confronto dei tempi caratteristici}
\begin{center}
	\begin{tabular}{l c c c}
		circuito & \(t_1\) & \(t_{1/2}\) & \(V_C(t \to \infty)\)\\
		\toprule
		scarica con NMOS & \(74.5 \, \ps\) & \(49 \, \ps\) & \(0 \V\) \\[4pt]
		carica con NMOS & \(115.6 \, \ps\) & \(88.6 \, \ps\) & \(V_{DD} - V_{TN} < V_{DD}\) \\[4pt]
		carica con PMOS & \(151.7 \, \ps\) & \(122 \, \ps\) & \(V_{DD}\) \\[4pt]
		scarica con PMOS & \(216.6 \, \ps\) & \(237 \, \ps\) & \(-V_{TP} > 0 \V\)
	\end{tabular}
\end{center}
Si nota che:
\begin{itemize}
	\item l'NMOS scarica totalmente il condensatore, ma lo carica parzialmente fino ad un valore inferiore a \(V_{DD}\),
	si dice che \textbf{l'NMOS trasmette bene il valore logico basso} (0V);
	\item il PMOS carica totalmente il condensatore, ma lo scarica parzialmente fino ad un valore superiore a \(0 \V\),
	si dice che \textbf{il PMOS trasmette bene il valore logico alto} (\(V_{DD}\));
	\item i tempi di dimezzamento per trasmettere il valore logico \say{cattivo} (carica con NMOS e scarica con PMOS) sono circa
	il doppio rispetto a quelli per trasmettere il valore logico \say{buono} (scarica con NMOS e carica con PMOS);
	\item il tempo di carica con PMOS è circa il doppio rispetto al tempo di scarica con NMOS, a parità del fattore di forma \(Z_n = Z_p\)
\end{itemize}

\subsubsection*{Effetto body nei circuiti di carica e scarica dei condensatori}
Spesso, nei circuiti digitali, i body degli NMOS sono collegati a massa e quelli dei PMOS a \(V_{DD}\), non necessariamente al loro
source. Nell'NMOS in scarica e nel PMOS in carica ciò non comporta variazioni, mentre nell'NMOS in carica e nel PMOS in scarica
si ha un aumento in modulo della soglia di tensione \(V_{TN}\) e \(V_{TP}\) rispettivamente, a causa dell'effetto body.
Ciò comporta un aumento dei tempi di carica e scarica dei condensatori e un \textbf{peggioramento ulteriore nella trasmissione
dei valori logici \say{cattivi}}.

\newpage

\subsection{Resistenza equivalente del MOSFET come interruttore}
Per facilitare l'analisi dei circuiti con MOSFET come interruttori, si può approssimare il comportamento del MOSFET acceso con
una resistenza equivalente \(R_{eq}\). In questo modo si linearizza il comportamento non lineare del MOSFET e si possono utilizzare
le tecniche di analisi dei circuiti lineari. In particolare le curve di scarica e carica diventano esponenziali con costante di
tempo \(\tau = R_{eq} C\).

\begin{center}
	\includegraphics[width=0.7\textwidth]{immagini/5_circuiti_mosfet/req_1.png}

	scarica con un NMOS \hspace{3cm} carica con un PMOS
\end{center}

\subsubsection*{Definizione di resistenza equivalente \(R_{eq}\)}

Si definisce la resistenza equivalente \(R_{eq}\) del MOSFET come interruttore acceso come la media della resistenza istantanea
agli estremi dell'intervallo di interesse, ovvero a \(t = 0\) e a \(t = t_{1/2}\), quando il MOSFET trasmette il valore logico
\say{buono}.

\begin{center}
	\begin{minipage}{0.35\textwidth}
		\centering \includegraphics[width=\textwidth]{immagini/5_circuiti_mosfet/req_2.png}
	\end{minipage}
	\begin{minipage}{0.6\textwidth}
		\[\text{NMOS:} \quad R_{n}(0) = \frac{V_{DD}}{I_{DSATN}}, \quad R_{n}(t_{1/2}) = \frac{V_{DD} / 2}{I_{DSATN}} \;\; \rightarrow \]
		\[\quad \rightarrow \;\; R_{n} = \frac{R_n(0) + R_n(t_{1/2})}{2} = \frac{3}{4} \frac{V_{DD}}{I_{DSATN}}\]
		\[\text{PMOS:} \quad R_{p}(0) = \frac{V_{DD}}{I_{DSATP}}, \quad R_{p}(t_{1/2}) = \frac{V_{DD} / 2}{I_{DSATP}} \;\; \rightarrow \]
		\[\quad \rightarrow \;\; R_{p} = \frac{R_p(0) + R_p(t_{1/2})}{2} = \frac{3}{4} \frac{V_{DD}}{I_{DSATP}}\]
	\end{minipage}
\end{center}

\noindent
NOTE:
\begin{itemize}
	\item nelle elaborazioni delle reti logiche digitali, si considera solo la prima parte del transitorio, ovvero fino al tempo
	di dimezzamento \(t_{1/2}\), in quanto è il momento in cui la tensione è vicina alla tensione di soglia logica, ovvero al
	valore in cui il segnale digitale cambia stato logico (da 0 a 1 o da 1 a 0);
	\item si calcola la resistenza solo nei casi in cui il MOSFET trasmette il valore logico \say{buono}, siccome è il comportamento
	più ricercato ed utilizzato nei circuiti digitali; più avanti viene anche approfondita la resistenza equivalente per valori logici
	\say{cattivi}.
\end{itemize}

\subsubsection*{Fattore di forma e resistenza equivalente}
La corrente di saturazione \(I_{DSAT}\) dipende dal fattore di forma \(Z\) del MOSFET e può esser riscritta come:
\[I_{DSATN} = Z_n V_{DSATN0} \qquad\qquad I_{DSATP} = Z_p V_{DSATP0}\]
Per cui la resistenza equivalente può essere riscritta in funzione delle costanti \(R_{n0}\) e \(R_{p0}\), ovvero le resistenze
equivalenti per fattore di forma unitario \(Z_n = Z_p = 1\).
\[R_{n} = \frac{R_{n0}}{Z_n} \;\; \text{con} \; R_{n0} = \frac{3}{4} \frac{V_{DD}}{V_{DSATN0}} \qquad\qquad R_{p} = \frac{R_{p0}}{Z_p} \;\; \text{con} \; R_{p0} = \frac{3}{4} \frac{V_{DD}}{V_{DSATP0}}\]
Si nota che la resistenza equivalente \(R_{eq}\) è inversamente proporzionale al fattore di forma \(Z\) del MOSFET, per cui
aumentando il fattore di forma \(Z\) del MOSFET, si riduce la resistenza equivalente \(R_{eq}\) e quindi si riduce anche
la costante di tempo \(\tau = R_{eq} C\) del circuito, migliorando le prestazioni del circuito.

Analizzando i valori tipici delle tecnologie CMOS si osserva che:
\[k_n' \approx 3k_p', \quad V_{TN} \approx -V_{TP}, \quad V_{DSATN} \approx -\frac{2}{3} V_{DSATP} \;\; \Rightarrow \;\; R_{n0} \approx 2R_{p0}\]
\[\frac{R_{n0}}{R_{p0}} = \frac{k_n' V_{DSATN} (V_{DD} - V_{TN} - \frac{V_{DSATN}}{2})}{k_p' V_{DSATP} (-V_{DD} - V_{TP} - \frac{V_{DSATP}}{2})} \approx \frac{k_n' V_{DSATN}}{-k_p' V_{DSATP}} \approx 3 \cdot \frac{2}{3} = 3\]

\subsubsection*{Resistenze equivalenti per valori logici \say{cattivi}}
Si osserva che per approssimare al meglio il transitorio dei mosfet nella trasmissione di valori logici \say{cattivi}, è necessario
raddoppiare le resistenze equivalenti calcolate in precedenza:
\[\text{NMOS (carica):} \quad R_{n, cattivo} = 2 R_{n,buono} \qquad\qquad \text{PMOS (scarica):} \quad R_{p, cattivo} = 2 R_{p,buono}\]

\begin{center}
	\centering \includegraphics[width=0.7\textwidth]{immagini/5_circuiti_mosfet/req_3.png}
\end{center}

\subsection{Reti di MOSFET e resistenza equivalente complessiva}
\subsubsection*{Serie di MOS}
La serie di MOSFET equivale ad un MOSFET equivalente con resistenza equivalente pari alla sommma delle resistenze equivalenti
dei singoli MOSFET e con fattore di forma pari al reciproco della somma dei recicproci dei fattori di forma dei singoli MOSFET:
\begin{center}
	\begin{minipage}{0.4\textwidth}
		\centering \includegraphics[width=0.7\textwidth]{immagini/5_circuiti_mosfet/serie_mos.png}
	\end{minipage}
	\begin{minipage}{0.55\textwidth}
		\begin{align*}
			R_{eq} &= R_1 + R_2 \\[5pt]
			\frac{1}{Z_{eq}} &= \frac{1}{Z_1} + \frac{1}{Z_2} \;\; \rightarrow \;\; Z_{eq} = \frac{Z_1 Z_2}{Z_1 + Z_2}
		\end{align*}
	\end{minipage}
\end{center}

\subsubsection*{Parallelo di MOS}
La parallelo di MOSFET equivale ad un MOSFET equivalente con resistenza equivalente pari al recicproco della somma dei recicproci
delle resistenze equivalenti dei singoli MOSFET e con fattore di forma pari alla somma dei fattori di forma dei singoli MOSFET:
\begin{center}
	\begin{minipage}{0.4\textwidth}
		\centering \includegraphics[width=0.7\textwidth]{immagini/5_circuiti_mosfet/parallelo_mos.png}
	\end{minipage}
	\begin{minipage}{0.55\textwidth}
		\begin{align*}
			\frac{1}{R_{eq}} &= \frac{1}{R_1} + \frac{1}{R_2} \;\; \rightarrow \;\; R_{eq} = \frac{R_1 R_2}{R_1 + R_2} \\[5pt]
			Z_{eq} &= Z_1 + Z_2
		\end{align*}
	\end{minipage}
\end{center}

\section{Breve cenno ai segnali digitali}
%\section{Invertitore CMOS}
\subsection{Schema circuitale}

\subsubsection*{Schema circuitale e simbolo logico}
L'invertitore CMOS (detto anche Complementary-MOS) è il circuito logico più semplice realizzabile con la tecnologia cmos.
Implementa la funzione logica NOT, ovvero l'operazione di negazione booleana. Lo schema circuitale dell'invertitore cmos
è riportato in figura (a sinistra), insieme al simbolo logico corrispondente (a destra).

\begin{center}
	\begin{minipage}{0.4\textwidth}
		\centering
		\begin{circuitikz}
			\ctikzset{tripoles/pmos style/emptycircle}
		
			\draw (0,-0.7) -- (0,0.7);
			\draw (-0.7,0) node[left] {\(V_{IN}\)} to[short, *-] (0,0);
		
			\draw (0,-0.7) node[nmos, anchor=G, scale=1.2] (nmos) {};
			\draw (0,0.7) node[pmos, anchor=G, scale=1.2] (pmos) {};
			\draw (nmos.S) -- ++(0,0.3) node[sground] {};
			\draw (pmos.S) -- ++(0,-0.3) node[rground, rotate=180, name=VDD]{};
		
			\draw (pmos.D) ++ (0,0.23) to[short, -*] ++(0.7,0) node[right] {\(V_{OUT}\)};
			\draw (pmos.D) ++ (-0.05,0.45) node[right] {\(_\text{D}\)};
			\draw (pmos.S) ++ (-0.05,-0.45) node[right] {\(_\text{S}\)};
			\draw (nmos.D) ++ (-0.05,-0.45) node[right] {\(_\text{D}\)};
			\draw (nmos.S) ++ (-0.05,0.45) node[right] {\(_\text{S}\)};
			\draw (nmos.centergap) ++ (1.2,0) node[left] {\(_\text{NMOS}\)};
			\draw (pmos.centergap) ++ (1.2,0) node[left] {\(_\text{PMOS}\)};
		
			\node at ($(VDD)+(0, 0.6)$) {\(V_{DD}\)};
		\end{circuitikz}
	\end{minipage}
	\begin{minipage}{0.4\textwidth}
		\centering
		\begin{circuitikz}
			\ctikzset{logic ports=ieee}
			\draw (0,0) node[not port, scale=0.8] (NOT) {} (NOT.in) node[left] {IN} (NOT.out) node[right] {OUT};
		\end{circuitikz}
	\end{minipage}	
\end{center}

\noindent
L'invertitore cmos è costituito da un transistor nmos e un transistor pmos collegati in serie tra \(V_{DD}\) e massa. L'ingresso
del circuito è collegato ai gate di entrambi i transistor, mentre l'uscita è prelevata dal nodo di connessione tra i due transistor.

\subsubsection*{Funzionamento}
Per comprendere meglio il funzionamento dell'invertitore cmos si sostituiscono i transistor (usati come interruttori) con la
serie di interruttore e resistenza equivalente, come mostrato in figura. Si distinguono due casi principali in base al valore
logico dell'ingresso \(V_{IN}\):

\begin{center}
	\begin{minipage}{0.3\textwidth}
		\centering
		\begin{circuitikz}[scale=0.75, transform shape]
			\draw (0.7,0) node[right] {\(V_{OUT}\)} to[short, *-] (0,0);
			\draw (0,-0.4) to[short, *-*] (0,0.4);
			\draw (-0.3,-0.4) -- (0,-0.9); 
			\draw (0,-0.9) to[R, l_=\(R_{n,eq}\), *-] (0,-2.5) ++(0,0.2) node[sground] {};
			\draw (0,0.4) -- (-0.05,0.9);
			\draw (0,0.9) to[R, l=\(R_{p,eq}\), *-] (0,2.5) ++(0,-0.2) node[rground, rotate=180, name=VDD]{};
			\node at ($(VDD)+(0, 0.6)$) {\(V_{DD}\)};
		\end{circuitikz}
	\end{minipage}
	\begin{minipage} {0.6\textwidth}
		Quando \(V_{IN} = 0 \V\), il pmos è acceso, mentre l'nmos è spento. L'uscita \(V_{OUT}\) è, quindi, collegata a
		\(V_{DD}\) tramite la resistenza equivalente \(R_p\) del pmos e si ha \(V_{OUT} = V_{DD}\).

		Il pmos porta l'uscita al livello logico alto (valore logico \say{buono} per pmos) ed è chiamato transistor di pull-up.
	\end{minipage}
\end{center}

\begin{center}
	\begin{minipage}{0.3\textwidth}
		\centering
		\begin{circuitikz}[scale=0.75, transform shape]
			\draw (0.7,0) node[right] {\(V_{OUT}\)} to[short, *-] (0,0);
			\draw (0,-0.4) to[short, *-*] (0,0.4);
			\draw (-0.05,-0.4) -- (0,-0.9); 
			\draw (0,-0.9) to[R, l_=\(R_{n,eq}\), *-] (0,-2.5) ++(0,0.2) node[sground] {};
			\draw (0,0.4) -- (-0.3,0.9);
			\draw (0,0.9) to[R, l=\(R_{p,eq}\), *-] (0,2.5) ++(0,-0.2) node[rground, rotate=180, name=VDD]{};
			\node at ($(VDD)+(0, 0.6)$) {\(V_{DD}\)};
		\end{circuitikz}
	\end{minipage}
	\begin{minipage} {0.6\textwidth}
		Quando \(V_{IN} = V_{DD}\), il pmos è spento, mentre l'nmos è acceso. L'uscita \(V_{OUT}\) è, quindi, collegata a
		massa tramite la resistenza equivalente \(R_n\) dell'nmos e si ha \(V_{OUT} = 0\).

		L'nmos porta l'uscita al livello logico basso (valore logico \say{buono} per nmos) ed è chiamato transistor di pull-down.
	\end{minipage}
\end{center}

\noindent
In entrambi i casi l'uscita \(V_{OUT}\) assume il valore logico opposto rispetto all'ingresso \(V_{IN}\), come previsto
dalla funzione logica NOT.

\newpage

\subsection{Caratteristica di trasferimento ingresso-uscita}
La caratteristica statica di trasferimento, o caratteristica di trasferimento ingresso-uscita (VTC, voltage transfer
characteristic) è la curva che descrive il valore di \(V_{OUT}\) in funzione di \(V_{IN}\).

\subsubsection*{Calcolo grafico della VTC}
Per ottenerla, si uguagliano
le correnti attraverso i due transistor secondo la legge di Kirchhoff delle correnti (LKC).
\[I_{DS,n}(V_{IN},V_{OUT}) = I_{DS,p}(V_{IN},V_{OUT})\]

Per evitare di fare conti, si risolve l'equazione graficamente, ovvero si tracciano le curve delle due correnti in funzione
di \(V_{OUT}\) per diversi valori di \(V_{IN}\). I punti di intersezione delle due curve rappresentano le soluzioni
dell'equazione, ovvero i valori di \(V_{OUT}\) corrispondenti a ciascun valore di \(V_{IN}\).

\begin{center}
	\begin{minipage}{0.4\textwidth}
		\centering \includegraphics[width=0.9\textwidth]{immagini/7_invertitore/vtc_1.png}
	\end{minipage}
	\begin{minipage}{0.1\textwidth}
		\centering \(\longrightarrow\)
	\end{minipage}
	\begin{minipage}{0.4\textwidth}
		\centering \includegraphics[width=0.9\textwidth]{immagini/7_invertitore/vtc_2.png}
	\end{minipage}
\end{center}

\subsubsection*{Regioni di funzionamento dei transistor}
Si osserva che per \(V_{IN} = 0 \V\) o \(V_{IN} = V_{DD}\) un solo transistor è acceso in regime lineare, mentre l'altro è spento.
Per valori intermedi si ha che:
\begin{itemize}
	\item il pmos passa da lineare a saturazione e infine si spegne al crescere di \(V_{IN}\)
	\item l'nmos da interdizione, passa da saturazione e infine lineare al crescere di \(V_{IN}\)
\end{itemize}

\subsubsection*{Punti importanti}
La pendenza della VTC (ovvero la derivata di \(V_{OUT}\) in funzione di \(V_{IN}\)) corrisponde al guadagno in tensione
dell'invertitore. Si definiscono i seguenti punti importanti della VTC:

\begin{center}
	\begin{minipage}{0.4\textwidth}
		\centering \includegraphics[width=0.9\textwidth]{immagini/7_invertitore/vtc_3.png}
	\end{minipage}
	\begin{minipage}{0.5\textwidth}
		P0 e P1 corrispondo ai punti con guadagno unitario. Tali punti delimitano la regione di livello logico indefinito,
		con guadagno \(>\) 1 e le regioni di livello logico definito (a sinistra e a destra), con guadagno \(< 1\).
		\vspace{5pt}

		Si osserva che in corrispondenza delle regioni di livello logico definito la pendenza della curva è molto bassa
		(guadagno \(<\) 1), per cui piccole variazioni di \(V_{IN}\) producono variazioni trascurabili di \(V_{OUT}\)
		(proprietà di rigenerazione del segnale).
		\vspace{5pt}

		Il pallino rosso in cui \(V_{OUT} = V_{IN}\) è il punto di soglia di commutazione logica dell'invertitore. La
		tensione di ingresso associata a questo punto è detta soglia di commutazione logica \(V_{M}\).
	\end{minipage}
\end{center}

\subsection{Soglia di commutazione logica}
\subsubsection*{Calcolo della tensione di commutazione logica}
Per calcolare la soglia di commutazione logica \(V_{M}\) si impone la condizione \(V_{OUT} = V_{IN} = V_{M}\), ovvero si
cortocircuitano l'ingresso e l'uscita dell'invertitore. Si ottiene, quindi, la seguente equazione che dipende solo da \(V_{M}\):
\[I_{DS,n}(V_{M}) = I_{DS,p}(V_{M})\]
Si ottiene \(V_{M}\) risolvendo l'equazione:
\[V_M = \frac{\displaystyle V_{TN} + \frac{V_{DSATN}}{2} + r \left(V_{DD} + V_{TP} + \frac{V_{DSATP}}{2}\right)}{1+r} \qquad \text{con} \; r = - \frac{Z_p}{Z_n} \frac{k_p' V_{DSATP}}{k_n' V_{DSATN}}\]

\subsubsection*{Relazione tra tensione di commutazione logica e rapporto dei fattori di forma}
Si osserva che \(V_{M}\) oltre a dipendere dai parametri fisici dei transistor e dal valore di \(V_{DD}\), dipende anche dal
rapporto tra i fattori di forma dei due transistor \(Z_p / Z_n\). Modificando tale rapporto è possibile regolare il valore di
\(V_{M}\).
In particolare per avere \(V_M = V_{DD}/2\) si ottiene:
\[V_M = \frac{V_{DD}}{2} \quad \rightarrow \quad r = \frac{\frac{V_{DD}}{2} - V_{TN} - \frac{V_{DSATN}}{2}}{\frac{V_{DD}}{2} + V_{TP} + \frac{V_{DSATP}}{2}} \quad \rightarrow \quad \frac{Z_p}{Z_n} = - r \frac{k_n' V_{DSATN}}{k_p' V_{DSATP}} \approx 3 - 3.5\]
Analizzando graficamente come varia \(V_{M}\) al variare del rapporto \(Z_p / Z_n\), si osserva che:
\begin{itemize}
	\item per \(Z_p > 3 Z_n\) aumenta \(V_{M}\), prevale il pmos e la VTC si sposta verso destra (il valore logico alto si mantiene più a lungo)
	\item per \(Z_p < 3 Z_n\) diminuisce \(V_{M}\), prevale l'nmos e la VTC si sposta verso sinistra (il valore logico basso si mantiene più a lungo)
	\item per \(Z_p = 3 Z_n\) si ha \(V_{M} \approx V_{DD}/2\) e i due transistor si bilanciano e la VTC è simmetrica
\end{itemize}
Infine si nota che la curva è molto piatta in corrispondenza della soglia di commutazione logica, il che implica che \(V_M\)
è poco sensibile alle variazioni dei fattori di forma dei transistor.

\begin{center}
	\begin{minipage}{0.4\textwidth}
		\centering \includegraphics[width=0.9\textwidth]{immagini/7_invertitore/soglia_1.png}

		\small{\(V_M\) al variare di \(Z_p / Z_n\), in scala logaritmica}
	\end{minipage}
	\begin{minipage}{0.55\textwidth}
		\centering \includegraphics[width=0.9\textwidth]{immagini/7_invertitore/soglia_2.png}

		\small{VTC al variare di \(Z_p / Z_n\)}
	\end{minipage}
\end{center}

\subsection{Tolleranza al rumore}
\subsubsection*{Tipi di disturbi in un circuito elettrico}
In un circuito elettrico possono essere presenti diversi tipi di disturbi che possono alterare il corretto funzionamento del
circuito stesso. I principali tipi di disturbi sono:
\begin{itemize}
	\item \textbf{accoppiamenti induttivi}: disturbi causati dalla mutua induzione di due conduttori vicini dovuta all'induttanza
	parassita delle piste di collegamento
	\item \textbf{accoppiamenti capacitivi}: disturbi causati dalla capacità parassita tra due conduttori vicini
	\item \textbf{rumore dell'alimentazione}: variazioni indesiderate della tensione di alimentazione del circuito dovuta ad
	esempio dalla caduta di tensione per le resistenze lungo le linee di alimentazione
\end{itemize}

\subsubsection*{Margine di immunità al rumore}
Il margine di immunità al rumore o NM (noise margin) rappresenta l'ampiezza massima del disturbo che è possibile avere durante
la trasmissione del segnale tra due porte logiche, senza che si verifichino errori di interpretazione del segnale logico. In base
al livello logico si distinguono due margini di immunità:
\begin{itemize}[topsep=0pt]
	\item NM per il livello logico alto: \(N\!M_H = V_{O\!H} - V_{I\!H}\) ovvero la
	differenza tra la minima tensione di uscita di un segnale a livello logico alto e la minima tensione di ingresso per
	riconoscere un segnale come livello logico alto;
	\item NM per il livello logico basso: \(N\!M_L = V_{I\!L} - V_{O\!L}\) ovvero la
	differenza tra la massima tensione di ingresso per riconoscere un segnale come livello logico basso e la massima tensione
	di uscita di un segnale a livello logico basso.
\end{itemize}
Il margine di immunità complessivo, ovvero il massimo disturbo tollerabile, corrisponde al margine di immunità più piccolo tra
i due livelli logici \(N\!M = \min(N\!M_H, N\!M_L)\).

\subsubsection*{Rigenerazione del segnale}
La rigenerazione del segnale è la capacità di un circuito logico di ripristinare i livelli logici di uscita a valori
vicini ai valori ideali \(V_{O\!H}\) e \(V_{O\!L}\), attenuando gli effetti dei disturbi e del rumore sul segnale.

Affinché una porta logica sia in grado di rigenerare il segnale, è necessario che il guadagno della VTC sia minore di 1 nelle
regioni di livello logico definito (VTC tendente all'orizzontale) e maggiore di 1 nella regione di livello logico
indefinito (VTC tendente al verticale).

\begin{center}
	\begin{minipage}{0.34\textwidth}
		\centering \includegraphics[width=\textwidth]{immagini/7_invertitore/noise_1.png}
	\end{minipage}
	\begin{minipage}{0.64\textwidth}
		\centering \includegraphics[width=\textwidth]{immagini/7_invertitore/noise_2.png}
	\end{minipage}
\end{center}

\subsubsection*{Ruolo della tensione di commutazione logica nella rigenerazione del segnale}
I margini di immunità al rumore per un invertitore cmos possono essere approssimati come:
\[N\!M_H \approx V_{DD} - V_{M} \qquad N\!M_L \approx V_{M} \qquad N\!M \approx \min(N\!M_H, N\!M_L)\]
Si osserva che per massimizzare il margine di immunità complessivo \(N\!M\) è necessario bilanciare i due margini di immunità, ovvero
impostare la soglia di commutazione logica \(V_{M}\) a metà della tensione di alimentazione \(V_{DD}/2\). Si ottiene così che
il dimensionamento ottimo dei transistor per massimizzare la tolleranza al rumore è \(Z_p \approx 3 Z_n\).

\subsection{Invertitore ideale}
Un invertitore ideale è un invertitore logico che interpreta \(V_{IN} < V_M\) come livello logico basso e \(V_{IN} > V_M\)
come livello logico alto, inoltre rigenera l'uscita esattamente ai valori ideali \(V_{O\!H} = V_{DD}\) e \(V_{O\!L} = 0 \V\).

\begin{center}
	\begin{minipage}{0.34\textwidth}
		\centering \includegraphics[width=\textwidth]{immagini/7_invertitore/invertitore_ideale.png}
	\end{minipage}
	\begin{minipage}{0.64\textwidth}
		La curva VTL di un invertitore ideale è una curva a gradino che passa per il punto di soglia di commutazione logica
		e la funzione di trasferimeno è una funzione definita a tratti:
		\[V_{OUT} = \begin{cases}
			V_{DD} \;\; \text{se} \; V_{IN} < V_{M} \\
			V_{M} \;\; \text{se} \; V_{IN} = V_{M} \\
			0 \V \;\; \text{se} \; V_{IN} > V_{M}
		\end{cases}\]

		I margini di immunità al rumore diventanto esattamente
		\[N\!M_H = V_{DD} - V_{M} \qquad N\!M_L = V_{M}\]
	\end{minipage}
\end{center}

\subsection{Tempo di ritardo}
\subsubsection*{Modello RC dell'invertitore}
Ad ogni terminale dei transistor sono associate delle capacità parassite. Queste capacità vanno caricate e scaricate durante la
commutazione del segnale, causando un ritardo temporale tra l'ingresso e l'uscita dell'invertitore. Si rappresenta, quindi,
l'invertitore con un modello RC equivalente in cui le capacità parassite sono evidenziate in rosso ed è stata aggiunta la
capacità di carico \(C_L\) collegata all'uscita del circuito:

\begin{center}
	\begin{minipage}{0.25\textwidth}
		\centering
		\begin{circuitikz}
			\ctikzset{tripoles/pmos style/emptycircle}
		
			\draw (0,-0.7) -- (0,0.7);
			\draw (-0.3,0) node[left] {\(V_{IN}\)} to[short, *-] (0,0);
		
			\draw (0,-0.7) node[nmos, anchor=G, scale=1.2] (nmos) {};
			\draw (0,0.7) node[pmos, anchor=G, scale=1.2] (pmos) {};
			\draw (nmos.S) -- ++(0,0.3) node[sground] {};
			\draw (pmos.S) -- ++(0,-0.3) node[rground, rotate=180, name=VDD]{};
		
			\draw (pmos.D) ++ (0,0.23) to[short, -*] ++(0.5,0) node[right] {\(V_{OUT}\)};
			\draw (pmos.D) ++ (-0.05,0.45) node[right] {\(_\text{D}\)};
			\draw (pmos.S) ++ (-0.05,-0.45) node[right] {\(_\text{S}\)};
			\draw (nmos.D) ++ (-0.05,-0.45) node[right] {\(_\text{D}\)};
			\draw (nmos.S) ++ (-0.05,0.45) node[right] {\(_\text{S}\)};
			\draw (nmos.centergap) ++ (1.2,0) node[left] {\(_\text{NMOS}\)};
			\draw (pmos.centergap) ++ (1.2,0) node[left] {\(_\text{PMOS}\)};
		
			\node at ($(VDD)+(0, 0.6)$) {\(V_{DD}\)};
		\end{circuitikz}
	\end{minipage}
	\begin{minipage}{0.05\textwidth}
		\centering \(\longrightarrow\)
	\end{minipage}
	\begin{minipage}{0.38\textwidth}
		\centering
		\begin{circuitikz}
			\ctikzset{bipoles/length=0.8cm}

			\draw (-2,0) node[left] {\(V_{IN}\)} to[short, *-] (-1.3,0);
			\draw [<->] (-0.3,-0.6) -- (-1.3,-0.6) -- (-1.3,0.6) -- (-0.3,0.6);

			\draw (2,0) node[right] {\(V_{OUT}\)} to[short, *-] (0,0);
			\draw (0,-0.4) to[short, *-*] (0,0.4);
			\draw (-0.3,-0.4) -- (0,-0.9); 
			\draw (0,-0.9) to[R, l_=\(R_{n}\), *-] (0,-2.5) node[sground] {};
			\draw (0,0.4) -- (-0.3,0.9);
			\draw (0,0.9) to[R, l=\(R_{p}\), *-] (0,2.5) node[rground, rotate=180, name=VDD]{};
			\node at ($(VDD)+(0, 0.5)$) {\(V_{DD}\)};

			\color{red}
			\draw (-0.9,-0.6) to[C, l_=\(\;_{C_{gn}}\)] ++(0,-0.7) ++(0,0.2) node[sground, color=black]{};
			\draw (-0.9,0.6) to[C, l=\(\;_{C_{gp}}\)] ++(0,0.7) ++(0,-0.2) node[rground, rotate=180, name=VDD, color=black]{};
			\node at ($(VDD)+(0, 0.4)$) [color=black] {\(\;_{V_{DD}}\)};

			\draw (0.7,0) to[C, l=\(\;_{C_{dn}}\)] ++(0,-0.7) ++(0,0.2) node[sground, color=black]{};
			\draw (0.7,0) to[C, l_=\(\;_{C_{dp}}\)] ++(0,0.7) ++(0,-0.2) node[rground, rotate=180, name=VDD, color=black]{};
			\node at ($(VDD)+(0, 0.4)$)[color=black] {\(\;_{V_{DD}}\)};

			\draw (0,-2.5) -- ++(0.7,0) to[C, l_=\(\;_{C_{sn}}\)] ++(0,0.7) ++(0,-0.2) node[sground, rotate=180, color=black]{};
			\draw (0,2.5) -- ++(0.7,0) to[C, l=\(\;_{C_{sp}}\)] ++(0,-0.7) ++(0,0.2) node[rground, name=VDD, color=black]{};
			\node at ($(VDD)+(0, -0.4)$) [color=black] {\(\;_{V_{DD}}\)};

			\draw (2,0) -- ++(0,-0.2) to[C, l=\(C_{L}\)] ++(0,-0.7) -- ++(0,-0.2) ++(0,0.2) node[sground, color=black]{};
		\end{circuitikz}
	\end{minipage}
	\begin{minipage}{0.05\textwidth}
		\centering \(\longrightarrow\)
	\end{minipage}
	\begin{minipage}{0.24\textwidth}
		\centering
		\begin{circuitikz}
			\ctikzset{bipoles/length=0.8cm}

			\draw (-1,0) node[left] {\(V_{IN}\)} to[short, *-] (-0.7,0);
			\draw [<->] (-0.3,-0.6) -- (-0.7,-0.6) -- (-0.7,0.6) -- (-0.3,0.6);

			\draw (0.7,0) node[right] {\(V_{OUT}\)} to[short, *-] (0,0);
			\draw (0,-0.4) to[short, *-*] (0,0.4);
			\draw (-0.3,-0.4) -- (0,-0.9); 
			\draw (0,-0.9) to[R, l_=\(R_{n}\), *-] (0,-2.5) ++(0,0.2) node[sground] {};
			\draw (0,0.4) -- (-0.3,0.9);
			\draw (0,0.9) to[R, l=\(R_{p}\), *-] (0,2.5) ++(0,-0.2) node[rground, rotate=180, name=VDD]{};
			\node at ($(VDD)+(0, 0.5)$) {\(V_{DD}\)};

			\color{red}
			\draw (0.7,0) -- ++(0,-0.2) to[C, l=\(C_{OUT}\)] ++(0,-0.7) -- ++(0,-0.2) ++(0,0.2) node[sground, color=black]{};
		\end{circuitikz}
	\end{minipage}	
\end{center}

\noindent
Analizzando le resistenze e le capacità del modello RC equivalente per i nodi di ingresso ed uscita si ha:
\[R_{IN} = \infty \qquad C_{IN} = C_{gn} + C_{dp} \qquad R_{OUT} =\frac{R_{n} + R_{p}}{2} \qquad C_{OUT} = C_{gp} + C_{dn} + C_{L}\]
NOTA: le capacità dei source sono cortocircuitate per cui sono ininfluenti, inoltre le capacità del pmos possono essere considerate
collegate tutte a massa invece che a \(V_{DD}\), come visto alla fine del paragrafo delle \hyperref[capacità_parassite_mosfet]{capacità parassite dei mosfet}.

\subsubsection*{Calcolo dei tempi di ritardo intrinseci}
Il tempo di ritardo è il tempo che impiega la porta di uscita a raggiungere il 50\% del valore finale dopo un cambio di stato
dell'ingresso. Il tempo di ritardo si dice intrinseco se il carico esterno è nullo (\(C_L = 0\)).
Si definiscono due tempi di ritardo intrinseci:
\begin{itemize}
	\item tempo di ritardo di salita \(t_{pLH0}\): tempo che impiega l'uscita a salire dal 50\% del livello logico basso al
	50\% del livello logico alto dopo un fronte di salita dell'ingresso
	\[V_{OUT} = \frac{V_{DD}}{2} \;\; \rightarrow \;\; V_{DD} - V_{DD} e ^{-\tfrac{t_{pLH0}}{R_{p}C_{OUT}}} = \frac{V_{DD}}{2} \;\; \rightarrow \;\; t_{pLH0} = \ln(2) \, R_p \, C_{OUT} \approx 0.69 \, R_p \, C_{OUT}\]
	\item tempo di ritardo di discesa \(t_{PHL0}\): tempo che impiega l'uscita a scendere dal 50\% del livello logico alto al
	50\% del livello logico basso dopo un fronte di discesa dell'ingresso
	\[V_{OUT} = \frac{V_{DD}}{2} \;\; \rightarrow \;\; V_{DD} e ^{-\tfrac{t_{pHL0}}{R_{n}C_{OUT}}} = \frac{V_{DD}}{2} \;\; \rightarrow \;\; t_{pHL0} = \ln(2) \, R_n \,  C_{OUT} \approx 0.69 \; R_n \, C_{OUT}\]
\end{itemize}

Si calcola il tempo di propagazione intrinseco medio come:
\[t_{p0} = \frac{t_{pLH0} + t_{pHL0}}{2} = \ln(2) \, \frac{R_n + R_p}{2} \, C_{OUT} \approx 0.69 \,\frac{R_n + R_p}{2} \, C_{OUT} = 0.69 \,R_{OUT} \, C_{OUT}\]

\newpage

\subsubsection*{Carico esterno e fan-out}
Quando l'invertitore guida un carico esterno \(C_L\), i tempi di ritardo aumentano in quanto la capacità totale da caricare
o scaricare è maggiore. Si definiscono il fan-out \(f\) e il coefficiente di carico \(\gamma\):
\[f = \frac{C_L}{C_{IN}} \qquad \gamma = \frac{C_{OUT}}{C_{IN}}\]
Si ottengono, quindi, i tempi di ritardo con carico esterno:
\[t_{pLH} = 0.69 \, R_p \, (C_{OUT} + C_L) = 0.69 \, R_p \, C_{OUT} + 0.69 \, R_p C_L = t_{pLH0} + 0.69 \, R_p C_L\]
\[t_{pHL} = 0.69 \, R_n \, (C_{OUT} + C_L) = 0.69 \, R_n \, C_{OUT} + 0.69 \, R_n C_L = t_{pHL0} + 0.69 \, R_n C_L\]
\[t_p = \frac{t_{pLH} + t_{pHL}}{2} = 0.69 \, R_{OUT} \, (C_{OUT} + C_L) = t_{p0} + 0.69 \, R_{OUT} C_L = t_{p0} \left(1 + \frac{f}{\gamma}\right)\]

\subsubsection*{Ottimizzazione del tempo di ritardo intrinseco}
Si osserva che il tempo di ritardo dipende da una serie di parametri tecnologici (specifici dei materiali) e da altri parametri
di progettazione che possono essere scelti liberamente. In particolare si può agire su:
\begin{itemize}
	\item il rapporto tra i fattori di forma dei transistor \(Z_p / Z_n\)
	\item il fattore di carico \(f\) che dipende dall'architettura del circuito logico
	\item la tensione di alimentazione \(V_{DD}\), difficile da modificare in quanto spesso imposta da vincoli esterni
\end{itemize}
Il tempo di ritardo può essere ottimizzato secondo i seguenti criteri:
\begin{itemize}
	\item ottimizzare il tempo massimo di ritardo \(\max (t_{pHL}, t_{pLH})\)
	\item ottimizzare il tempo di ritardo medio \(t_p\)
\end{itemize}
Si definiscono quindi i coefficienti \(\beta\) e \(\rho\) e si procede all'ottimizzazione scegliendo il valore ottimale di
\(\beta\) in base al criterio scelto:
\[\beta = \frac{Z_p}{Z_n} = \frac{W_p}{W_n} \qquad \rho = \frac{R_{p0}}{R_{n0}}\]
\[t_{pHL0} = 0.69 R_n C_{OUT} = 0.69 \frac{R_{n0}}{Z_n} C_{d0} W_n (1+\beta) = 0.69 R_{n0}C_{d0}L \left(1 + \beta\right)\]
\[t_{pLH0} = 0.69 R_p C_{OUT} = 0.69 \frac{\rho R_{n0}}{\beta Z_n} C_{d0} W_n (1+\beta) = 0.69 R_{n0}C_{d0}L \left(\frac{\rho}{\beta} + \rho\right)\]
\[t_{p0} = 0.69 \, \frac{R_{n0}C_{d0}L}{2}\left(1 + \beta + \rho + \frac{\rho}{\beta}\right)\]

\begin{center}
	\begin{minipage}{0.45\textwidth}
		\centering \includegraphics[width=0.9\textwidth]{immagini/7_invertitore/ritardo_1.png}
	\end{minipage}
	\begin{minipage}{0.5\textwidth}
		Ottimizzando il tempo massimo di ritardo (ovvero imponendo \(t_{pHL0} = t_{pLH0}\), punto blu in figura):
		\begin{align*}
			t_{pHL0} = t_{pLH0} \;\; &\rightarrow \;\; 1 + \beta = \frac{\rho}{\beta} + \rho\\
			&\rightarrow \;\; \beta = \rho \approx 2 - 2.5
		\end{align*}

		Ottimizzando il tempo medio di ritardo (ovvero trovando il minimo di \(t_{p0}\), punto rosso in figura):
		\begin{align*}
			\frac{d t_{p0}}{d \beta} = 0 \;\; &\rightarrow \;\; 1 - \frac{\rho}{\beta^2} = 0 \\
			&\rightarrow \;\; \beta = \sqrt{\rho} \approx 1.4 - 1.6
		\end{align*}
	\end{minipage}
\end{center}

\noindent
Si nota che il valore di \(\beta\) ottimizza anche l'affidabilità del circuito in quanto agisce sulla soglia di commutazione
logica \(V_{M}\) e quindi sui margini di immunità al rumore. Per avere affidabilità massima (ovvero \(V_M = V_{DD}/2\)) bisogna
scegliere \(\beta \approx 3\), ma siccome \(V_M\) reagisce molto poco alle variazioni di \(\beta\), si predilige scegliere
\(\beta\) in modo da ottimizzare il tempo di ritardo.

\subsubsection*{Dimensionamento dell'invertitore con carico esterno}
Fissato un certo \(\beta\) si calcola come ottimizzare il tempo di ritardo in funzione della capacità di carico esterna \(C_L\)
andando a dimensionare opportunamente i transistor. Si ottiene:
\[t_p = 0.69 R_{OUT} C_{OUT} \left(1 + \frac{C_L}{C_{OUT}}\right) = t_{p0} \left(1 + \frac{C_L}{C_{OUT}}\right) \quad \text{con } t_{p0} \text{ indipendente da } C_L\]
\[1 + \frac{C_L}{C_{OUT}} = 1+ \frac{C_L}{C_{d0} W_n (1 + \beta)} \approx 1 + C_L \frac{\text{costante}}{W_n} \qquad R_{OUT} = \frac{1}{2} \frac{R_{n0}}{Z_n} \left(1 + \frac{\rho}{\beta}\right) \approx \frac{\text{costante}}{Z_n}\]

\noindent
Si osserva che per minimizzare il tempo di ritardo si può aumentare la larghezza \(W_n\) (e proporzionalmente
anche \(W_p = \beta W_n\) e \(Z_n = \beta Z_p\)) in modo da ridurre sia \(R_{OUT}\) che il termine
\(\left(1 + \frac{C_L}{C_{OUT}}\right)\). Facendo così, però, si aumenta l'area occupata dal circuito e la capacità
parassita di uscita \(C_{OUT}\).

\begin{center}
	\begin{minipage}{0.45\textwidth}
		\centering \includegraphics[width=0.9\textwidth]{immagini/7_invertitore/ritardo_2.png}
	\end{minipage}
	\begin{minipage}{0.5\textwidth}
		Osservando il grafico di \(t_p\) in funzione di \(Z_n = W_n/L\) si osserva che:
		\begin{itemize}
			\item  diminuendo \(Z_n\) il tempo di ritardo aumenta a causa dell'aumento di \(R_{OUT}\) che limita
			la corrente di carica/scarica di \(C_{OUT} + C_L\)
			\item  aumentando \(Z_n\) il tempo di ritardo diminuisce inizialmente a causa della diminuzione di \(R_{OUT}\),
			poi si stabilizza attorno ad un valore minimo dovuto al fatto che l'invertitore deve anche \say{auto-caricarsi} la
			capacità parassita \(C_{OUT}\) che è ormai diventata significativa rispetto al carico \(C_L\)
		\end{itemize}
	\end{minipage}
\end{center}

\subsection{Consumo statico}
Il consumo statico è il consumo di potenza dell'invertitore quando l'ingresso è mantenuto costante ad un livello logico definito
(alto o basso). In questo caso uno dei due transistor è sempre spento per cui non c'è corrente di drenaggio tra i due terminali
di alimentazione \(V_{DD}\) e massa. L'unico contributo al consumo statico è dovuto alle correnti di sottosoglia dei due transistor,
alle correnti di perdita dell'ossido di gate e alla corrente inversa del diodo drain-substrato. Questi contributi sono però
trascurabili, per cui si può considerare che il consumo statico di un invertitore cmos sia praticamente nullo.

\subsection{Consumo dinamico}
\subsubsection*{Analisi del consumo dinamico per una porta logica generale}
Il consumo dinamico è il consumo di potenza di una generica porta logica durante la commutazione del segnale di uscita. La potenza
assorbita dalla porta logica viene utilizzata per la carica e scarica delle capacità parassite e del carico esterno. Si analizza
il consumo dinamico per entrambe le fasi di commutazione dell'uscita (LH e HL):
\begin{itemize}
	\item commutazione dell'uscita LH:
		\begin{align*}
			&\text{energia assorbita} & &E_{V_{DD}} = \int_0^T V_{DD} \; i_{DD}(t) \; dt = V_{DD} \int_{V_{OL}}^{V_{OH}} C \cdot dV = C V_{DD} (V_{OH} - V_{OL}) \\
			&\text{energia immagazzinata} & &E_{C} = \int_{V_{OL}}^{V_{OH}} V \cdot C \cdot dV = \frac{1}{2} C {V_{OH}}^2 - \frac{1}{2} C {V_{OL}}^2 \\
			&\text{energia dissipata} & &E_{diss} = E_{V_{DD}} - E_{C}
		\end{align*}
	\item commutazione dell'uscita HL:
		\begin{align*}
			&\text{energia assorbita} & &E_{V_{DD}} = 0  \qquad \text{il pmos è spento e non viene assorbita corrente da } V_{DD}\\
			&\text{energia immagazzinata} & &E_{C} = \int_{V_{OH}}^{V_{OL}} V \cdot C \cdot dV = \frac{1}{2} C {V_{OL}}^2 - \frac{1}{2} C {V_{OH}}^2 \\
			&\text{energia dissipata} & &E_{diss} = - E_{C}
		\end{align*}
\end{itemize}
Complessivamente si ottiene che l'energia totale assorbita dal generatore \(V_{DD}\) durante un ciclo di commutazione completo
(LH + HL) è pari all'energia dissipata e vale:
\[E_{V_{DD}, tot} = E_{diss, tot} = C \cdot V_{DD} (V_{OH} - V_{OL})\]
Definita la frequenza di commutazione \(f\) (numero di cicli di commutazione al secondo), si ottiene la potenza dinamica
dissipata dalla porta logica:
\[P_{DYN} = E_{diss, tot} \cdot f = C \cdot V_{DD} (V_{OH} - V_{OL}) \cdot f\]

\subsubsection*{Consumo dinamico dell'invertitore}
Applicando la formula generale del consumo dinamico all'invertitore cmos si ottiene:
\[P_{DYN, invertitore} = C_{OUT} \cdot {V_{DD}}^2 \cdot f\]

\subsubsection*{Cammino diretto}
Si osserva che durante la commutazione del segnale di uscita, per un breve intervallo di tempo, entrambi i transistor
possono essere contemporaneamente accesi, creando un cammino diretto tra \(V_{DD}\) e massa. Ciò si verifica quando la tensione
di ingresso attiva entrami i mosfet per legge \(V_{TN} < V_{IN} < V_{DD} - V_{TP}\) e il tempo di durata del cammino diretto
si indica con \(t_{cc}\). In questo intervallo di tempo si ha una corrente di cortocircuito \(I_{CC}\) che causa una dissipazione
di potenza addizionale e sottrae corrente alla carica/scarica delle capacità parassite e del carico esterno, allungando i tempi
di ritardo. Per questo motivo si vuole minimizzare la durata del cammino diretto durante la progettazione dell'invertitore.

\subsection{Oscillatore ad anello}
L'oscillatore ad anello è un circuito costituito da un numero dispari di invertitori collegati in cascata, in cui l'uscita
dell'ultimo invertitore è collegata all'ingresso del primo. In questo modo si crea un circuito ad anello chiuso che genera
un segnale che commuta periodicamente tra i livelli logici alto e basso ad una data frequenza.

NOTA1: per il corretto funzionamento dell'oscillatore è necessario che il numero di invertitori sia dispari e maggiore di 3:
con un invertitore si ha un circuito stabile alla soglia logica, con un numero pari di invertitori si ha un circuito che si
stabilizza ad uno dei due livelli logici senza commutare mai.

NOTA2: è necessario, inoltre, che il tempo di salita e discesa del segnale attraverso un singolo invertitore sia molto inferiore
al tempo di propagazione del segnale su tutto l'anello (metà del periodo di oscillazione), altrimenti il circuito non riesce a
commutare correttamente e si comporta come un singolo invertitore stabilizzato alla soglia logica.

Analizzando periodo, frequenza di oscillazione e consumo dinamico si ottiene:
\[T = 2 N t_p \qquad f = \frac{1}{T} = \frac{1}{2 N t_p} \qquad P_{DYN} = N \cdot C_{X} \, {V_{DD}}^2 \, f \qquad \text{per } N > t_{r,f} / t_p\]
dove \(N\) è il numero di invertitori nell'anello, \(t_p\) è il tempo di propagazione di un singolo invertitore e \(C_X\)
è la capacità di un nodo interno all'anello.

\begin{center}
	\begin{minipage}{0.55\textwidth}
		\centering \includegraphics[width=0.9\textwidth]{immagini/7_invertitore/oscillatore_1.png}

		\small{Oscillatore ad anello con 7 invertitori}
	\end{minipage}
	\begin{minipage}{0.4\textwidth}
		\centering \includegraphics[width=0.9\textwidth]{immagini/7_invertitore/oscillatore_2.png}

		\small{Segnale di uscita dell'oscillatore ad anello}
	\end{minipage}
\end{center}

\subsection{Buffer cmos}
\subsubsection*{Struttura generale}
Il buffer cmos è un circuito costituito da una serie di invertitori collegati in cascata, utilizzato per aumentare la capacità di
pilotaggio del segnale di uscita e migliorare le prestazioni del circuito logico.
\begin{center}
	\includegraphics[width=0.8\textwidth]{immagini/7_invertitore/buffer.png}
\end{center}

\noindent
Si suppone per ipotesi che \(L\), \(C_{d0}\) e \(C_{g0}\) siano uguali per tutti i transistor (pmos e nmos) di ogni invertitore.
Di conseguenza anche il parametro \(\gamma\) risulta essere uguale per tutti gli invertitori del buffer e indipendente dalle
dimensioni dei transistor.
\[\gamma_i = \frac{C_{OUT}}{C_{IN}} = \frac{C_{d0}}{C_{g0}L} \cdot \frac{W_p + W_n}{W_p + W_n} = \frac{C_{d0}}{C_{g0}L} = \text{costante per ogni invertitore}\]

\noindent
Il tempo di ritardo complessivo è dato dalla somma dei tempi di ritardo di ogni singolo invertitore:
\[t_{p,tot} = \sum_{i=1}^{N} t_{p,i} \qquad \text{con} \;\; t_{p,i} = t_{p0,i} \left(1 + \frac{f_i}{\gamma_i}\right)\]

\noindent
Le possibili ottimizzazioni per minimizzare il tempo di ritardo del buffer sono:
\begin{itemize}
	\item ottimizzare il tempo medio di ritardo intrinseco \(t_{p0}\)
	\item ottimizzare il numero di stadi \(N\)
	\item ottimizzare il fattore di carico \(f\) tra uno stadio e l'altro
\end{itemize}

\subsubsection*{Ottimizzazione di \(t_{p0}\) - dimensionamento del singolo invertitore}
Negli invertitori il segnale viene ripetutamente invertito e il tempo di ritardo complessivo è dato dalla somma dei tempi di
salita \(t_{pLH}\) e discesa \(t_{pHL}\) di ogni invertitore. Il modo migliore per ottimizzare il tempo complessivo è
minimizzare la somma dei tempi di ritardo \(t_{pLH}\) e \(t_{pHL}\), ovvero minimizzare il tempo di ritardo medio \(t_{p0}\)
di ogni stadio. Di conseguenza (per come è stato visto in precedenza) si deve scegliere il rapporto tra i fattori di forma
dei transistor come \(\beta = \sqrt{\rho}\). In questo modo tutti gli invertitori hanno lo stesso tempo di ritardo intrinseco
minimo: \[t_{p0} = 0.69 \, \frac{R_{n0}C_{d0}L}{2}\left(1 + \beta + \rho + \frac{\rho}{\beta}\right) \qquad R_{n0}, C_{d0}, L, \beta = \sqrt{\rho} \; \text{costanti per ogni invertitore}\]

\subsubsection*{Ottimizzazione di \(f\) - dimensionamento progressivo degli stadi}
Il fanout di un singolo invertitore dipende dal rapporto tra il fattore di forma dei suoi transistor e quello dell'invertitore successivo.
\[f_i = \frac{C_{IN, i+1}}{C_{IN, i}} = \frac{C_{g0} L (W_{p,i+1} + W_{n,i+1})}{C_{g0} L (W_{p,i} + W_{n,i})} = \frac{Z_{p,i+1} + Z_{n,i+1}}{Z_{p,i} + Z_{n,i}} = \frac{\beta Z_{n,i+1} + Z_{n,i+1}}{\beta Z_{n,i} + Z_{n,i}} = \frac{Z_{n,i+1}}{Z_{n,i}}\]
Si assume che tale rapporto sia costante per ogni stadio ottenendo così un dimensionamento progressivo dei transistor nei vari
invertitori tale per cui ogni invertitore ha dimensioni \(f\) volte maggiori del precedente e ogni nodo ha capacità \(f\) volte
maggiore del precedente:
\[\begin{array}{c c}
	Z_{n,i+1} = f \, Z_{n,i} = f^{i} \, Z_{n,1} & W_{n,i+1} = f \, W_{n,i} = f^{i} \, W_{n,1} \\
	Z_{p,i+1} = f \, Z_{p,i} = f^{i} \, Z_{p,1} & W_{p,i+1} = f \, W_{p,i} = f^{i} \, W_{p,1}
\end{array} \qquad C_{X,j} = f \, C_{X,j-1} = f^i \, C_{X,1}\]

\subsubsection*{Calcolo del valore di \(f\) ottimale}
Si osserva che tutti gli stadi hanno lo stesso tempo di ritardo \(t_{p,i}\), essendo \(t_{p0}\), \(\gamma\) e \(f\) costanti
per ogni invertitore, per cui è possibile calcolare il tempo di ritardo complessivo del buffer come:
\[t_{p,tot} = N \cdot t_p = N \cdot t_{p0} \left(1+\frac{f}{\gamma}\right) \qquad \text{per} \;\; t_{p} = t_{p0} \left(1+\frac{f}{\gamma}\right)\]
Calcolando il fanout totale del buffer si ottiene il numero di stadi necessari per raggiungere il carico esterno \(C_L\):
\[F = \frac{C_{L}}{C_{IN}} = \frac{C_{in,2}}{C_{in,1}} \frac{C_{in,3}}{C_{in,2}} \dots \frac{C_{in,N}}{C_{in,N-1}} = f_1 f_2 f_3 \dots f_N = f^N \quad \rightarrow \qquad N = \frac{\ln(F)}{\ln(f)}\]
Sostituendo il valore di \(N\) con l'espressione appena trovata, si ottiene un'espressione per il tempo di ritardo complessivo
con l'unica incognita \(f\). Per determinare il valore ottimale di \(f\) si cerca il tempo di ritardo è minimo, analizzando
la derivata rispetto a \(f\):
\[t_{p,tot} = \frac{\ln F}{\ln f} t_{p0} \left(1+\frac{f}{\gamma}\right) \qquad\qquad \frac{\partial t_{p,tot}}{\partial f} \;\; \rightarrow \;\; f = e^{1+\frac{\gamma}{f}}\]
L'equazione precedente non ammette una soluzione analitica, per cui si ricorre a metodi numerici per trovare il valore ottimale
di \(f\). Per \(\gamma = 1\) risulta \(f = 3.6\).

\subsubsection*{Calcolo del valore di \(N\), buffer invertente e non invertente}
Una volta determinato il valore di \(f\) e conoscendo il fanout totale \(F\) si può calcolare il numero di stadi necessari per
realizzare il buffer arrontondando il valore ottenuto all'intero più vicino:
\[N = \frac{\ln(F)}{\ln(f)}\]
Si osserva che se si vuole realizzare un buffer non invertente, è necessario avere un numero pari di stadi, mentre se si vuole
realizzare un buffer invertente, è necessario avere un numero dispari di stadi.

\subsubsection*{Consumo dinamico del buffer}
L'energia necessaria a far commutare l'uscita del buffer è usata per caricare e scaricare le capacità parassite di ogni
invertitore e la capacità di carico esterna \(C_L\). Per cui nella formula generale del consumo dinamico, al posto di \(C_{OUT}\)
si deve usare la somma delle capacità dei nodi interni più la capacità del nodo di uscita:
\[C_{tot} = \sum_{i=1}^{N-1} C_{X,i} + C_F \qquad \begin{array}{l l}
	C_{X,i} = C_{OUT,i} + C_{IN,i+1} & \text{capacità nodo interno i-esimo} \\
	C_F = C_{OUT,N} + C_L & \text{capacità nodo di uscita}
\end{array}\]
\[P_{DYN} = {C_{tot} \; V_{DD}}^2 \; f_\text{requenza}\]

\subsubsection*{Impieghi}
Il buffer cmos viene utilizzato per ridurre il tempo di ritardo quando si deve pilotare un carico esterno elevato. Alcuni esempi
di impiego del buffer sono:
\begin{itemize}
	\item pilotaggio di bus di dati o passaggio di segnali per piste lunghe con elevata capacità parassita
	\item pilotaggio di ingressi di circuiti logici con elevata capacità di ingresso
	\item pilotaggio da parte del clock di tutti i registri del computer
\end{itemize}
Siccome il buffer è costituito da più invertitori in cascata, il consumo dinamico è maggiore rispetto ad un singolo invertitore
ed inoltre occupa più area sul chip. Per questo motivo si cerca di utilizzare il buffer solo quando strettamente necessario.

%\section{Logica statica complementare}
\subsection{Reti duali}
\subsubsection*{Rete di pull-up e rete di pull-down}
Il principio alla base della logica statica complementare è quello di basarsi sulla struttura dell'invertitore cmos e
generalizzarla per realizzare qualsiasi funzione logica combinatoria, in modo da sfruttare i vantaggi dell'invertitore:
\begin{itemize}
	\item resistenza di ingresso infinita;
	\item consumo di potenza praticamente nullo in stato stazionario;
	\item elevata immunità al rumore, proprietà rigenerativa del segnale;
	\item la soglia logica e il tempo di propagazione intrinseci dipendono solo da \(Z_p/Z_n\).
\end{itemize}
Per fare ciò, si sfrutta la presenza di due reti complementari: la rete di pull-up (PUN), costituita da mosfet di tipo
p, e la rete di pull-down (PDN), costituita da mosfet di tipo n che si attivano in modo complementare per caricare o scaricare
il nodo di uscita.

\subsubsection*{Serie e paralleli di mosfet}
\begin{center}
	\begin{tabular}{>{\centering\arraybackslash}m{2cm} >{\centering\arraybackslash}m{2.9cm} >{\centering\arraybackslash}m{2.9cm} >{\centering\arraybackslash}m{2.9cm} >{\centering\arraybackslash}m{2.9cm}}
		\textbf{tipo di circuito} & \textbf{serie di nmos} & \textbf{parallelo di nmos} & \textbf{serie di pmos} & \textbf{parallelo di pmos}\\
		\toprule
		\textbf{schema circuitale} & \begin{circuitikz}
			\node[nmos, anchor=D, rotate=-90] (M1) at (0,0) {};
			\node[nmos, anchor=D, rotate=-90] (M2) at (1.2,0) {};
			\draw (M1.D) -- (M2.S);
			\draw (M1.G) node[above] {\(A\)};
			\draw (M2.G) node[above] {\(B\)};
		\end{circuitikz} & \begin{circuitikz}
			\node[nmos, anchor=D, rotate=-90] (M1) at (0,0.2) {};
			\node[nmos, anchor=S, rotate=90] (M2) at (0,-0.2) {};
			\draw (M1.S) -- (M2.D);
			\draw (M1.D) -- (M2.S);
			\draw (M1.D) ++(0,-0.2) -- ++(0.5,0);
			\draw (M1.S) ++(0,-0.2) -- ++(-0.5,0);
			\draw (M1.G) node[above] {\(A\)};
			\draw (M2.G) node[below] {\(B\)};
		\end{circuitikz} & \begin{circuitikz}
			\ctikzset{tripoles/pmos style/emptycircle}
			\node[pmos, anchor=D, rotate=-90] (M1) at (0,0) {};
			\node[pmos, anchor=D, rotate=-90] (M2) at (1.2,0) {};
			\draw (M1.S) -- (M2.D);
			\draw (M1.G) node[above] {\(A\)};
			\draw (M2.G) node[above] {\(B\)};
		\end{circuitikz} &  \begin{circuitikz}
			\ctikzset{tripoles/pmos style/emptycircle}
			\node[pmos, anchor=D, rotate=-90] (M1) at (0,0.2) {};
			\node[pmos, anchor=S, rotate=90] (M2) at (0,-0.2) {};
			\draw (M1.S) -- (M2.D);
			\draw (M1.D) -- (M2.S);
			\draw (M1.D) ++(0,-0.2) -- ++(-0.5,0);
			\draw (M1.S) ++(0,-0.2) -- ++(0.5,0);
			\draw (M1.G) node[above] {\(A\)};
			\draw (M2.G) node[below] {\(B\)};
		\end{circuitikz} \\
		\midrule
		\textbf{funzione logica} & \(\begin{array}{c}
			\text{AND} \\[6pt]
			A \cdot B
		\end{array}\) & \(\begin{array}{c}
			\text{OR} \\[6pt]
			A + B
		\end{array}\) & \(\begin{array}{c}
			\text{NOR} \\[6pt]
			\overline{A + B}
		\end{array}\) & \(\begin{array}{c}
			\text{NAND} \\[6pt]
			\overline{A \cdot B}
		\end{array}\) \\
		\midrule
		\textbf{tabella di verità} & \begin{tabular}{c c | c}
			\(A\) & \(B\) & \(V_{out}\) \\
			\midrule
			0 & 0 & 0 \\
			0 & 1 & 0 \\
			1 & 0 & 0 \\
			1 & 1 & 1 \\
		\end{tabular} &
		\begin{tabular}{c c | c}
			\(A\) & \(B\) & \(V_{out}\) \\
			\midrule
			0 & 0 & 0 \\
			0 & 1 & 1 \\
			1 & 0 & 1 \\
			1 & 1 & 1 \\
		\end{tabular}  &
		\begin{tabular}{c c | c}
			\(A\) & \(B\) & \(V_{out}\) \\
			\midrule
			0 & 0 & 1 \\
			0 & 1 & 0 \\
			1 & 0 & 0 \\
			1 & 1 & 0 \\
		\end{tabular} &
		\begin{tabular}{c c | c}
			\(A\) & \(B\) & \(V_{out}\) \\
			\midrule
			0 & 0 & 1 \\
			0 & 1 & 1 \\
			1 & 0 & 1 \\
			1 & 1 & 0 \\
		\end{tabular}
	\end{tabular}
\end{center}

\subsubsection*{Interruttore duale o complementare}
Due interruttori pilotati dallo stesso segnale sono duali o complementari se fissato un qualunque valore del segnale uno e uno
solo dei due è acceso. Un esempio di interruttori duali sono un nmos e un pmos pilotati dallo stesso segnale, come avviene
in un invertitore cmos.

\subsubsection*{Rete duale}
Una rete duale è l'estensione del concetto di interruttori duali ad una rete di interruttori. Due reti pilotate dagli stessi
ingressi sono duali se e solo se per ogni combinazione di ingressi una e una sola delle due reti è attiva. Il principio delle
reti duali è alla base delle reti di pull-up e pull-down nella logica statica complementare. Da una rete di nmos si può ottenere
la rete duale di pmos sostituendo ogni serie di nmos con un parallelo di pmos e viceversa.

\newpage

\subsection{Implementazione di porte logiche elementari}
\subsubsection*{Principio generale}
Data una funzione logica con \(N\) ingressi \(F(A,B,C, \dots)\), si osserva che:
\begin{itemize}
	\item per ogni combinazione di ingressi per cui \(F(\dots) = 0\), l'uscita della porta logica deve essere \(0V\), ovvero
	la rete di pull-down (PDN) deve essere attiva e collegare l'uscita a massa;
	\item per ogni combinazione di ingressi per cui \(F(\dots) = 1\), l'uscita della porta logica deve essere \(V_{DD}\), ovvero
	la rete di pull-up (PUN) deve essere attiva e collegare l'uscita a \(V_{DD}\).
\end{itemize}

\noindent
Per implementare una porta logica con logica statica complementare si segue il seguente procedimento:
\begin{enumerate}
	\item si costruisce la rete di pull-down (PDN), che implementa la funzione logica \(X = \overline{F}\)
	\item si costruisce la rete di pull-up (PUN), duale della PDN che implementa la funzione logica \(F\)
	\item infine si collega la PUN tra \(V_{DD}\) e l'uscita, e la PDN tra l'uscita e massa
\end{enumerate}

\noindent
La caratteristica statica delle porte logiche realizzate con logica statica complementare è uguale a quella dell'invertitore
cmos. Inoltre vale la proprietà rigenerativa del segnale.

\subsubsection*{Esempio di porte logiche elementari invertenti}
\begin{center}
	\begin{tabular}{>{\centering\arraybackslash}m{2cm} >{\centering\arraybackslash}m{4cm} >{\centering\arraybackslash}m{4cm}}
		\textbf{tipo di circuito} & \textbf{porta NAND} & \textbf{porta NOR}\\
		\toprule
		\textbf{funzione logica} & \(\begin{array}{c}
			F = \overline{A \cdot B} \\[6pt]
			X = A \cdot B
		\end{array}\) & \(\begin{array}{c}
			F = \overline{A + B} \\[6pt]
			X = A + B
		\end{array}\) \\
		\midrule
		\textbf{simbolo logico} & \begin{circuitikz}
			\ctikzset{logic ports=ieee}
			\draw (0,0) node[nand port, anchor=in 1, scale=0.8] (AND1) {};
			\draw (AND1.in 1) node[left] {\(A\)};
			\draw (AND1.in 2) node[left] {\(B\)};
			\draw (AND1.out) node[right] {\(V_{out}\)};
		\end{circuitikz} & \begin{circuitikz}
			\ctikzset{logic ports=ieee}
			\draw (0,0) node[nor port, anchor=in 1, scale=0.8] (OR1) {};
			\draw (OR1.in 1) node[left] {\(A\)};
			\draw (OR1.in 2) node[left] {\(B\)};
			\draw (OR1.out) node[right] {\(V_{out}\)};
		\end{circuitikz} \\
		\midrule
		\textbf{rete di pull-up} & \begin{circuitikz}
			\ctikzset{tripoles/pmos style/emptycircle}
			\node[pmos, anchor=D] (M1) at (-0.2,0) {};
			\node[pmos, anchor=S, rotate=180] (M2) at (0.2,0) {};
			\draw (M1.S) -- (M2.D);
			\draw (M1.D) -- (M2.S);
			\draw (M1.D) ++(0.2,0) -- ++(0,-0.5);
			\draw (M1.S) ++(0.2,0) -- ++(0,0.5);
			\draw (M1.G) node[left] {\(A\)};
			\draw (M2.G) node[right] {\(B\)};
		\end{circuitikz} & \begin{circuitikz}
			\ctikzset{tripoles/pmos style/emptycircle}
			\node[pmos, anchor=D] (M1) at (0,0) {};
			\node[pmos, anchor=D] (M2) at (0,1.1) {};
			\draw (M1.S) -- (M2.D);
			\draw (M2.G) node[left] {\(A\)};
			\draw (M1.G) node[left] {\(B\)};
		\end{circuitikz} \\
		\midrule
		\textbf{rete di pull-down} & \begin{circuitikz}
			\node[nmos, anchor=D] (M1) at (0,0) {};
			\node[nmos, anchor=D] (M2) at (0,1.1) {};
			\draw (M1.D) -- (M2.S);
			\draw (M2.G) node[left] {\(A\)};
			\draw (M1.G) node[left] {\(B\)};
		\end{circuitikz} & \begin{circuitikz}
			\node[nmos, anchor=D] (M1) at (-0.2,0) {};
			\node[nmos, anchor=S, rotate=180] (M2) at (0.2,0) {};
			\draw (M1.S) -- (M2.D);
			\draw (M1.D) -- (M2.S);
			\draw (M1.D) ++(0.2,0) -- ++(0,0.5);
			\draw (M1.S) ++(0.2,0) -- ++(0,-0.5);
			\draw (M1.G) node[left] {\(A\)};
			\draw (M2.G) node[right] {\(B\)};
		\end{circuitikz} \\
		\midrule
		\textbf{porta logica completa} & \begin{circuitikz}
			\ctikzset{tripoles/pmos style/emptycircle}
			\node[pmos, anchor=D] (M1) at (-0.2,0) {};
			\node[pmos, anchor=S, rotate=180] (M2) at (0.2,0) {};
			\draw (M1.S) -- (M2.D);
			\draw (M1.D) -- (M2.S);
			\draw (M1.D) ++(0.2,0) node[](PUN){};
			\draw (M1.S) ++(0.2,0) node[rground, rotate=180](VDD){};
			\draw (M1.D) ++(0.2,-0.25) -- ++(1,0) node[right]{\(F\)};
			\draw (M1.G) node[left] {\(A\)};
			\draw (M2.G) node[right] {\(B\)};
			\node at ($(VDD)+(0, 0.6)$) {\(V_{DD}\)};
			\node[nmos, anchor=D] (M1) at (PUN) {};
			\node[nmos, anchor=D] (M2) at ($(M1.S)+(0,0.5)$) {};
			\draw (M1.G) node[left] {\(A\)};
			\draw (M2.G) node[left] {\(B\)};
			\draw (M2.S) ++(0,0.2) -- ++(0,0.1) node[sground]{};
		\end{circuitikz} & \begin{circuitikz}
			\ctikzset{tripoles/pmos style/emptycircle}
			\ctikzset{tripoles/pmos style/emptycircle}
			\node[nmos, anchor=D] (M1) at (-0.2,0) {};
			\node[nmos, anchor=S, rotate=180] (M2) at (0.2,0) {};
			\draw (M1.S) -- (M2.D);
			\draw (M1.D) -- (M2.S);
			\draw (M1.D) ++(0.2,0) node[](PDN){};
			\draw (M1.S) ++(0.2,0) node[sground]{};
			\draw (M1.D) ++(0.2,0.25) -- ++(1,0) node[right]{\(F\)};
			\draw (M1.G) node[left] {\(A\)};
			\draw (M2.G) node[right] {\(B\)};
			\node[pmos, anchor=D] (M1) at (PDN) {};
			\node[pmos, anchor=D] (M2) at ($(M1.S)+(0,-0.5)$) {};
			\draw (M1.G) node[left] {\(A\)};
			\draw (M2.G) node[left] {\(B\)};
			\draw (M2.S) ++(0,-0.3) node[rground, rotate=180](VDD){};
			\node at ($(VDD)+(0, 0.6)$) {\(V_{DD}\)};
		\end{circuitikz}
	\end{tabular}
\end{center}

\newpage

\subsection{Funzioni invertenti e porte logiche non invertenti}
\subsubsection*{Funzioni invertenti e non invertenti}
\begin{itemize}
	\item Una funzione logica \(F(X_1,X_2, \dots X_N)\) è invertente se facendo variare un ingresso \(X_i\) da \(0\) a \(1\),
	dopo aver fissato gli altri ingressi, l'uscita \(F\) commuta nel verso opposto di \(X_i\) o rimane costante. Un esempio
	di funzioni invertenti sono le porte NAND e NOR.
	\item Una funzione logica \(F(X_1,X_2, \dots X_N)\) è non invertente se facendo variare un ingresso \(X_i\) da \(0\) a \(1\),
	dopo aver fissato gli altri ingressi, l'uscita \(F\) commuta nello stesso verso di \(X_i\) o rimane costante. Un esempio
	di funzioni non invertenti sono le porte AND e OR.
	\item Una funzione può non essere né invertente né non invertente, ad esempio la funzione XOR.
\end{itemize}

\subsubsection*{Limiti della logica statica complementare}
Con la logica statica complementare è possibile implementare direttamente solo funzioni invertenti, come NAND e NOR (illustrate
sopra). Per implementare funzioni non invertenti, come AND e OR, è necessario aggiungere un ulteriore stadio di inversione
all'uscita oppure agli ingressi della porta logica invertente.

\subsubsection*{Esempi di porte logiche non invertenti}
\begin{itemize}
	\item AND non invertente: \(F = A \cdot B = \overline{\overline{A} + \overline{B}} = \overline{\overline{A \cdot B}}\)
	è implementabile come una porta NOR con ingressi invertiti, oppure come una porta NAND seguita da un invertitore.
	\item OR non invertente: \(F = A + B = \overline{\overline{A} \cdot \overline{B}} = \overline{\overline{A + B}}\)
	è implementabile come una porta NAND con ingressi invertiti, oppure come una porta NOR seguita da un invertitore.
\end{itemize}


\subsection{Tempi di ritardo}
\subsubsection*{Ricerca del percorso peggiore e modello di Elmore}
L'analisi dei tempi di ritardo di una porta logica a più ingressi consiste nell'analizzare la carica e scarica del nodo di uscita
attraverso i percorsi di carica (nella PUN) e scarica (nella PDN) con resistenza equivalente massima, ovvero lungo il percorso peggiore.
In genere il percorso peggiore è quello con maggior numero di mosfet in serie e minor numero di mosfet in parallelo.

Una volta individuato il percorso peggiore, si modella il circuito come una rete di Elmore, in cui ogni mosfet viene rappresentato
come una resistenza (\(R_n\) o \(R_p\)) collegata a massa attraverso una capacità complessiva del nodo interno. Una volta tracciata
la rete di Elmore, è possibile calcolare il tempo di ritardo \(t_{pHL}\) usando la rete di scarica (PDN) e il tempo di ritardo \(t_{pLH}\)
usando la rete di carica (PUN), secondo le formule del modello di Elmore.

Di seguito l'esempio dell'individuazione dei percorsi peggiori nella PUN e PDN per il calcolo dei due tempi di ritardo \(t_{pLH}\)
e \(t_{pHL}\) nel caso di una nand a 3 ingressi.
\begin{center}
	\begin{minipage}{0.52\textwidth}
		\centering \includegraphics[width=\textwidth]{immagini/8_logica_statica_complementare/tp_1.png}

		\small{modellizzazione con resistenze e capacità}
	\end{minipage}
	\begin{minipage}{0.23\textwidth}
		\centering \includegraphics[width=\textwidth]{immagini/8_logica_statica_complementare/tp_2.png}

		\small{percorso peggiore della PUN per \(t_{pLH}\)}
	\end{minipage}
	\begin{minipage}{0.23\textwidth}
		\centering \includegraphics[width=\textwidth]{immagini/8_logica_statica_complementare/tp_3.png}

		\small{percorso peggiore della PDN per \(t_{pHL}\)}
	\end{minipage}
\end{center}
Per il caso sopra, i tempi di ritardo risultano:
\[t_{pLH} = 0.69 \, R_p \, (C_{OUT} + C_F) \qquad t_{pHL} = 0.69 \, (R_n C_1 + 2R_n C_2 + 3R_n(C_{OUT} + C_L))\]
\[\text{con} \quad C_1 = C_{dn,C} + C_{sn,B}, \quad C_2 = C_{dn,B} + C_{sn,A} \quad C_{OUT} = C_{dn,a} + C_{dp,A} + C_{dp,B} + C_{dp,C}\]

\subsubsection*{Tempo di ritardo medio}
Il tempo di ritardo medio \(t_p\) di una porta logica in statica complementare risulta:
\[t_{p0} = \frac{t_{pHL0} + t_{pLH0}}{2} = 0.69 \frac{R_n(C_1 + 2C_2 + 3C_{OUT}) + R_p C_{OUT}}{2}\]
\[t_p = t_{p0} + 0.69 \frac{3R_n + R_p}{2} C_L = 0.69 \, R_{OUT} \, C_L \qquad R_{OUT} = \frac{R_{PUN} + R_{PDN}}{2}\]

\subsubsection*{Dipendenza dal numero di ingressi}
Dalle formule del modello di Elmore per una porta logica a \(N\) ingressi (\(N\) mosfet in parallelo e altretttanti in serie),
il tempo di ritardo intrinseco dipende quadraticamente dal numero di ingressi \(N\). Questo effetto non è desiderabile, in quanto
porta ad un aumento significativo del tempo di ritardo al crescere del numero di ingressi della porta logica.
\begin{align*}
	t_{p0} &= 0.69\frac{R_n(C_1 + C_2 + \dots + (N-1)C_{N-1} + NC_{OUT}) + R_p C_{OUT}}{2} \qquad \text{con} \; C_{OUT} = C_{dn} + N C_{dp} \\
	&= 0.69\frac{R_n(C_1 + C_2 + \dots + (N-1)C_{N-1} + N(C_{dn} + NC_{dp})) + R_p (C_{dn} + NC_{dp})}{2} \\
	&= 0.69\frac{R_n(C_1 + C_2 + \dots + (N-1)C_{N-1} + NC_{dn} + \color{red}{\mathbf{N^2C_{dp}}}\color{black}) + R_p (C_{dn} + NC_{dp})}{2}
\end{align*}
Il contributo del carico esterno, invece, cresce linearmente con \(N\):
\[t_p = t_{p0} + 0.69 \frac{\color{red}{NR_n}\color{black} + R_p}{2} C_L\]

\begin{center}
	\begin{minipage}{0.5\textwidth}
		\centering \includegraphics[width=0.9\textwidth]{immagini/8_logica_statica_complementare/tp_4.png}
	\end{minipage}
	\begin{minipage}{0.48\textwidth}
		Analizzando il grafico del tempo di ritardo in funzione del numero di ingressi per una porta logica NAND, \(t_p\) cresce
		parabolicamente con \(N\).
	\end{minipage}
\end{center}

\subsubsection*{Bilanciamento PUN e PDN}
Per uniformare il caso peggiore tra \(t_{pLH}\) e \(t_{pHL}\), si può agire sul rapporto \(Z_p/Z_n\) dei mosfet, per bilanciare
le resistenze equivalenti della PUN e PDN al caso peggiore.
\[R_{PU} = \alpha \frac{R_{p0}}{Z_p} \qquad R_{PD} = \beta \frac{R_{n0}}{Z_n} \qquad R_{PU} = R_{PD} \;\; \rightarrow \;\; Z_{p} = \frac{\alpha R_{p0}}{\beta R_{n0}} Z_n\]
Si osserva che per ottimizzare il tempo di ritardo è ulteriormente possibile avere \(Z_p\) e \(Z_n\) minori per i mosfet che sono
collegati direttamente al nodo di uscita (coinvolti nel termine quadratico) e maggiori per i mosfet più lontani dall'uscita, senza
però alterare il bilanciamento tra PUN e PDN.

\subsubsection*{Tempo di ritardo di più porte logiche in cascata}
Quando si collegano in cascata più porte logiche (ad esempio per realizzare funzioni non invertenti come AND = NAND + NOT e
OR = NOR + NOT), il tempo di ritardo complessivo è dato dalla somma dei tempi di ritardo delle singole porte logiche.
In particolare il pedice del tempo di ritardo \(t_{pHL}\) o \(t_{pLH}\) complessivo è determinato dall'uscita dell'ultima porta
logica della catena. Ad esempio per una AND realizzata come NAND + NOT, il tempo di ritardo complessivo risulta:
\[t_{pHL,A\!N\!D} = t_{pLH,N\!A\!N\!D} + t_{pHL,N\!OT} \qquad t_{pLH,A\!N\!D} = t_{pHL,N\!A\!N\!D} + t_{pLH,N\!OT}\]

\subsection{Problema dei nodi interni}
\subsubsection*{Problema dei nodi interni}
Il problema dei nodi interni si verifica quando, a causa della presenza di nodi interni tra mosfet in serie, il tempo di ritardo
aumenta in modo significativo siccome, insieme alla capacità di uscita, devono essere caricate/scaricate anche tutte le capacità
parassite dei nodi interni. Il numero di nodi interni è legato al numero di ingressi della porta logica, e quindi il problema
si aggrava al crescere del numero di ingressi.

\subsubsection*{Soluzione 1: riordino degli ingressi}
Si suppone che gli \(N\) ingressi di una porta logica provengano da altre reti combinatorie con tempi di ritardo diversi.
Per cercare di contenere il problema dei nodi interni, si può riordinare gli ingressi in modo che la rete combinatoria con
tempo di ritardo minore venga collegato al mosfet più vicino a massa (o a \(V_{DD}\) nella PUN) e la rete con tempo di ritardo
maggiore venga collegata al mosfet più lontano dall'uscita. In questo modo, quando la rete più lenta commuta, le reti più veloci
hanno già precaricato/scaricato i nodi interni e l'unica capacità da caricare/scaricare è quella di uscita.

\subsubsection*{Soluzione 2: suddivisione in più stadi}
Quando il numero di ingressi resta comunque elevato, è possibile risolvere la questione dei nodi interni suddividendo la porta
logica in più stadi di porte logiche con meno ingressi ciascuna. In questo modo si riduce il numero di nodi interni per ogni
stadio, e si riduce il tempo di ritardo complessivo. Si osserva che il valore ottimo di ingressi per stadio si aggira tra 4 e 5.
Ovviamente suddividendo il numero di ingressi in più stadi si aumenta il numero totale di mosfet necessari per realizzare la
stessa funzione logica e quindi l'area occupata sul chip.

Per suddividere una porta logica in più stadi si possono utilizzare le leggi di De Morgan:
\[A + B + C + D \;_\text{(OR a 4 ingressi)} = \overline{\overline{A + B} \cdot \overline{C + D}} \;_\text{(NOR + NAND a 2 ingressi)}\]
\[A \cdot B \cdot C \cdot D \;_\text{(AND a 4 ingressi)} = \overline{\overline{A \cdot B} + \overline{C \cdot D}} \;_\text{(NAND + NOR a 2 ingressi)}\]

Di seguito un confronto delle prestazioni di una porta AND a 16 ingressi realizzata con numero differente di stadi:
\begin{center}
	\begin{tabular}{c c c}
		\textbf{numero di stadi} & \textbf{tempo di ritardo intrinseco} & \textbf{numero di mosfet} \\
		\toprule
		1 stadio da 16 ingressi & \(t_{p0} \approx 0.69 R C N^2 = 0.69 \, 256 \, RC\) & 32 mosfet \\
		\midrule
		2 stadi da 4 ingressi ciascuno & \(t_{p0} \approx 2 \cdot 0.69 R C N^2 = 0.69\, 32 \, RC\) & 40 mosfet \\
		\midrule
		4 stadi da 2 ingressi ciascuno & \(t_{p0} \approx 4 \cdot 0.69 R C N^2 = 0.69 \, 16 \, RC\) & 60 mosfet
	\end{tabular}
\end{center}
\begin{center}
	\begin{minipage}{0.38\textwidth}
		\centering \includegraphics[width=\textwidth]{immagini/8_logica_statica_complementare/and_2stadi.png}

		\small{AND a 16 ingressi in 2 stadi}
	\end{minipage}
	\begin{minipage}{0.6\textwidth}
		\centering \includegraphics[width=\textwidth]{immagini/8_logica_statica_complementare/and_4stadi.png}

		\small{AND a 16 ingressi in 4 stadi}
	\end{minipage}
\end{center}

\subsection{Consumo di potenza}
\subsubsection*{Consumo dinamico}
Il consumo dinamico di una porta logica si basa sulla formula generale definita in precedenza, solo che essendoci più ingressi
la frequenza media di commutazione \(f\) viene sostituita dalla frequenza effettiva di commutazione dell'uscita calcolata come
prodotto tra la frequenza di commutazione degli ingressi \(f\) e il fattore di attività \(\alpha_F\) dell'uscita.
\[P_{DYN} = C_F \cdot V_{DD} (V_H - V_L) \cdot f \cdot \alpha_F \qquad f \cdot \alpha_F = \text{frequenza effettiva di commutazione dell'uscita}\]

\subsubsection*{Fattore di attività}
Il fattore di attività \(\alpha_F\) di una porta logica è definito come la probabilità che l'uscita della porta logica commuti
da \(0\) a \(1\) in un dato intervallo di tempo. Definito \(p_F\) la probabilità che l'uscita della porta logica sia \(1\) e
supponendo che l'uscita di una porta logica non dipenda dal suo stato precedente (tempo invariante), il fattore di attività risulta:
\[\alpha_F = P(0 \rightarrow 1) = P(F = 0 {\text{ per } t = i-1}) \cdot P(F = 1 {\text{ per } t = i}) = p_F \cdot (1 - p_F)\]

\noindent
La probabilità \(p_F\) che l'uscita della porta logica sia a \(1\) dipende dalla funzione logica implementata e dalle probabilità
degli ingressi \(p_{X_i}\) (che si suppone siano tutte indipendenti e tempo invarianti). Per calcolare \(p_F\) si può utilizzare:
\begin{itemize}
	\item la tabella di verità della porta logica, calcolando la somma delle probabilità delle combinazioni di ingressi che portano
	all'uscita a \(1\)
	\item l'espressione logica della funzione \(F\), convertendo le operazioni logiche tra ingressi in operazioni algebriche sulle
	probabilità degli ingressi secondo le seguenti regole e facendo attenzione ad eliminare le intersezioni (sulla OR):
	\[\text{NOT} \; \rightarrow \; p_{\overline{X}} = 1 - p_X \qquad \text{AND} \; \rightarrow \;  p_{A \cdot B} = p_A \cdot p_B \qquad \text{OR} \; \rightarrow \; p_{A + B} = p_A + p_B - p_A \cdot p_B\]
	\item le mappe di Karnaugh per semplificare l'espressione logica in somme esclusive di prodotti, in modo da evitare le intersezioni
	e facilitare il calcolo delle probabilità
\end{itemize}

\noindent
Di seguito le probabilità \(p_F\) e i fattori di attività \(\alpha_F\) per le porte logiche elementari a 2 ingressi:
\begin{center}
	\begin{tabular}{>{\centering\arraybackslash}m{3cm} l m{6.5cm}}
		\toprule
		\textbf{Funzione logica} & \textbf{\(P_F\)} & \textbf{Fattore di attività \(\alpha_F\)} \\
		\midrule
		AND & \(p_A p_B\) & \(p_A p_B \cdot (1 - p_A p_B)\) \\[6pt]
		NAND & \(1 - p_A p_B\) & \(p_A p_B \cdot (1 - p_A p_B)\) \\[6pt]
		OR & \(p_A + (1 - p_A)p_B\) & \((p_A + (1 - p_A)p_B) \cdot (1 - (p_A + (1 - p_A)p_B))\) \\[6pt]
		NOR & \((1 - p_A)(1 - p_B)\) & \((p_A + (1 - p_A)p_B) \cdot (1 - (p_A + (1 - p_A)p_B))\) \\[6pt]
		XOR & \((p_A + p_B - 2 p_A p_B)\) & \((p_A + p_B - 2 p_A p_B) \cdot (1 - p_A - p_B + 2 p_A p_B)\) \\[6pt]
		XNOR & \(1 - p_A - p_B + 2 p_A p_B\) & \((p_A + p_B - 2 p_A p_B) \cdot (1 - p_A - p_B + 2 p_A p_B)\) \\
		\bottomrule
	\end{tabular}
\end{center}

\subsubsection*{Capacità logica}
Si nota che nella formula del consumo dinamico compare la capacità \(C_F\) che rappresenta la capacità totale del nodo di uscita
della porta logica. È possibile definire la capacità logica come il prodotto tra la capacità \(C_F\) e il fattore di attività
\(\alpha_F\).
\[C_\text{logica} = C_F \cdot \alpha_F\]

\noindent
In questo modo la formula del consumo dinamico può essere interpretata come il consumo dinamico per caricare e scaricare ad
una frequenza \(f\) una capacità effettiva pari alla capacità logica.
\[P_{DYN} = C_\text{logica} \cdot V_{DD} (V_H - V_L) \cdot f\]

%\section{Logica a Pass Transistor}
\subsection{Struttura base}
\subsubsection*{Principio di funzionamento}
I pass transistor sono transistor usati come interruttori, controllati da un segnale di controllo (gate) che permettono
o bloccano il passaggio di un certo segnale in ingresso verso l'uscita. A differenza delle porte logiche complementari,
i pass transistor non sono collegati necessariamente a \(V_{DD}\) o a massa, ma possono trasmettere direttamente i segnali
logici di ingresso. Inoltre le porte logiche a pass transistor possono essere costruite sia con nmos che con pmos e non
necessitano di reti complementari.

\subsubsection*{Multiplexer}
Il multiplexer è un circuito che seleziona uno tra più segnali di ingresso in base al valore di uno o più segnali di selezione.
La configurazione base di un multiplexer prevede due ingressi e un segnale di selezione. Sono facilmente realizzabili con pass
transistor.
\begin{center}
	\begin{minipage}{0.2\textwidth}
		\centering \begin{circuitikz}
			\draw (0, 0) node[muxdemux, muxdemux def={w=1, Lh=2, Rh=1.4, NL=2, NR=1, NB=0, NT=1}, anchor=center] (MUX) {};
			\draw (MUX.lpin 1) node[left] {$A_0$}; \draw (MUX.lpin 2) node[left] {$A_1$};
			\draw (MUX.tpin 1) node[above] {$S$}; \draw (MUX.rpin 1) node[right] {$F$};
		\end{circuitikz}
	\end{minipage}
	\begin{minipage}{0.38\textwidth}
		\centering \begin{circuitikz}
			\node[nmos, anchor=D, rotate=-90] (M1) at (0,0) {};
			\node[nmos, anchor=D, rotate=-90] (M2) at (0.3,-1.6) {};
			\draw (M1.D) -- ++(0.3,0) -- (M2.D);
			\node[left] at (M1.S) {\(A\)};
			\node[above] at (M1.G) {\(S\)};
			\draw (M2.S) -- ++(-0.3,0) node[left] {\(B\)};
			\node[above] at (M2.G) {\(\overline{S}\)};
			\draw (M1.D) ++(0.3,-0.8) -- ++(0.5,0) node[right] {\(F = S \ccdot A + \overline{S} \ccdot B\)};
		\end{circuitikz}
	\end{minipage}
	\begin{minipage}{0.38\textwidth}
		\centering \begin{circuitikz}
			\ctikzset{tripoles/pmos style/emptycircle}
			\node[pmos, anchor=D, rotate=90, xscale=-1] (M1) at (0,0) {};
			\node[pmos, anchor=D, rotate=90, xscale=-1] (M2) at (0.3,-1.6) {};
			\draw (M1.D) -- ++(0.3,0) -- (M2.D);
			\node[left] at (M1.S) {\(A\)};
			\node[above] at (M1.G) {\(S\)};
			\draw (M2.S) -- ++(-0.3,0) node[left] {\(B\)};
			\node[above] at (M2.G) {\(\overline{S}\)};
			\draw (M1.D) ++(0.3,-0.8) -- ++(0.5,0) node[right] {\(F = \overline{S} \ccdot A + S \ccdot B\)};
		\end{circuitikz}
	\end{minipage}

	\begin{minipage}{0.2\textwidth}
		\centering \small{simbolo logico del MUX 2:1}
	\end{minipage}
	\begin{minipage}{0.38\textwidth}
		\centering \small{MUX 2:1 a pass transistor con nmos}
	\end{minipage}
	\begin{minipage}{0.38\textwidth}
		\centering \small{MUX 2:1 a pass transistor con pmos}
	\end{minipage}
\end{center}

\subsubsection*{Porte elementari (AND, OR, XOR, NAND, NOR, XNOR) a pass transistor}
\begin{center}
	\begin{minipage}{0.32\textwidth}
		\centering \begin{circuitikz}
			\node[nmos, anchor=D, rotate=-90] (M1) at (0,0) {};
			\node[nmos, anchor=D, rotate=-90] (M2) at (0.3,-1.6) {};
			\draw (M1.D) -- ++(0.3,0) -- (M2.D);
			\node[left] at (M1.S) {\(B\)};
			\node[above] at (M1.G) {\(A\)};
			\draw (M2.S) -- ++(-0.3,0) node[left] {\(0\)};
			\node[above] at (M2.G) {\(\overline{A}\)};
			\draw (M1.D) ++(0.3,-0.8) -- ++(0.5,0) node[right] {\(F = A \cdot B\)};
		\end{circuitikz}

		\small{AND a pass-T con nmos}
	\end{minipage}
	\begin{minipage}{0.32\textwidth}
		\centering \begin{circuitikz}
			\node[nmos, anchor=D, rotate=-90] (M1) at (0,0) {};
			\node[nmos, anchor=D, rotate=-90] (M2) at (0.3,-1.6) {};
			\draw (M1.D) -- ++(0.3,0) -- (M2.D);
			\node[left] at (M1.S) {\(1\)};
			\node[above] at (M1.G) {\(A\)};
			\draw (M2.S) -- ++(-0.3,0) node[left] {\(B\)};
			\node[above] at (M2.G) {\(\overline{A}\)};
			\draw (M1.D) ++(0.3,-0.8) -- ++(0.5,0) node[right] {\(F = A + B\)};
		\end{circuitikz}

		\small{OR a pass-T con nmos}
	\end{minipage}
	\begin{minipage}{0.32\textwidth}
		\centering \begin{circuitikz}
			\node[nmos, anchor=D, rotate=-90] (M1) at (0,0) {};
			\node[nmos, anchor=D, rotate=-90] (M2) at (0.3,-1.6) {};
			\draw (M1.D) -- ++(0.3,0) -- (M2.D);
			\node[left] at (M1.S) {\(\overline{B}\)};
			\node[above] at (M1.G) {\(A\)};
			\draw (M2.S) -- ++(-0.3,0) node[left] {\(B\)};
			\node[above] at (M2.G) {\(\overline{A}\)};
			\draw (M1.D) ++(0.3,-0.8) -- ++(0.5,0) node[right] {\(F = A \oplus B\)};
		\end{circuitikz}

		\small{XOR a pass-T con nmos}
	\end{minipage}
\end{center}
\begin{center}
	\begin{minipage}{0.32\textwidth}
		\centering \begin{circuitikz}
			\node[nmos, anchor=D, rotate=-90] (M1) at (0,0) {};
			\node[nmos, anchor=D, rotate=-90] (M2) at (0.3,-1.6) {};
			\draw (M1.D) -- ++(0.3,0) -- (M2.D);
			\node[left] at (M1.S) {\(\overline{B}\)};
			\node[above] at (M1.G) {\(A\)};
			\draw (M2.S) -- ++(-0.3,0) node[left] {\(1\)};
			\node[above] at (M2.G) {\(\overline{A}\)};
			\draw (M1.D) ++(0.3,-0.8) -- ++(0.5,0) node[right] {\(F = \overline{A \cdot B}\)};
		\end{circuitikz}

		\small{NAND a pass-T con nmos}
	\end{minipage}
	\begin{minipage}{0.32\textwidth}
		\centering \begin{circuitikz}
			\node[nmos, anchor=D, rotate=-90] (M1) at (0,0) {};
			\node[nmos, anchor=D, rotate=-90] (M2) at (0.3,-1.6) {};
			\draw (M1.D) -- ++(0.3,0) -- (M2.D);
			\node[left] at (M1.S) {\(0\)};
			\node[above] at (M1.G) {\(A\)};
			\draw (M2.S) -- ++(-0.3,0) node[left] {\(\overline{B}\)};
			\node[above] at (M2.G) {\(\overline{A}\)};
			\draw (M1.D) ++(0.3,-0.8) -- ++(0.5,0) node[right] {\(F = \overline{A + B}\)};
		\end{circuitikz}

		\small{NOR a pass-T con nmos}
	\end{minipage}
	\begin{minipage}{0.32\textwidth}
		\centering \begin{circuitikz}
			\node[nmos, anchor=D, rotate=-90] (M1) at (0,0) {};
			\node[nmos, anchor=D, rotate=-90] (M2) at (0.3,-1.6) {};
			\draw (M1.D) -- ++(0.3,0) -- (M2.D);
			\node[left] at (M1.S) {\(B\)};
			\node[above] at (M1.G) {\(A\)};
			\draw (M2.S) -- ++(-0.3,0) node[left] {\(\overline{B}\)};
			\node[above] at (M2.G) {\(\overline{A}\)};
			\draw (M1.D) ++(0.3,-0.8) -- ++(0.5,0) node[right] {\(F = \overline{A \oplus B}\)};
		\end{circuitikz}

		\small{XNOR a pass-T con nmos}
	\end{minipage}
\end{center}

\subsubsection*{Teorema di Shannon e implementazione di funzioni complesse}
Ogni funzione logica di \(N\) variabili \(F(A,B,C \dots)\) può essere espressa nel seguente modo:
\[F = A \cdot F(1,B,C \dots) + \overline{A} \cdot F(0,B,C \dots) = A \cdot F_1(B,C \dots) + \overline{A} \cdot F_0(B,C \dots)\]
Questa espressione permette di implementare qualsiasi funzione logica usando multiplexer a pass transistor, dove la
variabile \(A\) viene usata come segnale di selezione e le uscite delle due funzioni \(F(0,B,C \dots)\) e \(F(1,B,C \dots)\)
vengono collegate agli ingressi del multiplexer. Il teorema può essere applicato in modo ricorsivo per ridurre ulteriormente
le funzioni \(F_0\) e \(F_1\) fino ad ottenere solo porte logiche elementari.

\subsection{Trasmissione dei valori logici cattivi}
\subsubsection*{Trasmissione del valore logico alto con nmos e basso con pmos}
La trasmissione di un valore logico alto tramite un pass transistor nmos avviene correttamente solo fino a \(V_{DD} - V_{TN}\).
Quando la tensione in ingresso raggiunge questo valore, la tensione tra source e gate uguaglia la tensione di soglia
\(V_G - V_S = V_{DD} - (V_{DD} - V_{TN}) = V_{TN}\) e il transistor entra in interdizione.

In analogo al valore alto per nmos, la trasmissione di un valore logico basso tramite un pass transistor pmos avviene
correttamente solo fino a \(-V_{TP}\). Quando la tensione in ingresso raggiunge questo valore, la tensione tra source e gate
uguaglia la tensione di soglia \(V_S - V_G = 0 - (-V_{TP}) = V_{TP}\) e il transistor entra in interdizione.
\begin{center}
	\begin{minipage}{0.3\textwidth}
		\centering \includegraphics[width=0.95\textwidth]{immagini/9_pass_transistor/tx_nmos.png}

		\small{caratteristiche di trasmissione di un pass transistor nmos}
	\end{minipage}
	\begin{minipage}{0.15\textwidth} \(\) \end{minipage}
	\begin{minipage}{0.3\textwidth}
		\centering \includegraphics[width=0.9\textwidth]{immagini/9_pass_transistor/tx_pmos.png}

		\small{caratteristiche di trasmissione di un pass transistor pmos}
	\end{minipage}
\end{center}

\noindent
Dai grafici si osserva che i pass transistor hanno guadagno statico unitario (la pendenza della caratteristica è 1), per cui
non hanno proprietà rigenerativa del segnale e il rumore viene trasmesso senza attenuazione. Inoltre l'effetto body peggiora
ulteriormente la trasmissione dei valori logici \say{cattivi}, aumentando in modulo la tensione di soglia \(V_{TN}\) e \(V_{TP}\).

\subsubsection*{Propagazione ed effetti dei valori logici cattivi per pass transistor nmos}
L'uscita di un pass trasistor nmos che trasmette un valore logico alto risulta degradata a \(V_{DD} - V_{TN}\) può essere
collegata in tre modi principali:
\begin{itemize}
	\item al source/drain di un altro nmos: la trasmissione del valore logico alto degradato dal primo nmos non viene
	ulteriormente peggiorata dal secondo nmos, in quanto il secondo riesce a trasmettere correttamente il valore logico
	degradato \(V_{DD} - V_{TN}\)
	\item al gate di un altro nmos: in questo caso il valore logico alto \(V_{DD} - V_{TN}\) limita la tensione di accensione
	del secondo nmos, che a sua volta peggiorerà ulteriormente il valore logico alto in uscita a \(V_{DD} - 2V_{TN}\)
	\item all'ingresso di una porta logica complementare: in questo caso il valore logico alto degradato \(V_{DD} - V_{TN}\)
	può causare la parziale accensione della pull-up-network, causando un aumento del consumo statico della porta logica
\end{itemize}

\begin{center}
	\begin{minipage}{0.25\textwidth}
		\centering \includegraphics[width=0.9\textwidth]{immagini/9_pass_transistor/tx_serie.png}
	\end{minipage}
	\begin{minipage}{0.01\textwidth} \(\) \end{minipage}
	\begin{minipage}{0.25\textwidth}
		\centering \includegraphics[width=0.9\textwidth]{immagini/9_pass_transistor/tx_gate.png}
	\end{minipage}
	\begin{minipage}{0.01\textwidth} \(\) \end{minipage}
	\begin{minipage}{0.35\textwidth}
		\centering \includegraphics[width=0.9\textwidth]{immagini/9_pass_transistor/tx_invertitore.png}
	\end{minipage}

	\begin{minipage}{0.25\textwidth}
		\centering \small{trasmissione del valore degradato attraverso due pass transistor nmos in serie}
	\end{minipage}
	\begin{minipage}{0.01\textwidth} \(\) \end{minipage}
	\begin{minipage}{0.25\textwidth}
		\centering \small{trasmissione del valore degradato verso il gate di un pass transistor nmos}
	\end{minipage}
	\begin{minipage}{0.01\textwidth} \(\) \end{minipage}
	\begin{minipage}{0.35\textwidth}
		\centering \small{trasmissione del valore degradato verso una porta logica complementare (invertitore)}
	\end{minipage}
\end{center}

\newpage

\subsection{Ottimizzazione dei valori logici - level restorer e transmission gate}
\subsubsection*{Level Restorer}
Per ripristinare i valori logici degradati in uscita da un pass transistor nmos o pmos, si può usare un circuito chiamato level
restorer, costituito da un invertitore e un transistor di pull-up o pull-down (in base al valore logico da ripristinare).
Analizzando il funzionamento di un level restorer usato per ripristinare il valore logico alto in uscita:
\begin{itemize}
	\item quando il segnale in ingresso al level restorer è basso, l'uscita dell'invertitore è alta e il transistor di pull-up
	(pmos) è spento, per cui il circuito di level restorer non entra in funzione
	\item quando invece il segnale in ingresso è alto, l'uscita dell'invertitore è bassa e il transistor di pull-up è acceso,
	forzando l'uscita della rete pass transistor \(F\) al valore logico alto \(V_{DD}\).
\end{itemize}

\noindent
Quando l'uscita del nodo \(F\) inizialmente a \(V_{DD}\) (con il transistor di pull-up è acceso), deve essere portata al valore
logico basso, è necessario che raggiunga la tensione \(V_M\) in modo da far commutare l'invertitore e spegnere il pmos, altrimenti
questo continuerà a forzare l'uscita al valore logico alto. Per fare in modo che ciò avvenga, è necessario che il pmos abbia un
fattore di forma \(Z_p\) inferiore al fattore di forma \(Z_{n,eq}\) della rete pass transistor usata per generare il segnale \(F\).
Se questo non avviene, il transistor di pull-up potrebbe causare un ritardo nella discesa del segnale in uscita o addirittura
la mancata commutazione del segnale.

\begin{center}
	\begin{minipage}{0.35\textwidth}
		\centering \includegraphics[width=0.85\textwidth]{immagini/9_pass_transistor/level_restorer_circuito.png}

		\small{circuito di un level restorer per pass transistor nmos}
	\end{minipage}
	\begin{minipage}{0.1\textwidth} \(\) \end{minipage}
	\begin{minipage}{0.35\textwidth}
		\vspace{0.65cm}

		\centering \includegraphics[width=0.85\textwidth]{immagini/9_pass_transistor/level_restorer_curva.png}

		\small{curva della commutazione HL del nodo \(F\) al variare di \(Z_p\)}
	\end{minipage}
\end{center}

\subsubsection*{Porta di trasmissione o transmission gate - TG}
Per prevenire la degradazione dei valori logici (in alternativa al level restorer) è possibile usare le porte di trasmissione
o transmission gate (TG) al posto dei semplici pass transistor nmos o pmos. Le porte TG sono costituite da un transistor nmos e
un transistor pmos collegati in parallelo, controllati da segnali di gate complementari. In questo modo si accendono e si spengono
insieme e si compensano a vicenda i difetti di trasmissione dei valori logici:
\begin{itemize}
	\item il transistor nmos trasmette correttamente il valore logico basso fino a 0V
	\item il transistor pmos trasmette correttamente il valore logico alto fino a \(V_{DD}\)
\end{itemize}
\begin{center}
	\centering \includegraphics[width=0.5\textwidth]{immagini/9_pass_transistor/tg.png}

	\small{schema circuitale di una porta di trasmissione (transmission gate) \\ e relativo simbolo di abbreviazione}
\end{center}

\subsection{Tempi di propagazione e ottimizzazioni}
\subsubsection*{Analisi dei tempi di propagazione}
Per calcolare il tempo di propagazione di un circuito a pass transistor si procede similmente a quanto visto per le reti
di pull-up e pull-down delle porte logiche complementari:
\begin{enumerate}
	\item si individua il percorso peggiore (serie più lunga di pass transistor)
	\item si modella il percorso peggiore secondo la rete di Elmore
	\item si calcolano i tempi di propagazione usando le formule viste per le reti di Elmore.
\end{enumerate}
Alcune osservazioni importanti riguardo al calcolo dei tempi di propagazione nei circuiti a pass transistor:
\begin{itemize}
	\item in base al tipo di commutazione (LH o HL) la resistenza equivalente del pass transistor raddoppia se si sta
	trasmettendo un valore logico \say{cattivo}
	\item se si usando insieme pmos e nmos, il percorso peggiore può variare in base al tipo di commutazione (LH o HL)
	e non necessariamente coincide con il percorso con la serie di mosfet più lunga, specialmente se \(R_n \neq R_p\)
	\item quando si usano le porte TG, la resistenza equivalente del TG è data dal parallelo formato dalle due resistenze
	\(R_n\) e \(R_p\) degli nmos e pmos, di cui necessariamente una e una sola raddoppiata siccome trasmette un valore
	logico cattivo
	\item i tempi di propagazione intrinseci delle reti a pass transistor dipendono linearmente dal numero di pass
	transistor in serie e tale dipendenza può provocare ritardi molto elevati in circuiti complessi
\end{itemize}

\subsubsection*{Ottimizzazione dei tempi di propagazione - buffering}
Per ottimizzare i tempi di propagazione dei circuiti a pass transistor si può usare la tecnica del buffering, che consiste nel
suddividere il percorso di propagazione del segnale in più stadi, inserendo, tra uno stadio e l'altro, dei buffer costituiti
da invertitori cmos in logica statica complementare. Questi hanno la funzione di interrompere la catena di pass transistor e
deviare le correnti di carica/scarica a massa o \(V_{DD}\). In questo modo si riduce la lunghezza del percorso peggiore e di
conseguenza il tempo di propagazione complessivo del circuito.
\begin{center}
	\includegraphics[width=0.7\textwidth]{immagini/9_pass_transistor/buffering.png}
\end{center}
Per calcolare il numero ottimale di stadi \(K\) e il numero di pass transistor per stadio \(M\) bisogna:
\begin{enumerate}
	\item calcolare il tempo di propagazione per un singolo stadio \(t_{p,stadio} = t_{bu\!f\!fer} + t_{p,pass}\) secondo la
	rete di Elmore, facendo attenzione che il nodo finale di ogni stadio ha come carico la capacità di ingresso dell'invertitore
	successivo e un source/drain in meno
	\item calcolare il tempo di propagazione complessivo come \(t_{p,tot} = K \cdot t_{p,stadio}\) ed effettuare le opportune
	sostituzioni \(K = N/M\) per avere \(t_{p,tot}\) in funzione di \(M\) solamente
	\item derivare \(t_{p,tot}\) rispetto a \(M\) e porre la derivata uguale a zero per trovare il valore ottimale di \(M\)
	che minimizza il tempo di propagazione complessivo (tipicamente \(3 \leq M \leq 5\))
	\item una volta trovato \(M\) intero, si può calcolare \(K = N/M\)
\end{enumerate}
NOTA: i buffer possono essere invertenti o non invertenti, se sono invertenti è opportuno fare attenzione ad eventuali negazioni
durante la catena di elaborazione logica del segnale.

\subsection{Consumo dinamico}
Il consumo dinamico dei circuiti a pass transistor si basa sempre sulla formula generale del consumo dinamico, con le stesse
variabili e considerazioni viste per le porte logiche complementari.
\[P_{DYN} = C_F \cdot V_{DD} (V_H - V_L) \cdot f \cdot \alpha_F\]
L'unica cosa di cui fare attenzione è l'escursione tra i valori logici alto e basso, che varia in base alla configurazione
usata (nmos, pmos o TG) e alla presenza di level restorer.

\subsection{Costruzioni di porte logiche}
\subsubsection*{MUX a pass-T e a TG}
\begin{center}
	\begin{minipage}{0.3\textwidth}
		\centering \begin{circuitikz}
			\ctikzset{logic ports=ieee}
			\node [nmos, anchor=D, rotate=-90] (M1) at (0,0) {};
			\node [nmos, anchor=D, rotate=-90] (M2) at (-1,-0.9) {};
			\draw (M2.G) to[short, -*] ++(0,1.1) node[not port, scale=0.5, anchor=in] (NOT) {};
			\draw (M1.G) -- ++(0,0.2) -- (NOT.out);
			\draw (NOT.in) -- ++(-0.8,0) node[left] {\(S\)};
			\draw (M1.S) -- ++(-1,0) node[left] {\(A_0\)};
			\draw (M2.S) -- ++(0,0) node[left] {\(A_1\)};
			\draw (M1.D) -- ++(0,-0.9) -- (M2.D);
			\draw (M1.D) ++(0,-0.45) to[short, *-] ++(0.5,0) node[right] {\(F\)};
		\end{circuitikz}
	\end{minipage}
	\begin{minipage}{0.3\textwidth}
		\centering \begin{circuitikz}
			\ctikzset{logic ports=ieee}
			\ctikzset{tripoles/pmos style=emptycircle}
			\node [pmos, anchor=D, rotate=-90] (M11) at (0,0) {};
			\node [nmos, anchor=D, rotate=90] (M12) at (0,0) {};

			\node [pmos, anchor=D, rotate=-90] (M21) at (0,-2) {};
			\node [nmos, anchor=D, rotate=90] (M22) at (0,-2) {};

			\node [not port, scale=0.5, anchor=out] (NOT) at (0.6,-1) {};
			\draw (M12.G) -- (M21.G);
			\draw (NOT.out) to[short, -*] ++(0.17,0);
			\draw (NOT.in) -- ++(-0.1,0);
			\draw (M11.G) -- ++(-1.67,0) node[left] {\(S\)};
			\draw (M22.G) -- ++(-1.145,0) to[short, -*] ++(0,1.98) to[short, -*] ++(0,1.98);

			\draw (M11.D) -- ++(-0.9,0) node[left] {\(A_0\)};
			\draw (M21.D) -- ++(-0.9,0) node[left] {\(A_1\)};

			\draw (M11.S) -- (M21.S);
			\draw (M11.S) ++(0,-1) to[short, *-] ++(0.5,0) node[right] {\(F\)}; 

		\end{circuitikz}
	\end{minipage}
	\begin{minipage}{0.38\textwidth}
		\centering \begin{circuitikz}
			\ctikzset{logic ports=ieee}
			\ctikzset{tripoles/pmos style=emptycircle}
			\node [pmos, anchor=D, rotate=-90] (M11) at (0,0) {};
			\node [nmos, anchor=D, rotate=90] (M12) at (0,0) {};

			\node [pmos, anchor=D, rotate=-90] (M21) at (0,-2) {};
			\node [nmos, anchor=D, rotate=90] (M22) at (0,-2) {};

			\node [not port, scale=0.5, anchor=out] (NOT) at (0.6,-1) {};
			\draw (M12.G) -- (M21.G);
			\draw (NOT.out) to[short, -*] ++(0.17,0);
			\draw (NOT.in) -- ++(-0.1,0);
			\draw (M11.G) -- ++(-1.67,0) node[left] {\(S\)};
			\draw (M22.G) -- ++(-1.145,0) to[short, -*] ++(0,1.98) to[short, -*] ++(0,1.98);
			
			\draw (M11.D) -- ++(-0.5,0) node[not port, scale=0.5, anchor=out] (NOT1) {} (NOT1.in) -- ++(-0.2,0) node[left] {\(A_0\)};
			\draw (M21.D) -- ++(-0.5,0) node[not port, scale=0.5, anchor=out] (NOT2) {} (NOT2.in) -- ++(-0.2,0) node[left] {\(A_1\)};

			\draw (M11.S) -- (M21.S);
			\draw (M11.S) ++(0,-1) to[short, *-] ++(0.2,0) node[not port, scale=0.5, anchor=in] (NOT1) {} (NOT1.out) -- ++(0.2,0) node[right] {\(F\)}; 

		\end{circuitikz}
	\end{minipage}

	\begin{minipage}{0.3\textwidth}
		\centering \small{MUX 2:1 a pass transistor con nmos, non rigenerativo, bidirezionale, con escursione logica limitata}
	\end{minipage}
	\begin{minipage}{0.3\textwidth}
		\centering \small{MUX 2:1 a transmission gate, non rigenerativo, bidirezionale, con escursione logica completa}
	\end{minipage}
	\begin{minipage}{0.38\textwidth}
		\centering \small{MUX 2:1 a transmission gate con invertitori, rigenerativo, unidirezionale con escursione logica completa}
	\end{minipage}
\end{center}

\subsubsection*{MUX a più stadi - struttura ad albero}
Di seguito una rappresentazione di un MUX 8:1 a più stadi organizzati secondo una struttura ad albero, realizzato con MUX 2:1.
I singoli MUX 2:1 possono essere realizzati con trasmission gate per garantire un'escursione logica completa. Inoltre è possibile
aggiungere degli invertitori per assicurare un funzionamento rigenerativo e unidirezionale, con l'accorgimento di non ripeterli
per i nodi intermedi \(X_0, X_1, X_2, X_3, Y_0, Y_1\) e lasciare solo quelli sugli ingressi e all'uscita e ad ogni \(M\) stadi,
come visto nella sezione precedente sul buffering.
\begin{center}
	\begin{minipage}{0.6\textwidth}
		\centering \begin{circuitikz}
			\draw (0,0) node[muxdemux, muxdemux def={w=1, Lh=2, Rh=1.4, NL=2, NR=1, NB=0, NT=1}, anchor=center] (MUX1) {};
			\draw (MUX1.lpin 1) node[left] {$A_0$}; \draw (MUX1.lpin 2) node[left] {$A_1$};
			\draw (MUX1.lpin 1) ++(0.25,0) node[right] {$0$}; \draw (MUX1.lpin 2) ++(0.25,0) node[right] {$1$};
			\draw (MUX1.tpin 1) -- ++(0,0.3) node[above] {$S_0$};
			
			\draw (0.5,-1.3) node[muxdemux, muxdemux def={w=1, Lh=2, Rh=1.4, NL=2, NR=1, NB=0, NT=1}, anchor=center] (MUX2) {};
			\draw (MUX2.lpin 1) -- ++(-0.5,0) node[left] {$A_2$}; \draw (MUX2.lpin 2) -- ++(-0.5,0) node[left] {$A_3$};
			\draw (MUX2.lpin 1) ++(0.25,0) node[right] {$0$}; \draw (MUX2.lpin 2) ++(0.25,0) node[right] {$1$};
			
			\draw (1,-2.6) node[muxdemux, muxdemux def={w=1, Lh=2, Rh=1.4, NL=2, NR=1, NB=0, NT=1}, anchor=center] (MUX3) {};
			\draw (MUX3.lpin 1) -- ++(-1,0) node[left] {$A_4$}; \draw (MUX3.lpin 2) -- ++(-1,0) node[left] {$A_5$};
			\draw (MUX3.lpin 1) ++(0.25,0) node[right] {$0$}; \draw (MUX3.lpin 2) ++(0.25,0) node[right] {$1$};
			
			\draw (1.5,-3.9) node[muxdemux, muxdemux def={w=1, Lh=2, Rh=1.4, NL=2, NR=1, NB=0, NT=1}, anchor=center] (MUX4) {};
			\draw (MUX4.lpin 1) -- ++(-1.5,0) node[left] {$A_6$}; \draw (MUX4.lpin 2) -- ++(-1.5,0) node[left] {$A_7$};
			\draw (MUX4.lpin 1) ++(0.25,0) node[right] {$0$}; \draw (MUX4.lpin 2) ++(0.25,0) node[right] {$1$};
			
			\draw (MUX1.tpin 1) to[short, *-] (MUX1.tpin 1 -| MUX2.tpin 1) to[short, *-] (MUX1.tpin 1 -| MUX3.tpin 1) to[short, *-] (MUX1.tpin 1 -| MUX4.tpin 1);
			\draw (MUX2.tpin 1) -- (MUX2.tpin 1 |- MUX1.tpin 1);
			\draw (MUX3.tpin 1) -- (MUX3.tpin 1 |- MUX1.tpin 1);
			\draw (MUX4.tpin 1) -- (MUX4.tpin 1 |- MUX1.tpin 1);
			
			\draw (MUX2.rpin 1) ++(1.2,0) node[muxdemux, muxdemux def={w=1, Lh=2, Rh=1.4, NL=2, NR=1, NB=0, NT=1}, anchor=lpin 2] (MUX5) {};
			\draw (MUX5.lpin 1) node[above] {$X_0$}; \draw (MUX5.lpin 2) node[below] {$X_1$};
			\draw (MUX5.lpin 1) ++(0.25,0) node[right] {$0$}; \draw (MUX5.lpin 2) ++(0.25,0) node[right] {$1$};
			\draw (MUX5.tpin 1) -- ++(0,1.5) node[above] {$S_1$};
	
			\draw (MUX3.rpin 1) ++(1.2,0) node[muxdemux, muxdemux def={w=1, Lh=2, Rh=1.4, NL=2, NR=1, NB=0, NT=1}, anchor=lpin 1] (MUX6) {};
			\draw (MUX6.lpin 1) ++(0.25,0) node[right] {$0$}; \draw (MUX6.lpin 2) ++(0.25,0) node[right] {$1$};
			\draw (MUX6.lpin 1) node[above] {$X_2$}; \draw (MUX6.lpin 2) node[below] {$X_3$};
	
			\draw (MUX5.tpin 1) ++(0,1.2) node[](MUX5_1){} to[short, *-] (MUX5_1 -| MUX6.tpin 1) -- (MUX6.tpin 1);
			\draw (MUX1.rpin 1) -- (MUX5.lpin 1);
			\draw (MUX2.rpin 1) -- (MUX5.lpin 2);
			\draw (MUX3.rpin 1) -- (MUX6.lpin 1);
			\draw (MUX4.rpin 1) -- (MUX6.lpin 2);
			
			\draw (MUX5.rpin 1) ++(1.2,0) node[muxdemux, muxdemux def={w=1, Lh=2, Rh=1.4, NL=2, NR=1, NB=0, NT=1}, anchor=lpin 1] (MUX7) {};
			\draw (MUX7.lpin 1) ++(0.25,0) node[right] {$0$}; \draw (MUX7.lpin 2) ++(0.25,0) node[right] {$1$};
			\draw (MUX7.lpin 1) node[above] {$Y_0$}; \draw (MUX7.lpin 2) node[below] {$Y_1$};
			\draw (MUX7.tpin 1) -- ++(0,1.78) node[above] {$S_2$}; \draw (MUX7.rpin 1) node[right] {$F$};
	
			\draw (MUX5.rpin 1) -- (MUX7.lpin 1);
			\draw (MUX6.rpin 1) -- (MUX7.lpin 2);
		\end{circuitikz}

		\small{MUX 8:1 a più stadi con struttura ad albero \\ realizzato con MUX 2:1}
	\end{minipage}
	\begin{minipage}{0.05\textwidth}
		\centering \(\rightarrow\)
	\end{minipage}
	\begin{minipage}{0.3\textwidth}
		\centering \begin{circuitikz}
			\draw (0, 0) node[muxdemux, muxdemux def={w=2, Rh=5, Lh=7, NL=8, NR=1, NB=0, NT=3}, anchor=center, align=center] (MUX) {MUX \\ 8:1};
			\draw (MUX.lpin 1) node[left] {$A_0$}; \draw (MUX.lpin 2) node[left] {$A_1$}; \draw (MUX.lpin 3) node[left] {$A_2$}; \draw (MUX.lpin 4) node[left] {$A_3$};
			\draw (MUX.lpin 5) node[left] {$A_4$}; \draw (MUX.lpin 6) node[left] {$A_5$}; \draw (MUX.lpin 7) node[left] {$A_6$}; \draw (MUX.lpin 8) node[left] {$A_7$};
			\draw (MUX.tpin 1) node[above] {$S_1$}; \draw (MUX.tpin 2) node[above] {$S_0$}; \draw (MUX.tpin 3) node[above] {$S_2$}; \draw (MUX.rpin 1) node[right] {$F$};
		\end{circuitikz}

		\small{simbolo di abbreviazione del MUX 8:1}
	\end{minipage}
\end{center}

\newpage

\subsubsection*{Porta XOR ibrida a pass-T / TG}
Di seguito un esempio di implementazioni di XOR usando pass transistor nmos, pmos e un'implementazione ibrida con trasmission
gate e pass transistor per correggere la trasmissione di valori logici degradati secondo le seguenti osservazioni:
\begin{itemize}
	\item la prima implementazione usa solo pass-T nmos, per cui il valore logico alto in uscita sarà degradato
	\item la seconda implementazione usa solo pass-T pmos, per cui se \(B=0\) il valore logico \(A\) sarà degradato
	\item la terza implementazione ibrida usa un transmission gate aggiuntivo per trasmettere il valore logico di \(A\)
	corretto quando \(B=0\)
\end{itemize}
\begin{center}
	\begin{minipage}{0.25\textwidth}
		\centering \begin{circuitikz}
			\node [nmos, anchor=D, xscale=-1] (M1) at (0,0) {};
			\node [nmos, anchor=D, xscale=-1] (M2) at (M1.S) {};
			\draw (M1.G) node[right] {\(\overline{A}\)} (M2.G) node[right] {\(A\)};
			\draw (M1.D) node[above] {\(B\)} (M2.S) node[below] {\(\overline{B}\)};
			\draw (M1.S) to[short,*-] ++(-0.5,0) node[left] {\(F\)};
		\end{circuitikz}
	\end{minipage}
	\begin{minipage}{0.05\textwidth}
		\centering \(\rightarrow\)
	\end{minipage}
	\begin{minipage}{0.25\textwidth}
		\centering \begin{circuitikz}
			\ctikzset{tripoles/pmos style=emptycircle}
			\node [pmos, anchor=D, xscale=-1] (M1) at (0,0) {};
			\node [nmos, anchor=D, xscale=-1] (M2) at (M1.D) {};
			\draw (M1.G) -- (M2.G) ($(M1.G)!0.5!(M2.G)$) to[short,*-] ++(0.5,0) node[right] {\(A\)};
			\draw (M1.S) node[above] {\(B\)} (M2.S) node[below] {\(\overline{B}\)};
			\draw (M1.D) to[short,*-] ++(-0.5,0) node[left] {\(F\)};
		\end{circuitikz}
	\end{minipage}
	\begin{minipage}{0.05\textwidth}
		\centering \(\rightarrow\)
	\end{minipage}
	\begin{minipage}{0.35\textwidth}
		\centering \begin{circuitikz}
			\ctikzset{tripoles/pmos style=emptycircle}
			\node [pmos, anchor=D, xscale=-1] (M1) at (0,0) {};
			\node [nmos, anchor=D, xscale=-1] (M2) at (M1.D) {};
			\draw (M1.G) -- (M2.G) ($(M1.G)!0.5!(M2.G)$) to[short,*-] ++(0.5,0) node[right] {\(A\)};
			\draw (M1.S) node[above] {\(B\)} (M2.S) node[below] {\(\overline{B}\)};
			\draw (M1.D) to[short,*-] ++(-0.75,0) node[](F){} to[short,*-] ++(0,0.5) node[above] {\(F\)};
			\draw (M1.D) -- ++(-1,0) node[pmos, anchor=D, rotate=90, xscale=-1] (M3) {};
			\node [nmos, anchor=D, rotate=-90,xscale=-1] (M4) at (M3.D) {};
			\node [left] at (M3.S) {\(A\)};
			\draw (M3.G) -- ++(0,0.2) to[short, -*] ++(1.77,0);
			\draw (M4.G) -- ++(0,-0.2) to[short, -*] ++(1.77,0);
		\end{circuitikz}
	\end{minipage}
\end{center}

\subsubsection*{Sommatore a 1 bit - Half Adder}
Un half adder (sommatore a 1 bit senza riporto) è costituito da due porte logiche: una XOR per il calcolo della somma e una AND
per il calcolo del riporto. Di seguito le implementazioni delle due funzioni logiche usando trasmission gate secondo la struttura a MUX.
\begin{center}
	\begin{minipage}{0.45\textwidth}
		\centering \includegraphics[width=0.9\textwidth]{immagini/9_pass_transistor/xor_tg.png}
	\end{minipage}
	\begin{minipage}{0.45\textwidth}
		\centering \includegraphics[width=0.9\textwidth]{immagini/9_pass_transistor/and_tg.png}
	\end{minipage}
\end{center}

\subsubsection*{Sommatore a 1 bit con riporto - Full Adder}
Un full adder (sommatore a 1 bit con riporto) è costituito, invece, da due funzioni più complesse:
\[S = A \oplus B \oplus C_{in} = P \oplus C_{in} \qquad C_{out} = A \ccdot (\overline{A \oplus B}) + (A \oplus B) \ccdot C_{in} = A \ccdot \overline{P} + P \ccdot C_{in} \qquad P = A \oplus B\]
Si nota, quindi, che è possibile impiegare una XOR e una XNOR per calcolare \(P\) e \(\overline{P}\) (1° stadio), una XOR per
calcolare la somma \(S\) (2° stadio) e un MUX 2:1 per calcolare il riporto \(C_{out}\) (3° stadio).

\begin{center}
	\begin{minipage}{0.25\textwidth}
		\centering \includegraphics[width=0.9\textwidth]{immagini/9_pass_transistor/fa_1.png}
	\end{minipage}
	\begin{minipage}{0.33\textwidth}
		\centering \includegraphics[width=0.9\textwidth]{immagini/9_pass_transistor/fa_2.png}
	\end{minipage}
	\begin{minipage}{0.33\textwidth}
		\centering \includegraphics[width=0.9\textwidth]{immagini/9_pass_transistor/fa_3.png}
	\end{minipage}
\end{center}

%\section{Fabbricazione dei circuiti integrati}
\subsection{Fasi di progettazione e costruzione}
\subsubsection*{Fasi di progettazione}
La fase di progettazione è svolta dal progettista del circuito integrato che ha conoscenze approfondite di elettronica e del
funzionamento dei componenti da realizzare. Le fasi di progettazione sono:
\begin{enumerate}
	\item definizione delle specifiche del circuito integrato
	\item progettazione dello schema a blocchi
	\item progettazione dello schema circuitale
	\item progettazione del layout
\end{enumerate}

\subsubsection*{Fasi di costruzione}
Le fasi di costruzione sono svolte in un impianto di fabbricazione (fab) da tecnici specializzati che non necessariamente hanno
conoscenze approfondite in elettronica. Le fasi di costruzione sono:
\begin{center}
	\begin{minipage}{0.55\textwidth}
		\begin{enumerate}
			\item fabbricazione del wafer (metodo di Czochralski)
		\end{enumerate}
	\end{minipage}
	\begin{minipage}{0.4\textwidth} \(\) \end{minipage}
	\vspace{-5pt}

	\color[gray]{0.75}\rule{0.9\textwidth}{0.4pt}\color{black}
	
	\begin{minipage}{0.4\textwidth}
		\begin{enumerate}[start=2]
			\item selezione delle regioni attive
			\item selezione del tipo di substrato
			\item ossido di gate ed elettrodo di gate
			\item diffusioni N+ e P+
		\end{enumerate}
	\end{minipage}
	\begin{minipage}{0.55\textwidth}
		\(\left. \vcenter{\hbox{\huge\bfseries\vrule height 2.2cm width 0pt}} \right\} \text{Front End of the Line - FEOL}\)
	\end{minipage}
	
	\color[gray]{0.75}\rule{0.9\textwidth}{0.4pt}\color{black}

	\begin{minipage}{0.4\textwidth}
		\begin{enumerate}[start=6]
			\item contatti e interconnessioni
			\item packaging
		\end{enumerate}
	\end{minipage}
	\begin{minipage}{0.55\textwidth}
		\(\left. \vcenter{\hbox{\huge\bfseries\vrule height 1.2cm width 0pt}} \right\} \text{Back End of the Line - BEOL}\)
	\end{minipage}
\end{center}

\noindent
Nelle spiegazioni successive si analizzeranno le varie fasi di produzione prendendo come modello la costruzione di un
invertitore CMOS.

\subsection{Fabbricazione del Wafer con metodo di Czochralski}
Attraverso il metodo di Czochralski si ottiene un lingotto di silicio monocristallino che viene successivamente tagliato in
fette sottili (wafer). Il processo prevede i seguenti passaggi:
\begin{enumerate}
	\item si fonde del silicio ad alta purezza in un crogiolo di quarzo, è possibile aggiungere elementi chimici (ad esempio
	arsenico o fosforo) in maniera controllata per il drogaggio di base
	\item si immerge di un seme di silicio monocristallino (barra di silicio puro) nella massa fusa e viene messo in rotazione
	attorno al suo asse verticale
	\item il seme viene estratto lentamente in modo che gli atomi di silicio fuso si solidificano e si dispongono naturalmente
	secondo la struttura cristallina del seme (per una proprietà propria del silicio), la velocità di estrazione controlla il
	diametro del lingotto
	\item il lingotto viene tagliato in fette sottili (wafer) tramite una sega a filo abrasivo di diamante
	\item i wafer vengono sottoposti a processi di lappatura e molatura per ottenere una superficie perfettamente piana che viene
	ricoperta con uno strato sottile di ossido di silicio per proteggerla
\end{enumerate}
\begin{center}
	\begin{minipage}{0.3\textwidth}
		\centering \includegraphics[width=0.8\textwidth]{immagini/10_tecniche_di_fabbricazione/wafer_1.png} \\
		\small{inserimento del seme nel crogiolo}
	\end{minipage}
	\begin{minipage}{0.3\textwidth}
		\centering \includegraphics[width=0.8\textwidth]{immagini/10_tecniche_di_fabbricazione/wafer_2.png} \\
		\small{estrazione e formazione del lingotto}
	\end{minipage}
	\begin{minipage}{0.3\textwidth}
		\centering \includegraphics[width=0.8\textwidth]{immagini/10_tecniche_di_fabbricazione/wafer_3.png} \\
		\small{taglio del lingotto in wafer}
	\end{minipage}
\end{center}

\subsection{Processo selettivo di costruzione del circuito integrato per litografia}
\subsubsection*{Divisione di un wafer in die}
Ogni wafer viene suddiviso a scacchiera in tante aree quadrate chiamate die (o chip). Ogni die compone un circuito integrato
completo che alla fine della lavorazione verrà tagliato e confezionato singolarmente. La dimensione dei die dipende dallo
spazio occupato dal circuito integrato e dal numero di difetti critici presenti nel wafer.

Un difetto critico è un difetto che rende inutilizzabile il circuito integrato. La probabilità di avere un difetto critico
aumenta con l'aumentare della superficie del die. Per questo motivo, per aumentare la resa di produzione, si tende a ridurre
la dimensione dei die.

\subsubsection*{Maschera}
Per identificare le aree del wafer che devono essere lavorate in ogni fase del processo di costruzione, si utilizza una maschera.
La maschera è una lastra di quarzo trasparente su cui sono incisi i disegni delle aree che devono essere lavorate.

\subsubsection*{Processo selettivo per litografia}
Per selezionare effettivamente le aree del wafer da lavorare e proteggere le altre si utilizza un processo selettivo per
litografia che prevede i seguenti passaggi:

\begin{center}
	\begin{minipage} {0.35\textwidth}
		\centering \includegraphics[width=\textwidth]{immagini/10_tecniche_di_fabbricazione/litografia.png}
	\end{minipage}
	\begin{minipage} {0.6\textwidth}
		\begin{enumerate}
			\item si ricopre il wafer con uno strato sottile di materiale fotosensibile chiamato fotoresist
		\end{enumerate}
		\vspace{18pt}
		\begin{enumerate}[start=2]
			\item si espone il wafer alla luce ultravioletta attraverso la maschera, le aree del fotoresist esposte alla luce
			cambiano le loro proprietà chimiche e diventano solubili in un apposito solvente
		\end{enumerate}
		\vspace{20pt}
		\begin{enumerate}[start=3]
			\item si sciolgono le aree del fotoresist diventate solubili, lasciando scoperte le aree del wafer che devono essere
			lavorate
		\end{enumerate}
		\vspace{12pt}
		\begin{enumerate}[start=4]
			\item si ottiene così il pattern desiderato sul wafer in cui le aree scoperte possono essere lavorate e il resto
			della superficie è protetta dal fotoresist
		\end{enumerate}
		\vspace{15pt}
		\begin{enumerate}[start=5]
			\item si esegue la lavorazione desiderata (ad esempio l'ossidazione o l'impiantazione ionica)
		\end{enumerate}
		\vspace{30pt}
		\begin{enumerate}[start=6]
			\item si rimuove il fotoresist rimanente con un altro solvente apposito
		\end{enumerate}
		\vspace{25pt}
		\begin{enumerate}[start=7]
			\item si ottiene così il wafer con le aree lavorate secondo il pattern desiderato
		\end{enumerate}
	\end{minipage}
\end{center}

\newpage
\subsection{Selezione delle regioni attive}
Il processo di selezione delle regioni attive consiste nel definire le regioni del wafer in cui verranno realizzati i singoli
mosfet e inserire delle barriere di ossido per isolare elettricamente le varie regioni. Il processo prevede i seguenti passaggi:

\begin{center}
	\begin{minipage} {0.35\textwidth}
		\centering \includegraphics[width=\textwidth]{immagini/10_tecniche_di_fabbricazione/regioni_attive.png}
	\end{minipage}
	\begin{minipage} {0.6\textwidth}
		\begin{enumerate}
			\item si deposita uno strato di nitruro di silicio (Si\(_3\)N\(_4\)) e si utilizza il processo di litografia selettiva
			per selezionare le aree in cui vanno inserite le barriere di ossido isolante
		\end{enumerate}
		\vspace{5pt}
		\begin{enumerate}[start=2]
			\item si rimuove il nitruro di silicio, l'ossido di silicio e una parte del substrato di silicio nelle aree
			selezionate tramite un attacco chimico con acido; per scavare verticalmente (e non anche lateralmente) le cavità,
			il solvente viene ionizzato e nebulizzato così, sotto l'azione di un campo elettrico verticale, gli ioni
			vengono direzionati e colpiscono il	wafer solo verticalmente senza intaccare le pareti laterali; questo processo
			è chiamato Reactive Ion Etching (RIE)
		\end{enumerate}
		\vspace{5pt}
		\begin{enumerate}[start=3]
			\item si rimuove il photoresist rimanente e si deposita l'ossido di isolamento (di bassa qualità siccome deve solo
			fungere da isolante) nelle regioni scoperte dal nitruro (ovvero nelle cavità scavate con il RIE) formando le
			Shallow Trench Isolation (STI)
		\end{enumerate}
		\vspace{5pt}
		\begin{enumerate}[start=4]
			\item si rimuove il nitruro di silicio e lo strato di ossido di silicio rimanenti, in modo da liberare la superficie di
			silicio puro del wafer
		\end{enumerate}
	\end{minipage}
\end{center}

\subsection{Selezione del tipo di substrato}
Il processo di selezione del tipo di substrato consiste nel creare un substrato di tipo P o N a seconda del tipo di mosfet
che si vuole realizzare all'interno delle aree attive definite nel punto precedente. Il processo prevede i seguenti passaggi:

\begin{center}
	\begin{minipage} {0.35\textwidth}
		\centering \includegraphics[width=\textwidth]{immagini/10_tecniche_di_fabbricazione/substrato.png}
	\end{minipage}
	\begin{minipage} {0.6\textwidth}
		\begin{enumerate}
			\item si utilizza il processo di litografia selettiva per selezionare le regioni attive in cui si vuole creare un
			substrato di tipo P (per la realizzazione di mosfet NMOS) o di tipo N (per la realizzazione di mosfet PMOS)
		\end{enumerate}
		\vspace{22pt}
		\begin{enumerate}[start=2]
			\item si bombardano le regioni scoperte con ioni droganti (ad esempio arsenico per il substrato di tipo P o fosforo
			per il substrato di tipo N) che penetrano nel silicio; questo processo è chiamato impiantazione ionica
		\end{enumerate}
		\vspace{5pt}
		\begin{enumerate}[start=3]
			\item si riscalda il wafer (annealing) per distribuire il drogante e farlo disporre correttamente nella struttura
			cristallina del silicio in modo da formare il substrato di tipo desiderato; questo è chiamato attivazione del drogaggio
		\end{enumerate}
		\vspace{22pt}
		\begin{enumerate}[start=4]
			\item si rimuove il fotoresist rimanente e si ripete tutto il processo di impiantazione ionica e attivazione del
			drogaggio con annealing per creare le aree di substrato del tipo opposto necessarie per la realizzazione dei mosfet
			complementari
		\end{enumerate}
	\end{minipage}
\end{center}

\subsection{Ossido di gate ed elettrodo di gate}
La fase successiva nella realizzazione dei mosfet è deposizione dell'ossido di gate e dell'elettrodo di gate in polisilicio.
Il processo prevede i seguenti passaggi:

\begin{center}
	\begin{minipage} {0.35\textwidth}
		\centering \includegraphics[width=\textwidth]{immagini/10_tecniche_di_fabbricazione/gate.png}
	\end{minipage}
	\begin{minipage} {0.6\textwidth}
		\begin{enumerate}
			\item si deposita uno strato sottile di ossido di silicio (SiO\(_2\)) di alta qualità (costituirà il dielettrico
			dei condensatori dei gate dei mosfet) su tutta la superficie del wafer, non richiede un processo selettivo in quanto
			lo strato di ossido ha spessore trascurabile rispetto alle STI
			\item si deposita uno strato di polisilicio (silicio policristallino con proprietà metalliche) sopra l'ossido di
			silicio per formare l'elettrodo di gate dei mosfet
			\item si utilizza il processo di litografia selettiva per selezionare le aree in cui si vogliono formare i gate dei
			mosfet e si elimina il polisilicio nelle aree scoperte perché non selezionate tramite un attacco chimico
		\end{enumerate}
	\end{minipage}
\end{center}

\vspace{2cm}

\subsection{Diffusioni N+ e P+}
Le diffusioni N+ e P+ servono a formare le regioni di source, drain e body dei mosfet. Il processo prevede i seguenti passaggi:

\begin{center}
	\begin{minipage} {0.35\textwidth}
		\centering \includegraphics[width=\textwidth]{immagini/10_tecniche_di_fabbricazione/diffusioni.png}
	\end{minipage}
	\begin{minipage} {0.6\textwidth}
		\begin{enumerate}
			\item si selezionano le aree in cui eseguire le diffusioni N+ (source e drain degli NMOS, body dei PMOS) tramite il
			processo di litografia selettiva
			\end{enumerate}
		\vspace{5pt}
		\begin{enumerate}[start=2]
			\item si bombarda il silicio con ioni di arsenico o fosforo attraverso il processo di impiantazione ionica con
			attivazione del drogaggio per annealing utilizzando per creare le regioni N+ nelle aree scoperte, infine si rimuove
			il fotoresist rimanente
			\end{enumerate}
		\vspace{5pt}
		\begin{enumerate}[start=3]
			\item si ripete il processo analogo di selezione delle aree, impiantazione ionica e attivazione del drogaggio per
			creare le regioni P+ (source e drain dei PMOS, body degli NMOS)
		\end{enumerate}
	\end{minipage}
\end{center}

\newpage

\subsection{Contatti e interconnessioni}
Dopo aver creato i mosfet, è necessario creare i terminali elettrici e le interconnessioni tra i vari componenti del circuito
integrato. Il processo prevede i seguenti passaggi:

\begin{center}
	\begin{minipage} {0.35\textwidth}
		\centering \includegraphics[width=\textwidth]{immagini/10_tecniche_di_fabbricazione/contatti.png}
	\end{minipage}
	\begin{minipage} {0.6\textwidth}
		\begin{enumerate}
			\item si deposita uno strato di ossido isolante (arancione) sopra tutta la superficie del wafer per isolare
			la struttura dei mosfet dalle piste metalliche di interconnessione costruite sopra
		\end{enumerate}
		\vspace{5pt}
		\begin{enumerate}[start=2]
			\item si utilizza un processo di litografia selettiva per forare l'ossido e raggiungere le aree di silicio dei
			terminali dei mosfet (source, drain, body e gate) dove vanno creati i contatti;
			\end{enumerate}
		\vspace{5pt}
		\begin{enumerate}[start=3]
			\item si deposita uno strato di metallo (ad esempio alluminio o rame) che riempie i fori sempre attraverso un
			processo di litografia selettiva in modo da formare i terminali del mosfet
		\end{enumerate}
		\vspace{5pt}
		\begin{enumerate}[start=4]
			\item si riveste l'intera superficie con un altro strato di ossido isolante (giallo) per separare i vari livelli di
			interconnessione
		\end{enumerate}
		\vspace{5pt}
		\begin{enumerate}[start=5]
			\item si riesegue l'intero processo di foratura e deposizione del metallo e deposizione del dielettrico più volte
			per creare i vari livelli di interconnessione necessari a collegare tra loro i vari componenti del circuito
		\end{enumerate}
	\end{minipage}
\end{center}

\vspace{1cm}

\subsection{Packaging}
Il processo di packaging consiste nel preparare il die per l'utilizzo esterno al fine di poterlo collegare ad altri circuiti
(ad esempio su una scheda madre) e proteggerlo da agenti esterni. Il processo prevede i seguenti passaggi:
\begin{enumerate}
	\item si taglia il wafer in singoli die tramite una sega a filo abrasivo di diamante
	\item si monta ogni die in un contenitore protettivo (package) che può essere di plastica o metallo
	\item si collegano i terminali del die (detti pad) ai terminali esterni del package tramite fili sottili di oro
\end{enumerate}
Alcuni esempi di package sono illustrai in figura a destra

\begin{center}
	\begin{minipage}{0.4\textwidth}
		\centering \includegraphics[width=0.9\textwidth]{immagini/10_tecniche_di_fabbricazione/packaging.png}

		\small{interconnessioni tra i pad del die e i terminali del package}
	\end{minipage}
	\begin{minipage}{0.55\textwidth}
		\centering
		\begin{tabular}{c c c}
			\includegraphics[width=0.25\textwidth]{immagini/10_tecniche_di_fabbricazione/th.png} &
			\includegraphics[width=0.27\textwidth]{immagini/10_tecniche_di_fabbricazione/smd.png} &
			\includegraphics[width=0.27\textwidth]{immagini/10_tecniche_di_fabbricazione/bga.png}\\
			\small{Through-Hole} & \small{Surface-Mount} & \small{Ball Grid Array} \\
		\end{tabular}
	\end{minipage}
\end{center}

\newpage

\subsection{Layout, regole di layout e sviluppo delle maschere}
\subsubsection*{Definizione del layout}
Il layout è la rappresentazione grafica in scala del circuito integrato che mostra la disposizione spaziale dei vari componenti
con le loro dimensioni e le interconnessioni tra di essi. Il layout è composto da una serie di maschere che verranno utilizzate
nel processo di costruzione del circuito integrato.

Il layout viene realizzato dal progettista del circuito integrato e viene poi usato dal tecnico di fabbricazione nelle varie
fasi di costruzione del circuito integrato. Funge da linguaggio comune tra progettista e tecnico di fabbricazione.

Ad ogni maschera è associta una fase del processo di costruzione del circuito integrato:
\begin{itemize}
	\item active: definizione delle regioni attive
	\item p-well/n-well: selezione del tipo di substrato
	\item n-diff/p-diff: diffusioni N+ e P+
	\item polysilicon: ossido di gate ed elettrodo di gate
	\item contact: deposizione del metallo per i terminali dei mosfet
	\item metal1, metal2, ... : deposizione del metallo per i vari livelli di interconnessione
	\item via 1-2, via 2-3, ... : foratura del dielettrico tra i vari livelli di interconnessione
\end{itemize}
\begin{center}
	\includegraphics[width=0.9\textwidth]{immagini/10_tecniche_di_fabbricazione/layout.png}

	\small{esempio di layout (maschere) per la realizzazione di un inverter CMOS, come illustrato nelle fasi di fabbricazione
	precedenti, a sinistra le FEOL e a destra le BEOL}
\end{center}

\subsubsection*{Regole di layout}
Affinché il layout possa essere effettivamente utilizzato per la costruzione del circuito integrato, deve rispettare una serie
di regole per garantire la corretta realizzazione fisica del circuito integrato. Le regole di layout sono dovute a:
\begin{itemize}
	\item lminima risoluzione e tolleranza della fotolitografia
	\item inevitabile disallineamento delle maschere
	\item imprecisioni del processo
	\item buon senso del progettista
\end{itemize}
Le regole di layout si dividono in:
\begin{itemize}
	\item \textbf{intra-layer}: regole che riguardano una singola maschera, ovvero le dimensioni e le distanze minime tra gli
	elementi di una singola maschera (per rispettare la risoluzione e le tolleranze del processo)
	\item \textbf{inter-layer}: regole che riguardano più maschere, ovvero le distanze minime che gli elementi di maschere
	diverse devono rispettare tra di loro (per contenere problemi di disallineamento)
\end{itemize}



\end{document}
