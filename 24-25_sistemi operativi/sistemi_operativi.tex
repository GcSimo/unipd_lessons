\documentclass[a4paper]{article}
\usepackage[utf8]{inputenc} % standard unicode
\usepackage[italian]{babel} % corretta sillabazione in italiano
\usepackage{geometry} % per impostare margini e layout pagina
\usepackage{amssymb} % per l'ambiente matematico
\usepackage{amsmath} % per l'ambiente matematico
\usepackage{enumitem} % per elenchi puntati
\usepackage{multirow} % per celle che si espandono su più righe
\usepackage{tabularx} % per tabelle con larghezza flessibile
\usepackage{booktabs} % per linee orizzontali tabelle
\usepackage{hyperref} % per collegamenti
\usepackage{dirtytalk} % per le ""
\usepackage{cancel} % per barrare testo

% per margini
\geometry{a4paper,left=25mm, right=25mm, bottom=25mm, top=30mm}

% per centrare testo nelle tabelleX
\renewcommand\tabularxcolumn[1]{m{#1}}

% per elenchi puntati
\setlist[itemize]{label=-, partopsep=0pt, topsep=3pt, itemsep=0pt}


\title{Sistemi Operativi}
\author{Giacomo Simonetto e Diego Chiesurin}
\date{Secondo Semestre 2024-25}

\begin{document}generale
\maketitle
\begin{abstract}
	Appunti del corso di Sistemi Operativi della facoltà di Ingegneria Informatica dell'Università di Padova.
\end{abstract}

\newpage

\tableofcontents                       % Genera l'indice

\newpage



\section{Introduzione}
\subsection{Cos'è un sistema operativo}
\newpage
\section{Ripasso Hardware}
\newpage
\section{Struttura dei Sistemi Operativi}
\newpage
\section{Concorrenza}
\subsection{Processi e Thread}
\subsubsection*{Processi}
Un Processo è un'istanza attiva di un programma che comprende tutte le informazioni necessarie per il suo funzionamento, risiede quindi in RAM. \\
Un processo è costituito da: 
\begin{itemize}
	\item Codice eseguibile
	\item Contesto di esecuzione (Process Control Block - PCB)
\end{itemize}
Inoltre un processo si può trovare nei seguenti stati:
\begin{itemize}
	\item New: Il processo è stato creato ma non è ancora stato avviato
	\item Ready: Il processo è pronto per essere eseguito, ma attende che la CPU sia libera
	\item Running: La CPU sta eseguendo il processo
	\item Waiting: Il processo è in pausa perché attende una risorsa o un evento
	\item Terminated: Il processo ha completato l'esecuzione o è stato terminato
\end{itemize}
\subsubsection*{Thread}
Un thread (filo) è l'unità base di esecuzione all'interno della CPU. Ogni processo deve essere quindi suddiviso in uno o più thread per essere eseguito.
Per gestirne l'esecuzione ogni thread possiede il proprio TCB, simile al PCB ma specifico per i thread. \\
Condivide con gli altri thread dello stesso processo:
\begin{itemize}
	\item Sezione del codice
	\item Sezione dati
	\item Risorse allocate al processo originale (file)
\end{itemize}
I thread vengono utilizzati nelle architetture multicore perché permettono il parallelismo e la concorrenza.
\subsubsection*{Process Control Block - PCB}
Il PCB è una struttura dati contenente tutte le informazioni per il sistema operativo per gestire un processo durante il suo ciclo di vita. Soprattutto per la gestione dei context switch. \\
Struttura di un PCB in generale:
\begin{itemize}
	\item Identificatore (PID)
	\item Stato del processo
	\item Contesto della CPU: Registri
	\item Informazioni sulla memoria:
	\begin{itemize}
		\item Base e limite della memoria
		\item Tabella delle pagine o dei segmenti
		\item Heap e Stack
		\item Initialized e uninitialized data segment
	\end{itemize}
	\item Informazioni sulle risorse
	\item Informazioni di scheduling
	\item Informazioni di comunicazione tra processi
\end{itemize} 
\subsubsection*{Multicore}
Un'architettura multicore è composta da più unità di calcolo all'interno della stessa CPU, che al sistema operativo appaiono come CPU separate.
Su questi tipi di sistemi è possibile eseguire in parallelo i thread, poiché il sistema può assegnare thread diversi a diverse unità di calcolo.
\subsection{Concorrenza vs Parallelismo}
\subsubsection*{Tabella di confronto}
\begin{center}
\centering
\begin{tabularx}{\textwidth}{|X|X|}
	\hline
	\textbf{Concorrenza} & \textbf{Parallelismo} \\
	\hline
	Atto di eseguire più calcoli nello stesso intervallo di tempo & Atto di eseguire più calcoli simultaneamente \\
	\hline
	Task multipli vengono eseguiti negli stessi intervalli di tempo, senza ordine specifico particolare & Task multipli vengono eseguiti contemporaneamente in sistemi multi-CPU o multicore \\
	\hline
	L'esecuzione del task sembra contemporanea, ma non lo è & Viene permesso da strutture hardware apposite \\
	\hline
	L'effetto è dovuto al time-slicing della CPU, ovvero allo scheduler(context switching) che dedica l'unità di calcolo ai vari task per unità di tempo inifnitesimali & Si ottiene trasformando un flusso di esecuzione sequenziale in uno parallelo \\
	\hline
\end{tabularx}
\end{center}
\subsubsection*{Tipi di Parallelismo}
Ci sono due tipi di parallelismo:
\begin{itemize}
	\item Data parallelism: la stessa operazione viene eseguita contemporaneamente sui dati, questi vengono suddivisi tra i vari thread in modo che non ci siano conflitti
	\item Task parallelism: parallelizzare del codice tra thread diversi
\end{itemize}
\subsection{Diagramma Temporale}
\subsubsection*{Tempi di esecuzione}
\subsubsection*{Grafo di Precedenza}
\subsubsection*{Sistemi di Processi}
\subsubsection*{Massimo Grado di Parallelismo}
\subsubsection*{Interferenza e Determinatezza}
\subsection{Risorse}
\subsection{Deadlock}
\subsubsection*{Definizione di Deadlock}
\subsubsection*{Definizione di Livelock}
\subsubsection*{Gestione dei Deadlock}
\subsubsection*{Ripristino dei Deadlock}
\subsection{Prevenzione del Deadlock}
\subsubsection*{Allocazione Globale}
\subsubsection*{Allocazione Gerarchica}
\subsubsection*{Algoritmo del Banchiere}
\subsection{Grafo delle Risorse}
\subsubsection*{Individuazione dei Deadlock}
\subsection{Race Condition e Sezioni Critiche}
\subsection{Semafori}
\subsubsection*{Semafori Contatore}
\subsubsection*{Semafori Binari - Lock Mutex}
\subsubsection*{Semafori Privati}
\subsubsection*{Busy Waiting}
\subsubsection*{Deadlock e Starvation}
\subsection{Monitor}
\subsubsection*{Problema dei 5 Filosofi}
\subsubsection*{Allocazione Risorse}
\subsection{Memory Barrier}
\newpage
\section{Scheduling CPU}
\newpage
\section{Gestione Memoria Principale}
\newpage
\section{Gestione File}
\newpage
\section{Affidabilità}




\end{document}