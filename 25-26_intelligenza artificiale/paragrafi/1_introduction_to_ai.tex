\section{Introduction to AI}
\subsection{Introduzione}
\subsubsection*{Definizione (una delle tante possibli)}
\begin{center}
    ``L'intelligenza artificiale è lo studio di come far compiere ai computer compiti che attualmente sono svolti in maniera
	migliore dagli esseri umani'' \\[5pt]
    \hspace*{\fill} {--- Elaine Rich \& Kevin Knight, 1965}
\end{center}

\subsubsection*{Branche dell'intelligenza artificiale}
L'intelligenza artificiale è multidisciplinare. Da domande e questi di varie discipline sono stati originati vari filoni di
ricerca del'IA, tra cui:
\begin{itemize}
	\item \textbf{filosofia}: origine della conoscenza, logica, metodi di ragionamento, struttura fisica della mente
	\item \textbf{matematica}: rappresentazione formale della conoscenza, teoria della probabilità, computabilità
	\item \textbf{neuroscienze}: funzionamento del cervello umano, neural networks, brain machine interfaces
	\item \textbf{psicologia}: percezione, controllo motorio, interazioni umano-macchina
	\item \textbf{linguistica}: rappresentazione della conoscenza, natural language processing
	\item \textbf{control theory}: cybernetica, robotica
	\item \textbf{economia}: teoria dell'utilità, teoria delle decisioni (game theory, multiagent systems)
	\item \textbf{informatica}: algoritmi, strutture dati, hardware dedicato (GPU, TPU) e Internet of Things (IoT)
\end{itemize}

\subsection{Storia}
\subsubsection*{Storia dell'IA}
\begin{center}
	\begin{tabularx}{\textwidth}{c X}
		1943 &
		\textbf{McCulloch} e \textbf{Pitts} propongono il primo modello di circuito booleano basato sulla struttura del cervello
		umano \\
		\midrule

		1950 &
		\textbf{Alan Turing} pubblica il paper ``Computing Machinery and Intelligence'', in cui propone il Turing Test come
		criterio per stabilire se una macchina possa essere considerata intelligente. \\
		\midrule

		1956 &
		\textbf{John McCarthy} conia il termine ``intelligenza artificiale'' \\
		\midrule

		1966-73 & 
		\textbf{primo inverno dell'IA} causato dall'aumento della complessità computazionale degli algoritmi intelligenti da
		sviluppare e dalla riduzione dei finanziamenti alla ricerca \\
		\midrule

		1970-80 &
		sviluppo di sistemi esperti e sviluppo delle \textbf{reti neurali} \\
		\midrule

		1980-87 &
		\textbf{secondo inverno dell'IA} causato dalla delusione per le limitazioni delle reti neurali dovute allo scarso
		sviluppo tecnologico di allora e dalla riduzione dei finanziamenti alla ricerca \\
		\midrule

		1997 &
		\textbf{Deep Blue} di IBM sconfigge il campione del mondo di scacchi Garry Kasparov \\
		\midrule

		2005 &
		\textbf{Stanley}, un'auto autonoma sviluppata dalla Stanford University, vince la DARPA Grand Challenge \\
		\midrule

		2011 &
		Apple lancia \textbf{Siri}, un assistente virtuale basato su intelligenza artificiale e riconoscimento vocale \\
		\midrule

		2012 &
		\textbf{AlexNet}, una rete neurale convoluzionale, vince la competizione ImageNet, segnando un grande passo avanti
		nel riconoscimento delle immagini e riportando popolarità alle reti neurali \\
		\midrule

		2017 &
		\textbf{AlphaGo} di DeepMind sconfigge il campione mondiale di Go, Lee Sedol \\
		\midrule

		2019 &
		Boston Dynamics presenta \textbf{Spot}, un robot quadrupede capace di muoversi in ambienti complessi \\
		\midrule

		2022 &
		OpenAI rilascia \textbf{ChatGPT}, un modello di linguaggio basato su GPT-3.5, che dimostra capacità avanzate di
		generazione di testo e comprensione del linguaggio naturale \\


	\end{tabularx}
\end{center}

\subsubsection*{Stato attuale basato sll'AI index di Stanford}
\begin{enumerate}
	\item \textbf{performance}: i modelli di intelligenza artificiale continuano a migliorare in vari compiti, avvicinandosi
	sempre di più alle capacità umane
	\item \textbf{presenza quotidiana}: la presenza di tecnologie basate su intelligenza artificiale è in crescita in vari settori della
	vita quotidiana, tra cui sanità, finanza, trasporti e intrattenimento
	\item \textbf{finanziamenti}: gli investimenti in ricerca e sviluppo nell'ambito dell'intelligenza artificiale sono in
	crescita, sia da parte di governi che di aziende private
	\item \textbf{divario USA-Cina}: il divario tra Stati Uniti e Cina in termini di ricerca e sviluppo nell'ambito
	dell'intelligenza artificiale si sta riducendo, con la Cina che sta facendo progressi significativi
	\item \textbf{etica}: c'è una crescente attenzione all'etica, alla regolamentazione e alle implicazioni sociali
	\item \textbf{impiego nazionale}: in generale l'impiego di intelligenza artificiale nei vari paesi è in crescita, anche se 
	è presente ancora un certo divario tra paesi sviluppati e in via di sviluppo
	\item \textbf{costi}: i costi associati allo sviluppo e all'implementazione di tecnologie basate su intelligenza artificiale
	stanno diminuendo, rendendo queste tecnologie sempre più accessibili
	\item \textbf{regolamentazione}: i governi stanno iniziando a sviluppare regolamentazioni specifiche sia per l'uso
	responsabile dell'intelligenza artificiale che per gli investimenti nel settore
	\item \textbf{educazione}: c'è un crescente diffusione di programmi educativi e corsi di formazione sull'informatica
	e sull'intelligenza artificiale, anche se c'è ancora divario tra paesi e regioni
	\item \textbf{sviluppo}: la ricerca e lo sviluppo nell'ambito dell'intelligenza artificiale stanno continuando a progredire
	rapidamente, con nuove tecnologie e approcci che emergono costantemente
	\item \textbf{impatto}: l'impatto dell'intelligenza artificiale sulle ricerche scientifiche di alto livello è molto
	significativo, dati anche i premi Nobel per la fisica e la chimica del 2024 assegnati a ricerche che hanno fatto uso
	di intelligenza artificiale
	\item \textbf{ragionamento complesso}: i modelli di intelligenza artificiale continuano ad avere difficoltà con compiti
	che richiedono ragionamento complesso e precisione
\end{enumerate}

\subsection{Tipi di intelligenza}
\subsubsection*{I 4 tipi di intelligenza}
Esistono 4 modi diversi per definire cosa si intende ``intelligenza artificiale''. In base alla definizione scelta si potrà
sviluppare un processo per decidere se un sistema è intelligente o meno e di conseguenza si avranno obiettivi e metodi di
sviluppo diversi. Le 4 definizioni sono:

\begin{center}
	\begin{minipage}{0.45\textwidth}
		\begin{itemize}
			\item[] think humanly - pensare come un umano
			\item[] acting humanly - agire come un umano
		\end{itemize}
	\end{minipage}
	\begin{minipage}{0.45\textwidth}
		\begin{itemize}
			\item[] think rationally - pensare razionalmente
			\item[] acting rationally - agire razionalmente
		\end{itemize}
	\end{minipage}
\end{center}

\subsubsection*{Think humanly - cognitive science}
Si basa sullo studio cognitive science, ovvero la branca della psicologia che studia i processi mentali come l'apprendimento,
la memoria, il ragionamento e la percezione. L'obiettivo è quello di creare modelli computazionali che simulino i processi
cognitivi della mente umana ottenuti dalla predizione dei comportamenti (introspezione ed esperimenti psicologici, top-down
approach) o dalla identificazione dei modelli umani (brain-imaging, bottom-up approach).

\subsubsection*{Acting humanly - Turing test}
Dalla definizione di Alan Turing: ``Una macchina è intelligente se fa cose intelligenti''. Questo principio è alla base del
Turing Test o Imitation Game, che serve appunto a valutare se una macchina agisce in modo intelligente come un essere umano.
Il test coinvolge vari settori come natural language processing, knowledge representation, automated reasoning e machine learning
a cui si aggiungono computer vision, speech recognition e robotics nei test più estesi.

\subsubsection*{Think rationally - logic}
Questa definizione si basa sul ragionamento logico ed è in contrapposizione con la cognitive science in quanto non sempre gli
esseri umani pensano in modo razionale. L'obiettivo è quello di creare sistemi che seguano le regole della logica formale per
ragionare correttamente e prendere decisioni basate su informazioni disponibili (sfruttando la logica della filosofia greca
e i metodi probabilistici per il calcolo dell'incertezza).

\subsubsection*{Acting rationally - rational agent}
Questa definizione si basa sulla teoria dei rational agent (o sistemi intelligenti), ovvero agenti che agiscono in modo
da massimizzare la loro performance measure, basandosi su ciò che percepiscono dall'ambiente in cui operano. Questa è
l'interpretazione più diffusa e accettata come standard per l'intelligenza artificiale attuale. Un rational agent deve
essere in grado di operare autonomamente, percepire l'ambiente, adattarsi ai cambiamenti, vivere per un tempo prolungato
e raggiungere i propri obiettivi, attraverso le stesse tecniche definite nel Turing Test e nel ragionamento logico.

\subsection{Rischi e benefici dell'IA}
\subsubsection*{Benefici}
\begin{itemize}
	\item diminuzione del lavoro ripetitivo e noioso
	\item aumento della produttività di servizi e benefici
	\item accelerazione di ricerche scientifiche
\end{itemize}

\subsubsection*{Rischi}
\begin{itemize}
	\item impiego in applicazioni militari
	\item violazione della privacy
	\item bias nei dati e nei modelli
	\item perdita di posti di lavoro
	\item cybersecurity
	\item value alignment problem: serve allineare i valori dell'IA per evitare che i goal dell'IA sovrastino i valori umani,
	creando superintelligenze pericolose e fuori controllo
\end{itemize}
