\documentclass[a4paper]{article}
\usepackage[utf8]{inputenc} % standard unicode
\usepackage[italian]{babel} % corretta sillabazione in italiano
\usepackage{geometry} % per impostare margini e layout pagina
\usepackage{amssymb} % per l'ambiente matematico
\usepackage{amsmath} % per l'ambiente matematico
%\usepackage{amsthm} % per l'ambiente matematico (simbolo qed alla fine delle dimostrazioni)
\usepackage{enumitem} % per elenchi puntati
\usepackage{multirow} % per celle che si espandono su più righe
\usepackage{tabularx} % per tabelle con larghezza flessibile
\usepackage{booktabs} % per linee orizzontali tabelle
%\usepackage{hyperref} % per collegamenti
%\usepackage{dirtytalk} % per le ""

% per margini
\geometry{a4paper,left=25mm, right=25mm, bottom=25mm, top=30mm}

% per centrare testo nelle tabelleX
\renewcommand\tabularxcolumn[1]{m{#1}}

\newcommand\dom{\text{dom}}   % dominio
\newcommand\dist{\text{dist}} % distanza
\newcommand\intr{\text{int}}  % intorno
\newcommand\R{\mathbb{R}}     % R
\newcommand\Rd{\mathbb{R}^2}  % R^2
\newcommand\Rt{\mathbb{R}^3}  % R^3
\newcommand\Rn{\mathbb{R}^n}  % R^n
\newcommand\tc{\text{t.c.}}   % tale che
\newcommand\dt{\frac{d}{dt}}  % d/dt
\newcommand\dx{\frac{d}{dx}}  % d/dx
\newcommand\dy{\frac{d}{dy}}  % d/dy

\title{Appunti di fondamenti di analisi e probabilità}
\author{Giacomo Simonetto}
\date{Primo semestre 2024-25}

\begin{document}

% -------------------------------------- Copertina e indice ---------------------------------------
\maketitle
\begin{abstract}
	Appunti del corso di Fondamenti di analisi e probabilità della facoltà di Ingegneria Informatica dell'Università di Padova.
\end{abstract}

\newpage

\tableofcontents

\newpage

% --------------------------------------- Curve e sostegni ----------------------------------------
\section{Curve e sostegni}
\subsection{Introduzione sugli intorni}
\subsubsection*{Definizioni su palle e cubi}
\begin{itemize}[topsep=3pt, itemsep=0pt]
	\item[-] Norma di un vettore: \(\left| x \right| := \sqrt{{x_1}^2 + {x_2}^2 + \dots + {x_n}^2}\)
	\item[-] Distanza tra due punti: \(\dist(x,y) := \left| x-y \right|\)
	\item[-] Disuguaglianza triangolare: \(\left| x-y \right| \leq \left| x \right| + \left| y \right|\)
	\item[-] Disco o palla chiusa: \(B(p,r] := \left\{ x \in \Rn : \left| x-p \right| \leq r \right\}\)
	\item[-] Disco o palla aperta: \(B(p,r[ \; := \left\{ x \in \Rn : \left| x-p \right| < r \right\}\)
	\item[-] Bordo di una palla: \(\partial B(p,r] = \partial B(p,r[ \; := \left\{ x \in \Rn : \left| x-p \right| = r \right\}\)
	\item[-] Quadrato o cubo chiuso: \(Q(p,r] := \left\{ \left( x_1, \dots, x_n \right) \in \Rn : \left| x_1-p_1 \right| \leq r, \dots, \left| x_n-p_n \right| \leq r \right\}\)
	\item[-] Quadrato o cubo aperto: \(Q(p,r[ \; =: \left\{ \left( x_1, \dots, x_n \right) \in \Rn : \left| x_1-p_1 \right| < r, \dots, \left| x_n-p_n \right| < r \right\}\)
	\item[-] Bordo di un quadrato: \(\partial Q(p,r] = \partial Q(p,r[ \;:= \left\{ \left( x_1, \dots, x_n \right) \in \Rn : \left| x_1-p_1 \right| = r, \dots, \left| x_n-p_n \right| = r \right\}\)
\end{itemize}

\subsubsection*{Teorema di inclusione tra palle e cubi}
Ogni palla contiene un cubo di stesso centro e viceversa. \\
Fissato \(p \in \Rn\) e \(r > 0\) vale \(B(p,r] \subseteq Q(p,r]\) e \(Q(p,r] \subseteq B(p,r\sqrt{n}]\)

\subsubsection*{Definizione di intorno}
Un intorno di \(p \in \Rn\) è un insieme che contiene una palla centrata in \(p\). Per il teorema precedente, la proposizione vale
anche per i quadrati.

\subsubsection*{Definizione di punto interno ad un insieme}
Il punto \(p \in D\) è un punto interno all'insieme \(D\) se \(\exists \delta > 0 : B(p,\delta[ \; \subset D\). \\
L'insieme dei punti interni di un insieme \(D\) si indica con \(\intr(D)\).

\subsubsection*{Insieme aperto, chiuso, frontiera e chiusura}
\begin{itemize}[topsep=3pt, itemsep=0pt]
	\item[-] Un insieme è aperto se ogni suo punto è un punto interno: \(D = \intr(D)\)
	\item[-] Un insieme è chiuso se il suo complementare è aperto
	\item[] Osservazione: \(\varnothing\) e \(\Rn\) sono sia aperti che chiusi
	\item[-] La frontiera \(\partial D\) è l'insieme dei punti tali che ogni loro intorno interseca sia \(D\), sia \(\Rn \backslash D\)
	\item[-] La chiusura \(\overline{D}\) è il più piccolo insieme chiuso contente \(D\): \(\overline{D} = D \cup \partial D\)
\end{itemize}

\subsubsection*{Prodotto scalare}
Il prodotto scalare tra due vettori \(x = \left( x_1, \dots, x_n \right)\) e \(y = \left( y_1, \dots, y_n \right)\) di \(\Rn\)
è il numero reale definito come \(x \cdot y = x_1 y_1 + \dots + x_n y_n\). \\
Due vettori sono ortogonali se il loro prodotto scalare è 0.

\subsubsection*{Disuguaglianza di Cauchy-Schwarz}
Siano \(x\), \(y \in \Rn\), allora \(\left| x \cdot y \right| \leq \left| x \right| \left| y \right|\). Si ha l'uguaglianza se
solo se uno è multiplo dell'altro.

%\subsubsection*{Esempi di insiemi notevoli}
%\begin{itemize}[topsep=3pt, itemsep=0pt]
%	\item[-] ellisse in \(\Rd\) con centro in \(a,b\) e semiassi \(A\) e \(B\) \[\left\{ \left(x, y\right) : \frac{\left(x - a\right)^2}{A^2} + \frac{\left(y - b\right)^2}{B^2} = 1 \right\}\]
%	\item[-] retta in \(\Rd\) perpendicolare al vettore \(a,b\) \[ax + by + c = 0\]
%	\item[-] piano in \(\Rt\) perpendicolare al vettore \(a,b,c\) \[ax + by + cz = 0\]
%	\item[-] cilindro in \(\Rt\) di asse parallelo all'asse z passante per \(a,b,0\) \[\left(x-a\right)^2 + \left(y-b\right)^2 = r^2\]
%\end{itemize}

\newpage

\subsection{Funzioni vettoriali e curve}
\subsubsection*{Definizione di funzioni vettoriali, curve, sostegni di curve}
\begin{itemize}[topsep=3pt, itemsep=0pt]
	\item[-] Una funzione vettoriale è una funzione \(f: I_{intervallo} \subset \R \rightarrow \Rn, \; t \mapsto f(t) = \left(f_1(t), f_2(t), \dots f_n(t)\right)\)
	\item[-] Una curva (parametrica) è una funzione vettoriale in cui \(f_1(t), \dots f_n(t)\) sono continue in \(I = \left[a,b\right]\)
	\item[-] Il sostegno di una curva \(f\) è l'insieme immagine di \(f\): \(f\left(\left[a,b\right]\right) := \left\{f(t) : t \in \left[a,b\right]\right\} \subset \Rn\)
	\item[-] Una curva si dice cartesiana se è della forma \(f(t) = \left(t, h(t)\right)\) o \(f(t) = \left(h(t),t\right), t \in \left[a,b\right]\)
	\item[-] Una curva si dice chiusa se \(f(a) = f(b)\)
	\item[-] Una curva si dice semplice se \(f(t_1) \neq f(t_2), \; \forall t_1,t_2\) con \(a < t_1 < t_2 \leq b\), ovvero se non si interseca mai ad eccezione degli estremi
\end{itemize}

\subsubsection*{Curve e sostegni}
\begin{itemize}[topsep=3pt, itemsep=0pt]
	\item[-] Data una curva \(f(t)\), per ottenere il sostegno di tale curva, bisogna eliminare il parametro \(t\), passando dalla
	forma parametrica a quella cartesiana.
	\item[-] Viceversa se, dato un sostegno, si vuole ottenere una curva, bisogna introdurre una parametrizzazione del sostegno,
	passando dalla forma cartesiana a quella paremtrica.
\end{itemize}

\subsection{Limiti di funzioni vettoriali}
\subsubsection*{Definizione di limite (finito) in una e più dimensioni}
\begin{align*}
	\lim_{t \to t_0} f(t) = \ell \in \Rn \quad &\Rightarrow \quad \forall V \; \text{intorno di} \; \ell \; \exists U \; \text{intorno di} \; p \; \tc \; x \in U \setminus \left\{p\right\} \Rightarrow f(x) \in V \\
	&\Rightarrow \quad \forall \varepsilon > 0 \; \exists \delta > 0 \; \tc \; 0 < \left|t-t_0\right| < \delta \Rightarrow \left|f(t) - \ell\right| < \varepsilon
\end{align*}
\[\lim_{t \to t_0} (f_1(t), f_2(t), \dots f_n(t)) = (\ell_1, \ell_2, \dots \ell_n) \quad \Leftrightarrow \quad \lim_{t \to t_0} f_1(t) = \ell_1, \; \lim_{t \to t_0} f_2(t) = \ell_2, \dots \lim_{t \to t_0} f_n(t) = \ell_n\]

\subsubsection*{Continuità}
Una funzione è continua se \(\lim_{t \to t_0} f(t) = f(t_0)\). Nel caso di funzioni vettoriali, essa è continua se ogni sua componente
è continua.

\newpage


\section{Derivate, gradienti, tangenti, massimi e minimi}
\subsection{Derivate di funzioni vettoriali}
\subsubsection*{Definizione di derivata in una e più dimensioni}
Una funzione \(f\) è derivabile in \(t_0\) se esiste il limite finito del rapporto incrementale.
\[f'(t_0) := \lim_{t \to t_0} \frac{f(t)-f(t_0)}{t-t_0} = \ell \in \Rn\]
\[f(t) = \left(f_1(t), f_2(t), \dots f_n(t) \right) \quad \Leftrightarrow \quad f'(t_0) = \left( f_1'(t_0), f'_2(t_0), \dots f'_n(t_0)\right)\]

\subsubsection*{Retta tangente ad una curva}
Se \(f\) è derivabile in \(t_0\) e \(f'(t_0) \neq 0\):
\begin{itemize}[topsep=3pt, itemsep=0pt]
	\item[-] il vettore tangente alla curva è \(f'(t_0)\)
	\item[-] la retta tangente alla curva è \(\left\{ f(t_0) + f'(t_0) \lambda, \; \lambda \in \Rn \right\}\)
\end{itemize}
Si parla di tangenza alla curva e non al sostegno perché una funzione può passare per lo stesso punto in due momenti diversi. In
questo caso si avrebbero due tangenti diverse per uno stesso punto del sostegno, quando invece sarebbero due tangeti associate a
due valori diversi del parametro della curva.

\subsubsection*{Funzioni o curve differenziabili e approssimazioni di primo ordine}
Una funzione \(f(t)\) si dice differenziabile se vale (specialmente il limite):
\[f(t) = f(t_0) + f'(t_0) (t-t_0) + R(t) \qquad \lim_{t \to t_0} \frac{R(t)}{t-t_0} = 0\]

\subsubsection*{Regole di derivazione di curve}
Siano \(f,g : \R \to \Rn\) curve derivabili, \(\varphi,u : \R \to \R\) funzioni derivabili, \(\alpha \in \R\), allora:
\begin{itemize}[topsep=3pt, itemsep=0pt]
	\item[1.] \(\dt \; \left(\text{costante}\right) = 0\)
	\item[2.] \(\dt \; \left(\alpha f\right) = \alpha f'\)
	\item[3.] \(\dt \; \left(\varphi(t)f(t)\right) = \varphi'(t)f(t) + \varphi(t)f'(t)\)
	\item[4.] \(\dt \; \left(f + g\right) = f' + g'\)
	\item[5.] \(\dt \; \left(f \cdot g\right) = f' \cdot g + f \cdot g'\)
	\item[6.] \(\dt \; \left(f \circ u\right) = f'(u)u'\)
\end{itemize}

\subsection{Derivate direzionali e derivate parziali}
\subsubsection*{Definizione di derivata direzionale}
La derivata direzionale di \(f\) in un punto \(p\) lungo la direzione \(\vec{u}\) è definita come il limite, se esiste finito
del rapporto incrementale.
\begin{align*}
	D_{\vec{u}} f(p) = \partial_{\vec{u}} f(p) &:= \lim_{t \to 0} \frac{f(p + t \vec{u}) - f(p)}{t} = \ell \in \R \\
	&:= g'(0) \text{ con } g(t) = f(p + t\vec{u})
\end{align*}
Per trovare la derivata direzionale bisogna fare:
\begin{itemize}[topsep=3pt, itemsep=0pt]
	\item[1.] trovare la funzione \(g(t) = f(p+t\vec{u})\)
	\item[2.] calcolare la derivata \(g'(t)\)
	\item[3.] valutare la derivata per \(t \to 0\)
	\item[Oss] nel caso in cui la funzione \(f\) non sia continua in \(p\) e non è possibile definire esattamente \(g(t)\) e \(g'(t)\),
	è consigliato usare la definizione per calcolare \(\displaystyle g'(0) = \lim_{t \to 0} \frac{g(t) - g(0)}{t} = \lim_{t \to 0} \frac{f(p + t \vec{u}) - f(p)}{t}\)
\end{itemize}

\subsubsection*{Definizione di derivata parziale}
La i-esima derivata parziale di \(f\) in \(p\) è la derivata direzionale lungo \(\vec{e_i}\) di \(f(x_1, \dots x_n)\),
\[\partial_{x_i} f(p) = D_{\vec{e_i}} f(p) = \frac{d}{d x_i} f(p) \qquad \text{casi particolari: } \begin{aligned}
	\partial_x f(p) &= D_{(1,0)} f(p) = \dx f(p) \\
	\partial_y f(p) &= D_{(0,1)} f(p) = \dy f(p)
\end{aligned}\]

\subsubsection*{Continuità e derivate direzionali}
\begin{itemize}[topsep=3pt, itemsep=0pt]
	\item[-] Una funzione può non essere continua in un punto \(p\), ma avere lo stesso derivate parziali e direzionali. In questo
	caso si sfrutta la definzione di derivata per calcolarne il valore.
\end{itemize}

\subsubsection*{Proprietà delle derivate direzionali}
\begin{itemize}[topsep=3pt, itemsep=0pt]
	\item[1.] \(D_u (f(p) + g(p)) = D_u f(p) + D_u f(p)\)
	\item[2.] \(D_u (f(p)g(p)) = D_u f(p) \; g(p) + f(p) D_u g(p)\)
	\item[3.] \(D_u (c f(p)) = c D_u f(p)\)
	\item[4.] \(D_u (\varphi \circ p)(p)) = D_u \varphi(f(p)) = \varphi(f(p)) D_u f(p)\)
\end{itemize}

\subsection{Gradiente}
\subsubsection*{Definizione di gradiente}
\[\vec{\nabla} f(p) := \left( \partial_{x_1} f(p), \partial_{x_2} f(p), \dots \partial_{x_n} f(p)\right)\]

\subsubsection*{Proprietà del gradiente} % tra cui gradiente della norma
\begin{itemize}[topsep=3pt, itemsep=0pt]
	\item[1.] \(\vec{\nabla} (f(p) + g(p)) = \vec{\nabla} f(p) + \vec{\nabla} g(p)\)
	\item[2.] \(\vec{\nabla} (f(p) g(p)) = \vec{\nabla} f(p) \; g(p) + f(p) \vec{\nabla} g(p)\)
	\item[3.] \(\vec{\nabla} (c f(p)) = c \vec{\nabla} f(p)\)
	\item[4.] \(\vec{\nabla} (\varphi \circ f)(p) = \vec{\nabla} (\varphi(f(p))) = \varphi'(f(p)) \vec{\nabla} f(p)\)
	\item[5.] gradiente della norma: \(\displaystyle \vec{\nabla} \left|x\right| = \frac{x}{\left|x\right|}\)
\end{itemize}

\subsubsection*{Relazione tra gradiente e derivate direzionali in funzioni C1}
Una funzione \(f\) è di classe \(C^1\) in un aperto se:
\begin{itemize}[topsep=3pt, itemsep=0pt]
	\item[-] \(f\) è continua
	\item[-] \(f\) ha derivate parziali continue
\end{itemize}
In una funzione \(C^1\), le derivate direzionali e il gradiente hanno la seguente relazione:
\[\forall u \in \Rn \qquad D_u f(p) = \vec{\nabla} f(p) \cdot \vec{u} = \partial_{x_1} f(p) u_1 + \partial_{x_2} f(p) u_2 + \dots \partial_{x_n} f(p) u_n\]
Osservazione: se in una funzione generica la derivata parziale \(D_{(u_1, u_2)} f(p)\) non è esprimibile come combinazione
lineare delle derivate parziali, allora non è \(C^1\). In altre parole se non vale l'equazione \(D_u f(p) = \vec{\nabla} f(p) \cdot \vec{u}\), allora \(f \notin C^1\).

\subsubsection*{Direzione di massima e minima crescita}
Se \(f\) è \(C^1\) in un intorno di \(p\), la direzione lungo cui si ha la massima pendenza è la direzione del vettore gradiente.
Viceversa la direzione di minore crescita è opposta a quella di massima crescita.
\begin{align*}
	D_{u_{max}} f(p) &= \left| \vec{\nabla} f(p)\right| \qquad \quad \Leftrightarrow \qquad u_{max} = \frac{\vec{\nabla} f(p)}{\left| \vec{\nabla} f(p) \right|} \\
	D_{u_{min}} f(p) &= -\left| \vec{\nabla} f(p)\right| \qquad \Leftrightarrow \qquad u_{min} = -\frac{\vec{\nabla} f(p)}{\left| \vec{\nabla} f(p) \right|}
\end{align*}

\subsection{Spazio tangente e differenziabilità}
\subsubsection*{Spazio tangente}
Lo spazio tangente al grafico di una curva \(f(p)\) in un punto \(p\) è l'insieme dei punti: \\
(n.b.: \(x\) è un punto sul piano, come lo è anche \(p\), non è una coordinata)
\[\left\{ (x,z) \; \tc \; z = f(p) + \vec{\nabla} f(p) \cdot (x-p)\right\}\] 

\subsubsection*{Differenziabilità}
Una funzione è differenziabile in un punto \(p\) se la funzione \(f\) è approssimabile al piano tangente in \(p\) in un suo intorno
con un errore trascurabile. \(L(x)\) è la funzione affine detta linearizzazione di \(f\) in \(p\)
\begin{align*}
	f(x) &=  f(p) + \vec{\nabla} f(p) \cdot (x-p) + R(x), \qquad \lim_{x \to p} \frac{R(x)}{\left|x-p\right| = 0} \\
	L(x) &:= f(p) + \vec{\nabla} f(p) \cdot (x-p)
\end{align*}
Per sapere se una funzione è differenziabile bisogna controllare che il resto \(R(x) = o(\left|x-p\right|)\), cioè bisogna risolvere
il limite per \(x \to p\) e verificare che faccia 0.

\subsubsection*{Continuità di funzioni differenziabili}
Una funzione differenziabile in \(p\) è anche continua in \(p\). Quindi valgono le seguenti inclusioni.
\[\text{Funzioni con derivate parziali} \subset \text{Funzioni continue} \subset \text{Funzioni differenziabili} \subset \text{Funzioni } C^1\]

\subsubsection*{Gradiente di funzioni differenziabili}
Nelle funzioni differenziabili il gradiente vale:
\[D_u f(p) = \vec{\nabla} f(p) \cdot u\]

\subsubsection*{Regola della catena di derivate parziali}
Per la regola della catena (con funzione a due variabili e in generale a \(n\) variabili):
\begin{align*}
	\dt f(x(t),y(t)) &= \partial_x f(x(t,y(t))) \cdot x'(t) + \partial_y f(x(t),y(t)) \cdot y'(t) = \vec{\nabla} f(x(t),y(t)) \cdot (x'(t), y'(t)) \\
	\dt f(r(t)) &= \vec{\nabla} f(r(t)) \cdot r'(t) = \partial_{x_1} f(x(t)) {x_1}'(t) + \partial_{x_2} f(x(t)) {x_2}'(t) + \dots + \partial_{x_n} f(x(t)) {x_n}'(t)
\end{align*}

\subsubsection*{Gradiente e curve di livello}
Il gradiente è perpendicolare alle curve di livello. Nelle curve di livello vale:
\[f(r(t)) = \text{costante} \quad \Rightarrow \quad \dt f(r(t)) = 0 \quad \Rightarrow \quad \vec{\nabla} f(r(t)) \cdot r'(t) = 0 \quad \Rightarrow \quad \vec{\nabla} f(r(t)) \perp r'(t)\]
con \(r'(t)\) un vettore con la stessa direzione della tangete alla curva di livello, per un certo \(t\).

\newpage

\subsection{Derivate seconde, matrice Hessiana}
\subsubsection*{Derivate seconde}
La derivata parziale di secondo ordine è definita come:
\[\partial^2_{x_i,x_j} f(x) = \frac{\partial^2 f(x)}{\partial_{x_i} \partial_{x_j}} = \partial_{x_i} (\partial_{x_j} f(x))\]

\subsubsection*{Matrice Hessiana}
\[\text{Hess}f(x) := \left( \begin{matrix}
	\partial^2_{{x_1}^2} f(x) & \cdots & \partial^2_{{x_1,x_n}} f(x) \\
	\vdots & \ddots & \vdots \\
	\partial^2_{{x_n,x_1}} f(x) & \cdots & \partial^2_{{x_n}^2} f(x) \\
\end{matrix} \right) \]

\subsubsection*{Teorema di Schwarz}
Data una funzione \(f\) di classe \(C^2\) (ovvero con derivate parziali doppie continue), allora vale:
\[\partial^2_{x_i,x_j} f(x) = \partial^2_{x_j,x_i} f(x) \qquad \forall i,j\]
Per questo principio, la matrice hessiana di funzioni \(C^2\) è una matrice simmetrica.

\subsection{Massimi e minimi}
\subsubsection*{Definizione di massimi, minimi e punti di sella}
\begin{itemize}[topsep=3pt, itemsep=0pt]
	\item[-] il punto \(p\) è punto di minimo assoluto se \(f(x) \geq f(p) \quad \forall x \in D\)
	\item[-] il punto \(p\) è punto di massimo assoluto se \(f(x) \leq f(p) \quad \forall x \in D\)
	\item[-] il punto \(p\) è punto di minimo relativo se \(\exists U_p \; \text{intorno di} \; p \; \tc \; f(x) \geq f(p) \quad \forall x \in U_p \cap D\)
	\item[-] il punto \(p\) è punto di massimo relativo se \(\exists U_p \; \text{intorno di} \; p \; \tc \; f(x) \leq f(p) \quad \forall x \in U_p \cap D\)
	\item[-] il punto \(p\) è punto di sella se \(\forall U_p \; \text{intorno di} \; p \; \exists x,y \in U_p \cap D \; \tc \; f(x) < f(p) < f(y)\)
\end{itemize}

\subsubsection*{Regola di Fermat e punti critici interni}
\begin{itemize}[topsep=3pt, itemsep=0pt]
	\item[-] Se \(f\) derivabile rispetto a \(\vec{u}\) in \(p\), con \(p\) punto di massimo o minimo locale \textbf{interno}
	al dominio, allora vale che \(D_u f(p) = 0\), per cui \(\vec{\nabla} f(p) = 0\)
	\item[-] I punti interni al dominio per cui \(\vec{\nabla} f(p) = 0\) si dicono punti critici e possono essere classificati
	come punti di massimo locale, di minimo locale o punti di sella.
	\item[-] Per trovare i punti critici \textbf{interni} al dominio, per definizione, bisogna risolvere \(\vec{\nabla} f(p) = 0\).
\end{itemize}

\subsubsection*{Criterio dell'Hessiana}
Data una funzione \(f\) di classe \(C^2\) e \(p\) punto critico interno al dominio, allora:
\begin{itemize}[topsep=3pt, itemsep=0pt]
	\item[-] se \(\det H_f(p) > 0\) e \(\partial^2_{x,x}f(p) > 0\) allora \(p\) è minimo locale
	\item[-] se \(\det H_f(p) > 0\) e \(\partial^2_{x,x}f(p) < 0\) allora \(p\) è massimo locale
	\item[-] se \(\det H_f(p) < 0\) allora \(p\) è punto di sella
	\item[-] se \(\det H_f(p) = 0\) allora non si può concludere nulla
\end{itemize}

\subsubsection*{Massimi e minimi assoluti su domini chiusi e limitati}
Per Weierstrass una funzione continua in un dominio chiuso e limitato ammette massimo e minimo assoluto. Per trovare il massimo e minimo
assoluto di una funzione bisogna controllare i punti critici interni al dominio \(D\) e i punti sul bordo del dominio \(\partial D\).

\begin{itemize}[topsep=3pt, itemsep=0pt]
	\item[-] \(\max f = \max \left\{ f(p) \; \tc \; p \in \left\{\text{punti di massimo interni a } D\right\} \cup \left\{\text{punti di massimo su } \partial D \right\} \right\}\)
	\item[-] \(\min f = \min \left\{ f(p) \; \tc \; p \in \left\{\text{punti di minimo interni a } D\right\} \cup \left\{\text{punti di minimo su } \partial D \right\} \right\}\)
\end{itemize}


\end{document}
