\documentclass[a4paper]{article}
\usepackage[utf8]{inputenc} % standard unicode
\usepackage[italian]{babel} % corretta sillabazione in italiano
\usepackage{geometry} % per impostare margini e layout pagina
\usepackage{amssymb} % per l'ambiente matematico
\usepackage{amsmath} % per l'ambiente matematico
%\usepackage{amsthm} % per l'ambiente matematico (simbolo qed alla fine delle dimostrazioni)
\usepackage{enumitem} % per elenchi puntati
\usepackage{multirow} % per celle che si espandono su più righe
\usepackage{tabularx} % per tabelle con larghezza flessibile
\usepackage{booktabs} % per linee orizzontali tabelle
%\usepackage{hyperref} % per collegamenti
%\usepackage{dirtytalk} % per le ""

% per margini
\geometry{a4paper,left=25mm, right=25mm, bottom=25mm, top=30mm}

% per centrare testo nelle tabelleX
\renewcommand\tabularxcolumn[1]{m{#1}}

\newcommand\dom{\text{dom}}   % dominio
\newcommand\dist{\text{dist}} % distanza
\newcommand\intr{\text{int}}  % intorno
\newcommand\cont[2]{C^{(#1)} \left({#2}\right)}
\newcommand\R{\mathbb{R}}     % R
\newcommand\Rd{\mathbb{R}^2}  % R^2
\newcommand\Rt{\mathbb{R}^3}  % R^3
\newcommand\Rn{\mathbb{R}^n}  % R^n
\newcommand\tc{\text{t.c.}}   % tale che
\newcommand\dt{\frac{d}{dt}}  % d/dt

\title{Appunti di fondamenti di analisi e probabilità}
\author{Giacomo Simonetto}
\date{Primo semestre 2024-25}

\begin{document}

% -------------------------------------- Copertina e indice ---------------------------------------
\maketitle
\begin{abstract}
	Appunti del corso di Fondamenti di analisi e probabilità della facoltà di Ingegneria Informatica dell'Università di Padova.
\end{abstract}

\newpage

\tableofcontents

\newpage

% --------------------------------------- Curve e sostegni ----------------------------------------
\section{Curve e sostegni}
\subsection{Introduzione sugli intorni}
\subsubsection*{Definizioni su palle e cubi}
\begin{itemize}[topsep=3pt, itemsep=0pt]
	\item[-] Norma di un vettore: \(\left| x \right| := \sqrt{{x_1}^2 + {x_2}^2 + \dots + {x_n}^2}\)
	\item[-] Distanza tra due punti: \(\dist(x,y) := \left| x-y \right|\)
	\item[-] Disuguaglianza triangolare: \(\left| x-y \right| \leq \left| x \right| + \left| y \right|\)
	\item[-] Disco o palla chiusa: \(B(p,r] := \left\{ x \in \Rn : \left| x-p \right| \leq r \right\}\)
	\item[-] Disco o palla aperta: \(B(p,r[ \; := \left\{ x \in \Rn : \left| x-p \right| < r \right\}\)
	\item[-] Bordo di una palla: \(\partial B(p,r] = \partial B(p,r[ \; := \left\{ x \in \Rn : \left| x-p \right| = r \right\}\)
	\item[-] Quadrato o cubo chiuso: \(Q(p,r] := \left\{ \left( x_1, \dots, x_n \right) \in \Rn : \left| x_1-p_1 \right| \leq r, \dots, \left| x_n-p_n \right| \leq r \right\}\)
	\item[-] Quadrato o cubo aperto: \(Q(p,r[ \; =: \left\{ \left( x_1, \dots, x_n \right) \in \Rn : \left| x_1-p_1 \right| < r, \dots, \left| x_n-p_n \right| < r \right\}\)
	\item[-] Bordo di un quadrato: \(\partial Q(p,r] = \partial Q(p,r[ \;:= \left\{ \left( x_1, \dots, x_n \right) \in \Rn : \left| x_1-p_1 \right| = r, \dots, \left| x_n-p_n \right| = r \right\}\)
\end{itemize}

\subsubsection*{Teorema di inclusione tra palle e cubi}
Ogni palla contiene un cubo di stesso centro e viceversa. \\
Fissato \(p \in \Rn\) e \(r > 0\) vale \(B(p,r] \subseteq Q(p,r]\) e \(Q(p,r] \subseteq B(p,r\sqrt{n}]\)

\subsubsection*{Definizione di intorno}
Un intorno di \(p \in \Rn\) è un insieme che contiene una palla centrata in \(p\). Per il teorema precedente, la proposizione vale
anche per i quadrati.

\subsubsection*{Definizione di punto interno ad un insieme}
Il punto \(p \in D\) è un punto interno all'insieme \(D\) se \(\exists \delta > 0 : B(p,\delta[ \; \subset D\). \\
L'insieme dei punti interni di un insieme \(D\) si indica con \(\intr(D)\).

\subsubsection*{Insieme aperto, chiuso, frontiera e chiusura}
\begin{itemize}[topsep=3pt, itemsep=0pt]
	\item[-] Un insieme è aperto se ogni suo punto è un punto interno: \(D = \intr(D)\)
	\item[-] Un insieme è chiuso se il suo complementare è aperto
	\item[] Osservazione: \(\varnothing\) e \(\Rn\) sono sia aperti che chiusi
	\item[-] La frontiera \(\partial D\) è l'insieme dei punti tali che ogni loro intorno interseca sia \(D\), sia \(\Rn \backslash D\)
	\item[-] La chiusura \(\overline{D}\) è il più piccolo insieme chiuso contente \(D\): \(\overline{D} = D \cup \partial D\)
\end{itemize}

\subsubsection*{Prodotto scalare}
Il prodotto scalare tra due vettori \(x = \left( x_1, \dots, x_n \right)\) e \(y = \left( y_1, \dots, y_n \right)\) di \(\Rn\)
è il numero reale definito come \(x \cdot y = x_1 y_1 + \dots + x_n y_n\). \\
Due vettori sono ortogonali se il loro prodotto scalare è 0.

\subsubsection*{Disuguaglianza di Cauchy-Schwarz}
Siano \(x\), \(y \in \Rn\), allora \(\left| x \cdot y \right| \leq \left| x \right| \left| y \right|\). Si ha l'uguaglianza se
solo se uno è multiplo dell'altro.

%\subsubsection*{Esempi di insiemi notevoli}
%\begin{itemize}[topsep=3pt, itemsep=0pt]
%	\item[-] ellisse in \(\Rd\) con centro in \(a,b\) e semiassi \(A\) e \(B\) \[\left\{ \left(x, y\right) : \frac{\left(x - a\right)^2}{A^2} + \frac{\left(y - b\right)^2}{B^2} = 1 \right\}\]
%	\item[-] retta in \(\Rd\) perpendicolare al vettore \(a,b\) \[ax + by + c = 0\]
%	\item[-] piano in \(\Rt\) perpendicolare al vettore \(a,b,c\) \[ax + by + cz = 0\]
%	\item[-] cilindro in \(\Rt\) di asse parallelo all'asse z passante per \(a,b,0\) \[\left(x-a\right)^2 + \left(y-b\right)^2 = r^2\]
%\end{itemize}

\newpage

\subsection{Funzioni vettoriali e curve}
\subsubsection*{Definizione di funzioni vettoriali, curve, sostegni di curve}
\begin{itemize}[topsep=3pt, itemsep=0pt]
	\item[-] Una funzione vettoriale è una funzione \(f: I_{intervallo} \subset \R \rightarrow \Rn, \; t \mapsto f(t) = \left(f_1(t), f_2(t), \dots f_n(t)\right)\)
	\item[-] Una curva (parametrica) è una funzione vettoriale in cui \(f_1(t), \dots f_n(t)\) sono continue in \(I = \left[a,b\right]\)
	\item[-] Il sostegno di una curva \(f\) è l'insieme immagine di \(f\): \(f\left(\left[a,b\right]\right) := \left\{f(t) : t \in \left[a,b\right]\right\} \subset \Rn\)
	\item[-] Una curva si dice cartesiana se è della forma \(f(t) = \left(t, h(t)\right)\) o \(f(t) = \left(h(t),t\right), t \in \left[a,b\right]\)
	\item[-] Una curva si dice chiusa se \(f(a) = f(b)\)
	\item[-] Una curva si dice semplice se \(f(t_1) \neq f(t_2), \; \forall t_1,t_2\) con \(a < t_1 < t_2 \leq b\), ovvero se non si interseca mai ad eccezione degli estremi
\end{itemize}

\subsubsection*{Curve e sostegni}
\begin{itemize}[topsep=3pt, itemsep=0pt]
	\item[-] Data una curva \(f(t)\), per ottenere il sostegno di tale curva, bisogna eliminare il parametro \(t\), passando dalla
	forma parametrica a quella cartesiana.
	\item[-] Viceversa se, dato un sostegno, si vuole ottenere una curva, bisogna introdurre una parametrizzazione del sostegno,
	passando dalla forma cartesiana a quella paremtrica.
\end{itemize}

\subsection{Limiti di funzioni vettoriali}
\subsubsection*{Definizione di limite (finito) in una e più dimensioni}
\begin{align*}
	\lim_{t \to t_0} f(t) = \ell \in \Rn \quad &\Rightarrow \quad \forall V \; \text{intorno di} \; \ell \; \exists U \; \text{intorno di} \; p \; \tc \; x \in U \setminus \left\{p\right\} \Rightarrow f(x) \in V \\
	&\Rightarrow \quad \forall \varepsilon > 0 \; \exists \delta > 0 \; \tc \; 0 < \left|t-t_0\right| < \delta \Rightarrow \left|f(t) - \ell\right| < \varepsilon
\end{align*}
\[\lim_{t \to t_0} (f_1(t), f_2(t), \dots f_n(t)) = (\ell_1, \ell_2, \dots \ell_n) \quad \Leftrightarrow \quad \lim_{t \to t_0} f_1(t) = \ell_1, \; \lim_{t \to t_0} f_2(t) = \ell_2, \dots \lim_{t \to t_0} f_n(t) = \ell_n\]

\subsubsection*{Continuità}
Una funzione è continua se \(\lim_{t \to t_0} f(t) = f(t_0)\). Nel caso di funzioni vettoriali, essa è continua se ogni sua componente
è continua.

\subsection{Derivate di funzioni vettoriali}
\subsubsection*{Definizione di derivata in una e più dimensioni}
Una funzione \(f\) è derivabile in \(t_0\) se esiste il limite finito del rapporto incrementale 
\[f'(t_0) := \lim_{t \to t_0} \frac{f(t)-f(t_0)}{t-t_0} = \ell \in \Rn\]
\[f(t) = \left(f_1(t), f_2(t), \dots f_n(t) \right) \quad \Leftrightarrow \quad f'(t_0) = \left( f_1'(t_0), f'_2(t_0), \dots f'_n(t_0)\right)\]

\subsubsection*{Retta tangente ad una curva}
Se \(f\) è derivabile in \(t_0\) e \(f'(t_0) \neq 0\):
\begin{itemize}[topsep=3pt, itemsep=0pt]
	\item[-] il vettore tangente alla curva è \(f'(t_0)\)
	\item[-] la retta tangente alla curva è \(\left\{ f(t_0) + f'(t_0) \lambda, \; \lambda \in \Rn \right\}\)
\end{itemize}
Si parla di tangenza alla curva e non al sostegno perché una funzione può passare per lo stesso punto in due momenti diversi. In
questo caso si avrebbero due tangenti diverse per uno stesso punto del sostegno, quando invece sarebbero due tangeti associate a
due valori diversi del parametro della curva.

\subsubsection*{Funzioni o curve differenziabili e approssimazioni di primo ordine}
Una funzione \(f(t)\) si dice differenziabile se vale (specialmente il limite):
\[f(t) = f(t_0) + f'(t_0) (t-t_0) + R(t) \qquad \lim_{t \to t_0} \frac{R(t)}{t-t_0} = 0\]

\subsubsection*{Regole di derivazione di curve}
Siano \(f,g : \R \to \Rn\) curve derivabili, \(\varphi,u : \R \to \R\) funzioni derivabili, \(\alpha \in \R\), allora:
\begin{itemize}[topsep=3pt, itemsep=0pt]
	\item[1.] \(\dt \; \left(\text{costante}\right) = 0\)
	\item[2.] \(\dt \; \left(\alpha f\right) = \alpha f'\)
	\item[3.] \(\dt \; \left(\varphi(t)f(t)\right) = \varphi'(t)f(t) + \varphi(t)f'(t)\)
	\item[4.] \(\dt \; \left(f + g\right) = f' + g'\)
	\item[5.] \(\dt \; \left(f \cdot g\right) = f' \cdot g + f \cdot g'\)
	\item[6.] \(\dt \; \left(f \circ u\right) = f'(u)u'\)
\end{itemize}

\subsection{Derivate direzionali e derivate parziali}
\subsubsection*{Definizione di derivata direzionale}
\subsubsection*{Continuità e derivate direzionali}
\subsubsection*{Proprietà delle derivate direzionali}
\subsubsection*{Definizione di derivata parziale}

\subsection{Gradiente}
\subsubsection*{Definizione di gradiente}
\subsubsection*{Proprietà del gradiente} % tra cui gradiente della norma
\subsubsection*{Funzioni C1}
\subsubsection*{Gradiente e derivate direzionali}
\subsubsection*{Direzione di massima e minima crescita}

\subsection{Spazio tangente e differenziabilità}
\subsubsection*{Spazio tangente}
\subsubsection*{Differenziabilità}
\subsubsection*{Continuità di funzioni differenziabili}
\subsubsection*{Gradiente di funzioni differenziabili}
\subsubsection*{Regola della catena di derivate parziali}
\subsubsection*{Gradiente e curve di livello}

\subsection{Derivate seconde, matrice Hessiana}
\subsubsection*{Derivate seconde}
\subsubsection*{Matrice Hessiana}
\subsubsection*{Teorema di Schwarz}

\subsection{Massimi e minimi}
\subsubsection*{Definizione di massimi, minimi e punti di sella}
\subsubsection*{Regola di Fermat}
\subsubsection*{Ricerca dei punti critici}
\subsubsection*{Criterio dell'Hessiana}

\end{document}
