\documentclass[a4paper]{article}
\usepackage[utf8]{inputenc} % standard unicode
\usepackage[italian]{babel} % corretta sillabazione in italiano
\usepackage{geometry} % per impostare margini e layout pagina
\usepackage{amssymb} % per l'ambiente matematico
\usepackage{amsmath} % per l'ambiente matematico
\usepackage{enumitem} % per elenchi puntati
\usepackage{multirow} % per celle che si espandono su più righe
\usepackage{tabularx} % per tabelle con larghezza flessibile
\usepackage{booktabs} % per linee orizzontali tabelle
\usepackage{hyperref} % per collegamenti
\usepackage{dirtytalk} % per le ""
\usepackage{circuitikz} % per circuiti elettrici
\usepackage{cancel} % per barrare testo

% per margini
\geometry{a4paper,left=25mm, right=25mm, bottom=25mm, top=30mm}

% per centrare testo nelle tabelleX
\renewcommand\tabularxcolumn[1]{m{#1}}

% per elenchi puntati
\setlist[itemize]{label=-, partopsep=0pt, topsep=3pt, itemsep=0pt}

% espressioni matematiche
\newcommand\nab{\vec{\nabla}} % nabla

% --- altro che si può eliminare ---
% percorso delle immagini da inserire
%\graphicspath{ {./ } }

% parte funzione reale e parte immaginaria
%\newcommand\Real{\text{Re}}
%\newcommand\Img{\text{Im}}

% versori
%\newcommand\ux{\vec{u}_x}
%\newcommand\uy{\vec{u}_y}
%\newcommand\uz{\vec{u}_z}
%\newcommand\uxp{\vec{u}_{x'}}
%\newcommand\uyp{\vec{u}_{y'}}
%\newcommand\uzp{\vec{u}_{z'}}
%\newcommand\ur{\vec{u}_r}
%\newcommand\uv{\vec{u}_v}
%\newcommand\un{\vec{u}_n}
%\newcommand\ug{\vec{u}_\gamma}
%\newcommand\uper{\vec{u}_\perp}
%\newcommand\upar{\vec{u}_\parallel}

% forza generica
%\newcommand\ft{\vec{F}\left(\vec{\gamma}(t), \; \dt \vec{\gamma}(t), \; t\right)}
%\newcommand\ftau{\vec{F}\left(\vec{\gamma}(\tau), \; \dtau \vec{\gamma}(\tau), \; \tau\right)}

% derivata
%\newcommand\dt{\frac{d}{dt}\,}
%\newcommand\dtau{\frac{d}{d\tau}\,}
%\newcommand\dts{\frac{d^2}{dt^2}\,}

% modulo vettore
%\newcommand\vmod[1]{\left|\left|{#1}\right|\right|}

\title{Appunti di Teoria dei circuiti}
\author{Giacomo Simonetto}
\date{Secondo semestre 2024-25}

\begin{document}

% -------------------------------------- Copertina e indice ---------------------------------------
\maketitle
\begin{abstract}
	Appunti del corso di Teoria dei circuiti della facoltà di Ingegneria Informatica dell'Università di Padova.
\end{abstract}

\newpage

\tableofcontents

\newpage

\section{Introduzione alla teoria dei circuiti}
\subsubsection*{Definizione di circuito}
Un circuito elettrico è un insieme di dispositivi elettrici interconnessi, deputati alla produzione, trasmissione ed utiizzazione
dell'energia elettrica.

\subsubsection*{Equazioni di Maxwell}
È possibile risolvere un circuito attraverso le equazioni di Maxwell, ma si otterrebbe un sistema troppo complesso da gestire e da
risolvere, per cui si utilizzano approssimazioni e modelli definiti dalla teoria dei circuiti.

%\begin{align*}
%	\nab \cdot \vec{D} = \rho & \quad \text{Flusso del campo elettrico - Gauss} \\
%	\nab \cdot \vec{B} = 0 & \quad \text{Flusso del campo magnetico - Gauss} \\
%	\nab \times \vec{E} = - \frac{\partial \vec{B}}{\partial t} & \quad \text{Circuitazione del campo elettrico - Faraday-Neumann-Lenz} \\
%	\nab \times \vec{H} = \vec{J} + \frac{\partial \vec{D}}{\partial t} & \quad \text{Circuitazione del campo magnetico - Ampère-Maxwell} \\
%	\nab \cdot \vec{J} = -\frac{\partial \rho}{\partial t} & \quad \text{Continuità} \\
%	\vec{D} = \varepsilon \vec{E} \qquad \vec{B} = \mu \vec{H} & \quad \text{Relazioni tra i vari campi}\\
%\end{align*}

\subsubsection*{Modello zero-dimensionale}
Il modello zero-dimensionale non tiene conto di cosa avviene all'interno dei componenti elettrici, ma solo di come interagiscono
tra di loro. In altre parole viene trascurata la loro dimensione.

\subsubsection*{Grandezze fisiche}
Le grandezze fisiche utilizzate sono: tensione, corrente, potenza, energia e frequenza

\subsubsection*{Modello a parametri concentrati}
Il modello a parametri concentrati prevede che:
\begin{itemize}
	\item[1.] i componenti RLC sono idealizzati e considerati puntiformi (modello zero-dimensionale)
	\item[2.] tensioni e correnti dipendono dal tempo e non dallo spazio: si può evitare di considerare eventuali propagazioni
	elettromagnetiche
	\item[3.] l'interazione tra componenti avviene solo attraverso connessioni elettriche
\end{itemize}
Il suo scopo è di:
\begin{itemize}
	\item analizzare i comportamenti di tensioni e correnti (flussi di potenza)
	\item prevedere comportamenti dei dispositivi reali mediante modelli semplificati
	\item progettare e ottimizzare sistemi elettrici
\end{itemize}

\subsubsection*{Validità}
La teoria dei circuiti è valida se la dimensione del circuito è inferiore alla lunghezza d'onda del segnale che circola all'interno:
\begin{itemize}
	\item corrente alternata di rete \(\rightarrow\) 50 Hz \(\rightarrow\) \(\lambda\) = 6000 km
	\item radiofrequenza \(\rightarrow\) 100 MHz \(\rightarrow\) \(\lambda\) = 3 m
	\item microonde \(\rightarrow\) 10 GHz \(\rightarrow\) \(\lambda\) = 3 cm (limite della TdC)
\end{itemize}

\subsubsection*{Tipi di circuiti}
\begin{itemize}
	\item circuiti elettrici di segnale, lavorano con mW
	\item circuiti elettrici di potenza, lavorano con kW
\end{itemize}

\subsubsection*{Flusso e trasmissione di energia}
Per flusso di energia si intende come viene utilizzata la potenza in un circuito. La trasmissione di energia può avvenire in due
modi: attraverso onde elettromagnetiche (radio, antenne, \dots) o per conduzione (linee elettriche).

\newpage

\section{Interpretazione fisica dell'elettrostatica}
\subsection{Campi e grandezze fisiche}
\subsubsection*{Campo fisico}
Un campo fisico è la distribuzione su un volume o su una superficie di una certa grandezza fisica rappresentabile tramite vettore
o scalare. I campi fisici di grandezze scalari si dicono campi scalari, mentre i campi fisici di grandezze vettoriali si dicono
campi vettoriali.

\subsubsection*{Grandezze fisiche}
Una grandezza fisica è una quantità misurabile di un oggetto. Il processo di misura consiste nel comparare una quantità campione
(detta unità di misura) con l'oggetto da misurare. Le grandezze fondamentali del Sistema Internazionale sono: m, kg, s, K, A, cd,
mol.

\subsection{Carica elettrica e densità di carica}
\subsubsection*{Carica elettrica}
\begin{itemize}
	\item la quantità di carica è una grandezza che misura la carica elettrica di un oggetto
	\item si osserva che esiste una forza che dipende dalla quantità di carica dei corpi e può essere attrattiva tra corpi con
	cariche di segno opposto o repulsiva tra corpi con cariche dello stesso segno
	\item la carica è quantizzata con quanto \(e = 1.6 \cdot 10^{-19} \; C\)
\end{itemize}

\subsubsection*{Densità di carica}
La carica di una distribuzione è data da \(\displaystyle q = \int_V \rho d\tau\), ovvero la somma complessiva delle cariche positive
e negative di un corpo:
\begin{itemize}
	\item densità volumica: \(\displaystyle \rho(P,t) = [C_{oulomb}/m^3] = \lim_{V \to 0} \frac{q}{V}, \quad q = \int_V \rho(P,t) d\tau\)
	\item densità superficiale: \(\displaystyle \sigma(P,t) = [C_{oulomb}/m^2] = \lim_{\Sigma \to 0} \frac{q}{\Sigma}, \quad q = \int_\Sigma \sigma(P,t) d\Sigma\)
\end{itemize}

\subsection{Corrente elettrica e densità di corrente}
\subsubsection*{Densità di corrente}
Si genera per conduzione elettrica attraverso due modi:
\begin{itemize}
	\item corrente di conduzione: moto delle cariche libere (es. nei metalli)
	\item corrente di convezione: moto delle cariche libere e/o vincolate (es. soluzioni elettrolitiche)
\end{itemize}
\[\vec{J}(P,t) = \rho^+ + v_d^+ + \rho^- + v_d^-  \qquad \begin{cases}
	\rho^+ \;\rightarrow\; \text{densità delle cariche positive} \\
	v_d^+ \;\rightarrow\; \text{velocità di deriva delle cariche positive} \\
	\rho^- \;\rightarrow\; \text{densità delle cariche negative} \\
	v_d^- \;\rightarrow\; \text{velocità di deriva delle cariche negative}
\end{cases}\]

\subsubsection*{Corrente}
La corrente è la quantità di cariche che attraversano una superficie in un'unità di tempo. Dipende dalla superficie e dal suo
orientamento. Non dipende dal resto dello spazio. Se si inverte l'orientamento della superficie o il riferimento, il segno della
corrente si inverte. Si misura in Ampère \([A_{mpere}] = [C_{oulomb}/s]\).
\[i(t) = \int_\Sigma \vec{J}(P,t) \cdot d\vec{\Sigma} \qquad \Leftrightarrow \qquad i(t) = \lim_{\Delta t \to 0} \frac{\Delta q_{\text{ attraverso }\Sigma}(t)}{\Delta t}\]
In caso di conduttori filiformi (dove \(\Sigma \ll \text{lunghezza}\)), vale \(i(t) = \vec{J} \cdot \vec{\Sigma}\)

\subsubsection*{Conservazione della carica e continuità della corrente}
La carica elettrica non si crea, non si distrugge, si conserva sempre.
\begin{align*}
	&q_\text{interna}(t + \Delta t) = q_\text{interna}(t) + \Delta q_\text{uscente} \\
	&i_\text{uscente}(t) = \lim_{\Delta t \to 0} \frac{\Delta q_\text{uscente}}{\Delta t} = -\frac{d q_\text{entrante}}{dt} = - i_\text{entrante}
\end{align*}
\begin{itemize}
	\item la variazione di carica corrisponde ad una corrente
	\item in assenza di corrente, la carica non varia
	\item la carica entrante è pari a quella uscente (in modulo)
\end{itemize}

\subsubsection*{Corrente solenoidale}
La corrente si dice solenoidale quando:
\begin{itemize}
	\item si è in regime stazionario: non si hanno accumuli o prelievi di carica in nessun punto del volume, la carica entrante
	e quella uscente sono uguali e il campo \(\vec{J}\) forma linee di flusso chiuse
	\item in regioni di carica nulla: \(\rho = 0\) ad esempio nei metalli
\end{itemize}
si è in regime stazionario

\subsubsection*{Tubo di flusso}
Il tubo di flusso è un conduttore rivestito da materiale isolante che può essere attraversato da corrente. In condizioni stazionarie
(con campo di corrente solenoidale) si ha che la corrente \(i_1\) attraverso una superficie \(\Sigma_1\) è uguale alla corrente
\(i_2\) attraverso una superficie \(\Sigma_2\). Ovvero non si hanno perdite di corrente: \(i_\text{uscente} = 0\).

\subsubsection*{Amperometro}
L'amperometro è uno strumento per misurare la corrente in un circuito. Il verso del sistema è dal \(+\) al \(-\) (ovvero la corrente
entra dal connettore \(+\) ed esce dal connettore \(-\)). Si usa in serie al circuito. Un amperometro si dice ideale se non influisce
sul circuito e se la misura avviene senza ritardi.

\begin{center}
	\begin{circuitikz}
		\draw (0,0) -- (0.21,0) ;
		\draw (0.42,0) [circ,thick] circle(0.2);
		\draw (1,0) [circ, thick] circle(0.38);
		\draw (1.58,0) [circ, thick] circle(0.2);
		\draw (1.78,0) -- (2,0) ;
		\draw[->] (0.6,-0.7) -- (1.4,-0.7);
		\node[] at (1,0) {A} ;
		\node[] at (0.42,0) {+} ;
		\node[] at (1.58,-0.02) {--} ;
		\node[] at (1,-0.9) {i} ;
	\end{circuitikz}
\end{center}

\newpage

\subsection{Campo elettrostatico}
\subsubsection*{Legge di Coulomb e campo elettrostatico}
Il campo elettrostatico si definisce a partire dalla forza di Coulomb, per questo è anche chiamato campo coulombiano.
\[\vec{F}_{1,2} = \frac{1}{4 \pi \varepsilon_0} \frac{q_1 \, q_2}{{r_{1,2}}^2} \hat{u}_{1,2} \qquad \qquad
\vec{F}_\text{elettr} = q \vec{E} \qquad \qquad
\vec{E} = \frac{\vec{F}}{q} = \frac{1}{4 \pi \varepsilon_0} \frac{q}{r^2} \hat{u}_{1,2} = [N/C_{oulob}] = [V_{olt}/m]\]
Il campo elettrostatico è additivo: \(\displaystyle \vec{E}(P) = \frac{1}{4 \pi \varepsilon_0} \sum_{k=1}^{n} \frac{q_k}{{r_{PO_k}}^2} \vec{u}_k(P) \qquad 
\vec{E}(P) = \frac{1}{4 \pi \varepsilon_0} \int_V \frac{\rho}{r^2} \vec{u}_k d\tau\)

\subsubsection*{Campo elettrostatico nei conduttori}
Un conduttore è un materiale che conduce corrente. Le cariche sono libere di muoversi e, muovendosi, generano una corrente. In
condizione di equilibrio il campo all'interno è nullo, ovvero non c'è nessuna forza che agisce sulle cariche e le cariche sono
ferme (altrimenti non ci sarebbe equilibrio).

\subsubsection*{Campo elettrostatico nei dielettrici - isolanti}
Un dielettrico o isolante è un materiale che non conduce corrente. Le cariche sono bloccate a meno di piccoli spostamenti responsabili
della polarizzazione dei dielettrici. I dielettrici possono essere:
\begin{itemize}
	\item omogenei: se \(\varepsilon\) non dipende dalla posizione
	\item lineari: se \(\varepsilon\) non dipende dal modulo del campo elettrico \(||\vec{E}||\)
	\item isotropi: se \(\varepsilon\) non dipende dalla direzione del campo elettrico \(\vec{u} = \vec{E} \big/ ||{\vec{E}}||\)
\end{itemize}
Un dielettrico omogeneo, lineare e isotropo si dice uniforme e per esso valgono tutte le leggi viste finora.

\subsubsection*{Permittività dielettrica di un mezzo}
La permittività dielettrica di un mezzo indica come tale mezzo reagisce al campo elettrico:
\[\varepsilon = \varepsilon_\text{relativa del mezzo} \cdot \varepsilon_\text{0 - nel vuoto} = [F_{araday}/m] = [{C_{oulomb}}^2/J] \qquad \quad \varepsilon > \varepsilon_0 \;\; \varepsilon_r > 1\]

\subsubsection*{Campo elettrico conservativo}
Il campo elettrostatico è conservativo ovvero:
\[\oint_\mathcal{L} = \vec{E} \cdot d\vec{l} = 0 \qquad \quad \oint_{\mathcal{L}_1} = \vec{E} \cdot d\vec{l} = \oint_{\mathcal{L}_2} = \vec{E} \cdot d\vec{l} \quad
\text{con} \; \begin{matrix} \mathcal{L}_{1,\text{iniziale}} = \mathcal{L}_{2,\text{iniziale}} \\ \mathcal{L}_{1,\text{finale}} = \mathcal{L}_{2,\text{finale}} \end{matrix}\]

\subsection{Potenziale elettrostatico}
\subsubsection*{Potenziale elettrostatico}
Essendo \(\vec{E}\) un campo conservativo, si definisce il potenziale elettrostatico:
\[\int_A^B \vec{E} \cdot d \vec{s} = -\Delta V = V_A - V_B \qquad \qquad V(P) = \int_P^C \vec{E} \cdot d \vec{s} = [V_{olt}] \qquad (V(C) = 0)\]

\subsubsection*{Lavoro di una forza elettrostatica}
Il lavoro compiuto dalla forza elettrostatica per spostare una carica \(q\) vale:
\[\mathcal{L}_{AB} = \int_A^B \vec{F} \cdot d \vec{l} = q \int_A^B \vec{E} \cdot d \vec{l} = - q \Delta V = q(V_A - V_B)\]

\subsubsection*{Energia potenziale elettrostatica}
Si definisce l'energia potenziale di una carica come il lavoro compiuto per portare la carica da distanza \(\infty\) alla posizione
in cui si trova: \[\psi(P) = q_0 V(P)\]

\subsubsection*{Sorgenti del campo elettrico - distribuzioni di carica}
\begin{itemize}
	\item \textbf{distribuzioni di carica statiche - condizioni elettrostatiche}: \\
	le cariche che generano il campo sono in quiete e non ci sono correnti
	\item \textbf{distribuzione di carica stazionarie - regime stazionario}: \\
	le cariche sono in moto a velocità costante nel tempo e di conseguenza sono presenti correnti costanti nel tempo
	\item \textbf{distribuzione di carica variabile - regime variabile}: \\
	le cariche sono in moto variabile e le correnti variano nel tempo, il campo non è più conservativo in quanto avrà una componente
	non conservativa \(\vec{E}_\text{indotta}\)
\end{itemize}

\subsubsection*{Forza di una carica in moto}
Le forze agenti su cariche in moto immerse in un campo magnetico hanno una componente dovuta al campo elettrico e una al campo
magnetico: \[\vec{F} = q_0 \vec{E} + q_0 \vec{v} \times \vec{B}\]

\subsection{Tensioni e forze elettromotrici}
\subsubsection*{Tensione}
La tensione è definita come il lavoro elettrico per unità di carica speso a muovere una carica elettrica di prova lungo una linea \(L\).
\[u(t) = \int_L \vec{E}(P,t) \cdot \vec{t}(P) \; dl = \frac{W_e}{q_0} = [V_{olt}] = [J/C] = [J/A_{mpere} \;\! s] \qquad \vec{t}(P) = \text{versore della curva in P} \]
Questa definizione permette di essere indipendenti dalla conservatività del campo elettrico: se il campo elettrico è conservativo,
la tensione equivale al potenziale (a meno di un segno), mentre se il campo elettrico non è conservativo non si definisce nessun
potenziale, ma si può calcolare lo stesso la tensione.
\[\begin{matrix}
	\text{campo conservativo} & \rightarrow & \text{potenziale} = -\text{tensione} \\
	\text{campo non conservativo} & \rightarrow & \xcancel{\text{potenziale}} = -\text{tensione}
\end{matrix}\]

\subsubsection*{Forza elettromotrice indotta}
Un campo elettrico non conservativo, è formato da una parte conservativa e da un campo elettrico indotto non conservativo.
Di conseguenza la circuitazione non è più nulla e si definisce la forza elettromotrice indotta \(fem\) come il lavoro
compiuto dal campo elettrico lungo una linea chiusa \(L\):
\[fem \;\; e(t) = \oint_L \vec{E}(P,t) \cdot \vec{t}(P) \; dl\]

\subsubsection*{Forza elettromotrice mozionale}
La forza elettromotrice complessiva agente sulle cariche in moto è data da una componente dovuta al campo elettrico e da una
dovuta al campo magnetico, detta forza elettromotrice mozionale, provocata dal movimento delle cariche in un campo magnetico:
\[\oint_L (\vec{E} + \vec{v}\times \vec{B}) \cdot \vec{t} \; dl = \oint_L \vec{E} \cdot \vec{t} \; dl +  \oint_L (\vec{v} \times \vec{B}) \cdot \vec{t} \; dl = e(t) + e_m(t)\]

\subsubsection*{Voltmetro}
Il voltmetro è uno strumento per misurare la tensione, si collega in parallelo alla sezione di circuito di cui si vuole conoscere
la tensione. La direzione del sistema è data dal vettore \(\vec{t}\) dal \(+\) al \(-\). Un voltmetro si dice ideale se non altera
il regime del circuito.

\begin{center}
	\begin{circuitikz}
		\draw (0,0) -- (0.21,0) ;
		\draw (0.42,0) [circ,thick] circle(0.2);
		\draw (1,0) [circ, thick] circle(0.38);
		\draw (1.58,0) [circ, thick] circle(0.2);
		\draw (1.78,0) -- (2,0) ;
		\node[] at (1,0) {V} ;
		\node[] at (0.42,0) {+} ;
		\node[] at (1.58,-0.02) {--} ;
		\node[] at (1,-0.6) {u} ;
	\end{circuitikz}
\end{center}

\section{Modello a parametri concentrati - \textit{mpc}}
\subsection{Teorema}
\subsubsection*{Obiettivo}
\begin{itemize}
	\item[1.] ogni componente elettrico si può modellare con equazioni algebriche o differenziali che dipendono solo da tensioni
	o correnti (ovvero non da componenti spaziali)
	\item[2.] le tensioni sono definite tra coppie di morsetti e le correnti sono definite ai terminali
	\item[3.] i terminali terminano con morsetti utilizzati per collegare il componente con il resto del circuito
\end{itemize}

\subsubsection*{Ipotesi}
\begin{itemize}
	\item[1.] all'esterno dei componenti elettrici
	\begin{itemize}[topsep=0pt]
		\item il campo \(\vec{E}\) è conservativo
		\item la densità \(\vec{J}\) è solenoidale (regime stazionario, no accumuli o prelievi)
	\end{itemize}
	\item[2.] i terminali e i morsetti sono superfici equipotenziali senza accumuli o prelievi di carica in essi
\end{itemize}

\subsubsection*{Teoria}
\begin{itemize}
	\item[1.] è possibile modellare ogni componente attraverso Equazioni
	\item[2.] per formare un circuito si collegano più morsetti tra loro attraverso conduttori detti connessioni o interconnessioni
	che soddisfano le ipotesi dei morsetti/terminali
	\item[3.] se più morsetti sono attaccati insieme, si formano nodi di volume e carica nulla per cui vale la legge dei nodi (o legge
	di continuità \(\rightarrow \;\; \sum i_\text{entranti} + \sum i_\text{uscenti} = 0\))
\end{itemize}

\subsection{Componenti}
\subsubsection*{Introduzione}
I componenti sono elementi del circuito:
\begin{itemize}
	\item rappresentati graficamente da una curva chiusa (detta superficie limite) con due (o più) tratti filiformi detti terminali
	con cui si possono collegare ad altri componenti
	\item modellabili con un'equazione differenziale o algebrica
\end{itemize}
Si definiscono:
\begin{itemize}
	\item corrente entrante e corrente uscente (per ogni morsetto) rappresentata con \(\rightarrow\)
	\item tensione (per ogni coppia di morsetti) rappresentata con \(+ \; -\)
\end{itemize}

\subsubsection*{Porte elettriche, bipoli, n-poli m-bipoli, (n-1)-bipoli}
\begin{itemize}
	\item \textbf{porta elettrica}: coppia di terminali in cui la corrente entrante in un terminale è pari alla corrente uscente
	dall'altro terminale; si definisce la corrente di porta \(i_{AB}(t)\) e la tensione di porta \(u_{AB}(t)\)
	\item \textbf{bipolo elettrico}: componente con due terminali
	\item \textbf{n-polo}: componente con n terminali
	\item \textbf{m-bipolo}: componente con \(m\) porte e \(2m\) terminali
	\item \textbf{n-polo come (n-1)-bipolo}: componente in cui si sceglie un polo \(N\) come polo di riferimento e si definiscono:
	\begin{itemize}[topsep=0pt]
		\item n-1 porte tra il polo di riferimento e gli altri n-1 poli del componente
		\item n-1 correnti \(i_{kN}(t)\) che entrano dagli n-1 morsetti ed escono dal morsetto \(N\), per cui la corrente uscente
		dal polo \(N\) è la somma di tutte le correnti di porta
		\item n-1 tensioni \(u_{kN}(t)\) di porta
	\end{itemize}
\end{itemize}

\subsection{Reti elettriche o circuiti}
\subsubsection*{Introduzione}
Una rete è formata da interconnessioni tra n-poli e m-bipoli. Le interconnessioni prendono il nome di nodi e i morsetti collegati
allo stesso nodo sono superfici equipotenziali.

\subsubsection*{Suddivisione}
\begin{itemize}
	\item \textbf{reti in regime stazionario}: valgono le ipotesi del \textit{mpc}
	\item \textbf{reti in regime variabile}: in teoria non si potrebbe applicare il \textit{mpc}, ma si definiscono le \dots
	\item \textbf{reti in regime variabile-quasi-stazionario}: ovvero reti in regime variabile in cui è possibile applicare
	il \textit{mpc} se i campi \(\vec{E}\) e \(\vec{J}\) variano lentamente nel tempo e non si hanno propagazioni di onde
	elettromagnetiche all'esterno dei componenti
\end{itemize}

\begin{description}[itemsep=0pt]
	\item[Tipologia] tipo dei componenti utilizzati nel circuito (resistivo, capacitativo, RLC, \dots)
	\item[Topologia] tipo di connessioni utilizzate nel circuito (serie, parallelo, \dots)
\end{description}

\subsection{Potenza di bipolo, convenzione dei generatori e utilizzatori}
\subsubsection*{Potenza di una porta}
Data una porta elettrica, la potenza è data dal prodotto:
\[p(t) = u(t) \cdot i(t) \qquad [W_\text{att}] = [V_\text{olt}] \cdot [A_\text{mpere}]\]
%\begin{align*}
%	p(t) = \frac{d \mathcal{L}_e(t)}{dt} &= \frac{d \mathcal{L}_e(t)}{dq} \cdot \frac{d q}{dt} = u(t) \cdot i(t) \\
%	&= \lim_{\Delta t \to 0} \frac{\Delta q (V_A - V_B)}{\Delta t} = i(t) \cdot \Delta V(t) = u(t) \cdot i(t)
%\end{align*}

\subsubsection*{Convenzione degli utilizzatori}
Un componente (o meglio una porta o bipolo) soddisfa la convenzione degli utilizzatori se la corrente entra nel morsetto \(+\).
Se \(p > 0\), la porta assorbe potenza, se \(p < 0\) la porta eroga energia.

\subsubsection*{Convenzione dei generatori}
Un componente (o meglio una porta o bipolo) soddisfa la convenzione dei generatori se la corrente entra nel morsetto \(-\).
Se \(p > 0\), la porta eroga energia, se \(p < 0\) la porta assorbe potenza.

\subsubsection*{Lavoro elettrico}
Il lavoro elettrico compiuto da una porta nell'intervallo \([t_0, t_1]\) vale:
\[\mathcal{L}(t_0,t_1) = \int_{t_0}^{t_1} p(t) dt = \int_{t_0}^{t_1} u(t) \cdot i(t) \qquad\qquad
\begin{matrix} [J_\text{oule}] = [W_\text{att}] \cdot [\text{sec}] \\[5pt] [kWh] = 3.6 \cdot 10^6 \; [J] \end{matrix}\]
Il lavoro è entrante per potenza entrante e viceversa.

\subsubsection*{Wattmetro}
Il wattmetro è uno strumento per misurare la potenza ed è costituito da un amperometro (in serie) e un voltmetro (in parallelo) combinati.
\begin{center}
	... disegno ....
\end{center}

\newpage

\subsubsection*{Potenza di un m-bipolo}
La potenza di un m-bipolo in cui tutte le porte sono convenzionate allo stesso modo è la somma di tutte le potenze delle singole porte.
\[p(t) = p_1(t) + p_2(t) + \dots + p_n(t) \qquad \qquad \mathcal{L}(t_0, t_1) = \int_{t_0}^{t_1} p(t) dt = \int_{t_0}^{t_1} \sum_{n=1}^{m} u_n(t) \cdot i_n(t) dt\]
In base al segno e alla convenzione utilizzata, si avrà potenza dissipata o erogata.

\subsubsection*{Potenza di un n-polo}
Per calcolare la potenza di un n-polo, uso la corrispondenza tra n-polo e (n-1)-bipolo.

\subsubsection*{Bipolo passivo}
Un bipolo si dice passivo se il lavoro elettrico uscente da un certo tempo \(t_0\) in poi è minore dell'energia immagazzinata
fino a \(t_0\). Ovvero un bipolo passivo è capace di accumulare energia, ma non ne può emettere più di quella che ha immagazzinato.
In condizioni stazionarie \(p_\text{uscente}(t) < 0\)
\[\mathcal{L}_\text{lavoro uscente}(t_0,t) \; < \; W_\text{energia immagazzinata}(t_0)\]

\subsubsection*{Bipolo attivo}
Un bipolo si dice attivo se per certe condizioni non è rispettata la legge sopra, ovvero se per certe condizioni vale:
\[\mathcal{L}_\text{lavoro uscente}(t_0,t) \; > \; W_\text{energia immagazzinata}(t_0)\]
In generale un bipolo attivo fornisce lavoro elettrico convertendolo da altre fonti e in condizioni stazionarie \(p_\text{uscente}(t) > 0\)

\newpage

\section{Componenti elettrici}
\subsection{Caratteristica esterna}
\subsubsection*{Caratteristica esterna di un bipolo e di un m-bipolo}
La caratteristica esterna è un'equazione che lega tutte le variabili (tensione e corrente) di ogni porta di un determinato
componente. Per un m-bipolo si avranno 2m variabili (tensione e corrente delle m-porte), per un bipolo si avranno 2 variabili.
\begin{itemize}
	\item caratteristica esterna m-bipolo: \(F(u_1(t), i_1(t), u_2(t), i_2(t),\dots, u_m(t), i_m(t)) = 0\)
	\item caratteristica esterna bipolo: \(F(u(t), i(t)) = 0\)
\end{itemize}

\subsubsection*{Bipolo controllato in corrente o in tensione}
È possibile scrivere l'equazione per un bipolo in funzione di una delle due variabili:
\begin{itemize}
	\item bipolo controllato in corrente: \(u(t) = f(i(t))\)
	\item bipolo controllato in tensione: \(i(t) = f(u(t))\)
\end{itemize}

\subsubsection*{Bipolo ideale}
Un bipolo si dice ideale se la sua caratteristica esterna è lineare a tratti, ovvero è possibile usare un'equazione lineare per
descriverne il comportamento in un intorno del punto \((U^*, I^*)\)

\subsection{Componenti}
\subsubsection*{Resistenza}
\begin{itemize}
	\item equazione costitutiva: \(\quad R=U/I \qquad [\Omega_\text{(ohm)}] = [V_\text{olt}] \; / \; [A_\text{mpere}]\)
	\item caratteristica esterna: \(\quad F(u,i) = 0 \; \rightarrow \; \frac{U}{I} - R = 0\)
	\item effetto Joule: \(\qquad \qquad \;\; \Delta Q = R \cdot I^2 \cdot \Delta t\)
	\item bilancio energetico: \(\quad \;\;\; \begin{matrix} P_\text{entrante} = u \cdot i \\ P_\text{uscente} = R \cdot i^2 \end{matrix} \qquad u \cdot i = R \cdot i^2 \; \rightarrow \; P_\text{entrante} = P_\text{uscente}\)
	\item resistività: \(\qquad\qquad\qquad R = \rho \cdot L/S \qquad \rho = \rho_0(1+ \alpha \Delta T) \qquad \begin{matrix} \rho_0 = \text{resistività a 20°C} \\ \Delta T = \text{\(\Delta\)temp. rispetto a 20°C} \end{matrix}\)
	\begin{itemize}[topsep=0pt]
		\item conduttori: \(\qquad\quad \rho_\text{Cu} = 1.8 \cdot 10^{-8} \Omega m\)
		\item semiconduttori: \(\quad \rho_\text{Si} = 2.3 \cdot 10^3 \Omega m\)
		\item isolanti: \(\qquad\qquad\; \rho_\text{PVC} = 10^{10} - 10^{13} \Omega m\)
	\end{itemize}
\end{itemize}

\subsubsection*{Resistore ideale}
\begin{center}
	\begin{tabularx}{\textwidth}{ c | c | X }
		modello grafico & equazioni costitutive & parametri caratteristici \\
		\begin{circuitikz} \draw (0,0) [R] to (2,0); \end{circuitikz} &
		\(\begin{cases} u(t) = R \, i(t) \\ i(t) = G \, u(t) \end{cases}\) &
		\(\begin{cases}
			R = \text{resistenza} \; [\Omega_\text{ohm}] \\
			G = \text{conduttanza} \; [S_\text{iemens}]
		\end{cases}\)
	\end{tabularx}
\end{center}

\subsubsection*{Cortocircuito ideale}
\begin{center}
	\begin{tabularx}{\textwidth}{ c | c | X }
		modello grafico & equazioni costitutive & descrizione \\
		\begin{circuitikz} \draw (0,0) -- (2,0); \end{circuitikz} &
		\(u(t) = 0 \quad \forall t\) &
		resistore con \(R = 0\)
	\end{tabularx}
\end{center}

\subsubsection*{Circuito aperto}
\begin{center}
	\begin{tabularx}{\textwidth}{ c | c | X }
		modello grafico & equazioni costitutive & descrizione \\
		\begin{circuitikz} \draw (0,0) -- (0.7,0) (1.3,0) -- (2,0); \end{circuitikz} &
		\(i(t) = 0 \quad \forall t\) &
		resistore con \(R = +\infty\)
	\end{tabularx}
\end{center}

\subsubsection*{Interruttore ideale}
\begin{center}
	\begin{tabularx}{\textwidth}{ c | X }
		modello grafico & descrizione \\
		\begin{circuitikz} \draw (0,0) to[nos] (2,0); \end{circuitikz} &
		\(\begin{array}{l}
			\text{dipolo in grado di commutarsi tra due stati:} \\
			\text{  1. cortocircuito ideale} \\
			\text{  2. circuito aperto} \end{array}\)
	\end{tabularx}
\end{center}

\subsubsection*{Interruttore reale}
\begin{center}
	\begin{tabularx}{\textwidth}{ c | X }
		modello grafico & descrizione \\
		\begin{circuitikz} \draw (0,0) to[nos] (2,0) (0,-0.2) -- (2,-0.2); \end{circuitikz} &
		unipolare: interrompe la continuità di un solo conduttore
	\end{tabularx}
\end{center}

\subsubsection*{Diodo ideale}
\begin{center}
	\begin{tabularx}{\textwidth}{ c | X }
		modello grafico & descrizione \\
		\begin{circuitikz} \draw (0,0) to[empty diode] (2,0); \end{circuitikz} &
		\(\begin{array}{l}
			\text{dipolo in grado di commutarsi tra due stati:} \\
			\text{ se } u>0 \rightarrow \text{ conduzione} \\
			\text{ se } u<0 \rightarrow \text{ interdizione}
		\end{array}\)
	\end{tabularx}
\end{center}

\subsubsection*{Diodo reale}
\begin{center}
	\begin{tabularx}{\textwidth}{ c | X }
		modello grafico & descrizione \\
		\begin{circuitikz} \draw (0,0) to[empty diode] (2,0); \end{circuitikz} &
		\(\begin{array}{l}
			\text{dipolo con tre stati:} \\
			\text{ se } u>0 \rightarrow \text{ conduzione} \\
			\text{ se } V_{bd}<u<0 \rightarrow \text{ interdizione} \\
			\text{ se } u<V_{bd} \rightarrow \text{ rottura/conduzione} \\
			\text{ottenuto con giunzione PN (materiali drogati positivamente o negativamente) }
		\end{array}\)
	\end{tabularx}
\end{center}

\subsubsection*{Induttore ideale}
\begin{center}
	\begin{tabularx}{\textwidth}{ c | c | c | X }
		modello grafico & equazioni costitutive & parametri caratteristici & descrizione \\
		\begin{circuitikz} \draw (0,0) [L] to (2,0); \end{circuitikz} &
		\(\displaystyle u(t) = L \frac{di(t)}{dt}\) &
		\(L = \text{induttanza} \; [H_\text{enry}]\) &
		in regime stazionario agisce come un cortocircuito
	\end{tabularx}
\end{center}

\subsubsection*{Induttore reale}
Usato nei circuiti AC, \(\displaystyle e(t) = -\frac{d\Phi}{dt}\)

\subsubsection*{Condensatore ideale}
\begin{center}
	\begin{tabularx}{\textwidth}{ c | c | c | X }
		modello grafico & equazioni costitutive & parametri caratteristici & descrizione \\
		\begin{circuitikz} \draw (0,0) [C] to (2,0); \end{circuitikz} &
		\(i(t) = C \frac{du(t)}{dt}\) &
		\(C = \text{capacità} \; [C_\text{oulomb}]\) &
		in regime stazionario agisce come un circuito aperto
	\end{tabularx}
\end{center}

\subsubsection*{Condensatore reale}
Usato nei circuiti AC, \(\displaystyle \oint \vec{D} \cdot d\vec{\Sigma} = q_\text{int}\)

\newpage

\subsection{Generatori}
\subsubsection*{Forze elettriche generatrici e generatori}
In condizioni stazionarie si ha che \(\vec{J}\) è solenoidale e che \(Q = L + \Delta W \rightarrow Q = L\) ovvero il lavoro
compiuto dal circuito è tutto dissipato in calore dalle resistenze (\(\Delta W = 0\)). Per cui si ha:
\[Q = L = \oint \vec{F}_\text{gen} \cdot d \vec{l} \neq 0\]
Esistono delle forze generatrici non conservative \(\vec{F}_\text{gen}\) la cui circuitazione non è nulla che mettono in moto
le cariche. Le parti del circuito in cui si sviluppano tali forze non conservative sono detti generatori.

\subsubsection*{Forza elettrica generatrice specifica}
Si definisce quindi la forza elettrica generatrice specifica come forza per unità di carica:
\[\vec{E}_\text{gen} = \frac{\vec{F}_\text{gen}}{q} \qquad [N_\text{ewton} / C_\text{oulomb}] = [V_\text{olt} / m]\]

\subsubsection*{Generatori a vuoto}
\begin{itemize}
	\item si hanno forze \(\vec{E}_\text{gen}\) che inducono accumuli di cariche ai capi del generatore e l'accumulo forma
	un campo \(\vec{E}_\text{coulombiano} \; (\vec{E}_c)\) contrario a \(\vec{E}_\text{gen}\)
	\item in regime stazionario \(\vec{E}_g = -\vec{E}_c\) e le cariche non si muovono (o hanno \(a = 0\), \(v\) costante)
	\item si ottiene che la forze elettromotrice è pari alla tensione ai capi del generatore:
	\[fem_{AB} = \int_B^A \vec{E}_g(P,t) \cdot d\vec{l} = \int_B^A -\vec{E}_c(P,t) \cdot d\vec{l} = \int_A^B \vec{E}_c(P,t) \cdot d\vec{l} = u_{0,AB}(t)\]
	\item se \(U_{0,AB}(t)\) è costante, il generatore è in regime DC e \(E_{AB} = U_{0,AB}\)
	\item se \(U_{0,AB}(t)\) è sinusoidale, il generatore è in regime AC e \(E_{AB}(t) = u_{0,AB}(t)\)
	\item il campo \(\vec{E}_c\) tende a muovere le cariche positive dal \(+\) al \(-\)
	\item il campo \(\vec{E}_g\) tende a muovere le cariche positive dal \(-\) al \(+\)
\end{itemize}

\subsubsection*{Generatori a carico}
\begin{itemize}
	\item \(\vec{E}_g \neq -\vec{E}_c\), si ha uno spostamento di cariche
	\item la caratteristica esterna vale \(U = E_{AB} - R_i I\) con \(R_i = \) resistenza interna
\end{itemize}

\subsubsection*{Generatori ideali}
\begin{tabularx}{0.6\textwidth}{ c  X }
	generatore ideale di tensione \(U = E\) &
	\begin{circuitikz}
		%\draw (0,0) -- (0.58,0);
		%\draw(1,0) [circ, thick] circle(0.42);
		%\draw (1.42,0) -- (2,0);
		%\node [] at (0.8,0) {--};
		%\node [] at (1.2,0) {+};
		\draw (0,0) to[V] (2,0);
		\node [] at (1,-0.7) {E};
		\node [] at (0.2,-0.25) {--};
		\node [] at (1.8,-0.25) {+};
	\end{circuitikz} \\
	\midrule
	generatore ideale di corrente \(I = J\) &
	\begin{circuitikz}
		%\draw (0,0) -- (0.58,0);
		%\draw(1,0) [circ, thick] circle(0.42);
		%\draw (1.42,0) -- (2,0);
		%\draw [->] (0.7,0) -- (1.3,0);
		\draw (0,0) to[I] (2,0);
		\draw[->] (0.6,-0.7) -- (1.4,-0.7);
		\node[] at (1,-0.9) {J} ;
	\end{circuitikz}
\end{tabularx}

\subsubsection*{Esempi di generatori}
\begin{itemize}
	\item elettrochimici (sede di reazioni chimiche, esempio pile a secco e accumulatori)
	\item fotovoltaici (fotone "convertito" in elettrone, funzionamento basato su fotodiodi)
	\item termoelettrici (giunzioni bimetalliche a temperature diverse formano una f.e.m. per effetto Seebeck)
	\item piezoelettrici (cristalli soggetti a stress meccanico formano un campo elettrico, quindi una f.e.m.)
	\item elettromeccanici (macchine rotanti con di statore e rotore, convertono energia meccanica in elettrica)
\end{itemize}

\newpage

\section{Topologia delle reti}
\section{Analisi di circuiti lineari a corrente continua}

\end{document}
