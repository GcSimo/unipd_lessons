\documentclass[a4paper]{article}
\usepackage[utf8]{inputenc} % standard unicode
\usepackage[italian]{babel} % corretta sillabazione in italiano
\usepackage{geometry} % per impostare margini e layout pagina
\usepackage{amssymb} % per l'ambiente matematico
\usepackage{amsmath} % per l'ambiente matematico
\usepackage{enumitem} % per elenchi puntati
\usepackage{multirow} % per celle che si espandono su più righe
\usepackage{tabularx} % per tabelle con larghezza flessibile
\usepackage{booktabs} % per linee orizzontali tabelle
\usepackage{hyperref} % per collegamenti
\usepackage{dirtytalk} % per le ""
\usepackage{circuitikz} % per circuiti elettrici
\usepackage{cancel} % per barrare testpo

% per margini
\geometry{a4paper,left=25mm, right=25mm, bottom=25mm, top=30mm}

% per centrare testo nelle tabelleX
\renewcommand\tabularxcolumn[1]{m{#1}}

% percorso delle immagini da inserire
\graphicspath{ {./ } }

% parte funzione reale e parte immaginaria
\newcommand\Real{\text{Re}}
\newcommand\Img{\text{Im}}

% versori
\newcommand\ux{\vec{u}_x}
\newcommand\uy{\vec{u}_y}
\newcommand\uz{\vec{u}_z}
\newcommand\uxp{\vec{u}_{x'}}
\newcommand\uyp{\vec{u}_{y'}}
\newcommand\uzp{\vec{u}_{z'}}
\newcommand\ur{\vec{u}_r}
\newcommand\uv{\vec{u}_v}
\newcommand\un{\vec{u}_n}
\newcommand\ug{\vec{u}_\gamma}
\newcommand\uper{\vec{u}_\perp}
\newcommand\upar{\vec{u}_\parallel}
\newcommand\nab{\vec{\nabla}} % nabla

% forza generica
\newcommand\ft{\vec{F}\left(\vec{\gamma}(t), \; \dt \vec{\gamma}(t), \; t\right)}
\newcommand\ftau{\vec{F}\left(\vec{\gamma}(\tau), \; \dtau \vec{\gamma}(\tau), \; \tau\right)}

% derivata
\newcommand\dt{\frac{d}{dt}\,}
\newcommand\dtau{\frac{d}{d\tau}\,}
\newcommand\dts{\frac{d^2}{dt^2}\,}

% modulo vettore
\newcommand\vmod[1]{\left|\left|{#1}\right|\right|}

% per elenci puntati
\setlist[itemize]{label=-, partopsep=0pt, topsep=3pt, itemsep=0pt}

\title{Appunti di Teoria dei circuiti}
\author{Giacomo Simonetto}
\date{Secondo semestre 2024-25}

\begin{document}

% -------------------------------------- Copertina e indice ---------------------------------------
\maketitle
\begin{abstract}
	Appunti del corso di Teoria dei circuiti della facoltà di Ingegneria Informatica dell'Università di Padova.
\end{abstract}

\newpage

\tableofcontents

\newpage

\section{Introduzione alla teoria dei circuiti}
\subsubsection*{Definizione di circuito}
Un circuito elettrico è un insieme di dispositivi elettrici interconnessi, deputati alla produzione, trasmissione ed utiizzazione
dell'energia elettrica.

\subsubsection*{Equazioni di Maxwell}
È possibile risolvere un circuito attraverso le equazioni di Maxwell, ma si otterrebbe un sistema troppo complesso da gestire e da
risolvere, per cui si utilizzano approssimazioni e modelli definiti dalla teoria dei circuiti.

%\begin{align*}
%	\nab \cdot \vec{D} = \rho & \quad \text{Flusso del campo elettrico - Gauss} \\
%	\nab \cdot \vec{B} = 0 & \quad \text{Flusso del campo magnetico - Gauss} \\
%	\nab \times \vec{E} = - \frac{\partial \vec{B}}{\partial t} & \quad \text{Circuitazione del campo elettrico - Faraday-Neumann-Lenz} \\
%	\nab \times \vec{H} = \vec{J} + \frac{\partial \vec{D}}{\partial t} & \quad \text{Circuitazione del campo magnetico - Ampère-Maxwell} \\
%	\nab \cdot \vec{J} = -\frac{\partial \rho}{\partial t} & \quad \text{Continuità} \\
%	\vec{D} = \varepsilon \vec{E} \qquad \vec{B} = \mu \vec{H} & \quad \text{Relazioni tra i vari campi}\\
%\end{align*}

\subsubsection*{Modello zero-dimensionale}
Il modello zero-dimensionale non tiene conto di cosa avviene all'interno dei componenti elettrici, ma solo di come interagiscono
tra di loro. In altre parole viene trascurata la loro dimensione.

\subsubsection*{Grandezze fisiche}
Le grandezze fisiche utilizzate sono: tensione, corrente, potenza, energia e frequenza

\subsubsection*{Modello a parametri concentrati}
Il modello a parametri concentrati prevede che:
\begin{itemize}
	\item[1.] i componenti RLC sono idealizzati e considerati puntiformi (modello zero-dimensionale)
	\item[2.] tensioni e correnti dipendono dal tempo e non dallo spazio: si può evitare di considerare eventuali propagazioni
	elettromagnetiche
	\item[3.] l'interazione tra componenti avviene solo attraverso connessioni elettriche
\end{itemize}
Il suo scopo è di:
\begin{itemize}
	\item analizzare i comportamenti di tensioni e correnti (flussi di potenza)
	\item prevedere comportamenti dei dispositivi reali mediante modelli semplificati
	\item progettare e ottimizzare sistemi elettrici
\end{itemize}

\subsubsection*{Validità}
La teoria dei circuiti è valida se la dimensione del circuito è inferiore alla lunghezza d'onda del segnale che circola all'interno:
\begin{itemize}
	\item corrente alternata di rete \(\rightarrow\) 50 Hz \(\rightarrow\) \(\lambda\) = 6000 km
	\item radiofrequenza \(\rightarrow\) 100 MHz \(\rightarrow\) \(\lambda\) = 3 m
	\item microonde \(\rightarrow\) 10 GHz \(\rightarrow\) \(\lambda\) = 3 cm (limite della TdC)
\end{itemize}

\subsubsection*{Tipi di circuiti}
\begin{itemize}
	\item circuiti elettrici di segnale, lavorano con mW
	\item circuiti elettrici di potenza, lavorano con kW
\end{itemize}

\subsubsection*{Flusso e trasmissione di energia}
Per flusso di energia si intende cone viene utilizzata la potenza in un circuito. La trasmissione di energia può avvenire in due
modi: attraverso onde elettromagnetiche (radio, antenne, \dots) o per conduzione (linee elettriche).

\newpage

\section{Interpretazione fisica dell'elettrostatica}
\subsection{Campi e grandezze fisiche}
\subsubsection*{Campo fisico}
Un campo fisico è la distribuzione su un volme o su una superficie di una certa grandezza fisica rappresentabile tramite vettore
o scalare. I campi fisici di grandezze scalari si dicono campi scalari, mentre i campi fisici di grandezze vettoriali si dicono
campi vettoriali.

\subsubsection*{Grandezze fisiche}
Una grandezza fisica è una quantità misurabile di un oggetto. Il processo di misura consiste nel comparare una quantità campione
(detta unità di misura) con l'oggetto da misurare. Le grandezze fondamentali del Sistema Internazionale sono: m, kg, s, K, A, cd,
mol.

\subsection{Carica elettrica e densità di carica}
\subsubsection*{Carica elettrica}
\begin{itemize}
	\item la quantità di carica è una grandezza che misura la carica elettrica di un oggetto
	\item si osserva che esiste una forza che dipende dalla quantità di carica dei corpi e può essere attrattiva tra corpi con
	cariche di segno opposto o repulsiva tra corpi con cariche dello stesso segno
	\item la carica è quantizzata con quanto \(e = 1.6 \cdot 10^{-19} \; C\)
\end{itemize}

\subsubsection*{Densità di carica}
La carica di una distribuzione è data da \(\displaystyle q = \int_V \rho d\tau\), ovvero la somma complessiva delle cariche positive
e negative di un corpo:
\begin{itemize}
	\item densità volumica: \(\displaystyle \rho(P,t) = [C_{oulomb}/m^3] = \lim_{V \to 0} \frac{q}{V}, \quad q = \int_V \rho(P,t) d\tau\)
	\item densità superficiale: \(\displaystyle \sigma(P,t) = [C_{oulomb}/m^2] = \lim_{\Sigma \to 0} \frac{q}{\Sigma}, \quad q = \int_\Sigma \sigma(P,t) d\Sigma\)
\end{itemize}

\subsection{Corrente elettrica e densità di corrente}
\subsubsection*{Densità di corrente}
Si genera per conduzione elettrica attraverso due modi:
\begin{itemize}
	\item corrente di conduzione: moto delle cariche libere (es. nei metalli)
	\item corrente di convezione: moto delle cariche libere e/o vincolate (es. soluzioni elettrolitiche)
\end{itemize}
\[\vec{J}(P,t) = \rho^+ + v_d^+ + \rho^- + v_d^-  \qquad \begin{cases}
	\rho^+ \;\rightarrow\; \text{densità delle cariche positive} \\
	v_d^+ \;\rightarrow\; \text{velocità di deriva delle cariche positive} \\
	\rho^- \;\rightarrow\; \text{densità delle cariche negative} \\
	v_d^- \;\rightarrow\; \text{velocità di deriva delle cariche negative}
\end{cases}\]

\subsubsection*{Corrente}
La corrente è la quantità di cariche che attraversano una superficie in un'unità di tempo. Dipende dalla superficie e dal suo
orientamento. Non dipende dal resto dello spazio. Se si inverte l'orientamento della superficie o il riferimento, il segno della
corrente si inverte. Si misura in Ampère \([A_{mpere}] = [C_{oulomb}/s]\).
\[i(t) = \int_\Sigma \vec{J}(P,t) \cdot d\vec{\Sigma} \qquad \Leftrightarrow \qquad i(t) = \lim_{\Delta t \to 0} \frac{\Delta q_{\text{ attraverso }\Sigma}(t)}{\Delta t}\]
In caso di conduttori filiformi (dove \(\Sigma \ll \text{lunghezza}\)), vale \(i(t) = \vec{J} \cdot \vec{\Sigma}\)

\subsubsection*{Conservazione della carica e continuità della corrente}
La carica elettrica non si crea, non si distrugge, si conserva sempre.
\begin{align*}
	&q_\text{interna}(t + \Delta t) = q_\text{interna}(t) + \Delta q_\text{uscente} \\
	&i_\text{uscente}(t) = \lim_{\Delta t \to 0} \frac{\Delta q_\text{uscente}}{\Delta t} = -\frac{d q_\text{entrante}}{dt} = - i_\text{entrante}
\end{align*}
\begin{itemize}
	\item la variazione di carica corrisponde ad una corrente
	\item in assenza di corrente, la carica non varia
	\item la carica entrante è pari a quella uscente (in modulo)
\end{itemize}

\subsubsection*{Corrente solenoidale}
La corrente si dice solenoidale quando:
\begin{itemize}
	\item si è in regime stazionario: non si hanno accumuli o prelievi di carica in nessun punto del volume, la carica entrante
	e quella uscente sono uguali e il campo \(\vec{J}\) forma linee di flusso chiuse
	\item in regioni di carica nulla: \(\rho = 0\) ad esempio nei metalli
\end{itemize}
si è in regime stazionario

\subsubsection*{Tubo di flusso}
Il tubo di flusso è un conduttore rivestito da materiale isolante che può essere attraversato da corrente. In condizioni stazionarie
(con campo di corrente solenoidale) si ha che la corrente \(i_1\) attraverso una superficie \(\Sigma_1\) è uguale alla corrente
\(i_2\) attraverso una superficie \(\Sigma_2\). Ovvero non si hanno perdite di corrente: \(i_\text{uscente} = 0\).

\subsubsection*{Amperometro}
L'amperometro è uno strumento per misurare la corrente in un circuito. Il verso del sistema è dal \(+\) al \(-\) (ovvero la corrente
entra dal connettore \(+\) ed esce dal connettore \(-\)). Si usa in serie al circuito. Un amperometro si dice ideale se non influisce
sul circuito e se la misura avviene senza ritardi.

\begin{center}
	\begin{circuitikz}
		\draw (0,0) -- (0.21,0) ;
		\draw (0.42,0) [circ,thick] circle(0.2);
		\draw (1,0) [circ, thick] circle(0.38);
		\draw (1.58,0) [circ, thick] circle(0.2);
		\draw (1.78,0) -- (2,0) ;
		\draw[->] (0.6,-0.7) -- (1.4,-0.7);
		\node[] at (1,0) {A} ;
		\node[] at (0.42,0) {+} ;
		\node[] at (1.58,-0.02) {--} ;
		\node[] at (1,-0.9) {i} ;
	\end{circuitikz}
\end{center}

\newpage

\subsection{Campo elettrostatico}
\subsubsection*{Legge di Coulomb e campo elettrostatico}
Il campo elettrostatico si definisce a partire dalla forza di Coulomb, per questo è anche chiamato campo coulombiano.
\[\vec{F}_{1,2} = \frac{1}{4 \pi \varepsilon_0} \frac{q_1 \, q_2}{{r_{1,2}}^2} \hat{u}_{1,2} \qquad \qquad
\vec{F}_\text{elettr} = q \vec{E} \qquad \qquad
\vec{E} = \frac{\vec{F}}{q} = \frac{1}{4 \pi \varepsilon_0} \frac{q}{r^2} \hat{u}_{1,2} = [N/C_{oulob}] = [V_{olt}/m]\]
Il campo elettrostatico è additivo: \(\displaystyle \vec{E}(P) = \frac{1}{4 \pi \varepsilon_0} \sum_{k=1}^{n} \frac{q_k}{{r_{PO_k}}^2} \vec{u}_k(P) \qquad 
\vec{E}(P) = \frac{1}{4 \pi \varepsilon_0} \int_V \frac{\rho}{r^2} \vec{u}_k d\tau\)

\subsubsection*{Campo elettrostatico nei conduttori}
Un conduttore è un materiale che conduce corrente. Le cariche sono libere di muoversi e, muovendonsi, generano una corrente. In
condizione di equilibrio il campo all'interno è nullo, ovvero non c'è nessuna forza che agisce sulle cariche e le cariche sono
ferme (altrimenti non ci sarebbe equilibrio).

\subsubsection*{Campo elettrostatico nei dielettrici - isolanti}
Un dielettrico o isolante è un materiale che non conduce corrente. Le cariche sono bloccate a meno di piccoli spostamenti responsabili
della polarizzazione dei dielettrici. I dielettrici possono essere:
\begin{itemize}
	\item omogenei: se \(\varepsilon\) non dipenden dalla posizione
	\item lineari: se \(\varepsilon\) non dipende dal modulo del campo elettrico \(||\vec{E}||\)
	\item isotropi: se \(\varepsilon\) non dipende dalla direzione del campo elettrico \(\vec{u} = \vec{E} \big/ ||{\vec{E}}||\)
\end{itemize}
Un dielettrio omogeneo, lineare e isotropo si dice uniforme e per esso valgono tutte le leggi viste finora.

\subsubsection*{Permittività dielettrica di un mezzo}
La permittività dielettrica di un mezzo indica come tale mezzo reagisce al campo elettrico:
\[\varepsilon = \varepsilon_\text{relativa del mezzo} \cdot \varepsilon_\text{0 - nel vuoto} = [F_{araday}/m] = [{C_{oulomb}}^2/J] \qquad \quad \varepsilon > \varepsilon_0 \;\; \varepsilon_r > 1\]

\subsubsection*{Campo elettrico conservativo}
Il campo elettrostatico è conservativo ovvero:
\[\oint_\mathcal{L} = \vec{E} \cdot d\vec{l} = 0 \qquad \quad \oint_{\mathcal{L}_1} = \vec{E} \cdot d\vec{l} = \oint_{\mathcal{L}_2} = \vec{E} \cdot d\vec{l} \quad
\text{con} \; \begin{matrix} \mathcal{L}_{1,\text{iniziale}} = \mathcal{L}_{2,\text{iniziale}} \\ \mathcal{L}_{1,\text{finale}} = \mathcal{L}_{2,\text{finale}} \end{matrix}\]

\subsection{Potenziale elettrostatico}
\subsubsection*{Potenziale elettrostatico}
Essendo \(\vec{E}\) un campo conservativo, si definisce il potenziale elettrostatico:
\[\int_A^B \vec{E} \cdot d \vec{s} = -\Delta V = V_A - V_B \qquad \qquad V(P) = \int_P^C \vec{E} \cdot d \vec{s} = [V_{olt}] \qquad (V(C) = 0)\]

\subsubsection*{Lavoro di una forza elettrostatica}
Il lavoro compiuto dalla forza elettrostatica per spostare una carica \(q\) vale:
\[\mathcal{L}_{AB} = \int_A^B \vec{F} \cdot d \vec{l} = q \int_A^B \vec{E} \cdot d \vec{l} = - q \Delta V = q(V_A - V_B)\]

\subsubsection*{Energia potenziale elettrostatica}
Si definisce l'energia potenziale di una carica come il lavoro compiuto per portare la carica da distanza \(\infty\) alla posizione
in cui si trova: \[\psi(P) = q_0 V(P)\]

\subsubsection*{Sorgenti del campo elettrico - distribuzioni di carica}
\begin{itemize}
	\item \textbf{distribuzioni di carica statiche - condizioni elettrostatiche}: \\
	le cariche che generano il campo sono in quiete e non ci sono correnti
	\item \textbf{distribuzione di carica stazionarie - regime stazionario}: \\
	le cariche sono in moto a velocità costante nel tempo e di conseguenza sono presenti correnti costanti nel tempo
	\item \textbf{distribuzione di carica variabile - regime variabile}: \\
	le cariche sono in moto variabile e le correnti variano nel tempo, il campo non è più conservativo in quanto avrà una componente
	non conservativa \(\vec{E}_\text{indotta}\)
\end{itemize}

\subsubsection*{Forza di una carica in moto}
Le forze agenti su cariche in moto immerse in un campo magnetico hanno una componente dovuta al campo elettrico e una al campo
magnetico: \[\vec{F} = q_0 \vec{E} + q_0 \vec{v} \times \vec{B}\]

\subsection{Tensioni e forze elettromotrici}
\subsubsection*{Tensione}
La tensione è definita come il lavoro elettrico per unità di carica speso a muovere una carica elettrica di prova lungo una linea \(L\).
\[u(t) = \int_L \vec{E}(P,t) \cdot \vec{t}(P) \; dl = \frac{W_e}{q_0} = [V_{olt}] = [J/C] = [J/A_{mpere} \;\! s] \qquad \vec{t}(P) = \text{versore della curva in P} \]
Questa definizione permette di essere indipendenti dalla conservatività del campo elettrico: se il campo elettrico è conservativo,
la tensione equivale al potenziale (a meno di un segno), mentre se il campo elettrico non è conservativo non si definisce nessun
potenziale, ma si può calcolare lo stesso la tensione.
\[\begin{matrix}
	\text{campo conservativo} & \rightarrow & \text{potenziale} = -\text{tensione} \\
	\text{campo non conservativo} & \rightarrow & \xcancel{\text{potenziale}} = -\text{tensione}
\end{matrix}\]

\subsubsection*{Forza elettromotrice indotta}
Un campo elettrico non conservativo, è formato da una parte conservativa e da un campo elettrico indotto non conservativo.
Di conseguenza la circuitazione non è più nulla e si definisce la forza elettromotrice indotta \(f.e.m.\) come il lavoro
compiuto dal campo elettrico lungo una linea chiusa \(L\):
\[f.e.m. \;\; e(t) = \oint_L \vec{E}(P,t) \cdot \vec{t}(P) \; dl\]

\subsubsection*{Forza elettromotrice mozionale}
La forza elettromotrice complessiva agente sulle cariche in moto è data da una componente dovuta al campo elettrico e da una
dovuta al campo magnetico, detta forza elettromotrice mozionale:
\[\oint_L (\vec{E} + \vec{v}\times \vec{B}) \cdot \vec{t} \; dl = \oint_L \vec{E} \cdot \vec{t} \; dl +  \oint_L (\vec{v} \times \vec{B}) \cdot \vec{t} \; dl = e(t) + e_m(t)\]

\subsubsection*{Voltmetro}
Il voltmetro è uno strumento per misurare la tensione, si collega in parallelo alla sezione di circuito di cui si vuole conoscere
la tensione. La direzione del sistema è data dal vettore \(\vec{t}\) dal \(+\) al \(-\). Un voltmetro si dice ideale se non altera
il regime del circuito.

\begin{center}
	\begin{circuitikz}
		\draw (0,0) -- (0.21,0) ;
		\draw (0.42,0) [circ,thick] circle(0.2);
		\draw (1,0) [circ, thick] circle(0.38);
		\draw (1.58,0) [circ, thick] circle(0.2);
		\draw (1.78,0) -- (2,0) ;
		\node[] at (1,0) {V} ;
		\node[] at (0.42,0) {+} ;
		\node[] at (1.58,-0.02) {--} ;
		\node[] at (1,-0.6) {u} ;
	\end{circuitikz}
\end{center}

\section{Modello a parametri concentrati}
\section{Componenti elettrici}
\section{Topologia delle reti}
\section{Analisi di circuiti lineari a corrente continua}

\end{document}
