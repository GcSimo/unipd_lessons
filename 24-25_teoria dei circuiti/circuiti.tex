\documentclass[a4paper]{article}
\usepackage[utf8]{inputenc} % standard unicode
\usepackage[italian]{babel} % corretta sillabazione in italiano
\usepackage{geometry} % per impostare margini e layout pagina
\usepackage{amssymb} % per l'ambiente matematico
\usepackage{amsmath} % per l'ambiente matematico
\usepackage{enumitem} % per elenchi puntati
\usepackage{multirow} % per celle che si espandono su più righe
\usepackage{tabularx} % per tabelle con larghezza flessibile
\usepackage{booktabs} % per linee orizzontali tabelle
\usepackage{hyperref} % per collegamenti
\usepackage{graphicx} % per immagini
\usepackage{subcaption} % per immagini
\usepackage{multicol} % per pagina in colonne
\usepackage{dirtytalk} % per le ""
\usepackage{circuitikz} % per circuiti elettrici
\usepackage{cancel} % per barrare testo
\usepackage{xcolor} % per caratteri bianchi

% per margini
\geometry{a4paper,left=25mm, right=25mm, bottom=25mm, top=30mm}

% per centrare testo nelle tabelleX
\renewcommand\tabularxcolumn[1]{m{#1}}

% per elenchi puntati
\setlist[itemize]{label=-, partopsep=0pt, topsep=3pt, itemsep=0pt}

% espressioni matematiche
\newcommand\nab{\vec{\nabla}} % nabla

% percorso delle immagini da inserire
\graphicspath{{./}}

% --- altro che si può eliminare ---
% parte funzione reale e parte immaginaria
\newcommand\Real{\text{Re}}
\newcommand\Img{\text{Im}}

% versori
%\newcommand\ux{\vec{u}_x}
%\newcommand\uy{\vec{u}_y}
%\newcommand\uz{\vec{u}_z}
%\newcommand\uxp{\vec{u}_{x'}}
%\newcommand\uyp{\vec{u}_{y'}}
%\newcommand\uzp{\vec{u}_{z'}}
%\newcommand\ur{\vec{u}_r}
%\newcommand\uv{\vec{u}_v}
%\newcommand\un{\vec{u}_n}
%\newcommand\ug{\vec{u}_\gamma}
%\newcommand\uper{\vec{u}_\perp}
%\newcommand\upar{\vec{u}_\parallel}

% forza generica
%\newcommand\ft{\vec{F}\left(\vec{\gamma}(t), \; \dt \vec{\gamma}(t), \; t\right)}
%\newcommand\ftau{\vec{F}\left(\vec{\gamma}(\tau), \; \dtau \vec{\gamma}(\tau), \; \tau\right)}

% derivata
%\newcommand\dt{\frac{d}{dt}\,}
%\newcommand\dtau{\frac{d}{d\tau}\,}
%\newcommand\dts{\frac{d^2}{dt^2}\,}

% modulo vettore
%\newcommand\vmod[1]{\left|\left|{#1}\right|\right|}

\title{Appunti di Teoria dei circuiti}
\author{Giacomo Simonetto}
\date{Secondo semestre 2024-25}

\begin{document}

% -------------------------------------- Copertina e indice ---------------------------------------
\maketitle
\begin{abstract}
	Appunti del corso di Teoria dei circuiti della facoltà di Ingegneria Informatica dell'Università di Padova.
\end{abstract}

\newpage

\tableofcontents

\newpage

\section{Introduzione alla teoria dei circuiti}
\subsubsection*{Definizione di circuito}
Un circuito elettrico è un insieme di dispositivi elettrici interconnessi, deputati alla produzione, trasmissione ed utilizzazione
dell'energia elettrica.

\subsubsection*{Equazioni di Maxwell}
È possibile risolvere un circuito attraverso le equazioni di Maxwell, ma si otterrebbe un sistema troppo complesso da gestire e da
risolvere, per cui si utilizzano approssimazioni e modelli definiti dalla teoria dei circuiti.

%\begin{align*}
%	\nab \cdot \vec{D} = \rho & \quad \text{Flusso del campo elettrico - Gauss} \\
%	\nab \cdot \vec{B} = 0 & \quad \text{Flusso del campo magnetico - Gauss} \\
%	\nab \times \vec{E} = - \frac{\partial \vec{B}}{\partial t} & \quad \text{Circuitazione del campo elettrico - Faraday-Neumann-Lenz} \\
%	\nab \times \vec{H} = \vec{J} + \frac{\partial \vec{D}}{\partial t} & \quad \text{Circuitazione del campo magnetico - Ampère-Maxwell} \\
%	\nab \cdot \vec{J} = -\frac{\partial \rho}{\partial t} & \quad \text{Continuità} \\
%	\vec{D} = \varepsilon \vec{E} \qquad \vec{B} = \mu \vec{H} & \quad \text{Relazioni tra i vari campi}\\
%\end{align*}

\subsubsection*{Modello zero-dimensionale}
Il modello zero-dimensionale non tiene conto di cosa avviene all'interno dei componenti elettrici, ma solo di come interagiscono
tra di loro. In altre parole viene trascurata la loro dimensione.

\subsubsection*{Grandezze fisiche}
Le grandezze fisiche utilizzate sono: tensione, corrente, potenza, energia e frequenza

\subsubsection*{Modello a parametri concentrati}
Il modello a parametri concentrati prevede che:
\begin{itemize}
	\item[1.] i componenti RLC sono idealizzati e considerati puntiformi (modello zero-dimensionale)
	\item[2.] tensioni e correnti dipendono dal tempo e non dallo spazio: si può evitare di considerare eventuali propagazioni
	elettromagnetiche
	\item[3.] l'interazione tra componenti avviene solo attraverso connessioni elettriche
\end{itemize}
Il suo scopo è di:
\begin{itemize}
	\item analizzare i comportamenti di tensioni e correnti (flussi di potenza)
	\item prevedere comportamenti dei dispositivi reali mediante modelli semplificati
	\item progettare e ottimizzare sistemi elettrici
\end{itemize}

\subsubsection*{Validità}
La teoria dei circuiti è valida se la dimensione del circuito è inferiore alla lunghezza d'onda del segnale che circola all'interno:
\begin{itemize}
	\item corrente alternata di rete \(\rightarrow\) 50 Hz \(\rightarrow\) \(\lambda\) = 6000 km
	\item radiofrequenza \(\rightarrow\) 100 MHz \(\rightarrow\) \(\lambda\) = 3 m
	\item microonde \(\rightarrow\) 10 GHz \(\rightarrow\) \(\lambda\) = 3 cm (limite della TdC)
\end{itemize}

\subsubsection*{Tipi di circuiti}
\begin{itemize}
	\item circuiti elettrici di segnale, lavorano con mW
	\item circuiti elettrici di potenza, lavorano con kW
\end{itemize}

\subsubsection*{Flusso e trasmissione di energia}
Per flusso di energia si intende come viene utilizzata la potenza in un circuito. La trasmissione di energia può avvenire in due
modi: attraverso onde elettromagnetiche (radio, antenne, \dots) o per conduzione (linee elettriche).

\newpage

\section{Interpretazione fisica dell'elettrostatica}
\subsection{Campi e grandezze fisiche}
\subsubsection*{Campo fisico}
Un campo fisico è la distribuzione su un volume o su una superficie di una certa grandezza fisica rappresentabile tramite vettore
o scalare. I campi fisici di grandezze scalari si dicono campi scalari, mentre i campi fisici di grandezze vettoriali si dicono
campi vettoriali.

\subsubsection*{Grandezze fisiche}
Una grandezza fisica è una quantità misurabile di un oggetto. Il processo di misura consiste nel comparare una quantità campione
(detta unità di misura) con l'oggetto da misurare. Le grandezze fondamentali del Sistema Internazionale sono: m, kg, s, K, A, cd,
mol.

\subsection{Carica elettrica e densità di carica}
\subsubsection*{Carica elettrica}
\begin{itemize}
	\item la quantità di carica è una grandezza che misura la carica elettrica di un oggetto
	\item si osserva che esiste una forza che dipende dalla quantità di carica dei corpi e può essere attrattiva tra corpi con
	cariche di segno opposto o repulsiva tra corpi con cariche dello stesso segno
	\item la carica è quantizzata con quanto \(e = 1.6 \cdot 10^{-19} \; C\)
\end{itemize}

\subsubsection*{Densità di carica}
La carica di una distribuzione è data da \(\displaystyle q = \int_V \rho d\tau\), ovvero la somma complessiva delle cariche positive
e negative di un corpo:
\begin{itemize}
	\item densità volumica: \(\displaystyle \rho(P,t) = [C_{oulomb}/m^3] = \lim_{V \to 0} \frac{q}{V}, \quad q = \int_V \rho(P,t) d\tau\)
	\item densità superficiale: \(\displaystyle \sigma(P,t) = [C_{oulomb}/m^2] = \lim_{\Sigma \to 0} \frac{q}{\Sigma}, \quad q = \int_\Sigma \sigma(P,t) d\Sigma\)
\end{itemize}

\subsection{Corrente elettrica e densità di corrente}
\subsubsection*{Densità di corrente}
Si genera per conduzione elettrica attraverso due modi:
\begin{itemize}
	\item corrente di conduzione: moto delle cariche libere (es. nei metalli)
	\item corrente di convezione: moto delle cariche libere e/o vincolate (es. soluzioni elettrolitiche)
\end{itemize}
\[\vec{J}(P,t) = \rho^+ + v_d^+ + \rho^- + v_d^-  \qquad \begin{cases}
	\rho^+ \;\rightarrow\; \text{densità delle cariche positive} \\
	v_d^+ \;\rightarrow\; \text{velocità di deriva delle cariche positive} \\
	\rho^- \;\rightarrow\; \text{densità delle cariche negative} \\
	v_d^- \;\rightarrow\; \text{velocità di deriva delle cariche negative}
\end{cases}\]

\subsubsection*{Corrente}
La corrente è la quantità di cariche che attraversano una superficie in un'unità di tempo. Dipende dalla superficie e dal suo
orientamento. Non dipende dal resto dello spazio. Se si inverte l'orientamento della superficie o il riferimento, il segno della
corrente si inverte. Si misura in Ampère \([A_{mpere}] = [C_{oulomb}/s]\).
\[i(t) = \int_\Sigma \vec{J}(P,t) \cdot d\vec{\Sigma} \qquad \Leftrightarrow \qquad i(t) = \lim_{\Delta t \to 0} \frac{\Delta q_{\text{ attraverso }\Sigma}(t)}{\Delta t}\]
In caso di conduttori filiformi (dove \(\Sigma \ll \text{lunghezza}\)), vale \(i(t) = \vec{J} \cdot \vec{\Sigma}\)

\subsubsection*{Conservazione della carica e continuità della corrente}
La carica elettrica non si crea, non si distrugge, si conserva sempre.
\begin{align*}
	&q_\text{interna}(t + \Delta t) = q_\text{interna}(t) + \Delta q_\text{uscente} \\
	&i_\text{uscente}(t) = \lim_{\Delta t \to 0} \frac{\Delta q_\text{uscente}}{\Delta t} = -\frac{d q_\text{entrante}}{dt} = - i_\text{entrante}
\end{align*}
\begin{itemize}
	\item la variazione di carica corrisponde ad una corrente
	\item in assenza di corrente, la carica non varia
	\item la carica entrante è pari a quella uscente (in modulo)
\end{itemize}

\subsubsection*{Corrente solenoidale}
La corrente si dice solenoidale quando:
\begin{itemize}
	\item si è in regime stazionario: non si hanno accumuli o prelievi di carica in nessun punto del volume, la carica entrante
	e quella uscente sono uguali e il campo \(\vec{J}\) forma linee di flusso chiuse
	\item in regioni di carica nulla: \(\rho = 0\) ad esempio nei metalli
\end{itemize}
si è in regime stazionario

\subsubsection*{Tubo di flusso}
Il tubo di flusso è un conduttore rivestito da materiale isolante che può essere attraversato da corrente. In condizioni stazionarie
(con campo di corrente solenoidale) si ha che la corrente \(i_1\) attraverso una superficie \(\Sigma_1\) è uguale alla corrente
\(i_2\) attraverso una superficie \(\Sigma_2\). Ovvero non si hanno perdite di corrente: \(i_\text{uscente} = 0\).

\subsubsection*{Amperometro}
L'amperometro è uno strumento per misurare la corrente in un circuito. Il verso del sistema è dal \(+\) al \(-\) (ovvero la corrente
entra dal connettore \(+\) ed esce dal connettore \(-\)). Si usa in serie al circuito. Un amperometro si dice ideale se non influisce
sul circuito e se la misura avviene senza ritardi.

\begin{center}
	\begin{circuitikz}
		\draw (0,0) -- (0.21,0);
		\draw (0.42,0) [circ,thick] circle(0.2);
		\draw (1,0) [circ, thick] circle(0.38);
		\draw (1.58,0) [circ, thick] circle(0.2);
		\draw (1.78,0) -- (2,0);
		\draw[->] (0.6,-0.7) -- (1.4,-0.7);
		\node[] at (1,0) {A};
		\node[] at (0.42,0) {+};
		\node[] at (1.58,-0.02) {--};
		\node[] at (1,-0.9) {i};
	\end{circuitikz}
\end{center}

\newpage

\subsection{Campo elettrostatico}
\subsubsection*{Legge di Coulomb e campo elettrostatico}
Il campo elettrostatico si definisce a partire dalla forza di Coulomb, per questo è anche chiamato campo coulombiano.
\[\vec{F}_{1,2} = \frac{1}{4 \pi \varepsilon_0} \frac{q_1 \, q_2}{{r_{1,2}}^2} \hat{u}_{1,2} \qquad \qquad
\vec{F}_\text{elettr} = q \vec{E} \qquad \qquad
\vec{E} = \frac{\vec{F}}{q} = \frac{1}{4 \pi \varepsilon_0} \frac{q}{r^2} \hat{u}_{1,2} = [N/C_{oulob}] = [V_{olt}/m]\]
Il campo elettrostatico è additivo: \(\displaystyle \vec{E}(P) = \frac{1}{4 \pi \varepsilon_0} \sum_{k=1}^{n} \frac{q_k}{{r_{PO_k}}^2} \vec{u}_k(P) \qquad 
\vec{E}(P) = \frac{1}{4 \pi \varepsilon_0} \int_V \frac{\rho}{r^2} \vec{u}_k d\tau\)

\subsubsection*{Campo elettrostatico nei conduttori}
Un conduttore è un materiale che conduce corrente. Le cariche sono libere di muoversi e, muovendosi, generano una corrente. In
condizione di equilibrio il campo all'interno è nullo, ovvero non c'è nessuna forza che agisce sulle cariche e le cariche sono
ferme (altrimenti non ci sarebbe equilibrio).

\subsubsection*{Campo elettrostatico nei dielettrici - isolanti}
Un dielettrico o isolante è un materiale che non conduce corrente. Le cariche sono bloccate a meno di piccoli spostamenti responsabili
della polarizzazione dei dielettrici. I dielettrici possono essere:
\begin{itemize}
	\item omogenei: se \(\varepsilon\) non dipende dalla posizione
	\item lineari: se \(\varepsilon\) non dipende dal modulo del campo elettrico \(||\vec{E}||\)
	\item isotropi: se \(\varepsilon\) non dipende dalla direzione del campo elettrico \(\vec{u} = \vec{E} \big/ ||{\vec{E}}||\)
\end{itemize}
Un dielettrico omogeneo, lineare e isotropo si dice uniforme e per esso valgono tutte le leggi viste finora.

\subsubsection*{Permittività dielettrica di un mezzo}
La permittività dielettrica di un mezzo indica come tale mezzo reagisce al campo elettrico:
\[\varepsilon = \varepsilon_\text{relativa del mezzo} \cdot \varepsilon_\text{0 - nel vuoto} = [F_{araday}/m] = [{C_{oulomb}}^2/J] \qquad \quad \varepsilon > \varepsilon_0 \;\; \varepsilon_r > 1\]

\subsubsection*{Campo elettrico conservativo}
Il campo elettrostatico è conservativo ovvero:
\[\oint_\mathcal{L} = \vec{E} \cdot d\vec{l} = 0 \qquad \quad \oint_{\mathcal{L}_1} = \vec{E} \cdot d\vec{l} = \oint_{\mathcal{L}_2} = \vec{E} \cdot d\vec{l} \quad
\text{con} \; \begin{matrix} \mathcal{L}_{1,\text{iniziale}} = \mathcal{L}_{2,\text{iniziale}} \\ \mathcal{L}_{1,\text{finale}} = \mathcal{L}_{2,\text{finale}} \end{matrix}\]

\subsection{Potenziale elettrostatico}
\subsubsection*{Potenziale elettrostatico}
Essendo \(\vec{E}\) un campo conservativo, si definisce il potenziale elettrostatico:
\[\int_A^B \vec{E} \cdot d \vec{s} = -\Delta V = V_A - V_B \qquad \qquad V(P) = \int_P^C \vec{E} \cdot d \vec{s} = [V_{olt}] \qquad (V(C) = 0)\]

\subsubsection*{Lavoro di una forza elettrostatica}
Il lavoro compiuto dalla forza elettrostatica per spostare una carica \(q\) vale:
\[\mathcal{L}_{AB} = \int_A^B \vec{F} \cdot d \vec{l} = q \int_A^B \vec{E} \cdot d \vec{l} = - q \Delta V = q(V_A - V_B)\]

\subsubsection*{Energia potenziale elettrostatica}
Si definisce l'energia potenziale di una carica come il lavoro compiuto per portare la carica da distanza \(\infty\) alla posizione
in cui si trova: \[\psi(P) = q_0 V(P)\]

\subsubsection*{Sorgenti del campo elettrico - distribuzioni di carica}
\begin{itemize}
	\item \textbf{distribuzioni di carica statiche - condizioni elettrostatiche}: \\
	le cariche che generano il campo sono in quiete e non ci sono correnti
	\item \textbf{distribuzione di carica stazionarie - regime stazionario}: \\
	le cariche sono in moto a velocità costante nel tempo e di conseguenza sono presenti correnti costanti nel tempo
	\item \textbf{distribuzione di carica variabile - regime variabile}: \\
	le cariche sono in moto variabile e le correnti variano nel tempo, il campo non è più conservativo in quanto avrà una componente
	non conservativa \(\vec{E}_\text{indotta}\)
\end{itemize}

\subsubsection*{Forza di una carica in moto}
Le forze agenti su cariche in moto immerse in un campo magnetico hanno una componente dovuta al campo elettrico e una al campo
magnetico: \[\vec{F} = q_0 \vec{E} + q_0 \vec{v} \times \vec{B}\]

\subsection{Tensioni e forze elettromotrici}
\subsubsection*{Tensione}
La tensione è definita come il lavoro elettrico per unità di carica speso a muovere una carica elettrica di prova lungo una linea \(L\).
\[u(t) = \int_L \vec{E}(P,t) \cdot \vec{t}(P) \; dl = \frac{W_e}{q_0} = [V_{olt}] = [J/C] = [J/A_{mpere} \;\! s] \qquad \vec{t}(P) = \text{versore della curva in P} \]
Questa definizione permette di essere indipendenti dalla conservatività del campo elettrico: se il campo elettrico è conservativo,
la tensione equivale al potenziale (a meno di un segno), mentre se il campo elettrico non è conservativo non si definisce nessun
potenziale, ma si può calcolare lo stesso la tensione.
\[\begin{matrix}
	\text{campo conservativo} & \rightarrow & \text{potenziale} = -\text{tensione} \\
	\text{campo non conservativo} & \rightarrow & \xcancel{\text{potenziale}} = -\text{tensione}
\end{matrix}\]

\subsubsection*{Forza elettromotrice indotta}
Un campo elettrico non conservativo, è formato da una parte conservativa e da un campo elettrico indotto non conservativo.
Di conseguenza la circuitazione non è più nulla e si definisce la forza elettromotrice indotta \(fem\) come il lavoro
compiuto dal campo elettrico lungo una linea chiusa \(L\):
\[fem \;\; e(t) = \oint_L \vec{E}(P,t) \cdot \vec{t}(P) \; dl\]

\subsubsection*{Forza elettromotrice mozionale}
La forza elettromotrice complessiva agente sulle cariche in moto è data da una componente dovuta al campo elettrico e da una
dovuta al campo magnetico, detta forza elettromotrice mozionale, provocata dal movimento delle cariche in un campo magnetico:
\[\oint_L (\vec{E} + \vec{v}\times \vec{B}) \cdot \vec{t} \; dl = \oint_L \vec{E} \cdot \vec{t} \; dl +  \oint_L (\vec{v} \times \vec{B}) \cdot \vec{t} \; dl = e(t) + e_m(t)\]

\subsubsection*{Voltmetro}
Il voltmetro è uno strumento per misurare la tensione, si collega in parallelo alla sezione di circuito di cui si vuole conoscere
la tensione. La direzione del sistema è data dal vettore \(\vec{t}\) dal \(+\) al \(-\). Un voltmetro si dice ideale se non altera
il regime del circuito.

\begin{center}
	\begin{circuitikz}
		\draw (0,0) -- (0.21,0);
		\draw (0.42,0) [circ,thick] circle(0.2);
		\draw (1,0) [circ, thick] circle(0.38);
		\draw (1.58,0) [circ, thick] circle(0.2);
		\draw (1.78,0) -- (2,0);
		\node[] at (1,0) {V};
		\node[] at (0.42,0) {+};
		\node[] at (1.58,-0.02) {--};
		\node[] at (1,-0.6) {u};
	\end{circuitikz}
\end{center}

\section{Modello a parametri concentrati - \textit{mpc}}
\subsection{Teorema}
\subsubsection*{Obiettivo}
\begin{itemize}
	\item[1.] ogni componente elettrico si può modellare con equazioni algebriche o differenziali che dipendono solo da tensioni
	o correnti (ovvero non da componenti spaziali)
	\item[2.] le tensioni sono definite tra coppie di morsetti e le correnti sono definite ai terminali
	\item[3.] i terminali terminano con morsetti utilizzati per collegare il componente con il resto del circuito
\end{itemize}

\subsubsection*{Ipotesi}
\begin{itemize}
	\item[1.] all'esterno dei componenti elettrici
	\begin{itemize}[topsep=0pt]
		\item il campo \(\vec{E}\) è conservativo
		\item la densità \(\vec{J}\) è solenoidale (regime stazionario, no accumuli o prelievi)
	\end{itemize}
	\item[2.] i terminali e i morsetti sono superfici equipotenziali senza accumuli o prelievi di carica in essi
\end{itemize}

\subsubsection*{Teoria}
\begin{itemize}
	\item[1.] è possibile modellare ogni componente attraverso Equazioni
	\item[2.] per formare un circuito si collegano più morsetti tra loro attraverso conduttori detti connessioni o interconnessioni
	che soddisfano le ipotesi dei morsetti/terminali
	\item[3.] se più morsetti sono attaccati insieme, si formano nodi di volume e carica nulla per cui vale la legge dei nodi (o legge
	di continuità \(\rightarrow \;\; \sum i_\text{entranti} + \sum i_\text{uscenti} = 0\))
\end{itemize}

\subsection{Componenti}
\subsubsection*{Introduzione}
I componenti sono elementi del circuito:
\begin{itemize}
	\item rappresentati graficamente da una curva chiusa (detta superficie limite) con due (o più) tratti filiformi detti terminali
	con cui si possono collegare ad altri componenti
	\item modellabili con un'equazione differenziale o algebrica
\end{itemize}
Si definiscono:
\begin{itemize}
	\item corrente entrante e corrente uscente (per ogni morsetto) rappresentata con \(\rightarrow\)
	\item tensione (per ogni coppia di morsetti) rappresentata con \(+ \; -\)
\end{itemize}

\subsubsection*{Porte elettriche, bipoli, n-poli m-bipoli, (n-1)-bipoli}
\begin{itemize}
	\item \textbf{porta elettrica}: coppia di terminali in cui la corrente entrante in un terminale è pari alla corrente uscente
	dall'altro terminale; si definisce la corrente di porta \(i_{AB}(t)\) e la tensione di porta \(u_{AB}(t)\)
	\item \textbf{bipolo elettrico}: componente con due terminali
	\item \textbf{n-polo}: componente con n terminali
	\item \textbf{m-bipolo}: componente con \(m\) porte e \(2m\) terminali
	\item \textbf{n-polo come (n-1)-bipolo}: componente in cui si sceglie un polo \(N\) come polo di riferimento e si definiscono:
	\begin{itemize}[topsep=0pt]
		\item n-1 porte tra il polo di riferimento e gli altri n-1 poli del componente
		\item n-1 correnti \(i_{kN}(t)\) che entrano dagli n-1 morsetti ed escono dal morsetto \(N\), per cui la corrente uscente
		dal polo \(N\) è la somma di tutte le correnti di porta
		\item n-1 tensioni \(u_{kN}(t)\) di porta
	\end{itemize}
\end{itemize}

\subsection{Reti elettriche o circuiti}
\subsubsection*{Introduzione}
Una rete è formata da interconnessioni tra n-poli e m-bipoli. Le interconnessioni prendono il nome di nodi e i morsetti collegati
allo stesso nodo sono superfici equipotenziali.

\subsubsection*{Suddivisione}
\begin{itemize}
	\item \textbf{reti in regime stazionario}: valgono le ipotesi del \textit{mpc}
	\item \textbf{reti in regime variabile}: in teoria non si potrebbe applicare il \textit{mpc}, ma si definiscono le \dots
	\item \textbf{reti in regime variabile quasi stazionario}: ovvero reti in regime variabile in cui è possibile applicare
	il \textit{mpc} se i campi \(\vec{E}\) e \(\vec{J}\) variano lentamente nel tempo e non si hanno propagazioni di onde
	elettromagnetiche all'esterno dei componenti
\end{itemize}

\begin{description}[itemsep=0pt]
	\item[Tipologia] tipo dei componenti utilizzati nel circuito (resistivo, capacitivo, RLC, \dots)
	\item[Topologia] tipo di connessioni utilizzate nel circuito (serie, parallelo, \dots)
\end{description}

\subsection{Potenza di bipolo, convenzione dei generatori e utilizzatori}
\subsubsection*{Potenza di una porta}
Data una porta elettrica, la potenza è data dal prodotto:
\[p(t) = u(t) \cdot i(t) \qquad [W_\text{att}] = [V_\text{olt}] \cdot [A_\text{mpere}]\]
%\begin{align*}
%	p(t) = \frac{d \mathcal{L}_e(t)}{dt} &= \frac{d \mathcal{L}_e(t)}{dq} \cdot \frac{d q}{dt} = u(t) \cdot i(t) \\
%	&= \lim_{\Delta t \to 0} \frac{\Delta q (V_A - V_B)}{\Delta t} = i(t) \cdot \Delta V(t) = u(t) \cdot i(t)
%\end{align*}

\subsubsection*{Convenzione degli utilizzatori}
Un componente (o meglio una porta o bipolo) soddisfa la convenzione degli utilizzatori se la corrente entra nel morsetto \(+\).
Se \(p > 0\), la porta assorbe potenza, se \(p < 0\) la porta eroga energia.

\subsubsection*{Convenzione dei generatori}
Un componente (o meglio una porta o bipolo) soddisfa la convenzione dei generatori se la corrente entra nel morsetto \(-\).
Se \(p > 0\), la porta eroga energia, se \(p < 0\) la porta assorbe potenza.

\subsubsection*{Lavoro elettrico}
Il lavoro elettrico compiuto da una porta nell'intervallo \([t_0, t_1]\) vale:
\[\mathcal{L}(t_0,t_1) = \int_{t_0}^{t_1} p(t) dt = \int_{t_0}^{t_1} u(t) \cdot i(t) \qquad\qquad
\begin{matrix} [J_\text{oule}] = [W_\text{att}] \cdot [\text{sec}] \\[5pt] [kWh] = 3.6 \cdot 10^6 \; [J] \end{matrix}\]
Il lavoro è entrante per potenza entrante e viceversa.

\subsubsection*{Wattmetro}
Il wattmetro è uno strumento per misurare la potenza ed è costituito da un amperometro (in serie) e un voltmetro (in parallelo) combinati.

\begin{center}
	\begin{minipage}{0.25\textwidth}
		\begin{circuitikz}[european]
			\draw (-0.5,0) -- (0.21,0);
			\draw (0.42,0) [circ,thick] circle(0.2);
			\draw (1,0.58) [circ,thick] circle(0.2);
			\draw (1,0) [circ, thick] circle(0.38);
			\draw (1.58,0) [circ, thick] circle(0.2);
			\draw (1,-0.58) [circ,thick] circle(0.2);
			\draw (1.78,0) -- (2.5,0);
			\draw (0,0) -- (0,1) -- (1,1) -- (1,0.79);
			\node[] at (1,0) {W};
			\node[] at (2.5,-1) {U};
			\node[] at (1,0.58) {+};
			\node[] at (0.42,0) {+};
			\node[] at (1.58,-0.02) {--};
			\node[] at (1,-0.6) {--};
			\draw (2.5,0) to[R] (2.5,-2) -- (-0.5,-2);
			\draw (1,-0.79) -- (1,-2);
		\end{circuitikz}
	\end{minipage}
	\begin{minipage}{0.05\textwidth}
		\(\longrightarrow\)
	\end{minipage}
	\begin{minipage}{0.25\textwidth}
		\begin{circuitikz}[european]
			\draw (-0.5,0) -- (0.21,0);
			\draw (0.42,0) [circ,thick] circle(0.2);
			\draw (1,0) [circ, thick] circle(0.38);
			\draw (1.58,0) [circ, thick] circle(0.2);
			\draw (1.78,0) -- (2.5,0);
			\node[] at (1,0) {A};
			\node[] at (2.5,-1) {U};
			\node[] at (0.42,0) {+};
			\node[] at (1.58,-0.02) {--};
			\draw (2.5,0) to[R] (2.5,-2) -- (-0.5,-2);
		\end{circuitikz}
	\end{minipage}
	\begin{minipage}{0.05\textwidth}
		\(+\)
	\end{minipage}
	\begin{minipage}{0.25\textwidth}
		\begin{circuitikz}[european]
			\draw (1,-0.42) [circ,thick] circle(0.2);
			\draw (1,-1) [circ, thick] circle(0.38);
			\draw (1,-1.58) [circ,thick] circle(0.2);
			\node[] at (1,-1) {V};
			\node[] at (2.5,-1) {U};
			\node[] at (1,-0.42) {+};
			\node[] at (1,-1.6) {--};
			\draw (-0.5,0) -- (2.5,0) to[R] (2.5,-2) -- (-0.5,-2);
			\draw (1,0) -- (1,-0.22);
			\draw (1,-1.78) -- (1,-2);
		\end{circuitikz}
	\end{minipage}
\end{center}

\newpage

\subsubsection*{Potenza di un m-bipolo}
La potenza di un m-bipolo in cui tutte le porte sono convenzionate allo stesso modo è la somma di tutte le potenze delle singole porte.
\[p(t) = p_1(t) + p_2(t) + \dots + p_n(t) \qquad \qquad \mathcal{L}(t_0, t_1) = \int_{t_0}^{t_1} p(t) dt = \int_{t_0}^{t_1} \sum_{n=1}^{m} u_n(t) \cdot i_n(t) dt\]
In base al segno e alla convenzione utilizzata, si avrà potenza dissipata o erogata.

\subsubsection*{Potenza di un n-polo}
Per calcolare la potenza di un n-polo, uso la corrispondenza tra n-polo e (n-1)-bipolo.

\subsubsection*{Bipolo passivo}
Un bipolo si dice passivo se il lavoro elettrico uscente da un certo tempo \(t_0\) in poi è minore dell'energia immagazzinata
fino a \(t_0\). Ovvero un bipolo passivo è capace di accumulare energia, ma non ne può emettere più di quella che ha immagazzinato.
In condizioni stazionarie \(p_\text{uscente}(t) < 0\)
\[\mathcal{L}_\text{lavoro uscente}(t_0,t) \; < \; W_\text{energia immagazzinata}(t_0)\]

\subsubsection*{Bipolo attivo}
Un bipolo si dice attivo se per certe condizioni non è rispettata la legge sopra, ovvero se per certe condizioni vale:
\[\mathcal{L}_\text{lavoro uscente}(t_0,t) \; > \; W_\text{energia immagazzinata}(t_0)\]
In generale un bipolo attivo fornisce lavoro elettrico convertendolo da altre fonti e in condizioni stazionarie \(p_\text{uscente}(t) > 0\)

\newpage

\section{Componenti elettrici}
\subsection{Caratteristica esterna}
\subsubsection*{Caratteristica esterna di un bipolo e di un m-bipolo}
La caratteristica esterna è un'equazione che lega tutte le variabili (tensione e corrente) di ogni porta di un determinato
componente. Per un m-bipolo si avranno 2m variabili (tensione e corrente delle m-porte), per un bipolo si avranno 2 variabili.
\begin{itemize}
	\item caratteristica esterna m-bipolo: \(F(u_1(t), i_1(t), u_2(t), i_2(t),\dots, u_m(t), i_m(t)) = 0\)
	\item caratteristica esterna bipolo: \(F(u(t), i(t)) = 0\)
\end{itemize}

\subsubsection*{Bipolo controllato in corrente o in tensione}
È possibile scrivere l'equazione per un bipolo in funzione di una delle due variabili:
\begin{itemize}
	\item bipolo controllato in corrente: \(u(t) = f(i(t))\)
	\item bipolo controllato in tensione: \(i(t) = f(u(t))\)
\end{itemize}

\subsubsection*{Bipolo ideale}
Un bipolo si dice ideale se la sua caratteristica esterna è lineare a tratti, ovvero è possibile usare un'equazione lineare per
descriverne il comportamento in un intorno del punto \((U^*, I^*)\)

\subsection{Componenti}
\subsubsection*{Resistenza}
\begin{itemize}
	\item equazione costitutiva: \(\quad R=U/I \qquad [\Omega_\text{(ohm)}] = [V_\text{olt}] \; / \; [A_\text{mpere}]\)
	\item caratteristica esterna: \(\quad F(u,i) = 0 \; \rightarrow \; \frac{U}{I} - R = 0\)
	\item effetto Joule: \(\qquad \qquad \;\; \Delta Q = R \cdot I^2 \cdot \Delta t\)
	\item bilancio energetico: \(\quad \;\;\; \begin{matrix} P_\text{entrante} = u \cdot i \\ P_\text{uscente} = R \cdot i^2 \end{matrix} \qquad u \cdot i = R \cdot i^2 \; \rightarrow \; P_\text{entrante} = P_\text{uscente}\)
	\item resistività: \(\qquad\qquad\qquad R = \rho \cdot L/S \qquad \rho = \rho_0(1+ \alpha \Delta T) \qquad \begin{matrix} \rho_0 = \text{resistività a 20°C} \\ \Delta T = \text{\(\Delta\)temp. rispetto a 20°C} \end{matrix}\)
	\begin{itemize}[topsep=0pt]
		\item conduttori: \(\qquad\quad \rho_\text{Cu} = 1.8 \cdot 10^{-8} \Omega m\)
		\item semiconduttori: \(\quad \rho_\text{Si} = 2.3 \cdot 10^3 \Omega m\)
		\item isolanti: \(\qquad\qquad\; \rho_\text{PVC} = 10^{10} - 10^{13} \Omega m\)
	\end{itemize}
\end{itemize}

\subsubsection*{Resistore ideale}
\begin{center}
	\begin{tabularx}{\textwidth}{ c | c | X }
		modello grafico & equazioni costitutive & parametri caratteristici \\
		\begin{circuitikz} \draw (0,0) [R] to (2,0); \end{circuitikz} &
		\(\begin{cases} u(t) = R \, i(t) \\ i(t) = G \, u(t) \end{cases}\) &
		\(\begin{cases}
			R = \text{resistenza} \; [\Omega_\text{ohm}] \\
			G = \text{conduttanza} \; [S_\text{iemens}]
		\end{cases}\)
	\end{tabularx}
\end{center}

\subsubsection*{Cortocircuito ideale}
\begin{center}
	\begin{tabularx}{\textwidth}{ c | c | X }
		modello grafico & equazioni costitutive & descrizione \\
		\begin{circuitikz} \draw (0,0) -- (2,0); \end{circuitikz} &
		\(u(t) = 0 \quad \forall t\) &
		resistore con \(R = 0\)
	\end{tabularx}
\end{center}

\subsubsection*{Circuito aperto}
\begin{center}
	\begin{tabularx}{\textwidth}{ c | c | X }
		modello grafico & equazioni costitutive & descrizione \\
		\begin{circuitikz} \draw (0,0) -- (0.7,0) (1.3,0) -- (2,0); \end{circuitikz} &
		\(i(t) = 0 \quad \forall t\) &
		resistore con \(R = +\infty\)
	\end{tabularx}
\end{center}

\subsubsection*{Interruttore ideale}
\begin{center}
	\begin{tabularx}{\textwidth}{ c | X }
		modello grafico & descrizione \\
		\begin{circuitikz} \draw (0,0) to[nos] (2,0); \end{circuitikz} &
		\(\begin{array}{l}
			\text{dipolo in grado di commutarsi tra due stati:} \\
			\text{  1. cortocircuito ideale} \\
			\text{  2. circuito aperto} \end{array}\)
	\end{tabularx}
\end{center}

\subsubsection*{Interruttore reale}
\begin{center}
	\begin{tabularx}{\textwidth}{ c | X }
		modello grafico & descrizione \\
		\begin{circuitikz} \draw (0,0) to[nos] (2,0) (0,-0.2) -- (2,-0.2); \end{circuitikz} &
		unipolare: interrompe la continuità di un solo conduttore
	\end{tabularx}
\end{center}

\subsubsection*{Diodo ideale}
\begin{center}
	\begin{tabularx}{\textwidth}{ c | X }
		modello grafico & descrizione \\
		\begin{circuitikz} \draw (0,0) to[empty diode] (2,0); \end{circuitikz} &
		\(\begin{array}{l}
			\text{dipolo in grado di commutarsi tra due stati:} \\
			\text{ se } u>0 \rightarrow \text{ conduzione} \\
			\text{ se } u<0 \rightarrow \text{ interdizione}
		\end{array}\)
	\end{tabularx}
\end{center}

\subsubsection*{Diodo reale}
\begin{center}
	\begin{tabularx}{\textwidth}{ c | X }
		modello grafico & descrizione \\
		\begin{circuitikz} \draw (0,0) to[empty diode] (2,0); \end{circuitikz} &
		\(\begin{array}{l}
			\text{dipolo con tre stati:} \\
			\text{ se } u>0 \rightarrow \text{ conduzione} \\
			\text{ se } V_{bd}<u<0 \rightarrow \text{ interdizione} \\
			\text{ se } u<V_{bd} \rightarrow \text{ rottura/conduzione} \\
			\text{ottenuto con giunzione PN (materiali drogati positivamente o negativamente) }
		\end{array}\)
	\end{tabularx}
\end{center}

\subsubsection*{Induttore ideale}
\begin{center}
	\begin{tabularx}{\textwidth}{ c | c | c | X }
		modello grafico & equazioni costitutive & parametri caratteristici & descrizione \\
		\begin{circuitikz} \draw (0,0) [L] to (2,0); \end{circuitikz} &
		\(\displaystyle u(t) = L \frac{di(t)}{dt}\) &
		\(L = \text{induttanza} \; [H_\text{enry}]\) &
		in regime stazionario agisce come un cortocircuito
	\end{tabularx}
\end{center}

\subsubsection*{Induttore reale}
Usato nei circuiti AC, \(\displaystyle e(t) = -\frac{d\Phi}{dt}\)

\subsubsection*{Condensatore ideale}
\begin{center}
	\begin{tabularx}{\textwidth}{ c | c | c | X }
		modello grafico & equazioni costitutive & parametri caratteristici & descrizione \\
		\begin{circuitikz} \draw (0,0) [C] to (2,0); \end{circuitikz} &
		\(i(t) = C \frac{du(t)}{dt}\) &
		\(C = \text{capacità} \; [C_\text{oulomb}]\) &
		in regime stazionario agisce come un circuito aperto
	\end{tabularx}
\end{center}

\subsubsection*{Condensatore reale}
Usato nei circuiti AC, \(\displaystyle \oint \vec{D} \cdot d\vec{\Sigma} = q_\text{int}\)

\newpage

\subsection{Generatori}
\subsubsection*{Forze elettriche generatrici e generatori}
In condizioni stazionarie si ha che \(\vec{J}\) è solenoidale e che \(Q = L + \Delta W \rightarrow Q = L\) ovvero il lavoro
compiuto dal circuito è tutto dissipato in calore dalle resistenze (\(\Delta W = 0\)). Per cui si ha:
\[Q = L = \oint \vec{F}_\text{gen} \cdot d \vec{l} \neq 0\]
Esistono delle forze generatrici non conservative \(\vec{F}_\text{gen}\) la cui circuitazione non è nulla che mettono in moto
le cariche. Le parti del circuito in cui si sviluppano tali forze non conservative sono detti generatori.

\subsubsection*{Forza elettrica generatrice specifica}
Si definisce quindi la forza elettrica generatrice specifica come forza per unità di carica:
\[\vec{E}_\text{gen} = \frac{\vec{F}_\text{gen}}{q} \qquad [N_\text{ewton} / C_\text{oulomb}] = [V_\text{olt} / m]\]

\subsubsection*{Generatori a vuoto}
\begin{itemize}
	\item si hanno forze \(\vec{E}_\text{gen}\) che inducono accumuli di cariche ai capi del generatore e l'accumulo forma
	un campo \(\vec{E}_\text{coulombiano} \; (\vec{E}_c)\) contrario a \(\vec{E}_\text{gen}\)
	\item in regime stazionario \(\vec{E}_g = -\vec{E}_c\) e le cariche non si muovono (o hanno \(a = 0\), \(v\) costante)
	\item si ottiene che la forze elettromotrice è pari alla tensione ai capi del generatore:
	\[fem_{AB} = \int_B^A \vec{E}_g(P,t) \cdot d\vec{l} = \int_B^A -\vec{E}_c(P,t) \cdot d\vec{l} = \int_A^B \vec{E}_c(P,t) \cdot d\vec{l} = u_{0,AB}(t)\]
	\item se \(U_{0,AB}(t)\) è costante, il generatore è in regime DC e \(E_{AB} = U_{0,AB}\)
	\item se \(U_{0,AB}(t)\) è sinusoidale, il generatore è in regime AC e \(E_{AB}(t) = u_{0,AB}(t)\)
	\item il campo \(\vec{E}_c\) tende a muovere le cariche positive dal \(+\) al \(-\)
	\item il campo \(\vec{E}_g\) tende a muovere le cariche positive dal \(-\) al \(+\)
\end{itemize}

\subsubsection*{Generatori a carico}
\begin{itemize}
	\item \(\vec{E}_g \neq -\vec{E}_c\), si ha uno spostamento di cariche
	\item la caratteristica esterna vale \(U = E_{AB} - R_i I\) con \(R_i = \) resistenza interna
\end{itemize}

\subsubsection*{Generatori ideali}
\begin{tabularx}{0.6\textwidth}{ c  X }
	generatore ideale di tensione \(U = E\) &
	\begin{circuitikz}
		%\draw (0,0) -- (0.58,0);
		%\draw(1,0) [circ, thick] circle(0.42);
		%\draw (1.42,0) -- (2,0);
		%\node [] at (0.8,0) {--};
		%\node [] at (1.2,0) {+};
		\draw (0,0) to[V] (2,0);
		\node [] at (1,-0.7) {E};
		\node [] at (0.2,-0.25) {--};
		\node [] at (1.8,-0.25) {+};
	\end{circuitikz} \\
	\midrule
	generatore ideale di corrente \(I = J\) &
	\begin{circuitikz}
		%\draw (0,0) -- (0.58,0);
		%\draw(1,0) [circ, thick] circle(0.42);
		%\draw (1.42,0) -- (2,0);
		%\draw [->] (0.7,0) -- (1.3,0);
		\draw (0,0) to[I] (2,0);
		\draw[->] (0.6,-0.7) -- (1.4,-0.7);
		\node[] at (1,-0.9) {J};
	\end{circuitikz}
\end{tabularx}

\subsubsection*{Esempi di generatori}
\begin{itemize}
	\item elettrochimici (sede di reazioni chimiche, esempio pile a secco e accumulatori)
	\item fotovoltaici (fotone "convertito" in elettrone, funzionamento basato su fotodiodi)
	\item termoelettrici (giunzioni bimetalliche a temperature diverse formano una f.e.m. per effetto Seebeck)
	\item piezoelettrici (cristalli soggetti a stress meccanico formano un campo elettrico, quindi una f.e.m.)
	\item elettromeccanici (macchine rotanti con di statore e rotore, convertono energia meccanica in elettrica)
\end{itemize}

\newpage

\section{Topologia delle reti}
\subsection{Introduzione}
La topologia di un circuito indica come gli n-poli sono connessi tra loro. Le interconnessioni tra i componenti sono dette
nodi del circuito.

\subsection{Teoria dei grafi}
\subsubsection*{Dal circuito al grafo}
Ad ogni circuito si associa un grafo:
\begin{itemize}
	\item i nodi del grafo corrispondono ai nodi del circuito
	\item gli archi del grafo corrispondono ai componenti
	\item ad ogni arco viene associata una tensione e una corrente
	\item gli archi sono orientati secondo le correnti e le tensioni
\end{itemize}

\subsubsection*{Componenti connesse}
Se si hanno grafi non connessi, le componenti connesse corrispondono a sottocircuiti non connessi tra loro che verranno
analizzati separatamente.

\subsubsection*{Definizioni}
\begin{itemize}
	\item \textbf{insieme di taglio}: insieme di archi che intersecano una superficie chiusa che taglia il circuito. La
	superficie di taglio può isolare nodi singoli o parti del circuito
	\item \textbf{nodo}: punto di interconnessione tra due o più archi del grafo
	\item \textbf{maglia}: insieme di archi che costituiscono un percorso chiuso che tocca ciascun nodo una sola volta
	\item \textbf{grafo piano}: grafo che può essere distribuito su un piano senza che gli archi si intersechino
	\item \textbf{anello}: maglia che non racchiude altre parti del grafo al suo interno
	\item \textbf{albero}: insieme di archi del grafo che collegano tutti i nodi del grafo senza formare maglie
	\item \textbf{coalbero}: insieme di archi del grafo complementare ad un albero
\end{itemize}
Per un grafo vale: \(\quad m_\text{ anelli} = l_\text{ archi} - n_\text{ nodi} + 1\)

\subsection{Leggi di Kirchhoff}
\subsubsection*{Legge di Kirchhoff delle correnti - LKC}
In un circuito chiuso la corrente entrante ed uscente in un insieme di taglio è nulla. Se l'insieme di taglio racchiude solo un
nodo, si avrà che la somma delle correnti entranti ed uscenti in un nodo è nulla.
\[\sum_{h \in \mathcal{T}_\text{insieme di taglio}} \alpha_h i_h(t) = 0\]
\begin{itemize}
	\item le correnti entranti nella sezione di taglio (o in un nodo) hanno \(\alpha_h = -1\)
	\item le correnti uscenti dalla sezione di taglio (o da un nodo) hanno segno \(\alpha_h = +1\)
\end{itemize}

\subsubsection*{Legge di Kirchhoff delle tensioni - LKT}
In un circuito chiuso, la somma di tutte le tensioni associate ai componenti di ogni maglia è nulla.
\[\sum_{h \in \mathcal{M}_\text{maglia}} \beta_h u_h(t) = 0\]
\begin{itemize}
	\item ad ogni maglia si associa una orientazione (oraria o antioraria)
	\item le tensioni concordi con il verso della maglia hanno \(\beta_h = +1\)
	\item le tensioni discordi con il verso della maglia hanno segno \(\beta_h = -1\)
\end{itemize}

\newpage

\subsection{Sistemi di equazioni topologiche}
\subsubsection*{Equazioni linearmente indipendenti}
Il sistema di equazioni topologiche è un sistema formato dalle equazioni LKT e LKC. Per avere un'unica soluzione e descrivere
completamente un circuito, devo avere il massimo numero di equazioni LKT/LKC linearmente indipendenti.

\subsubsection*{Equazioni indipendenti in grafo connesso}
Un grafo connesso ha:
\begin{itemize}
	\item \(n-1\) equazioni LKC indipendenti con insiemi di taglio (o nodi) che insistono su un solo arco dell'albero (detto lato
	peculiare specifico del taglio) e su altri archi del coalbero
	\item \(l-n+1\) equazioni LKT indipendenti con maglie costituite da un solo arco del coalbero (detto lato peculiare specifico
	della maglia) e altri archi dell'albero
	\item in totale \((l-n+1)+(n-1) = l\) equazioni indipendenti: una per ogni componente o arco del grafo
\end{itemize}

\subsubsection*{Equazioni indipendenti in grafo sconnesso}
Un grafo con \(p\) componenti connesse ha:
\begin{itemize}
	\item \(n_k-1\) equazioni LKC indipendenti per ogni componente connessa \(k \in 1,2,\dots,p\)
	\item \(l_k-n_k+1\) equazioni LKT indipendenti per ogni componente connessa \(k \in 1,2,\dots,p\)
	\item complessivamente \(\sum_{k=1}^p (l_k-n_k+1)+(n_k-1) = \sum_{k=1}^p l_k = l\) equazioni indipendenti, ovvero una per ogni
	componente o arco del grafo (anche se sconnesso)
\end{itemize}

\subsubsection*{Conservazione delle potenze}
La somma di tutte le potenze istantanee è nulla. La somma delle potenze entranti negli utilizzatori è uguale alla somma delle
potenze uscenti dai generatori.
\[P_\text{tot} = \sum_{h=1}^{l} u_h(t) \cdot i_h(t) = 0 \qquad\qquad \sum_{h_\text{utilizzatori}} P_{entrante,h}(t) = \sum_{h_\text{generatori}} P_{uscente,h}(t)\]

\subsection{Connessione in serie}
\subsubsection*{In generale}
\begin{itemize}
	\item due bipoli sono collegati in serie se hanno un solo morsetto in comune
	\item due bipoli si dicono in serie diretta se hanno anche lo stesso riferimento di tensione e corrente
\end{itemize}
\[\begin{cases} u_\text{serie}(t) = u_1(t) + u_2(t) \\ i_\text{serie}(t) = i_1(t) = i_2(t) \end{cases}\]

\subsubsection*{Serie di generatori ideali di corrente}
Non è possibile creare una serie di generatori ideali di corrente con \(J_1 \neq J_2\) in quanto nel nodo in comune ai due
generatori non è rispettata la LKC. Analogamente non può esistere una serie con un GIC e un circuito aperto.

\subsubsection*{Serie di un generatore ideale di tensione e un resistore}
In una serie tra generatore ideale di tensione e resistore ideale si ha:
\begin{itemize}
	\item \(I_\text{serie} = I_\text{resistore} = -I_\text{generatore} \;\; \rightarrow \;\;\) il generatore ha convenzione opposta
	\item \(U_\text{serie} = U_\text{generatore} + U_\text{resistore} = E + R I_2 = E - R I\)
\end{itemize}
Per determinare la curva caratteristica della serie servono almeno due punti del piano \(u,i\):
\begin{itemize}
	\item funzionamento a vuoto: \(u_0 = E = fem,\; i_0 = 0\)
	\item funzionamento a cortocircuito \(u_{cc} = 0,\; I_{cc} = E/R\)
\end{itemize}

\subsubsection*{Serie di resistori ideali e partitore di tensione}
\begin{itemize}
	\item \(u_\text{serie} = u_1(t) + u_2(t) + \dots + u_n(t) = (R_1 + R_2 + \dots + R_n) \cdot i_\text{serie}(t) = R_\text{eq,serie} \cdot i_\text{serie}(t)\)
	\item la resistenza equivalente della serie è la somma delle resistenze: \(R_\text{eq} = R_1 + R_2 + \dots + R_n\) 
	\item le tensioni si ripartiscono in modo proporzionale alle resistenze: \(u_k(t) = R_k / R_\text{eq} \cdot u_\text{serie}(t)\)
\end{itemize}

\subsection{Connessione in parallelo}
\subsubsection*{In generale}
\begin{itemize}
	\item due bipoli sono collegati in parallelo se entrambi i morsetti sono in comune
	\item due bipoli sono in parallelo diretto se hanno i riferimenti di tensione e corrente concordi
\end{itemize}
\[\begin{cases} u_\text{serie}(t) = u_1(t) = u_2(t) \\ i_\text{serie}(t) = i_1(t) + i_2(t) \end{cases}\]

\subsubsection*{Parallelo di generatori ideali di tensione}
Non è possibile creare un parallelo di generatori ideali di tensione con \(E_1 \neq E_2\) in quanto nella maglia ottenuta non
viene rispettata la \(LKT\). Analogamente non è possibile creare un parallelo tra GIT e cortocircuito.

\subsubsection*{Parallelo di un generatore ideale di corrente e un resistore}
In una serie tra generatore ideale di tensione e resistore ideale si ha:
\begin{itemize}
	\item \(U_\text{serie} = U_\text{resistore} = -U_\text{generatore} \;\; \rightarrow \;\;\) il generatore ha convenzione opposta
	\item \(I_\text{serie} = I_\text{generatore} + I_\text{resistore} = J + G U_2 = J - GU\)
\end{itemize}

\subsubsection*{Parallelo di resistori ideali e partitore di corrente}
\begin{itemize}
	\item \(i_\text{parallelo} = i_1(t) + i_2(t) + \dots + i_n(t) = (G_1 + G_2 + \dots + G_n) \cdot u_\text{parallelo}(t) = G_\text{eq,parallelo} \cdot u_\text{parallelo}(t)\)
	\item la conduttanza equivalente del parallelo è la somma delle conduttanze: \(G_\text{eq} = G_1 + G_2 + \dots + G_n\) 
	\item le correnti si ripartiscono in modo proporzionale alle conduttanze: \(i_k(t) = G_k / G_\text{eq} \cdot i_\text{parallelo}(t)\)
\end{itemize}

\subsection{Resistenze e conduttanze equivalenti di porta}
\begin{itemize}
	\item in una rete di resistori (passivi) è possibile definire una resistenza equivalente di porta \(R_\text{eq,AB} = U_\text{AB}/J_\text{AB}\),
	per cui è possibile sostituire la rete passiva con un resistore equivalente \(R_\text{eq,AB}\).
	\item analogo per le conduttanze
\end{itemize}

\subsubsection*{Resistenze a stella e a triangolo}
È possibile semplificare le reti di resistenze a maglie a triangolo con reti a stella attraverso la relazione:
\[R_{\Delta \; \text{triangolo}} = 3 R_{y \; \text{stella}}\]

\newpage

\section{Analisi di circuiti lineari a corrente continua}
\subsection{Componenti lineari - GLC, GLT}
Un circuito in regime stazionario in corrente continua formato solo da bipoli è lineare se costituito esclusivamente da
generatori lineari di tensione (GLT) o generatori lineari di corrente (GLC).
\subsubsection*{Generatore lineare di tensione - GLT}
Un generatore lineare di tensione è formato da una serie di generatore ideale di tensione e un resistore, entrambi con
convenzione di utilizzatore.
\begin{center}
	\begin{circuitikz}
		\draw (0,0) to[V] (2,0) to[R=\(R\)] (4,0);
		\node [] at (1,0.7) {\(E_\text{GIT}\)};
		\node [] at (0.3,-0.2) {--};
		\node [] at (1.7,-0.2) {+};
		\node [] at (2.3,-0.2) {--};
		\node [] at (3.7,-0.2) {+};
		\node [] at (-0.3,0) {--};
		\node [] at (4.3,0) {+};
		\node [draw,fill,circle,inner sep=1pt] at(0,0) {};
		\node [draw,fill,circle,inner sep=1pt] at(2,0) {};
		\node [draw,fill,circle,inner sep=1pt] at(4,0) {};
		\draw (0.5,-0.8) [<-] -- (3.5,-0.8);
		\node [] at (2,-1.1) {\(I_{GLT}\), \(U_\text{GLT}\)};
	\end{circuitikz}
\end{center}
Il sistema ha come equazione costitutiva:
\[U_\text{GLT} = U_\text{GIT} + U_\text{res} = E_\text{GIT} + R \cdot I_\text{GLT}\]

\subsubsection*{Generatore lineare di corrente - GLC}
Un generatore lineare di corrente è formato da un parallelo di generatore ideale di corrente e un resistore, entrambi con
convenzione di utilizzatore.

\begin{center}
	\begin{circuitikz}
		\draw (0,0) -- (0.7,0) -- (0.7,0.7) (3.3,0.7) to[I_=\(\)] (0.7,0.7) (3.3,0.7) -- (3.3,0) -- (4,0);
		\draw (0,0) -- (0.7,0) -- (0.7,-0.7) to[R=\(R\)] (3.3,-0.7) -- (3.3,0) -- (4,0);
		\node [] at (2,1.4) {\(J_\text{GIC}\)};
		\node [] at (1.3,0.5) {--};
		\node [] at (2.7,0.5) {+};
		\node [] at (1.3,-0.5) {--};
		\node [] at (2.7,-0.5) {+};
		\node [] at (-0.3,0) {--};
		\node [] at (4.3,0) {+};
		\node [draw,fill,circle,inner sep=1pt] at(0,0) {};
		\node [draw,fill,circle,inner sep=1pt] at(4,0) {};
		\node [draw,fill,circle,inner sep=1pt] at(0.7,0) {};
		\node [draw,fill,circle,inner sep=1pt] at(3.3,0) {};
		\draw (0.5,-1.2) [<-] -- (3.5,-1.2);
		\node [] at (2,-1.5) {\(I_{GLC}\), \(U_\text{GLC}\)};
	\end{circuitikz}
\end{center}
Il sistema ha come equazione costitutiva:
\[I_\text{GLC} = I_\text{GIC} + I_\text{res} = J_\text{GIC} + \frac{U_\text{GLC}}{R}\]

\subsubsection*{Sostituzione da GLT a GLC}
È possibile sostituire un GLT con un GLC:
\[I_\text{GLC} = I_\text{res} + J_\text{GIC} = \frac{U_\text{GLT}}{R} - \frac{E_\text{GIT}}{R} \qquad\qquad J_\text{GIC} = \frac{E_\text{GIT}}{R}\]

\subsubsection*{Sostituzione da GLC a GLT}
È possibile sostituire un GLT con un GLC:
\[U_\text{GLT} = U_\text{res} + E_\text{GIT} = R \cdot I_\text{GLC} - R \cdot J_\text{GIC} \qquad\qquad E_\text{GIT} = R \cdot J_\text{GIC}\]

\newpage

\subsection{Circuiti lineari ed equazioni indipendenti}
Un circuito lineare è descritto da equazioni linearmente indipendenti lineari di 1° grado ottenute dalle leggi di Kirchhoff o
dalle equazioni costitutive dei GLT e GLC. Si ottiene un sistema lineare \(A x = b\) risolvibile attraverso il calcolo
matriciale \(x = A^{-1} b\). \\
Il sistema ha \(2l\) incognite:
\begin{itemize}
	\item \(u_h(t)\) per ogni bipolo \(h = 1,2,3, \dots l\)
	\item \(i_h(t)\) per ogni bipolo \(h = 1,2,3, \dots l\)
\end{itemize}
Il sistema ha \((n-1) + (l-n+1) + l = 2l\) equazioni indipendenti:
\begin{itemize}
	\item \(n-1\) equazioni indipendenti per LKC
	\item \(l-n+1\) equazioni indipendenti per LKT
	\item \(l\) equazioni costitutive (una per ogni GLC e GLT)
\end{itemize}
Il sistema ha, quindi, \(2l\) equazioni linearmente indipendenti con \(2l\) incognite e \(\text{rango}\{A\} = 2l\), ovvero ha
un'unica soluzione.

\subsection{Metodi di analisi delle reti lineari}
\subsubsection*{Metodo diretto o di sostituzione}
Se un bipolo ha una certa tensione \(u_h(t)\) e corrente \(i_h(t)\) data dalla soluzione unica del sistema, è possibile sostituire
tale bipolo con:
\begin{itemize}
	\item un generatore ideale di tensione con \(e_h(t) = u_h(t)\)
	\item un generatore ideale di corrente con \(j_h(t) = i_h(t)\)
\end{itemize}
In questo modo si ottiene una rete semplificata e di conseguenza un sistema più semplice da risolvere.

\subsubsection*{Sovrapposizione degli effetti}
La tensione di un certo bipolo è data dalla somma dei contributi di tensione (detti tensioni parziali) di ogni generatore ideale
di tensione o corrente preso singolarmente, mentre tutti gli altri sono spenti.
\[u_h(t) = \sum_r u_{h,E_r}(t) + \sum_s u_{h,J_s}(t) \qquad\qquad u_{h,E_r}(t) = \alpha_{hr} \cdot e_r(t) \;\;\; / \;\;\; u_{h,J_s}(t) = R_{hs} \cdot j_s(t)\]
La corrente di un certo bipolo è data dalla somma dei contributi di corrente (detti correnti parziali) di ogni generatore ideale
di tensione o corrente preso singolarmente, mentre tutti gli altri sono spenti.
\[i_h(t) = \sum_r i_{h,E_r}(t) + \sum_s i_{h,J_s}(t) \qquad\qquad i_{h,E_r}(t) = G_{hr} \cdot e_r(t) \;\;\; / \;\;\; i_{h,J_s}(t) = \beta_{hs} \cdot j_s(t) \]
I coefficienti \(\alpha_{hr}, R_{hr}, \beta_{hr}, G_{hr}\) sono detti coefficienti di rete e sono costanti in una rete lineare.

\subsubsection*{Correnti cicliche o di anello}
\begin{itemize}
	\item si semplifica il sistema introducendo delle correnti fittizie dette correnti cicliche di anello, ottenute matematicamente
	da un cambio di variabile nelle LKC, il sistema finale avrà soltanto \(m\) equazioni, ovvero una per maglia
	\item è richiesto avere soltanto generatori ideali di tensione convenzionati da generatore: i GIC/GLC vanno trasformati
	in GIT/GLC con l'aggiunta delle seguenti condizioni \(E_h = U_h, \quad J_h = I_{Ar} - I_{As}\)
	\item per ogni anello \(A_k\) vale:
\end{itemize}
\[R_{A_{kk}} \cdot I_{A_k} - \sum_r (R_{A_{kr}} \cdot I_{A_r}) = E_{A_k} \qquad
\begin{array}{l}
	R_{A_{kk}} = \sum \text{resistenze della maglia } A_k \\
	R_{A_{kr}} = \sum \text{resistenze in comune alle maglie } A_k \text{ e } A_r \\
	E_{A_k} = \sum \text{fem di anello della maglia } A_k \text{ con conv. generatore} \\
	I_{A_k} = \text{corrente di anello della maglia } A_k \\
	I_{A_r} = \text{corrente di anello della maglia } A_r \\
\end{array}\]

\subsubsection*{Potenziali nodali}
\begin{itemize}
	\item si considerano come incognite solo i potenziali ai nodi del circuito in modo da soddisfare le LKT, per cui il sistema
	finale avrà solo \(n-1\) equazioni
	\item si sceglie un nodo di riferimento a cui si assegna potenziale nullo \(V_{rif} = 0\) e si calcolano i potenziali degli
	altri \(n-1\) nodi in modo da rispettare la relazione di tensione \(U_{kh} = V_{N_k}-V_{N_h}\)
	\item è richiesto avere soltanto generatori ideali di corrente convenzionati da generatore: i GIT/GLT vanno trasformati in
	GIC/GLC con l'aggiunta delle seguenti condizioni \(J_h = I_h, \quad V_{N_r}-V_{N_s} = E_h\)
	\item per ogni nodo \(N_k\) vale:
\end{itemize}
\[G_{N_{kk}} \cdot V_{N_k} - \sum_{r} (G_{N_{kr}} \cdot V_{N_r}) = J_{N_k} \qquad
\begin{array}{l}
	G_{N_{kk}} = \sum \text{conduttanze incidenti al nodo } N_k \\
	G_{N_{kr}} = \sum \text{conduttanze comprese tra i nodi } N_k \text{ e } N_r \\
	J_{N_k} = \sum \text{correnti dei GIC impresse su } N_k \text{ (+ se entranti)} \\
	V_{N_k} = \text{potenziale del nodo } N_k \\
	V_{N_r} = \text{potenziale del nodo } N_r \\
\end{array}\]

\subsubsection*{Metodo di riduzione - Teorema di Thévenin}
\begin{itemize}
	\item si individua una superficie di taglio che intercetta solo due conduttori, per cui è possibile individuare una porta 
	costituita dai morsetti \(a,b\)
	\item è possibile sostituire la parte di circuito individuata dalla superficie di taglio con un generatore di Thévenin
	costituito da un generatore lineare di tensione con:
\end{itemize}
\begin{align*}
	U_{eq} &= E_{ab} = U_{0,ab} = \text{tensione a vuoto della porta ab} \\
	R_{eq} &= R_{ab} = \frac{U_{0,ab}}{I_{cc,ab}} = \text{resistenza della rete inerte (con tutti i GIT/GIC spenti)} \\
	&\qquad\qquad U_{ab} = U_{0,ab} - R_{eq} \cdot I_{cc,ab} \;\;\rightarrow\;\; \text{equazione del GLT}
\end{align*}

\subsubsection*{Metodo di riduzione - Teorema di Norton}
\begin{itemize}
	\item si individua una superficie di taglio che intercetta solo due conduttori, per cui è possibile individuare una porta 
	costituita dai morsetti \(a,b\)
	\item è possibile sostituire la parte di circuito individuata dalla superficie di taglio con un generatore di Norton
	costituito da un generatore lineare di corrente con:
\end{itemize}
\begin{align*}
	I_{eq} &= J_{ab} = I_{cc,ab} = \text{corrente di cortocircuito della porta ab} \\
	R_{eq} &= R_{ab} = \frac{U_{0,ab}}{I_{cc,ab}} = \text{resistenza della rete inerte (con tutti i GIT/GIC spenti)} \\
	&\qquad\qquad I_{ab} = I_{cc,ab} - \frac{U_{ab}}{R_{eq}} \;\;\rightarrow\;\; \text{equazione del GLC}
\end{align*}

\newpage

\subsection{Rendimento di un generatore elettrico lineare}
Sia dato un circuito costituito da un generatore lineare e una resistenza
\begin{figure}[h]
	\centering
	\begin{minipage}{0.25\textwidth}
		\begin{circuitikz}
			\draw (0,0) to[V] (0,2) to[R=\(R_i\)] (2,2) to (2.5,2);
			\draw (2.5,2) to[R=\(R_u\)] (2.5,0) to (0,0);
			\node [] at (0.7,1) {\(E\)};
			\node [] at (-0.25,1.6) {+};
		\end{circuitikz}
	\end{minipage}
	\begin{minipage}{0.7\textwidth}
		\begin{align*}
			\text{potenza dissipata da \(R_u\)} \quad &P_{R_u} = U I = R_u I^2 = R_u \frac{E^2}{(R_u + R_i)^2} \\
			\text{potenza dissipata da \(R_i\)} \quad &P_{R_i} = U I = R_i I^2 = R_i \frac{E^2}{(R_u + R_i)^2} \\
			\text{potenza erogata dal GIT} \quad &P_{g} = U I = E I = \frac{E^2}{(R_u + R_i)} \\
			\text{rendimento} \quad \eta = \frac{P_{R_u}}{P_g} = &\frac{P_g-P_{R_i}}{P_g} = 1-\frac{P_{R_i}}{P_g} = 1-{R_i} \;\frac{I}{E} = \frac{R_u}{R_i + R_u}
		\end{align*}
	\end{minipage}
\end{figure}

Si osserva che:
\begin{itemize}
	\item il rendimento è sempre \(\leq 1\)
	\item in caso di generatore a vuoto \(\eta = 1\), in quanto \(I = 0\)
	\item in caso di cortocircuito \(\eta = 0\), in quanto \(E = 0\)
	\item nel mezzo si ha un andamento lineare \(\eta = 1 - R_i \; I / E\)
\end{itemize}

\subsubsection*{Rendimento nei circuiti di potenza}
Nei circuiti di potenza si vogliono rendimenti massimi, ovvero quando \(R_u \gg R_i\).

\subsubsection*{Rendimento nei circuiti di segnale}
Nei circuiti di segnale si vuole il massimo rapporto segnale/rumore e si analizzano le potenze massime. Analizzando la potenza dissipata
da \(R_u\) in funzione del carico si osserva che si ha massima potenza dissipata quando \(R_i = R_u\), con un rendimento di \(1/2\).
\[\frac{\partial P_{R_u}}{\partial R_u} = 0 \;\; \Leftrightarrow \;\; \frac{R_i - R_u}{(R_i + R_u)^3}E^2 = 0 \;\; \Leftrightarrow \;\; R_i = R_u \qquad\quad \eta = \frac{R_u}{R_i + R_u} = 0.5\]

\subsection{Punto di lavoro}
Si immagina di avere una rete di bipoli di ordine 0 e non necessariamente lineare. Nel seguente caso, si ha un fotodiodo
convenzionato da utilizzatore e un glt convenzionato da generatore con le rispettive caratteristiche esterne.

\begin{figure}[h]
	\centering
	\begin{minipage}{0.3\textwidth}
		\begin{circuitikz}
			\draw (0,0) to[V] (2,0) to[R=\(R\)] (4,0) to (4,1.5);
			\draw (0,0) to (0,1.5) to[pD] (4,1.5);
			\node [] at (1,0.7) {\(E\)};
			\node [] at (1.3,1.7) {+};
			\node [] at (2.7,1.7) {--};
			\node [] at (0.3,-0.25) {+};
		\end{circuitikz}
	\end{minipage}
	\begin{minipage}{0.2\textwidth}
		\centering
		\begin{subfigure}[h]{0.8\textwidth}
			\includegraphics[width=\textwidth]{immagini/photodiode.png}
			\caption*{caratt. esterna fotodiodo}
		\end{subfigure}
	\end{minipage}
	\begin{minipage}{0.2\textwidth}
		\begin{subfigure}[h]{0.8\textwidth}
			\includegraphics[width=\textwidth]{immagini/glt.png}
			\caption*{caratt. esterna glt}
		\end{subfigure}
	\end{minipage}
	\begin{minipage}{0.2\textwidth}
		\begin{subfigure}[h]{0.9\textwidth}
			\includegraphics[width=\textwidth]{immagini/puntodilavoro.png}
			\caption*{punto di lavoro}
		\end{subfigure}
	\end{minipage}
\end{figure}
\begin{itemize}
	\item Il punto di lavoro del circuito è il punto in cui le equazioni di Kirchhoff per le correnti e le tensioni sono
	soddisfatte e corrisponde al punto di intersezione delle caratteristiche esterne.
	\item Siccome l'intersezione si trova nel secondo quadrante, significa che l'utilizzatore (il fotodiodo) eroga energia
	e il generatore la assorbe. L'area evidenziata è la potenza erogata dal fotodiodo.
	\item In questo caso, le curve caratteristiche si intersecano in un unico punto, per cui la rete ha un'unica soluzione.
	Se le due curve caratteristiche si intersecano in più punti, non è possibile determinare a priori in quale dei punti
	la rete andrà a lavorare.
\end{itemize}

\newpage

\section{Doppi bipoli}
\subsection{Introduzione ai doppi bipoli}
\subsubsection*{Doppi bipoli generici}
I doppi bipoli sono componenti elettrici costituiti da due porte.
\begin{center}
	\begin{minipage}{0.3\textwidth}
		\begin{circuitikz}
			\draw (0,0) -- (1,0) -- (1,1.5) -- (0,1.5) -- (0,0);
			\draw (0,0.4) -- (-0.5,0.4);
			\node [] at (-0.7,0.4) {\(1'\)};
			\draw (0,1.1) -- (-0.5,1.1);
			\node [] at (-0.7,1.1) {\(1\)};
			\draw (1,0.4) -- (1.5,0.4);
			\node [] at (1.7,0.4) {\(2'\)};
			\draw (1,1.1) -- (1.5,1.1);
			\node [] at (1.7,1.1) {\(2\)};
		\end{circuitikz}
	\end{minipage}
	\begin{minipage}{0.4\textwidth}
		\(1-1'\) porta primaria, \(I_1 = I_{1'}\) \\[5pt]
		\(2-2'\) porta secondaria, \(I_2 = I_{2'}\)
	\end{minipage}
\end{center}
\begin{itemize}
	\item è possibile collegare i doppi bipoli con altri componenti attraverso le due porte
	\item valgono sempre le LKT, le LKC e le sezioni di taglio
\end{itemize}

\subsubsection*{Doppi bipoli di ordine zero}
I doppi bipoli di ordine zero sono doppi bipoli con:
\begin{itemize}
	\item 2 porte convenzionate da utilizzatore
	\item 2 equazioni costitutive di ordine zero, cioè senza derivate
	\item 4 variabili (\(v_1, v_2, i_1, i_2\)) di cui due sono grandezze libere e due sono grandezze pilotate esprimibili in
	funzione delle due grandezze libere
\end{itemize}

\begin{center}
	\begin{minipage}{0.25\textwidth}
		\begin{circuitikz}
			\draw (0,0) -- (1,0) -- (1,1.5) -- (0,1.5) -- (0,0);
			\draw (0,0.4) -- (-0.5,0.4);
			\draw (0,1.1) -- (-0.5,1.1);
			\draw (1,0.4) -- (1.5,0.4);
			\draw (1,1.1) -- (1.5,1.1);
			
			\node [] at (-0.9,1.3) {\(i_1\)};
			\draw[->] (-1.2,1.1) -- (-0.6,1.1);
			\node [] at (1.9,1.3) {\(i_2\)};
			\draw[->] (2.2,1.1) -- (1.6,1.1);
			\node [] at (-0.8,0.7) {\(u_1\)};
			\node [] at (1.8,0.7) {\(u_2\)};
		
			\node [] at (-0.4,0.2) {--};
			\node [] at (-0.4,1.3) {+};
			\node [] at (1.4,0.2) {--};
			\node [] at (1.4,1.3) {+};
		\end{circuitikz}
	\end{minipage}
	\begin{minipage}{0.7\textwidth}
		\[\begin{cases} f_1(v_1, v_2, i_1, i_2) = 0 \\ f_2(v_1, v_2, i_1, i_2) = 0 \end{cases} \Rightarrow \;\;
		\begin{cases} v_1 = f(i_1, i_2) \\ v_2 = g(i_1,i_2) \end{cases}, \;\;
		\begin{cases} v_1 = f(v_2, i_2) \\ i_1 = g(v_2,i_2) \end{cases}, \;\; ... \]
	\end{minipage}
\end{center}

\subsubsection*{Potenza nei doppi bipoli}
La potenza dei doppi bipoli vale: \(p_{entrante} = p_{e1} + p_{e2} = v_1 i_1 + v_2 i_2\), per un generico circuito vale la
conservazione delle potenze \(p_{tot} = 0\)

\subsubsection*{Doppi bipoli ideali e inerti}
\begin{itemize}
	\item \textbf{doppi bipoli ideali}: se modellabili con un sistema di eq. lineari del tipo \(Y = A \cdot X + B\)
	\item \textbf{doppi bipoli ideali inerti}: se \(B = 0 \;\; \Rightarrow \;\; Y = A \cdot X\), per cui sono rappresentabili
	attraverso una delle 6 rappresentazioni in base alle variabili indipendenti e controllate.
\end{itemize}

\subsection{Rappresentazioni dei doppi bipoli ideali inerti}
\subsubsection*{Rappresentazione controllata in corrente}
\[\begin{cases}
	v_1 = R_{11} i_1 + R_{12} i_2 \\
	v_2 = R_{21} i_1 + R_{22} i_2
\end{cases} \quad \Leftrightarrow \quad
\vec{v} = R \cdot \vec{i} \qquad
R = \left( \begin{matrix}
	R_{11} = \frac{v_1}{i_1} \Big|_{i_2=0} = [\Omega] & R_{12} = \frac{v_1}{i_2} \Big|_{i_1=0} = [\Omega] \\[8pt]
	R_{21} = \frac{v_2}{i_1} \Big|_{i_2=0} = [\Omega] & R_{22} = \frac{v_2}{i_2} \Big|_{i_1=0} = [\Omega]
\end{matrix} \right)\]
\begin{itemize}
	\item con \(R\) matrice di resistenza o matrice a vuoto
	\item per misurare \(R_{11}\) e \(R_{21}\), impongo la corrente \(i_1\), lascio la porta 2 a vuoto e misuro \(v_1\) e \(v_2\)
	\item per misurare \(R_{12}\) e \(R_{22}\), impongo la corrente \(i_2\), lascio la porta 1 a vuoto e misuro \(v_1\) e \(v_2\)
\end{itemize}

\subsubsection*{Rappresentazione controllata in tensione}
\[\begin{cases}
	i_1 = G_{11} v_1 + G_{12} v_2 \\
	i_2 = G_{21} v_1 + G_{22} v_2
\end{cases} \quad \Leftrightarrow \quad
\vec{i} = G \cdot \vec{v} \qquad
G = \left( \begin{matrix}
	G_{11} = \frac{i_1}{v_1} \Big|_{v_2=0} = [S] & G_{12} = \frac{i_1}{v_2} \Big|_{v_1=0} = [S] \\[8pt]
	G_{21} = \frac{i_2}{v_1} \Big|_{v_2=0} = [S] & G_{22} = \frac{i_2}{v_2} \Big|_{v_1=0} = [S]
\end{matrix} \right)\]
\begin{itemize}
	\item con \(G\) matrice di conduttanza o matrice di cortocircuito
	\item per misurare \(G_{11}\) e \(G_{21}\), impongo la tensione \(v_1\), cortocircuito la porta 2 e misuro \(i_1\) e \(i_2\)
	\item per misurare \(G_{12}\) e \(G_{22}\), impongo la tensione \(v_2\), cortocircuito la porta 1 e misuro \(i_1\) e \(i_2\)
\end{itemize}

\subsubsection*{Prima rappresentazione ibrida}
\[\begin{cases}
	v_1 = h_{11} i_1 + h_{12} v_2 \\
	i_2 = h_{21} i_1 + h_{22} v_2
\end{cases} \quad \Leftrightarrow \quad
\binom{v_1}{i_2} = h \cdot \binom{i_1}{v_2} \qquad
h = \left( \begin{matrix}
	h_{11} = \frac{v_1}{i_1} \Big|_{v_2=0} = [\Omega] & h_{12} = \frac{v_1}{v_2} \Big|_{i_1=0} = [/] \\[8pt]
	h_{21} = \frac{i_2}{i_1} \Big|_{v_2=0} = [/] & h_{22} = \frac{i_2}{v_2} \Big|_{i_1=0} = [S]
\end{matrix} \right)\]
\begin{itemize}
	\item con \(h\) prima matrice ibrida perché composta da grandezze ibride \([R]\) e \([S_{iemens}]\)
	\item per misurare \(h_{11}\) e \(h_{21}\), impongo la corrente \(i_1\), cortocircuito la porta 2 e misuro \(v_1\) e \(i_2\)
	\item per misurare \(h_{12}\) e \(h_{22}\), impongo la tensione \(v_2\), lascio la porta 1 a vuoto e misuro \(v_1\) e \(i_2\)
\end{itemize}

\subsubsection*{Seconda rappresentazione ibrida}
\[\begin{cases}
	i_1 = g_{11} v_1 + g_{12} i_2 \\
	v_2 = g_{21} v_1 + g_{22} i_2
\end{cases} \quad \Leftrightarrow \quad
\binom{i_1}{v_2} = g \cdot \binom{v_1}{i_2} \qquad
g = \left( \begin{matrix}
	g_{11} = \frac{i_1}{v_1} \Big|_{i_2=0} = [S] & g_{12} = \frac{i_1}{i_2} \Big|_{v_1=0} = [/] \\[8pt]
	g_{21} = \frac{v_2}{v_1} \Big|_{i_2=0} = [/] & g_{22} = \frac{v_2}{i_2} \Big|_{v_1=0} = [\Omega]
\end{matrix} \right)\]
\begin{itemize}
	\item con \(g\) seconda matrice ibrida perché composta da grandezze ibride \([R]\) e \([S_{iemens}]\)
	\item per misurare \(g_{11}\) e \(g_{21}\), impongo la tensione \(v_1\), lascio la porta 2 a vuoto e misuro \(i_1\) e \(v_2\)
	\item per misurare \(g_{12}\) e \(g_{22}\), impongo la corrente \(i_2\), cortocircuito la porta 1 e misuro \(i_1\) e \(v_2\)
\end{itemize}

\subsubsection*{Prima rappresentazione di trasmissione}
\[\begin{cases}
	v_1 = A v_2 - B i_2 \\
	i_1 = C v_2 - D i_2
\end{cases} \quad \Leftrightarrow \quad
\binom{v_1}{i_1} = T \cdot \binom{v_2}{-i_2} \qquad
T = \left( \begin{matrix}
	A = \frac{v_1}{v_2} \Big|_{i_2=0} & B = -\frac{v_1}{i_2} \Big|_{v_2=0} \\[8pt]
	C = \frac{i_1}{v_2} \Big|_{i_2=0} & D = -\frac{i_1}{i_2} \Big|_{v_2=0}
\end{matrix} \right)\]
\begin{itemize}
	\item con \(T\) prima matrice di trasmissione
	\item siccome si dovrebbe imporre sia \(v_2\) che \(i_2\) contemporaneamente, per determinare \(A, B, C, D\) si misurano
	i reciproci \(1/A, 1/B, 1/C, 1/D\) imponendo le opportune grandezze alle opportune porte
\end{itemize}

\subsubsection*{Seconda rappresentazione di trasmissione}
\[\begin{cases}
	v_2 = A' v_1 + B' i_1 \\
	-i_2 = C' v_1 + D' i_1
\end{cases} \quad \Leftrightarrow \quad
\binom{v_2}{-i_2} = T' \cdot \binom{v_1}{i_1} \qquad
T' = \left( \begin{matrix}
	A' = \frac{v_2}{v_1} \Big|_{i_1=0} & B' = \frac{v_2}{i_1} \Big|_{v_1=0} \\[8pt]
	C' = -\frac{i_2}{v_1} \Big|_{i_1=0} & D' = -\frac{i_2}{i_1} \Big|_{v_1=0}
\end{matrix} \right)\]
\begin{itemize}
	\item con \(T'\) seconda matrice di trasmissione
	\item siccome si dovrebbe imporre sia \(v_1\) che \(i_1\) contemporaneamente, per determinare \(A', B', C', D'\) si misurano
	i reciproci \(1/A', 1/B', 1/C', 1/D'\) imponendo le opportune grandezze alle opportune porte
\end{itemize}

\subsubsection*{Matrici e funzioni di trasferimento}
\begin{itemize}
	\item una matrice di trasferimento (che rappresenta una funzione di trasferimento) si può esprimere come rapporto tra causa
	ed effetto: \(A = \frac{\text{effetto}}{\text{causa}} = \frac{X(s)}{U(s)}\)
	\item siccome per determinare le matrici \(R, G, h, g\), si imporre una causa per misurarne l'effetto, sono dette matrici di trasferimento
	\item siccome per determinare le matrici \(T, T'\) non è possibile imporre la causa previsa, allora non sono matrici di trasferimento
\end{itemize}

\subsubsection*{Cambio di rappresentazioni}
Per i doppi bipoli ideali inerti per cui valgono tutte le rappresentazioni, allora vale:
\[R = G^{-1} \qquad\qquad h = g^{-1} \qquad\qquad T = T'^{-1}\]

\subsubsection*{Cambio dei riferimenti nelle rappresentazioni}
Se cambio i riferimenti ad una porta, è necessario invertire i segni dei parametri mutui \(X_{12}\) e \(X_{21}\) per mantenere
la stessa rappresentazione. Di seguito un esempio di doppi bipoli in rappresentazione controllata in corrente in cui si invertono
i riferimenti alla porta 2. 

\begin{center}
	% --- primo circuito ---
	\begin{minipage}{0.25\textwidth}
		\centering
		\begin{circuitikz}
			\draw (0,0) -- (1,0) -- (1,1.5) -- (0,1.5) -- (0,0);
			\draw (0,0.4) -- (-0.5,0.4);
			\draw (0,1.1) -- (-0.5,1.1);
			\draw (1,0.4) -- (1.5,0.4);
			\draw (1,1.1) -- (1.5,1.1);
			
			\node [] at (-0.9,1.3) {\(i_1\)};
			\draw[->] (-1.2,1.1) -- (-0.6,1.1);
			\node [] at (1.9,1.3) {\(i_2\)};
			\draw[->] (2.2,1.1) -- (1.6,1.1);
			\node [] at (-0.8,0.7) {\(u_1\)};
			\node [] at (1.8,0.7) {\(u_2\)};
		
			\node [] at (-0.4,0.2) {--};
			\node [] at (-0.4,1.3) {+};
			\node [] at (1.4,0.2) {--};
			\node [] at (1.4,1.3) {+};

			\node [] at (0.2,0.4) {\(1'\)};
			\node [] at (0.15,1.1) {\(1\)};
			\node [] at (0.85,0.4) {\(2'\)};
			\node [] at (0.8,1.1) {\(2\)};
		\end{circuitikz}
	\end{minipage}
	\begin{minipage}{0.1\textwidth}
		\centering
		\(\Longrightarrow\)
	\end{minipage}
	\begin{minipage}{0.55\textwidth}
		\centering
		\(\left(\begin{matrix} v_1 \\ v_2 \end{matrix}\right) = R \left(\begin{matrix} i_1 \\ i_2 \end{matrix}\right) \quad\leftrightarrow\quad \begin{cases}
			v_1 = R_{11} i_1 + R_{12} i_2 \\
			v_2 = R_{21} i_1 + R_{22} i_2
		\end{cases}\)
	\end{minipage}

	% --- vertical space ---
	\begin{minipage}{\textwidth}
		\(\big.\)
	\end{minipage}

	% --- secondo circuito ---
	\begin{minipage}{0.25\textwidth}
		\centering
		\begin{circuitikz}
			\draw (0,0) -- (1,0) -- (1,1.5) -- (0,1.5) -- (0,0);
			\draw (0,0.4) -- (-0.5,0.4);
			\draw (0,1.1) -- (-0.5,1.1);
			\draw (1,0.4) -- (1.5,0.4);
			\draw (1,1.1) -- (1.5,1.1);
			
			\node [] at (-0.9,1.3) {\(i_1\)};
			\draw[->] (-1.2,1.1) -- (-0.6,1.1);
			\node [] at (1.9,0.2) {\(i_2\)};
			\draw[->] (2.2,0.4) -- (1.6,0.4);
			\node [] at (-0.8,0.7) {\(u_1\)};
			\node [] at (1.8,0.8) {\(u_2\)};
		
			\node [] at (-0.4,0.2) {--};
			\node [] at (-0.4,1.3) {+};
			\node [] at (1.4,0.2) {+};
			\node [] at (1.4,1.3) {--};

			\node [] at (0.2,0.4) {\(1'\)};
			\node [] at (0.15,1.1) {\(1\)};
			\node [] at (0.85,0.4) {\(2'\)};
			\node [] at (0.8,1.1) {\(2\)};
		\end{circuitikz}
	\end{minipage}
	\begin{minipage}{0.1\textwidth}
		\centering
		\(\Longrightarrow\)
	\end{minipage}
	\begin{minipage}{0.55\textwidth}
		\centering
		\(\left(\begin{matrix} v_1 \\ -v_2 \end{matrix}\right) = R \left(\begin{matrix} i_1 \\ -i_2 \end{matrix}\right) \;\leftrightarrow\; \begin{cases}
			v_1 = R_{11} i_1 - R_{12} i_2 \\
			v_2 = -R_{21} i_1 + R_{22} i_2
		\end{cases}\) \\[5pt]
		\(\left(\begin{matrix} v_1 \\ v_2 \end{matrix}\right) = R^* \left(\begin{matrix} i_1 \\ i_2 \end{matrix}\right) \;\leftrightarrow\; \begin{cases}
			v_1 = R_{11}^* i_1 + R_{12}^* i_2 \\
			v_2 = R_{21}^* i_1 + R_{22}^* i_2
		\end{cases}\)
	\end{minipage}
	
	% --- vertical space ---
	\begin{minipage}{\textwidth}
		\(\big.\)
	\end{minipage}

	% --- equazioni corrispondenti ---
	\begin{minipage}{0.3\textwidth}
		\centering
		grandezze corrispondenti:
	\end{minipage}
	\begin{minipage}{0.4\textwidth}
		\centering
		\(R^* = \left(\begin{matrix} R_{11}^* & R_{12}^* \\ R_{21}^* & R_{22}^* \end{matrix}\right) = \left(\begin{matrix} R_{11} & -R_{12} \\ -R_{21} & R_{22} \end{matrix}\right)\)
	\end{minipage}
\end{center}

\subsection{Collegamenti di doppi bipoli}
\begin{multicols}{2}
	\subsubsection*{Collegamento in cascata}
	\begin{center}
		\includegraphics[width=0.8\linewidth]{immagini/dbp_cascata.png}
	\end{center}
	\[T_{eq}' = T_b' \cdot T_a'\]
	
	\subsubsection*{Serie di doppi bipoli}
	\begin{center}
		\includegraphics[width=0.7\linewidth]{immagini/dbp_serie.png}
	\end{center}
	\[R_{eq} = R_a + R_b\]
	
	\subsubsection*{Parallelo di doppi bipoli}
	\begin{center}
		\includegraphics[width=0.9\linewidth]{immagini/dbp_parallelo.png}
	\end{center}
	\[G_{eq} = G_a + G_b\]

	\columnbreak

	\subsubsection*{Collegamento ibrido serie/parallelo}
	\begin{center}
		\includegraphics[width=0.8\linewidth]{immagini/dbp_serie-parallelo.png}
	\end{center}
	\[h_{eq} = h_a + h_b\]
	
	\vspace{20pt}

	\subsubsection*{Collegamento ibrido parallelo/serie}
	\begin{center}
		\includegraphics[width=0.8\linewidth]{immagini/dbp_parallelo-serie.png}
	\end{center}
	\[g_{eq} = g_a + g_b\]
	
	\vspace{30pt}
	\(\;\) % invisible character
\end{multicols}

\newpage

\subsection{Trasformatore ideale}
\subsubsection*{Trasformatore ideale}
Un trasformatore ideale è un doppio bipolo solitamente in rappresentazione di trasmissione in grado di modificare la tensione
in ingresso di un certo fattore \(n\) detto rapporto di trasformazione.

\begin{center}
	\begin{minipage}{0.3\textwidth}
		\centering
		\includegraphics[width=\textwidth]{immagini/trasformatore.png}
	\end{minipage}
	\begin{minipage}{0.07\textwidth}
		\(\big.\)
	\end{minipage}
	\begin{minipage}{0.3\textwidth}
		\[\begin{cases}
			\displaystyle v_1(t) = n \, v_2(t) \\[7pt]
			\displaystyle i_1(t) = -\frac{1}{n} i_2(t)
		\end{cases}\]
	\end{minipage}
	\begin{minipage}{0.25\textwidth}
		\[T = \left(\begin{matrix}
			h & 0 \\ 0 & 1/h
		\end{matrix}\right)\]
	\end{minipage}
\end{center}
Siccome non è possibile isolare \(i_1\) e \(i_2\), oppure \(v_1\) e \(v_2\), non è possibile usare le rappresentazioni controllate
in corrente \(R\) o in tensione \(G\), mentre è possibile usare le altre 4.

\subsubsection*{Trasformatore reale}
I trasformatori reali sono utilizzabili solo in corrente alternata, servono per trasformare la tensione o per isolare due parti
di circuito. I trasformatori ideali sono solo utilizzati per modellare componenti elettrici.

\subsubsection*{Proprietà trasformatori ideali}
\begin{itemize}
	\item[1.] trasparente alla potenza: \(p_{e1} + p_{u2} = 0\)
	\item[2.] passivo: la potenza istantanea entrante è nulla
	\item[3.] amplifica la tensione [o corrente] in ingresso, riducendone la corrente [o tensione]
	\item[4.] trasporto: se alla porta 2 è collegato un resistore \(R_2\), dalla porta 1 appare come una resistenza equivalente
	\(R_{eq} = n^2 R_2\): \(\;\;v_1(t) = n v_2(t) = n (-R_2 i_2(t)) = - n R_2(-n i_1(t)) = n^2 R_2 i_1(t) = R_{eq} i_1(t)\)
\end{itemize}

\subsection{Generatori pilotati}
I generatori pilotati sono generatori ideali rappresentati come doppi bipoli ideali. Una porta impone la corrente [o la tensione]
mentre l'altra controlla la corrente [o la tensione] impressa.

\subsubsection*{Generatore di tensione pilotato in tensione - GTPT}
\begin{center}
	\begin{minipage}{0.3\textwidth}
		\centering
		\begin{circuitikz}
			\draw (0,0) -- (2,0) -- (2,1.5) -- (0,1.5) -- (0,0);
			\draw (-0.5,0.4) -- (0.5,0.4) (0.5,1.1) -- (-0.5,1.1);
			\draw (2.5,0.4) -- (1.5,0.4) -- (1.5,1.1) -- (2.5,1.1);
			\draw (1.5,0.5) -- (1.75,0.75) -- (1.5,1) -- (1.25,0.75) -- (1.5,0.5);
			\node [draw,fill,circle,inner sep=1pt] at(0.5,0.4) {};
			\node [draw,fill,circle,inner sep=1pt] at(0.5,1.1) {};

			\node [] at (-0.9,1.3) {\(i_1\)};
			\draw[->] (-1.2,1.1) -- (-0.6,1.1);
			\node [] at (2.9,1.3) {\(i_2\)};
			\draw[->] (3.2,1.1) -- (2.6,1.1);
			\node [] at (-0.5,0.7) {\(v_1\)};
			\node [] at (2.5,0.7) {\(v_2\)};

			\node [] at (-0.4,0.2) {--};
			\node [] at (-0.4,1.3) {+};
			\node [] at (2.4,0.2) {--};
			\node [] at (2.4,1.3) {+};
			\node [] at (1.3,1) {+};
		\end{circuitikz}
	\end{minipage}
	\begin{minipage}{0.6\textwidth}
		\[\begin{cases}
			i_1 = 0 \\
			v_2 = e_2 = k_\alpha v_1
		\end{cases} \qquad g = \left(\begin{matrix} 0 & 0 \\ k_\alpha & 0 \end{matrix}\right)\]
	\end{minipage}
\end{center}

\subsubsection*{Generatore di tensione pilotato in corrente - GTPC}
\begin{center}
	\begin{minipage}{0.3\textwidth}
		\centering
		\begin{circuitikz}
			\draw (0,0) -- (2,0) -- (2,1.5) -- (0,1.5) -- (0,0);
			\draw (-0.5,0.4) -- (0.5,0.4) -- (0.5,1.1) -- (-0.5,1.1);
			\draw (2.5,0.4) -- (1.5,0.4) -- (1.5,1.1) -- (2.5,1.1);
			\draw (1.5,0.5) -- (1.75,0.75) -- (1.5,1) -- (1.25,0.75) -- (1.5,0.5);
			\node [draw,fill,circle,inner sep=1pt] at(0.5,0.4) {};
			\node [draw,fill,circle,inner sep=1pt] at(0.5,1.1) {};

			\node [] at (-0.9,1.3) {\(i_1\)};
			\draw[->] (-1.2,1.1) -- (-0.6,1.1);
			\node [] at (2.9,1.3) {\(i_2\)};
			\draw[->] (3.2,1.1) -- (2.6,1.1);
			\node [] at (-0.5,0.7) {\(v_1\)};
			\node [] at (2.5,0.7) {\(v_2\)};

			\node [] at (-0.4,0.2) {--};
			\node [] at (-0.4,1.3) {+};
			\node [] at (2.4,0.2) {--};
			\node [] at (2.4,1.3) {+};
			\node [] at (1.3,1) {+};
		\end{circuitikz}
	\end{minipage}
	\begin{minipage}{0.6\textwidth}
		\[\begin{cases}
			v_1 = 0 \\
			v_2 = e_2 = k_r i_1
		\end{cases} \qquad R = \left(\begin{matrix} 0 & 0 \\ k_r & 0 \end{matrix}\right)\]
	\end{minipage}
\end{center}

\subsubsection*{Generatore di corrente pilotato in tensione - GCPT}
\begin{center}
	\begin{minipage}{0.3\textwidth}
		\centering
		\begin{circuitikz}
			\draw (0,0) -- (2,0) -- (2,1.5) -- (0,1.5) -- (0,0);
			\draw (-0.5,0.4) -- (0.5,0.4) (0.5,1.1) -- (-0.5,1.1);
			\draw (2.5,0.4) -- (1.5,0.4) -- (1.5,0.5) (1.5,1) -- (1.5,1.1) -- (2.5,1.1);
			\draw (1.75,0.75) -- (1.25,0.75);
			\draw (1.5,0.5) -- (1.75,0.75) -- (1.5,1) -- (1.25,0.75) -- (1.5,0.5);
			\node [draw,fill,circle,inner sep=1pt] at(0.5,0.4) {};
			\node [draw,fill,circle,inner sep=1pt] at(0.5,1.1) {};

			\draw[->] (1.2,0.9) -- (1.2,0.5);
			\node [] at (1.05,0.7) {\(j\)};

			\node [] at (-0.9,1.3) {\(i_1\)};
			\draw[->] (-1.2,1.1) -- (-0.6,1.1);
			\node [] at (2.9,1.3) {\(i_2\)};
			\draw[->] (3.2,1.1) -- (2.6,1.1);
			\node [] at (-0.5,0.7) {\(v_1\)};
			\node [] at (2.5,0.7) {\(v_2\)};

			\node [] at (-0.4,0.2) {--};
			\node [] at (-0.4,1.3) {+};
			\node [] at (2.4,0.2) {--};
			\node [] at (2.4,1.3) {+};
			\node [] at (1.3,1) {+};
		\end{circuitikz}
	\end{minipage}
	\begin{minipage}{0.6\textwidth}
		\[\begin{cases}
			i_1 = 0 \\
			i_2 = j_2 = k_g v_1
		\end{cases} \qquad G = \left(\begin{matrix} 0 & 0 \\ k_g & 0 \end{matrix}\right)\]
	\end{minipage}
\end{center}

\subsubsection*{Generatore di corrente pilotato in corrente - GCPC}
\begin{center}
	\begin{minipage}{0.3\textwidth}
		\centering
		\begin{circuitikz}
			\draw (0,0) -- (2,0) -- (2,1.5) -- (0,1.5) -- (0,0);
			\draw (-0.5,0.4) -- (0.5,0.4) -- (0.5,1.1) -- (-0.5,1.1);
			\draw (2.5,0.4) -- (1.5,0.4) -- (1.5,0.5) (1.5,1) -- (1.5,1.1) -- (2.5,1.1);
			\draw (1.75,0.75) -- (1.25,0.75);
			\draw (1.5,0.5) -- (1.75,0.75) -- (1.5,1) -- (1.25,0.75) -- (1.5,0.5);
			\node [draw,fill,circle,inner sep=1pt] at(0.5,0.4) {};
			\node [draw,fill,circle,inner sep=1pt] at(0.5,1.1) {};

			\draw[->] (1.2,0.9) -- (1.2,0.5);
			\node [] at (1.05,0.7) {\(j\)};

			\node [] at (-0.9,1.3) {\(i_1\)};
			\draw[->] (-1.2,1.1) -- (-0.6,1.1);
			\node [] at (2.9,1.3) {\(i_2\)};
			\draw[->] (3.2,1.1) -- (2.6,1.1);
			\node [] at (-0.5,0.7) {\(v_1\)};
			\node [] at (2.5,0.7) {\(v_2\)};

			\node [] at (-0.4,0.2) {--};
			\node [] at (-0.4,1.3) {+};
			\node [] at (2.4,0.2) {--};
			\node [] at (2.4,1.3) {+};
			\node [] at (1.3,1) {+};
		\end{circuitikz}
	\end{minipage}
	\begin{minipage}{0.6\textwidth}
		\[\begin{cases}
			v_1 = 0 \\
			i_2 = j_2 = k_\beta i_1
		\end{cases} \qquad h = \left(\begin{matrix} 0 & 0 \\ k_\beta & 0 \end{matrix}\right)\]
	\end{minipage}
\end{center}

\subsubsection*{Osservazioni sui generatori pilotati}
\begin{itemize}
	\item Non esistono componenti reali corrispondenti, ma si utilizzano per modellare altri componenti, ad esempio per il transistor
	NPN come composizione di diodi, resistori e GP.
	\item Ogni tipo di generatore ha solo una delle quattro rappresentazioni R, G, h, g e tutti hanno la rappresentazione \(T'\).
\end{itemize}

\subsection{Circuiti lineari in corrente continua con doppi bipoli}
\subsubsection*{Introduzione e teoremi}
\begin{itemize}
	\item in circuito in corrente continua con doppi bipoli si dice lineare se è costituito solo da GRT, GRC e doppi bipoli lineari
	inerti di ordine zero
	\item valgono ancora le LKC e LKT e si può modellare il circuito con il sistema lineare \(Ax = b\)
	\item valgono i teoremi di sostituzione, sovrapposizione degli effetti, Thévenin e Norton
	\item vale il principio di conservazione delle potenze
\end{itemize}

\subsubsection*{Resistenza equivalente di porta}
Si definisce la resistenza equivalente di porta in una sezione di circuito con generatori pilotati come la resistenza misurata quando
si applica una corrente di 1 \(A_{mpere}\), dopo aver spento tutti i generatori ideali e mantenendo accesi tutti i generatori pilotati.

\subsubsection*{Conduttanza equivalente di porta}
Si definisce la conduttanza equivalente di porta in una sezione di circuito con generatori pilotati come la resistenza misurata quando
si applica una tensione di 1 \(V_{olt}\), dopo aver spento tutti i generatori ideali e mantenendo accesi tutti i generatori pilotati.

\subsubsection*{Esercizi con la sovrapposizione degli effetti}
Quando si applica la sovrapposizione degli effetti in un circuito con generatori pilotati bisogna:
\begin{itemize}
	\item[1.] considerare individualmente ogni generatore ideale, spegnendo anche i generatori pilotati, in modo da calcolare la
	grandezza desiderata e le grandezze che controllano i generatori pilotati
	\item[2.] alla fine considerare individualmente ogni generatore pilotato utilizzando le grandezze pilotate ottenute prima per
	calcolare le grandezze richieste
\end{itemize}

\newpage

\section{Componenti elettrici pt.2}
\subsection{Condensatore}
\subsubsection*{Introduzione}
Un condensatore è un componente elettrico costituito da una coppia di conduttori (armature) separati da un isolante (dielettrico).

\subsubsection*{Capacità di un condensatore}
Sulle armature sono presenti cariche uguali ed opposte \(q_A(t) = -q_B(t)\). Si definisce la capacità di un condensatore (propria
del componente in quanto dipende dalla forma \(S/l\) e dal dielettrico \(\varepsilon\)):
\[C = \frac{q_A(t)}{u_{AB}(t)} = \frac{q(t)}{u(t)} \qquad \qquad C = [F_{arad}]\]
\begin{itemize}
	\item la capacità dipende dalla forma e dal dielettrico, per un condensatore piano vale \(C = \varepsilon \; S/l\)
	\item vale la legge di conservazione della carica \(i_\text{entrante} = \frac{d q_A(t)}{dt} = - \frac{d q_B(t)}{dt} = i_\text{uscente}\)
\end{itemize}

\subsubsection*{Bipolo condensatore}
\begin{center}
	\begin{tabularx}{\textwidth}{ c | X }
		modello grafico & equazioni costitutive  \\
		\begin{circuitikz} \draw (0,0) [C] to (2,0); \end{circuitikz} &
		\(\begin{cases}
			\displaystyle i(t) = C \frac{d u(t)}{dt} \\
			\displaystyle u(t) = u(0) + \frac{1}{C} \int_0^t i(\tau) d\tau
		\end{cases} \displaystyle \qquad \text{ottenute da} \;\; i(t) = \frac{d q(t)}{dt} = \frac{d \; C u(t)}{dt} = C \frac{d u(t)}{dt}\)
	\end{tabularx}
\end{center}

\subsubsection*{Lavoro elettrico in carica}
Il lavoro compiuto per caricare il condensatore dipende solo dalla tensione finale, non dipende dalle altre grandezze e da come
esse variano nel tempo.
\[L_\text{entrante}(0,t) = \int_0^t p_\text{entr}(\tau) d\tau = \int_0^t u(\tau) \cdot i(\tau) d\tau = C \int_0^t u(\tau) \frac{d u(\tau)}{d\tau} d\tau = C \int_{u_0=0}^{u_1} u \; du = \frac{1}{2}Cu(t)^2\]

\subsubsection*{Lavoro elettrico in scarica}
Il lavoro compiuto dal condensatore in scarica è pari all'energia immagazzinata durante la carica.
\[L_\text{entrante}(t,t') = \int_t^{t'} p_\text{entr}(\tau) d\tau = \int_t^{t'} u(\tau) \cdot i(\tau) d\tau = \dots = C \int_{u_1}^{u_2=0} u \; du = -\frac{1}{2}Cu(t)^2 = -L_\text{entr}(0,t)\]

\subsubsection*{Energia capacitiva}
L'energia immagazzinata dal condensatore è detta energia capacitiva e vale \(\quad\displaystyle W_C(t) = \frac{1}{2} C u(t)^2\)

\subsubsection*{Bipolo passivo}
Il condensatore è un bipolo passivo in quanto rispetta la condizione di passività: \(L_\text{uscente}(t_0,t) \leq W(t_0)\)
\[L_\text{uscente}(t_0,t) = -L_\text{entrante}(t_0,t) = W_C(t_0)-W_C(t) \leq W_C(t_0)\]

\subsubsection*{Condensatore in regime stazionario}
In regime stazionario il condensatore si comporta come un circuito aperto: \(\displaystyle i(t) = \frac{d u(t)}{dt} = 0 \quad \forall t\)

\subsection{Induttore}
\subsubsection*{Introduzione}
Un induttore è un conduttore filiforme che forma una spira (o meglio una bobina) non chiusa.

\subsubsection*{Induttanza di un induttore}
Essendo un conduttore filiforme si ha \(i_A(t) = i_B(t)\) e dall'equazione dell'autoflusso si ottiene l'induttanza
\[L = \frac{\varphi_L(t)}{i(t)} \qquad\qquad L=[H_{enry}]\]

\subsubsection*{Bipolo induttore}
\begin{center}
	\begin{tabularx}{\textwidth}{ c | X }
		modello grafico & equazioni costitutive  \\
		\begin{circuitikz} \draw (0,0) [L] to (2,0); \end{circuitikz} &
		\(\begin{cases}
			\displaystyle u(t) = L \frac{d i(t)}{dt} \\
			\displaystyle i(t) = i(0) + \frac{1}{L} \int_0^t u(\tau) d\tau
		\end{cases} \displaystyle \qquad \text{ottenute da} \;\; u(t) = \frac{d}{dt} \varphi(t) = \frac{d}{dt} L i(t) = L \frac{di(t)}{dt}\)
	\end{tabularx}
\end{center}

\subsubsection*{Lavoro elettrico in carica}
Il lavoro compiuto per caricare l'induttore dipende solo dalla corrente finale, non dipende dalle altre grandezze e da come esse
variano nel tempo.
\[L_\text{entrante}(0,t) = \int_0^t p_\text{entr}(\tau) d\tau = \int_0^t u(\tau) \cdot i(\tau) d\tau = L \int_0^t i(\tau) \frac{d i(\tau)}{d\tau} d\tau = L \int_{i_0=0}^{i_1} i \; di = \frac{1}{2}Li(t)^2\]

\subsubsection*{Lavoro elettrico in scarica}
Il lavoro compiuto dall'induttore in scarica è pari all'energia immagazzinata durante la carica.
\[L_\text{entrante}(t,t') = \int_t^{t'} p_\text{entr}(\tau) d\tau = \int_t^{t'} u(\tau) \cdot i(\tau) d\tau = \dots = L \int_{i_1}^{i_2=0} i \; di = -\frac{1}{2}Li(t)^2 = -L_\text{entr}(0,t)\]

\subsubsection*{Energia induttiva o energia magnetica}
L'energia immagazzinata dall'induttore è detta energia induttiva e vale \(\quad\displaystyle W_L(t) = \frac{1}{2} L i(t)^2\)

\subsubsection*{Bipolo passivo}
L'induttore è un bipolo passivo in quanto rispetta la condizione di passività: \(L_\text{uscente}(t_0,t) \leq W(t_0)\)
\[L_\text{uscente}(t_0,t) = -L_\text{entrante}(t_0,t) = W_L(t_0)-W_L(t) \leq W_L(t_0)\]

\subsubsection*{Induttore in regime stazionario}
In regime stazionario l'induttore si comporta come un cortocircuito: \(\displaystyle u(t) = \frac{d i(t)}{dt} = 0 \quad \forall t\)

\newpage


\section{Reti in regime sinusoidale}
\subsection{Introduzione e funzioni periodiche}
Una rete si dice in regime sinusoidale quando tutte le grandezze variano in regime sinusoidale isofrequenziale. Esempi di reti in
regime sinusoidale si trovano in telecomunicazioni, impianti radio e impianti di trasporto delle correnti.

\subsubsection*{Funzioni periodiche}
\[a(t) = a(t + T) \;\; \forall t \qquad\qquad T = \text{periodo} \qquad f = 1/T = \text{frequenza}\]
\begin{itemize}
	\item valore medio: \(\displaystyle \qquad\qquad\quad\;\;\, A_0 = \frac{1}{T} \int_T a(t) dt\)
	\item valore medio in modulo: \(\displaystyle\;\; A_m = \frac{1}{T} \int_T \left|a(t)\right| dt\)
	\item valore efficace: \(\displaystyle \qquad\qquad\quad\;\;\, A = \sqrt{\frac{1}{T} \int_T a(t)^2 dt} \qquad\; \text{anche detto R}_\text{oot} \text{M}_\text{ean} \text{S}_\text{quare}\)
	\item fattore di forma: \(\displaystyle \qquad\qquad\;\, k_f = \frac{A}{A_m} \qquad\qquad\qquad\quad \text{per sinusoidi} \; k_f = \frac{\pi}{2\sqrt{2}}\)
\end{itemize}

\subsection{Funzioni sinusoidali}
\subsubsection*{Notazione}
\[a(t) = A_M \sin(\omega t + \alpha) \qquad \qquad \begin{aligned}
	&A_M = \text{ampiezza} \\
	&A = A_M/\sqrt{2} = \text{amp. efficace}
\end{aligned} \qquad \begin{aligned}
	&\omega = 2 \pi f = \text{pulsazioni} \\
	&\alpha = \text{fase iniziale}
\end{aligned}\]
Si osserva che \(\cos(\theta) = \sin(\theta + \pi/2)\) e inoltre le funzioni sinusoidali isofrequenziali definite sopra costituiscono
uno spazio vettoriale (in quanto insieme chiuso per somma e prodotto per uno scalare) definito come segue:
\(a(t) = A_M \sin(\omega t + \alpha) = \left\{A_M, \omega, \alpha\right\}\)

\subsubsection*{Sinusoidi isofrequenziali e sfasamento}
Due sinusoidi si dicono isofrequenziali se hanno la stessa pulsazione \(\omega\) o la stessa frequenza \(f\). Date due sinusoidi
isofrequenziali è possibile definire lo sfasamento \(\varphi\):
\[\begin{aligned} a(t) &= A_M \sin (\omega t + \alpha) \\ b(t) &= B_M \sin (\omega t + \beta) \end{aligned}
\qquad \Leftrightarrow \qquad \varphi = \alpha - \beta + 2 k \pi \qquad \text{con} \; -\pi \leq \varphi \leq \pi\]
In base allo sfasamento le due sinusoidi si dicono:
\begin{center}
	\begin{minipage}{0.35\textwidth}
		\begin{itemize}
			\item \(\varphi = 0 \;\; \rightarrow \;\;\) \(a\) e \(b\) sono in fase
			\item \(\varphi > 0 \;\; \rightarrow \;\;\) \(a\) è in anticipo su \(b\)
			\item \(\varphi < 0 \;\; \rightarrow \;\;\) \(a\) è in ritardo su \(b\)
		\end{itemize}
	\end{minipage}
	\begin{minipage}{0.6\textwidth}
		\begin{itemize}
			\item \(\varphi = \pm \pi \;\; \rightarrow \;\;\) \(a\) e \(b\) sono in opposizione di fase
			\item \(\varphi = \pi/2 \;\; \rightarrow \;\;\) \(a\) è in quadratura in anticipo rispetto a \(b\)
			\item \(\varphi = -\pi/2 \;\; \rightarrow \;\;\) \(a\) è in quadratura in ritardo rispetto a \(b\)
		\end{itemize}
	\end{minipage}
\end{center}

\newpage

\subsection{Trasformata di Steinmetz, fasori e numeri complessi}
\subsubsection*{Trasformata di Steinmetz}
La trasformata di Steinmetz converte equazioni di onde sinusoidali isofrequenziali in numeri complessi detti fasori. La trasformazione
permette uno studio facilitato delle equazioni.
\begin{align*}
	a(t) = A_M \sin (\omega t + \alpha) \quad \rightarrow \quad \bar{A} = S\left[a(t)\right] = \frac{A_M}{\sqrt{2}} e^{j\alpha} = Ae^{j\alpha} \\
	\bar{A} = Ae^{j\alpha} \quad \rightarrow \quad a(t) = S^{-1}\left[\;\bar{A}\;\right] = \sqrt{2} A \sin(\omega t + \alpha) = A_M \sin (\omega t + \alpha)
\end{align*}
La trasformata è lineare in \(\mathbb{R}\):
\[S[\lambda a(t) + \mu b(t)] = \lambda S[a(t)] + \mu S[b(t)] \qquad\qquad S^{-1}\left[\lambda\bar{A} + \mu\bar{B}\;\right] = \lambda S^{-1}\left[\;\bar{A}\;\right] + \mu S^{-1}\left[\;\bar{B}\;\right]\]

\subsubsection*{Fasori e numeri complessi}
Il fasore è definito come immagine di una sinusoide attraverso la trasformata di Steinmetz. Nel piano complesso è rappresentato da
un vettore.
\[\bar{A} = S[a(t)] = \quad \begin{cases} Ae^{j\alpha} &\text{forma polare} \\ A\cos\alpha + jA\sin\alpha &\text{forma cartesiana}\end{cases}\]
L'angolo tra i fasori nel piano corrisponde al loro sfasamento:
\begin{center}
	\begin{minipage}{0.5\textwidth}
		\begin{itemize}
			\item in fase \(\;\; \rightarrow \;\;\) fasori sovrapposti
			\item in opposizione di fase \(\;\; \rightarrow \;\;\) fasori opposti
			\item in quadratura di fase \(\;\; \rightarrow \;\;\) fasori perpendicolari
		\end{itemize}
	\end{minipage}
	\begin{minipage}{0.49\textwidth}
		\begin{itemize}
			\item \(\bar{A}\) in anticipo su \(\bar{B}\) \(\;\; \rightarrow \;\;\) angolo di \(\bar{A}\) maggiore
			\item \(\bar{A}\) in ritardo su \(\bar{B}\) \(\;\; \rightarrow \;\;\) angolo di \(\bar{B}\) maggiore
		\end{itemize}
	\end{minipage}
\end{center}

\subsubsection*{Operazioni tra fasori}
\begin{itemize}
	\item \textbf{somma tra fasori} (somma di sinusoidi):
	\[\bar{C} = \bar{A} + \bar{B} = [\Real(A) + \Real(B)] + j [\Img(A) + \Img(B)] = \Real(C) + j \Img(C)\]
	\item \textbf{prodotto per uno scalare reale}:
	\[\bar{C} = k \bar{A} = k A e^{j\alpha} = \left|k\right| A e^{\alpha \pm n \pi} = Ce^{j\gamma}
	\qquad\quad C = \left|k\right| A \qquad \gamma = \begin{cases} \alpha & k > 0 \\ \alpha \pm \pi & k < 0 \end{cases}\]
	\item \textbf{prodotto per uno scalare immaginario} (derivata temporale, anticipo di \(\pi/2\)):
	\[\bar{C} = j\omega \bar{A} = \omega A e^{j\alpha}e^{j \pi / 2} = \omega A e^{j(\alpha + \pi/2)} = Ce^{j\gamma}
	\qquad\quad C = \omega A \qquad \gamma = \alpha + \pi/2\]
	\item \textbf{prodotto per il proprio coniugato} (il risultato è un numero complesso, ma non è un fasore):
	\[\dot{P} = \bar{A} \check{B} = Ae^{j\alpha}Be^{-j\beta} = AB e^{j\varphi} = AB\cos\varphi + jAB\sin\varphi\]
	\item \textbf{rapporto tra fasori} (il risultato è un numero complesso, ma non è un fasore):
	\[\dot{O} = \frac{\bar{A}}{\bar{B}} = \frac{A}{B} e^{j\alpha - j\beta} = Oe^{j\varphi} \qquad\quad \bar{A} = \dot{O} \bar{B} \qquad \dot{O} = \text{operatore complesso}\]
	\item \textbf{derivata temporale} (corrisponde a moltiplicare per \(j\) o ad aggiungere uno sfasamento di \(\pi/2\)):
	\[\frac{d \, a(t)}{dt} = A_M \omega \cos(\omega t + \alpha) = A_M \omega \sin(\omega t + \alpha + \pi/2 ) \quad\Leftrightarrow\quad \omega A e^{j(\alpha + \pi/2)} = j \omega A e^{j\alpha}\]
\end{itemize}

\newpage

\subsection{Potenze in regime sinusoidale}
\subsubsection*{Potenze in funzione del tempo}
La potenza in un bipolo in regime sinusoidale è data da una parte costante \(P_\text{costante}\) dipendente da \(U, I, \varphi\)
e da una parte fluttuante \(P_\text{fluttuante}(t)\) a frequenza doppia dipendente in particolare dal tempo.
\begin{align*}
	p(t) &= u(t) i(t) = U_M I_M \sin(\omega t + \alpha) \sin (\omega + \beta) = 2UI \left(\frac{1}{2} \cos(\alpha-\beta) - \frac{1}{2} \cos(2\omega t + \alpha + \beta)\right) = \\
	&= UI \cos(\varphi) - UI \cos (2 \omega t + \alpha + \beta) = P_\text{costante} + P_\text{fluttuante}(t)
\end{align*}

\subsubsection*{Potenza attiva}
Si definisce la potenza attiva come media della potenza in un periodo di tempo \(T\).
\[P = \frac{1}{T} \int_T p(t) dt = UI \cos \varphi = [W_{att}] \qquad \qquad \begin{cases}
	\text{potenza massima per} \; \varphi = 0 \\
	\text{potenza opposta per} \; \varphi = \pi \\
	\text{potenza nulla per} \; \varphi = \pm\pi/2
\end{cases}\]

\subsubsection*{Lavoro elettrico e significato del valore efficace}
Si definisce il lavoro elettrico per un intervallo di tempo \([t_1, t_2]\) con \(t_2 - t_1 = k T\):
\[L(t_1,t_2) = \int_{t_1}^{t_2} p(t) dt = \int_{t_1}^{t_2} P + p_f(t) dt = P \Delta t + \int_{t_1}^{t_2} p_f(t) dt \; \stackrel{t_2-t_1 \gg T}{\approx} \; P \Delta t\]
Il lavoro entrante calcolato in un periodo \(T\) vale:
\[L(t_1, t_1 + T) = \int_{t_1}^{t_1+T} p(t) dt = \int_{t_1}^{t_1+T} R i^2(t) dt = R {I_\text{eff}}^2 \cdot T \qquad L_\text{per unità di periodo} = R{I_\text{eff}}^2\]
Si osserva che il valore efficace o RMS è il corrispettivo valore che in corrente continua produce una stessa energia di
quella prodotta in circuito AC.

\subsubsection*{Potenza reattiva}
Si definisce la potenza reattiva come grandezza proporzionale al \(W_{att}\), ma non legata all'accumulo elettrico, bensì associata
all'accumulo di energia all'interno dei componenti (condensatori e induttori).
\[Q = UI \sin \varphi = [V_{olt} A_{mpere} R_{eattivi}]\]

\subsubsection*{Potenza apparente}
Si definisce la potenza apparente come modulo della potenza attiva e reattiva, sempre proporzionale al \(W_{att}\) e sempre positiva,
associata allo scambio di energia dei componenti e alla loro taglia.
\[S = U I = \sqrt{P^2 + Q^2} = [V_{olt}A_{mpere}] \qquad\qquad P = S \cos \varphi \quad Q = S \sin \varphi\]

\subsubsection*{Fattore di potenza}
Si definisce il fattore di potenza come rapporto tra la potenza attiva e quella apparente: \(\cos \varphi = P/S\)

\subsubsection*{Potenza complessa}
La potenza complessa contiene tutte le informazioni delle potenze attiva, reattiva e apparente in AC. La parte reale è legata al
lavoro elettrico, la parte immaginaria è legata all'accumulo di energia.
\[\dot{S} = \bar{U} \check{I} = UI e^{j\alpha} e^{-j\beta} = UI e^{j\varphi} = UI(cos\varphi + j \sin\varphi) = P + jQ\]
\[P = \Real(\dot{S}) \qquad Q = \Img(\dot{S}) \qquad S = \left|\dot{S}\right|\]

\subsection{Bipoli in regime sinusoidale}
\subsubsection*{Notazione}
\begin{center}
	\begin{minipage}{0.25\textwidth}
		\begin{circuitikz}[european]
			\draw (0,0) to[R=R, i_>=i] (3,0);
			\node [] at (0,0.2) {+};
			\node [] at (3,0.2) {--};
		\end{circuitikz}
	\end{minipage}
	\begin{minipage}{0.7\textwidth}
		\begin{align*}
			\text{corrente} \quad i(t) &= I_M \sin(\omega t + \beta) = \sqrt{2} I \sin(\omega t + \beta) \\
			\text{tensione} \quad u(t) &= U_M \sin(\omega t + \alpha) = \sqrt{2} U \sin(\omega t + \alpha) \\
			\text{sfasamento} \quad \varphi &= \alpha - \beta + 2k\pi \\
			\text{relazione} \quad u(t) &= \frac{U_M}{I_M} i\left(t + \frac{\varphi}{\omega}\right) \qquad \frac{\varphi}{\omega} = \text{sfasamento in secondi}
		\end{align*}
	\end{minipage}
\end{center}

\subsubsection*{Strumenti di misura}
\begin{itemize}
	\item \textbf{voltmetro a valore efficace}: misura il valore efficace della tensione (sempre positiva)
	\item \textbf{amperometro a valore efficace}: misura il valore efficace della corrente (sempre positiva)
	\item \textbf{wattametro a valor medio}: misura la potenza attiva (il segno dipende dalle convenzioni)
	\item \textbf{varmetro}: misura la potenza reattiva (il segno dipende dalle convenzioni)
\end{itemize}

\subsubsection*{Generatori ideali in AC}
\begin{center}
	\begin{minipage}{0.25\textwidth}
		\begin{circuitikz}
			\draw (0,0) to[V, i_>=i] (3,0);
			\draw (1.5,0) [circ, thick, fill, color=white] circle(0.39);
			\node[] at (0,0.2) {--};
			\node[] at (3,0.2) {+};
			\node[] at (1.5,0.7) {\(e(t)\)};
			\node[] at (1.5,0) {\(\underline{\sim}\)};
		\end{circuitikz}
	\end{minipage}
	\begin{minipage}{0.7\textwidth}
		\begin{align*}
			v(t) &= e(t) = E_M \sin(\omega t + \alpha) = \sqrt{2} E \sin(\omega t + \alpha) \\
			i(t) &= I_M \sin(\omega t + \beta) = \sqrt{2} I \sin(\omega t + \beta) \\
			&\text{con } I_M \text{ e } \beta \text{ incognite dipendenti dal circuito}
		\end{align*}
	\end{minipage}
\end{center}

\begin{center}
	\begin{minipage}{0.25\textwidth}
		\begin{circuitikz}
			\draw (0,0) to[V, i_>=i] (3,0);
			\draw (1.5,0) [circ, thick, fill, color=white] circle(0.39);
			\node[] at (0,0.2) {--};
			\node[] at (3,0.2) {+};
			\node[] at (1.5,0.7) {\(j(t)\)};
			\node[] at (1.5,0) {\(\underline{\sim}\)};
		\end{circuitikz}
	\end{minipage}
	\begin{minipage}{0.7\textwidth}
		\begin{align*}
			i(t) &= j(t) = I_M \sin(\omega t + \beta) = \sqrt{2} I \sin(\omega t + \beta) \\
			v(t) &= E_M \sin(\omega t + \alpha) = \sqrt{2} E \sin(\omega t + \alpha) \\
			&\text{con } E_M \text{ e } \alpha \text{ incognite dipendenti dal circuito}
		\end{align*}
	\end{minipage}
\end{center}

\subsubsection*{Bipoli lineari in AC}
Un bipolo si dice lineare in AC se la sua caratteristica esterna è combinazione lineare dei fasori \(\bar{U}\) e \(\bar{I}\)
(analogo ai bipoli lineari in DC): \(\;\; a_n\bar{U} + b_n\bar{I} = c_n\)

\subsubsection*{Bipoli passivi in AC}
Un bipolo si dice passivo in AC se può solo assorbire potenza attiva: \[P = UI \cos\varphi > 0 \;\; \rightarrow \;\; -\pi/2 \leq \varphi \leq \pi/2\]

\subsubsection*{Impedenza di un bipolo passivo}
\[\dot{Z} = \frac{\bar{U}}{\bar{I}} = \frac{Ue^{j\alpha}}{Ie^{j\beta}} = \frac{U}{I} e^{j\varphi} \qquad\qquad
Z = \left|\dot{Z}\right| = \frac{U_\text{eff}}{I_\text{eff}} = [\Omega] \qquad \varphi = \text{arg} \dot{Z} = \arctan\left(\frac{\Img(\dot{Z})}{\Real(\dot{Z})}\right) = [rad]\]

\subsubsection*{Ammettenza di un bipolo passivo}
\[\dot{Y} = \dot{Z}^{-1} = \frac{\bar{I}}{\bar{U}} = \frac{Ie^{j\beta}}{Ue^{j\alpha}} = \frac{I}{U} e^{-j\varphi}\]

\subsubsection*{Potenze in un bipolo passivo}
\[\dot{S} = \bar{U} I^* = \dot{Z} \bar{I} I^* = \dot{Z} I^2 \qquad\qquad \begin{aligned}
	&\text{potenza apparente} \quad\! S = \left|Z\right| I^2 = ZI^2 \\
	&\;\;\text{potenza attiva} \qquad P = \Real(Z) I^2 = ZI^2 \cos\varphi \\
	&\;\text{potenza reattiva} \quad\;\, Q = \Img(Z) I^2 = ZI^2 \sin \varphi
\end{aligned}\]

\subsubsection*{Resistore ideale in AC}
\begin{center}
	\begin{minipage}{0.25\textwidth}
		\begin{circuitikz}
			\draw (0,0) to[R=R, i>_=i(t)] (3,0);
			\node[] at (0,0.2) {+};
			\node[] at (3,0.2) {--};
		\end{circuitikz}
	\end{minipage}
	\begin{minipage}{0.74\textwidth}
		\begin{align*}
			u(t) = R i(t) \;\; \rightarrow \;\; &u(t) = U_M \sin(\omega t + \alpha) = R I_M \sin (\omega t + \beta) \\
			&\varphi = \alpha - \beta = 0
		\end{align*}
	\end{minipage}
\end{center}
\begin{align*}
	&\dot{Z}_\text{impedenza} = \frac{\bar{U}}{\bar{I}} = \frac{R \bar{I}}{\bar{I}} = R \qquad \dot{Y}_\text{ammettenza} = \frac{1}{\dot{Z}} = \frac{1}{R} = G \qquad R_\text{resistenza} = \frac{U_M}{I_M} = \frac{U}{I} \\
	&p(t)_\text{istantanea} = R i^2 \qquad S_\text{apparente} = RI^2 \qquad P_\text{attiva} = RI^2 \cos\varphi = RI^2 \qquad Q_\text{reattiva} = RI^2 \sin\varphi = 0
\end{align*}


\subsubsection*{Induttore ideale in AC}
\begin{center}
	\begin{minipage}{0.25\textwidth}
		\begin{circuitikz}
			\draw (0,0) to[L=L, i>_=i(t)] (3,0);
			\node[] at (0,0.2) {+};
			\node[] at (3,0.2) {--};
		\end{circuitikz}
	\end{minipage}
	\begin{minipage}{0.74\textwidth}
		\begin{align*}
			u(t) = L \frac{di}{dt} \;\; \rightarrow \;\; &u(t) = U_M \sin(\omega t + \alpha) = \omega L I_M \sin (\omega t + \beta + \pi/2) \\
			&\varphi = \alpha - \beta = \pi/2
		\end{align*}
	\end{minipage}
\end{center}
\begin{align*}
	&\dot{Z}_\text{impedenza} = \frac{\bar{U}}{\bar{I}} = \frac{U}{I} e^{j\pi/2} = j \omega L = j X_L \qquad {X_L}_\text{reattanza induttiva} = \omega L = [\Omega] \\
	&\dot{Y}_\text{ammettenza} = \frac{1}{\dot{Z}} = \frac{1}{j\omega L} = -\frac{j}{\omega L}= j B_L  \qquad\; {B_L}_\text{scuscettanza induttiva} = -\frac{1}{\omega L} = [S_{iemens}] \\
	&u(t) = \frac{1}{2}Li^2(t) \geq 0 \qquad S_\text{apparente} = \left|X_L\right| I^2 \qquad P_\text{attiva} = S \cos\varphi = 0 \qquad Q_\text{reattiva} = S \sin\varphi = X_L I^2
\end{align*}

\subsubsection*{Condensatore ideale in AC}
\begin{center}
	\begin{minipage}{0.25\textwidth}
		\begin{circuitikz}
			\draw (0,0) to[C=C, i>_=i(t)] (3,0);
			\node[] at (0,0.2) {+};
			\node[] at (3,0.2) {--};
		\end{circuitikz}
	\end{minipage}
	\begin{minipage}{0.74\textwidth}
		\begin{align*}
			i(t) = C \frac{du}{dt} \;\; \rightarrow \;\; &i(t) = I_M \sin(\omega t + \beta) = \omega C U_M \sin (\omega t + \alpha + \pi/2) \\
			&\varphi = \alpha - \beta = -\pi/2
		\end{align*}
	\end{minipage}
\end{center}
\begin{align*}
	&\dot{Z}_\text{impedenza} = \frac{\bar{U}}{\bar{I}} = \frac{U}{I} e^{-j\pi/2} = - \frac{j}{\omega C} = j X_C \qquad {X_C}_\text{reattanza capacitiva} = -\frac{1}{\omega C} = [\Omega] \\
	&\dot{Y}_\text{ammettenza} = \frac{1}{\dot{Z}} = -\frac{\omega C}{j} = j \omega C = j B_L  \qquad\quad\;\; {B_C}_\text{scuscettanza capacitiva} = \omega C = [S_{iemens}] \\
	&u(t) = \frac{1}{2}Cu^2(t) \geq 0 \qquad S_\text{apparente} = \left|X_L\right| I^2 \qquad P_\text{attiva} = S \cos\varphi = 0 \qquad Q_\text{reattiva} = S \sin\varphi = X_L I^2
\end{align*}

\subsubsection*{Condensatore reale (per curiosità personale, non in programma)}
\begin{center}
	\begin{minipage}{0.55\textwidth}
		\centering
		\begin{circuitikz}
			\draw (0,0) to[short, *-] (0.5,0) to[L=L] (1.5,0) to[R=\(R_S\)] (4,0) -- (4,0.5) to[R=\(R_C\)] (6,0.5) -- (6,0) to[short, -*] (6.5,0);
			\draw (4,0) -- (4,-0.75) to[C=C] (6,-0.75) -- (6,0);
		\end{circuitikz}
	\end{minipage}
	\begin{minipage}{0.4\textwidth}
		\begin{align*}
			L \quad &\text{induttanza dei terminali, }  L \to 0 \\
			R_S \quad &\text{resistenza dei terminali, } R_S \to 0 \\
			R_C \quad &\text{resistenza dei dielettrici, } R_C \to +\infty \\
			C \quad &\text{capacità del condensatore}
		\end{align*}
	\end{minipage}
\end{center}

\newpage

\subsection{Leggi di Kirchhoff, serie e paralleli in AC}
\subsubsection*{Legge di Kirchhoff per correnti}
Dato un insieme di taglio \(\mathcal{T}\), la legge di Kirchhoff per i fasori delle correnti  vale:
\[\sum_{h \in \mathcal{T}} \alpha_h \bar{I}_h = \bar{0} \qquad \qquad \begin{cases}\alpha = 1 &\text{correnti uscenti} \\ \alpha = -1 &\text{correnti entranti}\end{cases}\]

\subsubsection*{Legge di Kirchhoff per tensioni}
Data una maglia \(\mathcal{M}\), la legge di Kirchhoff per i fasori delle tensioni vale:
\[\sum_{h \in \mathcal{M}} \beta_h \bar{U}_h = \bar{0} \qquad \qquad \begin{cases}\beta = 1 &\text{tensioni concordi con maglia} \\ \beta = -1 &\text{tensioni discordi con maglia}\end{cases}\]

\subsubsection*{Serie di impedenze}
Data una serie di impedenze, l'impedenza equivalente è \(\dot{Z}_{eq} = \dot{Z}_1 + \dot{Z}_2 + ... + \dot{Z}_n\)

\subsubsection*{Partitore di tensione}
In un partitore di tensione, la tensione è proporzionale all'impedenza:
\[\bar{U}_k = \frac{\dot{Z}_k}{\sum_j \dot{Z}_j} \bar{U}_{tot} \qquad\qquad \bar{U}_1 = \frac{\dot{Z}_1}{\dot{Z}_1 + \dot{Z}_2} \bar{U}_{tot} \qquad \bar{U}_2 = \frac{\dot{Z}_2}{\dot{Z}_1 + \dot{Z}_2} \bar{U}_{tot}\]

\subsubsection*{Parallelo di impedenze}
Dato un parallelo di impedenze, l'impedenza equivalente è \(\dot{Z}_{eq} = \left(\frac{1}{\dot{Z}_1} + \frac{1}{\dot{Z}_2} + ... + \frac{1}{\dot{Z}_n}\right)^{-1}\)

\subsubsection*{Partitore di corrente}
In un partitore di corrente, la corrente è proporzionale all'ammettenza:
\[\bar{I}_k = \frac{\dot{Y}_k}{\sum_j \dot{Y}_j} \bar{I}_{tot} \qquad\qquad \bar{I}_1 = \frac{\dot{Z}_2}{\dot{Z}_1 + \dot{Z}_2} \bar{I}_{tot} \qquad \bar{I}_2 = \frac{\dot{Z}_1}{\dot{Z}_1 + \dot{Z}_2} \bar{I}_{tot}\]

\subsection{Rete simbolica}
\subsubsection*{Rete lineare e analisi nel tempo}
Una rete si dice lineare in regime sinusoidale se è composta soltanto da bipoli R, L, C, GIT e GIC. È possibile analizzare la
rete in funzione del tempo. Ciò, però, risulta complesso, lungo e poco vantaggioso.

\subsubsection*{Rete simbolica e analisi in frequenza}
Se si ha una rete lineare con GIT e GIC che generano sinusoidi isofrequenziali, è possibile trasformarla attraverso la trasformata
di Steinmetz ottenendo una rete simbolica analizzabile con i fasori con le seguenti proprietà:
\begin{itemize}
	\item si analizza in funzione della frequenza
	\item ha lo stesso grafo della rete di partenza
	\item valgono sostituzione, SVE, correnti di anello, potenziali nodali, Thévenin e Norton
\end{itemize}

\begin{align*}
	&\text{bipoli passivi} \;\; \rightarrow \;\; \bar{U} = \dot{Z} \bar{I} \qquad\qquad\quad \bar{I} = \dot{Y} \bar{U} \\
	&\;\, \text{GIT e GIC} \quad \rightarrow \;\; \bar{U} = \bar{E} = Ee^{j\alpha} \qquad \bar{I} = \bar{J} = Je^{j\beta} \\
	&\;\text{GLT e GLC} \!\!\quad \rightarrow \;\; \bar{U} = \bar{E} - \dot{Z} \bar{I} \qquad\quad \bar{I} = \bar{J} - \dot{Y} \bar{U}
\end{align*}

\subsubsection*{Teorema di sostituzione}
È possibile sostituire un lato della rete con \(\bar{U}\) e \(\bar{I}\) con un GIT di tensione \(\bar{E} = \bar{U}\)
o con un GIC di \(\bar{J} = \bar{I}\).

\subsubsection*{Sovrapposizione degli effetti}
\[\bar{U}_h = \sum_r \bar{U}_{h,E_r} + \sum_s \bar{U}_{h,J_s} \qquad \qquad \begin{aligned} &\bar{U}_{h,E_r} = \text{costributo dell'r-esimo GIT} \\ &\bar{U}_{h,J_s} = \text{costributo dell's-esimo GIC}\end{aligned}\]
\[\bar{I}_h = \sum_r \bar{I}_{h,E_r} + \sum_s \bar{I}_{h,J_s} \qquad \qquad \begin{aligned} &\bar{I}_{h,E_r} = \text{costributo dell'r-esimo GIT} \\ &\bar{I}_{h,J_s} = \text{costributo dell's-esimo GIC}\end{aligned}\]

\subsubsection*{Metodo delle correnti di anello (poco usato)}
\[\dot{Z}_{A_{kk}} \bar{I}_{A_k} - \sum_h \dot{Z}_{A_{kh}} \bar{I}_{A_h} = \bar{E}_{A_k} \qquad
\begin{aligned}
	\dot{Z}_{A_{kk}} &= \sum \text{impedenze della maglia } A_k \\
	\dot{Z}_{A_{kh}} &= \sum \text{impedenze in comune alle maglie } A_k \text{ e } A_h \\
	\bar{E}_{A_k} &= \sum \text{fem di anello della maglia } A_k \text{ con conv. generatore} \\
	\bar{I}_{A_k} &= \text{corrente di anello della maglia } A_k \\
	\bar{I}_{A_h} &= \text{corrente di anello della maglia } A_h
\end{aligned}\]

\subsubsection*{Metodo dei potenziali nodali (poco usato)}
\[\dot{Y}_{N_{kk}} \bar{U}_{N_k} - \sum_{h} \dot{Y}_{N_{kh}} \bar{U}_{N_h} = \bar{J}_{N_k} \qquad
\begin{aligned}
	\dot{Y}_{N_{kk}} &= \sum \text{ammettenze incidenti al nodo } N_k \\
	\dot{Y}_{N_{kh}} &= \sum \text{ammettenze comprese tra i nodi } N_k \text{ e } N_h \\
	\bar{J}_{N_k} &= \sum \text{correnti dei GIC impresse su } N_k \text{ (+ se entranti)} \\
	\bar{U}_{N_k} &= \text{potenziale del nodo } N_k \\
	\bar{U}_{N_h} &= \text{potenziale del nodo } N_h
\end{aligned}\]

\subsubsection*{Teorema di Thévenin}
\begin{align*}
	\bar{U}_{eq} &= \bar{E}_{ab} = \bar{U}_{0,ab} = \text{fasore della tensione a vuoto della porta ab} \\
	\dot{Z}_{eq} &= \dot{Z}_{ab} = \frac{\bar{U}_{0,ab}}{\bar{I}_{cc,ab}} = \text{impedenza della rete inerte (con tutti i GIT/GIC spenti)} \\
	\bar{U}_{ab} &= \bar{U}_{0,ab} - \dot{Z}_{eq} \; \bar{I}_{cc,ab} \;\;\rightarrow\;\; \text{equazione del GLT}
\end{align*}

\subsubsection*{Teorema di Norton}
\begin{align*}
	\bar{I}_{eq} &= \bar{J}_{ab} = \bar{I}_{cc,ab} = \text{fasore della corrente di cortocircuito della porta ab} \\
	\dot{Z}_{eq} &= \dot{Z}_{ab} = \frac{\bar{U}_{0,ab}}{\bar{I}_{cc,ab}} = \text{impedenza della rete inerte (con tutti i GIT/GIC spenti)} \\
	\bar{I}_{ab} &= \bar{I}_{cc,ab} - \frac{\bar{U}_{ab}}{\dot{Z}_{eq}} \;\;\rightarrow\;\; \text{equazione del GLC}
\end{align*}

\subsubsection*{Convenzione delle potenze complesse}
\[\sum \dot{S}_e = \dot{0} \quad \Leftrightarrow \quad \begin{cases}
	\sum P_e = 0 \\
	\sum Q_e = 0
\end{cases} \qquad \qquad \sum \dot{S}_u = \dot{0} \quad \Leftrightarrow \quad \begin{cases}
	\sum P_u = 0 \\
	\sum Q_u = 0
\end{cases} \qquad \qquad \sum \dot{S}_e = \sum \dot{S}_u\]

\newpage

\subsection{Analisi in frequenza e risonanza in una serie RLC}
\subsubsection*{Circuito ed equazioni caratteristiche}
\begin{center}
	\begin{minipage}{0.4\textwidth}
		\centering
		\begin{circuitikz}
			\draw (0,0) to[short, *-, i>_=i(t)] (1,0);
			\draw (1,0) to[R=R] (2.5,0) to[L=L] (4,0) to[C=C] (5,0) -- (5,-1.5) to[short, -*] (0,-1.5);
			\node[] at (-0.3,0) {+};
			\node[] at (1,0.2) {+};
			\node[] at (2.6,0.2) {+};
			\node[] at (4.1,0.2) {+};
			\node[] at (-0.3,-1.5) {--};
			\node[] at (-0.4,-0.75) {u(t)};
		\end{circuitikz}
		\begin{align*}
			&\quad\bar{U} = \dot{Z} \bar{I} \qquad \dot{Z}(\omega) = R + jX(\omega) \\
			&X(\omega) = X_L(\omega) + X_C(\omega) = \omega L - \frac{1}{\omega C} 
		\end{align*}
	\end{minipage}
	\begin{minipage}{0.1\textwidth}
		\textcolor{white}{.}
	\end{minipage}
	\begin{minipage}{0.4\textwidth}
		\centering
		\includegraphics[width=\textwidth]{immagini/serie RLC.png}
	\end{minipage}
\end{center}

\subsubsection*{Impedenza e ammettenza}
\begin{align*}
	\text{impedenza:} \quad &\dot{Z}(\omega) = R + j(X_L(\omega) + X_(\omega)) = R + j\left(\omega L - \frac{1}{\omega C}\right) \\
	&\left|\dot{Z}(\omega)\right| = \sqrt{R^2 + \left(\omega L - \frac{1}{\omega C}\right)^2} \qquad \qquad \varphi(\omega) = \arg \dot{Z} = \arctan \left(\frac{\omega L - \frac{1}{\omega C}}{R}\right) \\
	& \\
	\text{ammettenza:} \quad &\dot{Y}(\omega) = \frac{1}{\dot{Z}(\omega)} = \frac{1}{R + j(X_L + X_C)} = \frac{1}{R + j\left(\omega L - \frac{1}{\omega C}\right)} \\
	&\left|\dot{Y}(\omega)\right| = \frac{1}{\sqrt{R^2 + \left(\omega L - \frac{1}{\omega C}\right)^2}} \qquad\qquad \varphi'(\omega) = \arg \dot{Y} = \arctan \left(-\frac{\omega L - \frac{1}{\omega C}}{R}\right)
\end{align*}

\subsubsection*{Risonanza}
Si osserva che l'impedenza e l'ammettenza dipendono da \(\omega\), ovvero dalla frequenza delle correnti e delle tensioni impresse
sul circuito. Per un certo valore \(\omega_0 = \frac{1}{\sqrt{LC}}\) si ha un fenomeno di risonanza:
\begin{itemize}
	\item i fenomeni induttivi e capacitivi si equivalgono e si cancellano a vicenda
	\item si ha un picco di massimo dell'ammettenza per \(\omega = \omega_0\)
	\item si ha una valle di minimo dell'induttanza per \(\omega = \omega_0\)
\end{itemize}
In base al valore di \(\omega\) si avranno due comportamenti
\begin{itemize}
	\item per \(\omega < \omega_0\) prevale il comportamento ohmico-capacitivo (\(\varphi < 0\), \(\dot{Z} \stackrel{\omega \ll \omega_0}{\longrightarrow} R - j / \omega C\))
	\item per \(\omega > \omega_0\) prevale il comportamento ohmico-induttivo (\(\varphi > 0\), \(\dot{Z} \stackrel{\omega \gg \omega_0}{\longrightarrow} R + j\omega L\))
\end{itemize}

\subsubsection*{Fattore di merito in risonanza}
Il fattore di merito indica l'ampiezza della campana di risonanza e, in generale, la selettività del circuito:
\begin{itemize}
	\item fattore di merito alto \(\;\; \rightarrow \;\;\) campana ampia, bassa selettività
	\item fattore di merito basso \(\;\; \rightarrow \;\;\) campana stretta, alta selettività
\end{itemize}

\[Q_0 = \frac{U_L}{U_R} = \frac{U_C}{U_R} = \frac{\left|X_L(\omega_0)\right|}{R} = \frac{\left|X_C(\omega_0)\right|}{R} = \frac{\omega L}{R} = -\frac{1}{\omega C R} = \frac{1}{R} \sqrt{\frac{L}{C}} \qquad\qquad \text{indipendente da } \omega_0\]

\newpage

\subsubsection*{Analisi dei fasori di correnti, tensioni e potenze in risonanza}
\begin{align*}
	\bar{U}_L = - \bar{U}_C \qquad &\text{i fasori sono uguali e opposti tra loro e si cancellano a vicenda} \\
	\bar{U}_L = j \omega_0 L \, \bar{I} \qquad &\text{la corrente è in ritardo (e in quadratura) rispetto alla tensione nell'induttanza} \\
	\bar{U}_C = - j \frac{1}{\omega_0 C} \, \bar{I} \qquad &\text{la corrente è in anticipo (e in quadratura) rispetto alla tensione nel condensatore} \\
	Q_L + Q_C = 0 \qquad &\text{le potenze reattive sono opposte, induttore e condensatore agiscono in cortocircuito}
\end{align*}
Siccome vale \(U_L = U_C = Q_0 \, U_\text{serie}\), se \(Q_0 > 1\) le tensioni all'interno dei singoli componenti vengono amplificate.
Questo fenomeno può essere voluto nei circuiti di amplificazione di segnale, ma può provocare danni nei circuiti di potenza.

\subsubsection*{Analisi nel tempo di correnti, tensioni e potenze in risonanza}
\begin{align*}
	u_{LC}(t) &= u_L(t) + u_C(t) = 0 \quad \rightarrow \quad u_L(t) = - u_C(t), \qquad\quad i_L(t) = i_C(t) \\
	p_L(t) &= u_L(t) i(t) = - u_C(t) i(t) = - p_C(t) \quad \rightarrow \quad p_L(t) + p_C(t) = 0 \\
	\omega_L(t) &= \frac{1}{2} Li(t)^2, \qquad \omega_C(t) = \frac{1}{2} Cu(t)^2 \quad \rightarrow \quad \omega_{LC} = \text{costante}
\end{align*}
Si osserva che, in risonanza, il condensatore e l'induttanza agiscono come cortocircuito e si scambiano energia a vicenda, senza
influire sulla potenza istantanea del circuito.

\vspace{20pt}

\subsection{Analisi in frequenza e antisonanza in un parallelo RLC}
\subsubsection*{Circuito ed equazioni caratteristiche}
\begin{center}
	\begin{minipage}{0.4\textwidth}
		\centering
		\begin{circuitikz}
			\draw (0,0) to[short, *-, i>_=i(t)] (1,0) -- (4,0) (4,-2) to[short, -*] (0,-2);
			\draw (1.2,0) to[R=R] (1.2,-2) (2.5,0) to[L=L] (2.5,-2) (4,0) to[C=C] (4,-2);
			\node[] at (-0.3,0) {+};
			\node[] at (-0.3,-2) {--};
			\node[] at (-0.4,-1) {u(t)};
		\end{circuitikz}
		\begin{align*}
			&\quad \bar{I} = \dot{Y} \bar{U} \qquad\quad \dot{Y} = G + jB(\omega) \\
			&B(\omega) = B_L(\omega) + B_C(\omega) = -\frac{1}{\omega L} + \omega C
		\end{align*}
	\end{minipage}
	\begin{minipage}{0.1\textwidth}
		\textcolor{white}{.}
	\end{minipage}
	\begin{minipage}{0.4\textwidth}
		\centering
		\includegraphics[width=\textwidth]{immagini/parallelo RLC.png}
	\end{minipage}
\end{center}
\subsubsection*{Impedenza e ammettenza}
\begin{align*}
	\text{ammettenza:} \quad &\dot{Y}(\omega) = G + j(B_L + B_C) = G + j\left(\omega C - \frac{1}{\omega L}\right) \\
	&\left|\dot{Y}(\omega)\right| = \sqrt{G^2 + \left(\omega C - \frac{1}{\omega L}\right)^2} \qquad\qquad \varphi'(\omega) = \arg \dot{Y} = \arctan \left(\frac{\omega C - \frac{1}{\omega L}}{G}\right) \\
	& \\
	\text{impedenza:} \quad &\dot{Z}(\omega) = \frac{1}{\dot{Y}(\omega)} = \frac{1}{G + j(B_L + B_C)} = \frac{1}{G + j\left(\omega C - \frac{1}{\omega L}\right)} \\
	&\left|\dot{Z}(\omega)\right| = \frac{1}{\sqrt{G^2 + \left(\omega C - \frac{1}{\omega L}\right)^2}} \qquad\qquad \varphi(\omega) = \arg \dot{Z} = \arctan \left(-\frac{\omega C - \frac{1}{\omega L}}{G}\right)
\end{align*}

\newpage

\subsubsection*{Antisonanza}
Si ha un fenomeno di antisonanza (opposto alla risonanza) per \(\omega_0 = \frac{1}{\sqrt{LC}}\)
\begin{itemize}
	\item i fenomeni induttivi e capacitivi si equivalgono e si sommano
	\item si ha una valle di minimo dell'ammettenza per \(\omega = \omega_0\)
	\item si ha un picco di massimo dell'induttanza per \(\omega = \omega_0\)
\end{itemize}
In base al valore di \(\omega\) si avranno due comportamenti
\begin{itemize}
	\item per \(\omega < \omega_0\) prevale il comportamento ohmico-induttivo (\(\varphi > 0\), \(\dot{Z} \stackrel{\omega \ll \omega_0}{\longrightarrow} R - j / \omega L\))
	\item per \(\omega > \omega_0\) prevale il comportamento ohmico-capacitivo (\(\varphi < 0\), \(\dot{Z} \stackrel{\omega \gg \omega_0}{\longrightarrow} R + j\omega C\))
\end{itemize}

\subsubsection*{Fattore di merito in antisonanza}
\[Q_0 = \frac{I_L}{I_R} = \frac{I_C}{I_R} = \frac{R}{\left|X_L(\omega_0)\right|} = \frac{R}{\left|X_C(\omega_0)\right|} = \frac{R}{\omega L} = -\omega C R = R \sqrt{\frac{C}{L}} \qquad\qquad \text{indipendente da } \omega_0\]

\subsubsection*{Analisi dei fasori di correnti, tensioni e potenze in antisonanza}
\begin{align*}
	\bar{I}_L = - \bar{I}_C \qquad &\text{i fasori sono uguali ed opposti tra loro e si cancellano a vicenda} \\
	\bar{I}_C = j \omega_0 C \bar{U} \qquad &\text{la tensione è in ritardo (e in quadratura) rispetto alla corrente nel condensatore} \\
	\bar{I}_L = - j \frac{1}{\omega_0 L} \bar{U} \qquad &\text{la tensione è in anticipo (e in quadratura) rispetto alla corrente nell'induttanza} \\
	Q_L + Q_C = 0 \qquad &\text{le potenze reatt. sono opposte, induttore e condensatore formano un circuito aperto}
\end{align*}
Siccome vale \(I_L = I_C = Q_0 I_\text{parallelo}\), se \(Q_0 > 0\) le correnti all'interno dei singoli componenti vengono amplificate.

\subsubsection*{Analisi nel tempo di correnti, tensioni e potenze in antisonanza}
\begin{align*}
	i_{LC}(t) &= i_L(t) + i_C(t) = 0 \quad \rightarrow \quad i_L(t) = - i_C(t), \qquad\quad u_L(t) = u_C(t) \\
	p_L(t) &= u(t) i_L(t) = - u(t) i_C(t) = - p_C(t) \quad \rightarrow \quad p_L(t) + p_C(t) = 0 \\
	\omega_L(t) &= \frac{1}{2} Li(t)^2, \qquad \omega_C(t) = \frac{1}{2} Cu(t)^2 \quad \rightarrow \quad \omega_{LC} = \text{costante}
\end{align*}
Si osserva che, in antisonanza, il condensatore e l'induttore agiscono come circuito aperto e si scambiano energia a vicenda, senza
influire sulla potenza istantanea del circuito.

\vspace{20pt}

\subsection{Teorema del massimo trasferimento di potenza}
\begin{minipage}{0.3\textwidth}
	\centering
	\begin{circuitikz}[european]
		\draw (0,0) to[V, i_>=i] (2,0) to[R=\(\dot{Z}_i\)] (4,0) -- (4,-1.5) to[R=\(\dot{Z}_u\)] (0,-1.5) -- (0,0);
		\draw (1,0) [circ, thick, fill, color=white] circle(0.39);
		\node[] at (0.4,0.2) {--};
		\node[] at (1.6,0.2) {+};
		\node[] at (1,0.7) {\(e(t)\)};
		\node[] at (1,0) {\(\underline{\sim}\)};
	\end{circuitikz}
\end{minipage}
\begin{minipage}{0.69\textwidth}
	La potenza attiva ceduta al carico è massima se \(\dot{Z_i} = Z_u^*\), ovvero se il coniugato dell'impedenza di carico è pari
	all'impedenza interna del generatore (adattamento coniugato).
\end{minipage}

\newpage


\section{Reti in regime variabile}
\subsection{Regime variabile quasi stazionario}
Di tutte le possibili reti a regime variabile, si analizzano solo quelle a regime variabile quasi stazionario, in modo da poter
applicare le proprietà della teoria dei circuiti. Nello specifico verranno analizzati i regimi variabili che si verificano
all' apertura o alla chiusura di un interruttore che modifica il circuito, ciò provoca un transitorio tra il regime precedente
e quello nuovo.

\subsubsection*{Deviatore ideale}
\begin{center}
	\begin{minipage}{0.17\textwidth}
		\begin{circuitikz}
			\draw (1,0) node[spdt] (myspdt) {}
			(0,0) -- (myspdt.in) 
			(myspdt.out 1) -- +(0.5,0)
			(myspdt.out 2) -- +(0.5,0);
			\node[] at (0.3,0.2) {\(_\text{COM}\)};
			\node[] at (1.5,0.6) {1};
			\node[] at (1.5,-0.6) {2};
		\end{circuitikz}
	\end{minipage}
	\begin{minipage}{0.75\textwidth}
		dipolo in grado di commutarsi tra due stati: \\
		1. cortocircuito tra mors. 1 e comune, circuito aperto tra mors. 2 e comune \\
		2. cortocircuito tra mors. 1 e comune, circuito aperto tra mors. 2 e comune
	\end{minipage}
\end{center}

\subsection{Istante critico e variabili di stato all'istante critico}
\subsubsection*{Istante critico}
L'istante critico è l'ultimo istante di tempo in cui vale ancora il regime \say{vecchio} che corrisponde alle condizioni iniziali
del transitorio variabile. Studiare una rete variabile corrisponde a studiare cosa avviene nell'istante critico e nell'immediato
futuro.

\subsubsection*{Variabili di stato e continuità}
All'istante critico alcune grandezze possono ammettere discontinuità, altre sono continue. Le variabili che non ammettono discontinuità
all'istante critico sono dette variabili di stato.

\begin{itemize}
	\item interruttore ideale ha discontinuità quando si inverte lo stato \(u(0^-) \neq u(0^+)\)
	\item generatore ideale ha discontinuità in accensione o spegnimento \(e(0^-) \neq e(0^+), \; j(0^-) \neq j(0^+)\)
	\item resistenza ideale può presentare discontinuità di correnti e tensioni \(u(0^-) \neq u(0^+), \; i(0^-) \neq i(0^+)\)
	\item condensatore ideale non ammette discontinuità di tensione \(u(0^-) = u(0^+)\)
	\item induttore ideale non ammette discontinuità di corrente \(i(0^-) = i(0^+)\)
\end{itemize}

\subsection{Reti lineari in regime variabile}
\subsubsection*{Sistema di equazioni di reti lineari in regime variabile}
Per risolvere una rete lineare a regime variabile:
\begin{itemize}
	\item[1.] si scrive il sistema composto da \(2l\) equazioni differenziali ordinarie ottenute dalle LKC, LKT e dalle equazioni
	costitutive dei singoli bipoli
	\item[2.] si individuano le variabili di stato \(u_k(t)\) e \(i_k(t)\) e si identificano come uscite \(y(t)\)
	\item[3.] si individuano le variabili dei generatori \(e(t)\) e \(j(t)\) e si identificano come ingressi \(x(t)\)
	\item[4.] si risolve il sistema ottenendo una equazione differenziale ordinaria con unica incognita \(y(t)\)
	\item[5.] si risolve l'EDO ottenuta e di conseguenza si risolve l'intera rete
\end{itemize}

\subsubsection*{Equazione differenziale ordinaria}
\[\sum_{i=0}^{p} a_i \frac{d^i y(t)}{dt} = \sum_{i=0}^{q} b_i \frac{d^i x(t)}{dt} \qquad \begin{cases}
	\sum_{i=0}^{p} a_i \frac{d^i y(t)}{dt} \quad &\text{combinazione delle derivate degli ingressi} \\
	\sum_{i=0}^{q} b_i \frac{d^i x(t)}{dt} \quad &\text{combinazione delle derivate delle uscite}
\end{cases}\]
I coefficienti \(a_i\) e \(b_i\) dipendono dall'uscita considerata, il numero \(p\) \((\geq q)\) determina il grado dell'EDO ed è
sempre minore del numero di condensatori e induttori presenti nella rete. La soluzione dell'EDO si compone della somma tra la
soluzione dell'omogenea associata e una soluzione particolare.

\subsubsection*{Soluzione del'omogenea}
Supponendo che la soluzione dell'omogenea sia di tipo esponenziale \(y_0(t) = Ye^{st}\), si ottiene
\[\sum_{i=0}^{p} a_i \frac{d^i y(t)}{dt^i} = \sum_{i=0}^{p} a_i Y s^i e^{st} = Ye^{st} \sum_{i=0}^{p} a_i s^i = 0 \quad \Leftrightarrow \quad \sum_{i=0}^{p} a_i s^i = 0\]
\begin{itemize}
	\item il polinomio \(\sum_{i=0}^{p} a_i s^i = 0\) è definito equazione algebrica caratteristica del circuito, di grado \(p\)
	\item le soluzioni \(s = s_1, s_2, ... s_p\) sono esattamente \(p\) e sono chiamate zeri del polinomio
	\item le soluzioni possono essere reali distinte, reali coincidenti o complesse coniugate e dipendono dai valori R, L, C del circuito
	\item ciascuno zero \(s_i\) fornisce una soluzione parziale dell'omogenea del tipo \(y_0(t) = Y_i e ^{s_it}\) chiamato modo
	naturale o modo caratteristico dell'EDO.
\end{itemize}
La soluzione completa dell'omogenea è una combinazione lineare di tutti i modi naturali:
\[y_0(t) = Y_1e^{s_1t} + Y_2e^{s_2t} + \dots + Y_pe^{s_pt}\]

\subsubsection*{Radici reali e modi naturali unidirezionali}
Ad ogni radice reale \(s_i = \sigma_i\) è associato un modo naturale esponenziale del tipo \(y_0(t) = Y_i e^{\sigma_i t}\)
\begin{itemize}
	\item se le soluzioni sono negative o nulle \(\sigma_i \leq 0\), l'evoluzione del sistema converge ad un regime di stabilità (per reti di bipoli passivi si ha sempre stabilità)
	\item se ci sono due soluzioni coincidenti \(\sigma_{i} = \sigma_{i+1}\), il contributo risultante cumulativo di entrambi sarà della forma \(y_0(t) = Y_i e^{\sigma_i t} + Y_i t e^{\sigma_i t}\)
\end{itemize}

\subsubsection*{Radici complesse coniugate e modi natali oscillanti}
Ad ogni coppia di radici complesse coniugate è associato un modo naturale oscillatorio del tipo:
\begin{align*}
	y_0(t) &= Y_i e^{(\sigma_i + j\omega_i)t} + Y_{i+1} e^{(\sigma_i - j\omega_i)t} = e^{\sigma_i t} Y_i e^{j\omega_i t} + e^{\sigma_i t} Y_{i+1} e^{-j\omega_i t} \\
	&= e^{\sigma_i t} Y_i (\cos (\omega_i t) + j\sin(\omega_i t)) + e^{\sigma_i t} Y_{i+1} (\cos (\omega_i t) - j\sin(\omega_i t)) \\
	&= e^{\sigma_i t} (Y_i + Y_{i+1}) \cos (\omega_i t) + e^{\sigma_i t} j (Y_i - Y_{i+1}) \sin(\omega_i t) \\
	y_0(t) &= e^{\sigma_i t} A_i \cos (\omega_i t) + e^{\sigma_i t} B_i \sin(\omega_i t)
\end{align*}
Se la parte reale è negativa \(\sigma_i < 0\), si ha un moto armonico esponenzialmente smorzato e si converge ad un regime di stabilità.

\subsubsection*{Frequenze naturali e costanti di tempo}
Si osserva che le radici \(s_i\) esprimono una frequenza associata al modo naturale, per cui si definisce il tempo caratteristico
associato al modo naturale \(\tau_i = -1/\sigma_i\). In questo modo si ottengono:
\begin{itemize}
	\item modi naturali unidirezionali: \(y_0(t) = Y_i e^{-t/\tau_i}\)
	\item modi naturali oscillanti: \(y_0(t) = e^{-t/\tau_i} (A_i \cos(\omega_i t) + B_i \sin(\omega_i t))\)
\end{itemize}

\subsubsection*{Soluzione completa dell'EDO}
La soluzione completa dell'equazione differenziale è data dalla somma di quella dell'omogenea associata e di una soluzione particolare
\[y(t) = y_0(t) + y_p(t) = Y_1e^{s_1t} + Y_2e^{s_2t} + ... + Y_pe^{s_pt} + y_p(t)\]
Le costanti di integrazione \(Y_1, \dots Y_p\) si determinano imponendo le condizioni iniziali per l'uscita \(y(t)\) e le sue
\(p-1\)-esime derivate in \(t = 0\).

\newpage

\subsection{Carica del condensatore in circuito RC}
\begin{minipage}{0.4\textwidth}
	\centering
	\begin{circuitikz}
		\draw (1,0) node[spdt] (myspdt) {}
		(0,0) -- (myspdt.in)
		(myspdt.out 1) to[R=R, i>_=\(i(t)\)] +(2.5,0) to[C=C] +(2.5,-2) -- +(0,-2)
		(myspdt.out 2) -- +(0,-1.37) -- +(-1.6,-1.37) to[V=\(e(t)\)] (0,0);
		\node[] at (0.6,0.2) {1};
		\node[] at (1.4,0.55) {\(_\text{COM}\)};
		\node[] at (1.4,-0.55) {2};
	\end{circuitikz}
\end{minipage}
\begin{minipage}{0.5\textwidth}
	\begin{itemize}
		\item inizialmente lo switch si trova in posizione \(2\):
		\begin{itemize}[topsep=0pt]
			\item \(i(t) = 0\), \(u_C(t) = 0\)
		\end{itemize}
		\item a \(t=0\) passa in posizione \(1\):
		\begin{itemize}[topsep=0pt]
			\item \(u_C(t) = 0\) perché variabile di stato
		\end{itemize}
	\end{itemize}
\end{minipage}
\vspace{15pt}
\begin{itemize}[itemsep=0pt]
	\item[1.] scrivo le equazioni del sistema per \(t \geq 0\)
	\[\text{LKT:} \;\; u_R(t) + u_C(t) = E \quad\; \text{LKC:} \;\; i_R(t) = i_C(t) = i(t) \quad\; \text{R:} \;\; u_R(t) = Ri_R(t) \quad\; \text{C:} \;\; i_C(t) = C \frac{du_C(t)}{dt}\]
	\item[2.] ottengo la EDO e l'omogenea associata
	\[RC\frac{du_C(t)}{dt} + u_C(t) = E \qquad \rightarrow \qquad RC\frac{du_C(t)}{dt} + u_C(t) = 0 \quad\text{omogenea}\]
	\item[3.] soluzione particolare
	\[u_p(t) = U_P \text{ con } U_P = E \quad \rightarrow \quad RC\frac{d U_P}{dt} + U_P = E \;\;\rightarrow\;\; U_P = E\]
	\item[4.] soluzione dell'omogenea
	\[RC s + 1 = 0 \;\; \rightarrow \;\; s = -\frac{1}{RC}, \quad \tau = RC, \quad u_o(t) = U_0 e^{st} = U_0 e^{-t/\tau}\]
	\item[5.] soluzione completa e calcolo delle costanti con condizioni iniziali \(u_C(0) = 0\)
	\[u_C(t) = U_P + U_0 e^{-t/\tau} \;\; \rightarrow\;\; U_P = -U_0 = E \;\; \rightarrow \;\; u_C(t) = E(1-e^{-t/\tau}), \quad i(t) = \frac{E}{R} e^{-t/\tau}\]
\end{itemize}

\subsection{Scarica del condensatore in circuito RC}
\begin{minipage}{0.4\textwidth}
	\centering
	\begin{circuitikz}
		\draw (1,0) node[spdt] (myspdt) {}
		(0,0) -- (myspdt.in)
		(myspdt.out 1) to[R=R, i>_=\(i(t)\)] +(2.5,0) to[C=C] +(2.5,-2) -- +(0,-2)
		(myspdt.out 2) -- +(0,-1.37) -- +(-1.6,-1.37) to[V=\(e(t)\)] (0,0);
		\node[] at (0.6,0.2) {1};
		\node[] at (1.4,0.55) {\(_\text{COM}\)};
		\node[] at (1.4,-0.55) {2};
	\end{circuitikz}
\end{minipage}
\begin{minipage}{0.5\textwidth}
	\begin{itemize}
		\item inizialmente lo switch si trova in posizione \(1\):
		\begin{itemize}[topsep=0pt]
			\item \(u_C(t) = E\)
		\end{itemize}
		\item a \(t=0\) passa in posizione \(2\):
		\begin{itemize}[topsep=0pt]
			\item \(u_C(t) = E\) perché variabile di stato
		\end{itemize}
	\end{itemize}
\end{minipage}
\vspace{15pt}
\begin{itemize}[itemsep=0pt]
	\item[1.] scrivo le equazioni del sistema per \(t \geq 0\)
	\[\text{LKT:} \;\; u_R(t) + u_C(t) = 0 \quad\; \text{LKC:} \;\; i_R(t) = i_C(t) = i(t) \quad\; \text{R:} \;\; u_R(t) = Ri_R(t) \quad\; \text{C:} \;\; i_C(t) = C \frac{du_C(t)}{dt}\]
	\item[2.] ottengo la EDO che è già omogenea
	\[RC\frac{du_C(t)}{dt} + u_C(t) = 0 \quad\text{già omogenea, non serve la soluzione particolare}\]
	\item[4.] soluzione dell'omogenea
	\[RC s + 1 = 0 \;\; \rightarrow \;\; s = -\frac{1}{RC}, \quad \tau = RC, \quad u_o(t) = U_0 e^{st} = U_0 e^{-t/\tau}\]
	\item[5.] soluzione completa e calcolo delle costanti con condizioni iniziali \(u_C(0) = E\)
	\[u_C(t) = U_0 e^{-t/\tau} \;\; \rightarrow\;\; U_0 = E \;\; \rightarrow \;\; u_C(t) = Ee^{-t/\tau}, \quad i(t) = -\frac{E}{R} e^{-t/\tau}, \quad u_R = -Ee^{-t/\tau}\]
\end{itemize}

\newpage

\subsection{Carica dell'induttore in circuito RL}
\begin{minipage}{0.4\textwidth}
	\centering
	\begin{circuitikz}
		\draw (1,0) node[spdt] (myspdt) {}
		(0,0) -- (myspdt.in)
		(myspdt.out 1) to[R=R, i>_=\(i(t)\)] +(2.5,0) to[L=L] +(2.5,-2) -- +(0,-2)
		(myspdt.out 2) -- +(0,-1.37) -- +(-1.6,-1.37) to[V=\(e(t)\)] (0,0);
		\node[] at (0.6,0.2) {1};
		\node[] at (1.4,0.55) {\(_\text{COM}\)};
		\node[] at (1.4,-0.55) {2};
	\end{circuitikz}
\end{minipage}
\begin{minipage}{0.5\textwidth}
	\begin{itemize}
		\item inizialmente lo switch si trova in posizione \(2\):
		\begin{itemize}[topsep=0pt]
			\item \(i(t) = 0\), \(u_L(t) = 0\)
		\end{itemize}
		\item a \(t=0\) passa in posizione \(1\):
		\begin{itemize}[topsep=0pt]
			\item \(i_L(t) = 0\) perché variabile di stato
		\end{itemize}
	\end{itemize}
\end{minipage}
\vspace{15pt}
\begin{itemize}[itemsep=0pt]
	\item[1.] scrivo le equazioni del sistema per \(t \geq 0\)
	\[\text{LKT:} \;\; u_R(t) + u_L(t) = E \quad\; \text{LKC:} \;\; i_R(t) = i_L(t) = i(t) \quad\; \text{R:} \;\; u_R(t) = Ri_R(t) \quad\; \text{L:} \;\; u_L(t) = L \frac{di_L(t)}{dt}\]
	\item[2.] ottengo la EDO e l'omogenea associata
	\[L\frac{di_L(t)}{dt} + Ri_L(t) = E \qquad \rightarrow \qquad L\frac{di_L(t)}{dt} + Ri_L(t) = 0 \quad\text{omogenea}\]
	\item[3.] soluzione particolare
	\[i_p(t) = I_P \text{ con } i_P = \frac{E}{R} \quad \rightarrow \quad L\frac{dI_P}{dt} + RI_P = E \;\;\rightarrow\;\; I_P = \frac{E}{R}\]
	\item[4.] soluzione dell'omogenea
	\[L s + R = 0 \;\; \rightarrow \;\; s = -\frac{R}{L}, \quad \tau = \frac{L}{R}, \quad i_o(t) = I_0 e^{st} = I_0 e^{-t/\tau}\]
	\item[5.] soluzione completa e calcolo delle costanti con condizioni iniziali \(i_L(0) = 0\)
	\[i_L(t) = I_P + I_0 e^{-t/\tau} \;\; \rightarrow\;\; I_P = I_0 = \frac{E}{R} \;\; \rightarrow \;\; i_L(t) = \frac{E}{R}(1-e^{-t/\tau}), \quad \begin{array}{l} u_L(t) = E e^{-t/\tau} \\ u_R(t) = E(1-e^{-t/\tau}) \end{array}\]
\end{itemize}

\subsection{Scarica dell'induttore in circuito RL}
\begin{minipage}{0.4\textwidth}
	\centering
	\begin{circuitikz}
		\draw (1,0) node[spdt] (myspdt) {}
		(0,0) -- (myspdt.in)
		(myspdt.out 1) to[R=R, i>_=\(i(t)\)] +(2.5,0) to[L=L] +(2.5,-2) -- +(0,-2)
		(myspdt.out 2) -- +(0,-1.37) -- +(-1.6,-1.37) to[V=\(e(t)\)] (0,0);
		\node[] at (0.6,0.2) {1};
		\node[] at (1.4,0.55) {\(_\text{COM}\)};
		\node[] at (1.4,-0.55) {2};
	\end{circuitikz}
\end{minipage}
\begin{minipage}{0.5\textwidth}
	\begin{itemize}
		\item inizialmente lo switch si trova in posizione \(1\):
		\begin{itemize}[topsep=0pt]
			\item \(i_L(t) = E/R\)
		\end{itemize}
		\item a \(t=0\) passa in posizione \(2\):
		\begin{itemize}[topsep=0pt]
			\item \(i_L(t) = E/R\) perché variabile di stato
		\end{itemize}
	\end{itemize}
\end{minipage}
\vspace{15pt}
\begin{itemize}[itemsep=0pt]
	\item[1.] scrivo le equazioni del sistema per \(t \geq 0\)
	\[\text{LKT:} \;\; u_R(t) + u_L(t) = 0 \quad\; \text{LKC:} \;\; i_R(t) = i_L(t) = i(t) \quad\; \text{R:} \;\; u_R(t) = Ri_R(t) \quad\; \text{L:} \;\; u_L(t) = L \frac{di_L(t)}{dt}\]
	\item[2.] ottengo la EDO che è già omogenea
	\[L\frac{di_L(t)}{dt} + Ri_L(t) = 0 \quad\text{già omogenea, non serve la soluzione particolare}\]
	\item[4.] soluzione dell'omogenea
	\[L s + R = 0 \;\; \rightarrow \;\; s = -\frac{R}{L}, \quad \tau = \frac{L}{R}, \quad i_o(t) = I_0 e^{st} = I_0 e^{-t/\tau}\]
	\item[5.] soluzione completa e calcolo delle costanti con condizioni iniziali \(i_L(0) = E/R\)
	\[i_L(t) = I_0 e^{-t/\tau} \;\; \rightarrow\;\; I_0 = \frac{E}{R} \;\; \rightarrow \;\; i_L(t) = \frac{E}{R}e^{-t/\tau}, \quad u_L(t) = - E e^{-t/\tau}, \quad u_R(t) = Ee^{-t/\tau}\]
\end{itemize}

\newpage

\subsection{Circuito RLC e fattore di smorzamento}
\begin{minipage}{0.4\textwidth}
	\centering
	\begin{circuitikz}
		\draw (1,0) node[spdt] (myspdt) {}
		(0,0) -- (myspdt.in)
		(myspdt.out 1) to[R=R, i>_=\(i(t)\)] +(2.5,0) to[L=L] +(2.5,-2) to[C=C] +(0,-2)
		(myspdt.out 2) -- +(0,-1.37) -- +(-1.6,-1.37) to[V=\(e(t)\)] (0,0);
		\node[] at (0.6,0.2) {1};
		\node[] at (1.4,0.55) {\(_\text{COM}\)};
		\node[] at (1.4,-0.55) {2};
	\end{circuitikz}
\end{minipage}
\begin{minipage}{0.5\textwidth}
	\begin{itemize}
		\item inizialmente lo switch si trova in posizione \(2\):
		\begin{itemize}[topsep=0pt]
			\item \(i(t) = 0\), \(u_R(t) = u_L(t) = u_C(t) = 0\)
		\end{itemize}
		\item a \(t=0\) passa in posizione \(1\):
		\begin{itemize}[topsep=0pt]
			\item \(i_L(t) = 0\) e \(u_C(t) = 0\) perché variabili di stato
		\end{itemize}
	\end{itemize}
\end{minipage}
\vspace{10pt}

\begin{itemize}[itemsep=0pt]
	\item[1.] scrivo le equazioni del sistema per \(t \geq 0\)
	\[\begin{array}{l}
		\text{LKT:} \;\; u_R(t) + u_L(t) + u_C(t) = E \\[7pt]
		\text{LKC:} \;\; i_R(t) = i_L(t) = i_C(t) = i(t)
	\end{array} \qquad u_R(t) = Ri_R(t) \qquad u_L(t) = L \frac{di_L(t)}{dt} \qquad i_C(t) = C \frac{du_C(t)}{dt}\]
	\item[2.] ottengo le EDO in funzione della corrente e della tensione e le relative omogenee associate
	\[\begin{array}{l}
		\displaystyle LC\frac{d^2u_C(t)}{dt^2} + RC\frac{du_C(t)}{dt} + u_C(t) = E \\[10pt]
		\displaystyle LC\frac{d^2i_L(t)}{dt^2} + RC\frac{di_L(t)}{dt} + i_L(t) = 0
	\end{array}
	\quad \rightarrow \quad \begin{array}{l}
		\displaystyle LC\frac{d^2u_C(t)}{dt^2} + RC\frac{du_C(t)}{dt} + u_C(t) = 0 \\[10pt]
		\displaystyle LC\frac{d^2i_L(t)}{dt^2} + RC\frac{di_L(t)}{dt} + i_L(t) = 0
	\end{array}\]
	\item[3.] si risolve la EDO in funzione delle tensioni, introducendo i parametri \(T\), \(\omega_0\), \(\xi\) e si risolve
	distinguendo tre casi in base al valore di \(\xi > 1\), \(\xi = 1\), \(\xi < 1\)
	\[\begin{array}{l}
		\displaystyle T = \frac{2L}{R} \;\; \text{costante di tempo} \\[10pt]
		\displaystyle \omega_0 = \frac{1}{\sqrt{LC}} \;\; \text{pulsazione}
	\end{array} \qquad\qquad \begin{array}{l}
		\displaystyle \xi = \frac{1}{\omega_0 T} = \frac{R}{2\omega_0 L} = \frac{R}{R_{cr}} \;\; \text{smorzamento} \\[10pt]
		\displaystyle R_{cr} = 2 \omega_0 L = 2\sqrt{\frac{L}{C}} \;\; \text{resistenza critica}
	\end{array}\]
	
	\item[4.] caso \(\xi > 1\), andamento sovrasmorzato, \(R > R_{cr}\) \\
	si ha un andamento esponenziale monotono e si definiscono due costanti di tempo \(0 < T_2 < T < T_1\) tali per cui per \(t \to +\infty\)
	prevale la costante di tempo maggiore \(T_1\):
	
	\begin{minipage}{0.5\textwidth}
		\begin{align*}
			u_C(t) &= E - \frac{T_1}{T_1 - T_2} E e^{-t/T_1} + \frac{T_2}{T_1 - T_2} E e^{-t/T_2} \\
			i_L(t) &= \frac{C E}{T_1 - T_2} (e^{-t/T_1} - e^{-t/T_2})
		\end{align*}
	\end{minipage}
	\begin{minipage}{0.4\textwidth}
		\centering
		\includegraphics[width=0.6\linewidth]{immagini/rlc sovrasmorzato.png}
	\end{minipage}
	
	\item[5.] caso \(\xi = 1\), andamento criticamente smorzato \(R = R_{cr}\) \\
	si ha un andamento esponenziale monotono con una sola costante di tempo \(T = T_1 = T_2\):
	
	\begin{minipage}{0.5\textwidth}
		\begin{align*}
			u_C(t) &= E - E e^{-t/T} - E \frac{t}{T} e^{-t/T} \\
			i_L(t) &= \frac{C E}{T^2}t e^{-t/T}
		\end{align*}
	\end{minipage}
	\begin{minipage}{0.4\textwidth}
		\centering
		\includegraphics[width=0.6\linewidth]{immagini/rlc criticamente smorzato.png}
	\end{minipage}
	
	\item[6.] caso \(\xi < 1\), andamento sottosmorzato \(R < R_{cr}\) \\
	si ha un andamento oscillatorio esponenzialmente smorzato con una singola costante di tempo \(T\):
	
	\begin{minipage}{0.5\textwidth}
		\begin{align*}
			u_C(t) &= E - E e^{-t/T} \cos (\omega t) - \frac{E}{T\omega} e^{-t/T}\sin (\omega t) \\
			i_L(t) &= \frac{E}{L\omega}t e^{-t/T} \sin (\omega t)
		\end{align*}
	\end{minipage}
	\begin{minipage}{0.4\textwidth}
		\centering
		\includegraphics[width=0.6\linewidth]{immagini/rlc sottosmorzato.png}
	\end{minipage}
\end{itemize}

\subsection{Considerazioni finali su reti a regime variabile che tendono alla stabilità}
\begin{itemize}
	\item nelle reti a regime smorzato, il transitorio termina dopo \(\approx 5T\), nel caso in cui si hanno più
	costanti di tempo, domina quella maggiore
	\item dopo il transitorio iniziale , la rete tende ad un regime che dipende dai generatori (stazionario, sinusoidale), in
	assenza di generatori (con smorzamento), il regime è nullo \(u(t) = 0\), \(i(t) = 0\) 
	\item il contributo dello smorzamento è dato dai componenti resistivi \(R\)
	\item il contributo delle oscillazioni è dato dai componenti induttivo-capacitivi \(L\) e \(C\)
\end{itemize}

\end{document}
